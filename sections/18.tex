The hybrid controller is modelled formally by an 8-tuple
$$H=(Q,X,Init,f,D,E,G,R)$$


\begin{figure}[H]\centering
  \scalebox{0.7}{% Generated with LaTeXDraw 2.0.8
% Wed Mar 02 01:38:00 CET 2016
% \usepackage[usenames,dvipsnames]{pstricks}
% \usepackage{epsfig}
% \usepackage{pst-grad} % For gradients
% \usepackage{pst-plot} % For axes
\scalebox{1} % Change this value to rescale the drawing.
{
\begin{pspicture}(0,-2.01)(21.079063,2.01)
\usefont{T1}{ptm}{m}{n}
\rput(4.4445314,0.77){$q_1 = \texttt{Rotation}$}
\pscircle[linewidth=0.04,dimen=outer](4.43,0.01){2.0}
\usefont{T1}{ptm}{m}{n}
\rput(4.4045315,0.05){$\dot{x} = f(q_1,x)$}
\usefont{T1}{ptm}{m}{n}
\rput(4.474531,-0.73){$x \in D(q_1)$}
\usefont{T1}{ptm}{m}{n}
\rput(11.774531,0.73){$q_2 = \texttt{Translation}$}
\pscircle[linewidth=0.04,dimen=outer](11.81,-0.01){2.0}
\usefont{T1}{ptm}{m}{n}
\rput(11.784532,0.03){$\dot{x} = f(q_2,x)$}
\usefont{T1}{ptm}{m}{n}
\rput(11.854531,-0.75){$x \in D(q_2)$}
\psline[linewidth=0.04cm,arrowsize=0.05291667cm 2.0,arrowlength=1.4,arrowinset=0.4]{->}(0.0,-0.02)(2.48,-0.02)
\psline[linewidth=0.04cm,arrowsize=0.05291667cm 2.0,arrowlength=1.4,arrowinset=0.4]{->}(6.392,-0.02)(9.812,-0.04)
\usefont{T1}{ptm}{m}{n}
\rput(19.054531,0.71){$q_3 = \texttt{Stop}$}
\pscircle[linewidth=0.04,dimen=outer](19.03,0.03){1.87}
\usefont{T1}{ptm}{m}{n}
\rput(19.064531,-0.01){$\dot{x} = f(q_3,x)$}
\usefont{T1}{ptm}{m}{n}
\rput(19.134531,-0.79){$x \in D(q_3)$}
\psline[linewidth=0.04cm,arrowsize=0.05291667cm 2.0,arrowlength=1.4,arrowinset=0.4]{->}(13.792,-0.02)(17.212,-0.04)
\usefont{T1}{ptm}{m}{n}
\rput(8.064531,0.59){$E_1$}
\usefont{T1}{ptm}{m}{n}
\rput(15.484531,0.61){$E_2$}
\usefont{T1}{ptm}{m}{n}
\rput(8.014531,-0.41){$g_1(x) \in G(q_1, q_2)$}
\usefont{T1}{ptm}{m}{n}
\rput(15.494532,-0.39){$g_2(x) \in G(q_2, q_3)$}
\pscircle[linewidth=0.04,dimen=outer](19.03,0.03){1.8}
\end{pspicture} 
}

}
  \caption{The hybrid automaton that controls the navigation of the robot
    from an initial state $(x_0, y_0, \theta_0)$ to a goal location $(x_g, y_g)$.}
  \label{}
\end{figure}

\begin{itemize}

\item
  $Q \equiv \{q_1 ,q_2 ,q_3\}$ denotes the set of discrete states. The robot is in
  state $q_1$ when executing a rotation, in state $q_2$ when executing
  line-following, and in state $q_3$ when it has stopped.

\item
  $Init \equiv \{q_1\}$. The initial state is taken to be $q_1$.

\item
  $X \equiv \{(x,y,\theta): x,y \in \mathbb{R}^2, \theta \in(-180^{\circ}, 180^{\circ}] \}$
  denotes the continuous states

\item
  Vector fields $f$

  \begin{equation*}
    f(q_1, X) =
    \begin{bmatrix}
      \dot{x} = R u_{\omega} cos\theta      \\
      \dot{y} = R u_{\omega} sin\theta      \\
      \dot{\theta} = \dfrac{R}{L} u_{\Psi}
    \end{bmatrix}
  \end{equation*}

  \begin{equation*}
    f(q_2, X) =
    \begin{bmatrix}
      \dot{x} = R u_{\omega} cos\theta      \\
      \dot{y} = R u_{\omega} sin\theta      \\
      \dot{\theta} = \dfrac{R}{L} u_{\Psi}
    \end{bmatrix}
  \end{equation*}

  \begin{equation*}
    f(q_2, X) =
    \begin{bmatrix}
      \dot{x} = 0       \\
      \dot{y} = 0       \\
      \dot{\theta} = 0
    \end{bmatrix}
  \end{equation*}

\item
  $D$ shows what conditions need to be satisfied in order for the automaton to
  stay in a state.

  $D(q_1)=\{(x,y,\theta): x,y \in \mathbb{R}^2, |\theta_g-\theta| > \delta\}$

  $D(q_2)=\{(x,y,\theta): x,y \in \mathbb{R}^2, (x_g-x)^2+(y_g-y)^2>\xi, \theta \in(-180^{\circ}, 180^{\circ}]\}$

  $D(q_3)= \{x,y \in \mathbb{R}^2, \theta \in(-180^{\circ}, 180^{\circ}] \}$

\item
  $E$: The edges show which transitions are possible.

  $E_1 = \{ q_1,q_2\}$

  $E_2 = \{ q_2,q_3\}$

  $E = E_1 \cup E_2$

\item
  $G$: The guards show under what conditions the system can transition from one to
  another state.

  $G(\{q_1,q_2\})=\{\theta_g-\theta\le \delta \}$

  $G(\{q_2,q_3\})=\{(x_g-x)^2+(y_g-y)^2 \leq \xi\}$

\item
  $R$: Resets illustrate the values that the state takes when transitioning
  between states.

  $R=\{x,y,\theta\}$

\end{itemize}
