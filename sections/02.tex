On the basis of the continuous state equations for translation.

\begin{equation}
\begin{array}{ccc}
  \dot{x} = R\:u_{\omega}\:cos\theta \\

  \dot{y} = R\:u_{\omega}\:sin\theta
\end{array}
\end{equation}


Since the robot itself uses centimeters as reference distance unit, the  given
200 value for the speed turns to be: $u_{\omega} = 2\,[m/s]$. Furthermore,
assuming:

 \begin{enumerate}
 \item $ \dot{x} = \frac{\Delta x}{\Delta t}$
 \item $ \dot{y} = \frac{\Delta y}{\Delta t}$
 \item $\Delta x = x_{f} - x_{o};\ \Delta y = y_{f} - y_{o}\ \Delta t = t_{f} - t_{o}$
 \end{enumerate}

 and multiplying each term of (1) by itself and adding them toghether,
 one gets the following equation  that can be solved with values of the
 variations in position coordinates and time, plus the above mentioned value of
 the speed.

\begin{equation*}
 \left.
  \begin{array}{ll}
    R &= \sqrt{\dfrac{1}{u_{\omega}^{2}}\:\left[ \left( \dfrac{\Delta x}{\Delta t}\right)^2 + \:\left( \dfrac{\Delta y}{\Delta t}\right)^2 \right]} \\
    \Delta x &= \;1.3\:[m] \\
    \Delta y &=\;-0.07\:[m] \\
    \Delta t &=\;6.4250\:[s] \\
    u_{\omega} &=\;2\,[m/s] \\
\end{array}
\right\}\:R = 0.1013
 \end{equation*} \\

For the calculation of L, the equation for the robot pose is used.
$$\dot{\theta} = \dfrac{R}{L}\,u_{\Psi}$$

As earlier, we have considered $\dot{\theta} = \dfrac{\Delta \theta}{\Delta t}$.
From file \texttt{Rotation.cvs}, created for $u_{\Psi} = 400\,[cm/s]$ one gets
$\Delta\theta = 330.27^{\circ}$ , $\Delta t = 4.1667\,[s]$. Hence:

$$L = \dfrac{\Delta t}{\Delta \theta}\,R\,u_{\Psi} = 0.5113$$
