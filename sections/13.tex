For the purpose of simulating this part of the controller, the initial point of
the robot was taken to be $I(x_0, y_0) \equiv (0, 0)$. The goal was set to
$G(x_g, y_g) \equiv (1.0, 0.0)$.

Figures \ref{fig:13_0.1_max}, \ref{fig:13_0.2_max}, \ref{fig:13_0.5_max},
\ref{fig:13_0.75_max} and \ref{fig:13_max} show the displacemental response of
the robot for various values of $K_{\omega}^T$ inside the interval set by
inequality \ref{eq:12.2}. Figure \ref{fig:13_max_plus_one} verifies that the
upper limit for $K_{\omega}^T$ is indeed $\frac{2}{T_s R}$ by showing that the
displacemental response of the robot cannot converge for
$K_{\omega}^T > K_{\omega,max}^T$.

Here, one can see that the smaller the value of $K_{\omega}^T$ is, the larger
the settling time, the lower the rise time and the smoother the response is.
However, as the value of $K_{\omega}^T$ increases, the steady-state response
begins to oscillate, with the amplitude of this oscillation proportional to the
value of $K_{\omega}^T$.

Figures \ref{fig:13_0.1_max_magnified}, \ref{fig:13_0.2_max_magnified},
\ref{fig:13_0.5_max_magnified}, \ref{fig:13_0.75_max_magnified} and
\ref{fig:13_max_magnified} focus on the steady-state value of the aforementioned
responses. In contrast to the proportional rotational controller, the robot
\textit{can} arrive to its reference signal. The equations that govern the
robot's translational movement are non-linear, as opposed to the one that
governs its rotational movement, which in turn means that the translational
system's behaviour is not bounded within the laws that govern control with
proportional controllers on linear systems.


\noindent\makebox[\textwidth][c]{%
\begin{minipage}{\linewidth}
  \begin{minipage}{0.45\linewidth}
    \begin{figure}[H]
      \scalebox{0.6}{% This file was created by matlab2tikz.
%
%The latest updates can be retrieved from
%  http://www.mathworks.com/matlabcentral/fileexchange/22022-matlab2tikz-matlab2tikz
%where you can also make suggestions and rate matlab2tikz.
%
\definecolor{mycolor1}{rgb}{0.00000,0.44700,0.74100}%
%
\begin{tikzpicture}

\begin{axis}[%
width=4.133in,
height=3.26in,
at={(0.693in,0.44in)},
scale only axis,
xmin=0,
xmax=14,
xmajorgrids,
ymin=0,
ymax=3,
ymajorgrids,
axis background/.style={fill=white}
]
\addplot [color=mycolor1,solid,forget plot]
  table[row sep=crcr]{%
0	0\\
0.0191209039989989	0.01\\
0.0317448579999991	0.02\\
0.0476948880009986	0.03\\
0.0637059360009991	0.04\\
0.079731515000999	0.05\\
0.0958207519990001	0.07\\
0.111901427999999	0.08\\
0.127871894001	0.09\\
0.143821001999	0.1\\
0.159869505000999	0.12\\
0.175798412000999	0.13\\
0.191888835000999	0.14\\
0.207800022	0.16\\
0.223841464001	0.17\\
0.239784652999999	0.18\\
0.255804436000999	0.19\\
0.271791806	0.21\\
0.287798354	0.22\\
0.303801628999999	0.23\\
0.319748356000999	0.25\\
0.335802155999999	0.26\\
0.351806124001	0.27\\
0.367960881999999	0.28\\
0.383952647	0.3\\
0.399957356000999	0.31\\
0.415914512001	0.32\\
0.432000033001	0.34\\
0.447905792998999	0.35\\
0.463881607999999	0.36\\
0.479782874001	0.37\\
0.495847639999999	0.39\\
0.511907553998999	0.4\\
0.527952298999999	0.41\\
0.543848596999999	0.43\\
0.559911287000999	0.44\\
0.575915850001	0.45\\
0.591961120999999	0.46\\
0.607833249001	0.48\\
0.623788345998999	0.49\\
0.639936274	0.5\\
0.655926089001	0.52\\
0.671866296	0.53\\
0.687973953001	0.54\\
0.703914631999999	0.55\\
0.719926953001	0.57\\
0.735993943001	0.58\\
0.751901421	0.59\\
0.767961295	0.61\\
0.783941061998999	0.62\\
0.800072544001	0.63\\
0.815918922999998	0.64\\
0.831928382999999	0.66\\
0.847943565001	0.67\\
0.863970208999999	0.68\\
0.879923515998999	0.69\\
0.895938733002	0.71\\
0.911958682001	0.72\\
0.927989159001	0.73\\
0.943956984000999	0.75\\
0.959921216	0.76\\
0.975810679001	0.77\\
0.991876691	0.78\\
1.007932045	0.8\\
1.025383128	0.81\\
1.040465445	0.81\\
1.055805858	0.81\\
1.072044210001	0.82\\
1.088000753	0.82\\
1.103895887999	0.82\\
1.119998302	0.83\\
1.136032021	0.83\\
1.151911667	0.83\\
1.167809403	0.84\\
1.184044754	0.84\\
1.199784637999	0.84\\
1.215934591	0.85\\
1.231925553	0.85\\
1.247859456999	0.85\\
1.263942548	0.86\\
1.279947603001	0.86\\
1.295859105	0.86\\
1.311939084	0.87\\
1.327875212	0.87\\
1.343847692999	0.87\\
1.359795097	0.88\\
1.375801555	0.88\\
1.391793159001	0.88\\
1.407798834999	0.89\\
1.423971023001	0.89\\
1.440059371	0.89\\
1.455916767	0.9\\
1.471936523998	0.9\\
1.488026426001	0.9\\
1.50395373	0.91\\
1.519894984001	0.91\\
1.535790887001	0.91\\
1.551794573	0.92\\
1.567794665999	0.92\\
1.583929791001	0.92\\
1.599944206	0.93\\
1.615887876999	0.93\\
1.631785814001	0.93\\
1.64786884	0.94\\
1.663844738001	0.94\\
1.679944617001	0.94\\
1.696022923	0.95\\
1.711819539999	0.95\\
1.727754355	0.95\\
1.743748945999	0.96\\
1.760006894001	0.96\\
1.775985013001	0.96\\
1.791925441	0.97\\
1.807973071999	0.97\\
1.823937719	0.97\\
1.840027732001	0.98\\
1.855939689999	0.98\\
1.871902153	0.98\\
1.887908075001	0.99\\
1.903852841999	0.99\\
1.919859927	0.99\\
1.935787476	1\\
1.951807414002	1\\
1.967801959	1\\
1.983774977999	1.01\\
1.999781815001	1.01\\
2.020706031	1.01\\
2.032605297999	1.01\\
2.04778199	1.01\\
2.063812898001	1.01\\
2.079762657	1.01\\
2.095936444	1.01\\
2.111831444	1.01\\
2.127938709999	1.01\\
2.143859483999	1.01\\
2.159994483	1.01\\
2.175965796	1.01\\
2.191909406	1.01\\
2.207948443001	1.01\\
2.224158743	1.01\\
2.239957053002	1.01\\
2.255928487	1.01\\
2.27196626	1.01\\
2.287931199	1.02\\
2.303808278	1.02\\
2.319793954001	1.02\\
2.335759955	1.02\\
2.351774100001	1.02\\
2.367751070999	1.02\\
2.383809316	1.02\\
2.399933731001	1.02\\
2.415974923	1.02\\
2.431955782999	1.02\\
2.448076159001	1.02\\
2.463935888001	1.02\\
2.479935140001	1.02\\
2.496032723999	1.02\\
2.511970131001	1.02\\
2.527964004	1.02\\
2.544056964001	1.02\\
2.559931903	1.02\\
2.575936978001	1.02\\
2.591967259001	1.02\\
2.607906238999	1.02\\
2.624017685	1.02\\
2.639832722	1.02\\
2.65584306	1.02\\
2.671930983	1.02\\
2.687937469999	1.02\\
2.703921021	1.02\\
2.719937540999	1.02\\
2.735947808001	1.02\\
2.751951744999	1.02\\
2.767922636999	1.02\\
2.783944655001	1.02\\
2.800224357001	1.02\\
2.81593767	1.02\\
2.831980390001	1.02\\
2.847938973999	1.02\\
2.863819166001	1.02\\
2.879967455	1.02\\
2.896072636999	1.02\\
2.911883841999	1.02\\
2.928030778	1.02\\
2.943921122002	1.02\\
2.959844321001	1.02\\
2.975798886	1.02\\
2.991822067	1.02\\
3.007839444	1.02\\
3.025126094	1.02\\
3.040156354	1.02\\
3.055964818001	1.02\\
3.071971766998	1.02\\
3.088482152001	1.02\\
3.103921012999	1.02\\
3.119914084999	1.02\\
3.135996460001	1.02\\
3.151861797999	1.02\\
3.167885279001	1.02\\
3.184026833	1.02\\
3.199935167999	1.02\\
3.216803811001	1.02\\
3.232285448	1.02\\
3.247895858	1.02\\
3.263908695999	1.02\\
3.279917028	1.02\\
3.295939140999	1.02\\
3.311957224999	1.02\\
3.328091115999	1.02\\
3.343928981999	1.02\\
3.359942355	1.02\\
3.375902167	1.02\\
3.391814587	1.02\\
3.407810952002	1.02\\
3.423811827	1.02\\
3.439948674999	1.02\\
3.455933508999	1.02\\
3.471803287001	1.02\\
3.487797862	1.02\\
3.503819779999	1.02\\
3.51978317	1.02\\
3.535815471001	1.02\\
3.552061382	1.02\\
3.567935766001	1.02\\
3.58394331	1.02\\
3.600002708	1.02\\
3.616063765	1.02\\
3.631947844999	1.02\\
3.647986466002	1.02\\
3.663838792	1.02\\
3.679921803	1.02\\
3.695935730999	1.02\\
3.712000664999	1.02\\
3.727938613001	1.02\\
3.743969711001	1.02\\
3.759934780001	1.02\\
3.775941953999	1.02\\
3.791940374001	1.02\\
3.807807255001	1.02\\
3.823790404999	1.02\\
3.839859996	1.02\\
3.856007804001	1.02\\
3.871931224001	1.02\\
3.887929227001	1.02\\
3.903917794001	1.02\\
3.919932005001	1.02\\
3.935975199999	1.02\\
3.951959072001	1.02\\
3.967932328999	1.02\\
3.98385181	1.02\\
3.999994933001	1.02\\
4.021564753	1.02\\
4.033896726999	1.02\\
4.049015378	1.02\\
4.064002204001	1.02\\
4.079734881001	1.02\\
4.095791049002	1.02\\
4.111788586001	1.02\\
4.127791396	1.02\\
4.143757156	1.03\\
4.159832024	1.03\\
4.175817023001	1.03\\
4.191795091	1.03\\
4.20783273	1.03\\
4.223915656	1.03\\
4.239759944	1.03\\
4.255810198	1.03\\
4.271772431	1.03\\
4.287799555	1.03\\
4.303791815001	1.03\\
4.319807528999	1.03\\
4.335824445	1.03\\
4.351800364	1.03\\
4.367799816	1.03\\
4.383940091999	1.03\\
4.399935324001	1.03\\
4.415898594	1.03\\
4.431932057999	1.03\\
4.447984931	1.03\\
4.463892605	1.03\\
4.479911646	1.03\\
4.495947455	1.03\\
4.512049274	1.03\\
4.527966348999	1.03\\
4.543816692999	1.03\\
4.559810587	1.03\\
4.575808355	1.03\\
4.591795233999	1.03\\
4.607812379999	1.03\\
4.623785503	1.03\\
4.639788402001	1.03\\
4.655808316	1.03\\
4.671883439001	1.03\\
4.687932313002	1.03\\
4.703978531	1.03\\
4.719944832001	1.03\\
4.735983954001	1.03\\
4.751938935	1.03\\
4.767992800001	1.03\\
4.783936907	1.03\\
4.799944716999	1.03\\
4.815809629	1.03\\
4.831804640999	1.03\\
4.847937455	1.03\\
4.863968739	1.03\\
4.879964403	1.03\\
4.895844764	1.03\\
4.911835922001	1.03\\
4.927940528	1.03\\
4.94393114	1.03\\
4.959909604	1.03\\
4.975974619002	1.03\\
4.991918730999	1.03\\
5.007901088	1.03\\
5.02556765	1.03\\
5.040783645001	1.04\\
5.056165657	1.04\\
5.072689999001	1.04\\
5.088068674	1.04\\
5.103995636	1.04\\
5.119996241001	1.04\\
5.135790011002	1.04\\
5.151798784001	1.04\\
5.167989274	1.04\\
5.183980282002	1.04\\
5.200048419001	1.04\\
5.215820401001	1.04\\
5.231836477999	1.04\\
5.247783809	1.04\\
5.263965744002	1.04\\
5.27995574	1.04\\
5.296054928002	1.04\\
5.311947121	1.04\\
5.327807577999	1.04\\
5.343789424999	1.04\\
5.359907200001	1.04\\
5.375943174999	1.04\\
5.391821807999	1.04\\
5.407784581	1.04\\
5.423782348999	1.04\\
5.439890453001	1.04\\
5.455935216999	1.04\\
5.471868434	1.04\\
5.487789939001	1.04\\
5.503789376	1.04\\
5.519768391001	1.04\\
5.535792559	1.04\\
5.551788612999	1.04\\
5.567936892	1.04\\
5.584045782002	1.04\\
5.599870027001	1.04\\
5.615819746	1.04\\
5.631782951	1.05\\
5.647802789	1.05\\
5.663906171	1.05\\
5.679943813999	1.05\\
5.695998706999	1.05\\
5.71193676	1.05\\
5.727936833	1.05\\
5.743972818001	1.05\\
5.759887121	1.05\\
5.775808866999	1.05\\
5.791954530001	1.05\\
5.807943173	1.05\\
5.824020436001	1.05\\
5.839922417	1.05\\
5.855795572001	1.05\\
5.871797188999	1.05\\
5.888009297999	1.05\\
5.903933319	1.05\\
5.919862316	1.05\\
5.935820253	1.05\\
5.951914656	1.05\\
5.967905328001	1.05\\
5.983867349001	1.05\\
5.999967679999	1.05\\
6.022098035999	1.05\\
6.031811312	1.05\\
6.047826155001	1.05\\
6.063960222	1.05\\
6.079932361	1.05\\
6.096095107001	1.05\\
6.11181467	1.05\\
6.127797524	1.06\\
6.143803800999	1.06\\
6.159824039999	1.06\\
6.175827892	1.06\\
6.19182781	1.06\\
6.208002344	1.06\\
6.223887336001	1.06\\
6.239901762001	1.06\\
6.25581356	1.06\\
6.27191382	1.06\\
6.287860592001	1.06\\
6.303945512999	1.06\\
6.319912361	1.06\\
6.335869562	1.06\\
6.351669364	1.06\\
6.367661191	1.06\\
6.383752320999	1.06\\
6.399936822001	1.06\\
6.416405067	1.06\\
6.431954659001	1.06\\
6.4479075	1.06\\
6.463964843001	1.06\\
6.479918349999	1.07\\
6.495817156	1.07\\
6.512127496	1.07\\
6.527928983999	1.07\\
6.543961615999	1.07\\
6.559928528999	1.07\\
6.575926843001	1.07\\
6.591871105	1.07\\
6.60794034	1.07\\
6.623823396	1.07\\
6.639936989	1.07\\
6.65595374	1.07\\
6.671885518	1.07\\
6.688012653999	1.07\\
6.703966277001	1.07\\
6.719978903999	1.07\\
6.735919969	1.07\\
6.751875072001	1.07\\
6.767738150999	1.07\\
6.783897997999	1.07\\
6.799807767	1.07\\
6.815812520001	1.07\\
6.83184434	1.08\\
6.847791960001	1.08\\
6.863770118	1.08\\
6.879799132	1.08\\
6.895796139	1.08\\
6.911920664	1.08\\
6.927941539	1.08\\
6.943941717999	1.08\\
6.959918282999	1.08\\
6.975913826	1.08\\
6.992376177	1.08\\
7.007974154001	1.08\\
7.025648888001	1.08\\
7.040824483999	1.08\\
7.05620815	1.08\\
7.071928122	1.08\\
7.087939800999	1.08\\
7.103963899	1.08\\
7.119929845001	1.09\\
7.135947473002	1.09\\
7.151914361	1.09\\
7.167821723999	1.09\\
7.183804261999	1.09\\
7.200037890001	1.09\\
7.215937235001	1.09\\
7.231906864	1.09\\
7.247916993	1.09\\
7.264011558001	1.09\\
7.279855331	1.09\\
7.295799588	1.09\\
7.311795422001	1.09\\
7.327806315001	1.1\\
7.343807651001	1.1\\
7.35993825	1.1\\
7.375835223999	1.1\\
7.391803601	1.1\\
7.407802536001	1.1\\
7.423819423	1.1\\
7.439794965	1.1\\
7.45578249	1.1\\
7.471811970001	1.1\\
7.487933068001	1.1\\
7.504011629	1.1\\
7.519897202	1.1\\
7.535927955	1.11\\
7.55193076	1.11\\
7.567843360001	1.11\\
7.584008803	1.11\\
7.599968354	1.11\\
7.615958174	1.11\\
7.631930709999	1.11\\
7.647922169001	1.11\\
7.663923470001	1.11\\
7.679940626999	1.11\\
7.695939091002	1.11\\
7.711927809	1.11\\
7.727896226	1.11\\
7.744501902001	1.12\\
7.760122188999	1.12\\
7.775958130001	1.12\\
7.791912645001	1.12\\
7.807936828001	1.12\\
7.823901660999	1.12\\
7.83995739	1.12\\
7.855806426001	1.12\\
7.871816677	1.12\\
7.887934323	1.12\\
7.903911835001	1.12\\
7.919894062	1.12\\
7.935940681	1.12\\
7.951942258001	1.12\\
7.967847796	1.13\\
7.983828331	1.13\\
7.999944572001	1.13\\
8.022143829001	1.13\\
8.03176674	1.13\\
8.047909237	1.13\\
8.063840939001	1.13\\
8.079811976999	1.13\\
8.095808933001	1.13\\
8.111792668999	1.13\\
8.127803807999	1.14\\
8.143767584	1.14\\
8.159792819	1.14\\
8.175815062	1.14\\
8.191776042	1.14\\
8.207776957001	1.14\\
8.223863644999	1.14\\
8.239792784001	1.14\\
8.255808632	1.14\\
8.271791312	1.15\\
8.287837344	1.15\\
8.303846485001	1.15\\
8.319791502001	1.15\\
8.335786001999	1.15\\
8.351885501	1.15\\
8.367974976999	1.15\\
8.383933296002	1.15\\
8.399879325998	1.16\\
8.415917921999	1.16\\
8.431904730002	1.16\\
8.447809602001	1.16\\
8.463807754	1.16\\
8.480426292	1.16\\
8.495848455	1.16\\
8.511795182001	1.16\\
8.527819916001	1.16\\
8.543771483999	1.17\\
8.559769368	1.17\\
8.575934079001	1.17\\
8.591984141001	1.17\\
8.607964559	1.17\\
8.623929708	1.17\\
8.63993333	1.17\\
8.655922869	1.17\\
8.671857559	1.17\\
8.687815140999	1.18\\
8.703790643999	1.18\\
8.719822272999	1.18\\
8.735780577	1.18\\
8.751865347	1.18\\
8.768006185	1.18\\
8.783808008	1.18\\
8.799849726	1.18\\
8.815937268	1.18\\
8.831816862	1.19\\
8.847786492001	1.19\\
8.863802134998	1.19\\
8.87981191	1.19\\
8.895795060999	1.19\\
8.911797283001	1.19\\
8.927914482001	1.19\\
8.94394534	1.19\\
8.960040434	1.2\\
8.975811973999	1.2\\
8.991831632	1.2\\
9.007812211001	1.2\\
9.025194007	1.2\\
9.040525539002	1.2\\
9.055939248001	1.2\\
9.07185716	1.21\\
9.087951397001	1.21\\
9.103847721001	1.21\\
9.119963303	1.21\\
9.13592218	1.21\\
9.151950377001	1.21\\
9.167910986	1.22\\
9.183894835999	1.22\\
9.199942409001	1.22\\
9.215933937	1.22\\
9.232221652001	1.22\\
9.247938567999	1.23\\
9.263816043999	1.23\\
9.279929269001	1.23\\
9.29594858	1.23\\
9.311895844	1.23\\
9.327808975001	1.23\\
9.343817437	1.24\\
9.359928016001	1.24\\
9.376018765	1.24\\
9.391958271	1.24\\
9.407844412001	1.24\\
9.423884520001	1.25\\
9.439924207998	1.25\\
9.455875856001	1.25\\
9.471799767	1.25\\
9.487777588999	1.25\\
9.503839197001	1.25\\
9.519801817999	1.26\\
9.536132758999	1.26\\
9.551987743	1.26\\
9.567902298	1.26\\
9.583835038	1.26\\
9.599911087	1.27\\
9.615925188999	1.27\\
9.631984957001	1.27\\
9.647803374001	1.27\\
9.663852162001	1.27\\
9.679869024	1.27\\
9.695800271	1.28\\
9.711791077999	1.28\\
9.727751376	1.28\\
9.743795194	1.28\\
9.759800409001	1.28\\
9.77580401	1.28\\
9.791803085999	1.29\\
9.807789944	1.29\\
9.823799459	1.29\\
9.839843907	1.29\\
9.855930792999	1.29\\
9.872123332001	1.3\\
9.887933654999	1.3\\
9.903823108	1.3\\
9.919861011999	1.3\\
9.935943983	1.3\\
9.951934881001	1.3\\
9.967966626	1.31\\
9.983773209999	1.31\\
9.999969949	1.31\\
10.022023677	1.31\\
10.031812503	1.32\\
10.047789814001	1.32\\
10.063802126	1.32\\
10.079824435	1.32\\
10.095803956	1.33\\
10.111796187	1.33\\
10.127803657	1.33\\
10.144013108999	1.34\\
10.159951914	1.34\\
10.175915419001	1.34\\
10.191981407	1.35\\
10.207934940001	1.35\\
10.223879692	1.35\\
10.239865368	1.35\\
10.255943744002	1.36\\
10.271916485001	1.36\\
10.287946271	1.36\\
10.303977044001	1.37\\
10.320086685	1.37\\
10.335939278999	1.37\\
10.351958656	1.37\\
10.367887387001	1.38\\
10.383951074999	1.38\\
10.399952647001	1.38\\
10.415792476999	1.39\\
10.431804431999	1.39\\
10.447893858	1.39\\
10.463911025	1.4\\
10.479906979	1.4\\
10.49600108	1.4\\
10.511940371	1.4\\
10.527927860001	1.41\\
10.543788862999	1.41\\
10.559904528999	1.41\\
10.575815831	1.42\\
10.591904966	1.42\\
10.607940088	1.42\\
10.623940382	1.43\\
10.63992698	1.43\\
10.655854213	1.43\\
10.671820875	1.43\\
10.687909184	1.44\\
10.703931863001	1.44\\
10.719905927	1.44\\
10.735931202999	1.45\\
10.751930940001	1.45\\
10.76780589	1.45\\
10.783806189001	1.45\\
10.799935004	1.46\\
10.816078126002	1.46\\
10.831828468001	1.46\\
10.847791372	1.47\\
10.863825237	1.47\\
10.87993783	1.47\\
10.896050258001	1.48\\
10.911940908999	1.48\\
10.927802487999	1.48\\
10.943797720001	1.48\\
10.959794019999	1.49\\
10.978267125	1.49\\
10.993696895001	1.49\\
11.009072928	1.5\\
11.026768831001	1.5\\
11.042072049	1.51\\
11.057188199	1.51\\
11.072339207001	1.51\\
11.087768632	1.52\\
11.103740702	1.52\\
11.119905232001	1.53\\
11.135982511	1.53\\
11.151918213001	1.54\\
11.16807859	1.54\\
11.183817441	1.55\\
11.199740567	1.55\\
11.215791813999	1.56\\
11.231867684	1.56\\
11.247780074	1.57\\
11.263675295	1.57\\
11.281674249001	1.58\\
11.296812684	1.58\\
11.311906365	1.58\\
11.327961229	1.59\\
11.343884994999	1.59\\
11.359839665001	1.6\\
11.375703053002	1.6\\
11.393987279999	1.61\\
11.409034959	1.61\\
11.424086267	1.62\\
11.439905515	1.62\\
11.455790001	1.63\\
11.474046790001	1.63\\
11.489098848999	1.64\\
11.504157969	1.64\\
11.519741173	1.64\\
11.535802024	1.65\\
11.554238962	1.65\\
11.569597956	1.66\\
11.584941862	1.66\\
11.600338932999	1.67\\
11.615970426001	1.67\\
11.631911976	1.68\\
11.647963541001	1.68\\
11.663931396	1.69\\
11.680162664	1.69\\
11.695947579001	1.7\\
11.711848548	1.7\\
11.7279804	1.71\\
11.743913972	1.71\\
11.759784407	1.71\\
11.775790949	1.72\\
11.794073095999	1.73\\
11.809807503	1.73\\
11.825340401001	1.73\\
11.840925213	1.74\\
11.85656076	1.74\\
11.872015131001	1.75\\
11.887899482001	1.75\\
11.903811130001	1.76\\
11.919799973	1.76\\
11.935804550999	1.77\\
11.951805976999	1.77\\
11.967786421	1.78\\
11.983828278999	1.78\\
11.999807122002	1.78\\
12.023693945	1.79\\
12.032409576	1.8\\
12.047892076	1.8\\
12.063823573	1.81\\
12.079778389	1.82\\
12.095841912001	1.83\\
12.11178041	1.83\\
12.127778479	1.84\\
12.143856534001	1.85\\
12.159815649	1.86\\
12.175896521	1.86\\
12.191847357001	1.87\\
12.207830770001	1.88\\
12.223819497	1.89\\
12.239806254002	1.89\\
12.255837905001	1.9\\
12.271836223	1.91\\
12.287875419001	1.92\\
12.303837838	1.92\\
12.319908884001	1.93\\
12.335792446001	1.94\\
12.351794834002	1.95\\
12.367880386	1.95\\
12.383941707998	1.96\\
12.399805341999	1.97\\
12.415794671	1.98\\
12.431807242001	1.98\\
12.447776907	1.99\\
12.46379225	2\\
12.479773144001	2.01\\
12.495798701	2.01\\
12.511794855002	2.02\\
12.527913006001	2.03\\
12.543793172001	2.04\\
12.559996421002	2.04\\
12.575927613001	2.05\\
12.591932112	2.06\\
12.607792693001	2.07\\
12.623906133999	2.07\\
12.639903521	2.08\\
12.655856852999	2.09\\
12.671791152001	2.1\\
12.687931664	2.1\\
12.703853365	2.11\\
12.719803424	2.12\\
12.735804560002	2.13\\
12.751818244	2.13\\
12.767795225001	2.14\\
12.783806574001	2.15\\
12.799795345999	2.16\\
12.815803879	2.16\\
12.831997074999	2.17\\
12.847793251002	2.18\\
12.864055126999	2.19\\
12.880017574999	2.19\\
12.895962098	2.2\\
12.911974469	2.21\\
12.927912636999	2.22\\
12.943940528	2.22\\
12.959929972	2.23\\
12.975937909001	2.24\\
12.991935030001	2.25\\
13.007852855	2.25\\
13.025182544001	2.26\\
13.040323901001	2.27\\
13.055808460001	2.29\\
13.071751655001	2.3\\
13.087819071001	2.31\\
13.103786244	2.32\\
13.119792296	2.33\\
13.135781971001	2.35\\
13.151741051001	2.36\\
13.167940976999	2.37\\
13.183973875	2.38\\
13.199976624001	2.39\\
13.215947757	2.41\\
13.23283659	2.42\\
13.248498231001	2.43\\
13.263942884998	2.44\\
13.280094602001	2.45\\
13.302140011	2.47\\
13.311759275	2.48\\
13.32775883	2.49\\
13.343963332001	2.5\\
13.359941494999	2.51\\
13.375926099001	2.53\\
13.392017594	2.54\\
13.407948105002	2.55\\
13.423914476	2.56\\
13.439915311001	2.57\\
13.455808257	2.59\\
13.471796087	2.6\\
13.488143622999	2.61\\
13.503959664	2.62\\
13.519972406	2.63\\
13.535881403	2.65\\
13.551914767	2.66\\
13.567803673	2.67\\
13.583861700001	2.68\\
13.599802702999	2.69\\
13.621572249001	2.71\\
13.633459313999	2.72\\
13.648557292	2.73\\
13.664049905001	2.74\\
13.679922429001	2.75\\
13.695805491001	2.76\\
13.711859382	2.78\\
13.727781237999	2.79\\
13.744876164	2.8\\
13.760310385	2.81\\
13.775949046999	2.83\\
13.791813724001	2.84\\
13.808074633001	2.85\\
13.823971989	2.86\\
13.839903812	2.87\\
13.858844482001	2.89\\
13.871803404999	2.9\\
13.887881831	2.91\\
13.903962427	2.92\\
};
\end{axis}
\end{tikzpicture}%}
      \caption{The orientation of the robot over time for
        $K_{\omega}^T = 0.1 K_{\omega, max}^T$}
      \label{fig:13_0.1_max}
    \end{figure}
  \end{minipage}
  \hfill
  \begin{minipage}{0.45\linewidth}
    \begin{figure}[H]
      \scalebox{0.6}{% This file was created by matlab2tikz.
%
%The latest updates can be retrieved from
%  http://www.mathworks.com/matlabcentral/fileexchange/22022-matlab2tikz-matlab2tikz
%where you can also make suggestions and rate matlab2tikz.
%
\definecolor{mycolor1}{rgb}{0.00000,0.44700,0.74100}%
%
\begin{tikzpicture}

\begin{axis}[%
width=4.133in,
height=3.26in,
at={(0.693in,0.44in)},
scale only axis,
xmin=25.102618944001,
xmax=50,
xmajorgrids,
xlabel={Time (seconds)},
ymin=0.95,
ymax=1.05,
ymajorgrids,
ylabel={Distance (meters)},
axis background/.style={fill=white}
]
\addplot [color=mycolor1,solid,forget plot]
  table[row sep=crcr]{%
25.102618944001	1\\
25.117703814999	1\\
25.133273894001	1\\
25.149114404	1\\
25.165053735001	1\\
25.181026446003	1\\
25.19705739	1\\
25.212986478001	1\\
25.229091474003	1\\
25.244987111001	1\\
25.261024267998	1\\
25.277027599003	1\\
25.293062442002	1\\
25.308936379002	1\\
25.325086318001	1\\
25.340928055001	1\\
25.357069541001	1\\
25.373031338002	1\\
25.389055101002	1\\
25.405016521	1\\
25.421064349003	1\\
25.437037378003	1\\
25.453167830002	1\\
25.469016976002	1\\
25.485466467999	1\\
25.500994196	1\\
25.517026117001	1\\
25.532957781998	1\\
25.548946788002	1\\
25.565025085	1\\
25.581044627999	1\\
25.597065793999	1\\
25.613073264	1\\
25.629061675999	1\\
25.644967668999	1\\
25.661090158001	1\\
25.677073638001	1\\
25.693076646	1\\
25.709009612999	1\\
25.725068467	1\\
25.741053154999	1\\
25.757120797001	1\\
25.773044628003	1\\
25.788992334004	1\\
25.805079414002	1\\
25.821243226998	1\\
25.83698646	1\\
25.853248247002	1\\
25.869091359002	1\\
25.885036322999	1\\
25.901002420002	1\\
25.917098756001	1\\
25.932951651997	1\\
25.949053991002	1\\
25.964998799	1\\
25.990045301999	1\\
25.998747736001	1\\
26.013997348	1\\
26.029361778004	1\\
26.044943611001	1\\
26.060961375	1\\
26.076931450001	1\\
26.093039088002	1\\
26.109026188	1\\
26.125212328	1\\
26.141003442997	1\\
26.157056010003	1\\
26.173023166001	1\\
26.189053538002	1\\
26.205048815999	1\\
26.221087389	1\\
26.237082778	1\\
26.253161058999	1\\
26.269062694001	1\\
26.285010616002	1\\
26.301073075001	1\\
26.317142064	1\\
26.333048370999	1\\
26.349053300004	1\\
26.365051519001	1\\
26.380873752999	1\\
26.397112654999	1\\
26.412971892998	1\\
26.429066407002	1\\
26.444966123002	1\\
26.460995628003	1\\
26.477049539002	1\\
26.49310166	1\\
26.509067581002	1\\
26.525066210999	1\\
26.541043168	1\\
26.557068213002	1\\
26.573019866002	1\\
26.591711148999	1\\
26.606981178002	1\\
26.622172765004	1\\
26.637237393002	1\\
26.653126218003	1\\
26.669076564999	1\\
26.685368917	1\\
26.700997227002	1\\
26.71699673	1\\
26.733035701001	1\\
26.748976626004	1\\
26.765105245999	1\\
26.781012789998	1\\
26.797171376999	1\\
26.812994269001	1\\
26.828964756001	1\\
26.844960305001	1\\
26.861390678998	1\\
26.877053626999	1\\
26.893095666001	1\\
26.909013578999	1\\
26.925147791001	1\\
26.941009466	1\\
26.957110549	1\\
26.973091829998	1\\
26.990666543999	1\\
27.005942154	1\\
27.021364588002	1\\
27.037067257	1\\
27.053273027001	1\\
27.069002483002	1\\
27.085113501999	1\\
27.101025124001	1\\
27.117155578999	1\\
27.133283272	1\\
27.149193151001	1\\
27.164943522999	1\\
27.181000157998	1\\
27.197042936001	1\\
27.213019530999	1\\
27.229047347001	1\\
27.245017057999	1\\
27.261247819001	1\\
27.277047584004	1\\
27.293229218998	1\\
27.309028661999	1\\
27.325100484002	1\\
27.341035906003	1\\
27.357103883	1\\
27.373005263001	1\\
27.389040872002	1\\
27.405020842	1\\
27.421136565999	1\\
27.436966851998	1\\
27.455355932999	1\\
27.470420080002	1\\
27.485544985001	1\\
27.501155781003	1\\
27.517057171002	1\\
27.533061284001	1\\
27.549013193001	1\\
27.564970808999	1\\
27.581090182999	1\\
27.597076768998	1\\
27.612956842003	1\\
27.62904396	1\\
27.644926314999	1\\
27.660975498002	1\\
27.677031607003	1\\
27.693080851998	1\\
27.709066959	1\\
27.72491882	1\\
27.741014962998	1\\
27.757155284001	1\\
27.772988513001	1\\
27.789069943001	1\\
27.805090342999	1\\
27.821170943001	1\\
27.837035876004	1\\
27.853041431004	1\\
27.869131604001	1\\
27.885101916001	1\\
27.901076790001	1\\
27.916968922001	1\\
27.932965935002	1\\
27.949034310997	1\\
27.965068696999	1\\
27.986211626	1\\
27.998100347001	1\\
28.013221128003	1\\
28.028965531003	1\\
28.045051305001	1\\
28.061065916001	1\\
28.077014147999	1\\
28.093077002999	1\\
28.109011383	1\\
28.125191968003	1\\
28.141048599	1\\
28.157132897999	1\\
28.173030625	1\\
28.189070589001	1\\
28.205026483002	1\\
28.221064372002	1\\
28.237055811001	1\\
28.253114727002	1\\
28.269004842003	1\\
28.285129724	1\\
28.301058751	1\\
28.317148700001	1\\
28.333033476002	1\\
28.349118453	1\\
28.364986978001	1\\
28.381076710003	1\\
28.397071826001	1\\
28.413023218003	1\\
28.428904188	1\\
28.445070508	1\\
28.461049483998	1\\
28.477001718998	1\\
28.493132990002	1\\
28.509028679001	1\\
28.525064105999	1\\
28.541029141999	1\\
28.557399553002	1\\
28.57305955	1\\
28.58922125	1\\
28.605029383999	1\\
28.621095116002	1\\
28.637048444001	1\\
28.653174102002	1\\
28.669045043	1\\
28.687351648999	1\\
28.702495312001	1\\
28.717589512002	1\\
28.733035384999	1\\
28.749088478001	1\\
28.764955406998	1\\
28.781091592	1\\
28.796990233998	1\\
28.813055051003	1\\
28.829032012002	1\\
28.845025095002	1\\
28.861015097001	1\\
28.877086802002	1\\
28.893200463002	1\\
28.908947280999	1\\
28.925071668	1\\
28.941017351998	1\\
28.957061276001	1\\
28.973039855999	1\\
28.990627763001	1\\
29.005786505002	1\\
29.021129840001	1\\
29.037065512002	1\\
29.053267278	1\\
29.069049117001	1\\
29.084997667	1\\
29.100970900002	1\\
29.117025699002	1\\
29.132977734002	1\\
29.149135344002	1\\
29.165066835999	1\\
29.181103298001	1\\
29.196972647999	1\\
29.212994918999	1\\
29.229097404999	1\\
29.244930737999	1\\
29.261106778	1\\
29.276995314	1\\
29.293156135998	1\\
29.308983143002	1\\
29.325192961998	1\\
29.340983360001	1\\
29.357198490998	1\\
29.37293739	1\\
29.389111166001	1\\
29.405038844998	1\\
29.421224033001	1\\
29.437084881001	1\\
29.453308271	1\\
29.469082067002	1\\
29.485227956002	1\\
29.501072586998	1\\
29.517069883999	1\\
29.53289132	1\\
29.549052463002	1\\
29.564975746998	1\\
29.581008915001	1\\
29.597061527001	1\\
29.61304375	1\\
29.629030300999	1\\
29.644999932	1\\
29.660985503003	1\\
29.676949559998	1\\
29.693218393002	1\\
29.709100904999	1\\
29.725092605	1\\
29.741011222001	1\\
29.756956031003	1\\
29.773100042	1\\
29.789106564	1\\
29.805097895001	1\\
29.821120229001	1\\
29.837135686001	1\\
29.853302959	1\\
29.869024616002	1\\
29.885206135998	1\\
29.900972615002	1\\
29.917083663002	1\\
29.933054597001	1\\
29.949077032002	1\\
29.965117567002	1\\
29.985913145001	1\\
29.997936417	1\\
30.01300875	1\\
30.029178453999	1\\
30.045063597001	1\\
30.061096880002	1\\
30.076943441002	1\\
30.093065295998	1\\
30.108869232998	1\\
30.124895104001	1\\
30.140886668	1\\
30.157063119004	1\\
30.173167564003	1\\
30.188943568001	1\\
30.205083089001	1\\
30.221050886002	1\\
30.237113502003	1\\
30.253089809002	1\\
30.269037620003	1\\
30.284994828999	1\\
30.301047942997	1\\
30.317126042	1\\
30.333400172001	1\\
30.349117665001	1\\
30.365058363003	1\\
30.380970529999	1\\
30.397879593998	1\\
30.413211812001	1\\
30.429016488003	1\\
30.444900493001	1\\
30.461175126999	1\\
30.477072058999	1\\
30.493072392003	1\\
30.509034239003	1\\
30.524944558999	1\\
30.540918142003	1\\
30.557034896	1\\
30.572973382	1\\
30.58921939	1\\
30.605033807999	1\\
30.621090540001	1\\
30.637005420998	1\\
30.653082316002	1\\
30.669001533001	1\\
30.685619297001	1\\
30.700990190999	1\\
30.717040852002	1\\
30.733013534001	1\\
30.748983522004	1\\
30.765032920002	1\\
30.781047952	1\\
30.797132246003	1\\
30.813043198002	1\\
30.829097836998	1\\
30.845124083001	1\\
30.861097316002	1\\
30.877054688	1\\
30.893083421002	1\\
30.909035099999	1\\
30.925100682	1\\
30.941026862999	1\\
30.957084156998	1\\
30.973062195	1\\
30.990588402001	1\\
31.005756605999	1\\
31.021182668	1\\
31.036974523999	1\\
31.053034783001	1\\
31.068953381001	1\\
31.085056100003	1\\
31.101080115002	1\\
31.117133237	1\\
31.133033143002	1\\
31.149056809998	1\\
31.165023302998	1\\
31.181023832001	1\\
31.19702	1\\
31.212963569001	1\\
31.229070403	1\\
31.245050682	1\\
31.261030635003	1\\
31.277019557999	1\\
31.293113349	1\\
31.309020918004	1\\
31.325087753003	1\\
31.341047194001	1\\
31.357052630002	1\\
31.373055334	1\\
31.389112571	1\\
31.405066792	1\\
31.421200001999	1\\
31.437022123002	1\\
31.453202785	1\\
31.468926588002	1\\
31.487375056	1\\
31.502535902001	1\\
31.517630589001	1\\
31.532999822003	1\\
31.549086077004	1\\
31.565026054001	1\\
31.581057138001	1\\
31.597131134003	1\\
31.613115697003	1\\
31.629007221001	1\\
31.645049388001	1\\
31.661064641999	1\\
31.676934601998	1\\
31.693076518002	1\\
31.709001529	1\\
31.725096381001	1\\
31.741013071999	1\\
31.757127930001	1\\
31.773125097001	1\\
31.789064914002	1\\
31.805072023999	1\\
31.821172945	1\\
31.837054877003	1\\
31.853271952004	1\\
31.869023122002	1\\
31.885247317997	1\\
31.901019159001	1\\
31.917031260003	1\\
31.932970413002	1\\
31.949054734002	1\\
31.965022471001	1\\
31.989131994	1\\
31.997818476002	1\\
32.013077127999	1\\
32.028991044999	1\\
32.04507905	1\\
32.061046187001	1\\
32.077198013001	1\\
32.093068247002	1\\
32.109148486001	1\\
32.125067742001	1\\
32.141005949997	1\\
32.157129939999	1\\
32.17302041	1\\
32.189074064999	1\\
32.205043809998	1\\
32.221211558003	1\\
32.237173771	1\\
32.253250287999	1\\
32.268982023003	1\\
32.285252404	1\\
32.301013175	1\\
32.317007793999	1\\
32.333044703999	1\\
32.349097857998	1\\
32.365079452	1\\
32.381055969002	1\\
32.397019333001	1\\
32.412990557999	1\\
32.429071330002	1\\
32.44508771	1\\
32.460954778	1\\
32.477052928002	1\\
32.493083741002	1\\
32.509008859002	1\\
32.525106234002	1\\
32.541051101998	1\\
32.557120843998	1\\
32.573053093998	1\\
32.588989477002	1\\
32.604968289998	1\\
32.621036592999	1\\
32.637025785	1\\
32.653235985001	1\\
32.669022450001	1\\
32.685644217	1\\
32.700964232998	1\\
32.717066925004	1\\
32.732974955998	1\\
32.749078257	1\\
32.765138247002	1\\
32.781031619004	1\\
32.797129228001	1\\
32.813122355999	1\\
32.829123627999	1\\
32.845037725998	1\\
32.861103484002	1\\
32.877038244	1\\
32.893124759003	1\\
32.909047958001	1\\
32.925109236001	1\\
32.941031858998	1\\
32.957044014	1\\
32.973070243001	1\\
32.990585330002	1\\
33.005844488999	1\\
33.021191537999	1\\
33.037048677002	1\\
33.053057351002	1\\
33.068990573002	1\\
33.085072809002	1\\
33.101051297001	1\\
33.117045021	1\\
33.132962686001	1\\
33.149087738999	1\\
33.165059333001	1\\
33.181111304001	1\\
33.197270554001	1\\
33.212972839001	1\\
33.229073916001	1\\
33.245052253998	1\\
33.261078697003	1\\
33.277016009003	1\\
33.293085249001	1\\
33.309050168	1\\
33.324965281003	1\\
33.341059792	1\\
33.357051622002	1\\
33.373087096001	1\\
33.389121647	1\\
33.405070843003	1\\
33.421061351002	1\\
33.437044152001	1\\
33.453472626	1\\
33.469047556	1\\
33.485380414002	1\\
33.501055942002	1\\
33.517041251999	1\\
33.533033833001	1\\
33.548933473	1\\
33.565027666001	1\\
33.581092256001	1\\
33.596989744999	1\\
33.613030649998	1\\
33.629113954998	1\\
33.645375837998	1\\
33.660986829003	1\\
33.677038380002	1\\
33.693110917	1\\
33.709182187001	1\\
33.72507155	1\\
33.741080964001	1\\
33.757105643002	1\\
33.773056907002	1\\
33.789086196003	1\\
33.804949165001	1\\
33.821187869	1\\
33.837003900002	1\\
33.853266414002	1\\
33.869062975998	1\\
33.885132802002	1\\
33.901098476002	1\\
33.917114698002	1\\
33.933099497002	1\\
33.949042480999	1\\
33.965028211998	1\\
33.985995773003	1\\
33.998063872002	1\\
34.013262919999	1\\
34.029011417	1\\
34.045047493001	1\\
34.061060967003	1\\
34.077063615998	1\\
34.093192317002	1\\
34.109046001999	1\\
34.125094258999	1\\
34.141004756001	1\\
34.157216026001	1\\
34.172983140004	1\\
34.188978772	1\\
34.205026549	1\\
34.221165593998	1\\
34.237048130002	1\\
34.253139480999	1\\
34.268971724	1\\
34.287435144997	1\\
34.302567923001	1\\
34.317662599999	1\\
34.333064402001	1\\
34.348949091	1\\
34.365130389	1\\
34.381097426999	1\\
34.398202831002	1\\
34.413284774003	1\\
34.429007703999	1\\
34.445027091	1\\
34.460897707001	1\\
34.477014409001	1\\
34.493040407002	1\\
34.508925559002	1\\
34.524962918	1\\
34.541085414002	1\\
34.557052583001	1\\
34.573031975003	1\\
34.589146102002	1\\
34.60498246	1\\
34.621097304001	1\\
34.637035135998	1\\
34.653173553998	1\\
34.669055362	1\\
34.685200531003	1\\
34.701085536	1\\
34.716952444001	1\\
34.733058180001	1\\
34.749051782002	1\\
34.764997456002	1\\
34.781035188	1\\
34.797050372002	1\\
34.813030482003	1\\
34.829082749001	1\\
34.844943958001	1\\
34.860960689	1\\
34.877025358998	1\\
34.893067688004	1\\
34.909131812001	1\\
34.925073097001	1\\
34.941011612	1\\
34.957116548001	1\\
34.973153578999	1\\
34.990754816998	1\\
35.006022331002	1\\
35.021467056	1\\
35.037076519001	1\\
35.053358437001	1\\
35.069121531003	1\\
35.085042807999	1\\
35.100899948002	1\\
35.117050096001	1\\
35.133006007	1\\
35.149047602002	1\\
35.164894197999	1\\
35.181029959	1\\
35.197076298001	1\\
35.213034239998	1\\
35.228996432004	1\\
35.244951349	1\\
35.261111441002	1\\
35.277015060997	1\\
35.292874176003	1\\
35.309025737	1\\
35.325059831997	1\\
35.341015791001	1\\
35.357128224	1\\
35.373026155999	1\\
35.389051718998	1\\
35.405074292004	1\\
35.421146930001	1\\
35.437037379998	1\\
35.453391085003	1\\
35.469050751999	1\\
35.485273305001	1\\
35.501032222001	1\\
35.517102196999	1\\
35.533045083001	1\\
35.548934604001	1\\
35.565056202	1\\
35.580990988003	1\\
35.597555343003	1\\
35.613050091	1\\
35.629033757	1\\
35.645007111001	1\\
35.661161678002	1\\
35.677020834	1\\
35.693110074002	1\\
35.709171687001	1\\
35.724912686001	1\\
35.74102207	1\\
35.757106906003	1\\
35.773007178002	1\\
35.789111046998	1\\
35.805119393998	1\\
35.82111175	1\\
35.837180358998	1\\
35.853146831002	1\\
35.86901357	1\\
35.885496200001	1\\
35.901133992001	1\\
35.917029228001	1\\
35.932975022	1\\
35.949066540001	1\\
35.965007555001	1\\
35.988941168999	1\\
35.997657630002	1\\
36.013033728001	1\\
36.029087523999	1\\
36.045160693001	1\\
36.061111601002	1\\
36.077034057999	1\\
36.092902875	1\\
36.108982549	1\\
36.125031442002	1\\
36.141048186001	1\\
36.157191436001	1\\
36.173009753998	1\\
36.189115511002	1\\
36.205050724	1\\
36.221280703	1\\
36.237025293	1\\
36.253157781003	1\\
36.269059321	1\\
36.285313340001	1\\
36.300982592999	1\\
36.317098520001	1\\
36.332973103001	1\\
36.349047029999	1\\
36.365071645001	1\\
36.381085768002	1\\
36.396997436001	1\\
36.413071199002	1\\
36.429117648003	1\\
36.445054818001	1\\
36.461038700001	1\\
36.476998397	1\\
36.493001578	1\\
36.509104303998	1\\
36.525107029	1\\
36.541050018002	1\\
36.557386078999	1\\
36.573029654	1\\
36.589149633999	1\\
36.605016481003	1\\
36.621105077	1\\
36.637044560997	1\\
36.653223043999	1\\
36.669060885003	1\\
36.685139179001	1\\
36.701027487999	1\\
36.717011369004	1\\
36.732980786999	1\\
36.748998631001	1\\
36.765036164002	1\\
36.7810393	1\\
36.797091632	1\\
36.813048238003	1\\
36.829158089001	1\\
36.845066890004	1\\
36.861099252003	1\\
36.877089153	1\\
36.893102086003	1\\
36.909067077	1\\
36.925178726002	1\\
36.941019819001	1\\
36.957060705002	1\\
36.972988804001	1\\
36.991093494	1\\
37.006324433999	1\\
37.021715653	1\\
37.037064426003	1\\
37.053021509003	1\\
37.068947317002	1\\
37.084999487004	1\\
37.101033097001	1\\
37.117111966	1\\
37.133089739003	1\\
37.149142904	1\\
37.164866779	1\\
37.181029703	1\\
37.197034911999	1\\
37.213019564	1\\
37.229043544003	1\\
37.244995992001	1\\
37.260953958001	1\\
37.279199746003	1\\
37.294519443001	1\\
37.309751224999	1\\
37.325361305001	1\\
37.340975735001	1\\
37.357068998997	1\\
37.373050530999	1\\
37.389117255002	1\\
37.405199049004	1\\
37.421286099	1\\
37.437230029	1\\
37.453142110001	1\\
37.468962329003	1\\
37.487413104001	1\\
37.502690118001	1\\
37.517891792	1\\
37.534055001999	1\\
37.549415135003	1\\
37.565011638001	1\\
37.581102006001	1\\
37.597018705998	1\\
37.613020734002	1\\
37.629065032002	1\\
37.645099613999	1\\
37.661091535	1\\
37.677083463997	1\\
37.693233460999	1\\
37.708920862999	1\\
37.725099346001	1\\
37.741022529999	1\\
37.757328943001	1\\
37.773062154	1\\
37.789088304001	1\\
37.805109295002	1\\
37.821221734002	1\\
37.837061945	1\\
37.853111260003	1\\
37.868991943001	1\\
37.885035427002	1\\
37.901107411	1\\
37.917122770001	1\\
37.932844774003	1\\
37.949061288002	1\\
37.964846574002	1\\
37.985380106999	1\\
37.9972458	1\\
38.013069479001	1\\
38.028951574002	1\\
38.045086194001	1\\
38.061092386002	1\\
38.076971706002	1\\
38.093086064003	1\\
38.109071985001	1\\
38.125064977002	1\\
38.141046107998	1\\
38.157053217999	1\\
38.173024154	1\\
38.189067315999	1\\
38.205003019001	1\\
38.221127264	1\\
38.237143508999	1\\
38.253177905999	1\\
38.269065204003	1\\
38.28753964	1\\
38.302855486001	1\\
38.318201586003	1\\
38.333593785	1\\
38.349019734002	1\\
38.365120906998	1\\
38.381095946999	1\\
38.396990902001	1\\
38.413022557999	1\\
38.429057485001	1\\
38.445075174	1\\
38.460874495999	1\\
38.477060050999	1\\
38.493025766999	1\\
38.509068661	1\\
38.525098564	1\\
38.541010507	1\\
38.557061402001	1\\
38.573029608002	1\\
38.589309134003	1\\
38.605006286	1\\
38.621044789002	1\\
38.636967707001	1\\
38.655432905999	1\\
38.670785159001	1\\
38.686165455998	1\\
38.701548040001	1\\
38.717049528	1\\
38.733061561001	1\\
38.749030390999	1\\
38.765064011002	1\\
38.781033871003	1\\
38.797046063	1\\
38.813021665001	1\\
38.829137445	1\\
38.845044694001	1\\
38.861142642003	1\\
38.877034918	1\\
38.892966763001	1\\
38.909029589001	1\\
38.925045502999	1\\
38.941073224	1\\
38.957227587998	1\\
38.973056859002	1\\
38.990312193001	1\\
39.005439388001	1\\
39.021129604001	1\\
39.037034336998	1\\
39.055399285	1\\
39.070700533997	1\\
39.086218292	1\\
39.101425786999	1\\
39.117165205002	1\\
39.132946441998	1\\
39.149043605999	1\\
39.165023489003	1\\
39.181040515999	1\\
39.197069524998	1\\
39.213044868001	1\\
39.229036551999	1\\
39.245009549	1\\
39.261099744	1\\
39.277057535	1\\
39.293095109002	1\\
39.309063344002	1\\
39.325038983002	1\\
39.341017022	1\\
39.357083919003	1\\
39.372967429001	1\\
39.389209438	1\\
39.405032995999	1\\
39.421162368001	1\\
39.437029245003	1\\
39.453138536	1\\
39.469008599999	1\\
39.485225619999	1\\
39.500997143998	1\\
39.516962899998	1\\
39.533043026001	1\\
39.549094291001	1\\
39.564972212002	1\\
39.583225117001	1\\
39.598262629002	1\\
39.613270880002	1\\
39.629099945	1\\
39.644988943001	1\\
39.661072815999	1\\
39.677046524998	1\\
39.693098526001	1\\
39.709045599003	1\\
39.725086716	1\\
39.741061140999	1\\
39.757009520001	1\\
39.773073942002	1\\
39.789022745999	1\\
39.805110613003	1\\
39.821198840001	1\\
39.837077138001	1\\
39.853233060002	1\\
39.869025578	1\\
39.887552213002	1\\
39.902712207001	1\\
39.917836736001	1\\
39.933066394001	1\\
39.949035351998	1\\
39.96505746	1\\
39.985831157002	1\\
39.997808625	1\\
40.013108315003	1\\
40.029115154004	1\\
40.045022085	1\\
40.0610735	1\\
40.07701782	1\\
40.093058798001	1\\
40.109057550999	1\\
40.125021878998	1\\
40.141023757	1\\
40.157070616002	1\\
40.17300023	1\\
40.189098837002	1\\
40.205058644001	1\\
40.221124334	1\\
40.237047381001	1\\
40.253418675	1\\
40.269025655999	1\\
40.287471349	1\\
40.302647451001	1\\
40.317891751	1\\
40.333235030999	1\\
40.349062799	1\\
40.364961214001	1\\
40.381091085003	1\\
40.396994865998	1\\
40.413003744004	1\\
40.429074807999	1\\
40.445134477002	1\\
40.461077561001	1\\
40.477040876999	1\\
40.493167888001	1\\
40.509027638001	1\\
40.525101718998	1\\
40.541051935002	1\\
40.55707707	1\\
40.572988829003	1\\
40.589236569001	1\\
40.604976605999	1\\
40.621007596001	1\\
40.636957634003	1\\
40.653205482998	1\\
40.669061970002	1\\
40.685408720002	1\\
40.701122721001	1\\
40.717046203999	1\\
40.733034350003	1\\
40.749127902001	1\\
40.764973633999	1\\
40.781215380997	1\\
40.797044638001	1\\
40.813079078999	1\\
40.829193271	1\\
40.845029679001	1\\
40.861003224	1\\
40.877019623002	1\\
40.893132967999	1\\
40.909033171002	1\\
40.925097529999	1\\
40.941106652001	1\\
40.957122588002	1\\
40.973014276001	1\\
40.989052377999	1\\
41.005018263001	1\\
41.021140708001	1\\
41.037018179001	1\\
41.053253432	1\\
41.068932104001	1\\
41.085150744999	1\\
41.1010368	1\\
41.117022886002	1\\
41.133017347001	1\\
41.149036279	1\\
41.165093717	1\\
41.181005328999	1\\
41.197056349	1\\
41.213181235001	1\\
41.228982164002	1\\
41.244947791001	1\\
41.261058032002	1\\
41.277068383	1\\
41.293097658001	1\\
41.309076629002	1\\
41.325128998002	1\\
41.341037029004	1\\
41.357064551999	1\\
41.372990247002	1\\
41.389013905999	1\\
41.405167119	1\\
41.421128975998	1\\
41.437058575001	1\\
41.453393307	1\\
41.469016577004	1\\
41.485281897	1\\
41.500917983998	1\\
41.517038266003	1\\
41.53306075	1\\
41.549051237	1\\
41.565041609002	1\\
41.581086298001	1\\
41.597101957001	1\\
41.613034450001	1\\
41.629065065999	1\\
41.645061906998	1\\
41.661060242001	1\\
41.677009931	1\\
41.693056383999	1\\
41.709104767003	1\\
41.725070202	1\\
41.741046828	1\\
41.757113287003	1\\
41.773050176999	1\\
41.789122605	1\\
41.805047276997	1\\
41.8212388	1\\
41.836995536	1\\
41.853303598	1\\
41.8690338	1\\
41.885151365998	1\\
41.900996112	1\\
41.916988066998	1\\
41.93303221	1\\
41.949112937001	1\\
41.964851726002	1\\
41.988811782998	1\\
41.997628661	1\\
42.012988634003	1\\
42.029056025002	1\\
42.045072655999	1\\
42.061119079998	1\\
42.077040284001	1\\
42.093841842999	1\\
42.109560615998	1\\
42.124932894001	1\\
42.141003876004	1\\
42.157057494999	1\\
42.173062465001	1\\
42.188986613003	1\\
42.205070567997	1\\
42.221189588002	1\\
42.237105403004	1\\
42.253249515999	1\\
42.269036851998	1\\
42.287392494	1\\
42.302516818001	1\\
42.317627370999	1\\
42.333067571	1\\
42.349088416001	1\\
42.365018847001	1\\
42.381079498002	1\\
42.397036007	1\\
42.412942925	1\\
42.429066724	1\\
42.445021525002	1\\
42.461052067002	1\\
42.477061628003	1\\
42.493098034001	1\\
42.509039589001	1\\
42.525109324002	1\\
42.541020578	1\\
42.557084982003	1\\
42.573066243001	1\\
42.589087890999	1\\
42.605015310002	1\\
42.621092733002	1\\
42.637000874001	1\\
42.653193950001	1\\
42.668977528	1\\
42.687506873997	1\\
42.702817258	1\\
42.717995311001	1\\
42.733352887002	1\\
42.749080914002	1\\
42.765535516999	1\\
42.781067917	1\\
42.797049513001	1\\
42.813029426003	1\\
42.829056208001	1\\
42.845066562001	1\\
42.861070354001	1\\
42.877050406998	1\\
42.893085926003	1\\
42.909023242001	1\\
42.924943505002	1\\
42.941007809002	1\\
42.957130771	1\\
42.972980037999	1\\
42.991114025002	1\\
43.006346035	1\\
43.021703853001	1\\
43.037034925999	1\\
43.053067688004	1\\
43.069027845002	1\\
43.085036955002	1\\
43.100996724	1\\
43.117135611001	1\\
43.133081516003	1\\
43.149060743001	1\\
43.165065046002	1\\
43.181030471001	1\\
43.197066451001	1\\
43.213018857003	1\\
43.229128447999	1\\
43.245035961998	1\\
43.261075716	1\\
43.277047360001	1\\
43.293551541001	1\\
43.309017545998	1\\
43.325063617001	1\\
43.341070461003	1\\
43.357108435002	1\\
43.37304948	1\\
43.389101675999	1\\
43.405062506001	1\\
43.421174217999	1\\
43.437009413002	1\\
43.453190221001	1\\
43.468956899003	1\\
43.487395035	1\\
43.502603153	1\\
43.517713771	1\\
43.533000493001	1\\
43.549046661	1\\
43.565038967003	1\\
43.581099803002	1\\
43.597037771	1\\
43.613280133999	1\\
43.628982497002	1\\
43.644916703003	1\\
43.661138979001	1\\
43.677062344002	1\\
43.693119896	1\\
43.709173973	1\\
43.725100194001	1\\
43.741052314	1\\
43.756996602002	1\\
43.772989409001	1\\
43.789067800004	1\\
43.805026043	1\\
43.821199430001	1\\
43.837040699002	1\\
43.853177892998	1\\
43.869079984997	1\\
43.885446042004	1\\
43.901011315003	1\\
43.917130263001	1\\
43.93293546	1\\
43.949037104001	1\\
43.96543091	1\\
43.986245465001	1\\
43.998337166001	1\\
44.013625166001	1\\
44.029028321	1\\
44.045029476998	1\\
44.061107342003	1\\
44.077047946003	1\\
44.093062697003	1\\
44.109052483002	1\\
44.125099453	1\\
44.141041485001	1\\
44.157046370999	1\\
44.172973923001	1\\
44.189081861004	1\\
44.205022268998	1\\
44.223362651001	1\\
44.238450893998	1\\
44.253555718998	1\\
44.268954147999	1\\
44.285339685997	1\\
44.300986304001	1\\
44.316915449002	1\\
44.333083260003	1\\
44.349006661	1\\
44.365742411	1\\
44.381031113003	1\\
44.397045318001	1\\
44.413040119	1\\
44.429095879002	1\\
44.445054178002	1\\
44.461084119004	1\\
44.477016113003	1\\
44.493052424004	1\\
44.509040104001	1\\
44.525049810002	1\\
44.540999915001	1\\
44.556968790001	1\\
44.575310644001	1\\
44.590560122002	1\\
44.605590763001	1\\
44.621214703999	1\\
44.637043344002	1\\
44.653228266999	1\\
44.669037367001	1\\
44.687474599	1\\
44.702717202	1\\
44.71795141	1\\
44.733244871003	1\\
44.748941672001	1\\
44.764887099	1\\
44.783178719002	1\\
44.798467831002	1\\
44.813757851002	1\\
44.82911905	1\\
44.844989962002	1\\
44.861072060997	1\\
44.877064157002	1\\
44.893010283001	1\\
44.909019136998	1\\
44.925101837998	1\\
44.941034243001	1\\
44.957096080998	1\\
44.973052847001	1\\
44.99067439	1\\
45.005930045998	1\\
45.021232258	1\\
45.037066005002	1\\
45.053178664002	1\\
45.069043849999	1\\
45.085052191002	1\\
45.101067918	1\\
45.117028006001	1\\
45.132868107003	1\\
45.149053467999	1\\
45.165065481003	1\\
45.181023993001	1\\
45.197098647999	1\\
45.213081958001	1\\
45.229084939	1\\
45.245022078999	1\\
45.261074182	1\\
45.277010696	1\\
45.293103269001	1\\
45.309077884999	1\\
45.325134466	1\\
45.341031541001	1\\
45.357046193001	1\\
};
\end{axis}
\end{tikzpicture}%
}
      \caption{The steady state orientation of the robot for
        $K_{\omega}^T = 0.1 K_{\omega, max}^T$}
      \label{fig:13_0.1_max_magnified}
    \end{figure}
  \end{minipage}
\end{minipage}
}

\noindent\makebox[\textwidth][c]{%
\begin{minipage}{\linewidth}
  \begin{minipage}{0.45\linewidth}
    \begin{figure}[H]
      \scalebox{0.6}{% This file was created by matlab2tikz.
%
%The latest updates can be retrieved from
%  http://www.mathworks.com/matlabcentral/fileexchange/22022-matlab2tikz-matlab2tikz
%where you can also make suggestions and rate matlab2tikz.
%
\definecolor{mycolor1}{rgb}{0.00000,0.44700,0.74100}%
%
\begin{tikzpicture}

\begin{axis}[%
width=4.133in,
height=3.26in,
at={(0.693in,0.44in)},
scale only axis,
xmin=0,
xmax=20,
xmajorgrids,
ymin=0,
ymax=1.1,
ymajorgrids,
axis background/.style={fill=white}
]
\addplot [color=mycolor1,solid,forget plot]
  table[row sep=crcr]{%
0	0\\
0.014963641001001	0.01\\
0.0300829570010006	0.02\\
0.0461252509989999	0.02\\
0.0620985889989994	0.03\\
0.0802899249990024	0.04\\
0.0962416730000004	0.04\\
0.111498872999001	0.05\\
0.126782621998001	0.05\\
0.142226162998002	0.06\\
0.158225564998001	0.07\\
0.174244792000003	0.07\\
0.190220315000001	0.08\\
0.206236538999002	0.09\\
0.222171284000002	0.09\\
0.238197066	0.1\\
0.254312570999	0.1\\
0.270187115000002	0.11\\
0.286212675999001	0.12\\
0.302222230999	0.12\\
0.318190291000001	0.13\\
0.334364558998	0.14\\
0.350377155999999	0.14\\
0.366332863998	0.15\\
0.382359511999	0.16\\
0.398331120998003	0.16\\
0.41428977	0.17\\
0.430367764999002	0.18\\
0.446380386999	0.18\\
0.462473701000001	0.19\\
0.478359375	0.19\\
0.494388734999001	0.2\\
0.510294002998002	0.21\\
0.52642793	0.21\\
0.542357472999003	0.22\\
0.558309898001	0.23\\
0.574151733999003	0.23\\
0.590043894001	0.24\\
0.60613891	0.25\\
0.622021358999003	0.25\\
0.638027778000001	0.26\\
0.656257486999001	0.27\\
0.671450350000001	0.27\\
0.68652268	0.28\\
0.702230816999	0.28\\
0.718189540998001	0.29\\
0.734240158998001	0.3\\
0.750222611999001	0.3\\
0.766171243999001	0.31\\
0.782188816999	0.32\\
0.798199779999003	0.32\\
0.814201511999	0.33\\
0.830187457998	0.33\\
0.846131497000001	0.34\\
0.862126744999003	0.35\\
0.878245107000001	0.35\\
0.894270524999	0.36\\
0.910320065998003	0.37\\
0.926210979000002	0.37\\
0.942163073999003	0.38\\
0.95820182	0.39\\
0.974194195999	0.39\\
0.990194333999	0.4\\
1.007554309999	0.4\\
1.022614322998	0.41\\
1.038203219	0.41\\
1.05420043	0.42\\
1.070336817999	0.42\\
1.086305745001	0.42\\
1.102341809999	0.43\\
1.118312676998	0.43\\
1.134338945	0.44\\
1.150348559999	0.44\\
1.166328511999	0.44\\
1.182185064998	0.45\\
1.198186024999	0.45\\
1.214336790001	0.45\\
1.230309816999	0.46\\
1.24632274	0.46\\
1.262343063999	0.47\\
1.278313264	0.47\\
1.294381285999	0.47\\
1.310336111999	0.48\\
1.326300799999	0.48\\
1.342284261	0.49\\
1.358408951999	0.49\\
1.374431104999	0.49\\
1.39034134	0.5\\
1.406345045999	0.5\\
1.422349573999	0.51\\
1.438262555	0.51\\
1.454252861	0.51\\
1.470252788999	0.52\\
1.486261247	0.52\\
1.502282813	0.52\\
1.518309106001	0.53\\
1.534275718998	0.53\\
1.550432361	0.54\\
1.566346124	0.54\\
1.582367962	0.54\\
1.598314975	0.55\\
1.614330402998	0.55\\
1.630327278999	0.56\\
1.646536976999	0.56\\
1.663181271	0.56\\
1.678287422998	0.57\\
1.694315934999	0.57\\
1.710308277	0.58\\
1.726173288	0.58\\
1.742184542	0.58\\
1.75817166	0.59\\
1.774346494999	0.59\\
1.790341656999	0.6\\
1.806263612999	0.6\\
1.822217587	0.6\\
1.838338329	0.61\\
1.854456810999	0.61\\
1.870314666	0.61\\
1.886334649999	0.62\\
1.902328904999	0.62\\
1.918192073	0.63\\
1.934200530001	0.63\\
1.950174017999	0.63\\
1.966336318998	0.64\\
1.982311019001	0.64\\
1.999996603998	0.65\\
2.015141438	0.65\\
2.030245280001	0.65\\
2.046165927999	0.65\\
2.062227875999	0.65\\
2.078207563	0.66\\
2.094183140999	0.66\\
2.110197201	0.66\\
2.126197274999	0.66\\
2.142550913999	0.67\\
2.158470584999	0.67\\
2.174350271	0.67\\
2.190455551998	0.67\\
2.206259195	0.67\\
2.222273059	0.68\\
2.238340451	0.68\\
2.254292850998	0.68\\
2.270189389	0.68\\
2.286201729999	0.69\\
2.302176365999	0.69\\
2.31819211	0.69\\
2.334177684999	0.69\\
2.350354819	0.7\\
2.366348604	0.7\\
2.382466247999	0.7\\
2.398210125999	0.7\\
2.414208941	0.7\\
2.430370280001	0.71\\
2.446336811998	0.71\\
2.462416911999	0.71\\
2.478362757	0.71\\
2.494360028999	0.72\\
2.510356463999	0.72\\
2.526327662	0.72\\
2.542213757	0.72\\
2.558208593	0.73\\
2.574192124	0.73\\
2.590348500999	0.73\\
2.606218725	0.73\\
2.622328651998	0.73\\
2.638393491999	0.74\\
2.654249208	0.74\\
2.670306112999	0.74\\
2.686337900999	0.74\\
2.70233836	0.75\\
2.718336549	0.75\\
2.734227935999	0.75\\
2.750320167999	0.75\\
2.7662406	0.75\\
2.782146049	0.76\\
2.798146993999	0.76\\
2.814213162998	0.76\\
2.830180275999	0.76\\
2.846201292	0.77\\
2.862379824999	0.77\\
2.878320423	0.77\\
2.894405373998	0.77\\
2.910335058	0.78\\
2.92620242	0.78\\
2.942504365	0.78\\
2.958164752998	0.78\\
2.974184918999	0.78\\
2.990336410999	0.79\\
3.008078621	0.79\\
3.024020982	0.79\\
3.03926777	0.79\\
3.054729233999	0.79\\
3.070366504	0.79\\
3.086345985998	0.8\\
3.102353054998	0.8\\
3.118313163	0.8\\
3.134334764999	0.8\\
3.150375549999	0.8\\
3.166330109999	0.8\\
3.182345719	0.8\\
3.198346307998	0.81\\
3.214341511	0.81\\
3.230362135	0.81\\
3.246298949999	0.81\\
3.262346917	0.81\\
3.278320254999	0.81\\
3.294344321998	0.81\\
3.310299868999	0.82\\
3.326285965999	0.82\\
3.342284911999	0.82\\
3.358335434999	0.82\\
3.374311487999	0.82\\
3.390197713001	0.82\\
3.406335052999	0.82\\
3.422330976999	0.83\\
3.438201465999	0.83\\
3.454190644998	0.83\\
3.470221331999	0.83\\
3.486182251999	0.83\\
3.502190668999	0.83\\
3.518232280998	0.83\\
3.534215455999	0.83\\
3.550329092001	0.84\\
3.566454904999	0.84\\
3.582331969999	0.84\\
3.598322461	0.84\\
3.614448804998	0.84\\
3.630412595999	0.84\\
3.646420281999	0.84\\
3.662340348999	0.85\\
3.678391403	0.85\\
3.694305525999	0.85\\
3.710325129	0.85\\
3.726318016001	0.85\\
3.743143816	0.85\\
3.758826448999	0.85\\
3.774364614998	0.86\\
3.790276145	0.86\\
3.806212401998	0.86\\
3.822190880998	0.86\\
3.838203899	0.86\\
3.85427152	0.86\\
3.870269396	0.86\\
3.886306644998	0.87\\
3.902303146	0.87\\
3.918315285999	0.87\\
3.934349486999	0.87\\
3.950125730999	0.87\\
3.966150636999	0.87\\
3.982179956999	0.87\\
3.999843136	0.88\\
4.015192764	0.88\\
4.030377974998	0.88\\
4.046116212999	0.88\\
4.062242112999	0.88\\
4.078284677999	0.88\\
4.096150596001	0.88\\
4.111426998001	0.88\\
4.126709649999	0.88\\
4.14218982	0.88\\
4.158216573999	0.88\\
4.174183937998	0.88\\
4.190190596001	0.88\\
4.206179424	0.89\\
4.222144577999	0.89\\
4.238199801	0.89\\
4.254189024999	0.89\\
4.270210275999	0.89\\
4.286188785	0.89\\
4.302220469999	0.89\\
4.318175697	0.89\\
4.334205863001	0.89\\
4.350334343	0.89\\
4.366331769998	0.89\\
4.382345785	0.89\\
4.398215306999	0.9\\
4.414228410999	0.9\\
4.430192301	0.9\\
4.446393565	0.9\\
4.462352941	0.9\\
4.478316316	0.9\\
4.494308816999	0.9\\
4.51029684	0.9\\
4.526290387001	0.9\\
4.542214747999	0.9\\
4.558358372	0.9\\
4.574287174	0.9\\
4.590240243	0.91\\
4.606059047998	0.91\\
4.622246386	0.91\\
4.638157502001	0.91\\
4.654121442999	0.91\\
4.670149295999	0.91\\
4.686164008999	0.91\\
4.702392382999	0.91\\
4.718372810999	0.91\\
4.734228632	0.91\\
4.750399361999	0.91\\
4.766384566999	0.91\\
4.782321967998	0.92\\
4.798218872	0.92\\
4.814204116999	0.92\\
4.830203527998	0.92\\
4.846339158998	0.92\\
4.862349285	0.92\\
4.878338317999	0.92\\
4.894157295999	0.92\\
4.910478502001	0.92\\
4.926352518999	0.92\\
4.942364865999	0.92\\
4.958421917999	0.92\\
4.974316306	0.93\\
4.990191897001	0.93\\
5.007476512998	0.93\\
5.022703873998	0.93\\
5.038240938	0.93\\
5.054204036999	0.93\\
5.070416476999	0.93\\
5.086331928999	0.93\\
5.102308081999	0.93\\
5.118250354999	0.93\\
5.134309933	0.93\\
5.150314816999	0.93\\
5.166339335998	0.93\\
5.182310733999	0.93\\
5.198315322998	0.93\\
5.214339173998	0.93\\
5.230518493999	0.93\\
5.246339924	0.93\\
5.262257783998	0.93\\
5.278155670999	0.93\\
5.294378103	0.94\\
5.310300083999	0.94\\
5.326715649	0.94\\
5.342338783001	0.94\\
5.358335899999	0.94\\
5.374326626999	0.94\\
5.390306994999	0.94\\
5.406227944	0.94\\
5.422204518999	0.94\\
5.438184766998	0.94\\
5.454206389999	0.94\\
5.470199882999	0.94\\
5.486205348999	0.94\\
5.502190401998	0.94\\
5.518202924	0.94\\
5.534186208999	0.94\\
5.550203747	0.94\\
5.566225934999	0.94\\
5.582209576999	0.94\\
5.598293223999	0.94\\
5.614240582998	0.95\\
5.630336237999	0.95\\
5.646419965999	0.95\\
5.662228839998	0.95\\
5.678198354	0.95\\
5.694331319	0.95\\
5.710443677	0.95\\
5.726378822998	0.95\\
5.743104491001	0.95\\
5.758618562	0.95\\
5.774436528	0.95\\
5.790309156999	0.95\\
5.806419205	0.95\\
5.822206625999	0.95\\
5.838307953998	0.95\\
5.854453802999	0.95\\
5.870348927999	0.95\\
5.889271170999	0.95\\
5.902338541	0.95\\
5.918387278999	0.95\\
5.934324149	0.96\\
5.95031016	0.96\\
5.966314292999	0.96\\
5.982498833999	0.96\\
6.000228729	0.96\\
6.015424758999	0.96\\
6.030702424999	0.96\\
6.046135798	0.96\\
6.062198424	0.96\\
6.078220901998	0.96\\
6.094188677999	0.96\\
6.110325660999	0.96\\
6.126354219	0.96\\
6.142303521999	0.96\\
6.158333918999	0.96\\
6.174266202999	0.96\\
6.190262071998	0.96\\
6.206370416	0.96\\
6.222397549999	0.96\\
6.238334820999	0.96\\
6.254348640999	0.96\\
6.270301597	0.96\\
6.286331863998	0.96\\
6.302316361999	0.96\\
6.318413801	0.96\\
6.334369659	0.96\\
6.350332547998	0.96\\
6.366267639	0.96\\
6.382213303999	0.96\\
6.398193146	0.96\\
6.414228705	0.97\\
6.430188601999	0.97\\
6.446181463999	0.97\\
6.462194134998	0.97\\
6.478156903999	0.97\\
6.494184990999	0.97\\
6.510247154999	0.97\\
6.52618133	0.97\\
6.542183521999	0.97\\
6.558339499	0.97\\
6.574338184	0.97\\
6.590307586998	0.97\\
6.606416903	0.97\\
6.622361384998	0.97\\
6.638143734999	0.97\\
6.654208674999	0.97\\
6.670250164999	0.97\\
6.686204842998	0.97\\
6.702176542	0.97\\
6.718223014999	0.97\\
6.734241494999	0.97\\
6.750160363998	0.97\\
6.766175000999	0.97\\
6.782312871	0.97\\
6.798472947	0.97\\
6.814398065	0.97\\
6.830190934	0.97\\
6.846344865999	0.97\\
6.862376114998	0.97\\
6.878302215999	0.97\\
6.894338656	0.97\\
6.910353773001	0.97\\
6.926284208999	0.97\\
6.942361667999	0.98\\
6.958316103	0.98\\
6.974184187	0.98\\
6.99033668	0.98\\
7.008225314998	0.98\\
7.023665188	0.98\\
7.039243470999	0.98\\
7.054623026998	0.98\\
7.070364634001	0.98\\
7.086345979999	0.98\\
7.102402615999	0.98\\
7.118242691999	0.98\\
7.134359744999	0.98\\
7.150316315	0.98\\
7.166382472999	0.98\\
7.182547274	0.98\\
7.198349789999	0.98\\
7.214330448999	0.98\\
7.230238349001	0.98\\
7.246307833999	0.98\\
7.262237803999	0.98\\
7.278365049999	0.98\\
7.294216976999	0.98\\
7.310168960998	0.98\\
7.326210097999	0.98\\
7.342212515999	0.98\\
7.358277856998	0.98\\
7.374201146999	0.98\\
7.390231277998	0.98\\
7.406264767	0.98\\
7.422315459999	0.98\\
7.438309945999	0.98\\
7.454330945999	0.98\\
7.470276079998	0.98\\
7.486335772999	0.98\\
7.502366295999	0.98\\
7.518447779001	0.98\\
7.534201283001	0.98\\
7.550168178999	0.98\\
7.566151094999	0.98\\
7.582199648998	0.98\\
7.598202743	0.98\\
7.614285769998	0.98\\
7.630240881	0.98\\
7.646139050999	0.98\\
7.662123838999	0.98\\
7.678170057998	0.98\\
7.694301269998	0.98\\
7.710283684	0.98\\
7.726316926	0.98\\
7.742381731998	0.98\\
7.758271268999	0.98\\
7.774484094999	0.98\\
7.790227581999	0.99\\
7.806193396999	0.99\\
7.822361933	0.99\\
7.838208076999	0.99\\
7.854380215	0.99\\
7.870210228	0.99\\
7.886181035	0.99\\
7.902340258999	0.99\\
7.918454606001	0.99\\
7.934359061998	0.99\\
7.950334028999	0.99\\
7.966310167999	0.99\\
7.982321275999	0.99\\
8.000283782999	0.99\\
8.015718375	0.99\\
8.031201585998	0.99\\
8.046643809	0.99\\
8.062113108999	0.99\\
8.078118130998	0.99\\
8.094245530998	0.99\\
8.110382078001	0.99\\
8.12645416	0.99\\
8.142312456999	0.99\\
8.158333747999	0.99\\
8.174291179998	0.99\\
8.190193802	0.99\\
8.206203320999	0.99\\
8.222283847	0.99\\
8.238377151998	0.99\\
8.254259187	0.99\\
8.270192264	0.99\\
8.286177372999	0.99\\
8.302307812	0.99\\
8.318319380998	0.99\\
8.334325543001	0.99\\
8.350347139999	0.99\\
8.366169410999	0.99\\
8.382164031999	0.99\\
8.398169738001	0.99\\
8.414173048	0.99\\
8.430240293999	0.99\\
8.44632542	0.99\\
8.462324354999	0.99\\
8.478336911999	0.99\\
8.494325438999	0.99\\
8.510347076	0.99\\
8.526342027	0.99\\
8.542317619999	0.99\\
8.558342293999	0.99\\
8.574280882999	0.99\\
8.590401679998	0.99\\
8.606407601	0.99\\
8.622430771999	0.99\\
8.638315504	0.99\\
8.654127752001	0.99\\
8.670198053999	0.99\\
8.686279987999	0.99\\
8.702338438	0.99\\
8.718336181	0.99\\
8.734167918999	0.99\\
8.750245785	0.99\\
8.766388198	0.99\\
8.782379958	0.99\\
8.798381695	0.99\\
8.814352365999	0.99\\
8.830363917999	0.99\\
8.846357483999	0.99\\
8.862353340999	0.99\\
8.878359601	0.99\\
8.894227965999	0.99\\
8.910278271999	0.99\\
8.926205711	0.99\\
8.942196427	0.99\\
8.958202695	0.99\\
8.974167593	0.99\\
8.992984431999	0.99\\
9.010720622	0.99\\
9.022989660999	0.99\\
9.038213635	0.99\\
9.054212405001	0.99\\
9.070181532999	0.99\\
9.086328294	0.99\\
9.102427396999	0.99\\
9.118388031	0.99\\
9.134257617998	0.99\\
9.150187236	0.99\\
9.166350143999	0.99\\
9.182296495998	0.99\\
9.198216131	0.99\\
9.214210045999	0.99\\
9.230341706999	1\\
9.246308113998	1\\
9.262319397999	1\\
9.278410712999	1\\
9.294321634001	1\\
9.310447252001	1\\
9.326371036999	1\\
9.342223209999	1\\
9.358332981998	1\\
9.374415813	1\\
9.39030276	1\\
9.406218826999	1\\
9.422342757	1\\
9.438211622	1\\
9.454193102999	1\\
9.470189382999	1\\
9.486213500999	1\\
9.502192646	1\\
9.518227463999	1\\
9.534361625999	1\\
9.550318055	1\\
9.566548820999	1\\
9.582376854999	1\\
9.598338274999	1\\
9.614349102999	1\\
9.630318055	1\\
9.646492904999	1\\
9.662411217998	1\\
9.678232154999	1\\
9.694311743	1\\
9.710375550999	1\\
9.726255286999	1\\
9.743574445	1\\
9.758812014999	1\\
9.774375612999	1\\
9.790310644998	1\\
9.806371931	1\\
9.82250242	1\\
9.838207734999	1\\
9.854305830999	1\\
9.870272257999	1\\
9.886309134001	1\\
9.902339111	1\\
9.918415681999	1\\
9.934324893999	1\\
9.950658177999	1\\
9.966360490999	1\\
9.982311513	1\\
10.000047806999	1\\
10.01519166	1\\
10.030700929998	1\\
10.046235477999	1\\
10.062313988998	1\\
10.078404797998	1\\
10.094426425999	1\\
10.110231351	1\\
10.126195688999	1\\
10.142324177999	1\\
10.158182003998	1\\
10.174125993999	1\\
10.190331358	1\\
10.206499674	1\\
10.222191187	1\\
10.238217083	1\\
10.254345356998	1\\
10.2703485	1\\
10.286326380001	1\\
10.302273499	1\\
10.318286547001	1\\
10.334112705999	1\\
10.350239109999	1\\
10.366152234999	1\\
10.382277693998	1\\
10.398176445	1\\
10.414301583999	1\\
10.430286798	1\\
10.446336157999	1\\
10.46223201	1\\
10.478208414999	1\\
10.494157762998	1\\
10.51048369	1\\
10.526348011	1\\
10.542336882999	1\\
10.558298584999	1\\
10.574346517	1\\
10.590202788	1\\
10.606208945	1\\
10.622337745998	1\\
10.638406259001	1\\
10.654511295	1\\
10.670445013	1\\
10.686215049999	1\\
10.702169609999	1\\
10.718192088999	1\\
10.734255901998	1\\
10.750367191999	1\\
10.766336375	1\\
10.782347549	1\\
10.798431732998	1\\
10.814233536999	1\\
10.830197998998	1\\
10.846199961	1\\
10.862386722999	1\\
10.878469858999	1\\
10.894239666	1\\
10.910195428999	1\\
10.926216724998	1\\
10.942203191999	1\\
10.95815233	1\\
10.974193139999	1\\
10.990286059	1\\
11.008112859001	1\\
11.023539793001	1\\
11.039064222999	1\\
11.054543280001	1\\
11.070345458999	1\\
11.086477997999	1\\
11.102309870001	1\\
11.118455698999	1\\
11.134329917	1\\
11.150332108999	1\\
11.166269529999	1\\
11.182331825001	1\\
11.198342677999	1\\
11.214322288999	1\\
11.230223544	1\\
11.246385854999	1\\
11.262365054998	1\\
11.27833685	1\\
11.294340149999	1\\
11.310253792	1\\
11.326311435999	1\\
11.342329820999	1\\
11.358318639	1\\
11.374303747999	1\\
11.390343396	1\\
11.406440017999	1\\
11.422235152	1\\
11.438195584999	1\\
11.454189945	1\\
11.470193275999	1\\
11.486179472999	1\\
11.502212650999	1\\
11.518171896999	1\\
11.534182232998	1\\
11.55017933	1\\
11.566335833	1\\
11.582261498001	1\\
11.598298525999	1\\
11.614226198	1\\
11.630328217998	1\\
11.646376133999	1\\
11.662245998001	1\\
11.678298754999	1\\
11.694360438	1\\
11.71046161	1\\
11.726377151998	1\\
11.742225477001	1\\
11.75830741	1\\
11.774206744999	1\\
11.790202507999	1\\
11.80634511	1\\
11.822278611999	1\\
11.838216034	1\\
11.854191550999	1\\
11.870186603	1\\
11.886217119999	1\\
11.902195885	1\\
11.918191212999	1\\
11.934230294	1\\
11.95034592	1\\
11.966370538999	1\\
11.982386682998	1\\
12.000117530001	1\\
12.015180308	1\\
12.030381125	1\\
12.046197220999	1\\
12.062308816999	1\\
12.078341795	1\\
12.094300725	1\\
12.110208804001	1\\
12.126406892	1\\
12.142326446998	1\\
12.158307021	1\\
12.174308132	1\\
12.190319375999	1\\
12.206329903	1\\
12.222344633999	1\\
12.238315986999	1\\
12.254263465999	1\\
12.270195544998	1\\
12.286309390999	1\\
12.302311017999	1\\
12.318244825998	1\\
12.334281281999	1\\
12.350328255998	1\\
12.366405418999	1\\
12.382252155001	1\\
12.398306966999	1\\
12.414268601	1\\
12.430234052999	1\\
12.44619167	1\\
12.462199119999	1\\
12.478178164999	1\\
12.494194631	1\\
12.510247479	1\\
12.526378629	1\\
12.542259286999	1\\
12.558209492	1\\
12.574198830999	1\\
12.590215898001	1\\
12.606191795999	1\\
12.622182377998	1\\
12.638204261	1\\
12.654173975	1\\
12.670178342998	1\\
12.686230604	1\\
12.70233525	1\\
12.718202376999	1\\
12.734271606998	1\\
12.750203951999	1\\
12.766204334999	1\\
12.782340115999	1\\
12.79843725	1\\
12.814379250999	1\\
12.830316566999	1\\
12.846329885	1\\
12.862220315	1\\
12.878189377998	1\\
12.894209115999	1\\
12.910181752001	1\\
12.926196357998	1\\
12.942191915998	1\\
12.958179681	1\\
12.974185462999	1\\
12.990213474001	1\\
13.006170465	1\\
13.022159603	1\\
13.038250035	1\\
13.054227879	1\\
13.070196570999	1\\
13.086201059999	1\\
13.102210208	1\\
13.118216144998	1\\
13.134205137998	1\\
13.150202465	1\\
13.166197431	1\\
13.182222198999	1\\
13.198210513	1\\
13.214198908998	1\\
13.230184645	1\\
13.246262695	1\\
13.262348798	1\\
13.278314454	1\\
13.294291028	1\\
13.310316101	1\\
13.326340786999	1\\
13.342393642999	1\\
13.358349011999	1\\
13.374290107	1\\
13.390271597999	1\\
13.406377986999	1\\
13.422099028	1\\
13.438096604999	1\\
13.454126372	1\\
13.470191594	1\\
13.486207153	1\\
13.502305465999	1\\
13.518340903999	1\\
13.534300202999	1\\
13.550441388	1\\
13.566386191999	1\\
13.582313540001	1\\
13.59835441	1\\
13.614311125	1\\
13.630344904999	1\\
13.646209386	1\\
13.662143359001	1\\
13.678167145998	1\\
13.694162233	1\\
13.710147297998	1\\
13.726143382	1\\
13.743604957998	1\\
13.758822342001	1\\
13.774120026998	1\\
13.790197156	1\\
13.806203663999	1\\
13.822204207998	1\\
13.838193861999	1\\
13.854206424	1\\
13.870187013	1\\
13.886196497999	1\\
13.902214136	1\\
13.918169087	1\\
13.934337215999	1\\
13.950305411999	1\\
13.966121060999	1\\
13.982155122	1\\
13.999838226	1\\
14.016006125	1\\
14.031135643999	1\\
14.046359212999	1\\
14.062170388	1\\
14.078187655001	1\\
14.094220401998	1\\
14.1101896	1\\
14.126201995998	1\\
14.142228559999	1\\
14.158187044	1\\
14.174214583999	1\\
14.190213874998	1\\
14.206211893999	1\\
14.222188805	1\\
14.240797312	1\\
14.25607858	1\\
14.271403597999	1\\
14.286853774999	1\\
14.302370683	1\\
14.318367848999	1\\
14.334323771999	1\\
14.350167009998	1\\
14.366155351999	1\\
14.382149339001	1\\
14.398348937	1\\
14.414381698999	1\\
14.430328646999	1\\
14.446192598999	1\\
14.462163202999	1\\
14.478164636999	1\\
14.494118477999	1\\
14.510236838999	1\\
14.526184704	1\\
14.542332184	1\\
14.558333151998	1\\
14.574352887001	1\\
14.590249307998	1\\
14.606177893999	1\\
14.622191548	1\\
14.638330932001	1\\
14.654342771999	1\\
14.670349219999	1\\
14.686208683998	1\\
14.702192349998	1\\
14.718195104	1\\
14.734249636	1\\
14.750315613998	1\\
14.766403837	1\\
14.782237125999	1\\
14.798154973999	1\\
14.814217129	1\\
14.830215545999	1\\
14.846321795999	1\\
14.862392679998	1\\
14.878308070999	1\\
14.894253586	1\\
14.910215247	1\\
14.926331743999	1\\
14.942262214998	1\\
14.958324538999	1\\
14.974310209999	1\\
14.990391628	1\\
15.008914243999	1\\
15.024202103	1\\
15.039761125999	1\\
15.055353305	1\\
15.070958849001	1\\
15.086523163999	1\\
15.102217563	1\\
15.118196250999	1\\
15.134139358	1\\
15.150175031999	1\\
15.166234469999	1\\
15.182167525999	1\\
15.198193868999	1\\
15.214129370998	1\\
15.230141240999	1\\
15.246275874	1\\
15.262318212999	1\\
15.278331566999	1\\
15.294172919998	1\\
15.310135347999	1\\
15.326205594999	1\\
15.342546757	1\\
15.358350153999	1\\
15.374330632999	1\\
15.390198958	1\\
15.406248306999	1\\
15.422201961	1\\
15.438318533001	1\\
15.454183609999	1\\
15.470228530001	1\\
15.486339278999	1\\
15.502283630998	1\\
15.518308344999	1\\
15.534336191	1\\
15.550325810999	1\\
15.566287536999	1\\
15.582378970999	1\\
15.598340076999	1\\
15.614188671999	1\\
15.630363150999	1\\
15.646216345999	1\\
15.662303913	1\\
15.678192267999	1\\
15.694172119999	1\\
15.710137980999	1\\
15.726139427	1\\
15.744045707001	1\\
15.759467243999	1\\
15.77497901	1\\
15.790430172998	1\\
15.806307027	1\\
15.822332156999	1\\
15.838261465	1\\
15.854458609999	1\\
15.870201154999	1\\
15.886171746	1\\
15.902196691	1\\
15.91816841	1\\
15.934198311001	1\\
15.950211758999	1\\
15.966247990999	1\\
15.982313646	1\\
15.999949798	1\\
16.015319462	1\\
16.030872782999	1\\
16.046306107	1\\
16.062293431999	1\\
16.078336340999	1\\
16.094597119999	1\\
16.110214291	1\\
16.126257395998	1\\
16.142331978	1\\
16.158336511999	1\\
16.174300733999	1\\
16.190327082001	1\\
16.206346172998	1\\
16.222305523998	1\\
16.238288975	1\\
16.254360901001	1\\
16.270213113998	1\\
16.286205333999	1\\
16.302336959999	1\\
16.318318441999	1\\
16.334362764999	1\\
16.350292469	1\\
16.36630674	1\\
16.382221428999	1\\
16.398221341999	1\\
16.414355578998	1\\
16.430819379	1\\
16.446431483999	1\\
16.462443545	1\\
16.478207571001	1\\
16.496578407999	1\\
16.512222344999	1\\
16.527518193998	1\\
16.542590131	1\\
16.558200806	1\\
16.574155560999	1\\
16.590164976	1\\
16.606160794998	1\\
16.622202122	1\\
16.638198415998	1\\
16.654325025999	1\\
16.670346196998	1\\
16.686342031	1\\
16.702310086	1\\
16.718329863998	1\\
16.734222138	1\\
16.750273382	1\\
16.766265067999	1\\
16.782314165998	1\\
16.798287568998	1\\
16.814201320999	1\\
16.830201710998	1\\
16.846204906	1\\
16.862210491001	1\\
16.878188774999	1\\
16.894218410999	1\\
16.910199670999	1\\
16.926202033998	1\\
16.942189508999	1\\
16.958190504	1\\
16.974341826	1\\
16.990263902	1\\
};
\end{axis}
\end{tikzpicture}%}
      \caption{The orientation of the robot over time for
        $K_{\omega}^T = 0.2 K_{\omega, max}^T$}
      \label{fig:13_0.2_max}
    \end{figure}
  \end{minipage}
  \hfill
  \begin{minipage}{0.45\linewidth}
    \begin{figure}[H]
      \scalebox{0.6}{% This file was created by matlab2tikz.
%
%The latest updates can be retrieved from
%  http://www.mathworks.com/matlabcentral/fileexchange/22022-matlab2tikz-matlab2tikz
%where you can also make suggestions and rate matlab2tikz.
%
\definecolor{mycolor1}{rgb}{0.00000,0.44700,0.74100}%
%
\begin{tikzpicture}

\begin{axis}[%
width=4.133in,
height=3.26in,
at={(0.693in,0.44in)},
scale only axis,
xmin=10.078617628998,
xmax=25,
xmajorgrids,
xlabel={Time (seconds)},
ymin=0.95,
ymax=1.05,
ymajorgrids,
ylabel={Distance (meters)},
axis background/.style={fill=white}
]
\addplot [color=mycolor1,solid,forget plot]
  table[row sep=crcr]{%
10.078617628998	1\\
10.09453273	1\\
10.110628740997	1\\
10.126662831997	1\\
10.142589829998	1\\
10.158683521	1\\
10.174577838997	1\\
10.190661571999	1\\
10.206636429001	1\\
10.222662684998	1\\
10.238616379997	1\\
10.254669479	1\\
10.270736105999	1\\
10.286725587998	1\\
10.302583829998	1\\
10.321186086998	1\\
10.336582651997	1\\
10.351792404999	1\\
10.367180736	1\\
10.382529225998	1\\
10.398508654999	1\\
10.414634563999	1\\
10.430614896999	1\\
10.446558698998	1\\
10.46264582	1\\
10.478576580998	1\\
10.494917752998	1\\
10.510596742996	1\\
10.526577748997	1\\
10.542604842999	1\\
10.558712725998	1\\
10.574606538998	1\\
10.590631609001	1\\
10.606624507	1\\
10.622534969998	1\\
10.638601055996	1\\
10.654608496998	1\\
10.670672398998	1\\
10.686640238999	1\\
10.702574398998	1\\
10.720928888001	1\\
10.736046661995	1\\
10.751262612995	1\\
10.766947415997	1\\
10.782552861996	1\\
10.798619378998	1\\
10.814725714996	1\\
10.830556493996	1\\
10.846491698998	1\\
10.862562997997	1\\
10.878546325996	1\\
10.894545731999	1\\
10.910401018997	1\\
10.926737618	1\\
10.942439582996	1\\
10.958616211998	1\\
10.974512067997	1\\
10.995218178997	1\\
11.007107019997	1\\
11.022655319996	1\\
11.038610853001	1\\
11.054594187996	1\\
11.070616936997	1\\
11.086776854996	1\\
11.102494546997	1\\
11.121103194	1\\
11.136358598999	1\\
11.151664183998	1\\
11.166991582996	1\\
11.182667055	1\\
11.198466543999	1\\
11.214580675999	1\\
11.230661577999	1\\
11.246614496998	1\\
11.262708486	1\\
11.278561164997	1\\
11.294645765999	1\\
11.310584063995	1\\
11.326628528	1\\
11.342634754997	1\\
11.358635726997	1\\
11.374642114998	1\\
11.3905155	1\\
11.406617906998	1\\
11.422558836998	1\\
11.438594975998	1\\
11.454709308998	1\\
11.470607801998	1\\
11.486646483997	1\\
11.502609291996	1\\
11.521151417	1\\
11.536248388996	1\\
11.551333376999	1\\
11.566620124996	1\\
11.582625340996	1\\
11.598636218998	1\\
11.614697460999	1\\
11.630929361	1\\
11.646685881996	1\\
11.662717915997	1\\
11.678624834	1\\
11.694651533997	1\\
11.710602265995	1\\
11.726738653	1\\
11.742650091	1\\
11.758614348	1\\
11.774586307999	1\\
11.790710522999	1\\
11.806601935001	1\\
11.822705818997	1\\
11.838548374996	1\\
11.854660975998	1\\
11.870587223999	1\\
11.886808183998	1\\
11.902546870998	1\\
11.920982127998	1\\
11.936070891998	1\\
11.951114742996	1\\
11.966629195999	1\\
11.982711800999	1\\
12.000595138001	1\\
12.015813084	1\\
12.031135769997	1\\
12.046655998997	1\\
12.062558675999	1\\
12.078643468998	1\\
12.094634188995	1\\
12.110533511997	1\\
12.126666117996	1\\
12.142614770001	1\\
12.158543902996	1\\
12.174577918999	1\\
12.190623494999	1\\
12.2065865	1\\
12.222669084999	1\\
12.238608591999	1\\
12.254536361996	1\\
12.270605506001	1\\
12.286787591999	1\\
12.302515421997	1\\
12.320955019001	1\\
12.336358033997	1\\
12.351867292999	1\\
12.367440162998	1\\
12.382751345997	1\\
12.398625834995	1\\
12.414694497997	1\\
12.430581341	1\\
12.446596047997	1\\
12.462653300995	1\\
12.478654761997	1\\
12.494592844998	1\\
12.510577342999	1\\
12.526729691998	1\\
12.542661349999	1\\
12.558595054001	1\\
12.574602974999	1\\
12.590528065998	1\\
12.606599937	1\\
12.622582558994	1\\
12.638626863999	1\\
12.654679726997	1\\
12.670563240997	1\\
12.686728467995	1\\
12.702635972	1\\
12.721015107998	1\\
12.736106687	1\\
12.751183452995	1\\
12.766559727997	1\\
12.782602833996	1\\
12.798564404999	1\\
12.814639899998	1\\
12.830552664997	1\\
12.846628455998	1\\
12.862710976997	1\\
12.878603671002	1\\
12.894621139	1\\
12.910611801998	1\\
12.926571487999	1\\
12.942613626995	1\\
12.958699033997	1\\
12.974615974999	1\\
12.995336737	1\\
13.007186433998	1\\
13.022715374996	1\\
13.038475924	1\\
13.054547505997	1\\
13.070654011997	1\\
13.086703977997	1\\
13.102589491997	1\\
13.118839135998	1\\
13.134523499001	1\\
13.150573585995	1\\
13.166548401997	1\\
13.182692712998	1\\
13.198560848	1\\
13.214628063	1\\
13.230660256996	1\\
13.246568508999	1\\
13.262692854996	1\\
13.278581275997	1\\
13.294685774998	1\\
13.31058682	1\\
13.326641481999	1\\
13.342638247997	1\\
13.358605992996	1\\
13.374698377998	1\\
13.390653998997	1\\
13.406614122997	1\\
13.422664079998	1\\
13.438603393997	1\\
13.454654458	1\\
13.470603716	1\\
13.486651609997	1\\
13.502607401997	1\\
13.521006822998	1\\
13.536209634998	1\\
13.551362494	1\\
13.566554600998	1\\
13.582642482998	1\\
13.598477541	1\\
13.614633970997	1\\
13.630589984997	1\\
13.646617915997	1\\
13.662672121998	1\\
13.678623199997	1\\
13.694541960999	1\\
13.710575258	1\\
13.726784991997	1\\
13.742545662998	1\\
13.758597538998	1\\
13.774655187996	1\\
13.790688491997	1\\
13.806612071999	1\\
13.822639376999	1\\
13.838554638996	1\\
13.854666309998	1\\
13.870619877998	1\\
13.886729334995	1\\
13.902603447998	1\\
13.921024232998	1\\
13.936157731999	1\\
13.951372232998	1\\
13.966617298	1\\
13.982575922001	1\\
14.000681494999	1\\
14.015945572998	1\\
14.031236583	1\\
14.046593862999	1\\
14.062707717999	1\\
14.078608028996	1\\
14.094584254997	1\\
14.110584157997	1\\
14.126724238999	1\\
14.142559595997	1\\
14.158699346996	1\\
14.174599667	1\\
14.190628715996	1\\
14.206586141998	1\\
14.222688337998	1\\
14.238663391998	1\\
14.254635830998	1\\
14.270650446999	1\\
14.286802369999	1\\
14.302635684998	1\\
14.318951473999	1\\
14.334589106999	1\\
14.350569269997	1\\
14.366561129997	1\\
14.382574107998	1\\
14.398582630997	1\\
14.414513832996	1\\
14.430628054997	1\\
14.446647963997	1\\
14.462817420998	1\\
14.478599134998	1\\
14.494679105	1\\
14.510614546997	1\\
14.526709043999	1\\
14.542659800999	1\\
14.558516264999	1\\
14.574567693997	1\\
14.590736734997	1\\
14.606631570995	1\\
14.622563946999	1\\
14.638652938999	1\\
14.654617092999	1\\
14.670640971996	1\\
14.686797473995	1\\
14.702492	1\\
14.721090837998	1\\
14.736387496998	1\\
14.751572087998	1\\
14.767719874996	1\\
14.782994334999	1\\
14.798642603001	1\\
14.814670437996	1\\
14.830591516998	1\\
14.846553624001	1\\
14.862723803997	1\\
14.878635709999	1\\
14.894642430996	1\\
14.910542528	1\\
14.926666086998	1\\
14.942622119995	1\\
14.958644834995	1\\
14.974582697998	1\\
14.995349939999	1\\
15.007240955998	1\\
15.022698636997	1\\
15.038648055	1\\
15.054660686997	1\\
15.070671057999	1\\
15.086789008995	1\\
15.102598926998	1\\
15.118972195999	1\\
15.134625711998	1\\
15.150585993	1\\
15.166591392998	1\\
15.182557880997	1\\
15.198489037998	1\\
15.214570069996	1\\
15.230496895996	1\\
15.246578396	1\\
15.262691550995	1\\
15.278495558998	1\\
15.294699107998	1\\
15.310579666996	1\\
15.326560597996	1\\
15.342640927998	1\\
15.358673651997	1\\
15.374618924996	1\\
15.390643062996	1\\
15.406627816998	1\\
15.422534733997	1\\
15.438649557999	1\\
15.454552681	1\\
15.470583760998	1\\
15.486709683998	1\\
15.502630117001	1\\
15.519132719998	1\\
15.534631877998	1\\
15.550561731999	1\\
15.566606722996	1\\
15.582605567997	1\\
15.598598030998	1\\
15.614618084	1\\
15.630686887997	1\\
15.646619313	1\\
15.662719438999	1\\
15.678583909996	1\\
15.694654462998	1\\
15.710586687996	1\\
15.726584390999	1\\
15.742572186997	1\\
15.758497696999	1\\
15.774572794998	1\\
15.790604810997	1\\
15.806392456997	1\\
15.824801280998	1\\
15.839907372997	1\\
15.855044209999	1\\
15.870667339996	1\\
15.886825547001	1\\
15.902604945	1\\
15.921203587998	1\\
15.936479199997	1\\
15.951761691998	1\\
15.967187543999	1\\
15.982698198998	1\\
15.998528877998	1\\
16.014626508	1\\
16.030574798996	1\\
16.046735279999	1\\
16.062405982998	1\\
16.078573816998	1\\
16.094647416996	1\\
16.110621774998	1\\
16.126597580998	1\\
16.142550100998	1\\
16.158627224999	1\\
16.174617921997	1\\
16.190649616997	1\\
16.206637132	1\\
16.222688332001	1\\
16.238577478996	1\\
16.254726930996	1\\
16.270604917996	1\\
16.286714211998	1\\
16.302606612	1\\
16.321011493	1\\
16.336275948998	1\\
16.351500248997	1\\
16.366726072998	1\\
16.382524069996	1\\
16.398641537998	1\\
16.414665992001	1\\
16.430382719998	1\\
16.446629039997	1\\
16.462935248001	1\\
16.478645398998	1\\
16.496754752998	1\\
16.512126206997	1\\
16.527382817997	1\\
16.542655173996	1\\
16.558654515999	1\\
16.574643504997	1\\
16.590683232998	1\\
16.606625760998	1\\
16.622625388996	1\\
16.638627156998	1\\
16.654645996998	1\\
16.670566014999	1\\
16.686749153999	1\\
16.702619702999	1\\
16.718745571995	1\\
16.734699153	1\\
16.750904568997	1\\
16.766491322998	1\\
16.784802723	1\\
16.800128111996	1\\
16.815237296997	1\\
16.830599409	1\\
16.846637313999	1\\
16.862660336998	1\\
16.878604748997	1\\
16.894685893997	1\\
16.910508966	1\\
16.926615821999	1\\
16.942585987999	1\\
16.958578077	1\\
16.974585076996	1\\
16.998561467999	1\\
17.007747083	1\\
17.022955756996	1\\
17.038629300995	1\\
17.054560463997	1\\
17.070552287998	1\\
17.086611345997	1\\
17.102602503998	1\\
17.118756565998	1\\
17.134609376	1\\
17.150813106999	1\\
17.166608753998	1\\
17.182632159996	1\\
17.198496563	1\\
17.214625448998	1\\
17.230562339996	1\\
17.246699057999	1\\
17.262684389	1\\
17.278580671997	1\\
17.294603120998	1\\
17.310555716999	1\\
17.326612804997	1\\
17.342589798996	1\\
17.358630596996	1\\
17.374636613999	1\\
17.390548707996	1\\
17.406606061997	1\\
17.422605332996	1\\
17.438541537998	1\\
17.454724966	1\\
17.470592227997	1\\
17.486859262001	1\\
17.502580022999	1\\
17.518799718998	1\\
17.534591219998	1\\
17.550666072998	1\\
17.566654329998	1\\
17.582624390995	1\\
17.598503353996	1\\
17.614523646999	1\\
17.630559482998	1\\
17.646599819	1\\
17.662676800999	1\\
17.678512216999	1\\
17.694623571999	1\\
17.710614815998	1\\
17.726490830998	1\\
17.742534048	1\\
17.758619287998	1\\
17.774631655998	1\\
17.790667472	1\\
17.806658318997	1\\
17.822543663998	1\\
17.838556831997	1\\
17.854648396996	1\\
17.870600577	1\\
17.886770466	1\\
17.902616970997	1\\
17.918870464001	1\\
17.934571539997	1\\
17.950589824997	1\\
17.966598306999	1\\
17.982459427998	1\\
17.999995216	1\\
18.015304303997	1\\
18.030680647999	1\\
18.046612652001	1\\
18.062720659	1\\
18.078571735996	1\\
18.094664549999	1\\
18.110586306999	1\\
18.126735833	1\\
18.142638400997	1\\
18.15863016	1\\
18.174600180996	1\\
18.190634291996	1\\
18.206583910999	1\\
18.222681139	1\\
18.238592042999	1\\
18.254613327995	1\\
18.270629384998	1\\
18.286847975998	1\\
18.302746443997	1\\
18.318905508	1\\
18.334448353996	1\\
18.350628778	1\\
18.366597018997	1\\
18.382635399998	1\\
18.398625992996	1\\
18.414578193997	1\\
18.430641924999	1\\
18.446729306	1\\
18.462663695995	1\\
18.478569501	1\\
18.494719579998	1\\
18.510579003998	1\\
18.526629261997	1\\
18.542523324997	1\\
18.558632139999	1\\
18.574593399998	1\\
18.590715890995	1\\
18.60659466	1\\
18.622701481999	1\\
18.638496716995	1\\
18.654532136997	1\\
18.670619233997	1\\
18.686727823998	1\\
18.702607749996	1\\
18.720977263996	1\\
18.736052711998	1\\
18.751155255997	1\\
18.766620035999	1\\
18.782560459999	1\\
18.798515185001	1\\
18.814640677998	1\\
18.830464172997	1\\
18.846587925999	1\\
18.862642482998	1\\
18.878581983997	1\\
18.894647855999	1\\
18.910570930996	1\\
18.926650475998	1\\
18.942581531998	1\\
18.958618633999	1\\
18.974591745998	1\\
18.996241072998	1\\
19.008335855999	1\\
19.023667300999	1\\
19.039091577999	1\\
19.054663722	1\\
19.070541206997	1\\
19.086762782997	1\\
19.102601610996	1\\
19.119168223999	1\\
19.134540610001	1\\
19.150639962998	1\\
19.166485973	1\\
19.184812859997	1\\
19.200098503998	1\\
19.215216552998	1\\
19.23057816	1\\
19.246607323998	1\\
19.262574061997	1\\
19.278592612995	1\\
19.294669598	1\\
19.310469938999	1\\
19.326619889996	1\\
19.342608476997	1\\
19.358701761997	1\\
19.374468518997	1\\
19.390646959999	1\\
19.406581073998	1\\
19.422649131001	1\\
19.438631123997	1\\
19.454686307999	1\\
19.470661919998	1\\
19.486845226997	1\\
19.502493948998	1\\
19.520914691998	1\\
19.536191185997	1\\
19.551436838997	1\\
19.56657318	1\\
19.582581561997	1\\
19.598567842999	1\\
19.614620251999	1\\
19.630659734997	1\\
19.646526721001	1\\
19.662656729	1\\
19.678691947998	1\\
19.694678359997	1\\
19.710600571999	1\\
19.726684709999	1\\
19.742513343998	1\\
19.758618892998	1\\
19.774686705998	1\\
19.790618073998	1\\
19.806621469998	1\\
19.822639291	1\\
19.838614064999	1\\
19.854665418999	1\\
19.870606759998	1\\
19.886615298996	1\\
19.902662723995	1\\
19.918907931999	1\\
19.934551080998	1\\
19.950619691998	1\\
19.966654267998	1\\
19.982589681999	1\\
19.999994301998	1\\
20.015166143997	1\\
20.030635061997	1\\
20.046587773998	1\\
20.062546194	1\\
20.078527206997	1\\
20.094706703999	1\\
20.110596878998	1\\
20.126696527996	1\\
20.142553862999	1\\
20.158681376	1\\
20.174585531998	1\\
20.190637868	1\\
20.206603152996	1\\
20.222736338001	1\\
20.238616154999	1\\
20.254682057999	1\\
20.270563762997	1\\
20.286746950996	1\\
20.302653761997	1\\
20.318757613999	1\\
20.334630231999	1\\
20.350679365997	1\\
20.366542695	1\\
20.382737053997	1\\
20.398600507	1\\
20.414638351997	1\\
20.430631429997	1\\
20.446609808998	1\\
20.462594232998	1\\
20.478620219998	1\\
20.494675994999	1\\
20.510599238999	1\\
20.526726580998	1\\
20.542591292999	1\\
20.558568715996	1\\
20.574616606999	1\\
20.590661306	1\\
20.606593069996	1\\
20.622631530998	1\\
20.638601390995	1\\
20.654761952999	1\\
20.670646454998	1\\
20.686721262997	1\\
20.702603488999	1\\
20.721031521	1\\
20.736172702	1\\
20.751307953999	1\\
20.766693711998	1\\
20.782558418999	1\\
20.798571188999	1\\
20.814655279999	1\\
20.830626638996	1\\
20.846607612999	1\\
20.862714476997	1\\
20.878590324997	1\\
20.894594896999	1\\
20.910628574997	1\\
20.926651570999	1\\
20.942605173996	1\\
20.958674814999	1\\
20.974977669994	1\\
20.995923052998	1\\
21.007889402996	1\\
21.023028785999	1\\
21.038406278999	1\\
21.054509058998	1\\
21.070582664997	1\\
21.086597669998	1\\
21.105056886997	1\\
21.120247462998	1\\
21.135532120998	1\\
21.150851440998	1\\
21.166535438999	1\\
21.182653068997	1\\
21.198544937	1\\
21.214603487999	1\\
21.230589528996	1\\
21.246696484997	1\\
21.262673466	1\\
21.278590160999	1\\
21.294650686997	1\\
21.310600125996	1\\
21.326539772999	1\\
21.342609934998	1\\
21.358640311997	1\\
21.374575948998	1\\
21.390602185997	1\\
21.406580335999	1\\
21.422650869	1\\
21.438597285999	1\\
21.454643358997	1\\
21.470644319996	1\\
21.486741432999	1\\
21.502601659996	1\\
21.521228402996	1\\
21.536544172997	1\\
21.551937007	1\\
21.567119862999	1\\
21.582693695	1\\
21.598550664997	1\\
21.614603161995	1\\
21.630616795998	1\\
21.646669233997	1\\
21.662546288998	1\\
21.678581695999	1\\
21.694685522999	1\\
21.710638722996	1\\
21.726616129997	1\\
21.742608445999	1\\
21.758663378998	1\\
21.774629096996	1\\
21.790677723999	1\\
21.806628366997	1\\
21.822613598999	1\\
21.838589087998	1\\
21.854622285999	1\\
21.870632827999	1\\
21.886758917996	1\\
21.902684501995	1\\
21.921165372997	1\\
21.936439726997	1\\
21.951742043999	1\\
21.966985337998	1\\
21.982668758	1\\
22.000307899998	1\\
22.015418724995	1\\
22.030734626999	1\\
22.046663755997	1\\
22.062577357998	1\\
22.080977841999	1\\
22.096110008999	1\\
22.111391057999	1\\
22.12671791	1\\
22.142587603996	1\\
22.158697179997	1\\
22.174613658997	1\\
22.190629909996	1\\
22.206600455998	1\\
22.222621097	1\\
22.238643095997	1\\
22.254868112999	1\\
22.270586116997	1\\
22.286789826996	1\\
22.302729077995	1\\
22.319006614998	1\\
22.334638880997	1\\
22.350592347	1\\
22.366545700996	1\\
22.384988882	1\\
22.400381473999	1\\
22.415577519997	1\\
22.430925353996	1\\
22.446658668999	1\\
22.462656141995	1\\
22.478574224999	1\\
22.494673021999	1\\
22.510560033997	1\\
22.526528979	1\\
22.542586274998	1\\
22.558660257	1\\
22.574614851997	1\\
22.590657380997	1\\
22.606620971996	1\\
22.622622594998	1\\
22.638641896999	1\\
22.654657997997	1\\
22.670642950996	1\\
22.686754196999	1\\
22.702601521999	1\\
};
\end{axis}
\end{tikzpicture}%
}
      \caption{The steady state orientation of the robot for
        $K_{\omega}^T = 0.2 K_{\omega, max}^T$}
      \label{fig:13_0.2_max_magnified}
    \end{figure}
  \end{minipage}
\end{minipage}
}

\noindent\makebox[\textwidth][c]{%
\begin{minipage}{\linewidth}
  \begin{minipage}{0.45\linewidth}
    \begin{figure}[H]
      \scalebox{0.6}{% This file was created by matlab2tikz.
%
%The latest updates can be retrieved from
%  http://www.mathworks.com/matlabcentral/fileexchange/22022-matlab2tikz-matlab2tikz
%where you can also make suggestions and rate matlab2tikz.
%
\definecolor{mycolor1}{rgb}{0.00000,0.44700,0.74100}%
%
\begin{tikzpicture}

\begin{axis}[%
width=4.133in,
height=3.26in,
at={(0.693in,0.44in)},
scale only axis,
xmin=0,
xmax=14,
xmajorgrids,
xlabel={Time (seconds)},
ymin=0,
ymax=1.2,
ymajorgrids,
ylabel={Distance (meters)},
axis background/.style={fill=white}
]
\addplot [color=mycolor1,solid,forget plot]
  table[row sep=crcr]{%
0	0\\
0.0173427809990048	0.01\\
0.0324416520000026	0.02\\
0.048009686999999	0.03\\
0.0639578379999995	0.04\\
0.0799599459990039	0.06\\
0.0960899690000042	0.07\\
0.112039111999005	0.08\\
0.128050004999004	0.09\\
0.144017645000003	0.11\\
0.159998209999003	0.12\\
0.176007808999002	0.13\\
0.191972407999001	0.15\\
0.208036761999004	0.16\\
0.224215745999999	0.17\\
0.239886369000003	0.18\\
0.258035140999004	0.2\\
0.271978326999001	0.21\\
0.287971217999004	0.22\\
0.303931765000001	0.24\\
0.319939917000003	0.25\\
0.336073412999004	0.26\\
0.351906618999004	0.27\\
0.367937327999003	0.29\\
0.383898169999003	0.3\\
0.399911940000005	0.31\\
0.415891466999002	0.33\\
0.431981017999001	0.34\\
0.447998129999004	0.35\\
0.463976037999004	0.36\\
0.480018734000005	0.38\\
0.495989921999	0.39\\
0.511995287999004	0.4\\
0.527999229999006	0.42\\
0.543990372999005	0.43\\
0.559997678000003	0.44\\
0.575982105999	0.45\\
0.591956607000001	0.47\\
0.608006548	0.48\\
0.623974745000005	0.49\\
0.639961473000005	0.51\\
0.656008082000002	0.52\\
0.671962747999005	0.53\\
0.688000175999001	0.54\\
0.704014425999001	0.56\\
0.720009808999002	0.57\\
0.735962043999001	0.58\\
0.751981296999	0.59\\
0.767958492999002	0.61\\
0.783960430999003	0.62\\
0.800013053998999	0.63\\
0.816014546000004	0.65\\
0.832005152000003	0.66\\
0.847961827999003	0.67\\
0.864042528999999	0.68\\
0.880010018999003	0.7\\
0.896028814998001	0.71\\
0.912076004999004	0.72\\
0.928008701000005	0.74\\
0.943985206	0.75\\
0.960007479999006	0.76\\
0.975997576000005	0.77\\
0.992281697999	0.79\\
1.014564687	0.8\\
1.024067584999	0.8\\
1.040053037	0.81\\
1.056233412999	0.81\\
1.072132928	0.81\\
1.088243161	0.82\\
1.104598954	0.82\\
1.120147791999	0.83\\
1.136206087999	0.83\\
1.152113335999	0.83\\
1.168098129999	0.84\\
1.184151919999	0.84\\
1.19995270200101	0.84\\
1.215997803	0.85\\
1.232169454999	0.85\\
1.248139499999	0.85\\
1.264055147	0.86\\
1.280343794999	0.86\\
1.296153777999	0.86\\
1.312111018999	0.87\\
1.328148923	0.87\\
1.344095330999	0.87\\
1.36000672599901	0.88\\
1.376041185	0.88\\
1.39214030800001	0.88\\
1.40822716800001	0.89\\
1.423989491999	0.89\\
1.439998445999	0.9\\
1.456003871999	0.9\\
1.471994669999	0.9\\
1.487959718	0.91\\
1.504028622999	0.91\\
1.52008297	0.91\\
1.536140708999	0.92\\
1.552159820999	0.92\\
1.567993617999	0.92\\
1.584010513999	0.93\\
1.599998225	0.93\\
1.61613411599901	0.93\\
1.63217293199901	0.94\\
1.648124462	0.94\\
1.664131345	0.94\\
1.68014522	0.95\\
1.696120902999	0.95\\
1.712084017999	0.95\\
1.728191169	0.96\\
1.744112353999	0.96\\
1.76003300199901	0.97\\
1.775975510999	0.97\\
1.792005937999	0.97\\
1.808134116999	0.98\\
1.824143831999	0.98\\
1.840105945	0.98\\
1.8561379	0.99\\
1.872107394999	0.99\\
1.888088063999	0.99\\
1.904201750999	1\\
1.920113037999	1\\
1.935997378999	1\\
1.95198065999901	1.01\\
1.968012063999	1.01\\
1.984250423	1.01\\
2.000222086999	1.02\\
2.018086761	1.02\\
2.033438987	1.02\\
2.048715937	1.02\\
2.064161086999	1.02\\
2.08004193	1.02\\
2.09696248	1.02\\
2.112118326	1.02\\
2.1280066	1.02\\
2.144029501	1.02\\
2.160000283	1.02\\
2.176014378	1.02\\
2.191990167	1.02\\
2.20799695100001	1.02\\
2.224010034	1.02\\
2.240015914999	1.02\\
2.255990004999	1.02\\
2.272015916999	1.02\\
2.28798452	1.02\\
2.304011237999	1.02\\
2.320009492999	1.02\\
2.336025116999	1.02\\
2.351985109	1.02\\
2.368024511999	1.02\\
2.384061934999	1.02\\
2.400025169999	1.02\\
2.41600730800001	1.02\\
2.432004444999	1.02\\
2.447996550999	1.02\\
2.463990204999	1.02\\
2.479986394999	1.02\\
2.496114171999	1.02\\
2.511998548999	1.02\\
2.528105348999	1.02\\
2.543991289	1.02\\
2.559982237	1.02\\
2.57835130699901	1.01\\
2.593841645999	1.01\\
2.609182016	1.01\\
2.624693989	1.01\\
2.640106184999	1.01\\
2.656057603	1.01\\
2.672138028999	1.01\\
2.688128013999	1.01\\
2.704068968999	1.01\\
2.720137513	1.01\\
2.736072912999	1.01\\
2.75201283999901	1.01\\
2.768134445999	1.01\\
2.78413956900001	1.01\\
2.800131125	1.01\\
2.816018609	1.01\\
2.832013576	1.01\\
2.848138381999	1.01\\
2.864594779999	1.01\\
2.880106020999	1.01\\
2.896149832999	1.01\\
2.912131094	1.01\\
2.928173388	1.01\\
2.944145718999	1.01\\
2.960246877	1.01\\
2.976134114999	1.01\\
2.992172734999	1.01\\
3.014336347999	1.01\\
3.023994282999	1.01\\
3.039982789	1.01\\
3.055975055999	1.01\\
3.072022030999	1.01\\
3.089404648	1.01\\
3.104525883999	1.01\\
3.119967065999	1.01\\
3.135969707999	1.01\\
3.151963112	1.01\\
3.167881685	1.01\\
3.183882552999	1.01\\
3.199872352999	1.01\\
3.216126211999	1.01\\
3.232121960999	1.01\\
3.248054772	1.01\\
3.263893141999	1.01\\
3.282000468999	1.01\\
3.296982727999	1.01\\
3.31197523	1.01\\
3.32794466	1.01\\
3.34404803200001	1.01\\
3.36000360499901	1.01\\
3.376036622999	1.01\\
3.391994169999	1.01\\
3.408020272999	1.01\\
3.424038221999	1.01\\
3.43999627	1.01\\
3.456063951999	1.01\\
3.472040652999	1.01\\
3.488020315	1.01\\
3.503999326	1.01\\
3.520014142999	1.01\\
3.536031640999	1.01\\
3.551935291	1.01\\
3.568027378	1.01\\
3.583988372999	1.01\\
3.600002763999	1.01\\
3.616003669	1.01\\
3.632000916	1.01\\
3.648024452999	1.01\\
3.664129874999	1.01\\
3.680147954999	1\\
3.69613454300001	1\\
3.712134531998	1\\
3.728132304999	1\\
3.744181626	1\\
3.760179282999	1\\
3.776113317	1\\
3.792163678999	1\\
3.808154892999	1\\
3.824010257999	1\\
3.840255702999	1\\
3.856157173	1\\
3.87215068199901	1\\
3.888586552999	1\\
3.904036880999	1\\
3.920287852999	1\\
3.936251072999	1\\
3.952138435999	1\\
3.968136905999	1\\
3.984198402	1\\
4.000170227999	1\\
4.01795863500001	1\\
4.033494676999	1\\
4.048953303	1\\
4.064365924999	1\\
4.080177446999	1\\
4.096174282999	1\\
4.112139150999	1\\
4.128100987999	1\\
4.14413338500001	1\\
4.160133708999	1\\
4.176172951999	1\\
4.192132678999	1\\
4.208019659	1\\
4.224131803999	1\\
4.240167321999	1\\
4.256142685	1\\
4.272043484	1\\
4.28802321600001	1\\
4.303990324999	1\\
4.320004055	1\\
4.335978810999	1\\
4.351985824999	1\\
4.36801656799901	1\\
4.38404177399901	1\\
4.400022076999	1\\
4.416136735999	1\\
4.431964190999	1\\
4.448020954999	1\\
4.463966527999	1\\
4.479997744	1\\
4.495982783999	1\\
4.511946138999	1\\
4.527886606999	1\\
4.544000839998	1\\
4.560048701	1\\
4.576184051999	1\\
4.592161644	1\\
4.608115047	1\\
4.624133513	1\\
4.640122430999	1\\
4.655993372	1\\
4.67199547999901	1\\
4.688143202999	1\\
4.704156225999	1\\
4.720184859	1\\
4.736134719	1\\
4.752151414999	1\\
4.768141345999	1\\
4.784158747999	1\\
4.800183815999	1\\
4.816136544999	1\\
4.832204116999	1\\
4.848122089	1\\
4.864160558999	1\\
4.880195041	1\\
4.896061787999	1\\
4.912009527	1\\
4.928004134	1\\
4.94400119000001	1\\
4.959912536	1\\
4.976008405999	1\\
4.991983289999	1\\
5.012779287999	1\\
5.024759008999	1\\
5.04009385099901	1\\
5.05622336199901	1\\
5.072282083999	1\\
5.08861982700101	1\\
5.10430498800001	1\\
5.120190035999	1\\
5.136127813	1\\
5.152138987	1\\
5.168139992999	1\\
5.184127277	1\\
5.200145957999	1\\
5.216115391999	1\\
5.232034235999	1\\
5.247993171	1\\
5.26399159100001	1\\
5.279986482999	1\\
5.29598140999901	1\\
5.311995960999	1\\
5.327989415	1\\
5.343986547	1\\
5.360128641999	1\\
5.37617021499901	1\\
5.392136803	1\\
5.408033997998	1\\
5.424029129999	1\\
5.440115359	1\\
5.456121771999	1\\
5.472321863999	1\\
5.488197876	1\\
5.50402145500001	1\\
5.520010282999	1\\
5.536008543	1\\
5.552141883	1\\
5.568152379999	1\\
5.584092518001	1\\
5.600049879999	1\\
5.616031987	1\\
5.631992977999	1\\
5.647959305	1\\
5.663982755999	1\\
5.680137499999	1\\
5.696284862999	1\\
5.711998201	1\\
5.727996670999	1\\
5.743978398	1\\
5.760147359	1\\
5.776294701999	1\\
5.792107846999	1\\
5.807957275	1\\
5.824008813999	1\\
5.840135199	1\\
5.856144423	1\\
5.872120858	1\\
5.888131510999	1\\
5.904277334999	1\\
5.920134544	1\\
5.936276263999	1\\
5.952146112999	1\\
5.968110796	1\\
5.98422916800001	1\\
6.000134937	1\\
6.017955201	1\\
6.033309273	1\\
6.048876217999	1\\
6.06429315	1\\
6.080002895	1\\
6.096062297999	1\\
6.111881426999	1\\
6.127885428999	1\\
6.143868822	1\\
6.159875400999	1\\
6.17599493199901	1\\
6.19195014	1\\
6.207970449	1\\
6.226229096	1\\
6.241365681999	1\\
6.256421803999	1\\
6.271906629999	1\\
6.287928734	1\\
6.30396896600001	1\\
6.319925305	1\\
6.335984506999	1\\
6.351972570999	1\\
6.368024635	1\\
6.383984956999	1\\
6.399954975	1\\
6.415962308999	1\\
6.431950315	1\\
6.44796581900001	1\\
6.463934064998	1\\
6.480011228999	1\\
6.495981088999	1\\
6.512010287999	1\\
6.528015412999	1\\
6.543988515999	1\\
6.560113602	1\\
6.57623202399901	1\\
6.592112581999	1\\
6.608156415	1\\
6.624192093999	1\\
6.640285789999	1\\
6.656030088999	1\\
6.671945850999	1\\
6.687987673999	1\\
6.704132525999	1\\
6.719970772999	1\\
6.736104492999	1\\
6.751992808999	1\\
6.768153673	1\\
6.784008402	1\\
6.800087455999	1\\
6.815991595999	1\\
6.832028021	1\\
6.84800047700001	1\\
6.863998350999	1\\
6.880137013	1\\
6.896031652999	1\\
6.912099544999	1\\
6.92799246600001	1\\
6.943987893999	1\\
6.95999299900001	1\\
6.975959536999	1\\
6.992001123998	1\\
7.012547803	1\\
7.024444907	1\\
7.040010154999	1\\
7.056013525999	1\\
7.071990638	1\\
7.088082072999	1\\
7.10410756900001	1\\
7.120209539	1\\
7.13609844	1\\
7.152152320999	1\\
7.168127973999	1\\
7.18413543	1\\
7.200124492999	1\\
7.216148109999	1\\
7.23212560200001	1\\
7.248110951999	1\\
7.26410397999901	1\\
7.280107596999	1\\
7.296172657999	1\\
7.312138798999	1\\
7.328389675999	1\\
7.344246749999	1\\
7.360117089999	1\\
7.376133881	1\\
7.392209396	1\\
7.408179556999	1\\
7.424222657999	1\\
7.44000718300001	1\\
7.456025652999	1\\
7.472134357999	1\\
7.488177669999	1\\
7.504138473999	1\\
7.520244607999	1\\
7.53604430800001	1\\
7.552130796999	1\\
7.568160241999	1\\
7.584255089999	1\\
7.600133463	1\\
7.616101708999	1\\
7.63204390700001	1\\
7.647998036	1\\
7.664003041	1\\
7.679998017999	1\\
7.695990885	1\\
7.712008453999	1\\
7.727992511	1\\
7.743994619	1\\
7.760015684999	1\\
7.7761136	1\\
7.79213700199901	1\\
7.808005763	1\\
7.824004462999	1\\
7.840012945999	1\\
7.855997082999	1\\
7.871995398	1\\
7.888128537999	1\\
7.904146536999	1\\
7.920063818	1\\
7.935988891999	1\\
7.951988369	1\\
7.9681476	1\\
7.98413824	1\\
8.000612523999	1\\
8.016034668	1\\
8.032197544999	1\\
8.048233449999	1\\
8.064130945999	1\\
8.080042408999	1\\
8.096745635999	1\\
8.11199411599901	1\\
8.127977255999	1\\
8.144003828999	1\\
8.159983020999	1\\
8.176010616999	1\\
8.191979845	1\\
8.207932214	1\\
8.224133254999	1\\
8.240156291	1\\
8.256083689998	1\\
8.272019272	1\\
8.288004355999	1\\
8.303998991999	1\\
8.319992360999	1\\
8.336004511999	1\\
8.351987041	1\\
8.367957978	1\\
8.383950017999	1\\
8.400012699	1\\
8.415996359999	1\\
8.434228331999	1\\
8.449309287999	1\\
8.464665822999	1\\
8.48016129300001	1\\
8.496174355999	1\\
8.512203083999	1\\
8.528132131999	1\\
8.54410647700001	1\\
8.55998614899901	1\\
8.575896333999	1\\
8.591907415999	1\\
8.608007352999	1\\
8.623986579999	1\\
8.639997304999	1\\
8.655966428999	1\\
8.672141423	1\\
8.688144359	1\\
8.703889088	1\\
8.719969639999	1\\
8.735959333999	1\\
8.752001319999	1\\
8.767993622999	1\\
8.783962692	1\\
8.799895529	1\\
8.815891782999	1\\
8.831999345	1\\
8.848015089999	1\\
8.864001756999	1\\
8.88013552399901	1\\
8.896260289999	1\\
8.912318574999	1\\
8.927992385999	1\\
8.943993982	1\\
8.962487746999	1\\
8.977696001	1\\
8.993163161999	1\\
9.014677282	1\\
9.02420156799901	1\\
9.040034315999	1\\
9.055962577999	1\\
9.072012706	1\\
9.087974944999	1\\
9.103997427999	1\\
9.11998719400001	1\\
9.136002798	1\\
9.151967381	1\\
9.167965449999	1\\
9.183957525	1\\
9.19999506900001	1\\
9.215975593	1\\
9.231968679999	1\\
9.247920869999	1\\
9.263926097999	1\\
9.280811586	1\\
9.296480320999	1\\
9.312409241999	1\\
9.327927327001	1\\
9.344082577999	1\\
9.359891491999	1\\
9.375897297	1\\
9.391944416999	1\\
9.40792215999901	1\\
9.423910246999	1\\
9.440049171999	1\\
9.45604045100001	1\\
9.472007007999	1\\
9.488013025	1\\
9.503998536	1\\
9.520007794999	1\\
9.535998773	1\\
9.551945906998	1\\
9.56798611599901	1\\
9.583949917999	1\\
9.599947189998	1\\
9.616007850999	1\\
9.631994909999	1\\
9.647979673999	1\\
9.663996980999	1\\
9.680136555999	1\\
9.696141800999	1\\
9.712104319999	1\\
9.727991311999	1\\
9.744024981999	1\\
9.760014749999	1\\
9.776009364	1\\
9.791996983	1\\
9.808017914	1\\
9.824136475999	1\\
9.84024418899901	1\\
9.8561944	1\\
9.872094957999	1\\
9.888154319999	1\\
9.904002499999	1\\
9.920018839999	1\\
9.936166623998	1\\
9.951939121	1\\
9.967966963	1\\
9.983988049	1\\
9.999951971	1\\
10.015939260999	1\\
10.032005598999	1\\
10.048136503999	1\\
10.064182556999	1\\
10.080009338	1\\
10.095885478999	1\\
10.112037838999	1\\
10.127996753999	1\\
10.144014808	1\\
10.160024895	1\\
10.176027673999	1\\
10.192019724999	1\\
10.208014485999	1\\
10.224042216999	1\\
10.240038312	1\\
10.255981581999	1\\
10.271993081999	1\\
10.288021132999	1\\
10.304004473999	1\\
10.320000217999	1\\
10.335967391999	1\\
10.352050148	1\\
10.367985403999	1\\
10.384071648	1\\
10.400144030999	1\\
10.416052440999	1\\
10.432010529999	1\\
10.448006859999	1\\
10.463956964999	1\\
10.480009227	1\\
10.495998245	1\\
10.511991523999	1\\
10.527991362	1\\
10.54412823	1\\
10.560157239	1\\
10.576144258999	1\\
10.59213452	1\\
10.608117409	1\\
10.624230253999	1\\
10.640230151999	1\\
10.656163489	1\\
10.672074956999	1\\
10.688172822	1\\
10.704001777	1\\
10.720112350999	1\\
10.73611902	1\\
10.752115675999	1\\
10.767999877	1\\
10.784007503	1\\
10.800141559999	1\\
10.816192779	1\\
10.832131358	1\\
10.847994295999	1\\
10.864133174999	1\\
10.880115102999	1\\
10.896005315999	1\\
10.912139624	1\\
10.928084263999	1\\
10.94413697	1\\
10.960135244	1\\
10.976144374	1\\
10.992005461999	1\\
11.012826115999	1\\
11.024793680999	1\\
11.04019215	1\\
11.055966015999	1\\
11.071960581999	1\\
11.088071942999	1\\
11.103997010999	1\\
11.120057802999	1\\
11.13601502	1\\
11.152030071999	1\\
11.167997545999	1\\
11.184006148999	1\\
11.200006673999	1\\
11.21599915	1\\
11.232004174999	1\\
11.247978726	1\\
11.264012991999	1\\
11.279993095	1\\
11.295999477999	1\\
11.311994966	1\\
11.328026615999	1\\
11.343976275	1\\
11.360166435	1\\
11.375999213999	1\\
11.392290683	1\\
11.408147888	1\\
11.424121016	1\\
11.439997378999	1\\
11.455985730999	1\\
11.472001655	1\\
11.488001497999	1\\
11.503976039	1\\
11.519990103999	1\\
11.536173513	1\\
11.552074452001	1\\
11.567994799999	1\\
11.583999799999	1\\
11.600016185	1\\
11.616124333999	1\\
11.632007573	1\\
11.648007164999	1\\
11.664147621999	1\\
11.680134690999	1\\
11.696044439998	1\\
11.711998659	1\\
11.728142168	1\\
11.744165694999	1\\
11.760217459999	1\\
11.776007463	1\\
11.792135256	1\\
11.808139017999	1\\
11.824165194999	1\\
11.840140713	1\\
11.856122474999	1\\
11.872108732999	1\\
11.888149272999	1\\
11.904211131999	1\\
11.920141402	1\\
11.935997955	1\\
11.952007039	1\\
11.967997852	1\\
11.983965866999	1\\
12.0001704	1\\
12.017821334999	1\\
12.033239130999	1\\
12.048577015999	1\\
12.064184214999	1\\
};
\end{axis}
\end{tikzpicture}%
}
      \caption{The orientation of the robot over time for
        $K_{\omega}^T = 0.5 K_{\omega, max}^T$}
      \label{fig:13_0.5_max}
    \end{figure}
  \end{minipage}
  \hfill
  \begin{minipage}{0.45\linewidth}
    \begin{figure}[H]
      \scalebox{0.6}{% This file was created by matlab2tikz.
%
%The latest updates can be retrieved from
%  http://www.mathworks.com/matlabcentral/fileexchange/22022-matlab2tikz-matlab2tikz
%where you can also make suggestions and rate matlab2tikz.
%
\definecolor{mycolor1}{rgb}{0.00000,0.44700,0.74100}%
%
\begin{tikzpicture}

\begin{axis}[%
width=4.133in,
height=3.26in,
at={(0.693in,0.44in)},
scale only axis,
xmin=2.096022590996,
xmax=12,
xmajorgrids,
xlabel={Time (seconds)},
ymin=0.95,
ymax=1.1,
ymajorgrids,
ylabel={Distance (meters)},
axis background/.style={fill=white}
]
\addplot [color=mycolor1,solid,forget plot]
  table[row sep=crcr]{%
2.096022590996	1.01\\
2.112087074997	1.01\\
2.127999820995	1.01\\
2.143977532997	1.01\\
2.159982082996	1.01\\
2.176002051998	1.01\\
2.192015499996	1.01\\
2.208007447998	1.01\\
2.224459890999	1.01\\
2.240007907997	1.01\\
2.256080520996	1.01\\
2.272073768997	1.01\\
2.288037327999	1.01\\
2.303981934997	1.01\\
2.320082363998	1.01\\
2.335971921997	1.01\\
2.352033344997	1.01\\
2.368001910999	1.01\\
2.384014407997	1.01\\
2.400069802997	1.01\\
2.416131327995	1.01\\
2.431916724998	1.01\\
2.447923059997	1.01\\
2.464107966999	1.01\\
2.480397913998	1.01\\
2.495981161999	1.01\\
2.512028392998	1.01\\
2.527942147995	1.01\\
2.543984420997	1.01\\
2.559975512997	1.01\\
2.576014087997	1.01\\
2.591864806	1.01\\
2.608008722999	1.01\\
2.623906190998	1.01\\
2.640003503998	1.01\\
2.656083752998	1.01\\
2.671999486	1.01\\
2.687951456997	1.01\\
2.703973458999	1.01\\
2.720082553997	1.01\\
2.735969706997	1.01\\
2.752007981998	1.01\\
2.768010681	1.01\\
2.784016006996	1.01\\
2.800023075996	1.01\\
2.816077763996	1.01\\
2.831920890999	1.01\\
2.848013479996	1.01\\
2.864017278999	1.01\\
2.882501274998	1.01\\
2.897865827999	1.01\\
2.912965963997	1.01\\
2.928325323997	1.01\\
2.944029927997	1.01\\
2.960025552997	1.01\\
2.975959856998	1.01\\
2.991975328995	1.01\\
3.009379820999	1.01\\
3.024369664997	1.01\\
3.039862670997	1.01\\
3.055913688	1.01\\
3.071898794998	1.01\\
3.090094557999	1.01\\
3.105332393997	1.01\\
3.121056405998	1.01\\
3.136462840996	1.01\\
3.152088174999	1.01\\
3.168152115997	1.01\\
3.184008599995	1.01\\
3.200012526996	1.01\\
3.216128021999	1.01\\
3.231986509998	1.01\\
3.248137299995	1.01\\
3.263992344997	1.01\\
3.280170757999	1.01\\
3.296013243996	1.01\\
3.312025535995	1.01\\
3.328001965996	1.01\\
3.344001796997	1.01\\
3.359951140999	1.01\\
3.376074223995	1.01\\
3.391939231998	1.01\\
3.407993153999	1.01\\
3.423974320995	1.01\\
3.439864591995	1.01\\
3.455862829998	1.01\\
3.471998537998	1.01\\
3.488111950996	1.01\\
3.504100440998	1.01\\
3.520074504997	1.01\\
3.536005727997	1.01\\
3.552023982998	1.01\\
3.568007783996	1.01\\
3.584079817997	1.01\\
3.599904656998	1.01\\
3.616000499996	1.01\\
3.631988195	1.01\\
3.648138542995	1.01\\
3.664005699997	1.01\\
3.682337954994	1.01\\
3.697476035999	1.01\\
3.712532188995	1.01\\
3.728012097999	1.01\\
3.744104485996	1.01\\
3.760151521995	1.01\\
3.775955547001	1.01\\
3.791935686001	1.01\\
3.808074070999	1.01\\
3.823904945999	1.01\\
3.839875599998	1.01\\
3.856115539997	1.01\\
3.871999454998	1.01\\
3.888148531998	1.01\\
3.903980458999	1.01\\
3.919980727997	1.01\\
3.935981923996	1.01\\
3.952035112995	1.01\\
3.968095911995	1.01\\
3.984010623997	1.01\\
4.000018369999	1.01\\
4.017659521	1.01\\
4.032651418999	1.01\\
4.047869371998	1.01\\
4.063874577995	1.01\\
4.079914458996	1.01\\
4.096005841999	1.01\\
4.112081942997	1.01\\
4.127971717995	1.01\\
4.143948535999	1.01\\
4.159966245998	1.01\\
4.176001515999	1.01\\
4.192035113998	1.01\\
4.208093846996	1.01\\
4.223999518997	1.01\\
4.239979562996	1.01\\
4.255949508999	1.01\\
4.271955935997	1.01\\
4.288082318996	1.01\\
4.304107706997	1.01\\
4.320027192997	1.01\\
4.336007606998	1.01\\
4.352161588997	1.01\\
4.367960000999	1.01\\
4.384024780998	1.01\\
4.399953778	1.01\\
4.416036433998	1.01\\
4.432012470997	1.01\\
4.448273601997	1.01\\
4.464014378998	1.01\\
4.480116733997	1.01\\
4.495966077	1.01\\
4.512020438995	1.01\\
4.527905726997	1.01\\
4.543999908996	1.01\\
4.560024286999	1.01\\
4.576027614998	1.01\\
4.592062615997	1.01\\
4.607996896	1.01\\
4.624032684997	1.01\\
4.639967279999	1.01\\
4.658292297996	1.01\\
4.673396821998	1.01\\
4.688626704998	1.01\\
4.703972855995	1.01\\
4.720021835995	1.01\\
4.736027511997	1.01\\
4.752215511997	1.01\\
4.768015495998	1.01\\
4.784094388	1.01\\
4.800026727997	1.01\\
4.816174902996	1.01\\
4.831998632999	1.01\\
4.848084218998	1.01\\
4.864098619995	1.01\\
4.880296481998	1.01\\
4.896001090996	1.01\\
4.912141487999	1.01\\
4.927964249	1.01\\
4.944070557999	1.01\\
4.960089309997	1.01\\
4.975956925999	1.01\\
4.992077724995	1.01\\
5.009622090996	1.01\\
5.024627269997	1.01\\
5.039870412998	1.01\\
5.055909371998	1.01\\
5.071876896999	1.01\\
5.088071426998	1.01\\
5.104006957996	1.01\\
5.120013084995	1.01\\
5.136088203998	1.01\\
5.152102229999	1.01\\
5.167931626999	1.01\\
5.184114257995	1.01\\
5.199990540996	1.01\\
5.216014503998	1.01\\
5.231950211994	1.01\\
5.247993079998	1.01\\
5.263994604	1.01\\
5.282373288998	1.01\\
5.297569038998	1.01\\
5.31282684	1.01\\
5.328114626999	1.01\\
5.344069851997	1.01\\
5.359874322998	1.01\\
5.375935339996	1.01\\
5.391983059997	1.01\\
5.407986154999	1.01\\
5.424131846996	1.01\\
5.440028761997	1.01\\
5.455939915996	1.01\\
5.472064786999	1.01\\
5.487999457996	1.01\\
5.503982773998	1.01\\
5.520074549995	1.01\\
5.535968537998	1.01\\
5.554341542999	1.01\\
5.569569124	1.01\\
5.584772043999	1.01\\
5.599996078998	1.01\\
5.616079997997	1.01\\
5.632035299995	1.01\\
5.648161707996	1.01\\
5.664033800999	1.01\\
5.680219115997	1.01\\
5.695974381	1.01\\
5.712088421997	1.01\\
5.728001973999	1.01\\
5.743968092998	1.01\\
5.759918325996	1.01\\
5.776015178997	1.01\\
5.791996118996	1.01\\
5.810378858997	1.01\\
5.825653858997	1.01\\
5.840804220997	1.01\\
5.856060413998	1.01\\
5.871921021	1.01\\
5.887903015999	1.01\\
5.904025452995	1.01\\
5.920074530994	1.01\\
5.935984302997	1.01\\
5.952089339996	1.01\\
5.967982079998	1.01\\
5.984063086998	1.01\\
6.000036135998	1.01\\
6.017485625995	1.01\\
6.032396501999	1.01\\
6.047861370998	1.01\\
6.065924472996	1.01\\
6.080852286999	1.01\\
6.095860303997	1.01\\
6.114483093998	1.01\\
6.129607878998	1.01\\
6.144791240997	1.01\\
6.160104892998	1.01\\
6.175896732998	1.01\\
6.192079107998	1.01\\
6.208081997997	1.01\\
6.223901037998	1.01\\
6.239986497997	1.01\\
6.256024912998	1.01\\
6.271992492996	1.01\\
6.288084564998	1.01\\
6.304102092998	1.01\\
6.320020285999	1.01\\
6.336036445999	1.01\\
6.352025804996	1.01\\
6.367983368	1.01\\
6.384081613998	1.01\\
6.399972944996	1.01\\
6.416018477997	1.01\\
6.433276476997	1.01\\
6.448492042	1.01\\
6.464084287998	1.01\\
6.480010934997	1.01\\
6.496086227997	1.01\\
6.512013978996	1.01\\
6.527981817997	1.01\\
6.544157539997	1.01\\
6.559866420997	1.01\\
6.576084214996	1.01\\
6.592033288998	1.01\\
6.608084121998	1.01\\
6.624011286999	1.01\\
6.640012746994	1.01\\
6.656096810997	1.01\\
6.671947667995	1.01\\
6.688087415996	1.01\\
6.704009953998	1.01\\
6.720037218998	1.01\\
6.735971438995	1.01\\
6.751939540996	1.01\\
6.770174161999	1.01\\
6.785333134998	1.01\\
6.800459663998	1.01\\
6.816005763996	1.01\\
6.831972441997	1.01\\
6.848126268997	1.01\\
6.864009602997	1.01\\
6.882429409996	1.01\\
6.897504728001	1.01\\
6.912643468994	1.01\\
6.928021256996	1.01\\
6.943989222999	1.01\\
6.960082752994	1.01\\
6.975976484997	1.01\\
6.991959054996	1.01\\
7.009544974995	1.01\\
7.024735305996	1.01\\
7.039987028995	1.01\\
7.056076553997	1.01\\
7.071932912994	1.01\\
7.088013442997	1.01\\
7.103985799	1.01\\
7.120022359997	1.01\\
7.136004528995	1.01\\
7.152015977997	1.01\\
7.168208645996	1.01\\
7.183858207996	1.01\\
7.199871558998	1.01\\
7.217990857998	1.01\\
7.232887071995	1.01\\
7.247900396	1.01\\
7.266062018997	1.01\\
7.281290202995	1.01\\
7.296492874	1.01\\
7.312037605999	1.01\\
7.328008127998	1.01\\
7.344001484997	1.01\\
7.359875908996	1.01\\
7.376175336998	1.01\\
7.391900213997	1.01\\
7.410228070999	1.01\\
7.425436051998	1.01\\
7.440627473999	1.01\\
7.456082768997	1.01\\
7.472004208996	1.01\\
7.488115268997	1.01\\
7.504278662998	1.01\\
7.520087412998	1.01\\
7.536034612999	1.01\\
7.552029009998	1.01\\
7.568067767998	1.01\\
7.584027089001	1.01\\
7.60003836	1.01\\
7.616010768997	1.01\\
7.631999080997	1.01\\
7.648091083	1.01\\
7.663946643997	1.01\\
7.682541497997	1.01\\
7.697945937996	1.01\\
7.713199299	1.01\\
7.729267007999	1.01\\
7.744502891994	1.01\\
7.759989245998	1.01\\
7.776014085995	1.01\\
7.792060959995	1.01\\
7.808125835998	1.01\\
7.823956134998	1.01\\
7.839957257995	1.01\\
7.855972578998	1.01\\
7.871960193996	1.01\\
7.888008793995	1.01\\
7.903956532997	1.01\\
7.920010411999	1.01\\
7.936019188995	1.01\\
7.952001348999	1.01\\
7.96808607	1.01\\
7.983995307995	1.01\\
8.000017798	1.01\\
8.017534567997	1.01\\
8.032851715996	1.01\\
8.048005393997	1.01\\
8.064002013996	1.01\\
8.082506879997	1.01\\
8.097767141994	1.01\\
8.112978775997	1.01\\
8.128452854996	1.01\\
8.143995035999	1.01\\
8.160004926998	1.01\\
8.176070347999	1.01\\
8.191934208	1.01\\
8.208166216999	1.01\\
8.224076636997	1.01\\
8.239946428997	1.01\\
8.256059642998	1.01\\
8.272024819996	1.01\\
8.288015973995	1.01\\
8.303993330997	1.01\\
8.319982954998	1.01\\
8.335955536995	1.01\\
8.352035280994	1.01\\
8.367992421997	1.01\\
8.384056294998	1.01\\
8.400017841999	1.01\\
8.416018140995	1.01\\
8.431991120994	1.01\\
8.448156365997	1.01\\
8.463939656998	1.01\\
8.480089556995	1.01\\
8.495953608997	1.01\\
8.512153938995	1.01\\
8.527977867996	1.01\\
8.544022113998	1.01\\
8.559921480999	1.01\\
8.576017733997	1.01\\
8.592090067997	1.01\\
8.608066702	1.01\\
8.623963300995	1.01\\
8.640011297996	1.01\\
8.656022285999	1.01\\
8.672001055996	1.01\\
8.688014990997	1.01\\
8.703967150997	1.01\\
8.720020855999	1.01\\
8.736028431995	1.01\\
8.752131829998	1.01\\
8.767995082996	1.01\\
8.784071063995	1.01\\
8.799988409996	1.01\\
8.816074528	1.01\\
8.832006647995	1.01\\
8.848014097999	1.01\\
8.863944067997	1.01\\
8.882396634998	1.01\\
8.897624438	1.01\\
8.912811237999	1.01\\
8.928062945995	1.01\\
8.943918809997	1.01\\
8.960028584999	1.01\\
8.975903560997	1.01\\
8.991999079994	1.01\\
9.009394887997	1.01\\
9.025091055996	1.01\\
9.040406477997	1.01\\
9.056062438995	1.01\\
9.071989268997	1.01\\
9.088025359997	1.01\\
9.103994106994	1.01\\
9.120091445999	1.01\\
9.135979921997	1.01\\
9.152074293995	1.01\\
9.168032206997	1.01\\
9.184071659996	1.01\\
9.199993144997	1.01\\
9.216092573997	1.01\\
9.232028326996	1.01\\
9.248111282997	1.01\\
9.263960920997	1.01\\
9.282480664997	1.01\\
9.297786695999	1.01\\
9.312942385998	1.01\\
9.328341937996	1.01\\
9.343977845997	1.01\\
9.359999011997	1.01\\
9.376140531998	1.01\\
9.392008114998	1.01\\
9.408033079998	1.01\\
9.424021728996	1.01\\
9.440014600998	1.01\\
9.455977583999	1.01\\
9.471978448997	1.01\\
9.488089850998	1.01\\
9.504032063	1.01\\
9.520001875	1.01\\
9.535993872997	1.01\\
9.551936670997	1.01\\
9.567976955997	1.01\\
9.584354043999	1.01\\
9.599936386997	1.01\\
9.616010492996	1.01\\
9.631958314998	1.01\\
9.647986642998	1.01\\
9.663979181995	1.01\\
9.680228735996	1.01\\
9.695999494995	1.01\\
9.712017277996	1.01\\
9.727965348995	1.01\\
9.744095384998	1.01\\
9.759991314995	1.01\\
9.776072371998	1.01\\
9.793826314998	1.01\\
9.808872491997	1.01\\
9.824083211998	1.01\\
9.840001304996	1.01\\
9.856074016998	1.01\\
9.871961565998	1.01\\
9.888026441997	1.01\\
9.904026504997	1.01\\
9.920030311996	1.01\\
9.936072265999	1.01\\
9.952201606994	1.01\\
9.968005811996	1.01\\
9.984095452995	1.01\\
10.000009661999	1.01\\
10.017750520996	1.01\\
10.032952300995	1.01\\
10.048121976997	1.01\\
10.063995716999	1.01\\
10.080287413998	1.01\\
10.096002410999	1.01\\
10.111998355999	1.01\\
10.127915986999	1.01\\
10.143968589996	1.01\\
10.159985159996	1.01\\
10.176123293995	1.01\\
10.192006104996	1.01\\
10.207968183994	1.01\\
10.223857250999	1.01\\
10.240689630997	1.01\\
10.255976891998	1.01\\
10.272002358997	1.01\\
10.288006637997	1.01\\
10.304031714996	1.01\\
10.320141505997	1.01\\
10.335861390999	1.01\\
10.351897709999	1.01\\
10.368074546997	1.01\\
10.383944612995	1.01\\
10.400005851997	1.01\\
10.416335112995	1.01\\
10.432092190998	1.01\\
10.448224231994	1.01\\
10.464025846996	1.01\\
10.480423247997	1.01\\
10.496021248997	1.01\\
10.512004473999	1.01\\
10.527981880997	1.01\\
10.544123625995	1.01\\
10.559876181999	1.01\\
10.575984897995	1.01\\
10.592033626999	1.01\\
10.608020195	1.01\\
10.624070899998	1.01\\
10.639919933998	1.01\\
10.656032898998	1.01\\
10.671968596996	1.01\\
10.687903911999	1.01\\
10.703897055996	1.01\\
10.720035947998	1.01\\
10.736002751995	1.01\\
10.752019338997	1.01\\
10.768019917999	1.01\\
10.783941722999	1.01\\
10.799916480999	1.01\\
10.816083086998	1.01\\
10.832026561996	1.01\\
10.848107057999	1.01\\
10.864003520996	1.01\\
10.882472154995	1.01\\
10.897605393997	1.01\\
10.912744730999	1.01\\
10.928127832996	1.01\\
10.943999588997	1.01\\
10.959916672996	1.01\\
10.975983328995	1.01\\
10.991994607998	1.01\\
11.009596246994	1.01\\
11.024956900997	1.01\\
11.040209130997	1.01\\
11.056028854996	1.01\\
11.072069670997	1.01\\
11.088087919998	1.01\\
11.103981196995	1.01\\
11.120028063	1.01\\
11.135937722	1.01\\
11.152125674995	1.01\\
11.168003861999	1.01\\
11.184904405998	1.01\\
11.200194875995	1.01\\
11.216078585998	1.01\\
11.231996704998	1.01\\
11.248372303997	1.01\\
11.264019241997	1.01\\
11.282447042999	1.01\\
11.297554653999	1.01\\
11.312715811001	1.01\\
11.328061202999	1.01\\
11.344018241997	1.01\\
11.360026884998	1.01\\
11.375984916996	1.01\\
11.392173488998	1.01\\
11.408031960998	1.01\\
11.424080623997	1.01\\
11.440022405994	1.01\\
11.455980049999	1.01\\
11.472008080997	1.01\\
11.488083299999	1.01\\
11.503982007995	1.01\\
11.520129991997	1.01\\
11.536059746998	1.01\\
11.552069782997	1.01\\
11.567972200996	1.01\\
11.584080803997	1.01\\
11.600012117	1.01\\
11.616289718994	1.01\\
11.631955003998	1.01\\
11.648127553997	1.01\\
11.664001626999	1.01\\
11.682357248997	1.01\\
11.697463000995	1.01\\
11.712666252998	1.01\\
11.728430869999	1.01\\
11.744104891994	1.01\\
11.759953337997	1.01\\
11.775947126999	1.01\\
};
\end{axis}
\end{tikzpicture}%
}
      \caption{The steady state orientation of the robot for
        $K_{\omega}^T = 0.5 K_{\omega, max}^T$}
      \label{fig:13_0.5_max_magnified}
    \end{figure}
  \end{minipage}
\end{minipage}
}

\noindent\makebox[\textwidth][c]{%
\begin{minipage}{\linewidth}
  \begin{minipage}{0.45\linewidth}
    \begin{figure}[H]
      \scalebox{0.6}{% This file was created by matlab2tikz.
%
%The latest updates can be retrieved from
%  http://www.mathworks.com/matlabcentral/fileexchange/22022-matlab2tikz-matlab2tikz
%where you can also make suggestions and rate matlab2tikz.
%
\definecolor{mycolor1}{rgb}{0.00000,0.44700,0.74100}%
%
\begin{tikzpicture}

\begin{axis}[%
width=4.133in,
height=3.26in,
at={(0.693in,0.44in)},
scale only axis,
xmin=0,
xmax=25,
xmajorgrids,
xlabel={Time (seconds)},
ymin=0,
ymax=1.2,
ymajorgrids,
ylabel={Distance (meters)},
axis background/.style={fill=white}
]
\addplot [color=mycolor1,solid,forget plot]
  table[row sep=crcr]{%
0	0\\
0.0174626789989993	0.01\\
0.0324618559999987	0.02\\
0.0477497429989986	0.03\\
0.0637612349979989	0.05\\
0.079796608998999	0.06\\
0.0957639009989984	0.07\\
0.111869542998999	0.08\\
0.127875028998997	0.1\\
0.143826290998999	0.11\\
0.159866701998997	0.12\\
0.175886700998999	0.14\\
0.191928839998998	0.15\\
0.207950666998997	0.16\\
0.223924871998999	0.17\\
0.239894341998998	0.19\\
0.255991995998997	0.2\\
0.271899179998998	0.21\\
0.288006640998997	0.23\\
0.304006428999998	0.24\\
0.320006823998999	0.25\\
0.335976723999999	0.26\\
0.352267300997999	0.28\\
0.367867536998	0.29\\
0.384008771998998	0.3\\
0.400108340999	0.32\\
0.415965058999997	0.33\\
0.431999845998999	0.34\\
0.447994988998998	0.35\\
0.463866821998998	0.37\\
0.479971577999998	0.38\\
0.496002675998997	0.39\\
0.512302989999	0.41\\
0.527770042998999	0.42\\
0.543726174998999	0.43\\
0.561921533999	0.45\\
0.577121719998997	0.46\\
0.592586705998998	0.47\\
0.608058571998	0.48\\
0.623936478999998	0.5\\
0.640011511999	0.51\\
0.656013784998998	0.52\\
0.672201122998998	0.53\\
0.688035005998999	0.55\\
0.704007550998997	0.56\\
0.719867597998999	0.57\\
0.735888818999	0.58\\
0.751866711998999	0.6\\
0.767982162999999	0.61\\
0.783882118998997	0.62\\
0.799887584999997	0.64\\
0.815874598998997	0.65\\
0.831871204999	0.66\\
0.848007626998998	0.67\\
0.864026618998997	0.69\\
0.880011332998998	0.7\\
0.896030543998998	0.71\\
0.912229077998999	0.73\\
0.928031542998999	0.74\\
0.943890847998999	0.75\\
0.959805181998998	0.76\\
0.975953391998999	0.78\\
0.991871729998998	0.79\\
1.007894388999	0.8\\
1.023880014999	0.8\\
1.039887780999	0.81\\
1.056002774999	0.82\\
1.071975200999	0.82\\
1.087933700999	0.83\\
1.103999497999	0.83\\
1.120049143999	0.84\\
1.136025082999	0.84\\
1.151995764999	0.85\\
1.168004522999	0.85\\
1.183996713998	0.86\\
1.20001125	0.86\\
1.216046422999	0.87\\
1.231892632999	0.87\\
1.247979302999	0.88\\
1.263979952999	0.88\\
1.280011146999	0.89\\
1.296000139999	0.89\\
1.312053559998	0.9\\
1.328035536999	0.9\\
1.345222101999	0.91\\
1.360627019999	0.92\\
1.375937640999	0.92\\
1.392156120999	0.93\\
1.408007920999	0.93\\
1.424012589999	0.94\\
1.439946263999	0.94\\
1.455990391999	0.95\\
1.471904375999	0.95\\
1.488017804999	0.96\\
1.503984559	0.96\\
1.519856070999	0.97\\
1.535872807999	0.97\\
1.552002033999	0.98\\
1.567960683999	0.98\\
1.583882012998	0.99\\
1.599868094999	0.99\\
1.615869250999	1\\
1.631882064	1\\
1.648002665999	1.01\\
1.664002038	1.02\\
1.679880920999	1.02\\
1.695861901999	1.03\\
1.712064102999	1.03\\
1.727956337999	1.04\\
1.744009471999	1.04\\
1.760015141999	1.05\\
1.776939580999	1.05\\
1.792561998998	1.06\\
1.808254419999	1.06\\
1.824014679999	1.07\\
1.839771476999	1.07\\
1.856032097	1.08\\
1.8719713	1.08\\
1.887997198999	1.09\\
1.903909316	1.09\\
1.919988365999	1.1\\
1.935880574999	1.11\\
1.951862682999	1.11\\
1.967858	1.12\\
1.983893118999	1.12\\
1.999828334999	1.13\\
2.017554871999	1.13\\
2.032992194999	1.12\\
2.048494826	1.12\\
2.064031468999	1.12\\
2.080014037999	1.12\\
2.096020344999	1.11\\
2.112013227998	1.11\\
2.127975800999	1.11\\
2.144010438999	1.11\\
2.159933271999	1.1\\
2.175960967999	1.1\\
2.191988419999	1.1\\
2.207879248998	1.09\\
2.223973632	1.09\\
2.239968362999	1.09\\
2.255862175999	1.09\\
2.271863762999	1.08\\
2.287834875999	1.08\\
2.303865514	1.08\\
2.319874121998	1.07\\
2.335881993	1.07\\
2.351719312998	1.07\\
2.367714635999	1.07\\
2.383878951	1.06\\
2.400001381999	1.06\\
2.416038143999	1.06\\
2.431921732998	1.05\\
2.448142000999	1.05\\
2.463998854999	1.05\\
2.479967526999	1.04\\
2.495970882999	1.04\\
2.511929863999	1.04\\
2.528030519999	1.04\\
2.544001356999	1.03\\
2.559980398999	1.03\\
2.575926439999	1.03\\
2.591874529999	1.02\\
2.607830793999	1.02\\
2.623854335999	1.02\\
2.639877517	1.02\\
2.655941996999	1.01\\
2.671865175998	1.01\\
2.687973434999	1.01\\
2.704020108999	1\\
2.720236382999	1\\
2.736025668999	1\\
2.752025230999	1\\
2.768025775999	0.99\\
2.784161230999	0.99\\
2.800042293999	0.99\\
2.816009472998	0.98\\
2.831889148999	0.98\\
2.847979477999	0.98\\
2.863999882	0.98\\
2.880038102	0.97\\
2.896001266999	0.97\\
2.91208301	0.97\\
2.928022260999	0.96\\
2.943986032999	0.96\\
2.959884109999	0.96\\
2.975945165999	0.96\\
2.991873988999	0.95\\
3.009893503999	0.95\\
3.024857837999	0.95\\
3.039763448999	0.95\\
3.055765007999	0.96\\
3.071756852999	0.96\\
3.089894011999	0.96\\
3.104931804999	0.96\\
3.119996914999	0.96\\
3.135735412999	0.96\\
3.151710100999	0.96\\
3.167745311999	0.96\\
3.186015708999	0.97\\
3.201147929999	0.97\\
3.216298368	0.97\\
3.231878208999	0.97\\
3.247862316999	0.97\\
3.263890758999	0.97\\
3.279882007999	0.97\\
3.295904356999	0.97\\
3.311854212	0.97\\
3.327895202999	0.98\\
3.343888425999	0.98\\
3.359865012999	0.98\\
3.375877001999	0.98\\
3.391863359999	0.98\\
3.407798933999	0.98\\
3.423921429	0.98\\
3.440024755999	0.98\\
3.456024292999	0.99\\
3.472209856999	0.99\\
3.488034462999	0.99\\
3.503919125	0.99\\
3.519752715999	0.99\\
3.535809607999	0.99\\
3.551789886999	0.99\\
3.567849690999	0.99\\
3.583870681999	0.99\\
3.599900752999	1\\
3.615889776998	1\\
3.631867349999	1\\
3.647869447999	1\\
3.663844705	1\\
3.679905996999	1\\
3.695892058999	1\\
3.711935722	1\\
3.728118107999	1.01\\
3.744059903999	1.01\\
3.760029102998	1.01\\
3.776011861999	1.01\\
3.791986132999	1.01\\
3.807988120999	1.01\\
3.823980706998	1.01\\
3.839947845999	1.01\\
3.855974399999	1.01\\
3.872044762999	1.02\\
3.888011179999	1.02\\
3.903967012999	1.02\\
3.920027407999	1.02\\
3.936031084999	1.02\\
3.952036770999	1.02\\
3.967975144999	1.02\\
3.983914480999	1.02\\
4.000024553999	1.03\\
4.017641288999	1.03\\
4.033169054999	1.03\\
4.048550791998	1.03\\
4.063990855999	1.02\\
4.080062290999	1.02\\
4.095986056999	1.02\\
4.111886839999	1.02\\
4.127819606999	1.02\\
4.144148214999	1.02\\
4.160019936999	1.02\\
4.175939662999	1.02\\
4.191938700999	1.02\\
4.207932804999	1.02\\
4.223896087999	1.02\\
4.239875096999	1.02\\
4.255840655999	1.02\\
4.271893568999	1.02\\
4.287825327999	1.02\\
4.303856401999	1.02\\
4.319832300998	1.02\\
4.335883843999	1.02\\
4.351877715999	1.02\\
4.367894812999	1.02\\
4.383907183999	1.02\\
4.399894278	1.01\\
4.415893284999	1.01\\
4.431888521999	1.01\\
4.447873927999	1.01\\
4.463875729999	1.01\\
4.479999613999	1.01\\
4.496013052999	1.01\\
4.512019530999	1.01\\
4.527957357998	1.01\\
4.543984542999	1.01\\
4.559819309999	1.01\\
4.575788898999	1.01\\
4.591944604998	1.01\\
4.607971146	1.01\\
4.623987125999	1.01\\
4.639949057999	1.01\\
4.656000559	1.01\\
4.672094489999	1.01\\
4.687879585999	1.01\\
4.703872743999	1.01\\
4.719920294999	1.01\\
4.735913502999	1.01\\
4.752004809	1\\
4.767906703999	1\\
4.7840268	1\\
4.800014279998	1\\
4.816018146999	1\\
4.832113486999	1\\
4.848008852999	1\\
4.863978727999	1\\
4.880075606	1\\
4.895960963999	1\\
4.911919694	1\\
4.928117305999	1\\
4.944018524999	1\\
4.959973190999	1\\
4.976074583999	1\\
4.991821239999	1\\
5.009709425999	1\\
5.024790245999	1\\
5.039959992999	1\\
5.055860730999	1\\
5.071870651998	1\\
5.087885264	1\\
5.103864805998	1\\
5.119891273999	1\\
5.135869408	1\\
5.151974896	1\\
5.167975449998	1\\
5.184000109999	1\\
5.199977587	1\\
5.215995691	1\\
5.232078156999	1\\
5.247989017998	1\\
5.264130586999	1\\
5.280031565999	1\\
5.296000518999	1\\
5.311984481999	1\\
5.327879347	1\\
5.344526842	1\\
5.360167326	1\\
5.376033319999	1\\
5.391934930999	1\\
5.407866172999	1\\
5.423862442999	1\\
5.440078811999	1\\
5.455893900999	1\\
5.471908196999	1\\
5.487860140999	1\\
5.503845420999	1\\
5.519990914999	1\\
5.536007400999	1\\
5.552005429999	1.01\\
5.568058236999	1.01\\
5.583978154998	1.01\\
5.599842710999	1.01\\
5.615993640998	1.01\\
5.632001827999	1.01\\
5.647966010999	1.01\\
5.664125094999	1.01\\
5.680018024998	1.01\\
5.695994619999	1.01\\
5.711972122999	1.01\\
5.728010167999	1.01\\
5.743999901999	1.01\\
5.760137276999	1.01\\
5.775986257999	1.01\\
5.791877895999	1.01\\
5.807874892998	1.01\\
5.823862403	1.01\\
5.839878727998	1.01\\
5.855882776	1.01\\
5.871841069998	1.01\\
5.887893015999	1.01\\
5.903862028999	1.01\\
5.919852976999	1.01\\
5.935972630999	1.01\\
5.952036358999	1.01\\
5.967929956999	1.01\\
5.983915331999	1.01\\
5.999831213999	1.01\\
6.015858248999	1.01\\
6.032022179999	1.01\\
6.047903363	1.01\\
6.063898144998	1.01\\
6.079849321999	1.01\\
6.095824825999	1.01\\
6.111996868999	1.01\\
6.128108946	1.01\\
6.143892538999	1.01\\
6.159875934999	1.01\\
6.175874920999	1.01\\
6.191878186999	1.01\\
6.207870341999	1.01\\
6.224013875999	1.01\\
6.240037853999	1.01\\
6.256029332999	1.01\\
6.272011274998	1.01\\
6.287981575999	1.01\\
6.304012769999	1.01\\
6.320118439	1.01\\
6.336020288999	1.01\\
6.352054033999	1.01\\
6.368088232999	1.01\\
6.384032316999	1.01\\
6.400122574999	1.01\\
6.415921917999	1.01\\
6.431835581999	1.01\\
6.447886223999	1.01\\
6.463870130999	1.01\\
6.479849332999	1\\
6.495893700999	1\\
6.511857382	1\\
6.527753397999	1\\
6.545861970999	1\\
6.561200842999	1\\
6.576875936	1\\
6.592423679	1\\
6.608025845999	1\\
6.623878969999	1\\
6.639914929999	1\\
6.655987047	1\\
6.671989922	1\\
6.688013889999	1\\
6.704004952999	1\\
6.720068890999	1\\
6.736150619999	1\\
6.752037370999	1\\
6.768069101999	1\\
6.784006903999	1\\
6.799993771999	1\\
6.815971454999	1\\
6.832009890999	1\\
6.848113629999	1\\
6.863999198	1\\
6.879939759999	1\\
6.895942045999	1\\
6.912124261999	1\\
6.928029630999	1\\
6.944165456999	1\\
6.959887706998	1\\
6.976007034999	1\\
6.992002968999	1\\
7.009838833998	1\\
7.025679385999	1\\
7.041055420999	1\\
7.056333552	1\\
7.071906168999	1\\
7.087904962	1\\
7.103939302999	1\\
7.119829943999	1\\
7.135919274999	1\\
7.151877708999	1\\
7.167859114998	1\\
7.183868178999	1\\
7.199928163998	1\\
7.21587009	1\\
7.231894618999	1\\
7.247876318999	1\\
7.263869458999	1\\
7.279869301999	1\\
7.295881863999	1\\
7.311856719999	1\\
7.327867943999	1\\
7.345051779998	1\\
7.360105609999	1\\
7.375888284999	1\\
7.392002146999	1\\
7.407868977999	1\\
7.424025245999	1\\
7.439967533999	1\\
7.456005848999	1\\
7.472011455	1\\
7.487946560999	1\\
7.503897037999	1\\
7.51987912	1\\
7.535871385999	1.01\\
7.551890172	1.01\\
7.567868736999	1.01\\
7.583851144998	1.01\\
7.600030705999	1.01\\
7.616024731	1.01\\
7.632148464999	1.01\\
7.647892998999	1.01\\
7.663871372999	1.01\\
7.679867328	1.01\\
7.695905652999	1.01\\
7.712040416999	1.01\\
7.727974279998	1.01\\
7.744201542999	1.01\\
7.760011510999	1.01\\
7.776020624999	1.01\\
7.791975200999	1.01\\
7.807878444999	1.01\\
7.823853807999	1.01\\
7.839718198	1.01\\
7.855874836999	1.01\\
7.872153871998	1.01\\
7.888426733999	1.01\\
7.903962525999	1.01\\
7.919943816999	1.01\\
7.935950971999	1.01\\
7.951970047999	1.01\\
7.968014396999	1.01\\
7.984256819998	1.01\\
7.999868898999	1.01\\
8.017523672999	1.01\\
8.032769827999	1.01\\
8.048181281999	1.01\\
8.063852641999	1.01\\
8.079988225999	1.01\\
8.096011376999	1.01\\
8.111881876999	1.01\\
8.127861363999	1.01\\
8.143864819998	1.01\\
8.159873452999	1.01\\
8.175861033	1.01\\
8.191867125999	1.01\\
8.20789384	1.01\\
8.224126896	1.01\\
8.240062849999	1.01\\
8.256004815999	1.01\\
8.272012847999	1.01\\
8.287997465999	1.01\\
8.303886958999	1.01\\
8.319999602999	1.01\\
8.336002507998	1.01\\
8.351875853999	1.01\\
8.367826648999	1.01\\
8.383846338998	1.01\\
8.400120475999	1.01\\
8.415868680999	1.01\\
8.431865111998	1.01\\
8.447849346999	1.01\\
8.464007896999	1.01\\
8.479980798998	1.01\\
8.495999103999	1\\
8.511890731999	1\\
8.527875389999	1\\
8.543876912999	1\\
8.559879225999	1\\
8.575974816	1\\
8.592018262999	1\\
8.608191769999	1\\
8.623983005	1\\
8.639886895999	1\\
8.655816042999	1\\
8.671847306999	1\\
8.687840845999	1\\
8.703875452999	1\\
8.720051936999	1\\
8.735987774999	1\\
8.751941096999	1\\
8.767866587	1\\
8.783913270999	1\\
8.799988318999	1\\
8.815868791998	1\\
8.831758026	1\\
8.847757928999	1\\
8.863753619999	1\\
8.879776915998	1\\
8.895898163999	1\\
8.911913162999	1\\
8.927955927999	1\\
8.943859601999	1\\
8.959826149999	1\\
8.975892308999	1\\
8.991934672999	1\\
9.009796186999	1\\
9.024851808999	1\\
9.040352314999	1\\
9.055973941	1\\
9.071889517999	1\\
9.087868648999	1\\
9.103862141999	1\\
9.119865803999	1\\
9.135848403	1\\
9.151879411998	1\\
9.167857349999	1\\
9.183874266999	1\\
9.199871829999	1\\
9.215789665998	1\\
9.231778349999	1\\
9.247869997999	1\\
9.263901124999	1\\
9.279885957999	1\\
9.295871227999	1\\
9.312064785999	1\\
9.328011553999	1\\
9.345068497999	1\\
9.360465411999	1\\
9.376250740999	1\\
9.392020221999	1\\
9.40791275	1\\
9.424006712999	1\\
9.439902239998	1\\
9.455808727998	1\\
9.471910439998	1\\
9.487867375999	1\\
9.503928098999	1\\
9.519914112999	1.01\\
9.535739799999	1.01\\
9.553918728999	1.01\\
9.569053169999	1.01\\
9.584160199999	1.01\\
9.599880293998	1.01\\
9.615890866	1.01\\
9.631871109998	1.01\\
9.647800667999	1.01\\
9.66382362	1.01\\
9.679866453998	1.01\\
9.695945489999	1.01\\
9.711977482999	1.01\\
9.727964887999	1.01\\
9.744040682998	1.01\\
9.759991865999	1.01\\
9.776074914999	1.01\\
9.791985743998	1.01\\
9.807888295999	1.01\\
9.823824059	1.01\\
9.840301824998	1.01\\
9.856038801999	1.01\\
9.872071654999	1.01\\
9.887947401999	1.01\\
9.903939657999	1.01\\
9.920013944999	1.01\\
9.935993977999	1.01\\
9.952146225999	1.01\\
9.968013177999	1.01\\
9.983968986	1.01\\
10.000006344999	1.01\\
10.017793639	1.01\\
10.032987644999	1.01\\
10.04815437	1.01\\
10.063932545999	1.01\\
10.079878902999	1.01\\
10.095849278999	1.01\\
10.111813595999	1.01\\
10.127744679999	1.01\\
10.143792674999	1.01\\
10.159945451998	1.01\\
10.176089620999	1.01\\
10.192013415999	1.01\\
10.207924677999	1.01\\
10.223880931999	1.01\\
10.239864337999	1.01\\
10.255879761999	1.01\\
10.271838797	1.01\\
10.287875809999	1.01\\
10.303893522999	1.01\\
10.319999225999	1.01\\
10.336030885998	1.01\\
10.352521937999	1.01\\
10.368013724	1.01\\
10.383902615999	1.01\\
10.399859325999	1.01\\
10.416017181999	1.01\\
10.432008184998	1.01\\
10.447979909999	1.01\\
10.463928876999	1.01\\
10.479875559999	1.01\\
10.496002660999	1.01\\
10.512021667999	1\\
10.527997806999	1\\
10.543876148999	1\\
10.559876052998	1\\
10.575865872999	1\\
10.591872476999	1\\
10.607873561999	1\\
10.623836616999	1\\
10.63991808	1\\
10.655894960999	1\\
10.672026045999	1\\
10.687889148999	1\\
10.703961847999	1\\
10.719855910999	1\\
10.735855448999	1\\
10.752441494999	1\\
10.767938347999	1\\
10.783903351999	1\\
10.800133792999	1\\
10.815979069998	1\\
10.831966731	1\\
10.847945881999	1\\
10.863932434999	1\\
10.879851890999	1\\
10.895872731999	1\\
10.91187267	1\\
10.927865987999	1\\
10.943878604999	1\\
10.959866886999	1\\
10.975885196999	1\\
10.991902714999	1\\
11.007766406	1\\
11.023891645999	1\\
11.039868075999	1\\
11.055886861998	1\\
11.071872658999	1\\
11.087828451999	1\\
11.103884778	1\\
11.119873521	1\\
11.135897863999	1\\
11.151866982998	1\\
11.167878511999	1\\
11.183767511999	1\\
11.199869653999	1\\
11.215875677999	1\\
11.231947301999	1\\
11.247863646	1\\
11.263872102999	1\\
11.279864719999	1\\
11.295871533999	1\\
11.311898597999	1\\
11.327867104998	1\\
11.343923445999	1\\
11.359877368999	1\\
11.375776203999	1\\
11.391795619999	1\\
11.407871136999	1\\
11.423810663	1\\
11.440027732999	1\\
11.456017923999	1\\
11.471920291998	1\\
11.487979137999	1\\
11.504008823999	1.01\\
11.519970048998	1.01\\
11.536172616999	1.01\\
11.552012541999	1.01\\
11.567996398999	1.01\\
11.584009026	1.01\\
11.600030377999	1.01\\
11.615983375	1.01\\
11.631988128998	1.01\\
11.647981005999	1.01\\
11.664029103999	1.01\\
11.680063118999	1.01\\
11.69597162	1.01\\
11.711927839999	1.01\\
11.727916025999	1.01\\
11.744182533999	1.01\\
11.759909778999	1.01\\
11.775948429999	1.01\\
11.792005050998	1.01\\
11.808126524998	1.01\\
11.823887375998	1.01\\
11.839876153999	1.01\\
11.855874002999	1.01\\
11.871833479999	1.01\\
11.887898950999	1.01\\
11.903855521	1.01\\
11.919908769999	1.01\\
11.935868863	1.01\\
11.952007781999	1.01\\
11.967878556999	1.01\\
11.983897372999	1.01\\
11.999874995999	1.01\\
12.017372452999	1.01\\
12.032499602998	1.01\\
12.04809071	1.01\\
12.063751670999	1.01\\
12.080007963998	1.01\\
12.096025827999	1.01\\
12.111897920999	1.01\\
12.127869384999	1.01\\
12.144070478999	1.01\\
12.160027936	1.01\\
12.175977890999	1.01\\
12.191950953999	1.01\\
12.207983245	1.01\\
12.223885341999	1.01\\
12.239881897999	1.01\\
12.255897359999	1.01\\
12.272004341999	1.01\\
12.287908431999	1.01\\
12.303858757999	1.01\\
12.319885932999	1.01\\
12.335885267998	1.01\\
12.351980729999	1.01\\
12.368015307	1.01\\
12.384028113999	1.01\\
12.400004028	1.01\\
12.416009077999	1.01\\
12.431974347999	1.01\\
12.448066765999	1.01\\
12.463986915999	1.01\\
12.480027989999	1.01\\
12.495847696	1.01\\
12.511992058999	1.01\\
12.527791808999	1\\
12.543917266999	1\\
12.559838015999	1\\
12.575841952999	1\\
12.591862286998	1\\
12.607863583999	1\\
12.623866227998	1\\
12.640008809999	1\\
12.656006592999	1\\
12.671990905999	1\\
12.687881846999	1\\
12.704004452999	1\\
12.720035982998	1\\
12.735984915999	1\\
12.751976700999	1\\
12.767975976999	1\\
12.784005231999	1\\
12.799983215999	1\\
12.816006756999	1\\
12.832049336999	1\\
12.848142917999	1\\
12.863981454999	1\\
12.879865365999	1\\
12.895867915	1\\
12.911802208999	1\\
12.930192609999	1\\
12.945252537999	1\\
12.960315422999	1\\
12.975783778	1\\
12.991901500999	1\\
13.010107160999	1\\
13.025458703999	1\\
13.041084667999	1\\
13.056613111999	1\\
13.071932117999	1\\
13.087884537999	1\\
13.103879258999	1\\
13.119888232999	1\\
13.135877291998	1\\
13.151850649999	1\\
13.167887380999	1\\
13.183868085999	1\\
13.200012500999	1\\
13.216007523999	1\\
13.231977969999	1\\
13.248005826998	1\\
13.263944239998	1\\
13.280005627999	1\\
13.295976499999	1\\
13.311853213999	1\\
13.327858945999	1\\
13.343906737999	1\\
13.360058821999	1\\
13.376154134999	1\\
13.391876789999	1\\
13.408005059	1\\
13.423926222999	1\\
13.439892374999	1\\
13.455861974999	1\\
13.471880569999	1\\
13.487839159999	1.01\\
13.503983926999	1.01\\
13.520005368	1.01\\
13.535982810999	1.01\\
13.551996960999	1.01\\
13.568013186	1.01\\
13.583883936999	1.01\\
13.599865908999	1.01\\
13.615878305999	1.01\\
13.631850365999	1.01\\
13.647869978999	1.01\\
13.663891720998	1.01\\
13.679885895999	1.01\\
13.696014436	1.01\\
13.712121276999	1.01\\
13.728048651	1.01\\
13.744076106	1.01\\
13.760014028999	1.01\\
13.776012292999	1.01\\
13.792015879999	1.01\\
13.808030321	1.01\\
13.824019815999	1.01\\
13.840011920999	1.01\\
13.856030486999	1.01\\
13.872060682998	1.01\\
13.888389146	1.01\\
13.904055262999	1.01\\
13.919969361999	1.01\\
13.935873196999	1.01\\
13.951840561	1.01\\
13.968049021999	1.01\\
13.984080967999	1.01\\
13.999951738999	1.01\\
14.017460587	1.01\\
14.033022908	1.01\\
14.048397994999	1.01\\
14.063885302	1.01\\
14.079887108999	1.01\\
14.095864936999	1.01\\
14.112116270999	1.01\\
14.128085653999	1.01\\
14.144026186	1.01\\
14.159925828999	1.01\\
14.175982705999	1.01\\
14.191986934	1.01\\
14.208011041998	1.01\\
14.223884483999	1.01\\
14.240035809999	1.01\\
14.256014419999	1.01\\
14.272017800999	1.01\\
14.288005821999	1.01\\
14.304008962999	1.01\\
14.320099365999	1.01\\
14.336007696999	1.01\\
14.353467021999	1.01\\
14.368968871999	1.01\\
14.384398286998	1.01\\
14.399818203	1.01\\
14.416286820999	1.01\\
14.432029465998	1.01\\
14.448215350999	1.01\\
14.464114635999	1.01\\
14.480037669999	1.01\\
14.495988748	1.01\\
14.511863070999	1.01\\
14.527993925999	1.01\\
14.543988536999	1\\
14.559941799999	1\\
14.575931048998	1\\
14.591998213998	1\\
14.608009648999	1\\
14.623993189999	1\\
14.640117604999	1\\
14.656036930999	1\\
14.672004565999	1\\
14.687978153999	1\\
14.703974112999	1\\
14.719961557998	1\\
14.735983841999	1\\
14.751878135999	1\\
14.767878449998	1\\
14.783866882999	1\\
14.799832161999	1\\
14.815865325999	1\\
14.831852163999	1\\
14.847794109999	1\\
14.864187585999	1\\
14.880111637999	1\\
14.895991673998	1\\
14.911927516999	1\\
14.927833058999	1\\
14.943997953999	1\\
14.959962731999	1\\
14.976036501999	1\\
14.991970846999	1\\
15.007707370999	1\\
15.023977323999	1\\
15.040189161998	1\\
15.055907716999	1\\
15.071883410999	1\\
15.087862571999	1\\
15.103899247999	1\\
15.119881902999	1\\
15.135851757999	1\\
15.151841437999	1\\
15.167906904999	1\\
15.183851918	1\\
15.199819948999	1\\
15.215867579999	1\\
15.231768948999	1\\
15.247921514999	1\\
15.263756436999	1\\
15.279813299999	1\\
15.295876493	1\\
15.311903531999	1\\
15.327868144998	1\\
15.343895894999	1\\
15.359854927999	1\\
15.375875887999	1\\
15.391998794999	1\\
15.408026781999	1\\
15.423954436999	1\\
15.440094306999	1\\
15.456013206999	1\\
15.471914109999	1.01\\
15.488004758999	1.01\\
15.503969060999	1.01\\
15.519806646	1.01\\
15.535782852999	1.01\\
15.551760686999	1.01\\
15.567751392	1.01\\
15.585856403999	1.01\\
15.600853201	1.01\\
15.616121299999	1.01\\
15.631908931999	1.01\\
15.647856451998	1.01\\
15.663853122999	1.01\\
15.679877290999	1.01\\
15.695821266999	1.01\\
15.711854941999	1.01\\
15.727887398999	1.01\\
15.743695851999	1.01\\
15.761789156999	1.01\\
15.776661535999	1.01\\
15.791696012999	1.01\\
15.809872545	1.01\\
15.824902128998	1.01\\
15.840131154999	1.01\\
15.856026085	1.01\\
15.871926823999	1.01\\
15.888012988	1.01\\
15.904001735999	1.01\\
15.920024490999	1.01\\
15.935962791998	1.01\\
15.952139246999	1.01\\
15.968011837	1.01\\
15.983976073999	1.01\\
16.000029628999	1.01\\
16.017752226999	1.01\\
16.033357785999	1.01\\
16.048791044999	1.01\\
16.064235493999	1.01\\
16.080060635999	1.01\\
16.096003434999	1.01\\
16.112029705999	1.01\\
16.127949265999	1.01\\
16.144003214999	1.01\\
16.160057360999	1.01\\
16.175969334999	1.01\\
16.192146784999	1.01\\
16.208059039999	1.01\\
16.223969319999	1.01\\
16.240004938999	1.01\\
16.255941347999	1.01\\
16.271872495	1.01\\
16.287870844999	1.01\\
16.303986675999	1.01\\
16.319926768999	1.01\\
16.336011453999	1.01\\
16.351979946	1.01\\
16.367997884999	1.01\\
16.383885387999	1.01\\
16.399825530999	1.01\\
16.415927882	1.01\\
16.431961726999	1.01\\
16.447965743999	1.01\\
16.463865148999	1.01\\
16.479867153	1.01\\
16.496002061999	1.01\\
16.511985743	1.01\\
16.528060009999	1.01\\
16.544030871999	1.01\\
16.559953345999	1\\
16.575829071	1\\
16.591954083	1\\
16.608010027999	1\\
16.624013619999	1\\
16.639986546999	1\\
16.655870531999	1\\
16.671987245999	1\\
16.688167351999	1\\
16.704000632999	1\\
16.720106600999	1\\
16.736036931999	1\\
16.752033015999	1\\
16.767971614999	1\\
16.784007133999	1\\
16.800016623999	1\\
16.815974110999	1\\
16.832023626999	1\\
16.847980311999	1\\
16.863880092999	1\\
16.880003285999	1\\
16.896001484999	1\\
16.912081559999	1\\
16.927982835	1\\
16.944016032999	1\\
16.960012253	1\\
16.975908305999	1\\
16.991890668999	1\\
17.007739736	1\\
17.023903024	1\\
17.039844598999	1\\
17.055803491999	1\\
17.071920238999	1\\
17.087858581998	1\\
17.104216842	1\\
17.119893607999	1\\
17.13588833	1\\
17.152005738999	1\\
17.167986523999	1\\
17.183913145999	1\\
17.199868753999	1\\
17.215860693999	1\\
17.231824519999	1\\
17.247921765998	1\\
17.263889428999	1\\
17.280022529999	1\\
17.295847583999	1\\
17.312018916998	1\\
17.327918423999	1\\
17.344831305999	1\\
17.360238329999	1\\
17.37590121	1\\
17.391985658999	1\\
17.407981342999	1\\
17.424009063999	1\\
17.440054823999	1\\
17.455935465	1.01\\
17.471862576998	1.01\\
17.487765373	1.01\\
17.503976151999	1.01\\
17.519997508999	1.01\\
17.535883474	1.01\\
17.551980140999	1.01\\
17.567876188999	1.01\\
17.583899133999	1.01\\
17.599843971999	1.01\\
17.615841790998	1.01\\
17.631866925999	1.01\\
17.647842604999	1.01\\
17.664007651999	1.01\\
17.680001210999	1.01\\
17.696078699999	1.01\\
17.711865885999	1.01\\
17.727884706998	1.01\\
17.743997575999	1.01\\
17.759882112999	1.01\\
17.775864500999	1.01\\
17.791975430999	1.01\\
17.808001122999	1.01\\
17.824029790999	1.01\\
17.840001114999	1.01\\
17.855999874999	1.01\\
17.872052422	1.01\\
17.887960081999	1.01\\
17.903999288998	1.01\\
17.920040801999	1.01\\
17.935990795999	1.01\\
17.951999554999	1.01\\
17.967958308999	1.01\\
17.984002141999	1.01\\
18.000020622999	1.01\\
18.017662005998	1.01\\
18.032744283999	1.01\\
18.048249165999	1.01\\
18.063834917999	1.01\\
18.079746109998	1.01\\
18.095825184999	1.01\\
18.112113958	1.01\\
18.128149992999	1.01\\
18.144374494999	1.01\\
18.160125644999	1.01\\
18.176028146999	1.01\\
18.191956792999	1.01\\
18.208006098999	1.01\\
18.224028692999	1.01\\
18.240014228999	1.01\\
18.255980214999	1.01\\
18.272131636999	1.01\\
18.287978662999	1.01\\
18.303999027999	1.01\\
18.320008773999	1.01\\
18.336030219999	1.01\\
18.351981122999	1.01\\
18.367959676999	1.01\\
18.384000732999	1.01\\
18.400113261999	1.01\\
18.416022679999	1.01\\
18.431975797	1.01\\
18.448011010999	1.01\\
18.463876097999	1.01\\
18.479705939999	1.01\\
18.495836470999	1.01\\
18.511890490999	1.01\\
18.527862356999	1.01\\
18.543882971999	1.01\\
18.559771896999	1.01\\
18.575891752999	1\\
18.591888373	1\\
18.607993078998	1\\
18.623906932	1\\
18.639976272999	1\\
18.656078396999	1\\
18.672196528999	1\\
18.687884454999	1\\
18.703863459999	1\\
18.719864561999	1\\
18.735873063999	1\\
18.751868536999	1\\
18.767842561999	1\\
18.783899594999	1\\
18.799867870999	1\\
18.815924875999	1\\
18.831922065999	1\\
18.848078918999	1\\
18.864000319999	1\\
18.880133133999	1\\
18.895902885999	1\\
18.912004448999	1\\
18.928033217999	1\\
18.943942334999	1\\
18.959868194998	1\\
18.976003420999	1\\
18.992013420999	1\\
19.009921919999	1\\
19.025271134999	1\\
19.040433017999	1\\
19.055850842	1\\
19.072000999999	1\\
19.088014398999	1\\
19.103876088999	1\\
19.119886169999	1\\
19.135880941999	1\\
19.151895778999	1\\
19.168011062999	1\\
19.183966205	1\\
19.199803913999	1\\
19.215874179	1\\
19.231878716999	1\\
19.247822622999	1\\
19.263980213998	1\\
19.280077814	1\\
19.295898134999	1\\
19.312138818999	1\\
19.328013627999	1\\
19.343854531999	1\\
19.360006528999	1\\
19.375880285999	1\\
19.391871920999	1\\
19.407865954999	1\\
19.423873811999	1\\
19.439877342999	1.01\\
19.455884732999	1.01\\
19.471885708998	1.01\\
19.487812994999	1.01\\
19.503972271999	1.01\\
19.520112868999	1.01\\
19.536042031999	1.01\\
19.552003352998	1.01\\
19.568029078999	1.01\\
19.583914176999	1.01\\
19.599964969999	1.01\\
19.616001666998	1.01\\
19.632062247999	1.01\\
19.648001638999	1.01\\
19.664032381999	1.01\\
19.679985033	1.01\\
19.695884878999	1.01\\
19.712042635999	1.01\\
19.728023419999	1.01\\
19.744009737999	1.01\\
19.759994646999	1.01\\
19.775926346999	1.01\\
19.792069771999	1.01\\
19.807959903	1.01\\
19.824003880999	1.01\\
19.839892165999	1.01\\
19.855929398999	1.01\\
19.871974419999	1.01\\
19.888077411999	1.01\\
19.904012407999	1.01\\
19.920034144999	1.01\\
19.935956226999	1.01\\
19.951984624999	1.01\\
19.967869496999	1.01\\
19.983860264999	1.01\\
19.999862091999	1.01\\
20.015858226999	1.01\\
20.031800538999	1.01\\
20.047993585999	1.01\\
20.063880652999	1.01\\
20.079920924999	1.01\\
20.095994512999	1.01\\
20.11210212	1.01\\
20.127998232999	1.01\\
20.143969486	1.01\\
20.159989504999	1.01\\
20.176094625998	1.01\\
20.192018055999	1.01\\
20.207836677999	1.01\\
20.223975401999	1.01\\
20.240026665999	1.01\\
20.256039434	1.01\\
20.272011142999	1.01\\
20.287891469999	1.01\\
20.303795922	1.01\\
20.319865717999	1.01\\
20.335906497999	1.01\\
20.352318882	1.01\\
20.36806929	1.01\\
20.384031251999	1.01\\
20.400036787999	1.01\\
20.416096009999	1.01\\
20.431989932998	1.01\\
20.447930241999	1.01\\
20.463866896999	1.01\\
20.479856	1.01\\
20.496009155999	1.01\\
20.511916719999	1.01\\
20.527999700999	1.01\\
20.543875087	1.01\\
20.560019286999	1.01\\
20.575881232998	1\\
20.591899787999	1\\
20.607863088999	1\\
20.623873966999	1\\
20.639979939999	1\\
20.656158868998	1\\
20.672007264	1\\
20.687946273999	1\\
20.704126886999	1\\
20.720002763999	1\\
20.736128517	1\\
20.751849562998	1\\
20.768091313999	1\\
20.784009311999	1\\
20.799975478999	1\\
20.815843187998	1\\
20.831843866999	1\\
20.847974872999	1\\
20.863869767999	1\\
20.879797553999	1\\
20.898116546999	1\\
20.913293253999	1\\
20.928390100999	1\\
20.943803007999	1\\
20.959780951999	1\\
20.975878290999	1\\
20.991806125	1\\
21.009667727999	1\\
21.024930359998	1\\
21.040263151998	1\\
21.056119223999	1\\
21.072002001999	1\\
21.088214365999	1\\
21.104024745	1\\
21.119891731999	1\\
21.135854936999	1\\
21.152000679999	1\\
21.168013512999	1\\
21.183885465999	1\\
21.199870720999	1\\
21.215861477998	1\\
21.231885591999	1\\
21.247867804999	1\\
21.263792942999	1\\
21.280036017999	1\\
21.296010888999	1\\
21.312001901999	1\\
21.328107900999	1\\
21.343999681999	1\\
21.359840334999	1\\
21.375964201999	1\\
21.391873185999	1\\
21.407966113	1\\
21.423894243999	1\\
21.439952005999	1.01\\
21.455994970998	1.01\\
21.472086617999	1.01\\
21.488010172999	1.01\\
21.504132172999	1.01\\
21.519764538999	1.01\\
21.536089888999	1.01\\
21.552433183999	1.01\\
21.567821071999	1.01\\
21.583871162999	1.01\\
21.599868826998	1.01\\
21.616016093999	1.01\\
21.632008043999	1.01\\
21.647998049999	1.01\\
21.663982542999	1.01\\
21.680008865999	1.01\\
21.696003792999	1.01\\
21.712034219999	1.01\\
21.728004271	1.01\\
21.744137385998	1.01\\
21.759990255999	1.01\\
21.775888608999	1.01\\
21.791864401998	1.01\\
21.807866805999	1.01\\
21.823855269	1.01\\
21.839862928999	1.01\\
21.85589042	1.01\\
21.871849499999	1.01\\
21.887964644999	1.01\\
21.904004357999	1.01\\
21.920090105999	1.01\\
21.935887184999	1.01\\
21.951840742999	1.01\\
21.967944220999	1.01\\
21.983883736999	1.01\\
21.999876483999	1.01\\
22.017572114999	1.01\\
22.032669221999	1.01\\
22.047870324999	1.01\\
22.063832557998	1.01\\
22.080003804999	1.01\\
22.095913745	1.01\\
22.112135534999	1.01\\
22.128003688999	1.01\\
22.143978765998	1.01\\
22.159999275999	1.01\\
22.175986059999	1.01\\
22.191933323	1.01\\
22.208024793998	1.01\\
22.224015981999	1.01\\
22.240073284999	1.01\\
22.256092848999	1.01\\
22.271978612999	1.01\\
22.287903195999	1.01\\
22.303997717999	1.01\\
22.319937043999	1.01\\
22.335950041998	1.01\\
22.352325706998	1.01\\
22.368009207999	1.01\\
22.384086901999	1.01\\
22.400042324999	1.01\\
22.415942731999	1.01\\
22.432438288999	1.01\\
22.448104750999	1.01\\
22.464030140999	1.01\\
22.480236906999	1.01\\
22.496270285999	1.01\\
};
\end{axis}
\end{tikzpicture}%
}
      \caption{The orientation of the robot over time for
        $K_{\omega}^T = 0.75 K_{\omega, max}^T$}
      \label{fig:13_0.75_max}
    \end{figure}
  \end{minipage}
  \hfill
  \begin{minipage}{0.45\linewidth}
    \begin{figure}[H]
      \scalebox{0.6}{% This file was created by matlab2tikz.
%
%The latest updates can be retrieved from
%  http://www.mathworks.com/matlabcentral/fileexchange/22022-matlab2tikz-matlab2tikz
%where you can also make suggestions and rate matlab2tikz.
%
\definecolor{mycolor1}{rgb}{0.00000,0.44700,0.74100}%
%
\begin{tikzpicture}

\begin{axis}[%
width=4.133in,
height=3.26in,
at={(0.693in,0.44in)},
scale only axis,
xmin=5.679941101002,
xmax=14,
xmajorgrids,
xlabel={Time (seconds)},
ymin=0.95,
ymax=1.1,
ymajorgrids,
ylabel={Distance (meters)},
axis background/.style={fill=white}
]
\addplot [color=mycolor1,solid,forget plot]
  table[row sep=crcr]{%
5.679941101002	1.01\\
5.695975913002	1.01\\
5.711892638001	1.01\\
5.727813021	1.01\\
5.744008267003	1.01\\
5.759936832001	1.01\\
5.775969526001	1.01\\
5.792018526001	1.01\\
5.807984577	1.01\\
5.823953236	1.01\\
5.840003055001	1.01\\
5.85595939	1.01\\
5.871989216	1.01\\
5.888314601998	1.01\\
5.903970800999	1.01\\
5.919957805001	1.01\\
5.935979351998	1.01\\
5.951958909001	1.01\\
5.967956953	1.01\\
5.983925791001	1.01\\
6.001431427002	1.01\\
6.016364973	1.01\\
6.031818804001	1.01\\
6.047809819001	1.01\\
6.063839921002	1.01\\
6.079800439999	1.01\\
6.098095126999	1.01\\
6.11330966	1.01\\
6.128562985001	1.01\\
6.144473747002	1.01\\
6.159900165001	1.01\\
6.175952303002	1.01\\
6.191973191998	1.01\\
6.208055192002	1.01\\
6.223924043	1.01\\
6.239919620999	1.01\\
6.255914964001	1.01\\
6.271884858002	1.01\\
6.287902713002	1.01\\
6.303944286999	1.01\\
6.319962666001	1.01\\
6.335980598999	1.01\\
6.351939945004	1.01\\
6.368009823002	1.01\\
6.383994356003	1.01\\
6.399954510003	1.01\\
6.415830843003	1.01\\
6.432089840001	1.01\\
6.447978003003	1.01\\
6.464252550999	1.01\\
6.479981472001	1.01\\
6.495931723	1.01\\
6.511927191002	1.01\\
6.527941240998	1.01\\
6.543929042	1.01\\
6.559898266999	1.01\\
6.575963547001	1.01\\
6.591958404	1.01\\
6.607958933003	1.01\\
6.624007552002	1.01\\
6.639981351002	1.01\\
6.655917954998	1.01\\
6.671971656002	1.01\\
6.687925539002	1.01\\
6.703998397	1.01\\
6.719913468998	1.01\\
6.735871611	1.01\\
6.751901487999	1.01\\
6.767998833001	1.01\\
6.783947331002	1.01\\
6.799979701001	1.01\\
6.815945582001	1.01\\
6.832065587002	1.01\\
6.847901689003	1.01\\
6.866248286	1.01\\
6.881420066998	1.01\\
6.896605079003	1.01\\
6.911913917	1.01\\
6.927954777001	1.01\\
6.943952170002	1.01\\
6.960036939	1.01\\
6.975943867001	1.01\\
6.996726499001	1.01\\
7.008482016003	1.01\\
7.023796702	1.01\\
7.039818418999	1.01\\
7.055818508	1.01\\
7.071834447003	1.01\\
7.089984586003	1.01\\
7.105108985001	1.01\\
7.120363263001	1.01\\
7.136284664002	1.01\\
7.151990419999	1.01\\
7.167998792	1.01\\
7.183951504002	1.01\\
7.199990701001	1.01\\
7.215962727997	1.01\\
7.231894051999	1.01\\
7.247914397999	1.01\\
7.266262719002	1.01\\
7.281521497002	1.01\\
7.296837425003	1.01\\
7.312100022	1.01\\
7.327948853001	1.01\\
7.343927998997	1.01\\
7.359854473999	1.01\\
7.375993223999	1.01\\
7.391995282002	1.01\\
7.408058067997	1.01\\
7.423880432999	1.01\\
7.439977065998	1.01\\
7.455949296002	1.01\\
7.472054535	1.01\\
7.487964908001	1.01\\
7.504042077	1.01\\
7.519958495999	1.01\\
7.536032051003	1.01\\
7.551938237	1.01\\
7.567983798001	1.01\\
7.583910025998	1.01\\
7.600108397	1.01\\
7.615966711003	1.01\\
7.632059714001	1.01\\
7.648022207001	1.01\\
7.666468143998	1.01\\
7.681771619	1.01\\
7.697009610001	1.01\\
7.711933047001	1.01\\
7.728034923001	1.01\\
7.745690436001	1.01\\
7.760974258999	1.01\\
7.776107912999	1.01\\
7.791954018002	1.01\\
7.807970591999	1.01\\
7.823890112999	1.01\\
7.842261227002	1.01\\
7.857360805001	1.01\\
7.872594141003	1.01\\
7.887933376999	1.01\\
7.903974968998	1.01\\
7.919998420998	1.01\\
7.935907680001	1.01\\
7.951931434002	1.01\\
7.968023633	1.01\\
7.983905634999	1.01\\
8.001466812001	1.01\\
8.016356680001	1.01\\
8.031816524998	1.01\\
8.047803689	1.01\\
8.063857279004	1.01\\
8.079829097001	1.01\\
8.096042062001	1.01\\
8.112033636002	1.01\\
8.127998244999	1.01\\
8.143967733998	1.01\\
8.159937272999	1.01\\
8.175927048001	1.01\\
8.191975652001	1.01\\
8.208031501999	1.01\\
8.223887312001	1.01\\
8.239982838002	1.01\\
8.255887815998	1.01\\
8.271980946	1.01\\
8.287964522	1.01\\
8.303997380001	1.01\\
8.320038401001	1.01\\
8.335973884003	1.01\\
8.352761644001	1.01\\
8.368004093998	1.01\\
8.383977687001	1.01\\
8.400024903	1.01\\
8.415955564	1.01\\
8.432073875	1.01\\
8.447898370999	1.01\\
8.464046615998	1.01\\
8.479919471001	1.01\\
8.495905713002	1.01\\
8.511923682	1.01\\
8.527931313004	1.01\\
8.543838675999	1.01\\
8.559874023999	1.01\\
8.57589032	1.01\\
8.591922521	1.01\\
8.607976571999	1.01\\
8.623929463002	1.01\\
8.640026582001	1.01\\
8.655957078003	1.01\\
8.671935312001	1.01\\
8.687962466999	1.01\\
8.703984591	1.01\\
8.719816671002	1.01\\
8.735854422001	1.01\\
8.754042695	1.01\\
8.769207961998	1.01\\
8.784624834	1.01\\
8.800130196	1.01\\
8.81596257	1.01\\
8.832100466	1.01\\
8.847905903	1.01\\
8.866206270001	1.01\\
8.881446887001	1.01\\
8.896659009999	1.01\\
8.912214364998	1.01\\
8.928017205998	1.01\\
8.943857891003	1.01\\
8.959768420998	1.01\\
8.975951540001	1.01\\
8.996965255001	1.01\\
9.008750529999	1.01\\
9.023817404999	1.01\\
9.039830235001	1.01\\
9.055809403	1.01\\
9.071847277001	1.01\\
9.08800114	1.01\\
9.103961328003	1.01\\
9.119987125	1.01\\
9.135942833001	1.01\\
9.151918897999	1.01\\
9.167983045002	1.01\\
9.183914064	1.01\\
9.199884222001	1.01\\
9.215932814	1.01\\
9.232047870999	1.01\\
9.247946946003	1.01\\
9.266367771	1.01\\
9.281447293	1.01\\
9.296558662999	1.01\\
9.311882077	1.01\\
9.328068068001	1.01\\
9.344741407002	1.01\\
9.360038351998	1.01\\
9.375924612	1.01\\
9.391916829003	1.01\\
9.408028862999	1.01\\
9.423933091999	1.01\\
9.440000388001	1.01\\
9.455917373002	1.01\\
9.471962213002	1.01\\
9.487921048001	1.01\\
9.504001594002	1.01\\
9.519909647	1.01\\
9.535963494	1.01\\
9.551926338002	1.01\\
9.568094456002	1.01\\
9.583907863003	1.01\\
9.599976016003	1.01\\
9.615857685002	1.01\\
9.632261584999	1.01\\
9.647896281002	1.01\\
9.666407792	1.01\\
9.681685758	1.01\\
9.696860697003	1.01\\
9.712158252999	1.01\\
9.727959937001	1.01\\
9.743817842999	1.01\\
9.759914430001	1.01\\
9.775988111	1.01\\
9.791846461003	1.01\\
9.810203614003	1.01\\
9.825308544003	1.01\\
9.840627134999	1.01\\
9.856006094002	1.01\\
9.871980709999	1.01\\
9.888056434002	1.01\\
9.903976021	1.01\\
9.920013355	1.01\\
9.936031383	1.01\\
9.951926218003	1.01\\
9.968026970002	1.01\\
9.983986963002	1.01\\
10.001447153	1.01\\
10.016368028	1.01\\
10.031822548001	1.01\\
10.047799773003	1.01\\
10.065953720002	1.01\\
10.081139986	1.01\\
10.096363348999	1.01\\
10.111990807999	1.01\\
10.127993402001	1.01\\
10.143955395001	1.01\\
10.1599673	1.01\\
10.175904297001	1.01\\
10.192038328003	1.01\\
10.207997799	1.01\\
10.223928977002	1.01\\
10.239946835999	1.01\\
10.255963852002	1.01\\
10.271988622002	1.01\\
10.287912659001	1.01\\
10.304022381001	1.01\\
10.319958464001	1.01\\
10.336966664002	1.01\\
10.352348422001	1.01\\
10.367991989998	1.01\\
10.383991360001	1.01\\
10.399999832001	1.01\\
10.415900642998	1.01\\
10.432120465001	1.01\\
10.447901938004	1.01\\
10.466265418	1.01\\
10.481570071	1.01\\
10.496732486	1.01\\
10.511993499001	1.01\\
10.527957880001	1.01\\
10.543941994004	1.01\\
10.559937933999	1.01\\
10.575945018002	1.01\\
10.591919674	1.01\\
10.608060993	1.01\\
10.623935797001	1.01\\
10.639997861	1.01\\
10.655798183003	1.01\\
10.672005631001	1.01\\
10.687946520001	1.01\\
10.703983913002	1.01\\
10.719950079998	1.01\\
10.735995128998	1.01\\
10.751882338997	1.01\\
10.768000453999	1.01\\
10.783988823002	1.01\\
10.800073001999	1.01\\
10.815974385003	1.01\\
10.832087485001	1.01\\
10.847822848999	1.01\\
10.866425420998	1.01\\
10.881679506001	1.01\\
10.896918587002	1.01\\
10.912436464001	1.01\\
10.927991723999	1.01\\
10.943918095002	1.01\\
10.959866934998	1.01\\
10.975860588997	1.01\\
10.997217120999	1.01\\
11.009020912999	1.01\\
11.024013866002	1.01\\
11.039920965001	1.01\\
11.055820975998	1.01\\
11.072045386002	1.01\\
11.087917604	1.01\\
11.104006294003	1.01\\
11.119900591999	1.01\\
11.135972093003	1.01\\
11.151902643002	1.01\\
11.167958985001	1.01\\
11.183931797001	1.01\\
11.199975774998	1.01\\
11.215956008	1.01\\
11.232109494999	1.01\\
11.247964314	1.01\\
11.266374096001	1.01\\
11.281675716999	1.01\\
11.296979420002	1.01\\
11.312237937001	1.01\\
11.327933126	1.01\\
11.343936507	1.01\\
11.359863712002	1.01\\
11.375909222001	1.01\\
11.391946874001	1.01\\
11.408053736	1.01\\
11.423932635003	1.01\\
11.440000036	1.01\\
11.455955042	1.01\\
11.472015062001	1.01\\
11.487913634999	1.01\\
11.503958619	1.01\\
11.519934634003	1.01\\
11.535976954003	1.01\\
11.551961778	1.01\\
11.567964202	1.01\\
11.583957779999	1.01\\
11.600041866002	1.01\\
11.616045230999	1.01\\
11.632096801003	1.01\\
11.647924112999	1.01\\
11.664200252999	1.01\\
11.679919793999	1.01\\
11.695988860001	1.01\\
11.711868842003	1.01\\
11.727812280999	1.01\\
11.743986565003	1.01\\
11.760135897	1.01\\
11.776031024002	1.01\\
11.791942584	1.01\\
11.808089005001	1.01\\
11.823937340001	1.01\\
11.840050530003	1.01\\
11.855838647999	1.01\\
11.871979265004	1.01\\
11.887938797001	1.01\\
11.903997458001	1.01\\
11.919909097001	1.01\\
11.936039830998	1.01\\
11.951967419999	1.01\\
11.967970810997	1.01\\
11.983969050999	1.01\\
12.001599522	1.01\\
12.016568808003	1.01\\
12.031829284001	1.01\\
12.047790703	1.01\\
12.065890964001	1.01\\
12.080839043	1.01\\
12.095954201001	1.01\\
12.111942924	1.01\\
12.127987378003	1.01\\
12.143974041001	1.01\\
12.159968442002	1.01\\
12.175981953999	1.01\\
12.192007419999	1.01\\
12.207928412003	1.01\\
12.223861725003	1.01\\
12.240017630001	1.01\\
12.255962188	1.01\\
12.271990681	1.01\\
12.287961799	1.01\\
12.304034225998	1.01\\
12.319926145001	1.01\\
12.337927534001	1.01\\
12.353058877003	1.01\\
12.368194218003	1.01\\
12.383958808003	1.01\\
12.399961751999	1.01\\
12.415900472001	1.01\\
12.432049869	1.01\\
12.447965190003	1.01\\
12.464299647	1.01\\
12.479981182999	1.01\\
12.495930702	1.01\\
12.511939582001	1.01\\
12.527865420002	1.01\\
12.543899912999	1.01\\
12.560012851002	1.01\\
12.576015375	1.01\\
12.592003563	1.01\\
12.607940945	1.01\\
12.623926502999	1.01\\
12.639913945	1.01\\
12.655966564999	1.01\\
12.674400277001	1.01\\
12.689705180001	1.01\\
12.705288760003	1.01\\
12.720368246998	1.01\\
12.735971184002	1.01\\
12.751955522999	1.01\\
12.767997742001	1.01\\
12.783914523999	1.01\\
12.799949864998	1.01\\
12.815886452	1.01\\
12.832077256001	1.01\\
12.847910300999	1.01\\
12.866254262001	1.01\\
12.881557084004	1.01\\
12.896644744999	1.01\\
12.911919200001	1.01\\
12.927896482003	1.01\\
12.943907230004	1.01\\
12.959965614998	1.01\\
};
\end{axis}
\end{tikzpicture}%
}
      \caption{The steady state orientation of the robot for
        $K_{\omega}^T = 0.75 K_{\omega, max}^T$}
      \label{fig:13_0.75_max_magnified}
    \end{figure}
  \end{minipage}
\end{minipage}
}

\noindent\makebox[\textwidth][c]{%
\begin{minipage}{\linewidth}
  \begin{minipage}{0.45\linewidth}
    \begin{figure}[H]
      \scalebox{0.6}{% This file was created by matlab2tikz.
%
%The latest updates can be retrieved from
%  http://www.mathworks.com/matlabcentral/fileexchange/22022-matlab2tikz-matlab2tikz
%where you can also make suggestions and rate matlab2tikz.
%
\definecolor{mycolor1}{rgb}{0.00000,0.44700,0.74100}%
%
\begin{tikzpicture}

\begin{axis}[%
width=4.133in,
height=3.26in,
at={(0.693in,0.44in)},
scale only axis,
xmin=0,
xmax=70,
xmajorgrids,
xlabel={Time (seconds)},
ymin=0,
ymax=1.4,
ymajorgrids,
ylabel={Distance (meters)},
axis background/.style={fill=white}
]
\addplot [color=mycolor1,solid,forget plot]
  table[row sep=crcr]{%
0	0\\
0.0149769679989982	0.02\\
0.0298814690019991	0.03\\
0.0452880099989995	0.04\\
0.0612703510019984	0.05\\
0.0796002390029989	0.07\\
0.0946679150009993	0.08\\
0.109841918002998	0.09\\
0.125338816001998	0.1\\
0.141428992000999	0.12\\
0.158509467002999	0.13\\
0.173637396	0.14\\
0.189329937000999	0.16\\
0.205383828002999	0.17\\
0.221377497002	0.18\\
0.237339203999	0.19\\
0.253386097999999	0.21\\
0.269347554001	0.22\\
0.285392455001999	0.23\\
0.301374026000999	0.25\\
0.317323059001999	0.26\\
0.333382006000999	0.27\\
0.349420687004	0.28\\
0.365401540000999	0.3\\
0.381272431	0.31\\
0.397377805001	0.32\\
0.413296605998998	0.33\\
0.429382096000999	0.35\\
0.445464361	0.36\\
0.461370506999999	0.37\\
0.477387725002998	0.39\\
0.493430584998999	0.4\\
0.509440825999999	0.41\\
0.525320861	0.42\\
0.541364389999999	0.44\\
0.557315308002999	0.45\\
0.573384262000998	0.46\\
0.589460001998998	0.48\\
0.605430954002999	0.49\\
0.621528953002999	0.5\\
0.637395786998998	0.51\\
0.653439784999998	0.53\\
0.669368181	0.54\\
0.685440713002	0.55\\
0.701333077998999	0.57\\
0.717335752002999	0.58\\
0.733369598003998	0.59\\
0.749385932999	0.6\\
0.765357541999999	0.62\\
0.781376743999999	0.63\\
0.797372306999	0.64\\
0.813307006000999	0.66\\
0.829388080001999	0.67\\
0.845509367999999	0.68\\
0.861298528004	0.69\\
0.877613804001	0.71\\
0.893487119998999	0.72\\
0.909414596999999	0.73\\
0.925325130000999	0.75\\
0.941318952998999	0.76\\
0.957309926003	0.77\\
0.973441596000999	0.78\\
0.989371496002	0.8\\
1.007007971001	0.81\\
1.022066040001	0.81\\
1.037252131001	0.82\\
1.053285112	0.83\\
1.071464532002	0.83\\
1.086681516003	0.84\\
1.101742265	0.85\\
1.117361381001	0.85\\
1.133379866001	0.86\\
1.149332189003	0.87\\
1.165413042	0.87\\
1.181427989998	0.88\\
1.197339849003	0.89\\
1.213419773003	0.89\\
1.229399119999	0.9\\
1.245338037998	0.91\\
1.261381567002	0.91\\
1.277585277001	0.92\\
1.293399287998	0.93\\
1.309508429001	0.93\\
1.325371382	0.94\\
1.341372177002	0.95\\
1.357368176998	0.95\\
1.373432195	0.96\\
1.389309027001	0.97\\
1.405355789002	0.97\\
1.421381063999	0.98\\
1.437370721001	0.99\\
1.453408243	0.99\\
1.469342492001	1\\
1.485329793	1.01\\
1.501334134999	1.01\\
1.517398274002	1.02\\
1.533454841	1.03\\
1.549405461998	1.04\\
1.565333402001	1.04\\
1.581372469002	1.05\\
1.597407687001	1.06\\
1.613426168003	1.06\\
1.629404816998	1.07\\
1.645467838002	1.08\\
1.661338794998	1.08\\
1.677607357003	1.09\\
1.693301672001	1.1\\
1.709457694001	1.1\\
1.725277534001	1.11\\
1.741324158001	1.12\\
1.757385851002	1.12\\
1.773262805001	1.13\\
1.789538755001	1.14\\
1.805362193001	1.14\\
1.821314223004	1.15\\
1.837344744999	1.16\\
1.853395218003	1.16\\
1.869204003002	1.17\\
1.885382274998	1.18\\
1.901343256001	1.18\\
1.917436723	1.19\\
1.933370349003	1.2\\
1.949385390999	1.2\\
1.965349418	1.21\\
1.981406695999	1.22\\
1.999607551998	1.22\\
2.014558227001	1.21\\
2.029465181	1.21\\
2.045250501	1.2\\
2.061263824002	1.19\\
2.077259140003	1.19\\
2.093373495003	1.18\\
2.109287881001	1.17\\
2.125351133999	1.17\\
2.141373163998	1.16\\
2.157291914997	1.15\\
2.173452651001	1.15\\
2.189184375004	1.14\\
2.205352441002	1.13\\
2.221839528	1.13\\
2.237376768002	1.12\\
2.253511723	1.11\\
2.269326581002	1.11\\
2.285448947003	1.1\\
2.301350367001	1.09\\
2.317396629002	1.09\\
2.333305115002	1.08\\
2.349340904999	1.07\\
2.365342512001	1.07\\
2.381389492001	1.06\\
2.397376757	1.05\\
2.413339677002	1.05\\
2.429540575001	1.04\\
2.445460762001	1.03\\
2.46137905	1.03\\
2.477555773999	1.02\\
2.493420399998	1.01\\
2.509576389	1.01\\
2.525346737999	1\\
2.541364058003	0.99\\
2.557255335999	0.99\\
2.573391908001	0.98\\
2.589465804001	0.97\\
2.605395203003	0.97\\
2.621264550999	0.96\\
2.639505773003	0.95\\
2.654834810002	0.94\\
2.669957365002	0.94\\
2.685394563	0.93\\
2.701406859001	0.92\\
2.717414896	0.92\\
2.733323843003	0.91\\
2.749376876	0.9\\
2.765345438004	0.9\\
2.781392524002	0.89\\
2.797340958	0.88\\
2.813391764	0.88\\
2.829386863999	0.87\\
2.845479986	0.86\\
2.861432025002	0.86\\
2.877348077999	0.85\\
2.893370603001	0.84\\
2.909584967003	0.84\\
2.92540166	0.83\\
2.941473605999	0.82\\
2.957318387001	0.82\\
2.973318404003	0.81\\
2.989467661999	0.8\\
3.005253164002	0.81\\
3.021216029	0.81\\
3.037206716999	0.82\\
3.053249308998	0.82\\
3.069228589001	0.83\\
3.087308997002	0.84\\
3.102410654	0.84\\
3.117685563999	0.85\\
3.133384836003	0.86\\
3.149410414002	0.86\\
3.165386832001	0.87\\
3.181370669003	0.88\\
3.197373001999	0.88\\
3.213319286	0.89\\
3.229396007	0.89\\
3.245465478001	0.9\\
3.261419076	0.91\\
3.277564022	0.91\\
3.293366774002	0.92\\
3.311595522999	0.93\\
3.326646561001	0.93\\
3.34170664	0.94\\
3.357403421002	0.95\\
3.373359715	0.95\\
3.389366624001	0.96\\
3.405383189999	0.97\\
3.421440047001	0.97\\
3.437330348	0.98\\
3.453353833	0.98\\
3.469297870003	0.99\\
3.485425944001	1\\
3.501383165001	1\\
3.517391222004	1.01\\
3.533329771	1.02\\
3.549430447003	1.02\\
3.565329993004	1.03\\
3.581426821003	1.04\\
3.597369545002	1.04\\
3.613281769001	1.05\\
3.629287781002	1.05\\
3.645408006001	1.06\\
3.661399471001	1.07\\
3.677696402001	1.07\\
3.693450278	1.08\\
3.709453353001	1.09\\
3.725388545002	1.09\\
3.741350631001	1.1\\
3.757338739003	1.11\\
3.773537082001	1.11\\
3.789333563999	1.12\\
3.805373278	1.12\\
3.821423414002	1.13\\
3.837272268002	1.14\\
3.853381501004	1.14\\
3.869410128998	1.15\\
3.885510995999	1.16\\
3.901370917	1.16\\
3.917391021	1.17\\
3.933429912998	1.18\\
3.949650385998	1.18\\
3.965337769001	1.19\\
3.981432369	1.2\\
3.999713365002	1.2\\
4.014725631001	1.19\\
4.029700876999	1.19\\
4.045270820999	1.18\\
4.061257923001	1.17\\
4.077300923001	1.17\\
4.093346117001	1.16\\
4.109413353001	1.16\\
4.125216148003	1.15\\
4.141351328999	1.14\\
4.157309388001	1.14\\
4.173321468998	1.13\\
4.189343163002	1.13\\
4.205424307999	1.12\\
4.221453462002	1.11\\
4.237438217999	1.11\\
4.253318677002	1.1\\
4.269264216999	1.1\\
4.285305999001	1.09\\
4.301343253998	1.08\\
4.317375741001	1.08\\
4.333368408001	1.07\\
4.34939657	1.07\\
4.365333115002	1.06\\
4.381375494	1.05\\
4.397349723999	1.05\\
4.413442178002	1.04\\
4.429373129002	1.04\\
4.445477028	1.03\\
4.461348544998	1.02\\
4.477520258	1.02\\
4.493341812001	1.01\\
4.509426998001	1\\
4.525360125	1\\
4.541334111	0.99\\
4.557324570999	0.99\\
4.573597779	0.98\\
4.589336711003	0.97\\
4.605404959	0.97\\
4.621482990002	0.96\\
4.637472346001	0.96\\
4.653406126	0.95\\
4.669338361	0.94\\
4.685408656998	0.94\\
4.701283744004	0.93\\
4.717372739003	0.93\\
4.733414783001	0.92\\
4.749485416001	0.91\\
4.765360656998	0.91\\
4.781279072003	0.9\\
4.797249211998	0.9\\
4.813416578999	0.89\\
4.829414508	0.88\\
4.84543605	0.88\\
4.861365863999	0.87\\
4.877583460999	0.87\\
4.893371278004	0.86\\
4.909438289002	0.85\\
4.92540891	0.85\\
4.941417298001	0.84\\
4.957359024002	0.83\\
4.973381859001	0.83\\
4.989315131001	0.82\\
5.005239699002	0.82\\
5.021223244999	0.83\\
5.039372347	0.84\\
5.054696112999	0.84\\
5.069741012001	0.85\\
5.085413785	0.85\\
5.101335411	0.86\\
5.117457565003	0.86\\
5.133338279	0.87\\
5.149402077	0.88\\
5.165369582001	0.88\\
5.18134991	0.89\\
5.197361170002	0.89\\
5.213329611	0.9\\
5.229354824002	0.9\\
5.245502465	0.91\\
5.261341881001	0.92\\
5.277596668	0.92\\
5.293379296002	0.93\\
5.309602290001	0.93\\
5.325277124001	0.94\\
5.341339987	0.94\\
5.357358989003	0.95\\
5.373399043003	0.96\\
5.389315382	0.96\\
5.405387449002	0.97\\
5.421397343998	0.97\\
5.437312243	0.98\\
5.453398458	0.98\\
5.469340716004	0.99\\
5.485468619004	1\\
5.501364206002	1\\
5.517268762001	1.01\\
5.533311288002	1.01\\
5.549261056	1.02\\
5.565372507	1.03\\
5.581266162003	1.03\\
5.597350851002	1.04\\
5.613335159001	1.04\\
5.629353920002	1.05\\
5.645544285	1.05\\
5.661384657002	1.06\\
5.677375300003	1.07\\
5.693321120003	1.07\\
5.709486214001	1.08\\
5.725350175999	1.08\\
5.741426439999	1.09\\
5.757319534001	1.09\\
5.773397939999	1.1\\
5.789319235001	1.11\\
5.805311	1.11\\
5.821357966	1.12\\
5.837298299	1.12\\
5.853411388001	1.13\\
5.869332086003	1.13\\
5.887841788002	1.14\\
5.902990973004	1.15\\
5.918126281998	1.15\\
5.933375108002	1.16\\
5.949415087002	1.16\\
5.965366066002	1.17\\
5.981412302002	1.17\\
5.999577535	1.18\\
6.014590677002	1.17\\
6.029539007	1.17\\
6.045281191002	1.16\\
6.061242902001	1.16\\
6.079628495003	1.15\\
6.094689445	1.15\\
6.109842111	1.14\\
6.125367713002	1.13\\
6.141450963002	1.13\\
6.157416508999	1.12\\
6.173427597	1.12\\
6.189356875	1.11\\
6.205424747002	1.11\\
6.221334973	1.1\\
6.23731132	1.1\\
6.253437131001	1.09\\
6.269373545002	1.09\\
6.285347054001	1.08\\
6.301338057003	1.08\\
6.317404195	1.07\\
6.333449836998	1.06\\
6.349416114003	1.06\\
6.365349346001	1.05\\
6.381316184002	1.05\\
6.397380027001	1.04\\
6.413484096001	1.04\\
6.429387466	1.03\\
6.445445982003	1.03\\
6.461352508	1.02\\
6.477352450001	1.02\\
6.493357832001	1.01\\
6.511678911	1\\
6.526936245999	1\\
6.542041330998	0.99\\
6.557382181	0.99\\
6.573427357003	0.98\\
6.589339882	0.98\\
6.605449714001	0.97\\
6.621272584004	0.97\\
6.637381375	0.96\\
6.653439239998	0.96\\
6.669365667	0.95\\
6.685425676998	0.94\\
6.701449718998	0.94\\
6.71739132	0.93\\
6.733390351002	0.93\\
6.749440463997	0.92\\
6.765257642002	0.92\\
6.781562532002	0.91\\
6.797308465	0.91\\
6.813331735001	0.9\\
6.829335678002	0.9\\
6.845281324002	0.89\\
6.861363838002	0.89\\
6.877680056	0.88\\
6.893409872998	0.87\\
6.909429232003	0.87\\
6.925460597004	0.86\\
6.941344223	0.86\\
6.957407658001	0.85\\
6.973356786	0.85\\
6.989382313	0.84\\
7.006822600003	0.84\\
7.021893678002	0.85\\
7.037195337002	0.85\\
7.055339411999	0.86\\
7.070270395001	0.86\\
7.085318609001	0.87\\
7.101340121002	0.87\\
7.117344704998	0.88\\
7.133237490002	0.88\\
7.149354514	0.89\\
7.165330918	0.89\\
7.181341196999	0.9\\
7.197310278	0.9\\
7.213407895001	0.91\\
7.229325385002	0.91\\
7.245308777001	0.92\\
7.261381533001	0.92\\
7.277633438	0.93\\
7.293380438	0.93\\
7.309373938999	0.94\\
7.325313443001	0.94\\
7.341422099003	0.95\\
7.357478218003	0.95\\
7.373459529	0.96\\
7.389458223999	0.96\\
7.405406149998	0.97\\
7.421370081002	0.97\\
7.437267685002	0.98\\
7.453397149002	0.98\\
7.469358154003	0.99\\
7.485395675003	0.99\\
7.501324525002	1\\
7.517347293	1\\
7.533364842999	1.01\\
7.549283461003	1.01\\
7.565375008004	1.02\\
7.581400925	1.02\\
7.597383719002	1.03\\
7.613536148003	1.03\\
7.629505161	1.04\\
7.645340801998	1.04\\
7.661436949002	1.05\\
7.677374039002	1.06\\
7.693319771	1.06\\
7.709352823002	1.07\\
7.72539614	1.07\\
7.741474702004	1.08\\
7.757339176998	1.08\\
7.773322151001	1.09\\
7.789354524002	1.09\\
7.805429737999	1.1\\
7.821313024002	1.1\\
7.837346195	1.11\\
7.853359556	1.11\\
7.869346763001	1.12\\
7.885435944001	1.12\\
7.901299538002	1.13\\
7.917425574002	1.13\\
7.933365756001	1.14\\
7.949288779	1.14\\
7.967441386002	1.15\\
7.982659162003	1.15\\
7.999972023003	1.16\\
8.014886067002	1.15\\
8.029774083	1.15\\
8.045261643002	1.14\\
8.061265826	1.14\\
8.077255430001	1.13\\
8.093355716	1.13\\
8.109419254002	1.12\\
8.125507081002	1.12\\
8.141307573002	1.11\\
8.157454117001	1.11\\
8.173446359001	1.1\\
8.189377797001	1.1\\
8.205445729	1.09\\
8.221387991001	1.09\\
8.237365895001	1.08\\
8.253413055001	1.08\\
8.269373313999	1.07\\
8.285257364003	1.07\\
8.301348145001	1.07\\
8.317457567002	1.06\\
8.333385543999	1.06\\
8.349365252003	1.05\\
8.365374577	1.05\\
8.381421677002	1.04\\
8.397388309002	1.04\\
8.413383912003	1.03\\
8.429372489003	1.03\\
8.445357075001	1.02\\
8.461321723999	1.02\\
8.477624767998	1.01\\
8.493503615998	1.01\\
8.509461378998	1\\
8.525279790001	1\\
8.541479322003	0.99\\
8.557407045998	0.99\\
8.573430537998	0.98\\
8.589296031002	0.98\\
8.605381683998	0.97\\
8.621312210003	0.97\\
8.637346556	0.96\\
8.653385377003	0.96\\
8.669276108002	0.95\\
8.685519748001	0.95\\
8.701383713002	0.95\\
8.717429981999	0.94\\
8.733240848999	0.94\\
8.749401969998	0.93\\
8.765407484001	0.93\\
8.781434987	0.92\\
8.79735982	0.92\\
8.813391911	0.91\\
8.829382688	0.91\\
8.845460518002	0.9\\
8.861267528	0.9\\
8.877493790001	0.89\\
8.893358828003	0.89\\
8.909339152001	0.88\\
8.925396248001	0.88\\
8.941364882	0.87\\
8.957315293	0.87\\
8.973387332001	0.86\\
8.989361875	0.86\\
9.006762369004	0.86\\
9.021879796002	0.86\\
9.037214304001	0.87\\
9.055392862	0.87\\
9.070402097	0.88\\
9.085460738999	0.88\\
9.101374908001	0.89\\
9.117432832001	0.89\\
9.133297550003	0.9\\
9.149476218003	0.9\\
9.165328927002	0.91\\
9.181421080002	0.91\\
9.197313494999	0.92\\
9.213323792	0.92\\
9.229287697998	0.93\\
9.245396652001	0.93\\
9.261260950001	0.93\\
9.277509691002	0.94\\
9.293423264	0.94\\
9.309343367001	0.95\\
9.325429419003	0.95\\
9.341460248001	0.96\\
9.357196768002	0.96\\
9.375531243	0.97\\
9.390777036999	0.97\\
9.405944534001	0.98\\
9.42138564	0.98\\
9.437540217003	0.99\\
9.453442374001	0.99\\
9.469380813999	1\\
9.485411483002	1\\
9.501353485001	1.01\\
9.517421402001	1.01\\
9.533385710003	1.02\\
9.549404290001	1.02\\
9.565386917	1.03\\
9.581420694001	1.03\\
9.597316530999	1.04\\
9.613415939999	1.04\\
9.629427316998	1.05\\
9.645485050003	1.05\\
9.661411265	1.05\\
9.679780488003	1.06\\
9.694958244999	1.06\\
9.710239336003	1.07\\
9.72560866	1.07\\
9.741376668	1.08\\
9.757374126	1.08\\
9.773354175999	1.09\\
9.789397969002	1.09\\
9.805305987999	1.1\\
9.821383486	1.1\\
9.837351456002	1.11\\
9.853448064003	1.11\\
9.869316270001	1.12\\
9.885326683003	1.12\\
9.901216493	1.13\\
9.917395734001	1.13\\
9.933381689999	1.14\\
9.949396312001	1.14\\
9.965336114003	1.15\\
9.983768124001	1.15\\
10.002080563	1.15\\
10.014038752003	1.15\\
10.029353948002	1.14\\
10.045315089001	1.14\\
10.061360675003	1.14\\
10.077481903	1.13\\
10.093366900002	1.13\\
10.111723073998	1.12\\
10.126941917	1.12\\
10.142233163998	1.11\\
10.157473095001	1.11\\
10.173447616001	1.1\\
10.189394505001	1.1\\
10.205402665001	1.09\\
10.221409930001	1.09\\
10.237402221001	1.08\\
10.253360728001	1.08\\
10.269359568001	1.07\\
10.285386105	1.07\\
10.301384033001	1.06\\
10.317353710003	1.06\\
10.333354122002	1.05\\
10.349388298001	1.05\\
10.365373122002	1.04\\
10.381366376004	1.04\\
10.397390854	1.03\\
10.413470403	1.03\\
10.429355813	1.03\\
10.445462542	1.02\\
10.461301665001	1.02\\
10.477492056	1.01\\
10.493380115002	1.01\\
10.509345768002	1\\
10.525378925003	1\\
10.541405105	0.99\\
10.557364148003	0.99\\
10.573438010002	0.98\\
10.589381223	0.98\\
10.605389585999	0.97\\
10.622190378998	0.97\\
10.637552716999	0.96\\
10.654129463002	0.96\\
10.669439076	0.95\\
10.685445484001	0.95\\
10.701368348	0.94\\
10.717390361	0.94\\
10.733326288998	0.93\\
10.749446450001	0.93\\
10.765345945004	0.92\\
10.781418119	0.92\\
10.79733364	0.92\\
10.813368221001	0.91\\
10.829367106999	0.91\\
10.845464265999	0.9\\
10.861376779003	0.9\\
10.877490727001	0.89\\
10.893420256001	0.89\\
10.909557481003	0.88\\
10.925413117001	0.88\\
10.941329906002	0.87\\
10.957369159001	0.87\\
10.973430882	0.86\\
10.989297069001	0.86\\
11.007490713002	0.86\\
11.022678649998	0.86\\
11.037966326	0.87\\
11.053431085003	0.87\\
11.0693725	0.88\\
11.085368981999	0.88\\
11.101435109001	0.89\\
11.117410482998	0.89\\
11.133263286999	0.9\\
11.149310591004	0.9\\
11.165370009999	0.91\\
11.181368249001	0.91\\
11.197360124001	0.92\\
11.213379171002	0.92\\
11.229362374001	0.93\\
11.245550019001	0.93\\
11.261388016999	0.94\\
11.27732798	0.94\\
11.293376376004	0.94\\
11.309407695999	0.95\\
11.325303663002	0.95\\
11.341446701	0.96\\
11.357318688999	0.96\\
11.373325394001	0.97\\
11.389487975003	0.97\\
11.405411124001	0.98\\
11.421411082001	0.98\\
11.437422387001	0.99\\
11.453436242001	0.99\\
11.469352785	1\\
11.485370168	1\\
11.501357433003	1.01\\
11.517373162003	1.01\\
11.533413972	1.02\\
11.549399598	1.02\\
11.565284268002	1.03\\
11.581447729	1.03\\
11.597394998001	1.04\\
11.613383586003	1.04\\
11.629361259003	1.05\\
11.645453640999	1.05\\
11.661424028	1.06\\
11.677578110001	1.06\\
11.693346604	1.06\\
11.709468758999	1.07\\
11.725442432999	1.07\\
11.741375820004	1.08\\
11.757390687001	1.08\\
11.773462624001	1.09\\
11.789336584	1.09\\
11.805445126999	1.1\\
11.821411157002	1.1\\
11.837311082001	1.11\\
11.853312256001	1.11\\
11.869457431999	1.12\\
11.885320278	1.12\\
11.901300613003	1.13\\
11.917404929001	1.13\\
11.933346931	1.14\\
11.949387420002	1.14\\
11.965334895001	1.15\\
11.981415002999	1.15\\
11.999598297001	1.15\\
12.014838031998	1.15\\
12.030269071003	1.14\\
12.045873152001	1.14\\
12.061395352001	1.13\\
12.077523462002	1.13\\
12.093375930001	1.13\\
12.109421361	1.12\\
12.125376487999	1.12\\
12.141326002999	1.11\\
12.157467050003	1.11\\
12.173417475003	1.1\\
12.189369946999	1.1\\
12.205363400002	1.09\\
12.221458870003	1.09\\
12.237342866001	1.08\\
12.253401151001	1.08\\
12.269346846001	1.07\\
12.285330021	1.07\\
12.301384118	1.06\\
12.317388952	1.06\\
12.333384896	1.05\\
12.349388016999	1.05\\
12.365200405003	1.04\\
12.383331507	1.04\\
12.398237828999	1.03\\
12.413240239998	1.03\\
12.429381138001	1.02\\
12.445489417	1.02\\
12.461389880001	1.01\\
12.477609862	1.01\\
12.493312432003	1.01\\
12.509492177002	1\\
12.525438253002	1\\
12.541349600003	0.99\\
12.557371482998	0.99\\
12.573311889	0.98\\
12.589303513001	0.98\\
12.605415452999	0.97\\
12.621345233002	0.97\\
12.637254295002	0.96\\
12.653283915001	0.96\\
12.669342236	0.95\\
12.685379865002	0.95\\
12.701314630001	0.94\\
12.717428440998	0.94\\
12.733334015999	0.93\\
12.749413534001	0.93\\
12.765311120003	0.92\\
12.781405172001	0.92\\
12.797342474003	0.91\\
12.813379962002	0.91\\
12.829367970001	0.9\\
12.845317221001	0.9\\
12.861315943001	0.9\\
12.877441363003	0.89\\
12.893417293999	0.89\\
12.909541054001	0.88\\
12.925374588002	0.88\\
12.94136523	0.87\\
12.957358955002	0.87\\
12.973408189999	0.86\\
12.989392668003	0.86\\
13.006757980999	0.86\\
13.021822809998	0.86\\
13.037378556	0.87\\
13.053402714001	0.87\\
13.069365711998	0.88\\
13.085418492001	0.88\\
13.101375663002	0.88\\
13.120098883	0.89\\
13.135332475003	0.9\\
13.150575868	0.9\\
13.165829290001	0.9\\
13.181384394001	0.91\\
13.197358128002	0.91\\
13.213417516999	0.92\\
13.229406200001	0.92\\
13.245519093003	0.93\\
13.261414605	0.93\\
13.27759834	0.94\\
13.293404684002	0.94\\
13.309609049	0.95\\
13.325432219002	0.95\\
13.341322388001	0.96\\
13.357315550999	0.96\\
13.373447077	0.97\\
13.389459245003	0.97\\
13.405505466999	0.98\\
13.42131316	0.98\\
13.437348804001	0.99\\
13.45342325	0.99\\
13.469268158001	0.99\\
13.485386449002	1\\
13.501375938999	1\\
13.517413617001	1.01\\
13.533359202	1.01\\
13.549411965	1.02\\
13.565585612	1.02\\
13.581455380001	1.03\\
13.597312772999	1.03\\
13.613388521	1.04\\
13.629373180001	1.04\\
13.645421122002	1.05\\
13.661440449002	1.05\\
13.677388056	1.06\\
13.693394107998	1.06\\
13.709480328999	1.07\\
13.725308721001	1.07\\
13.741458521	1.08\\
13.75732107	1.08\\
13.773417502003	1.09\\
13.789398992001	1.09\\
13.805369162003	1.1\\
13.821391712002	1.1\\
13.837313228001	1.11\\
13.853539544003	1.11\\
13.869373376004	1.11\\
13.885421213002	1.12\\
13.901472126	1.12\\
13.917414138001	1.13\\
13.933337756001	1.13\\
13.949266816998	1.14\\
13.965399450001	1.14\\
13.981432939999	1.15\\
14.000691298001	1.15\\
14.013531568001	1.15\\
14.029324218003	1.14\\
14.045429696003	1.14\\
14.061309109001	1.13\\
14.079653272003	1.13\\
14.09488975	1.12\\
14.110183964001	1.12\\
14.125464657002	1.12\\
14.141372471001	1.11\\
14.157340842003	1.11\\
14.173384966004	1.1\\
14.189363156002	1.1\\
14.205425544998	1.09\\
14.221409647999	1.09\\
14.237359457001	1.08\\
14.253270776001	1.08\\
14.269389617001	1.08\\
14.285388036999	1.07\\
14.3012953	1.07\\
14.317412870003	1.06\\
14.333337883	1.06\\
14.349286862004	1.05\\
14.365354603001	1.05\\
14.381423037003	1.04\\
14.397304665001	1.04\\
14.413423504002	1.03\\
14.429408628998	1.03\\
14.445382485001	1.03\\
14.461406211003	1.02\\
14.477648383	1.02\\
14.493396200001	1.01\\
14.509607502003	1.01\\
14.525332237	1\\
14.541441438	1\\
14.557380397	0.99\\
14.573453126	0.99\\
14.589218541001	0.99\\
14.605409985001	0.98\\
14.621362404999	0.98\\
14.637347514	0.97\\
14.653419712002	0.97\\
14.669411421002	0.96\\
14.685361141003	0.96\\
14.701326170002	0.95\\
14.717405466999	0.95\\
14.733325967003	0.95\\
14.749414956002	0.94\\
14.765412389	0.94\\
14.781424084004	0.93\\
14.797353292	0.93\\
14.813337839001	0.92\\
14.829333639004	0.92\\
14.845446718998	0.91\\
14.861378598004	0.91\\
14.877582493004	0.91\\
14.893289367001	0.9\\
14.909592007	0.9\\
14.925350508004	0.89\\
14.941399006001	0.89\\
14.957308074002	0.88\\
14.973320599999	0.88\\
14.989476710003	0.87\\
15.006956927998	0.87\\
15.022305621002	0.88\\
15.037462622002	0.88\\
15.053460429001	0.89\\
15.069271327	0.89\\
15.085436946999	0.89\\
15.101334531998	0.9\\
15.117409137001	0.9\\
15.133312622002	0.91\\
15.149332997002	0.91\\
15.165408920998	0.91\\
15.181416882	0.92\\
15.197384187001	0.92\\
15.213381468998	0.93\\
15.229344228001	0.93\\
15.245470914002	0.94\\
15.261388715	0.94\\
15.277626948998	0.94\\
15.293380168003	0.95\\
15.309469895001	0.95\\
15.325357036999	0.96\\
15.341429254002	0.96\\
15.357360622002	0.96\\
15.373379694001	0.97\\
15.389313381001	0.97\\
15.405398854	0.98\\
15.421404701	0.98\\
15.437446975003	0.99\\
15.453421061001	0.99\\
15.469366304001	0.99\\
15.486274937001	1\\
15.501387860001	1\\
15.517437745999	1.01\\
15.533284694001	1.01\\
15.549467595001	1.01\\
15.565400101002	1.02\\
15.581340914002	1.02\\
15.597344597	1.03\\
15.613394483002	1.03\\
15.629369159001	1.03\\
15.645304977001	1.04\\
15.661390975003	1.04\\
15.677544241001	1.05\\
15.693369870003	1.05\\
15.711739188	1.06\\
15.727025986	1.06\\
15.742149487999	1.06\\
15.757437077004	1.07\\
15.773336861	1.07\\
15.789316063	1.08\\
15.805445866001	1.08\\
15.821488523003	1.08\\
15.837412903004	1.09\\
15.853556261002	1.09\\
15.869345025002	1.1\\
15.885359261002	1.1\\
15.901330424	1.11\\
15.917306087002	1.11\\
15.933382542	1.11\\
15.949411047001	1.12\\
15.965372306999	1.12\\
15.981384083	1.13\\
15.999889755001	1.13\\
16.015334921002	1.12\\
16.030622394001	1.12\\
16.045839278	1.12\\
16.061390214001	1.11\\
16.077529156998	1.11\\
16.093264137001	1.11\\
16.111752182003	1.1\\
16.126880564999	1.1\\
16.142134591999	1.09\\
16.157455427002	1.09\\
16.173414046002	1.09\\
16.189362419998	1.08\\
16.205375463002	1.08\\
16.221394457001	1.08\\
16.237357291001	1.07\\
16.253296487	1.07\\
16.269382692002	1.06\\
16.285412566998	1.06\\
16.301342847	1.06\\
16.317435786	1.05\\
16.333403276001	1.05\\
16.349297794003	1.04\\
16.365399259003	1.04\\
16.381409584999	1.04\\
16.397439401001	1.03\\
16.413427042	1.03\\
16.429409199002	1.03\\
16.445463405999	1.02\\
16.461332383999	1.02\\
16.477496878002	1.01\\
16.493377876999	1.01\\
16.509463209	1.01\\
16.525334571003	1\\
16.541352699997	1\\
16.557354101998	0.99\\
16.573397282997	0.99\\
16.589203816998	0.99\\
16.607472252999	0.98\\
16.622631616001	0.98\\
16.637942470001	0.98\\
16.653388530003	0.97\\
16.669366663998	0.97\\
16.685359049	0.96\\
16.701280501004	0.96\\
16.717389456002	0.96\\
16.733341781002	0.95\\
16.749297411	0.95\\
16.765400647	0.94\\
16.781379968003	0.94\\
16.797396564003	0.94\\
16.813380060002	0.93\\
16.829448082001	0.93\\
16.845301110001	0.93\\
16.861376647	0.92\\
16.877608024998	0.92\\
16.893382323002	0.91\\
16.909451152001	0.91\\
16.925442831002	0.91\\
16.941293141003	0.9\\
16.957397430001	0.9\\
16.973432904	0.9\\
16.989346636002	0.89\\
17.005446185002	0.89\\
17.021352152001	0.9\\
17.037423711003	0.9\\
17.053530763001	0.9\\
17.069332477001	0.91\\
17.085386056	0.91\\
17.101393676998	0.91\\
17.117479709999	0.92\\
17.133335883999	0.92\\
17.149238216999	0.92\\
17.167547544998	0.93\\
17.182743208	0.93\\
17.197957375	0.93\\
17.213424658001	0.94\\
17.229329320004	0.94\\
17.245446367001	0.94\\
17.261391736	0.95\\
17.277520786999	0.95\\
17.293263248001	0.95\\
17.311676870999	0.96\\
17.326904985001	0.96\\
17.342001920998	0.97\\
17.357357058003	0.97\\
17.373433680001	0.97\\
17.389334013001	0.98\\
17.405381123001	0.98\\
17.421241649002	0.98\\
17.437352284001	0.99\\
17.45332734	0.99\\
17.469346147003	0.99\\
17.485393242001	1\\
17.501280595001	1\\
17.517381987999	1\\
17.533346356003	1.01\\
17.549386956002	1.01\\
17.565332351002	1.01\\
17.581516016003	1.02\\
17.597379124001	1.02\\
17.613346779	1.03\\
17.629479192002	1.03\\
17.645456620003	1.03\\
17.661434107003	1.04\\
17.677529502003	1.04\\
17.693340591999	1.04\\
17.711715362999	1.05\\
17.726859707001	1.05\\
17.741964666001	1.05\\
17.757363894001	1.06\\
17.773269182999	1.06\\
17.789390279999	1.06\\
17.805394170002	1.07\\
17.821503714001	1.07\\
17.83736609	1.07\\
17.853300943001	1.08\\
17.869375182999	1.08\\
17.885417252003	1.08\\
17.901398089001	1.09\\
17.917432564999	1.09\\
17.933411403	1.1\\
17.949400883	1.1\\
17.965377879002	1.1\\
17.981393688999	1.11\\
18.000128846001	1.11\\
18.013317285	1.1\\
18.029399446003	1.1\\
18.045395754002	1.1\\
18.061372041001	1.1\\
18.077465612004	1.09\\
18.093309556	1.09\\
18.111685495003	1.09\\
18.126894063	1.08\\
18.142331924	1.08\\
18.157615826	1.08\\
18.173365270001	1.07\\
18.189309548001	1.07\\
18.205361663002	1.07\\
18.221635472	1.06\\
18.23735698	1.06\\
18.253423554001	1.06\\
18.269327309002	1.05\\
18.285323826	1.05\\
18.301334833	1.05\\
18.317403076	1.04\\
18.333345499001	1.04\\
18.349215167	1.04\\
18.365365501	1.03\\
18.381380647999	1.03\\
18.397398633	1.03\\
18.413302554001	1.03\\
18.429393758	1.02\\
18.445448075001	1.02\\
18.461320970001	1.02\\
18.477403806999	1.01\\
18.493367232003	1.01\\
18.509999802002	1.01\\
18.525410952999	1\\
18.541513514	1\\
18.557390498001	1\\
18.573347161	0.99\\
18.589371111	0.99\\
18.605415004002	0.99\\
18.621420369	0.98\\
18.637567829998	0.98\\
18.653522988999	0.98\\
18.669440110001	0.97\\
18.685366180001	0.97\\
18.701440031002	0.97\\
18.717434527001	0.96\\
18.733383309002	0.96\\
18.749407510002	0.96\\
18.765387992001	0.95\\
18.781393548001	0.95\\
18.797350238003	0.95\\
18.813427987	0.95\\
18.829392611004	0.94\\
18.845464516003	0.94\\
18.861327997002	0.94\\
18.877426297001	0.93\\
18.893379317002	0.93\\
18.909445237	0.93\\
18.925366293	0.92\\
18.94130759	0.92\\
18.957339592999	0.92\\
18.973374726002	0.91\\
18.989382354	0.91\\
19.006792334	0.91\\
19.022070776001	0.91\\
19.037350922001	0.92\\
19.053457967999	0.92\\
19.069327073002	0.92\\
19.085374133	0.93\\
19.101328423001	0.93\\
19.117449716	0.93\\
19.133309527001	0.93\\
19.149307676003	0.94\\
19.165370086003	0.94\\
19.181445150002	0.94\\
19.197379123001	0.95\\
19.213453144001	0.95\\
19.229461126	0.95\\
19.245462487	0.95\\
19.261409436001	0.96\\
19.277696167	0.96\\
19.293397776001	0.96\\
19.309613689003	0.97\\
19.325308581002	0.97\\
19.341309895001	0.97\\
19.357327801003	0.97\\
19.373426843003	0.98\\
19.389292962998	0.98\\
19.405324223	0.98\\
19.421364769001	0.99\\
19.437527952999	0.99\\
19.453376551998	0.99\\
19.469349394001	0.99\\
19.485405188	1\\
19.501348192002	1\\
19.517456958	1\\
19.533307574002	1.01\\
19.549467318001	1.01\\
19.565416380001	1.01\\
19.581368771	1.01\\
19.597345166001	1.02\\
19.613328260998	1.02\\
19.629308976002	1.02\\
19.645522463002	1.03\\
19.661312695999	1.03\\
19.677623714001	1.03\\
19.693335419003	1.03\\
19.709670452	1.04\\
19.725272581002	1.04\\
19.743681938004	1.04\\
19.758999102001	1.05\\
19.774210149002	1.05\\
19.789449679001	1.05\\
19.805449376	1.05\\
19.821430290001	1.06\\
19.837382877003	1.06\\
19.853330655003	1.06\\
19.869310602001	1.07\\
19.885414243	1.07\\
19.901338901001	1.07\\
19.917389523999	1.07\\
19.933382131001	1.08\\
19.949402079003	1.08\\
19.965370597	1.08\\
19.981438569001	1.09\\
19.999567133999	1.09\\
20.014925511002	1.08\\
20.030227152001	1.08\\
20.045536904999	1.08\\
20.061337387001	1.08\\
20.077579008	1.07\\
20.093333791001	1.07\\
20.109383046002	1.07\\
20.125365559002	1.07\\
20.141466256001	1.06\\
20.157279384999	1.06\\
20.173420585003	1.06\\
20.18936	1.06\\
20.205411183003	1.05\\
20.221381386002	1.05\\
20.237355329998	1.05\\
20.253411175	1.05\\
20.269380987	1.04\\
20.285425855	1.04\\
20.301471111	1.04\\
20.317400790001	1.04\\
20.333347092003	1.03\\
20.349350006001	1.03\\
20.365319015	1.03\\
20.381419656998	1.03\\
20.397365958	1.02\\
20.413443959004	1.02\\
20.429332159001	1.02\\
20.445495111	1.02\\
20.461264825001	1.01\\
20.479688243004	1.01\\
20.495007754002	1.01\\
20.510107516999	1.01\\
20.525524518997	1\\
20.541730028	1\\
20.557341402001	1\\
20.573408281002	1\\
20.589368629002	0.99\\
20.605453699002	0.99\\
20.621309211003	0.99\\
20.637281259999	0.99\\
20.653376867001	0.98\\
20.669336439003	0.98\\
20.68545459	0.98\\
20.701334298001	0.98\\
20.717423857003	0.97\\
20.733355646	0.97\\
20.749246365998	0.97\\
20.765350314999	0.97\\
20.781213654999	0.96\\
20.797350512001	0.96\\
20.813473685002	0.96\\
20.829375570999	0.96\\
20.845396161999	0.95\\
20.861368061001	0.95\\
20.877632969002	0.95\\
20.893356797001	0.94\\
20.909366347	0.94\\
20.925328458	0.94\\
20.941376041001	0.94\\
20.957383163002	0.93\\
20.973432706002	0.93\\
20.989468064999	0.93\\
21.007127908001	0.93\\
21.022450499001	0.93\\
21.037818088002	0.93\\
21.053420675999	0.94\\
21.069363293999	0.94\\
21.085397770001	0.94\\
21.10133205	0.94\\
21.117397406002	0.95\\
21.133322115002	0.95\\
21.149383286999	0.95\\
21.165361200001	0.95\\
21.181393308003	0.96\\
21.197368598004	0.96\\
21.213379165001	0.96\\
21.229355074002	0.96\\
21.245474198002	0.97\\
21.261410422001	0.97\\
21.277639379002	0.97\\
21.293374136002	0.97\\
21.309487767002	0.98\\
21.325358783001	0.98\\
21.341373345001	0.98\\
21.357297354	0.99\\
21.373442543003	0.99\\
21.389401011997	0.99\\
21.405331400002	0.99\\
21.421423124001	1\\
21.437324879002	1\\
21.453570877003	1\\
21.469398430001	1\\
21.485427860001	1.01\\
21.501340234001	1.01\\
21.517428371002	1.01\\
21.533402206997	1.01\\
21.549218036999	1.02\\
21.565391062001	1.02\\
21.581400432003	1.02\\
21.597381157002	1.02\\
21.613382668003	1.03\\
21.629321248001	1.03\\
21.645336276001	1.03\\
21.661380272999	1.03\\
21.677451098	1.04\\
21.693352530003	1.04\\
21.709722127999	1.04\\
21.725347594998	1.04\\
21.741433857003	1.05\\
21.757362859001	1.05\\
21.773437542	1.05\\
21.789292365002	1.05\\
21.805394546002	1.06\\
21.821273742001	1.06\\
21.837416868	1.06\\
21.853408572998	1.06\\
21.869366281002	1.07\\
21.885471242001	1.07\\
21.901337206002	1.07\\
21.917408641999	1.07\\
21.933218039002	1.08\\
21.949374455998	1.08\\
21.965363515	1.08\\
21.981382195	1.08\\
22.000160365002	1.09\\
22.01544407	1.08\\
22.030594211998	1.08\\
22.045822978001	1.08\\
22.061311906002	1.08\\
22.077483561001	1.07\\
22.093282891003	1.07\\
22.111742778	1.07\\
22.125295492001	1.07\\
22.141447913002	1.06\\
22.157345479	1.06\\
22.17337975	1.06\\
22.189431696999	1.06\\
22.205359556999	1.05\\
22.221390144001	1.05\\
22.237538258	1.05\\
22.253327404	1.05\\
22.269372920002	1.04\\
22.285369420998	1.04\\
22.301334781002	1.04\\
22.317352553002	1.04\\
22.333342254998	1.03\\
22.349401477001	1.03\\
22.365350755001	1.03\\
22.381380907002	1.02\\
22.397387329998	1.02\\
22.413208239998	1.02\\
22.429397320999	1.02\\
22.44548355	1.01\\
22.461309539002	1.01\\
22.477569665001	1.01\\
22.493354541001	1.01\\
22.509486131001	1\\
22.525419926003	1\\
22.541373805001	1\\
22.557299487999	1\\
22.573424757	0.99\\
22.589455675999	0.99\\
22.605388617001	0.99\\
22.621410594998	0.99\\
22.637278041001	0.98\\
22.654175721001	0.98\\
22.669476765	0.98\\
22.68538891	0.98\\
22.701354765999	0.97\\
22.717403435002	0.97\\
22.733349110001	0.97\\
22.749373438	0.97\\
22.76536623	0.96\\
22.781422495999	0.96\\
22.797283186001	0.96\\
22.813375411	0.96\\
22.829284944001	0.95\\
22.845522436001	0.95\\
22.861360773003	0.95\\
22.8775345	0.95\\
22.893380114003	0.94\\
22.90939832	0.94\\
22.925379898999	0.94\\
22.941301598999	0.94\\
22.957364115998	0.93\\
22.973378071003	0.93\\
22.989346185002	0.93\\
23.006880417	0.93\\
23.022006515999	0.93\\
23.037473	0.93\\
23.053377026001	0.94\\
23.069373782002	0.94\\
23.085340877999	0.94\\
23.101331333	0.94\\
23.117198808998	0.95\\
23.135439848	0.95\\
23.150548712002	0.95\\
23.165629963002	0.95\\
23.181247477001	0.96\\
23.197351749001	0.96\\
23.21340864	0.96\\
23.229318303002	0.96\\
23.245417914002	0.97\\
23.261341607003	0.97\\
23.277304734001	0.97\\
23.293319952004	0.97\\
23.311700055001	0.98\\
23.326920455002	0.98\\
23.342255779	0.98\\
23.357513520001	0.98\\
23.373404209999	0.99\\
23.389428534001	0.99\\
23.407760613003	0.99\\
23.422949603001	0.99\\
23.438044685002	1\\
23.453349612999	1\\
23.469276267998	1\\
23.485399012001	1\\
23.501358994999	1.01\\
23.517515218003	1.01\\
23.533352858002	1.01\\
23.549389504002	1.01\\
23.565469659001	1.02\\
23.581391355	1.02\\
23.597337361	1.02\\
23.613297543003	1.02\\
23.629332487	1.03\\
23.645546747002	1.03\\
23.661380858002	1.03\\
23.677428994	1.03\\
23.69336973	1.04\\
23.711704007	1.04\\
23.726997646	1.04\\
23.742110729	1.05\\
23.757419246002	1.05\\
23.773412278	1.05\\
23.789326963002	1.05\\
23.805331834	1.05\\
23.821432444001	1.06\\
23.837413108998	1.06\\
23.853416058998	1.06\\
23.869343764	1.07\\
23.885385301003	1.07\\
23.901343825001	1.07\\
23.917419795998	1.07\\
23.933314260002	1.08\\
23.949393467999	1.08\\
23.965321969002	1.08\\
23.981427567002	1.08\\
23.999751347	1.08\\
24.014900487	1.08\\
24.030093420998	1.08\\
24.045612424	1.08\\
24.061455529003	1.07\\
24.077618415001	1.07\\
24.093387445	1.07\\
24.109697876999	1.07\\
24.125389182003	1.06\\
24.141369073002	1.06\\
24.157388751	1.06\\
24.173494884999	1.06\\
24.189413224003	1.05\\
24.205473153	1.05\\
24.221433209	1.05\\
24.23739614	1.05\\
24.253376624001	1.04\\
24.269334688	1.04\\
24.285427291001	1.04\\
24.301338613003	1.04\\
24.317388113999	1.03\\
24.333364534001	1.03\\
24.349389688999	1.03\\
24.365339131001	1.03\\
24.381341704003	1.02\\
24.397283028	1.02\\
24.413362906998	1.02\\
24.429398545998	1.02\\
24.445442506001	1.01\\
24.461437486	1.01\\
24.477405297001	1.01\\
24.493341265	1.01\\
24.509386651001	1\\
24.525402897003	1\\
24.541397112	1\\
24.557344967003	1\\
24.573357029999	0.99\\
24.58944259	0.99\\
24.605402281998	0.99\\
24.621207265003	0.99\\
24.637293777001	0.98\\
24.65343255	0.98\\
24.669379559002	0.98\\
24.685421958004	0.97\\
24.701408181	0.97\\
24.717268465	0.97\\
24.733346161999	0.97\\
24.749314292	0.96\\
24.765344690003	0.96\\
24.781366243004	0.96\\
24.797368289002	0.96\\
24.813317920002	0.95\\
24.829388849003	0.95\\
24.845460084004	0.95\\
24.861386830998	0.95\\
24.877557191002	0.94\\
24.893371182999	0.94\\
24.909567872002	0.94\\
24.925375959999	0.94\\
24.941337135998	0.93\\
24.957380758004	0.93\\
24.973412097	0.93\\
24.989343494999	0.93\\
25.006764598999	0.93\\
25.022055239998	0.93\\
25.037453204003	0.93\\
25.053374695	0.93\\
25.069330873001	0.94\\
25.085275705998	0.94\\
25.101339114003	0.94\\
25.117369792	0.94\\
25.133275225998	0.95\\
25.149388025002	0.95\\
25.165494114003	0.95\\
25.181390467999	0.95\\
25.197451703999	0.96\\
25.213432338997	0.96\\
25.229379773003	0.96\\
25.245360220001	0.96\\
25.261375745999	0.97\\
25.277474813999	0.97\\
25.293365304001	0.97\\
25.309389978001	0.97\\
25.325340395001	0.98\\
25.341391314999	0.98\\
25.357363398003	0.98\\
25.373316924	0.98\\
25.389423190003	0.99\\
25.405390916001	0.99\\
25.421342621002	0.99\\
25.437343991001	0.99\\
25.453525161999	1\\
25.469315712998	1\\
25.485353101002	1\\
25.501338227001	1.01\\
25.517394563004	1.01\\
25.533372746998	1.01\\
25.549653532002	1.01\\
25.565311715	1.02\\
25.581375287998	1.02\\
25.597393827999	1.02\\
25.613416417	1.02\\
25.629288830002	1.03\\
25.645486629002	1.03\\
25.661416448002	1.03\\
25.677496063999	1.03\\
25.693337306	1.04\\
25.711714852001	1.04\\
25.726942404999	1.04\\
25.741987419003	1.04\\
25.757355986	1.05\\
25.773362805001	1.05\\
25.789381486	1.05\\
25.805388244999	1.05\\
25.821586492001	1.06\\
25.837403570999	1.06\\
25.853354372002	1.06\\
25.869290501	1.06\\
25.885377952999	1.07\\
25.901374118	1.07\\
25.917396682003	1.07\\
25.933320905999	1.07\\
25.949421355999	1.08\\
25.965352813	1.08\\
25.981414126999	1.08\\
25.999913022	1.08\\
26.015191631001	1.08\\
26.030393666001	1.08\\
26.045547836998	1.08\\
26.061423484001	1.07\\
26.077581686001	1.07\\
26.093373923001	1.07\\
26.109634392002	1.07\\
26.125405241001	1.06\\
26.141375216	1.06\\
26.159408398003	1.06\\
26.174601713002	1.05\\
26.190087779	1.05\\
26.205399355	1.05\\
26.221403338002	1.05\\
26.237355226002	1.04\\
26.253278368	1.04\\
26.269311690003	1.04\\
26.28539425	1.04\\
26.301364827	1.03\\
26.317411937001	1.03\\
26.333386423001	1.03\\
26.349301392002	1.03\\
26.365458748001	1.02\\
26.3813233	1.02\\
26.397305588002	1.02\\
26.413344035004	1.02\\
26.429423317002	1.01\\
26.445412466	1.01\\
26.461363501999	1.01\\
26.479868920002	1.01\\
26.495021226002	1\\
26.510154037003	1\\
26.525502065003	1\\
26.541376534001	1\\
26.557342104	0.99\\
26.573257180001	0.99\\
26.589449038998	0.99\\
26.605404832001	0.99\\
26.621269553002	0.98\\
26.637284646	0.98\\
26.653329541001	0.98\\
26.669328186001	0.98\\
26.685389865002	0.97\\
26.701382303002	0.97\\
26.717310259999	0.97\\
26.733263367001	0.97\\
26.749450566002	0.96\\
26.765406092999	0.96\\
26.781398786	0.96\\
26.797381484001	0.96\\
26.813385716999	0.95\\
26.829382112999	0.95\\
26.845462779003	0.95\\
26.861386599003	0.95\\
26.877567904003	0.94\\
26.893355244999	0.94\\
26.909608633	0.94\\
26.925283246002	0.94\\
26.94142559	0.93\\
26.957385663002	0.93\\
26.973400801003	0.93\\
26.989424783001	0.93\\
27.006746038002	0.93\\
27.022010446999	0.93\\
27.037431749001	0.93\\
27.053488530003	0.93\\
27.069362113003	0.93\\
27.085479603001	0.94\\
27.101356988999	0.94\\
27.117417825001	0.94\\
27.133334966004	0.95\\
27.149395495999	0.95\\
27.165384780003	0.95\\
27.181387755001	0.95\\
27.197381289002	0.96\\
27.213364674	0.96\\
27.229368502003	0.96\\
27.245398528	0.96\\
27.261380863003	0.97\\
27.277502501999	0.97\\
27.293381950001	0.97\\
27.309724891003	0.97\\
27.325389404	0.98\\
27.341378744	0.98\\
27.357250568001	0.98\\
27.373361707001	0.98\\
27.389250694001	0.99\\
27.405404515	0.99\\
27.421408538002	0.99\\
27.437384242001	0.99\\
27.453372307999	1\\
27.469383168003	1\\
27.485543334999	1\\
27.501352067002	1\\
27.517393971001	1.01\\
27.533333642002	1.01\\
27.549412188999	1.01\\
27.565350633004	1.01\\
27.581374767002	1.02\\
27.597355512001	1.02\\
27.613450526001	1.02\\
27.629346672001	1.02\\
27.645462834	1.03\\
27.661401847	1.03\\
27.677572068001	1.03\\
27.693349641999	1.03\\
27.709585054001	1.04\\
27.725694287003	1.04\\
27.741286015999	1.04\\
27.757346639	1.04\\
27.773387380001	1.05\\
27.789429483002	1.05\\
27.805378246002	1.05\\
27.821418612	1.05\\
27.837313477001	1.06\\
27.853411151001	1.06\\
27.869445556999	1.06\\
27.885656635002	1.06\\
27.901346619	1.07\\
27.917407907002	1.07\\
27.933363770001	1.07\\
27.949501661003	1.07\\
27.965408161999	1.08\\
27.981462605	1.08\\
27.999702605	1.08\\
28.015042606999	1.08\\
28.030310827004	1.08\\
28.045563657002	1.07\\
28.061355713002	1.07\\
28.077450036	1.07\\
28.093344093998	1.07\\
28.111739466999	1.06\\
28.125412246002	1.06\\
28.141474678002	1.06\\
28.157389724003	1.06\\
28.173414657002	1.05\\
28.189427690998	1.05\\
28.205432752003	1.05\\
28.221345354	1.05\\
28.237358999001	1.04\\
28.253409929001	1.04\\
28.269400870003	1.04\\
28.285426121002	1.04\\
28.301386350003	1.03\\
28.317393108002	1.03\\
28.333379474999	1.03\\
28.349485057999	1.03\\
28.365333097	1.02\\
28.381437678002	1.02\\
28.397390850003	1.02\\
28.413329422001	1.02\\
28.429301043003	1.01\\
28.445526869	1.01\\
28.461393715	1.01\\
28.477575167	1.01\\
28.493364245003	1\\
28.509391097	1\\
28.525361045998	1\\
28.541388577999	1\\
28.557359207001	0.99\\
28.573332841	0.99\\
28.589689081002	0.99\\
28.605393174	0.98\\
28.621389258999	0.98\\
28.637447825001	0.98\\
28.653267554001	0.98\\
28.669325013001	0.97\\
28.685325036	0.97\\
28.703680126	0.97\\
28.718860994999	0.97\\
28.734110877999	0.96\\
28.749399454003	0.96\\
28.765371904999	0.96\\
28.781379901001	0.96\\
28.797391491001	0.95\\
28.813391669998	0.95\\
28.829367335003	0.95\\
28.845455667	0.95\\
28.861348708	0.94\\
28.877378272999	0.94\\
28.893298774002	0.94\\
28.909443883999	0.94\\
28.925311462002	0.93\\
28.94139723	0.93\\
28.957388428002	0.93\\
28.973360082001	0.93\\
28.989369492001	0.92\\
29.006963829003	0.92\\
29.022121928002	0.93\\
29.037434986	0.93\\
29.053408424	0.93\\
29.06933898	0.93\\
29.085308604	0.94\\
29.101372341999	0.94\\
29.117432862999	0.94\\
29.133306344002	0.94\\
29.149385297001	0.95\\
29.165426264	0.95\\
29.181295042	0.95\\
29.197336633999	0.95\\
29.213343562001	0.96\\
29.229343389	0.96\\
29.245591364003	0.96\\
29.261378069001	0.96\\
29.277457199002	0.97\\
29.293444732003	0.97\\
29.309662864998	0.97\\
29.325343083	0.97\\
29.341433527001	0.98\\
29.357257594998	0.98\\
29.373304805001	0.98\\
29.389330343003	0.98\\
29.405405527001	0.99\\
29.421401872002	0.99\\
29.437380337002	0.99\\
29.453528587002	0.99\\
29.469374220001	1\\
29.485412650002	1\\
29.501340557999	1\\
29.517395418	1\\
29.533341218003	1.01\\
29.549375469002	1.01\\
29.565380815003	1.01\\
29.581290097	1.01\\
29.597366483998	1.02\\
29.613313500004	1.02\\
29.629390881001	1.02\\
29.645478177002	1.03\\
29.661381155999	1.03\\
29.677638202	1.03\\
29.693349681999	1.03\\
29.709421548001	1.04\\
29.725362953999	1.04\\
29.741367101002	1.04\\
29.757308071999	1.04\\
29.773408241001	1.05\\
29.789300683003	1.05\\
29.805359039002	1.05\\
29.821472537003	1.05\\
29.837333600003	1.06\\
29.853348627999	1.06\\
29.869337942002	1.06\\
29.885379986	1.06\\
29.901309978001	1.07\\
29.917427134999	1.07\\
29.933365944001	1.07\\
29.949379944001	1.07\\
29.965387099003	1.08\\
29.981380181999	1.08\\
29.999583674004	1.08\\
30.014898265999	1.08\\
30.029981084	1.08\\
30.045462245999	1.07\\
30.061477303002	1.07\\
30.077463726002	1.07\\
30.093386264	1.07\\
30.109581063999	1.06\\
30.125294132004	1.06\\
30.141368180001	1.06\\
30.157461238999	1.06\\
30.173260245999	1.06\\
30.189325816998	1.05\\
30.205369809998	1.05\\
30.223699133	1.05\\
30.238814657002	1.05\\
30.254081158001	1.04\\
30.269349027001	1.04\\
30.285388369	1.04\\
30.301362464001	1.04\\
30.317409890003	1.04\\
30.333311061001	1.03\\
30.349393744999	1.03\\
30.365362483002	1.03\\
30.381403233002	1.03\\
30.397401348	1.02\\
30.413398428997	1.02\\
30.429402848	1.02\\
30.445347379998	1.02\\
30.461412665001	1.02\\
30.477508830002	1.01\\
30.493361783001	1.01\\
30.509446619	1.01\\
30.525342201	1.01\\
30.541349145001	1\\
30.557318749001	1\\
30.573299549	1\\
30.589411783001	1\\
30.605339598999	1\\
30.621460063	0.99\\
30.637310057999	0.99\\
30.653354235001	0.99\\
30.669382082001	0.99\\
30.685372438004	0.98\\
30.701373834999	0.98\\
30.717433834	0.98\\
30.733394995003	0.98\\
30.749428860001	0.98\\
30.765386362	0.97\\
30.781417867001	0.97\\
30.797448108998	0.97\\
30.813388482003	0.97\\
30.829319300999	0.96\\
30.845454909001	0.96\\
30.861384994004	0.96\\
30.877560190003	0.96\\
30.893329054001	0.96\\
30.909385654	0.95\\
30.925364346001	0.95\\
30.941460201	0.95\\
30.957394570999	0.95\\
30.973378262001	0.94\\
30.989337804001	0.94\\
31.006779435002	0.94\\
31.021991945999	0.94\\
31.037371771	0.95\\
31.053379698002	0.95\\
31.069375978001	0.95\\
31.085502146	0.95\\
31.101290659001	0.95\\
31.117460605	0.95\\
31.133372498001	0.96\\
31.149383822003	0.96\\
31.165814732003	0.96\\
31.181432197003	0.96\\
31.197367731003	0.96\\
31.21338584	0.97\\
31.229327935002	0.97\\
31.245371627999	0.97\\
31.261377948002	0.97\\
31.277613034001	0.97\\
31.293377966999	0.98\\
31.309822850998	0.98\\
31.32532257	0.98\\
31.341452855	0.98\\
31.357428598999	0.98\\
31.373489286003	0.99\\
31.389353084	0.99\\
31.405439467999	0.99\\
31.421261041001	0.99\\
31.437230820999	0.99\\
31.453423938	0.99\\
31.469311321999	1\\
31.485516730004	1\\
31.501325757	1\\
31.517401305004	1\\
31.533250501004	1\\
31.549405095001	1.01\\
31.565328375	1.01\\
31.581410604004	1.01\\
31.597378174	1.01\\
31.613373552998	1.01\\
31.629428215	1.02\\
31.645413324002	1.02\\
31.661359696003	1.02\\
31.677544458	1.02\\
31.693369881001	1.02\\
31.709648937001	1.03\\
31.725338321003	1.03\\
31.741371119	1.03\\
31.757297765999	1.03\\
31.773440126999	1.03\\
31.789436713002	1.04\\
31.805382985001	1.04\\
31.821407365002	1.04\\
31.837334717003	1.04\\
31.853306360001	1.04\\
31.869385639004	1.04\\
31.885324027001	1.05\\
31.901348421002	1.05\\
31.917380726002	1.05\\
31.933227223999	1.05\\
31.949347770001	1.05\\
31.965371212002	1.06\\
31.981418783001	1.06\\
31.999633724003	1.06\\
32.014820597	1.06\\
32.030133508	1.06\\
32.045471107998	1.05\\
32.061393453003	1.05\\
32.077589204003	1.05\\
32.093339873001	1.05\\
32.109550524002	1.05\\
32.125369271004	1.05\\
32.141400364998	1.04\\
32.157343760998	1.04\\
32.173379970001	1.04\\
32.189323723999	1.04\\
32.205379959	1.04\\
32.221369637001	1.04\\
32.237399824002	1.04\\
32.253225152001	1.03\\
32.269361905003	1.03\\
32.285458723	1.03\\
32.301353899002	1.03\\
32.317376639	1.03\\
32.333445619	1.03\\
32.349371805001	1.02\\
32.365388730999	1.02\\
32.381388915001	1.02\\
32.397264894001	1.02\\
32.413321935002	1.02\\
32.429422241001	1.02\\
32.445523068001	1.01\\
32.461338998001	1.01\\
32.477559883	1.01\\
32.493294588002	1.01\\
32.511759209999	1.01\\
32.527077317002	1.01\\
32.542179060002	1.01\\
32.557420219002	1\\
32.573387166001	1\\
32.589360735001	1\\
32.605370420998	1\\
32.621368438999	1\\
32.637274027001	1\\
32.653377632	0.99\\
32.669263429001	0.99\\
32.685394036003	0.99\\
32.701327714001	0.99\\
32.717423584999	0.99\\
32.733335319001	0.99\\
32.749412712002	0.98\\
32.765334228001	0.98\\
32.781301006001	0.98\\
32.797343415001	0.98\\
32.813375736	0.98\\
32.829395645001	0.98\\
32.845435675003	0.98\\
32.861401343998	0.97\\
32.877422687001	0.97\\
32.893338305001	0.97\\
32.909391187001	0.97\\
32.925342420002	0.97\\
32.941347071999	0.97\\
32.957377967999	0.96\\
32.973346409001	0.96\\
32.989331629002	0.96\\
33.005324226002	0.96\\
33.021396768002	0.96\\
33.037423753002	0.96\\
33.053411389	0.96\\
33.069324109001	0.97\\
33.085387772	0.97\\
33.101494292	0.97\\
33.117472876999	0.97\\
33.133423503002	0.97\\
33.149287149002	0.97\\
33.165397531002	0.97\\
33.181414247002	0.97\\
33.197323582001	0.98\\
33.213404050999	0.98\\
33.229417896	0.98\\
33.245365174	0.98\\
33.261405930001	0.98\\
33.277501460003	0.98\\
33.293374100003	0.98\\
33.309601127999	0.98\\
33.325355990002	0.99\\
33.341381021	0.99\\
33.357321523999	0.99\\
33.373412077999	0.99\\
33.389331448002	0.99\\
33.405325674	0.99\\
33.421858288998	0.99\\
33.437595691002	1\\
33.453403701	1\\
33.469324610001	1\\
33.485360008999	1\\
33.501278502003	1\\
33.517369517002	1\\
33.533331794998	1\\
33.549258117001	1\\
33.565356016999	1.01\\
33.581363677002	1.01\\
33.597368515999	1.01\\
33.613383954998	1.01\\
33.629383612999	1.01\\
33.645473555001	1.01\\
33.66137282	1.01\\
33.677509278	1.01\\
33.693366571999	1.02\\
33.711779475003	1.02\\
33.726922893002	1.02\\
33.741956065003	1.02\\
33.757362481999	1.02\\
33.773429483002	1.02\\
33.789416100003	1.02\\
33.805379501	1.02\\
33.821365477001	1.03\\
33.837374501	1.03\\
33.853371763001	1.03\\
33.869410269001	1.03\\
33.885782972	1.03\\
33.901284376999	1.03\\
33.917421865002	1.03\\
33.933326306999	1.03\\
33.949492807999	1.04\\
33.965438428002	1.04\\
33.981330932999	1.04\\
34.000012052998	1.04\\
34.015327883	1.04\\
34.030368848004	1.04\\
34.045610265	1.04\\
34.061300053002	1.04\\
34.077562647999	1.03\\
34.093373733002	1.03\\
34.111759563999	1.03\\
34.127018353001	1.03\\
34.142261626	1.03\\
34.157402088002	1.03\\
34.173254692002	1.03\\
34.189332551998	1.03\\
34.205358202004	1.03\\
34.221474068001	1.03\\
34.237363962998	1.02\\
34.253302806999	1.02\\
34.269264964001	1.02\\
34.285388668	1.02\\
34.301340442002	1.02\\
34.317279898003	1.02\\
34.333340919998	1.02\\
34.349386211003	1.02\\
34.365266501004	1.02\\
34.381423362	1.02\\
34.397380553002	1.02\\
34.413428125	1.01\\
34.429447306999	1.01\\
34.445384157002	1.01\\
34.461374827999	1.01\\
34.477473109001	1.01\\
34.493453538998	1.01\\
34.509431267002	1.01\\
34.525412612999	1.01\\
34.541388211003	1.01\\
34.557243830002	1.01\\
34.573373004002	1\\
34.589308521	1\\
34.605348469998	1\\
34.621407681999	1\\
34.637414957001	1\\
34.653383419003	1\\
34.669356637001	1\\
34.685410577999	1\\
34.701343363999	1\\
34.717369632	1\\
34.733354340004	1\\
34.749384923001	0.99\\
34.765410585003	0.99\\
34.781388378002	0.99\\
34.797299020001	0.99\\
34.813449296997	0.99\\
34.829382973	0.99\\
34.845476962002	0.99\\
34.861461881001	0.99\\
34.877572891003	0.99\\
34.893428036003	0.99\\
34.909545275002	0.99\\
34.925283791001	0.98\\
34.941371243	0.98\\
34.957389542	0.98\\
34.975623579003	0.98\\
34.990884357003	0.98\\
35.005411358002	0.98\\
35.021328153004	0.98\\
35.037251028	0.98\\
35.053378648003	0.98\\
35.069352091	0.98\\
35.085419854	0.98\\
35.101369413002	0.98\\
35.117390228001	0.98\\
35.133801307999	0.98\\
35.149289156002	0.99\\
35.165414508	0.99\\
35.181418666001	0.99\\
35.197351527001	0.99\\
35.213389547001	0.99\\
35.229442896	0.99\\
35.245452531998	0.99\\
35.261397850003	0.99\\
35.277495204003	0.99\\
35.293372634999	0.99\\
35.309589746002	0.99\\
35.325320438999	0.99\\
35.341384578003	0.99\\
35.357386470001	0.99\\
35.373403126999	0.99\\
35.389378244	0.99\\
35.405368830002	1\\
35.421260830002	1\\
35.437344409001	1\\
35.453388355	1\\
35.469333043003	1\\
35.485378906002	1\\
35.501329501999	1\\
35.517372488003	1\\
35.533407756001	1\\
35.549289759003	1\\
35.565391396	1\\
35.581390895001	1\\
35.597237001999	1\\
35.613422524998	1\\
35.629335557999	1\\
35.645411996002	1\\
35.661366161999	1.01\\
35.677452047001	1.01\\
35.693384908001	1.01\\
35.709443692997	1.01\\
35.725349515	1.01\\
35.741404919003	1.01\\
35.757369326	1.01\\
35.773414041001	1.01\\
35.789361620999	1.01\\
35.805409583	1.01\\
35.821380634999	1.01\\
35.837377246002	1.01\\
35.853413803002	1.01\\
35.869344464001	1.01\\
35.885228665001	1.01\\
35.903446684998	1.01\\
35.918584765	1.02\\
35.933968032002	1.02\\
35.949415967999	1.02\\
35.965445889	1.02\\
35.981386668003	1.02\\
35.999608106003	1.02\\
36.014842304001	1.02\\
36.030123964001	1.02\\
36.045478330002	1.02\\
36.061400009999	1.02\\
36.077610305001	1.02\\
36.093367091999	1.02\\
36.109662701	1.02\\
36.125390001004	1.02\\
36.141361434998	1.02\\
36.157379188999	1.02\\
36.173425225998	1.02\\
36.189393044998	1.01\\
36.205381072998	1.01\\
36.221465175	1.01\\
36.237343525002	1.01\\
36.253387408001	1.01\\
36.269373334	1.01\\
36.285339125	1.01\\
36.301465028	1.01\\
36.317412822998	1.01\\
36.333363058998	1.01\\
36.349637379002	1.01\\
36.365356225003	1.01\\
36.381442043	1.01\\
36.397389854	1.01\\
36.41333314	1.01\\
36.429414617001	1.01\\
36.445457041001	1.01\\
36.461479285	1.01\\
36.477691900002	1.01\\
36.493359285	1.01\\
36.511775755001	1.01\\
36.526943457001	1.01\\
36.542058534001	1.01\\
36.557728770001	1.01\\
36.573389050003	1.01\\
36.589387524002	1.01\\
36.605396413002	1.01\\
36.621444060002	1.01\\
36.637339049	1.01\\
36.653373395001	1.01\\
36.669199921997	1.01\\
36.685397211003	1.01\\
36.701354598	1.01\\
36.717325795998	1\\
36.733349413002	1\\
36.749379445	1\\
36.765295178002	1\\
36.781452667	1\\
36.797345292004	1\\
36.813495977001	1\\
36.829422899002	1\\
36.845370869999	1\\
36.861347120003	1\\
36.877517459	1\\
36.893385225003	1\\
36.90941309	1\\
36.925306968003	1\\
36.941373271004	1\\
36.957287154	1\\
36.973381766999	1\\
36.989390425	1\\
37.005357770001	1\\
37.021420801003	1\\
37.037407724003	1\\
37.053325100003	1\\
37.069328712002	1\\
37.085387112999	1\\
37.101338942002	1\\
37.117379698002	1\\
37.133318262001	1\\
37.149284965	1\\
37.165395434002	1\\
37.18134459	1\\
37.197391385998	1\\
37.213419970001	1\\
37.229356966	1\\
37.24535932	1\\
37.26137339	1\\
37.277448101998	1\\
37.293364205002	1.01\\
37.309507283001	1.01\\
37.325319857998	1.01\\
37.341303075001	1.01\\
37.357410563999	1.01\\
37.373606181004	1.01\\
37.389372549004	1.01\\
37.405376131001	1.01\\
37.421482855999	1.01\\
37.437379431004	1.01\\
37.453514374001	1.01\\
37.469371965	1.01\\
37.485429327999	1.01\\
37.501414805001	1.01\\
37.517348679001	1.01\\
37.533360107003	1.01\\
37.549368258999	1.01\\
37.565362320004	1.01\\
37.581320529	1.01\\
37.597347131001	1.01\\
37.613341028004	1.01\\
37.629320932999	1.01\\
37.645467369	1.01\\
37.661417083	1.01\\
37.677467737004	1.01\\
37.693298879002	1.01\\
37.709525869999	1.01\\
37.725406624001	1.01\\
37.741372554001	1.01\\
37.757330508	1.01\\
37.773355047001	1.01\\
37.789346039002	1.01\\
37.805367121002	1.01\\
37.821449300003	1.02\\
37.837466758	1.02\\
37.853378938004	1.02\\
37.869351292	1.02\\
37.885422333004	1.02\\
37.901459360001	1.02\\
37.917388914002	1.02\\
37.933390004002	1.02\\
37.949405522999	1.02\\
37.965342989998	1.02\\
37.981388190003	1.02\\
37.999709284001	1.02\\
38.015503702004	1.02\\
38.031214710003	1.02\\
38.0465055	1.02\\
38.061774824002	1.02\\
38.077530759003	1.02\\
38.093329937001	1.02\\
38.111762429001	1.02\\
38.126961685002	1.02\\
38.142168679001	1.02\\
38.157430649002	1.02\\
38.173479934002	1.02\\
38.189387869999	1.01\\
38.205380931004	1.01\\
38.221434987	1.01\\
38.237325493	1.01\\
38.253276797001	1.01\\
38.269364583	1.01\\
38.285396965	1.01\\
38.301342801003	1.01\\
38.317429580998	1.01\\
38.333370139	1.01\\
38.349408934998	1.01\\
38.365401963002	1.01\\
38.381401445	1.01\\
38.397713832001	1.01\\
38.413541055001	1.01\\
38.429370671002	1.01\\
38.445498202999	1.01\\
38.461262224999	1.01\\
38.477443801998	1.01\\
38.493355520001	1.01\\
38.511730301003	1.01\\
38.526906041001	1.01\\
38.542243568001	1.01\\
38.557589898999	1.01\\
38.573443982003	1.01\\
38.589315066002	1.01\\
38.605445478001	1.01\\
38.621441563	1.01\\
38.637388997002	1.01\\
38.653418848	1.01\\
38.669405195	1.01\\
38.685483631001	1.01\\
38.701352191002	1\\
38.717464595001	1\\
38.733361765	1\\
38.749390063	1\\
38.765371683998	1\\
38.781394069001	1\\
38.797331089001	1\\
38.813430156998	1\\
38.829398133	1\\
38.845445664002	1\\
38.861415278	1\\
38.877455258	1\\
38.893215798001	1\\
38.909393094002	1\\
38.925419036	1\\
38.941343999001	1\\
38.957393183998	1\\
38.973390315003	1\\
38.989348133	1\\
39.007068742001	1\\
39.022441057003	1\\
39.037572164002	1\\
39.053381867001	1\\
39.069378033001	1\\
39.085416609001	1\\
39.10132966	1\\
39.117198438004	1\\
39.133291794998	1\\
39.149421659001	1\\
39.165365098	1\\
39.181405103001	1\\
39.197267412003	1\\
39.213391970001	1\\
39.229260282997	1\\
39.245381924	1\\
39.26134975	1\\
39.277442208	1\\
39.293326789002	1\\
39.309602343003	1.01\\
39.325351743	1.01\\
39.341374337002	1.01\\
39.357293082001	1.01\\
39.373358694001	1.01\\
39.389417896	1.01\\
39.405398977001	1.01\\
39.421386562001	1.01\\
39.437345622998	1.01\\
39.453475236	1.01\\
39.469373693001	1.01\\
39.485408736	1.01\\
39.501335569001	1.01\\
39.517473018002	1.01\\
39.533399710999	1.01\\
39.549469517002	1.01\\
39.565354339001	1.01\\
39.581399064003	1.01\\
39.597385312001	1.01\\
39.613413414002	1.01\\
39.629376369004	1.01\\
39.645455331002	1.01\\
39.661349376004	1.01\\
39.677441286	1.01\\
39.693342926998	1.01\\
39.709566967999	1.01\\
39.725298084999	1.01\\
39.741358586003	1.01\\
39.757404681	1.01\\
39.773383515	1.01\\
39.789399544998	1.01\\
39.805221582001	1.01\\
39.821254505001	1.01\\
39.837340621002	1.02\\
39.853378250004	1.02\\
39.869334710999	1.02\\
39.885272101002	1.02\\
39.901388052002	1.02\\
39.917383098999	1.02\\
39.933424161003	1.02\\
39.949379924	1.02\\
39.965328201	1.02\\
39.981310958	1.02\\
39.999595615998	1.02\\
40.014762530999	1.02\\
40.030131254002	1.02\\
40.045445804001	1.02\\
40.061408356999	1.02\\
40.077327093003	1.02\\
40.093398	1.02\\
40.111754730004	1.02\\
40.127093207001	1.02\\
40.142488918999	1.02\\
40.157662055001	1.02\\
40.173299203003	1.01\\
40.189461801003	1.01\\
40.205305029003	1.01\\
40.22128532	1.01\\
40.237335825001	1.01\\
40.253403031002	1.01\\
40.269342101002	1.01\\
40.285430600003	1.01\\
40.301342684002	1.01\\
40.317407124001	1.01\\
40.333298982003	1.01\\
40.349359167	1.01\\
40.365295746002	1.01\\
40.381349030003	1.01\\
40.397308773003	1.01\\
40.413417152001	1.01\\
40.429403915001	1.01\\
40.445393193001	1.01\\
40.461377772	1.01\\
40.477600982003	1.01\\
40.493321166001	1.01\\
40.509586706002	1.01\\
40.525306995003	1.01\\
40.541335002999	1.01\\
40.557374318001	1.01\\
40.573513469002	1.01\\
40.589392889	1.01\\
40.605420470001	1.01\\
40.621397647	1.01\\
40.637323793	1.01\\
40.653236872002	1.01\\
40.669310363999	1.01\\
40.685245698002	1.01\\
40.701323055001	1\\
40.717388747002	1\\
40.733349671997	1\\
40.749423912998	1\\
40.765348165001	1\\
40.781392930001	1\\
40.797330222004	1\\
40.813350426003	1\\
40.829438724999	1\\
40.845462061001	1\\
40.861468930001	1\\
40.877720056	1\\
40.893349459999	1\\
40.909491835999	1\\
40.927061397	1\\
40.942203640999	1\\
40.957380479004	1\\
40.973392845001	1\\
40.989327940998	1\\
41.006991200001	1\\
41.022635993	1\\
41.037799105	1\\
41.053460279	1\\
41.069370220001	1\\
41.085383222004	1\\
41.101472994	1\\
41.117299247998	1\\
41.133403240002	1\\
41.14939166	1\\
41.165404868	1\\
41.181382144001	1\\
41.197398485001	1\\
41.213366455998	1\\
41.229331783001	1\\
41.245490739998	1\\
41.261424306999	1\\
41.277531018002	1\\
41.293374136002	1\\
41.309670997002	1.01\\
41.325356628002	1.01\\
41.341469050999	1.01\\
41.357319426998	1.01\\
41.373321193001	1.01\\
41.389430439003	1.01\\
41.405394972	1.01\\
41.421423246002	1.01\\
41.437421123001	1.01\\
41.453311732998	1.01\\
41.469360537003	1.01\\
41.485422851002	1.01\\
41.501369464001	1.01\\
41.517367270001	1.01\\
41.533365868	1.01\\
41.549514258	1.01\\
41.56534857	1.01\\
41.581386906002	1.01\\
41.597380369	1.01\\
41.613350184002	1.01\\
41.629394488003	1.01\\
41.645524777001	1.01\\
41.661391979	1.01\\
41.677565696003	1.01\\
41.693375043	1.01\\
41.709559354	1.01\\
41.725395445999	1.01\\
41.741387898003	1.01\\
41.757330280003	1.01\\
41.773400678002	1.01\\
41.789688228001	1.01\\
41.805406603001	1.01\\
41.821400551998	1.01\\
41.837403312001	1.02\\
41.853377245999	1.02\\
41.869349725998	1.02\\
41.885382209999	1.02\\
41.901306553002	1.02\\
41.917418082001	1.02\\
41.933403675	1.02\\
41.949321461003	1.02\\
41.965344130001	1.02\\
41.981429998001	1.02\\
41.999649693001	1.02\\
42.014924512001	1.02\\
42.030512566998	1.02\\
42.045792494	1.02\\
42.061338414997	1.02\\
42.077597342999	1.02\\
42.09337182	1.02\\
42.111835097	1.02\\
42.125386861	1.02\\
42.141404494999	1.02\\
42.157424688	1.02\\
42.173407834004	1.01\\
42.189331729	1.01\\
42.205328654999	1.01\\
42.221287471001	1.01\\
42.237302485001	1.01\\
42.253390205002	1.01\\
42.269334858002	1.01\\
42.285374361	1.01\\
42.301331458	1.01\\
42.317413612	1.01\\
42.333359583	1.01\\
42.349393629002	1.01\\
42.365372244	1.01\\
42.381370373001	1.01\\
42.397325440003	1.01\\
42.413389834	1.01\\
42.429399202	1.01\\
42.445478396	1.01\\
42.461286696999	1.01\\
42.477432123001	1.01\\
42.493241648003	1.01\\
42.511655444001	1.01\\
42.526797658001	1.01\\
42.541879620999	1.01\\
42.557393286999	1.01\\
42.573294346001	1.01\\
42.589389800999	1.01\\
42.605295720001	1.01\\
42.621382661	1.01\\
42.637385707001	1.01\\
42.653378887001	1.01\\
42.669341729	1.01\\
42.685374210999	1.01\\
42.701328145001	1\\
42.717406070004	1\\
42.733502186001	1\\
42.749419402001	1\\
42.765335661	1\\
42.781421057999	1\\
42.797465486	1\\
42.813362622002	1\\
42.829271444001	1\\
42.845467104004	1\\
42.861336864003	1\\
42.877399073002	1\\
42.893329324002	1\\
42.909317550999	1\\
42.925404843003	1\\
42.941330942997	1\\
42.957343868004	1\\
42.973400251004	1\\
42.989312375	1\\
43.007213558003	1\\
43.021937213002	1\\
43.037331709999	1\\
43.053392666001	1\\
43.069372573998	1\\
43.085295903004	1\\
43.101328009003	1\\
43.117416223004	1\\
43.133363031002	1\\
43.149335452	1\\
43.165508230999	1\\
43.181282036	1\\
43.197377897999	1\\
43.213418493	1\\
43.229367722	1\\
43.245396996002	1\\
43.261380111	1\\
43.277448029	1\\
43.293324071003	1\\
43.311795536	1.01\\
43.326950685002	1.01\\
43.342102896	1.01\\
43.357398153	1.01\\
43.373438967999	1.01\\
43.389236636002	1.01\\
43.405377676003	1.01\\
43.421429827999	1.01\\
43.437306613999	1.01\\
43.453394720001	1.01\\
43.469340371002	1.01\\
43.485432267002	1.01\\
43.501382984001	1.01\\
43.517392373001	1.01\\
43.533226865002	1.01\\
43.549323200001	1.01\\
43.565258216	1.01\\
43.581416607003	1.01\\
43.597387124001	1.01\\
43.613334418999	1.01\\
43.629321303002	1.01\\
43.645455433003	1.01\\
43.661371478001	1.01\\
43.677482391003	1.01\\
43.693402348999	1.01\\
43.709423084004	1.01\\
43.725256314999	1.01\\
43.741377802998	1.01\\
43.757424993	1.01\\
43.773412529	1.01\\
43.789463371998	1.01\\
43.805355596001	1.01\\
43.821354200001	1.01\\
43.837356034001	1.02\\
43.853400147999	1.02\\
43.869378131001	1.02\\
43.885395994999	1.02\\
43.901332574997	1.02\\
43.917283003002	1.02\\
43.933338826	1.02\\
43.949492567002	1.02\\
43.965274514	1.02\\
43.981402363999	1.02\\
43.999588989003	1.02\\
44.015002675003	1.02\\
44.030277774002	1.02\\
44.045446257004	1.02\\
44.061325655999	1.02\\
44.077417783001	1.02\\
44.093332367001	1.02\\
44.109345581002	1.02\\
44.125434614003	1.02\\
44.141395278	1.02\\
44.157349221001	1.02\\
44.173385743	1.01\\
44.189645343998	1.01\\
44.205475716	1.01\\
44.221417099003	1.01\\
44.23735873	1.01\\
44.253303001999	1.01\\
44.269350368	1.01\\
44.285453849999	1.01\\
44.301367177998	1.01\\
44.317425924	1.01\\
44.333321907002	1.01\\
44.349331912998	1.01\\
44.365373065003	1.01\\
44.381407185002	1.01\\
44.397294688	1.01\\
44.413375791001	1.01\\
44.429358168999	1.01\\
44.445429350998	1.01\\
44.461353586003	1.01\\
44.477480316002	1.01\\
44.493391528004	1.01\\
44.509669556999	1.01\\
44.525287641003	1.01\\
44.541340255001	1.01\\
44.557350985001	1.01\\
44.573419930001	1.01\\
44.589235730999	1.01\\
44.605404612	1.01\\
44.621383853001	1.01\\
44.637348577004	1.01\\
44.653412815998	1.01\\
44.669408700001	1.01\\
44.685277495999	1\\
44.701357471001	1\\
44.717383698002	1\\
44.733383727001	1\\
44.749441004998	1\\
44.765371065003	1\\
44.781550536003	1\\
44.797348851002	1\\
44.813295344002	1\\
44.829412096001	1\\
44.845422233002	1\\
44.861322156002	1\\
44.877498140999	1\\
44.893428627003	1\\
44.911804800999	1\\
44.927060359001	1\\
44.942221520001	1\\
44.957526816002	1\\
44.973389526001	1\\
44.989389278	1\\
45.006990134999	1\\
45.022879431	1\\
45.038034327	1\\
45.053322657002	1\\
45.069374036	1\\
45.085423346001	1\\
45.101338461998	1\\
45.117389837002	1\\
45.133334870999	1\\
45.149532477001	1\\
45.165461139	1\\
45.181413138001	1\\
45.197302776001	1\\
45.213319492001	1\\
45.229369430001	1\\
45.245388126	1\\
45.261289851998	1\\
45.277389106999	1\\
45.293415126004	1\\
45.309373686001	1\\
45.325453263001	1.01\\
45.341254233002	1.01\\
45.357269062001	1.01\\
45.373286100998	1.01\\
45.389324872998	1.01\\
45.406614429001	1.01\\
45.422472896	1.01\\
45.437721062001	1.01\\
45.453421737	1.01\\
45.469339237	1.01\\
45.485424573998	1.01\\
45.501362636997	1.01\\
45.517413746002	1.01\\
45.533372314999	1.01\\
45.549306543999	1.01\\
45.565332101002	1.01\\
45.581369810002	1.01\\
45.597303941998	1.01\\
45.613400662003	1.01\\
45.629363756001	1.01\\
45.645434425003	1.01\\
45.661387577004	1.01\\
45.677544992001	1.01\\
45.693479903	1.01\\
45.709617033001	1.01\\
45.725431264	1.01\\
45.741356549	1.01\\
45.757353168999	1.01\\
45.773465964001	1.01\\
45.789362998001	1.01\\
45.805334878002	1.01\\
45.821333481999	1.01\\
45.837343650002	1.01\\
45.853384711003	1.02\\
45.869367620999	1.02\\
45.885379661999	1.02\\
45.901344290001	1.02\\
45.917399584	1.02\\
45.933359245999	1.02\\
45.94941932	1.02\\
45.965329142002	1.02\\
45.981474708004	1.02\\
45.999562747998	1.02\\
46.014688947003	1.02\\
46.030286508004	1.02\\
46.045570849003	1.02\\
46.061517009999	1.02\\
46.077518397	1.02\\
46.093265914997	1.02\\
46.109416352001	1.02\\
46.125314539002	1.02\\
46.141387173001	1.02\\
46.157329336003	1.01\\
46.173441805001	1.01\\
46.189360222	1.01\\
46.205293117001	1.01\\
46.221419062001	1.01\\
46.237334738999	1.01\\
46.253336596001	1.01\\
46.269322549	1.01\\
46.285362588002	1.01\\
46.301342480004	1.01\\
46.317385490998	1.01\\
46.333375282002	1.01\\
46.349416464001	1.01\\
46.365372619	1.01\\
46.381363841	1.01\\
46.397358858002	1.01\\
46.413397062001	1.01\\
46.429385345001	1.01\\
46.445430496002	1.01\\
46.461418426998	1.01\\
46.477566465	1.01\\
46.493340897999	1.01\\
46.511854303002	1.01\\
46.527205937001	1.01\\
46.542274674	1.01\\
46.557503119	1.01\\
46.573369885998	1.01\\
46.589417830002	1.01\\
46.605370182999	1.01\\
46.621439753998	1.01\\
46.637386908001	1.01\\
46.653367456997	1.01\\
46.669339051003	1.01\\
46.685463576	1\\
46.701332663002	1\\
46.717499374001	1\\
46.733299627999	1\\
46.749422091004	1\\
46.765328212002	1\\
46.781439327004	1\\
46.797354763001	1\\
46.813414781998	1\\
46.829507588002	1\\
46.845449624001	1\\
46.861373615002	1\\
46.877465852001	1\\
46.893395556	1\\
46.909509176998	1\\
46.925323492001	1\\
46.94364505	1\\
46.958825761002	1\\
46.973903535	1\\
46.989337466999	1\\
47.006939203999	1\\
47.022104536003	1\\
47.037307627003	1\\
47.053390735001	1\\
47.069250237	1\\
47.085308497002	1\\
47.101371231003	1\\
47.117249508	1\\
47.133383789997	1\\
47.149423444001	1\\
47.165403432003	1\\
47.181392076	1\\
47.197373180001	1\\
47.213384685002	1\\
47.229403997002	1\\
47.245488126004	1\\
47.261359242001	1\\
47.277330228001	1\\
47.293394268002	1\\
47.311822751	1\\
47.327048605999	1.01\\
47.342266564999	1.01\\
47.357533984001	1.01\\
47.373439999001	1.01\\
47.389347943001	1.01\\
47.405395329998	1.01\\
47.421417487999	1.01\\
47.437340815998	1.01\\
47.45336025	1.01\\
47.469306925999	1.01\\
47.485421566002	1.01\\
47.501365827999	1.01\\
47.517399979	1.01\\
47.533357956002	1.01\\
47.549450951	1.01\\
47.565358229	1.01\\
47.581391353001	1.01\\
47.597387723	1.01\\
47.613418737	1.01\\
47.629360709	1.01\\
47.645498174	1.01\\
47.661384684002	1.01\\
47.677427744999	1.01\\
47.693363883999	1.01\\
47.709345939003	1.01\\
47.725374209999	1.01\\
47.741466713002	1.01\\
47.757326062001	1.01\\
47.773398068001	1.01\\
47.789340082001	1.01\\
47.805340067002	1.01\\
47.821304175	1.01\\
47.837412484001	1.01\\
47.853398366001	1.02\\
47.869347875	1.02\\
47.885392164002	1.02\\
47.901324800003	1.02\\
47.917444429001	1.02\\
47.933345877999	1.02\\
47.949505640003	1.02\\
47.965483879002	1.02\\
47.981372039002	1.02\\
47.999625078999	1.02\\
48.014910201	1.02\\
48.030238944001	1.02\\
48.045532691002	1.02\\
48.061384705002	1.02\\
48.077484083004	1.02\\
48.093400437001	1.02\\
48.109586562001	1.02\\
48.125377777001	1.02\\
48.141302650002	1.01\\
48.157397268002	1.01\\
48.173443229	1.01\\
48.189419918	1.01\\
48.205363702	1.01\\
48.221392520001	1.01\\
48.237344133999	1.01\\
48.253469437001	1.01\\
48.269402489003	1.01\\
48.285331062001	1.01\\
48.301386203999	1.01\\
48.317418470001	1.01\\
48.333364988003	1.01\\
48.349407306999	1.01\\
48.365392471001	1.01\\
48.381479857998	1.01\\
48.397354743	1.01\\
48.413409119	1.01\\
48.429411412998	1.01\\
48.445561877003	1.01\\
48.461411552002	1.01\\
48.477422628998	1.01\\
48.4933445	1.01\\
48.511755337002	1.01\\
48.526936214001	1.01\\
48.542236933003	1.01\\
48.557541184002	1.01\\
48.575530121002	1.01\\
48.590936178002	1.01\\
48.605917753002	1.01\\
48.621427835999	1.01\\
48.637340672001	1.01\\
48.653349746002	1.01\\
48.669308415001	1\\
48.685464613003	1\\
48.701383292	1\\
48.717402017998	1\\
48.733376368	1\\
48.749460947003	1\\
48.765385098999	1\\
48.781389954003	1\\
48.797375650002	1\\
48.813296828003	1\\
48.829369798001	1\\
48.845332294003	1\\
48.861208877999	1\\
48.877434555001	1\\
48.893386477001	1\\
48.911746728001	1\\
48.926976556	1\\
48.942228453003	1\\
48.957528921002	1\\
48.973333938	1\\
48.989341988999	1\\
49.005299707001	1\\
49.02136125	1\\
49.037326670002	1\\
49.053412538002	1\\
49.069388076	1\\
49.085576291001	1\\
49.101397899002	1\\
49.117387404999	1\\
49.133328276001	1\\
49.149377494	1\\
49.16540975	1\\
49.181400803002	1\\
49.197379848999	1\\
49.213470172001	1\\
49.229406867001	1\\
49.245452303002	1\\
49.261263112999	1\\
49.277506679001	1\\
49.293362916001	1\\
49.311777547001	1\\
49.327063341999	1.01\\
49.342245762001	1.01\\
49.357579125	1.01\\
49.373472606999	1.01\\
49.389288509999	1.01\\
49.405368928002	1.01\\
49.423941738003	1.01\\
49.439219109001	1.01\\
49.454571751	1.01\\
49.470000680001	1.01\\
49.485396363003	1.01\\
49.501335498001	1.01\\
49.517375735001	1.01\\
49.533263728001	1.01\\
49.549462942002	1.01\\
49.565398178002	1.01\\
49.581443775002	1.01\\
49.597269794003	1.01\\
49.613390851998	1.01\\
49.629365324002	1.01\\
49.645475636002	1.01\\
49.661355690003	1.01\\
49.677444106003	1.01\\
49.693408369	1.01\\
49.709500202999	1.01\\
49.725274886002	1.01\\
49.743649318001	1.01\\
49.759067157002	1.01\\
49.774040513001	1.01\\
49.789450788002	1.01\\
49.805402925999	1.01\\
49.821271766999	1.01\\
49.837322224003	1.01\\
49.853455074002	1.02\\
49.869378628998	1.02\\
49.885428779003	1.02\\
49.90133357	1.02\\
49.917424679001	1.02\\
49.933337725003	1.02\\
49.949412933003	1.02\\
49.965366056999	1.02\\
49.981428095001	1.02\\
49.999833555001	1.02\\
50.015117530003	1.02\\
50.030451148999	1.02\\
50.045575498001	1.02\\
50.061514627999	1.02\\
50.07757139	1.02\\
50.093387661999	1.02\\
50.109551542	1.02\\
50.125350438999	1.02\\
50.141370418003	1.01\\
50.157310869	1.01\\
50.173376388001	1.01\\
50.189430244004	1.01\\
50.205386162003	1.01\\
50.221486012001	1.01\\
50.237332322003	1.01\\
50.253333768002	1.01\\
50.269340447003	1.01\\
50.285487051998	1.01\\
50.301286104	1.01\\
50.317569061001	1.01\\
50.333337709999	1.01\\
50.349324531998	1.01\\
50.365326334999	1.01\\
50.381399555001	1.01\\
50.397355393002	1.01\\
50.413387904999	1.01\\
50.429382351002	1.01\\
50.445431525002	1.01\\
50.461373242001	1.01\\
50.477604405003	1.01\\
50.493522178002	1.01\\
50.509476757	1.01\\
50.525339962998	1.01\\
50.541460567002	1.01\\
50.557540742001	1.01\\
50.573361598999	1.01\\
50.589436758	1.01\\
50.605640824002	1.01\\
50.621387739003	1.01\\
50.637350234001	1.01\\
50.653413503998	1.01\\
50.669427765	1\\
50.685417510002	1\\
50.701409204003	1\\
50.717524072998	1\\
50.733374161999	1\\
50.749387465	1\\
50.765334488999	1\\
50.781326345001	1\\
50.797395444001	1\\
50.813394140003	1\\
50.829411351998	1\\
50.845416642002	1\\
50.861369523999	1\\
50.877589776001	1\\
50.893381187001	1\\
50.909611813	1\\
50.926650014	1\\
50.941845745003	1\\
50.957390932999	1\\
50.973408105	1\\
50.989358716	1\\
51.005281842003	1\\
51.02140605	1\\
51.037440380001	1\\
51.053413614003	1\\
51.069348159001	1\\
51.085410749001	1\\
51.101371569001	1\\
51.117375536	1\\
51.133323361	1\\
51.149502111	1\\
51.165495804001	1\\
51.181391602001	1\\
51.197397537998	1\\
51.213365499001	1\\
51.229363171002	1\\
51.245400946003	1\\
51.261360649002	1\\
51.277475803002	1\\
51.293340364003	1\\
51.309584550999	1\\
51.325319979	1\\
51.341435395001	1.01\\
51.357333645001	1.01\\
51.373431258999	1.01\\
51.389339702999	1.01\\
51.405413469002	1.01\\
51.421535001999	1.01\\
51.437397451004	1.01\\
51.453248998001	1.01\\
51.469357367001	1.01\\
51.485366835003	1.01\\
51.501370787998	1.01\\
51.517394253002	1.01\\
51.533347556	1.01\\
51.549373138001	1.01\\
51.565371612	1.01\\
51.581370984001	1.01\\
51.597398881001	1.01\\
51.613415237	1.01\\
51.629423743	1.01\\
51.645537650997	1.01\\
51.661402819001	1.01\\
51.677460192002	1.01\\
51.693381362004	1.01\\
51.709480906002	1.01\\
51.725408810002	1.01\\
51.741595889004	1.01\\
51.757300422001	1.01\\
51.773461746002	1.01\\
51.78986809	1.01\\
51.805417918	1.01\\
51.821316117001	1.01\\
51.83725141	1.01\\
51.853379802002	1.01\\
51.869394097	1.02\\
51.885423579998	1.02\\
51.901403300999	1.02\\
51.917408285	1.02\\
51.933401854	1.02\\
51.949385525002	1.02\\
51.965334668999	1.02\\
51.981454503998	1.02\\
51.999657890999	1.02\\
52.014850834999	1.02\\
52.030114494	1.02\\
52.045371168	1.02\\
52.061435056999	1.02\\
52.077492430001	1.02\\
52.093242997998	1.02\\
52.111573989003	1.02\\
52.126717335999	1.02\\
52.141801569001	1.01\\
52.157372656998	1.01\\
52.173412339001	1.01\\
52.18945325	1.01\\
52.205460457001	1.01\\
52.221359377999	1.01\\
52.237416906998	1.01\\
52.253510925	1.01\\
52.269400938999	1.01\\
52.285449515	1.01\\
52.301341284001	1.01\\
52.317446243	1.01\\
52.333297626	1.01\\
52.349434553997	1.01\\
52.365354471001	1.01\\
52.381376084999	1.01\\
52.397391323002	1.01\\
52.413378909001	1.01\\
52.429338180001	1.01\\
52.445362056999	1.01\\
52.461347077004	1.01\\
52.477496586003	1.01\\
52.493361387001	1.01\\
52.509325512001	1.01\\
52.525391862	1.01\\
52.541363867001	1.01\\
52.557287027001	1.01\\
52.573406952	1.01\\
52.589299275002	1.01\\
52.605428291001	1.01\\
52.621485419003	1.01\\
52.637385550999	1.01\\
52.653371418	1.01\\
52.669369006001	1\\
52.685384476002	1\\
52.701355352001	1\\
52.717484548001	1\\
52.733333975003	1\\
52.749263391003	1\\
52.765300735001	1\\
52.781427948002	1\\
52.797394093998	1\\
52.813416895001	1\\
52.829460100998	1\\
52.845532053002	1\\
52.861331709999	1\\
52.877431246002	1\\
52.893369629002	1\\
52.910780099999	1\\
52.926090890999	1\\
52.941482730999	1\\
52.957363033001	1\\
52.973384876	1\\
52.989358339001	1\\
53.005550016999	1\\
53.021395472	1\\
53.037417498001	1\\
53.053427915001	1\\
53.069290293	1\\
53.085387827	1\\
53.101347288002	1\\
53.117387246998	1\\
53.133286026001	1\\
53.149395457001	1\\
53.165410465	1\\
53.181398917	1\\
53.197334249001	1\\
53.213422734001	1\\
53.229378079998	1\\
53.245502570999	1\\
53.261335716	1\\
53.277474248001	1\\
53.293369229	1\\
53.309539097	1\\
53.325307913002	1\\
53.341444113003	1.01\\
53.357303257	1.01\\
53.373409225998	1.01\\
53.389366102001	1.01\\
53.405440027001	1.01\\
53.421251312001	1.01\\
53.437305087998	1.01\\
53.453328649002	1.01\\
53.469357514	1.01\\
53.485394201	1.01\\
53.501378783001	1.01\\
53.51738434	1.01\\
53.533364768997	1.01\\
53.549424167	1.01\\
53.565373295002	1.01\\
53.581441298001	1.01\\
53.597346450001	1.01\\
53.613467078003	1.01\\
53.629349474999	1.01\\
53.645508904	1.01\\
53.661369287003	1.01\\
53.677368101998	1.01\\
53.693326633999	1.01\\
53.711794452999	1.01\\
53.726983729	1.01\\
53.742195394001	1.01\\
53.757483009003	1.01\\
53.773418843003	1.01\\
53.789367619999	1.01\\
53.805373777001	1.01\\
53.821371375	1.01\\
53.837420535	1.01\\
53.853363258	1.01\\
53.869346143002	1.02\\
53.885354216	1.02\\
53.901330668	1.02\\
53.917416382	1.02\\
53.933328501999	1.02\\
53.949417728001	1.02\\
53.965370869	1.02\\
53.981447361	1.02\\
53.999949185002	1.02\\
54.015259199002	1.02\\
54.030398571999	1.02\\
54.045653837998	1.02\\
54.061313716999	1.02\\
54.077403352001	1.02\\
54.093317752999	1.02\\
54.111663966999	1.02\\
54.126616897	1.02\\
54.141721952	1.01\\
54.157448852001	1.01\\
54.173572861	1.01\\
54.189393769001	1.01\\
54.205328469002	1.01\\
54.221280985001	1.01\\
54.237411393997	1.01\\
54.253433242001	1.01\\
54.269328674	1.01\\
54.285277970001	1.01\\
54.301365824997	1.01\\
54.317304841	1.01\\
54.333356008999	1.01\\
54.349416960999	1.01\\
54.365337859001	1.01\\
54.381441777001	1.01\\
54.397303778004	1.01\\
54.413410249001	1.01\\
54.429404237999	1.01\\
54.445404607998	1.01\\
54.461367453003	1.01\\
54.477587828999	1.01\\
54.493338394001	1.01\\
54.509573925	1.01\\
54.52536748	1.01\\
54.54145416	1.01\\
54.557413146004	1.01\\
54.573290911003	1.01\\
54.589421315998	1.01\\
54.605370279003	1.01\\
54.621456292	1.01\\
54.637413138001	1.01\\
54.653390579998	1.01\\
54.669708950001	1\\
54.685382668999	1\\
54.701397566002	1\\
54.71740682	1\\
54.733235217999	1\\
54.749423745003	1\\
54.765457140003	1\\
54.781387229	1\\
54.797359029	1\\
54.813405141999	1\\
54.829462677002	1\\
54.845333546002	1\\
54.861303272999	1\\
54.879550278	1\\
54.894653944001	1\\
54.909907104	1\\
54.925506165001	1\\
54.941380544003	1\\
54.957369123001	1\\
54.973349445	1\\
54.989313436001	1\\
55.007365945999	1\\
55.022706671002	1\\
55.038086063	1\\
55.05338391	1\\
55.069375494	1\\
55.08541416	1\\
55.101367779	1\\
55.117433957001	1\\
55.13343589	1\\
55.14936857	1\\
55.16541107	1\\
55.181424459	1\\
55.197407020001	1\\
55.213459360001	1\\
55.229391991001	1\\
55.245421656002	1\\
55.261392704003	1\\
55.277454112999	1\\
55.293342092003	1\\
55.311748313	1\\
55.327062707001	1\\
55.342322348	1.01\\
55.357913675003	1.01\\
55.373398186001	1.01\\
55.389362057003	1.01\\
55.405326012001	1.01\\
55.421354279999	1.01\\
55.437374399002	1.01\\
55.453411071003	1.01\\
55.469381242001	1.01\\
55.485388377003	1.01\\
55.501338677002	1.01\\
55.517421194001	1.01\\
55.533333327	1.01\\
55.549337683998	1.01\\
55.565317251999	1.01\\
55.581370966999	1.01\\
55.597355700001	1.01\\
55.613414868	1.01\\
55.629395293	1.01\\
55.645415602001	1.01\\
55.661304623001	1.01\\
55.679626309002	1.01\\
55.694622538002	1.01\\
55.709809208	1.01\\
55.726704665001	1.01\\
55.74182305	1.01\\
55.757380772	1.01\\
55.773457174	1.01\\
55.789383023999	1.01\\
55.805396653	1.01\\
55.821326031002	1.01\\
55.837353466999	1.01\\
55.853337787003	1.01\\
55.869283749001	1.02\\
55.885390620999	1.02\\
55.901335685002	1.02\\
55.917390418	1.02\\
55.933809550999	1.02\\
55.949216156002	1.02\\
55.967510904	1.02\\
55.982802155999	1.02\\
56.000346632	1.02\\
56.015431153	1.02\\
56.030563184998	1.02\\
56.045807199002	1.02\\
56.061421895001	1.02\\
56.077428505001	1.02\\
56.093356158001	1.02\\
56.109547254002	1.02\\
56.125319574002	1.02\\
56.141430436001	1.01\\
56.157365201	1.01\\
56.175299956002	1.01\\
56.190537966	1.01\\
56.205617909001	1.01\\
56.221384895001	1.01\\
56.237259897	1.01\\
56.253380731999	1.01\\
56.26937834	1.01\\
56.285419010002	1.01\\
56.301374809002	1.01\\
56.317392152001	1.01\\
56.333382089001	1.01\\
56.349400056	1.01\\
56.365395790001	1.01\\
56.381443990998	1.01\\
56.397395641003	1.01\\
56.41342675	1.01\\
56.429435562001	1.01\\
56.445443460003	1.01\\
56.461354201	1.01\\
56.477448768002	1.01\\
56.493379710999	1.01\\
56.509675183998	1.01\\
56.526711355999	1.01\\
56.541784974003	1.01\\
56.557407926003	1.01\\
56.573359136002	1.01\\
56.589194312001	1.01\\
56.605350772999	1.01\\
56.621319452999	1.01\\
56.637500032002	1.01\\
56.653385097	1.01\\
56.669243164002	1\\
56.685364501999	1\\
56.701366919998	1\\
56.717408901001	1\\
56.733326942002	1\\
56.749419149002	1\\
56.765357212002	1\\
56.781427294003	1\\
56.797419082001	1\\
56.813369139	1\\
56.829441963002	1\\
56.845484875	1\\
56.861492705002	1\\
56.877569302002	1\\
56.893351681999	1\\
56.909573202	1\\
56.925325549	1\\
56.941383738999	1\\
56.957366464001	1\\
56.973458145001	1\\
56.989422342003	1\\
57.007404869	1\\
57.022801849003	1\\
57.038054027001	1\\
57.053351326	1\\
57.069344793999	1\\
57.085395101002	1\\
57.101301086003	1\\
57.117612305001	1\\
57.133342573002	1\\
57.149428039002	1\\
57.165375366001	1\\
57.181425305001	1\\
57.197360286003	1\\
57.213417703999	1\\
57.229286597	1\\
57.245362778004	1\\
57.261406287003	1\\
57.277319404	1\\
57.293322769001	1\\
57.309685191002	1\\
57.325343941002	1\\
57.341404867001	1.01\\
57.357403498001	1.01\\
57.373416099999	1.01\\
57.389373251	1.01\\
57.405438891003	1.01\\
57.421419898003	1.01\\
57.437296367001	1.01\\
57.453298920998	1.01\\
57.469360022999	1.01\\
57.485422799	1.01\\
57.501267612	1.01\\
57.517416626	1.01\\
57.533359189003	1.01\\
57.549207667004	1.01\\
57.567467605	1.01\\
57.582706313004	1.01\\
57.597955543999	1.01\\
57.613466785	1.01\\
57.629392588002	1.01\\
57.645472327	1.01\\
57.661406054001	1.01\\
57.677467543999	1.01\\
57.693348103001	1.01\\
57.709506206002	1.01\\
57.725276911003	1.01\\
57.741363051998	1.01\\
57.757332065003	1.01\\
57.773382293	1.01\\
57.789374484001	1.01\\
57.805342965	1.01\\
57.821422299	1.01\\
57.837429513001	1.01\\
57.853410383	1.01\\
57.869362553002	1.02\\
57.885226243	1.02\\
57.901362761002	1.02\\
57.917445936001	1.02\\
57.933327888001	1.02\\
57.949426309998	1.02\\
57.965297827999	1.02\\
57.981421370003	1.02\\
57.999762771	1.02\\
58.015062497002	1.02\\
58.030485953003	1.02\\
58.045648834	1.02\\
58.061348161999	1.02\\
58.077507937001	1.02\\
58.093362240002	1.02\\
58.109606482003	1.02\\
58.125291432999	1.01\\
58.141376161999	1.01\\
58.157342404999	1.01\\
58.173386441998	1.01\\
58.189357901001	1.01\\
58.205341892002	1.01\\
58.221459498001	1.01\\
58.237435792004	1.01\\
58.25345984	1.01\\
58.269391997002	1.01\\
58.285485119999	1.01\\
58.301375063004	1.01\\
58.317371904	1.01\\
58.333404441002	1.01\\
58.349379408001	1.01\\
58.365298042	1.01\\
58.381431502999	1.01\\
58.397378496002	1.01\\
58.413347829998	1.01\\
58.429437426998	1.01\\
58.445454334	1.01\\
58.461367430001	1.01\\
58.477559379002	1.01\\
58.493386731999	1.01\\
58.509650832001	1.01\\
58.525401902001	1.01\\
58.541449297001	1.01\\
58.557335422001	1.01\\
58.573371167	1.01\\
58.589305724999	1.01\\
58.605357469002	1.01\\
58.621434992001	1.01\\
58.637353987	1.01\\
58.653438993	1\\
58.669629924	1\\
58.685303313999	1\\
58.701391455002	1\\
58.717428465	1\\
58.733381919998	1\\
58.749435109001	1\\
58.765343270001	1\\
58.78145914	1\\
58.797405622002	1\\
58.813469456002	1\\
58.829425319001	1\\
58.845408599003	1\\
58.861347309002	1\\
58.877479071003	1\\
58.893356201	1\\
58.909643195	1\\
58.925395422001	1\\
58.941394156998	1\\
58.957340149002	1\\
58.973391371002	1\\
58.989182594002	1\\
59.005407040001	1\\
59.021387950001	1\\
59.037373467999	1\\
59.053371710003	1\\
59.069428221001	1\\
59.085388269001	1\\
59.101589723999	1\\
59.117342613003	1\\
59.133275053002	1\\
59.149410528	1\\
59.165411549004	1\\
59.181382335999	1\\
59.197344213002	1\\
59.213405895001	1\\
59.229364330002	1\\
59.245454844002	1\\
59.261390973	1\\
59.277705249001	1\\
59.293404985001	1\\
59.309596834	1\\
59.325299759999	1\\
59.341382220001	1\\
59.357374040001	1.01\\
59.373349934002	1.01\\
59.389342234001	1.01\\
59.405365058998	1.01\\
59.421326063	1.01\\
59.437194869999	1.01\\
59.453368583	1.01\\
59.469345464001	1.01\\
59.485399960999	1.01\\
59.501339086003	1.01\\
59.517490277001	1.01\\
59.533395816002	1.01\\
59.549415655003	1.01\\
59.565473584999	1.01\\
59.581491383	1.01\\
59.597336721001	1.01\\
59.613484770001	1.01\\
59.629390223999	1.01\\
59.645568261002	1.01\\
59.661398570999	1.01\\
59.677528533001	1.01\\
59.69337591	1.01\\
59.711598385998	1.01\\
59.726661	1.01\\
59.741785554001	1.01\\
59.757329939999	1.01\\
59.773334343003	1.01\\
59.789719133004	1.01\\
59.805344097	1.01\\
59.821343679001	1.01\\
59.837442256001	1.01\\
59.853391534001	1.01\\
59.869337706002	1.02\\
59.885416131001	1.02\\
59.901384248001	1.02\\
59.917381100003	1.02\\
59.933348684002	1.02\\
59.949387364998	1.02\\
59.965422393002	1.02\\
59.981372408001	1.02\\
59.999566389	1.02\\
60.014909449002	1.02\\
60.030103600998	1.02\\
60.045279624001	1.02\\
60.061376765	1.02\\
60.077586379002	1.02\\
60.093419444001	1.02\\
60.109373563	1.02\\
60.125423743	1.01\\
60.141319618	1.01\\
60.157281247002	1.01\\
60.173317642998	1.01\\
60.189215845001	1.01\\
60.205414216004	1.01\\
60.22134307	1.01\\
60.237427533001	1.01\\
60.253425221001	1.01\\
60.269346631001	1.01\\
60.285362971001	1.01\\
60.301335648003	1.01\\
60.317504222	1.01\\
60.333434233002	1.01\\
60.349435068001	1.01\\
60.365335195004	1.01\\
60.381471993	1.01\\
60.397175400002	1.01\\
60.413385034001	1.01\\
60.429390630001	1.01\\
60.445371144001	1.01\\
60.461328170002	1.01\\
60.479607504002	1.01\\
60.494748255001	1.01\\
60.509834568001	1.01\\
60.525414558003	1.01\\
60.541370799	1.01\\
60.557361154999	1.01\\
60.573399170002	1.01\\
60.589362785004	1.01\\
60.605459753998	1.01\\
60.621386586003	1.01\\
60.637378501	1.01\\
60.653377543	1\\
60.669353507	1\\
60.685359438	1\\
60.701362504002	1\\
60.717389433003	1\\
60.733738362	1\\
60.749396766003	1\\
60.765370626	1\\
60.781405769001	1\\
60.79738973	1\\
60.813288449002	1\\
60.829382390999	1\\
60.845723226002	1\\
60.861361165001	1\\
60.877550136002	1\\
60.893336711003	1\\
60.911614166001	1\\
60.926710156002	1\\
60.941962780003	1\\
60.957352976002	1\\
60.973336411999	1\\
60.989323501	1\\
61.006898159001	1\\
61.022402399002	1\\
61.037731582001	1\\
61.053414290001	1\\
61.069351998001	1\\
61.085411605999	1\\
61.101364072003	1\\
61.117396373001	1\\
61.133339694001	1\\
61.149294266003	1\\
61.165418488999	1\\
61.181481971001	1\\
61.197343710999	1\\
61.213480561001	1\\
61.229369737	1\\
61.245468950001	1\\
61.261261966	1\\
61.277469834	1\\
61.293372577004	1\\
61.311701078003	1\\
61.326935416001	1\\
61.342141255001	1\\
61.357391938999	1.01\\
61.375765536999	1.01\\
61.390954347	1.01\\
61.406018720001	1.01\\
61.421379209004	1.01\\
61.437355098004	1.01\\
61.453425977001	1.01\\
61.469334693001	1.01\\
61.485491006001	1.01\\
61.501380664002	1.01\\
61.517396248001	1.01\\
61.533358897	1.01\\
61.549380930001	1.01\\
61.565374303997	1.01\\
61.581421108002	1.01\\
61.597353899002	1.01\\
61.613344260002	1.01\\
61.629312311001	1.01\\
61.645430829003	1.01\\
61.661386246002	1.01\\
61.677439038998	1.01\\
61.693260506001	1.01\\
61.709619919998	1.01\\
61.725396141999	1.01\\
61.741305112	1.01\\
61.757365244	1.01\\
61.773382290001	1.01\\
61.789555737	1.01\\
61.805375893002	1.01\\
61.821458265	1.01\\
61.837411490002	1.01\\
61.853334194001	1.01\\
61.869268225003	1.01\\
61.885389019001	1.02\\
61.90129209	1.02\\
61.917383965	1.02\\
61.933329665001	1.02\\
61.949277075001	1.02\\
61.965357839001	1.02\\
61.981383984001	1.02\\
61.999633973	1.02\\
62.015040384999	1.02\\
62.030207555001	1.02\\
62.045477158001	1.02\\
62.061371347	1.02\\
62.077412629002	1.02\\
62.093370191002	1.02\\
62.111841310002	1.01\\
62.126949236	1.01\\
62.142216875	1.01\\
62.157574497002	1.01\\
62.173371757	1.01\\
62.189315566002	1.01\\
62.205413570004	1.01\\
62.221518379002	1.01\\
62.237390877999	1.01\\
62.253364522	1.01\\
62.269347737004	1.01\\
62.285393244004	1.01\\
62.301338042004	1.01\\
62.317381148003	1.01\\
62.333442982998	1.01\\
62.349701576	1.01\\
62.365367277001	1.01\\
62.381428925	1.01\\
62.397410336003	1.01\\
62.413402783001	1.01\\
62.429411570999	1.01\\
62.445502435002	1.01\\
62.461279538002	1.01\\
62.477470129002	1.01\\
62.493456883999	1.01\\
62.509656618	1.01\\
62.525328737999	1.01\\
62.541387108998	1.01\\
62.557307226002	1.01\\
62.573395925999	1.01\\
62.589427335999	1.01\\
62.605369293	1.01\\
62.621487567002	1.01\\
62.637342854	1\\
62.653437085999	1\\
62.669389757	1\\
62.685416426003	1\\
62.701342422001	1\\
62.717446098	1\\
62.733367305001	1\\
62.751782277001	1\\
62.767001977001	1\\
62.782200714001	1\\
62.797504019001	1\\
62.813398460999	1\\
62.829387789002	1\\
62.845490067002	1\\
62.861363875	1\\
62.877433756001	1\\
62.893248585003	1\\
62.909595509999	1\\
62.925307064003	1\\
62.941356605	1\\
62.957267084999	1\\
62.975505842999	1\\
62.990789032002	1\\
63.005431438	1\\
63.021341653	1\\
63.037379272	1\\
63.053398081002	1\\
63.069326419003	1\\
63.085467502003	1\\
63.101375587002	1\\
63.117409457001	1\\
63.133264602001	1\\
63.149359726002	1\\
63.165374938999	1\\
63.181412884003	1\\
63.197383524998	1\\
63.213379093998	1\\
63.229427140999	1\\
63.245476927002	1\\
63.261356808003	1\\
63.277528260002	1\\
63.293330861	1\\
63.309435827	1\\
63.325346153	1\\
63.341374495003	1\\
63.357382947003	1\\
63.373439610001	1.01\\
63.389383482003	1.01\\
63.405315440998	1.01\\
63.421461002999	1.01\\
63.437359551003	1.01\\
63.453402300003	1.01\\
63.469347032997	1.01\\
63.485379806999	1.01\\
63.501281133004	1.01\\
63.517381644001	1.01\\
63.533323149002	1.01\\
63.549443902001	1.01\\
63.565370280003	1.01\\
63.581333491001	1.01\\
63.597365033001	1.01\\
63.613372248001	1.01\\
63.629377483002	1.01\\
63.645321434002	1.01\\
63.661387593003	1.01\\
63.677581453003	1.01\\
63.693384969002	1.01\\
63.711775940998	1.01\\
63.726992208	1.01\\
63.742183317002	1.01\\
63.757514461003	1.01\\
63.773316770001	1.01\\
63.789462770001	1.01\\
63.805366459	1.01\\
63.821425658001	1.01\\
63.837421592003	1.01\\
63.853418128002	1.01\\
63.869357164002	1.01\\
63.885356362	1.01\\
63.901368666001	1.02\\
63.917412904	1.02\\
63.933382448998	1.02\\
63.949544775002	1.02\\
63.965324381001	1.02\\
63.981434557003	1.02\\
63.999626829003	1.02\\
};
\addplot [color=mycolor1,solid,forget plot]
  table[row sep=crcr]{%
63.999626829003	1.02\\
64.014765815998	1.02\\
64.030066880001	1.02\\
64.045489307999	1.02\\
64.061401860001	1.02\\
64.077348879002	1.02\\
64.093338211998	1.02\\
64.111725084999	1.01\\
64.127164043	1.01\\
64.142271563	1.01\\
64.157709525002	1.01\\
64.173440029	1.01\\
64.189384525002	1.01\\
64.205371443001	1.01\\
64.221378506001	1.01\\
64.237364306999	1.01\\
64.253310353001	1.01\\
64.269384508	1.01\\
64.285397354	1.01\\
64.301389251999	1.01\\
64.317509568001	1.01\\
64.333330162003	1.01\\
64.349385243	1.01\\
64.365359819001	1.01\\
64.381351814003	1.01\\
64.397344186001	1.01\\
64.413385066002	1.01\\
64.429429946999	1.01\\
64.445410145001	1.01\\
64.461286152001	1.01\\
64.477487247002	1.01\\
64.493362442002	1.01\\
64.509593609001	1.01\\
64.525320390999	1.01\\
64.541449870003	1.01\\
64.557258077	1.01\\
64.575483622002	1.01\\
64.590683801998	1.01\\
64.605775668	1.01\\
64.621383536999	1.01\\
64.637384177002	1\\
64.653391376999	1\\
64.669376577	1\\
64.685388372002	1\\
64.701363529	1\\
64.717422646	1\\
64.733342814003	1\\
64.749421205002	1\\
64.765361257	1\\
64.781381806999	1\\
64.797363223999	1\\
64.813424004998	1\\
64.829351929001	1\\
64.845329397003	1\\
64.861335414002	1\\
64.877414922001	1\\
64.893367268002	1\\
64.909879578003	1\\
64.925384529	1\\
64.941472422001	1\\
64.957388284001	1\\
64.973456522	1\\
64.989351999001	1\\
65.006902467003	1\\
65.022407548001	1\\
65.037807616001	1\\
65.053420447998	1\\
65.069314803002	1\\
65.085446179001	1\\
65.101327120999	1\\
65.117213196003	1\\
65.135465324002	1\\
65.150738169998	1\\
65.166029742001	1\\
65.181296282002	1\\
65.197375526001	1\\
65.21337998	1\\
65.229315822003	1\\
65.245430064999	1\\
65.261365692002	1\\
65.277469140003	1\\
65.293375707001	1\\
65.311782780999	1\\
65.326825117001	1\\
65.342024548001	1\\
65.35733034	1\\
65.373366067002	1\\
65.389388256001	1.01\\
65.405411672001	1.01\\
65.421323813999	1.01\\
65.437359868	1.01\\
65.453310482003	1.01\\
65.469280680001	1.01\\
65.485253950001	1.01\\
65.501339128002	1.01\\
65.517382884003	1.01\\
65.533332906002	1.01\\
65.549373134999	1.01\\
65.565323406002	1.01\\
65.581399667	1.01\\
65.597347026001	1.01\\
65.613394356999	1.01\\
65.629367179001	1.01\\
65.645489824002	1.01\\
65.661385270001	1.01\\
65.677442855004	1.01\\
65.693292386002	1.01\\
65.709352712002	1.01\\
65.725466258004	1.01\\
65.742127092999	1.01\\
65.757427477001	1.01\\
65.773330723	1.01\\
65.789369097	1.01\\
65.805354511997	1.01\\
65.821390543	1.01\\
65.837416028	1.01\\
65.853268527001	1.01\\
65.869478766999	1.01\\
65.885376985001	1.01\\
65.901329526001	1.02\\
65.917407	1.02\\
65.933377476998	1.02\\
65.949385468003	1.02\\
65.965346012001	1.02\\
65.981347597	1.02\\
65.999603630001	1.02\\
66.014744727001	1.02\\
66.029828409001	1.02\\
66.045396083004	1.02\\
};
\end{axis}
\end{tikzpicture}%
}
      \caption{The orientation of the robot over time for
        $K_{\omega}^T = K_{\omega, max}^T$.}
      \label{fig:13_max}
    \end{figure}
  \end{minipage}
  \hfill
  \begin{minipage}{0.45\linewidth}
    \begin{figure}[H]
      \scalebox{0.6}{% This file was created by matlab2tikz.
%
%The latest updates can be retrieved from
%  http://www.mathworks.com/matlabcentral/fileexchange/22022-matlab2tikz-matlab2tikz
%where you can also make suggestions and rate matlab2tikz.
%
\definecolor{mycolor1}{rgb}{0.00000,0.44700,0.74100}%
%
\begin{tikzpicture}

\begin{axis}[%
width=4.133in,
height=3.26in,
at={(0.693in,0.44in)},
scale only axis,
xmin=36.781452667,
xmax=70,
xmajorgrids,
ymin=0.95,
ymax=1.1,
ymajorgrids,
axis background/.style={fill=white}
]
\addplot [color=mycolor1,solid,forget plot]
  table[row sep=crcr]{%
36.781452667	1\\
36.797345292004	1\\
36.813495977001	1\\
36.829422899002	1\\
36.845370869999	1\\
36.861347120003	1\\
36.877517459	1\\
36.893385225003	1\\
36.90941309	1\\
36.925306968003	1\\
36.941373271004	1\\
36.957287154	1\\
36.973381766999	1\\
36.989390425	1\\
37.005357770001	1\\
37.021420801003	1\\
37.037407724003	1\\
37.053325100003	1\\
37.069328712002	1\\
37.085387112999	1\\
37.101338942002	1\\
37.117379698002	1\\
37.133318262001	1\\
37.149284965	1\\
37.165395434002	1\\
37.18134459	1\\
37.197391385998	1\\
37.213419970001	1\\
37.229356966	1\\
37.24535932	1\\
37.26137339	1\\
37.277448101998	1\\
37.293364205002	1.01\\
37.309507283001	1.01\\
37.325319857998	1.01\\
37.341303075001	1.01\\
37.357410563999	1.01\\
37.373606181004	1.01\\
37.389372549004	1.01\\
37.405376131001	1.01\\
37.421482855999	1.01\\
37.437379431004	1.01\\
37.453514374001	1.01\\
37.469371965	1.01\\
37.485429327999	1.01\\
37.501414805001	1.01\\
37.517348679001	1.01\\
37.533360107003	1.01\\
37.549368258999	1.01\\
37.565362320004	1.01\\
37.581320529	1.01\\
37.597347131001	1.01\\
37.613341028004	1.01\\
37.629320932999	1.01\\
37.645467369	1.01\\
37.661417083	1.01\\
37.677467737004	1.01\\
37.693298879002	1.01\\
37.709525869999	1.01\\
37.725406624001	1.01\\
37.741372554001	1.01\\
37.757330508	1.01\\
37.773355047001	1.01\\
37.789346039002	1.01\\
37.805367121002	1.01\\
37.821449300003	1.02\\
37.837466758	1.02\\
37.853378938004	1.02\\
37.869351292	1.02\\
37.885422333004	1.02\\
37.901459360001	1.02\\
37.917388914002	1.02\\
37.933390004002	1.02\\
37.949405522999	1.02\\
37.965342989998	1.02\\
37.981388190003	1.02\\
37.999709284001	1.02\\
38.015503702004	1.02\\
38.031214710003	1.02\\
38.0465055	1.02\\
38.061774824002	1.02\\
38.077530759003	1.02\\
38.093329937001	1.02\\
38.111762429001	1.02\\
38.126961685002	1.02\\
38.142168679001	1.02\\
38.157430649002	1.02\\
38.173479934002	1.02\\
38.189387869999	1.01\\
38.205380931004	1.01\\
38.221434987	1.01\\
38.237325493	1.01\\
38.253276797001	1.01\\
38.269364583	1.01\\
38.285396965	1.01\\
38.301342801003	1.01\\
38.317429580998	1.01\\
38.333370139	1.01\\
38.349408934998	1.01\\
38.365401963002	1.01\\
38.381401445	1.01\\
38.397713832001	1.01\\
38.413541055001	1.01\\
38.429370671002	1.01\\
38.445498202999	1.01\\
38.461262224999	1.01\\
38.477443801998	1.01\\
38.493355520001	1.01\\
38.511730301003	1.01\\
38.526906041001	1.01\\
38.542243568001	1.01\\
38.557589898999	1.01\\
38.573443982003	1.01\\
38.589315066002	1.01\\
38.605445478001	1.01\\
38.621441563	1.01\\
38.637388997002	1.01\\
38.653418848	1.01\\
38.669405195	1.01\\
38.685483631001	1.01\\
38.701352191002	1\\
38.717464595001	1\\
38.733361765	1\\
38.749390063	1\\
38.765371683998	1\\
38.781394069001	1\\
38.797331089001	1\\
38.813430156998	1\\
38.829398133	1\\
38.845445664002	1\\
38.861415278	1\\
38.877455258	1\\
38.893215798001	1\\
38.909393094002	1\\
38.925419036	1\\
38.941343999001	1\\
38.957393183998	1\\
38.973390315003	1\\
38.989348133	1\\
39.007068742001	1\\
39.022441057003	1\\
39.037572164002	1\\
39.053381867001	1\\
39.069378033001	1\\
39.085416609001	1\\
39.10132966	1\\
39.117198438004	1\\
39.133291794998	1\\
39.149421659001	1\\
39.165365098	1\\
39.181405103001	1\\
39.197267412003	1\\
39.213391970001	1\\
39.229260282997	1\\
39.245381924	1\\
39.26134975	1\\
39.277442208	1\\
39.293326789002	1\\
39.309602343003	1.01\\
39.325351743	1.01\\
39.341374337002	1.01\\
39.357293082001	1.01\\
39.373358694001	1.01\\
39.389417896	1.01\\
39.405398977001	1.01\\
39.421386562001	1.01\\
39.437345622998	1.01\\
39.453475236	1.01\\
39.469373693001	1.01\\
39.485408736	1.01\\
39.501335569001	1.01\\
39.517473018002	1.01\\
39.533399710999	1.01\\
39.549469517002	1.01\\
39.565354339001	1.01\\
39.581399064003	1.01\\
39.597385312001	1.01\\
39.613413414002	1.01\\
39.629376369004	1.01\\
39.645455331002	1.01\\
39.661349376004	1.01\\
39.677441286	1.01\\
39.693342926998	1.01\\
39.709566967999	1.01\\
39.725298084999	1.01\\
39.741358586003	1.01\\
39.757404681	1.01\\
39.773383515	1.01\\
39.789399544998	1.01\\
39.805221582001	1.01\\
39.821254505001	1.01\\
39.837340621002	1.02\\
39.853378250004	1.02\\
39.869334710999	1.02\\
39.885272101002	1.02\\
39.901388052002	1.02\\
39.917383098999	1.02\\
39.933424161003	1.02\\
39.949379924	1.02\\
39.965328201	1.02\\
39.981310958	1.02\\
39.999595615998	1.02\\
40.014762530999	1.02\\
40.030131254002	1.02\\
40.045445804001	1.02\\
40.061408356999	1.02\\
40.077327093003	1.02\\
40.093398	1.02\\
40.111754730004	1.02\\
40.127093207001	1.02\\
40.142488918999	1.02\\
40.157662055001	1.02\\
40.173299203003	1.01\\
40.189461801003	1.01\\
40.205305029003	1.01\\
40.22128532	1.01\\
40.237335825001	1.01\\
40.253403031002	1.01\\
40.269342101002	1.01\\
40.285430600003	1.01\\
40.301342684002	1.01\\
40.317407124001	1.01\\
40.333298982003	1.01\\
40.349359167	1.01\\
40.365295746002	1.01\\
40.381349030003	1.01\\
40.397308773003	1.01\\
40.413417152001	1.01\\
40.429403915001	1.01\\
40.445393193001	1.01\\
40.461377772	1.01\\
40.477600982003	1.01\\
40.493321166001	1.01\\
40.509586706002	1.01\\
40.525306995003	1.01\\
40.541335002999	1.01\\
40.557374318001	1.01\\
40.573513469002	1.01\\
40.589392889	1.01\\
40.605420470001	1.01\\
40.621397647	1.01\\
40.637323793	1.01\\
40.653236872002	1.01\\
40.669310363999	1.01\\
40.685245698002	1.01\\
40.701323055001	1\\
40.717388747002	1\\
40.733349671997	1\\
40.749423912998	1\\
40.765348165001	1\\
40.781392930001	1\\
40.797330222004	1\\
40.813350426003	1\\
40.829438724999	1\\
40.845462061001	1\\
40.861468930001	1\\
40.877720056	1\\
40.893349459999	1\\
40.909491835999	1\\
40.927061397	1\\
40.942203640999	1\\
40.957380479004	1\\
40.973392845001	1\\
40.989327940998	1\\
41.006991200001	1\\
41.022635993	1\\
41.037799105	1\\
41.053460279	1\\
41.069370220001	1\\
41.085383222004	1\\
41.101472994	1\\
41.117299247998	1\\
41.133403240002	1\\
41.14939166	1\\
41.165404868	1\\
41.181382144001	1\\
41.197398485001	1\\
41.213366455998	1\\
41.229331783001	1\\
41.245490739998	1\\
41.261424306999	1\\
41.277531018002	1\\
41.293374136002	1\\
41.309670997002	1.01\\
41.325356628002	1.01\\
41.341469050999	1.01\\
41.357319426998	1.01\\
41.373321193001	1.01\\
41.389430439003	1.01\\
41.405394972	1.01\\
41.421423246002	1.01\\
41.437421123001	1.01\\
41.453311732998	1.01\\
41.469360537003	1.01\\
41.485422851002	1.01\\
41.501369464001	1.01\\
41.517367270001	1.01\\
41.533365868	1.01\\
41.549514258	1.01\\
41.56534857	1.01\\
41.581386906002	1.01\\
41.597380369	1.01\\
41.613350184002	1.01\\
41.629394488003	1.01\\
41.645524777001	1.01\\
41.661391979	1.01\\
41.677565696003	1.01\\
41.693375043	1.01\\
41.709559354	1.01\\
41.725395445999	1.01\\
41.741387898003	1.01\\
41.757330280003	1.01\\
41.773400678002	1.01\\
41.789688228001	1.01\\
41.805406603001	1.01\\
41.821400551998	1.01\\
41.837403312001	1.02\\
41.853377245999	1.02\\
41.869349725998	1.02\\
41.885382209999	1.02\\
41.901306553002	1.02\\
41.917418082001	1.02\\
41.933403675	1.02\\
41.949321461003	1.02\\
41.965344130001	1.02\\
41.981429998001	1.02\\
41.999649693001	1.02\\
42.014924512001	1.02\\
42.030512566998	1.02\\
42.045792494	1.02\\
42.061338414997	1.02\\
42.077597342999	1.02\\
42.09337182	1.02\\
42.111835097	1.02\\
42.125386861	1.02\\
42.141404494999	1.02\\
42.157424688	1.02\\
42.173407834004	1.01\\
42.189331729	1.01\\
42.205328654999	1.01\\
42.221287471001	1.01\\
42.237302485001	1.01\\
42.253390205002	1.01\\
42.269334858002	1.01\\
42.285374361	1.01\\
42.301331458	1.01\\
42.317413612	1.01\\
42.333359583	1.01\\
42.349393629002	1.01\\
42.365372244	1.01\\
42.381370373001	1.01\\
42.397325440003	1.01\\
42.413389834	1.01\\
42.429399202	1.01\\
42.445478396	1.01\\
42.461286696999	1.01\\
42.477432123001	1.01\\
42.493241648003	1.01\\
42.511655444001	1.01\\
42.526797658001	1.01\\
42.541879620999	1.01\\
42.557393286999	1.01\\
42.573294346001	1.01\\
42.589389800999	1.01\\
42.605295720001	1.01\\
42.621382661	1.01\\
42.637385707001	1.01\\
42.653378887001	1.01\\
42.669341729	1.01\\
42.685374210999	1.01\\
42.701328145001	1\\
42.717406070004	1\\
42.733502186001	1\\
42.749419402001	1\\
42.765335661	1\\
42.781421057999	1\\
42.797465486	1\\
42.813362622002	1\\
42.829271444001	1\\
42.845467104004	1\\
42.861336864003	1\\
42.877399073002	1\\
42.893329324002	1\\
42.909317550999	1\\
42.925404843003	1\\
42.941330942997	1\\
42.957343868004	1\\
42.973400251004	1\\
42.989312375	1\\
43.007213558003	1\\
43.021937213002	1\\
43.037331709999	1\\
43.053392666001	1\\
43.069372573998	1\\
43.085295903004	1\\
43.101328009003	1\\
43.117416223004	1\\
43.133363031002	1\\
43.149335452	1\\
43.165508230999	1\\
43.181282036	1\\
43.197377897999	1\\
43.213418493	1\\
43.229367722	1\\
43.245396996002	1\\
43.261380111	1\\
43.277448029	1\\
43.293324071003	1\\
43.311795536	1.01\\
43.326950685002	1.01\\
43.342102896	1.01\\
43.357398153	1.01\\
43.373438967999	1.01\\
43.389236636002	1.01\\
43.405377676003	1.01\\
43.421429827999	1.01\\
43.437306613999	1.01\\
43.453394720001	1.01\\
43.469340371002	1.01\\
43.485432267002	1.01\\
43.501382984001	1.01\\
43.517392373001	1.01\\
43.533226865002	1.01\\
43.549323200001	1.01\\
43.565258216	1.01\\
43.581416607003	1.01\\
43.597387124001	1.01\\
43.613334418999	1.01\\
43.629321303002	1.01\\
43.645455433003	1.01\\
43.661371478001	1.01\\
43.677482391003	1.01\\
43.693402348999	1.01\\
43.709423084004	1.01\\
43.725256314999	1.01\\
43.741377802998	1.01\\
43.757424993	1.01\\
43.773412529	1.01\\
43.789463371998	1.01\\
43.805355596001	1.01\\
43.821354200001	1.01\\
43.837356034001	1.02\\
43.853400147999	1.02\\
43.869378131001	1.02\\
43.885395994999	1.02\\
43.901332574997	1.02\\
43.917283003002	1.02\\
43.933338826	1.02\\
43.949492567002	1.02\\
43.965274514	1.02\\
43.981402363999	1.02\\
43.999588989003	1.02\\
44.015002675003	1.02\\
44.030277774002	1.02\\
44.045446257004	1.02\\
44.061325655999	1.02\\
44.077417783001	1.02\\
44.093332367001	1.02\\
44.109345581002	1.02\\
44.125434614003	1.02\\
44.141395278	1.02\\
44.157349221001	1.02\\
44.173385743	1.01\\
44.189645343998	1.01\\
44.205475716	1.01\\
44.221417099003	1.01\\
44.23735873	1.01\\
44.253303001999	1.01\\
44.269350368	1.01\\
44.285453849999	1.01\\
44.301367177998	1.01\\
44.317425924	1.01\\
44.333321907002	1.01\\
44.349331912998	1.01\\
44.365373065003	1.01\\
44.381407185002	1.01\\
44.397294688	1.01\\
44.413375791001	1.01\\
44.429358168999	1.01\\
44.445429350998	1.01\\
44.461353586003	1.01\\
44.477480316002	1.01\\
44.493391528004	1.01\\
44.509669556999	1.01\\
44.525287641003	1.01\\
44.541340255001	1.01\\
44.557350985001	1.01\\
44.573419930001	1.01\\
44.589235730999	1.01\\
44.605404612	1.01\\
44.621383853001	1.01\\
44.637348577004	1.01\\
44.653412815998	1.01\\
44.669408700001	1.01\\
44.685277495999	1\\
44.701357471001	1\\
44.717383698002	1\\
44.733383727001	1\\
44.749441004998	1\\
44.765371065003	1\\
44.781550536003	1\\
44.797348851002	1\\
44.813295344002	1\\
44.829412096001	1\\
44.845422233002	1\\
44.861322156002	1\\
44.877498140999	1\\
44.893428627003	1\\
44.911804800999	1\\
44.927060359001	1\\
44.942221520001	1\\
44.957526816002	1\\
44.973389526001	1\\
44.989389278	1\\
45.006990134999	1\\
45.022879431	1\\
45.038034327	1\\
45.053322657002	1\\
45.069374036	1\\
45.085423346001	1\\
45.101338461998	1\\
45.117389837002	1\\
45.133334870999	1\\
45.149532477001	1\\
45.165461139	1\\
45.181413138001	1\\
45.197302776001	1\\
45.213319492001	1\\
45.229369430001	1\\
45.245388126	1\\
45.261289851998	1\\
45.277389106999	1\\
45.293415126004	1\\
45.309373686001	1\\
45.325453263001	1.01\\
45.341254233002	1.01\\
45.357269062001	1.01\\
45.373286100998	1.01\\
45.389324872998	1.01\\
45.406614429001	1.01\\
45.422472896	1.01\\
45.437721062001	1.01\\
45.453421737	1.01\\
45.469339237	1.01\\
45.485424573998	1.01\\
45.501362636997	1.01\\
45.517413746002	1.01\\
45.533372314999	1.01\\
45.549306543999	1.01\\
45.565332101002	1.01\\
45.581369810002	1.01\\
45.597303941998	1.01\\
45.613400662003	1.01\\
45.629363756001	1.01\\
45.645434425003	1.01\\
45.661387577004	1.01\\
45.677544992001	1.01\\
45.693479903	1.01\\
45.709617033001	1.01\\
45.725431264	1.01\\
45.741356549	1.01\\
45.757353168999	1.01\\
45.773465964001	1.01\\
45.789362998001	1.01\\
45.805334878002	1.01\\
45.821333481999	1.01\\
45.837343650002	1.01\\
45.853384711003	1.02\\
45.869367620999	1.02\\
45.885379661999	1.02\\
45.901344290001	1.02\\
45.917399584	1.02\\
45.933359245999	1.02\\
45.94941932	1.02\\
45.965329142002	1.02\\
45.981474708004	1.02\\
45.999562747998	1.02\\
46.014688947003	1.02\\
46.030286508004	1.02\\
46.045570849003	1.02\\
46.061517009999	1.02\\
46.077518397	1.02\\
46.093265914997	1.02\\
46.109416352001	1.02\\
46.125314539002	1.02\\
46.141387173001	1.02\\
46.157329336003	1.01\\
46.173441805001	1.01\\
46.189360222	1.01\\
46.205293117001	1.01\\
46.221419062001	1.01\\
46.237334738999	1.01\\
46.253336596001	1.01\\
46.269322549	1.01\\
46.285362588002	1.01\\
46.301342480004	1.01\\
46.317385490998	1.01\\
46.333375282002	1.01\\
46.349416464001	1.01\\
46.365372619	1.01\\
46.381363841	1.01\\
46.397358858002	1.01\\
46.413397062001	1.01\\
46.429385345001	1.01\\
46.445430496002	1.01\\
46.461418426998	1.01\\
46.477566465	1.01\\
46.493340897999	1.01\\
46.511854303002	1.01\\
46.527205937001	1.01\\
46.542274674	1.01\\
46.557503119	1.01\\
46.573369885998	1.01\\
46.589417830002	1.01\\
46.605370182999	1.01\\
46.621439753998	1.01\\
46.637386908001	1.01\\
46.653367456997	1.01\\
46.669339051003	1.01\\
46.685463576	1\\
46.701332663002	1\\
46.717499374001	1\\
46.733299627999	1\\
46.749422091004	1\\
46.765328212002	1\\
46.781439327004	1\\
46.797354763001	1\\
46.813414781998	1\\
46.829507588002	1\\
46.845449624001	1\\
46.861373615002	1\\
46.877465852001	1\\
46.893395556	1\\
46.909509176998	1\\
46.925323492001	1\\
46.94364505	1\\
46.958825761002	1\\
46.973903535	1\\
46.989337466999	1\\
47.006939203999	1\\
47.022104536003	1\\
47.037307627003	1\\
47.053390735001	1\\
47.069250237	1\\
47.085308497002	1\\
47.101371231003	1\\
47.117249508	1\\
47.133383789997	1\\
47.149423444001	1\\
47.165403432003	1\\
47.181392076	1\\
47.197373180001	1\\
47.213384685002	1\\
47.229403997002	1\\
47.245488126004	1\\
47.261359242001	1\\
47.277330228001	1\\
47.293394268002	1\\
47.311822751	1\\
47.327048605999	1.01\\
47.342266564999	1.01\\
47.357533984001	1.01\\
47.373439999001	1.01\\
47.389347943001	1.01\\
47.405395329998	1.01\\
47.421417487999	1.01\\
47.437340815998	1.01\\
47.45336025	1.01\\
47.469306925999	1.01\\
47.485421566002	1.01\\
47.501365827999	1.01\\
47.517399979	1.01\\
47.533357956002	1.01\\
47.549450951	1.01\\
47.565358229	1.01\\
47.581391353001	1.01\\
47.597387723	1.01\\
47.613418737	1.01\\
47.629360709	1.01\\
47.645498174	1.01\\
47.661384684002	1.01\\
47.677427744999	1.01\\
47.693363883999	1.01\\
47.709345939003	1.01\\
47.725374209999	1.01\\
47.741466713002	1.01\\
47.757326062001	1.01\\
47.773398068001	1.01\\
47.789340082001	1.01\\
47.805340067002	1.01\\
47.821304175	1.01\\
47.837412484001	1.01\\
47.853398366001	1.02\\
47.869347875	1.02\\
47.885392164002	1.02\\
47.901324800003	1.02\\
47.917444429001	1.02\\
47.933345877999	1.02\\
47.949505640003	1.02\\
47.965483879002	1.02\\
47.981372039002	1.02\\
47.999625078999	1.02\\
48.014910201	1.02\\
48.030238944001	1.02\\
48.045532691002	1.02\\
48.061384705002	1.02\\
48.077484083004	1.02\\
48.093400437001	1.02\\
48.109586562001	1.02\\
48.125377777001	1.02\\
48.141302650002	1.01\\
48.157397268002	1.01\\
48.173443229	1.01\\
48.189419918	1.01\\
48.205363702	1.01\\
48.221392520001	1.01\\
48.237344133999	1.01\\
48.253469437001	1.01\\
48.269402489003	1.01\\
48.285331062001	1.01\\
48.301386203999	1.01\\
48.317418470001	1.01\\
48.333364988003	1.01\\
48.349407306999	1.01\\
48.365392471001	1.01\\
48.381479857998	1.01\\
48.397354743	1.01\\
48.413409119	1.01\\
48.429411412998	1.01\\
48.445561877003	1.01\\
48.461411552002	1.01\\
48.477422628998	1.01\\
48.4933445	1.01\\
48.511755337002	1.01\\
48.526936214001	1.01\\
48.542236933003	1.01\\
48.557541184002	1.01\\
48.575530121002	1.01\\
48.590936178002	1.01\\
48.605917753002	1.01\\
48.621427835999	1.01\\
48.637340672001	1.01\\
48.653349746002	1.01\\
48.669308415001	1\\
48.685464613003	1\\
48.701383292	1\\
48.717402017998	1\\
48.733376368	1\\
48.749460947003	1\\
48.765385098999	1\\
48.781389954003	1\\
48.797375650002	1\\
48.813296828003	1\\
48.829369798001	1\\
48.845332294003	1\\
48.861208877999	1\\
48.877434555001	1\\
48.893386477001	1\\
48.911746728001	1\\
48.926976556	1\\
48.942228453003	1\\
48.957528921002	1\\
48.973333938	1\\
48.989341988999	1\\
49.005299707001	1\\
49.02136125	1\\
49.037326670002	1\\
49.053412538002	1\\
49.069388076	1\\
49.085576291001	1\\
49.101397899002	1\\
49.117387404999	1\\
49.133328276001	1\\
49.149377494	1\\
49.16540975	1\\
49.181400803002	1\\
49.197379848999	1\\
49.213470172001	1\\
49.229406867001	1\\
49.245452303002	1\\
49.261263112999	1\\
49.277506679001	1\\
49.293362916001	1\\
49.311777547001	1\\
49.327063341999	1.01\\
49.342245762001	1.01\\
49.357579125	1.01\\
49.373472606999	1.01\\
49.389288509999	1.01\\
49.405368928002	1.01\\
49.423941738003	1.01\\
49.439219109001	1.01\\
49.454571751	1.01\\
49.470000680001	1.01\\
49.485396363003	1.01\\
49.501335498001	1.01\\
49.517375735001	1.01\\
49.533263728001	1.01\\
49.549462942002	1.01\\
49.565398178002	1.01\\
49.581443775002	1.01\\
49.597269794003	1.01\\
49.613390851998	1.01\\
49.629365324002	1.01\\
49.645475636002	1.01\\
49.661355690003	1.01\\
49.677444106003	1.01\\
49.693408369	1.01\\
49.709500202999	1.01\\
49.725274886002	1.01\\
49.743649318001	1.01\\
49.759067157002	1.01\\
49.774040513001	1.01\\
49.789450788002	1.01\\
49.805402925999	1.01\\
49.821271766999	1.01\\
49.837322224003	1.01\\
49.853455074002	1.02\\
49.869378628998	1.02\\
49.885428779003	1.02\\
49.90133357	1.02\\
49.917424679001	1.02\\
49.933337725003	1.02\\
49.949412933003	1.02\\
49.965366056999	1.02\\
49.981428095001	1.02\\
49.999833555001	1.02\\
50.015117530003	1.02\\
50.030451148999	1.02\\
50.045575498001	1.02\\
50.061514627999	1.02\\
50.07757139	1.02\\
50.093387661999	1.02\\
50.109551542	1.02\\
50.125350438999	1.02\\
50.141370418003	1.01\\
50.157310869	1.01\\
50.173376388001	1.01\\
50.189430244004	1.01\\
50.205386162003	1.01\\
50.221486012001	1.01\\
50.237332322003	1.01\\
50.253333768002	1.01\\
50.269340447003	1.01\\
50.285487051998	1.01\\
50.301286104	1.01\\
50.317569061001	1.01\\
50.333337709999	1.01\\
50.349324531998	1.01\\
50.365326334999	1.01\\
50.381399555001	1.01\\
50.397355393002	1.01\\
50.413387904999	1.01\\
50.429382351002	1.01\\
50.445431525002	1.01\\
50.461373242001	1.01\\
50.477604405003	1.01\\
50.493522178002	1.01\\
50.509476757	1.01\\
50.525339962998	1.01\\
50.541460567002	1.01\\
50.557540742001	1.01\\
50.573361598999	1.01\\
50.589436758	1.01\\
50.605640824002	1.01\\
50.621387739003	1.01\\
50.637350234001	1.01\\
50.653413503998	1.01\\
50.669427765	1\\
50.685417510002	1\\
50.701409204003	1\\
50.717524072998	1\\
50.733374161999	1\\
50.749387465	1\\
50.765334488999	1\\
50.781326345001	1\\
50.797395444001	1\\
50.813394140003	1\\
50.829411351998	1\\
50.845416642002	1\\
50.861369523999	1\\
50.877589776001	1\\
50.893381187001	1\\
50.909611813	1\\
50.926650014	1\\
50.941845745003	1\\
50.957390932999	1\\
50.973408105	1\\
50.989358716	1\\
51.005281842003	1\\
51.02140605	1\\
51.037440380001	1\\
51.053413614003	1\\
51.069348159001	1\\
51.085410749001	1\\
51.101371569001	1\\
51.117375536	1\\
51.133323361	1\\
51.149502111	1\\
51.165495804001	1\\
51.181391602001	1\\
51.197397537998	1\\
51.213365499001	1\\
51.229363171002	1\\
51.245400946003	1\\
51.261360649002	1\\
51.277475803002	1\\
51.293340364003	1\\
51.309584550999	1\\
51.325319979	1\\
51.341435395001	1.01\\
51.357333645001	1.01\\
51.373431258999	1.01\\
51.389339702999	1.01\\
51.405413469002	1.01\\
51.421535001999	1.01\\
51.437397451004	1.01\\
51.453248998001	1.01\\
51.469357367001	1.01\\
51.485366835003	1.01\\
51.501370787998	1.01\\
51.517394253002	1.01\\
51.533347556	1.01\\
51.549373138001	1.01\\
51.565371612	1.01\\
51.581370984001	1.01\\
51.597398881001	1.01\\
51.613415237	1.01\\
51.629423743	1.01\\
51.645537650997	1.01\\
51.661402819001	1.01\\
51.677460192002	1.01\\
51.693381362004	1.01\\
51.709480906002	1.01\\
51.725408810002	1.01\\
51.741595889004	1.01\\
51.757300422001	1.01\\
51.773461746002	1.01\\
51.78986809	1.01\\
51.805417918	1.01\\
51.821316117001	1.01\\
51.83725141	1.01\\
51.853379802002	1.01\\
51.869394097	1.02\\
51.885423579998	1.02\\
51.901403300999	1.02\\
51.917408285	1.02\\
51.933401854	1.02\\
51.949385525002	1.02\\
51.965334668999	1.02\\
51.981454503998	1.02\\
51.999657890999	1.02\\
52.014850834999	1.02\\
52.030114494	1.02\\
52.045371168	1.02\\
52.061435056999	1.02\\
52.077492430001	1.02\\
52.093242997998	1.02\\
52.111573989003	1.02\\
52.126717335999	1.02\\
52.141801569001	1.01\\
52.157372656998	1.01\\
52.173412339001	1.01\\
52.18945325	1.01\\
52.205460457001	1.01\\
52.221359377999	1.01\\
52.237416906998	1.01\\
52.253510925	1.01\\
52.269400938999	1.01\\
52.285449515	1.01\\
52.301341284001	1.01\\
52.317446243	1.01\\
52.333297626	1.01\\
52.349434553997	1.01\\
52.365354471001	1.01\\
52.381376084999	1.01\\
52.397391323002	1.01\\
52.413378909001	1.01\\
52.429338180001	1.01\\
52.445362056999	1.01\\
52.461347077004	1.01\\
52.477496586003	1.01\\
52.493361387001	1.01\\
52.509325512001	1.01\\
52.525391862	1.01\\
52.541363867001	1.01\\
52.557287027001	1.01\\
52.573406952	1.01\\
52.589299275002	1.01\\
52.605428291001	1.01\\
52.621485419003	1.01\\
52.637385550999	1.01\\
52.653371418	1.01\\
52.669369006001	1\\
52.685384476002	1\\
52.701355352001	1\\
52.717484548001	1\\
52.733333975003	1\\
52.749263391003	1\\
52.765300735001	1\\
52.781427948002	1\\
52.797394093998	1\\
52.813416895001	1\\
52.829460100998	1\\
52.845532053002	1\\
52.861331709999	1\\
52.877431246002	1\\
52.893369629002	1\\
52.910780099999	1\\
52.926090890999	1\\
52.941482730999	1\\
52.957363033001	1\\
52.973384876	1\\
52.989358339001	1\\
53.005550016999	1\\
53.021395472	1\\
53.037417498001	1\\
53.053427915001	1\\
53.069290293	1\\
53.085387827	1\\
53.101347288002	1\\
53.117387246998	1\\
53.133286026001	1\\
53.149395457001	1\\
53.165410465	1\\
53.181398917	1\\
53.197334249001	1\\
53.213422734001	1\\
53.229378079998	1\\
53.245502570999	1\\
53.261335716	1\\
53.277474248001	1\\
53.293369229	1\\
53.309539097	1\\
53.325307913002	1\\
53.341444113003	1.01\\
53.357303257	1.01\\
53.373409225998	1.01\\
53.389366102001	1.01\\
53.405440027001	1.01\\
53.421251312001	1.01\\
53.437305087998	1.01\\
53.453328649002	1.01\\
53.469357514	1.01\\
53.485394201	1.01\\
53.501378783001	1.01\\
53.51738434	1.01\\
53.533364768997	1.01\\
53.549424167	1.01\\
53.565373295002	1.01\\
53.581441298001	1.01\\
53.597346450001	1.01\\
53.613467078003	1.01\\
53.629349474999	1.01\\
53.645508904	1.01\\
53.661369287003	1.01\\
53.677368101998	1.01\\
53.693326633999	1.01\\
53.711794452999	1.01\\
53.726983729	1.01\\
53.742195394001	1.01\\
53.757483009003	1.01\\
53.773418843003	1.01\\
53.789367619999	1.01\\
53.805373777001	1.01\\
53.821371375	1.01\\
53.837420535	1.01\\
53.853363258	1.01\\
53.869346143002	1.02\\
53.885354216	1.02\\
53.901330668	1.02\\
53.917416382	1.02\\
53.933328501999	1.02\\
53.949417728001	1.02\\
53.965370869	1.02\\
53.981447361	1.02\\
53.999949185002	1.02\\
54.015259199002	1.02\\
54.030398571999	1.02\\
54.045653837998	1.02\\
54.061313716999	1.02\\
54.077403352001	1.02\\
54.093317752999	1.02\\
54.111663966999	1.02\\
54.126616897	1.02\\
54.141721952	1.01\\
54.157448852001	1.01\\
54.173572861	1.01\\
54.189393769001	1.01\\
54.205328469002	1.01\\
54.221280985001	1.01\\
54.237411393997	1.01\\
54.253433242001	1.01\\
54.269328674	1.01\\
54.285277970001	1.01\\
54.301365824997	1.01\\
54.317304841	1.01\\
54.333356008999	1.01\\
54.349416960999	1.01\\
54.365337859001	1.01\\
54.381441777001	1.01\\
54.397303778004	1.01\\
54.413410249001	1.01\\
54.429404237999	1.01\\
54.445404607998	1.01\\
54.461367453003	1.01\\
54.477587828999	1.01\\
54.493338394001	1.01\\
54.509573925	1.01\\
54.52536748	1.01\\
54.54145416	1.01\\
54.557413146004	1.01\\
54.573290911003	1.01\\
54.589421315998	1.01\\
54.605370279003	1.01\\
54.621456292	1.01\\
54.637413138001	1.01\\
54.653390579998	1.01\\
54.669708950001	1\\
54.685382668999	1\\
54.701397566002	1\\
54.71740682	1\\
54.733235217999	1\\
54.749423745003	1\\
54.765457140003	1\\
54.781387229	1\\
54.797359029	1\\
54.813405141999	1\\
54.829462677002	1\\
54.845333546002	1\\
54.861303272999	1\\
54.879550278	1\\
54.894653944001	1\\
54.909907104	1\\
54.925506165001	1\\
54.941380544003	1\\
54.957369123001	1\\
54.973349445	1\\
54.989313436001	1\\
55.007365945999	1\\
55.022706671002	1\\
55.038086063	1\\
55.05338391	1\\
55.069375494	1\\
55.08541416	1\\
55.101367779	1\\
55.117433957001	1\\
55.13343589	1\\
55.14936857	1\\
55.16541107	1\\
55.181424459	1\\
55.197407020001	1\\
55.213459360001	1\\
55.229391991001	1\\
55.245421656002	1\\
55.261392704003	1\\
55.277454112999	1\\
55.293342092003	1\\
55.311748313	1\\
55.327062707001	1\\
55.342322348	1.01\\
55.357913675003	1.01\\
55.373398186001	1.01\\
55.389362057003	1.01\\
55.405326012001	1.01\\
55.421354279999	1.01\\
55.437374399002	1.01\\
55.453411071003	1.01\\
55.469381242001	1.01\\
55.485388377003	1.01\\
55.501338677002	1.01\\
55.517421194001	1.01\\
55.533333327	1.01\\
55.549337683998	1.01\\
55.565317251999	1.01\\
55.581370966999	1.01\\
55.597355700001	1.01\\
55.613414868	1.01\\
55.629395293	1.01\\
55.645415602001	1.01\\
55.661304623001	1.01\\
55.679626309002	1.01\\
55.694622538002	1.01\\
55.709809208	1.01\\
55.726704665001	1.01\\
55.74182305	1.01\\
55.757380772	1.01\\
55.773457174	1.01\\
55.789383023999	1.01\\
55.805396653	1.01\\
55.821326031002	1.01\\
55.837353466999	1.01\\
55.853337787003	1.01\\
55.869283749001	1.02\\
55.885390620999	1.02\\
55.901335685002	1.02\\
55.917390418	1.02\\
55.933809550999	1.02\\
55.949216156002	1.02\\
55.967510904	1.02\\
55.982802155999	1.02\\
56.000346632	1.02\\
56.015431153	1.02\\
56.030563184998	1.02\\
56.045807199002	1.02\\
56.061421895001	1.02\\
56.077428505001	1.02\\
56.093356158001	1.02\\
56.109547254002	1.02\\
56.125319574002	1.02\\
56.141430436001	1.01\\
56.157365201	1.01\\
56.175299956002	1.01\\
56.190537966	1.01\\
56.205617909001	1.01\\
56.221384895001	1.01\\
56.237259897	1.01\\
56.253380731999	1.01\\
56.26937834	1.01\\
56.285419010002	1.01\\
56.301374809002	1.01\\
56.317392152001	1.01\\
56.333382089001	1.01\\
56.349400056	1.01\\
56.365395790001	1.01\\
56.381443990998	1.01\\
56.397395641003	1.01\\
56.41342675	1.01\\
56.429435562001	1.01\\
56.445443460003	1.01\\
56.461354201	1.01\\
56.477448768002	1.01\\
56.493379710999	1.01\\
56.509675183998	1.01\\
56.526711355999	1.01\\
56.541784974003	1.01\\
56.557407926003	1.01\\
56.573359136002	1.01\\
56.589194312001	1.01\\
56.605350772999	1.01\\
56.621319452999	1.01\\
56.637500032002	1.01\\
56.653385097	1.01\\
56.669243164002	1\\
56.685364501999	1\\
56.701366919998	1\\
56.717408901001	1\\
56.733326942002	1\\
56.749419149002	1\\
56.765357212002	1\\
56.781427294003	1\\
56.797419082001	1\\
56.813369139	1\\
56.829441963002	1\\
56.845484875	1\\
56.861492705002	1\\
56.877569302002	1\\
56.893351681999	1\\
56.909573202	1\\
56.925325549	1\\
56.941383738999	1\\
56.957366464001	1\\
56.973458145001	1\\
56.989422342003	1\\
57.007404869	1\\
57.022801849003	1\\
57.038054027001	1\\
57.053351326	1\\
57.069344793999	1\\
57.085395101002	1\\
57.101301086003	1\\
57.117612305001	1\\
57.133342573002	1\\
57.149428039002	1\\
57.165375366001	1\\
57.181425305001	1\\
57.197360286003	1\\
57.213417703999	1\\
57.229286597	1\\
57.245362778004	1\\
57.261406287003	1\\
57.277319404	1\\
57.293322769001	1\\
57.309685191002	1\\
57.325343941002	1\\
57.341404867001	1.01\\
57.357403498001	1.01\\
57.373416099999	1.01\\
57.389373251	1.01\\
57.405438891003	1.01\\
57.421419898003	1.01\\
57.437296367001	1.01\\
57.453298920998	1.01\\
57.469360022999	1.01\\
57.485422799	1.01\\
57.501267612	1.01\\
57.517416626	1.01\\
57.533359189003	1.01\\
57.549207667004	1.01\\
57.567467605	1.01\\
57.582706313004	1.01\\
57.597955543999	1.01\\
57.613466785	1.01\\
57.629392588002	1.01\\
57.645472327	1.01\\
57.661406054001	1.01\\
57.677467543999	1.01\\
57.693348103001	1.01\\
57.709506206002	1.01\\
57.725276911003	1.01\\
57.741363051998	1.01\\
57.757332065003	1.01\\
57.773382293	1.01\\
57.789374484001	1.01\\
57.805342965	1.01\\
57.821422299	1.01\\
57.837429513001	1.01\\
57.853410383	1.01\\
57.869362553002	1.02\\
57.885226243	1.02\\
57.901362761002	1.02\\
57.917445936001	1.02\\
57.933327888001	1.02\\
57.949426309998	1.02\\
57.965297827999	1.02\\
57.981421370003	1.02\\
57.999762771	1.02\\
58.015062497002	1.02\\
58.030485953003	1.02\\
58.045648834	1.02\\
58.061348161999	1.02\\
58.077507937001	1.02\\
58.093362240002	1.02\\
58.109606482003	1.02\\
58.125291432999	1.01\\
58.141376161999	1.01\\
58.157342404999	1.01\\
58.173386441998	1.01\\
58.189357901001	1.01\\
58.205341892002	1.01\\
58.221459498001	1.01\\
58.237435792004	1.01\\
58.25345984	1.01\\
58.269391997002	1.01\\
58.285485119999	1.01\\
58.301375063004	1.01\\
58.317371904	1.01\\
58.333404441002	1.01\\
58.349379408001	1.01\\
58.365298042	1.01\\
58.381431502999	1.01\\
58.397378496002	1.01\\
58.413347829998	1.01\\
58.429437426998	1.01\\
58.445454334	1.01\\
58.461367430001	1.01\\
58.477559379002	1.01\\
58.493386731999	1.01\\
58.509650832001	1.01\\
58.525401902001	1.01\\
58.541449297001	1.01\\
58.557335422001	1.01\\
58.573371167	1.01\\
58.589305724999	1.01\\
58.605357469002	1.01\\
58.621434992001	1.01\\
58.637353987	1.01\\
58.653438993	1\\
58.669629924	1\\
58.685303313999	1\\
58.701391455002	1\\
58.717428465	1\\
58.733381919998	1\\
58.749435109001	1\\
58.765343270001	1\\
58.78145914	1\\
58.797405622002	1\\
58.813469456002	1\\
58.829425319001	1\\
58.845408599003	1\\
58.861347309002	1\\
58.877479071003	1\\
58.893356201	1\\
58.909643195	1\\
58.925395422001	1\\
58.941394156998	1\\
58.957340149002	1\\
58.973391371002	1\\
58.989182594002	1\\
59.005407040001	1\\
59.021387950001	1\\
59.037373467999	1\\
59.053371710003	1\\
59.069428221001	1\\
59.085388269001	1\\
59.101589723999	1\\
59.117342613003	1\\
59.133275053002	1\\
59.149410528	1\\
59.165411549004	1\\
59.181382335999	1\\
59.197344213002	1\\
59.213405895001	1\\
59.229364330002	1\\
59.245454844002	1\\
59.261390973	1\\
59.277705249001	1\\
59.293404985001	1\\
59.309596834	1\\
59.325299759999	1\\
59.341382220001	1\\
59.357374040001	1.01\\
59.373349934002	1.01\\
59.389342234001	1.01\\
59.405365058998	1.01\\
59.421326063	1.01\\
59.437194869999	1.01\\
59.453368583	1.01\\
59.469345464001	1.01\\
59.485399960999	1.01\\
59.501339086003	1.01\\
59.517490277001	1.01\\
59.533395816002	1.01\\
59.549415655003	1.01\\
59.565473584999	1.01\\
59.581491383	1.01\\
59.597336721001	1.01\\
59.613484770001	1.01\\
59.629390223999	1.01\\
59.645568261002	1.01\\
59.661398570999	1.01\\
59.677528533001	1.01\\
59.69337591	1.01\\
59.711598385998	1.01\\
59.726661	1.01\\
59.741785554001	1.01\\
59.757329939999	1.01\\
59.773334343003	1.01\\
59.789719133004	1.01\\
59.805344097	1.01\\
59.821343679001	1.01\\
59.837442256001	1.01\\
59.853391534001	1.01\\
59.869337706002	1.02\\
59.885416131001	1.02\\
59.901384248001	1.02\\
59.917381100003	1.02\\
59.933348684002	1.02\\
59.949387364998	1.02\\
59.965422393002	1.02\\
59.981372408001	1.02\\
59.999566389	1.02\\
60.014909449002	1.02\\
60.030103600998	1.02\\
60.045279624001	1.02\\
60.061376765	1.02\\
60.077586379002	1.02\\
60.093419444001	1.02\\
60.109373563	1.02\\
60.125423743	1.01\\
60.141319618	1.01\\
60.157281247002	1.01\\
60.173317642998	1.01\\
60.189215845001	1.01\\
60.205414216004	1.01\\
60.22134307	1.01\\
60.237427533001	1.01\\
60.253425221001	1.01\\
60.269346631001	1.01\\
60.285362971001	1.01\\
60.301335648003	1.01\\
60.317504222	1.01\\
60.333434233002	1.01\\
60.349435068001	1.01\\
60.365335195004	1.01\\
60.381471993	1.01\\
60.397175400002	1.01\\
60.413385034001	1.01\\
60.429390630001	1.01\\
60.445371144001	1.01\\
60.461328170002	1.01\\
60.479607504002	1.01\\
60.494748255001	1.01\\
60.509834568001	1.01\\
60.525414558003	1.01\\
60.541370799	1.01\\
60.557361154999	1.01\\
60.573399170002	1.01\\
60.589362785004	1.01\\
60.605459753998	1.01\\
60.621386586003	1.01\\
60.637378501	1.01\\
60.653377543	1\\
60.669353507	1\\
60.685359438	1\\
60.701362504002	1\\
60.717389433003	1\\
60.733738362	1\\
60.749396766003	1\\
60.765370626	1\\
60.781405769001	1\\
60.79738973	1\\
60.813288449002	1\\
60.829382390999	1\\
60.845723226002	1\\
60.861361165001	1\\
60.877550136002	1\\
60.893336711003	1\\
60.911614166001	1\\
60.926710156002	1\\
60.941962780003	1\\
60.957352976002	1\\
60.973336411999	1\\
60.989323501	1\\
61.006898159001	1\\
61.022402399002	1\\
61.037731582001	1\\
61.053414290001	1\\
61.069351998001	1\\
61.085411605999	1\\
61.101364072003	1\\
61.117396373001	1\\
61.133339694001	1\\
61.149294266003	1\\
61.165418488999	1\\
61.181481971001	1\\
61.197343710999	1\\
61.213480561001	1\\
61.229369737	1\\
61.245468950001	1\\
61.261261966	1\\
61.277469834	1\\
61.293372577004	1\\
61.311701078003	1\\
61.326935416001	1\\
61.342141255001	1\\
61.357391938999	1.01\\
61.375765536999	1.01\\
61.390954347	1.01\\
61.406018720001	1.01\\
61.421379209004	1.01\\
61.437355098004	1.01\\
61.453425977001	1.01\\
61.469334693001	1.01\\
61.485491006001	1.01\\
61.501380664002	1.01\\
61.517396248001	1.01\\
61.533358897	1.01\\
61.549380930001	1.01\\
61.565374303997	1.01\\
61.581421108002	1.01\\
61.597353899002	1.01\\
61.613344260002	1.01\\
61.629312311001	1.01\\
61.645430829003	1.01\\
61.661386246002	1.01\\
61.677439038998	1.01\\
61.693260506001	1.01\\
61.709619919998	1.01\\
61.725396141999	1.01\\
61.741305112	1.01\\
61.757365244	1.01\\
61.773382290001	1.01\\
61.789555737	1.01\\
61.805375893002	1.01\\
61.821458265	1.01\\
61.837411490002	1.01\\
61.853334194001	1.01\\
61.869268225003	1.01\\
61.885389019001	1.02\\
61.90129209	1.02\\
61.917383965	1.02\\
61.933329665001	1.02\\
61.949277075001	1.02\\
61.965357839001	1.02\\
61.981383984001	1.02\\
61.999633973	1.02\\
62.015040384999	1.02\\
62.030207555001	1.02\\
62.045477158001	1.02\\
62.061371347	1.02\\
62.077412629002	1.02\\
62.093370191002	1.02\\
62.111841310002	1.01\\
62.126949236	1.01\\
62.142216875	1.01\\
62.157574497002	1.01\\
62.173371757	1.01\\
62.189315566002	1.01\\
62.205413570004	1.01\\
62.221518379002	1.01\\
62.237390877999	1.01\\
62.253364522	1.01\\
62.269347737004	1.01\\
62.285393244004	1.01\\
62.301338042004	1.01\\
62.317381148003	1.01\\
62.333442982998	1.01\\
62.349701576	1.01\\
62.365367277001	1.01\\
62.381428925	1.01\\
62.397410336003	1.01\\
62.413402783001	1.01\\
62.429411570999	1.01\\
62.445502435002	1.01\\
62.461279538002	1.01\\
62.477470129002	1.01\\
62.493456883999	1.01\\
62.509656618	1.01\\
62.525328737999	1.01\\
62.541387108998	1.01\\
62.557307226002	1.01\\
62.573395925999	1.01\\
62.589427335999	1.01\\
62.605369293	1.01\\
62.621487567002	1.01\\
62.637342854	1\\
62.653437085999	1\\
62.669389757	1\\
62.685416426003	1\\
62.701342422001	1\\
62.717446098	1\\
62.733367305001	1\\
62.751782277001	1\\
62.767001977001	1\\
62.782200714001	1\\
62.797504019001	1\\
62.813398460999	1\\
62.829387789002	1\\
62.845490067002	1\\
62.861363875	1\\
62.877433756001	1\\
62.893248585003	1\\
62.909595509999	1\\
62.925307064003	1\\
62.941356605	1\\
62.957267084999	1\\
62.975505842999	1\\
62.990789032002	1\\
63.005431438	1\\
63.021341653	1\\
63.037379272	1\\
63.053398081002	1\\
63.069326419003	1\\
63.085467502003	1\\
63.101375587002	1\\
63.117409457001	1\\
63.133264602001	1\\
63.149359726002	1\\
63.165374938999	1\\
63.181412884003	1\\
63.197383524998	1\\
63.213379093998	1\\
63.229427140999	1\\
63.245476927002	1\\
63.261356808003	1\\
63.277528260002	1\\
63.293330861	1\\
63.309435827	1\\
63.325346153	1\\
63.341374495003	1\\
63.357382947003	1\\
63.373439610001	1.01\\
63.389383482003	1.01\\
63.405315440998	1.01\\
63.421461002999	1.01\\
63.437359551003	1.01\\
63.453402300003	1.01\\
63.469347032997	1.01\\
63.485379806999	1.01\\
63.501281133004	1.01\\
63.517381644001	1.01\\
63.533323149002	1.01\\
63.549443902001	1.01\\
63.565370280003	1.01\\
63.581333491001	1.01\\
63.597365033001	1.01\\
63.613372248001	1.01\\
63.629377483002	1.01\\
63.645321434002	1.01\\
63.661387593003	1.01\\
63.677581453003	1.01\\
63.693384969002	1.01\\
63.711775940998	1.01\\
63.726992208	1.01\\
63.742183317002	1.01\\
63.757514461003	1.01\\
63.773316770001	1.01\\
63.789462770001	1.01\\
63.805366459	1.01\\
63.821425658001	1.01\\
63.837421592003	1.01\\
63.853418128002	1.01\\
63.869357164002	1.01\\
63.885356362	1.01\\
63.901368666001	1.02\\
63.917412904	1.02\\
63.933382448998	1.02\\
63.949544775002	1.02\\
63.965324381001	1.02\\
63.981434557003	1.02\\
63.999626829003	1.02\\
};
\addplot [color=mycolor1,solid,forget plot]
  table[row sep=crcr]{%
63.999626829003	1.02\\
64.014765815998	1.02\\
64.030066880001	1.02\\
64.045489307999	1.02\\
64.061401860001	1.02\\
64.077348879002	1.02\\
64.093338211998	1.02\\
64.111725084999	1.01\\
64.127164043	1.01\\
64.142271563	1.01\\
64.157709525002	1.01\\
64.173440029	1.01\\
64.189384525002	1.01\\
64.205371443001	1.01\\
64.221378506001	1.01\\
64.237364306999	1.01\\
64.253310353001	1.01\\
64.269384508	1.01\\
64.285397354	1.01\\
64.301389251999	1.01\\
64.317509568001	1.01\\
64.333330162003	1.01\\
64.349385243	1.01\\
64.365359819001	1.01\\
64.381351814003	1.01\\
64.397344186001	1.01\\
64.413385066002	1.01\\
64.429429946999	1.01\\
64.445410145001	1.01\\
64.461286152001	1.01\\
64.477487247002	1.01\\
64.493362442002	1.01\\
64.509593609001	1.01\\
64.525320390999	1.01\\
64.541449870003	1.01\\
64.557258077	1.01\\
64.575483622002	1.01\\
64.590683801998	1.01\\
64.605775668	1.01\\
64.621383536999	1.01\\
64.637384177002	1\\
64.653391376999	1\\
64.669376577	1\\
64.685388372002	1\\
64.701363529	1\\
64.717422646	1\\
64.733342814003	1\\
64.749421205002	1\\
64.765361257	1\\
64.781381806999	1\\
64.797363223999	1\\
64.813424004998	1\\
64.829351929001	1\\
64.845329397003	1\\
64.861335414002	1\\
64.877414922001	1\\
64.893367268002	1\\
64.909879578003	1\\
64.925384529	1\\
64.941472422001	1\\
64.957388284001	1\\
64.973456522	1\\
64.989351999001	1\\
65.006902467003	1\\
65.022407548001	1\\
65.037807616001	1\\
65.053420447998	1\\
65.069314803002	1\\
65.085446179001	1\\
65.101327120999	1\\
65.117213196003	1\\
65.135465324002	1\\
65.150738169998	1\\
65.166029742001	1\\
65.181296282002	1\\
65.197375526001	1\\
65.21337998	1\\
65.229315822003	1\\
65.245430064999	1\\
65.261365692002	1\\
65.277469140003	1\\
65.293375707001	1\\
65.311782780999	1\\
65.326825117001	1\\
65.342024548001	1\\
65.35733034	1\\
65.373366067002	1\\
65.389388256001	1.01\\
65.405411672001	1.01\\
65.421323813999	1.01\\
65.437359868	1.01\\
65.453310482003	1.01\\
65.469280680001	1.01\\
65.485253950001	1.01\\
65.501339128002	1.01\\
65.517382884003	1.01\\
65.533332906002	1.01\\
65.549373134999	1.01\\
65.565323406002	1.01\\
65.581399667	1.01\\
65.597347026001	1.01\\
65.613394356999	1.01\\
65.629367179001	1.01\\
65.645489824002	1.01\\
65.661385270001	1.01\\
65.677442855004	1.01\\
65.693292386002	1.01\\
65.709352712002	1.01\\
65.725466258004	1.01\\
65.742127092999	1.01\\
65.757427477001	1.01\\
65.773330723	1.01\\
65.789369097	1.01\\
65.805354511997	1.01\\
65.821390543	1.01\\
65.837416028	1.01\\
65.853268527001	1.01\\
65.869478766999	1.01\\
65.885376985001	1.01\\
65.901329526001	1.02\\
65.917407	1.02\\
65.933377476998	1.02\\
65.949385468003	1.02\\
65.965346012001	1.02\\
65.981347597	1.02\\
65.999603630001	1.02\\
66.014744727001	1.02\\
66.029828409001	1.02\\
66.045396083004	1.02\\
};
\end{axis}
\end{tikzpicture}%
}
      \caption{The steady state orientation of the robot for
        $K_{\omega}^T = K_{\omega, max}^T$}
      \label{fig:13_max_magnified}
    \end{figure}
  \end{minipage}
\end{minipage}
}

\noindent\makebox[\textwidth][c]{%
\begin{minipage}{\linewidth}
  \begin{minipage}{0.45\linewidth}
    \begin{figure}[H]
      \scalebox{0.6}{% This file was created by matlab2tikz.
%
%The latest updates can be retrieved from
%  http://www.mathworks.com/matlabcentral/fileexchange/22022-matlab2tikz-matlab2tikz
%where you can also make suggestions and rate matlab2tikz.
%
\definecolor{mycolor1}{rgb}{0.00000,0.44700,0.74100}%
%
\begin{tikzpicture}

\begin{axis}[%
width=4.133in,
height=3.26in,
at={(0.693in,0.44in)},
scale only axis,
xmin=0,
xmax=40,
xmajorgrids,
xlabel={Time (seconds)},
ymin=0,
ymax=1.5,
ymajorgrids,
ylabel={Distance (meters)},
axis background/.style={fill=white}
]
\addplot [color=mycolor1,solid,forget plot]
  table[row sep=crcr]{%
0	0\\
0.0217409450000003	0.01\\
0.033792352998998	0.02\\
0.0489014580000006	0.03\\
0.0639562689989991	0.04\\
0.079772224998001	0.05\\
0.0957126689989976	0.07\\
0.111700037999999	0.08\\
0.127712689999998	0.09\\
0.143718015999001	0.1\\
0.159674435998999	0.12\\
0.175695530998997	0.13\\
0.191699789999001	0.14\\
0.207984773998	0.16\\
0.223942375999998	0.17\\
0.239810186999999	0.18\\
0.255936399999	0.19\\
0.271804438999999	0.21\\
0.287743208999	0.22\\
0.303842366999	0.23\\
0.319835173999	0.25\\
0.335700957998998	0.26\\
0.351873607999001	0.27\\
0.367856787999	0.28\\
0.383849574000001	0.3\\
0.399845874998998	0.31\\
0.415822118999001	0.32\\
0.431813292999998	0.34\\
0.447735967	0.35\\
0.463632089001	0.36\\
0.479713220998999	0.37\\
0.496025092999	0.39\\
0.512536688999001	0.4\\
0.527990765000001	0.41\\
0.543851838999998	0.43\\
0.559837134	0.44\\
0.575915023998998	0.45\\
0.591754673999999	0.46\\
0.607742785	0.48\\
0.623993175999001	0.49\\
0.639806823998999	0.5\\
0.655811074999001	0.52\\
0.671814993000001	0.53\\
0.687811587998997	0.54\\
0.703840316997997	0.55\\
0.719832000999	0.57\\
0.735794514998999	0.58\\
0.752070452999998	0.59\\
0.767854747998998	0.61\\
0.783832355999	0.62\\
0.799788184	0.63\\
0.815943594999999	0.64\\
0.831838619998001	0.66\\
0.847707610998999	0.67\\
0.863703535	0.68\\
0.879669551998999	0.69\\
0.895688748999998	0.71\\
0.911719160999998	0.72\\
0.927769202998999	0.73\\
0.943837119998001	0.75\\
0.959851741999999	0.76\\
0.975738938999001	0.77\\
0.991704634999998	0.78\\
1.007742484	0.8\\
1.023593245	0.81\\
1.039669651	0.81\\
1.055704668999	0.82\\
1.071690925999	0.83\\
1.087738823001	0.84\\
1.103697291999	0.84\\
1.119703379	0.85\\
1.135821735999	0.86\\
1.151863895	0.87\\
1.167942901998	0.87\\
1.183725021999	0.88\\
1.199694094999	0.89\\
1.215711302998	0.9\\
1.231704981999	0.9\\
1.247748313999	0.91\\
1.263682136999	0.92\\
1.279706960998	0.92\\
1.295837776999	0.93\\
1.311865351999	0.94\\
1.327904128	0.95\\
1.343844497	0.95\\
1.359796937999	0.96\\
1.375706421	0.97\\
1.391686408998	0.98\\
1.407694685999	0.98\\
1.423702791999	0.99\\
1.439780396998	1\\
1.456438506999	1.01\\
1.471830882999	1.01\\
1.487781976	1.02\\
1.503854253	1.03\\
1.519727313	1.04\\
1.535984058999	1.04\\
1.551865939	1.05\\
1.567836119998	1.06\\
1.583726945999	1.07\\
1.599813601	1.07\\
1.615751809998	1.08\\
1.631961211998	1.09\\
1.647857169999	1.09\\
1.663798535999	1.1\\
1.679889514998	1.11\\
1.695919285999	1.12\\
1.711888088999	1.12\\
1.727805549999	1.13\\
1.743836272999	1.14\\
1.759734249999	1.15\\
1.775701082999	1.15\\
1.791811361999	1.16\\
1.807844745999	1.17\\
1.82381444	1.18\\
1.839704914999	1.18\\
1.855841107999	1.19\\
1.871856014999	1.2\\
1.887802100999	1.21\\
1.903753418999	1.21\\
1.919691482	1.22\\
1.93575544	1.23\\
1.951829972999	1.24\\
1.968024772999	1.24\\
1.983730238998	1.25\\
1.999800972	1.26\\
2.020478611999	1.26\\
2.032574732999	1.25\\
2.047808990999	1.24\\
2.063719280999	1.23\\
2.079713961998	1.23\\
2.095706287999	1.22\\
2.111695567	1.21\\
2.127735451	1.2\\
2.143669105	1.19\\
2.159700787999	1.18\\
2.175711350999	1.17\\
2.191715825	1.16\\
2.207714475	1.16\\
2.223715164999	1.15\\
2.239696205999	1.14\\
2.255699772	1.13\\
2.271735791999	1.12\\
2.287733099999	1.11\\
2.303713158	1.1\\
2.319719810999	1.09\\
2.33565175	1.09\\
2.351674985999	1.08\\
2.367946930999	1.07\\
2.384069837999	1.06\\
2.399838272	1.05\\
2.415749297999	1.04\\
2.431858914999	1.03\\
2.447906821999	1.02\\
2.463786677999	1.01\\
2.480206233	1.01\\
2.495865804	1\\
2.511887482	0.99\\
2.527719649999	0.98\\
2.543695704	0.97\\
2.559692179	0.96\\
2.575697840999	0.95\\
2.591713940999	0.94\\
2.607783290999	0.94\\
2.623720096001	0.93\\
2.639750342999	0.92\\
2.655744468	0.91\\
2.671598926999	0.9\\
2.687552465999	0.89\\
2.705621228001	0.88\\
2.721036544999	0.87\\
2.736189399	0.86\\
2.751687565	0.86\\
2.767602123	0.85\\
2.783646452999	0.84\\
2.799696249	0.83\\
2.815645978999	0.82\\
2.831662695	0.81\\
2.847653924	0.8\\
2.863654798999	0.8\\
2.879649755999	0.79\\
2.895647148	0.78\\
2.911800023999	0.77\\
2.927837290999	0.76\\
2.943929473999	0.75\\
2.959822097999	0.74\\
2.975863440999	0.73\\
2.991708648	0.72\\
3.007725078999	0.72\\
3.025111278999	0.72\\
3.040410899999	0.73\\
3.055723790999	0.74\\
3.071841993	0.75\\
3.087841504999	0.76\\
3.103705214	0.77\\
3.119719676001	0.78\\
3.135694813	0.79\\
3.151695757998	0.8\\
3.167703773999	0.81\\
3.183712492999	0.82\\
3.199891671999	0.83\\
3.215885200999	0.84\\
3.231827984999	0.85\\
3.247920769999	0.86\\
3.263969712999	0.87\\
3.279756394	0.88\\
3.295949203	0.89\\
3.311854146	0.9\\
3.327891023	0.91\\
3.343716676	0.92\\
3.359863312	0.93\\
3.375818724	0.94\\
3.391715427998	0.95\\
3.407985358999	0.96\\
3.42369978	0.97\\
3.439714811998	0.98\\
3.455852863	0.99\\
3.471956699999	1\\
3.487770292999	1.01\\
3.503828988999	1.02\\
3.519934190999	1.03\\
3.535913547999	1.04\\
3.551814245998	1.05\\
3.567833543	1.06\\
3.583831688	1.07\\
3.599789394998	1.08\\
3.615704651	1.09\\
3.631815786999	1.1\\
3.647848644999	1.11\\
3.663979920998	1.12\\
3.679698535999	1.13\\
3.695698624999	1.14\\
3.711663373	1.15\\
3.727701427998	1.16\\
3.743725952	1.17\\
3.759699445	1.18\\
3.775685738	1.19\\
3.791716796	1.2\\
3.80772601	1.21\\
3.823844157	1.23\\
3.839883012999	1.24\\
3.855839438	1.25\\
3.871858459999	1.26\\
3.888336355999	1.27\\
3.903714203999	1.28\\
3.919701093	1.29\\
3.935697924	1.3\\
3.951705775	1.31\\
3.967661226999	1.32\\
3.983700175	1.33\\
3.999842415001	1.34\\
4.021004193	1.34\\
4.033038101999	1.33\\
4.048561395	1.32\\
4.063878795999	1.31\\
4.079841038999	1.3\\
4.095694165001	1.29\\
4.111906719999	1.27\\
4.127847905	1.26\\
4.143853084999	1.25\\
4.159851028	1.24\\
4.175721878999	1.23\\
4.191878513	1.22\\
4.207986755999	1.2\\
4.223943245999	1.19\\
4.239726648999	1.18\\
4.255785315	1.17\\
4.271978696999	1.16\\
4.287948230999	1.15\\
4.303832186	1.13\\
4.319825233999	1.12\\
4.335841623999	1.11\\
4.35180731	1.1\\
4.367755075	1.09\\
4.383704790001	1.08\\
4.399721785999	1.07\\
4.415699535	1.05\\
4.431670913	1.04\\
4.447761837	1.03\\
4.466118170999	1.02\\
4.481789378999	1.01\\
4.497157476	0.99\\
4.512424848999	0.98\\
4.527678620999	0.97\\
4.543635125999	0.96\\
4.559810696999	0.95\\
4.575845018	0.94\\
4.591871151	0.93\\
4.607852957999	0.91\\
4.623844566998	0.9\\
4.639781101999	0.89\\
4.655726822	0.88\\
4.672105413999	0.87\\
4.687934649999	0.86\\
4.703839617999	0.84\\
4.719771226	0.83\\
4.735718697	0.82\\
4.751722778999	0.81\\
4.767708854999	0.8\\
4.783709275	0.79\\
4.799752249999	0.78\\
4.815708689999	0.76\\
4.831705266001	0.75\\
4.847684577	0.74\\
4.863704812	0.73\\
4.879699858	0.72\\
4.895689859	0.71\\
4.911729157	0.69\\
4.927810199	0.68\\
4.943832728001	0.67\\
4.959839903	0.66\\
4.975845688999	0.65\\
4.991918993999	0.64\\
5.007806411999	0.62\\
5.025291686999	0.62\\
5.040445518	0.64\\
5.055715200998	0.65\\
5.071673959999	0.66\\
5.087696930999	0.67\\
5.103701574999	0.69\\
5.119691877	0.7\\
5.135833700998	0.71\\
5.151998653	0.73\\
5.167840001999	0.74\\
5.183767009999	0.75\\
5.199833032999	0.76\\
5.215753626	0.78\\
5.231886161	0.79\\
5.247809641	0.8\\
5.263854285	0.82\\
5.279823363	0.83\\
5.295819282999	0.84\\
5.311762973	0.85\\
5.327812879999	0.87\\
5.343750176	0.88\\
5.359853183	0.89\\
5.375915482	0.9\\
5.391819352999	0.92\\
5.407823611	0.93\\
5.423808716	0.94\\
5.439716955999	0.96\\
5.456136855999	0.97\\
5.471786713999	0.98\\
5.487795790999	0.99\\
5.503834476	1.01\\
5.519973454	1.02\\
5.536053294	1.03\\
5.552022662999	1.05\\
5.567821646	1.06\\
5.583842617	1.07\\
5.59983191	1.08\\
5.615695901	1.1\\
5.631707335	1.11\\
5.647702282	1.12\\
5.663607675999	1.14\\
5.679754063	1.15\\
5.69560783	1.16\\
5.711601714	1.17\\
5.727683021	1.19\\
5.743673146	1.2\\
5.759837452001	1.21\\
5.775837379999	1.23\\
5.791838321	1.24\\
5.807845249999	1.25\\
5.823842190999	1.26\\
5.839865728	1.28\\
5.855839071	1.29\\
5.871861399999	1.3\\
5.887759502999	1.31\\
5.903702726998	1.33\\
5.919652541999	1.34\\
5.935987348999	1.35\\
5.951703499	1.37\\
5.967859610999	1.38\\
5.98427019	1.39\\
6.0000699	1.4\\
6.021008075	1.41\\
6.033260475	1.4\\
6.048344165999	1.39\\
6.063750044999	1.37\\
6.079702253999	1.36\\
6.095715230999	1.35\\
6.111703667998	1.34\\
6.127699147	1.32\\
6.143666891999	1.31\\
6.159709582	1.3\\
6.175693304	1.28\\
6.191704587	1.27\\
6.207679217999	1.26\\
6.223719127	1.25\\
6.239671142999	1.23\\
6.255703413	1.22\\
6.271716891999	1.21\\
6.287694474998	1.2\\
6.30372469	1.18\\
6.319746214	1.17\\
6.335719747	1.16\\
6.351741094	1.14\\
6.367844692	1.13\\
6.383814666999	1.12\\
6.399712249	1.11\\
6.415716837998	1.09\\
6.431948530999	1.08\\
6.448975772999	1.07\\
6.464409909	1.05\\
6.479816608	1.04\\
6.495947792998	1.03\\
6.511895056999	1.02\\
6.527808488999	1\\
6.543833388999	0.99\\
6.559857476999	0.98\\
6.575824805999	0.96\\
6.591903552998	0.95\\
6.607698195999	0.94\\
6.623660929999	0.93\\
6.639700729	0.91\\
6.655808717	0.9\\
6.671842116	0.89\\
6.6877346	0.87\\
6.703724997	0.86\\
6.719835962	0.85\\
6.735841218	0.84\\
6.751900830999	0.82\\
6.767819569999	0.81\\
6.783688751999	0.8\\
6.799832470999	0.78\\
6.815897200998	0.77\\
6.831816221	0.76\\
6.847788130999	0.75\\
6.863823745	0.73\\
6.879726539	0.72\\
6.895709837999	0.71\\
6.911714447999	0.7\\
6.928063651999	0.68\\
6.944079766999	0.67\\
6.959799038999	0.66\\
6.975784324	0.64\\
6.991699234999	0.63\\
7.007706089	0.62\\
7.025053039999	0.62\\
7.040236160999	0.63\\
7.05573719	0.64\\
7.071711198999	0.65\\
7.087706561998	0.67\\
7.103717369998	0.68\\
7.119702518999	0.69\\
7.135702401999	0.71\\
7.151815982999	0.72\\
7.167715153	0.73\\
7.183734795998	0.74\\
7.199706093999	0.76\\
7.21570286	0.77\\
7.232001035	0.78\\
7.247916836998	0.8\\
7.263845528999	0.81\\
7.279826152999	0.82\\
7.295846388	0.83\\
7.311833472	0.85\\
7.327929879001	0.86\\
7.344228885	0.87\\
7.360611941	0.89\\
7.376076707	0.9\\
7.391775447999	0.91\\
7.407838113	0.92\\
7.423843704	0.94\\
7.439841682999	0.95\\
7.456209787	0.96\\
7.471821584001	0.98\\
7.487814617	0.99\\
7.503735946	1\\
7.519697953999	1.01\\
7.535709087998	1.03\\
7.551715850999	1.04\\
7.567701280999	1.05\\
7.583706040999	1.06\\
7.599749743	1.08\\
7.615698129999	1.09\\
7.631783486998	1.1\\
7.647748049999	1.12\\
7.663679518	1.13\\
7.679817443999	1.14\\
7.695819521999	1.15\\
7.711882643	1.17\\
7.727818817	1.18\\
7.743878384999	1.19\\
7.759851748	1.21\\
7.776293397999	1.22\\
7.791701849999	1.23\\
7.807716813	1.24\\
7.823698955999	1.26\\
7.839704466	1.27\\
7.855709055	1.28\\
7.871702544	1.3\\
7.887645458	1.31\\
7.903836363999	1.32\\
7.919866631	1.33\\
7.935850327	1.35\\
7.951762532	1.36\\
7.967795208999	1.37\\
7.983833188	1.39\\
7.999813794999	1.4\\
8.020962470999	1.4\\
8.033284518	1.39\\
8.048643344	1.38\\
8.064148078999	1.37\\
8.079848277999	1.35\\
8.095839591999	1.34\\
8.111887404	1.33\\
8.127843458	1.32\\
8.143836971	1.3\\
8.159846847	1.29\\
8.175792125999	1.28\\
8.191692824999	1.26\\
8.207791682	1.25\\
8.223873565999	1.24\\
8.239924183999	1.23\\
8.255865093999	1.21\\
8.271856542998	1.2\\
8.287846883999	1.19\\
8.303790809999	1.17\\
8.319890204	1.16\\
8.335715822999	1.15\\
8.352028108	1.14\\
8.367843292999	1.12\\
8.383911831999	1.11\\
8.399833202	1.1\\
8.415855585999	1.09\\
8.431868563999	1.07\\
8.447704328998	1.06\\
8.463757583	1.05\\
8.479845758	1.03\\
8.495861023999	1.02\\
8.511692176	1.01\\
8.527691428	1\\
8.543817853001	0.98\\
8.559849322999	0.97\\
8.575798956999	0.96\\
8.591786230999	0.94\\
8.60783716	0.93\\
8.623807155	0.92\\
8.639838826	0.91\\
8.655648439999	0.89\\
8.671686611998	0.88\\
8.687698613	0.87\\
8.703617538	0.85\\
8.719741532999	0.84\\
8.735705739	0.83\\
8.751722891	0.82\\
8.767664699	0.8\\
8.783715792	0.79\\
8.799722646999	0.78\\
8.815734517	0.77\\
8.831713116	0.75\\
8.847689065	0.74\\
8.863714290999	0.73\\
8.879709703998	0.71\\
8.895875884	0.7\\
8.911827869	0.69\\
8.927889431999	0.67\\
8.943796403998	0.66\\
8.959842056	0.65\\
8.975953811999	0.64\\
8.991858415001	0.62\\
9.007839023999	0.61\\
9.025711174999	0.61\\
9.041152842999	0.62\\
9.056739731999	0.64\\
9.072280948999	0.65\\
9.087788282999	0.66\\
9.103848497	0.67\\
9.119873756999	0.69\\
9.135810296	0.7\\
9.151830860999	0.71\\
9.168001189999	0.72\\
9.183972232999	0.74\\
9.199773271	0.75\\
9.215748483001	0.76\\
9.231783275	0.78\\
9.247697503999	0.79\\
9.263707223999	0.8\\
9.279843295998	0.81\\
9.295849107999	0.83\\
9.311914604999	0.84\\
9.327870532999	0.85\\
9.343851363	0.87\\
9.359850318998	0.88\\
9.375818315	0.89\\
9.391819962	0.9\\
9.407786149	0.92\\
9.423818826999	0.93\\
9.439809207999	0.94\\
9.456589559998	0.96\\
9.471987513999	0.97\\
9.487840488998	0.98\\
9.503817691	0.99\\
9.519705134	1.01\\
9.535895859999	1.02\\
9.551836544001	1.03\\
9.567840451	1.04\\
9.583875790001	1.06\\
9.599839686	1.07\\
9.615811701	1.08\\
9.631819086	1.1\\
9.647707847	1.11\\
9.663743410999	1.12\\
9.679769713999	1.13\\
9.695714206999	1.15\\
9.711869457	1.16\\
9.728196005999	1.17\\
9.743722386	1.19\\
9.759720941	1.2\\
9.775699516	1.21\\
9.791713325999	1.22\\
9.807776407	1.24\\
9.82384245	1.25\\
9.839846627999	1.26\\
9.856092623	1.28\\
9.871853048999	1.29\\
9.887840284	1.3\\
9.903827086998	1.31\\
9.919807295998	1.33\\
9.935855824	1.34\\
9.9516673	1.35\\
9.967719009999	1.36\\
9.983931716	1.38\\
9.999863278999	1.39\\
10.021012092	1.39\\
10.033078635999	1.38\\
10.048790551	1.37\\
10.064233566999	1.36\\
10.079710209	1.35\\
10.098198176	1.33\\
10.113471091999	1.32\\
10.128731839999	1.31\\
10.143822222	1.3\\
10.159714992	1.28\\
10.17570566	1.27\\
10.191697545	1.26\\
10.207648467	1.24\\
10.223707343999	1.23\\
10.239710583999	1.22\\
10.255654163999	1.21\\
10.271696403999	1.19\\
10.287682610998	1.18\\
10.303710221998	1.17\\
10.319750863	1.15\\
10.335723864999	1.14\\
10.351733719999	1.13\\
10.367835368999	1.12\\
10.383948828999	1.1\\
10.399706107999	1.09\\
10.415709654	1.08\\
10.431821807	1.06\\
10.447833749	1.05\\
10.463716790998	1.04\\
10.479707722999	1.03\\
10.495843894999	1.01\\
10.512155995998	1\\
10.527891841	0.99\\
10.543849964999	0.98\\
10.559767937	0.96\\
10.575748004999	0.95\\
10.591723087	0.94\\
10.607845563	0.92\\
10.623775733	0.91\\
10.639833498	0.9\\
10.655759828999	0.89\\
10.671864153999	0.87\\
10.687703838999	0.86\\
10.7036964	0.85\\
10.719701099	0.83\\
10.735716427	0.82\\
10.751706327999	0.81\\
10.767735567999	0.8\\
10.783702470999	0.78\\
10.799717583999	0.77\\
10.815732495999	0.76\\
10.831655599999	0.74\\
10.847855306	0.73\\
10.863882585999	0.72\\
10.879839802999	0.71\\
10.895933218	0.69\\
10.911709214001	0.68\\
10.927857704	0.67\\
10.943866636999	0.65\\
10.959946856999	0.64\\
10.975931997	0.63\\
10.991818943999	0.62\\
11.007695596	0.6\\
11.023704386	0.61\\
11.039670733998	0.62\\
11.055746021998	0.63\\
11.071765910999	0.64\\
11.087703543	0.66\\
11.103688466999	0.67\\
11.119654578998	0.68\\
11.135701367999	0.7\\
11.151663531	0.71\\
11.167697391	0.72\\
11.183690058	0.73\\
11.199684698998	0.75\\
11.215835934999	0.76\\
11.231961256999	0.77\\
11.247808123999	0.79\\
11.263852191999	0.8\\
11.279940831	0.81\\
11.295648809	0.82\\
11.311709243	0.84\\
11.327845099998	0.85\\
11.343922569	0.86\\
11.359881749999	0.88\\
11.375848968999	0.89\\
11.391826375998	0.9\\
11.407840541	0.91\\
11.423839252	0.93\\
11.439729532	0.94\\
11.455668316999	0.95\\
11.47167273	0.97\\
11.487701218	0.98\\
11.503667048	0.99\\
11.519708229999	1\\
11.535711420999	1.02\\
11.5517164	1.03\\
11.567719962	1.04\\
11.583819436	1.06\\
11.599840707	1.07\\
11.615833137999	1.08\\
11.631804753	1.09\\
11.647637267	1.11\\
11.663696235999	1.12\\
11.679674838999	1.13\\
11.69569185	1.14\\
11.711721766	1.16\\
11.727809474999	1.17\\
11.743710658	1.18\\
11.759717762999	1.2\\
11.775705709001	1.21\\
11.791701848	1.22\\
11.807675521999	1.23\\
11.823694724	1.25\\
11.839808325	1.26\\
11.855845562999	1.27\\
11.871758075	1.29\\
11.887842221998	1.3\\
11.903725135	1.31\\
11.919699185	1.32\\
11.935840353001	1.34\\
11.951729138999	1.35\\
11.967800278998	1.36\\
11.983694392	1.38\\
11.999977492	1.39\\
12.019002953	1.39\\
12.031840967	1.38\\
12.047888510999	1.37\\
12.063749136	1.36\\
12.079843776998	1.35\\
12.095828674999	1.33\\
12.111850778999	1.32\\
12.127720561999	1.31\\
12.143685856999	1.29\\
12.159808834	1.28\\
12.175836369998	1.27\\
12.191746102999	1.26\\
12.207700427999	1.24\\
12.223724891999	1.23\\
12.239666467999	1.22\\
12.255677929999	1.2\\
12.271701464	1.19\\
12.287793426	1.18\\
12.303807591999	1.17\\
12.319799497	1.15\\
12.335871799	1.14\\
12.351949186999	1.13\\
12.367924795998	1.11\\
12.383834709999	1.1\\
12.399772507001	1.09\\
12.415739586	1.08\\
12.431816448999	1.06\\
12.447799019999	1.05\\
12.463838194999	1.04\\
12.479906041999	1.02\\
12.495867934	1.01\\
12.511711347999	1\\
12.527725283	0.99\\
12.543727762	0.97\\
12.559687985	0.96\\
12.575719409	0.95\\
12.591697644	0.94\\
12.607708141	0.92\\
12.623695168999	0.91\\
12.639699653	0.9\\
12.655706562	0.88\\
12.671702566	0.87\\
12.687710217	0.86\\
12.703724252	0.85\\
12.719700867999	0.83\\
12.735759775	0.82\\
12.751811851	0.81\\
12.767962471	0.79\\
12.783848600999	0.78\\
12.799839944999	0.77\\
12.815919493999	0.76\\
12.831760076	0.74\\
12.847839969999	0.73\\
12.863873986	0.72\\
12.879871493	0.7\\
12.895868068999	0.69\\
12.911872162	0.68\\
12.927814553998	0.67\\
12.943832046999	0.65\\
12.959828793	0.64\\
12.975712663999	0.63\\
12.991684228	0.61\\
13.007697244998	0.6\\
13.025210434	0.6\\
13.040539330999	0.61\\
13.055842850999	0.63\\
13.071848671999	0.64\\
13.087879591	0.65\\
13.103962195999	0.66\\
13.119836582	0.68\\
13.135828736	0.69\\
13.151819232	0.7\\
13.167733837999	0.72\\
13.183824886999	0.73\\
13.199719341999	0.74\\
13.215719269999	0.75\\
13.231874699999	0.77\\
13.247852653999	0.78\\
13.263834846999	0.79\\
13.279844231999	0.81\\
13.295870273999	0.82\\
13.311886413999	0.83\\
13.327708384	0.84\\
13.343844826999	0.86\\
13.359876094999	0.87\\
13.375842762999	0.88\\
13.391882686999	0.89\\
13.407746376999	0.91\\
13.423706214	0.92\\
13.439702117999	0.93\\
13.4570759	0.95\\
13.472368425	0.96\\
13.487770082	0.97\\
13.503955205	0.98\\
13.519765899999	1\\
13.535953002	1.01\\
13.551846914999	1.02\\
13.567898489999	1.04\\
13.583837750998	1.05\\
13.599693852999	1.06\\
13.615702250999	1.07\\
13.631704259	1.09\\
13.647672013999	1.1\\
13.663657850999	1.11\\
13.679684575	1.13\\
13.695706913	1.14\\
13.711705544	1.15\\
13.727685091	1.16\\
13.743647544999	1.18\\
13.759837617999	1.19\\
13.775825364	1.2\\
13.79187508	1.22\\
13.807855187999	1.23\\
13.823838097999	1.24\\
13.839711614	1.25\\
13.855531428998	1.27\\
13.873570768999	1.28\\
13.888476393	1.29\\
13.903525382001	1.31\\
13.921634386999	1.32\\
13.936704681	1.33\\
13.952152091999	1.34\\
13.968016672999	1.36\\
13.983852113	1.37\\
13.999565086	1.38\\
14.018192505	1.39\\
14.033614934	1.37\\
14.049050073999	1.36\\
14.064371837999	1.35\\
14.079746784999	1.34\\
14.095704572999	1.33\\
14.111700391999	1.31\\
14.127664283	1.3\\
14.143712846999	1.29\\
14.159731637999	1.27\\
14.175737354	1.26\\
14.191730286999	1.25\\
14.207762787999	1.24\\
14.22373303	1.22\\
14.239737948999	1.21\\
14.255700727999	1.2\\
14.271709340999	1.18\\
14.287696489999	1.17\\
14.303693922998	1.16\\
14.319702325	1.15\\
14.335815184	1.13\\
14.351697095	1.12\\
14.367612803	1.11\\
14.383734657	1.09\\
14.399687477999	1.08\\
14.415781025	1.07\\
14.431824396	1.06\\
14.447696715999	1.04\\
14.463846742999	1.03\\
14.479802756999	1.02\\
14.495824124	1\\
14.511741278998	0.99\\
14.527715578998	0.98\\
14.543704452999	0.97\\
14.559680642	0.95\\
14.575845865	0.94\\
14.591816146	0.93\\
14.607838405	0.91\\
14.623706844999	0.9\\
14.6397297	0.89\\
14.655641125	0.88\\
14.671612453999	0.86\\
14.687614288999	0.85\\
14.70366033	0.84\\
14.719838142999	0.83\\
14.735839945001	0.81\\
14.752268344998	0.8\\
14.767845029	0.79\\
14.783808779999	0.77\\
14.799852722999	0.76\\
14.81596677	0.75\\
14.831806213999	0.73\\
14.847762968999	0.72\\
14.863691741999	0.71\\
14.880070507999	0.7\\
14.895900226	0.68\\
14.911968964999	0.67\\
14.927881850999	0.66\\
14.943831720999	0.65\\
14.959841615998	0.63\\
14.975828168999	0.62\\
14.991807752	0.61\\
15.007815818	0.59\\
15.025599599999	0.59\\
15.041043126999	0.61\\
15.056476415998	0.62\\
15.07180163	0.63\\
15.087848755	0.64\\
15.103849965999	0.66\\
15.119710238	0.67\\
15.135695016999	0.68\\
15.151694959999	0.69\\
15.167707541999	0.71\\
15.183828032999	0.72\\
15.199974748999	0.73\\
15.215948863999	0.75\\
15.231852533999	0.76\\
15.247844679999	0.77\\
15.263848982999	0.78\\
15.279723351999	0.8\\
15.295708625999	0.81\\
15.31177244	0.82\\
15.327816296999	0.84\\
15.343798991	0.85\\
15.359818283999	0.86\\
15.375752438	0.87\\
15.391957511	0.89\\
15.407850035999	0.9\\
15.42379517	0.91\\
15.439965995	0.93\\
15.456163115999	0.94\\
15.471634613	0.95\\
15.487706707999	0.96\\
15.503688349999	0.98\\
15.519700865999	0.99\\
15.535675917999	1\\
15.551700402	1.01\\
15.567702780999	1.03\\
15.583713858999	1.04\\
15.599749883999	1.05\\
15.615712176999	1.07\\
15.63170086	1.08\\
15.647698587999	1.09\\
15.663861179999	1.1\\
15.679888029999	1.12\\
15.695795993999	1.13\\
15.711835832999	1.14\\
15.727823543999	1.16\\
15.743696137999	1.17\\
15.759679533999	1.18\\
15.775861283	1.19\\
15.791854416999	1.21\\
15.807852941999	1.22\\
15.823756481999	1.23\\
15.839899172999	1.25\\
15.855780918999	1.26\\
15.871845989999	1.27\\
15.887899043	1.28\\
15.903815011999	1.3\\
15.919788499999	1.31\\
15.935749734999	1.32\\
15.951766266	1.34\\
15.967843995998	1.35\\
15.983887413	1.36\\
15.999813903999	1.37\\
16.020550563	1.38\\
16.032525057999	1.37\\
16.047830105999	1.36\\
16.063815411	1.34\\
16.079874363	1.33\\
16.095814335	1.32\\
16.111705452	1.3\\
16.127867076999	1.29\\
16.143712980999	1.28\\
16.159695729999	1.27\\
16.175878087	1.25\\
16.191885917998	1.24\\
16.20780591	1.23\\
16.223843995	1.21\\
16.239817734	1.2\\
16.255791653	1.19\\
16.271708683	1.18\\
16.287724477999	1.16\\
16.306561005	1.15\\
16.322026832998	1.14\\
16.337467022	1.12\\
16.352857268999	1.11\\
16.368286783999	1.1\\
16.383775023	1.09\\
16.399796243998	1.07\\
16.415775374999	1.06\\
16.431719537001	1.05\\
16.447767629999	1.04\\
16.463846014999	1.02\\
16.479794011	1.01\\
16.495627403999	1\\
16.511707156	0.98\\
16.52778912	0.97\\
16.543833222999	0.96\\
16.559822227	0.95\\
16.575840461998	0.93\\
16.591931375998	0.92\\
16.607850796999	0.91\\
16.623841257998	0.89\\
16.639757492	0.88\\
16.655718807999	0.87\\
16.671846389999	0.86\\
16.687793733999	0.84\\
16.703721513999	0.83\\
16.719674039999	0.82\\
16.735727839999	0.8\\
16.751844151998	0.79\\
16.767821162999	0.78\\
16.783679586	0.77\\
16.799673129999	0.75\\
16.816089370999	0.74\\
16.831959785999	0.73\\
16.847833497999	0.71\\
16.863823067999	0.7\\
16.879740373999	0.69\\
16.895808166	0.68\\
16.911854148998	0.66\\
16.927857163999	0.65\\
16.943853253	0.64\\
16.959838192999	0.62\\
16.975839825	0.61\\
16.991817909	0.6\\
17.007823263	0.59\\
17.025449778999	0.59\\
17.040897259999	0.6\\
17.056599002999	0.61\\
17.072063542	0.62\\
17.088059136	0.64\\
17.103846913	0.65\\
17.119845989999	0.66\\
17.135844164	0.67\\
17.151871664999	0.69\\
17.167834865	0.7\\
17.183782688	0.71\\
17.199835913999	0.73\\
17.215833681999	0.74\\
17.231751428998	0.75\\
17.247726184999	0.76\\
17.263591098999	0.78\\
17.279863507999	0.79\\
17.295829736	0.8\\
17.311731162999	0.81\\
17.328075417999	0.83\\
17.343683061999	0.84\\
17.359852018999	0.85\\
17.375907773999	0.87\\
17.391718629	0.88\\
17.407843134	0.89\\
17.423903419	0.9\\
17.439856215	0.92\\
17.455841760999	0.93\\
17.471847938999	0.94\\
17.48784057	0.96\\
17.503942133999	0.97\\
17.519930971999	0.98\\
17.535830722999	0.99\\
17.551993134999	1.01\\
17.567836492999	1.02\\
17.583975366	1.03\\
17.599884412999	1.05\\
17.615833812	1.06\\
17.631800544999	1.07\\
17.647711471	1.08\\
17.663572113	1.1\\
17.681651171999	1.11\\
17.696441011	1.12\\
17.712684207999	1.14\\
17.727973322	1.15\\
17.74362365	1.16\\
17.759709505999	1.17\\
17.775685028998	1.19\\
17.791698351999	1.2\\
17.807710739	1.21\\
17.823701751999	1.22\\
17.839717705999	1.24\\
17.855688768999	1.25\\
17.871699006	1.26\\
17.887712824999	1.28\\
17.903693119	1.29\\
17.919735698998	1.3\\
17.935740998999	1.31\\
17.951849573999	1.33\\
17.967855114999	1.34\\
17.98378074	1.35\\
17.999824170999	1.37\\
18.018457483	1.37\\
18.033933474	1.36\\
18.049291592	1.35\\
18.064836521	1.33\\
18.080193775999	1.32\\
18.095889455	1.31\\
18.111830234	1.3\\
18.127921148	1.29\\
18.143760543	1.27\\
18.159710357	1.26\\
18.175702627999	1.25\\
18.191699764998	1.23\\
18.207722670998	1.22\\
18.223962462	1.21\\
18.239850100999	1.2\\
18.255775013999	1.18\\
18.271776806999	1.17\\
18.287816290001	1.16\\
18.30390298	1.15\\
18.319928861999	1.13\\
18.335905913	1.12\\
18.351799365	1.11\\
18.367851019998	1.1\\
18.383849345	1.08\\
18.399693945999	1.07\\
18.415873918	1.06\\
18.431965648998	1.04\\
18.447694896	1.03\\
18.463715660998	1.02\\
18.479857208	1.01\\
18.495689396	0.99\\
18.511848304999	0.98\\
18.527829009999	0.97\\
18.543713936998	0.96\\
18.559688908999	0.94\\
18.575842284999	0.93\\
18.591714948999	0.92\\
18.607665701	0.91\\
18.625990753999	0.89\\
18.641063420998	0.88\\
18.656130920999	0.87\\
18.671696009	0.85\\
18.687820092998	0.84\\
18.703845193998	0.83\\
18.719768788	0.82\\
18.735745544	0.8\\
18.751814158	0.79\\
18.767831916999	0.78\\
18.783791833999	0.77\\
18.799701848999	0.75\\
18.815890565	0.74\\
18.831854091	0.73\\
18.847911679	0.72\\
18.863840280999	0.7\\
18.879830171999	0.69\\
18.895847493998	0.68\\
18.91198323	0.66\\
18.927852867999	0.65\\
18.943847894999	0.64\\
18.959826004	0.63\\
18.975888278999	0.61\\
18.991839512	0.6\\
19.007838146999	0.59\\
19.025620398999	0.59\\
19.041004342	0.6\\
19.056592887	0.61\\
19.072037607999	0.63\\
19.087898415998	0.64\\
19.103849899	0.65\\
19.119822585	0.66\\
19.135825560999	0.68\\
19.153376163	0.69\\
19.168496592999	0.7\\
19.183791076999	0.72\\
19.199797866	0.73\\
19.215699752	0.74\\
19.231654521	0.75\\
19.24770253	0.77\\
19.263665585	0.78\\
19.279699339999	0.79\\
19.295710292999	0.8\\
19.311664858	0.82\\
19.327832601999	0.83\\
19.343890546999	0.84\\
19.359882372001	0.86\\
19.375718072999	0.87\\
19.391724229999	0.88\\
19.407839323999	0.89\\
19.423845551999	0.91\\
19.439800578	0.92\\
19.456037834	0.93\\
19.471664115999	0.95\\
19.487650472999	0.96\\
19.503984293	0.97\\
19.519896824	0.98\\
19.535707973999	1\\
19.551697509	1.01\\
19.567703179	1.02\\
19.583659427998	1.04\\
19.599669817	1.05\\
19.615714879999	1.06\\
19.631721132998	1.07\\
19.647669205	1.09\\
19.663724864	1.1\\
19.679837146999	1.11\\
19.695912440999	1.13\\
19.711841618998	1.14\\
19.727909510999	1.15\\
19.743707210998	1.16\\
19.759738104999	1.18\\
19.775837512999	1.19\\
19.791846532	1.2\\
19.807723045	1.21\\
19.823720738999	1.23\\
19.839728589999	1.24\\
19.855701075999	1.25\\
19.871712746	1.27\\
19.887702084999	1.28\\
19.903701265999	1.29\\
19.919720472999	1.3\\
19.935692387999	1.32\\
19.951706849	1.33\\
19.967703089	1.34\\
19.983849078	1.36\\
19.999800863	1.37\\
20.017988118999	1.37\\
20.032994263999	1.36\\
20.048302575	1.35\\
20.063816963	1.34\\
20.079805595	1.33\\
20.095819820999	1.31\\
20.111968866999	1.3\\
20.127919713999	1.29\\
20.143875702999	1.27\\
20.159860185999	1.26\\
20.176089458999	1.25\\
20.19174421	1.24\\
20.207796250999	1.22\\
20.223965148	1.21\\
20.239721418999	1.2\\
20.255775649999	1.19\\
20.271738399	1.17\\
20.287685938	1.16\\
20.303850655	1.15\\
20.319838875999	1.14\\
20.335855132001	1.12\\
20.351824338	1.11\\
20.367822951999	1.1\\
20.383812855999	1.08\\
20.399849923999	1.07\\
20.4158467	1.06\\
20.431933228999	1.05\\
20.447762185	1.03\\
20.463846523998	1.02\\
20.479897194	1.01\\
20.495817850999	1\\
20.511881288999	0.98\\
20.527960075	0.97\\
20.543743573001	0.96\\
20.559836647999	0.95\\
20.575849846	0.93\\
20.591821148999	0.92\\
20.607653477	0.91\\
20.623853674	0.89\\
20.639717173999	0.88\\
20.655769334	0.87\\
20.671683603999	0.86\\
20.687682750998	0.84\\
20.703640806	0.83\\
20.719633834	0.82\\
20.735710187999	0.81\\
20.751731705	0.79\\
20.767835569	0.78\\
20.783773361999	0.77\\
20.799705035001	0.76\\
20.815664988	0.74\\
20.831815902999	0.73\\
20.847846315	0.72\\
20.863810748	0.71\\
20.879835433	0.69\\
20.895829512999	0.68\\
20.911842429999	0.67\\
20.927835386999	0.65\\
20.943768214	0.64\\
20.959866855999	0.63\\
20.976041966	0.62\\
20.992244342998	0.6\\
21.007852743999	0.59\\
21.025646788999	0.59\\
21.041070255	0.6\\
21.056749411	0.62\\
21.072172507001	0.63\\
21.087952050999	0.64\\
21.103847147	0.65\\
21.119854921999	0.67\\
21.135949094999	0.68\\
21.151714245999	0.69\\
21.167686054999	0.7\\
21.183840438	0.72\\
21.199845778999	0.73\\
21.215825240999	0.74\\
21.231832054	0.76\\
21.247926636999	0.77\\
21.263824751999	0.78\\
21.279807228	0.79\\
21.295778873	0.81\\
21.311807836	0.82\\
21.327844939	0.83\\
21.343991505	0.85\\
21.359836837998	0.86\\
21.375817492001	0.87\\
21.391835301	0.88\\
21.407874948999	0.9\\
21.423857497999	0.91\\
21.439840838	0.92\\
21.455966916999	0.94\\
21.471860049	0.95\\
21.490299986999	0.96\\
21.505648578999	0.98\\
21.520988266	0.99\\
21.536370163999	1\\
21.551930449999	1.01\\
21.567813575001	1.02\\
21.583846170998	1.04\\
21.599796728	1.05\\
21.615919608999	1.06\\
21.631848281	1.08\\
21.647748709	1.09\\
21.663981094	1.1\\
21.679890997999	1.11\\
21.695753243999	1.13\\
21.711807244998	1.14\\
21.727822052	1.15\\
21.743951483999	1.17\\
21.759787736	1.18\\
21.775829443999	1.19\\
21.791887243	1.2\\
21.807700198001	1.22\\
21.823671692	1.23\\
21.839951924999	1.24\\
21.855934592	1.26\\
21.871842222999	1.27\\
21.887701456999	1.28\\
21.903709474	1.29\\
21.919936807999	1.31\\
21.935833997999	1.32\\
21.951823919	1.33\\
21.967774549999	1.35\\
21.983732196999	1.36\\
21.999743686999	1.37\\
22.017652707	1.38\\
22.033164644999	1.36\\
22.048447853	1.35\\
22.063960591999	1.34\\
22.079832985999	1.33\\
22.095976695	1.31\\
22.111803742	1.3\\
22.127825866	1.29\\
22.143816387999	1.28\\
22.15988915	1.26\\
22.175802222	1.25\\
22.191810889999	1.24\\
22.207800719999	1.22\\
22.223698402	1.21\\
22.239723732999	1.2\\
22.255838109999	1.19\\
22.271783196	1.17\\
22.287862639999	1.16\\
22.303844559999	1.15\\
22.319829388999	1.13\\
22.335850389999	1.12\\
22.351691288	1.11\\
22.367699384	1.1\\
22.383676540999	1.08\\
22.399716980999	1.07\\
22.415701926	1.06\\
22.431669701	1.05\\
22.447768975999	1.03\\
22.463790484	1.02\\
22.479783203	1.01\\
22.495773368998	0.99\\
22.511840629999	0.98\\
22.527720841999	0.97\\
22.543662988999	0.96\\
22.559663078	0.94\\
22.575653324	0.93\\
22.591653393	0.92\\
22.607915665999	0.9\\
22.623881431	0.89\\
22.639837163999	0.88\\
22.655837357999	0.87\\
22.671662364	0.85\\
22.687695538999	0.84\\
22.703720412999	0.83\\
22.719890171999	0.81\\
22.735882167001	0.8\\
22.751849435999	0.79\\
22.767842232999	0.78\\
22.783732918999	0.76\\
22.799693052	0.75\\
22.815671355999	0.74\\
22.831815616	0.72\\
22.847816129999	0.71\\
22.863846195	0.7\\
22.880023432999	0.69\\
22.896149978	0.67\\
22.911890894	0.66\\
22.927800416999	0.65\\
22.943787221	0.63\\
22.959943754	0.62\\
22.975837933	0.61\\
22.991833466	0.6\\
23.007964259999	0.58\\
23.023745969	0.59\\
23.039735379	0.6\\
23.055755055999	0.61\\
23.071629629999	0.62\\
23.087683009998	0.64\\
23.103702102	0.65\\
23.119698056	0.66\\
23.135697788999	0.68\\
23.151702031999	0.69\\
23.167727323001	0.7\\
23.183697107999	0.71\\
23.199710686	0.73\\
23.215690539	0.74\\
23.231724825	0.75\\
23.247712139998	0.77\\
23.263779705999	0.78\\
23.279738040999	0.79\\
23.295696535	0.8\\
23.311703358998	0.82\\
23.327720260999	0.83\\
23.343746183	0.84\\
23.359832481999	0.86\\
23.375855677999	0.87\\
23.391729318998	0.88\\
23.407707563	0.89\\
23.423708695	0.91\\
23.439732813	0.92\\
23.455820618999	0.93\\
23.471678780999	0.94\\
23.487701641999	0.96\\
23.503707820001	0.97\\
23.519698967998	0.98\\
23.535673874	1\\
23.552051458999	1.01\\
23.567865165001	1.02\\
23.583847127	1.04\\
23.599763290001	1.05\\
23.615811191999	1.06\\
23.631699438	1.07\\
23.647981635999	1.09\\
23.663613342	1.1\\
23.679733981998	1.11\\
23.695719113999	1.12\\
23.711620252999	1.14\\
23.727610790001	1.15\\
23.743680655	1.16\\
23.759719570998	1.18\\
23.775813571999	1.19\\
23.794208101999	1.2\\
23.809284347	1.22\\
23.824334802999	1.23\\
23.839683406	1.24\\
23.855699751	1.25\\
23.871717540001	1.27\\
23.887680766	1.28\\
23.903697691999	1.29\\
23.919707407	1.3\\
23.935696741	1.32\\
23.951973497	1.33\\
23.967893077999	1.34\\
23.983912890999	1.36\\
23.999846238999	1.37\\
24.018070557999	1.37\\
24.033552497	1.36\\
24.048749956999	1.35\\
24.063798748999	1.34\\
24.0797049	1.33\\
24.095722161999	1.31\\
24.111655941998	1.3\\
24.12783044	1.29\\
24.143825371999	1.27\\
24.159703339999	1.26\\
24.17582414	1.25\\
24.191839503999	1.24\\
24.207753421999	1.22\\
24.223705502	1.21\\
24.239802926001	1.2\\
24.255734889999	1.19\\
24.272020776999	1.17\\
24.287923559999	1.16\\
24.303778182999	1.15\\
24.319919873	1.14\\
24.335869852999	1.12\\
24.351820118999	1.11\\
24.367821580999	1.1\\
24.383779132998	1.08\\
24.399843155999	1.07\\
24.415816404	1.06\\
24.431859421999	1.05\\
24.44775453	1.03\\
24.46373091	1.02\\
24.479706908999	1.01\\
24.495689928998	1\\
24.511693153999	0.98\\
24.527545592998	0.97\\
24.543742471	0.96\\
24.559609039999	0.95\\
24.575586565999	0.93\\
24.591721425	0.92\\
24.607575893999	0.91\\
24.623722860999	0.89\\
24.639836676999	0.88\\
24.655701809999	0.87\\
24.671651163999	0.86\\
24.687978705	0.84\\
24.704122393999	0.83\\
24.719850044999	0.82\\
24.735706527999	0.81\\
24.751811127998	0.79\\
24.767844846999	0.78\\
24.78384851	0.77\\
24.799833222999	0.76\\
24.815991013999	0.74\\
24.831751484999	0.73\\
24.847671021	0.72\\
24.863657263	0.71\\
24.879716125	0.69\\
24.895666685999	0.68\\
24.911702686999	0.67\\
24.927685665999	0.65\\
24.943665313	0.64\\
24.959834926	0.63\\
24.975762996999	0.62\\
24.991920016	0.6\\
25.007751031	0.59\\
25.023749851	0.59\\
25.039836688	0.61\\
25.056054799999	0.62\\
25.071708743998	0.63\\
25.087726312	0.64\\
25.103678301999	0.66\\
25.119699106999	0.67\\
25.135666622	0.68\\
25.151684782	0.7\\
25.167707000999	0.71\\
25.183706396999	0.72\\
25.199719741	0.73\\
25.215739133999	0.75\\
25.231692946999	0.76\\
25.247704592	0.77\\
25.263696021	0.79\\
25.279729099	0.8\\
25.295734013999	0.81\\
25.311712898999	0.82\\
25.327705986999	0.84\\
25.343687894999	0.85\\
25.359671484999	0.86\\
25.375852393999	0.88\\
25.391763205	0.89\\
25.407701273	0.9\\
25.423699738	0.91\\
25.439705384998	0.93\\
25.455606413999	0.94\\
25.471672985999	0.95\\
25.487708786999	0.97\\
25.503695495999	0.98\\
25.519740165001	0.99\\
25.535701455999	1\\
25.551793361999	1.02\\
25.567871473	1.03\\
25.583950229999	1.04\\
25.599704601	1.05\\
25.615704775	1.07\\
25.631806278999	1.08\\
25.647769354	1.09\\
25.663783533998	1.11\\
25.679826138999	1.12\\
25.695801197999	1.13\\
25.711866947999	1.14\\
25.727903429	1.16\\
25.743880926999	1.17\\
25.759779029	1.18\\
25.775816301999	1.2\\
25.791851915999	1.21\\
25.807859802998	1.22\\
25.823864787	1.23\\
25.839845151	1.25\\
25.85586292	1.26\\
25.871853852999	1.27\\
25.88805403	1.29\\
25.903811217	1.3\\
25.919942438001	1.31\\
25.935704762999	1.32\\
25.951653344998	1.34\\
25.967859038999	1.35\\
25.983854988999	1.36\\
25.999853663999	1.38\\
26.017757507999	1.38\\
26.033103733	1.37\\
26.049037291999	1.36\\
26.064586708999	1.34\\
26.080000589	1.33\\
26.095838730999	1.32\\
26.111730282999	1.31\\
26.127710291999	1.29\\
26.143877875	1.28\\
26.160000721	1.27\\
26.175865103	1.25\\
26.191833097999	1.24\\
26.207849995	1.23\\
26.223767115	1.22\\
26.239872587999	1.2\\
26.255879030999	1.19\\
26.271749272	1.18\\
26.287831038999	1.17\\
26.303816634999	1.15\\
26.319708166999	1.14\\
26.335851851	1.13\\
26.351874601	1.11\\
26.367837349998	1.1\\
26.383915841999	1.09\\
26.399731577	1.08\\
26.415668948999	1.06\\
26.431900341	1.05\\
26.447672455	1.04\\
26.46380388	1.02\\
26.479847618	1.01\\
26.495870585998	1\\
26.511838382999	0.99\\
26.527697010999	0.97\\
26.543949536999	0.96\\
26.559886943	0.95\\
26.575831295	0.93\\
26.591801216	0.92\\
26.607848651999	0.91\\
26.623712786999	0.9\\
26.639728931999	0.88\\
26.655722161	0.87\\
26.671760904999	0.86\\
26.687663556998	0.84\\
26.705777625	0.83\\
26.72071319	0.82\\
26.735880771999	0.81\\
26.751702154999	0.79\\
26.767687034	0.78\\
26.783706622	0.77\\
26.799842573998	0.76\\
26.815859038999	0.74\\
26.83177178	0.73\\
26.847824316999	0.72\\
26.863700334999	0.7\\
26.879833795999	0.69\\
26.895901301999	0.68\\
26.911912145	0.67\\
26.927851745998	0.65\\
26.943879311998	0.64\\
26.959837730999	0.63\\
26.975814833	0.61\\
26.991829348999	0.6\\
27.007897010999	0.59\\
27.025581842999	0.59\\
27.040759673	0.6\\
27.055958229999	0.61\\
27.071730170999	0.62\\
27.087705173	0.64\\
27.103691631	0.65\\
27.119696405	0.66\\
27.135702276998	0.68\\
27.151704696	0.69\\
27.167716827	0.7\\
27.183694955	0.71\\
27.199702701999	0.73\\
27.215728322999	0.74\\
27.231676915001	0.75\\
27.247707136999	0.76\\
27.263702524999	0.78\\
27.279726466999	0.79\\
27.295732499	0.8\\
27.311739492999	0.82\\
27.327678387999	0.83\\
27.343740162	0.84\\
27.359757258999	0.85\\
27.375729473999	0.87\\
27.391853187999	0.88\\
27.407861059	0.89\\
27.423756148001	0.91\\
27.439813388	0.92\\
27.455788895	0.93\\
27.471828067999	0.94\\
27.487725285998	0.96\\
27.503835731999	0.97\\
27.520202591999	0.98\\
27.536179564	1\\
27.551667691999	1.01\\
27.567717734	1.02\\
27.583651330998	1.03\\
27.599654636999	1.05\\
27.6157757	1.06\\
27.63180261	1.07\\
27.647790549999	1.09\\
27.663747646	1.1\\
27.679784408999	1.11\\
27.695722320999	1.12\\
27.713899894999	1.14\\
27.728958854999	1.15\\
27.744092023998	1.16\\
27.759679060999	1.18\\
27.775697477	1.19\\
27.791845411	1.2\\
27.807845283999	1.21\\
27.82372291	1.23\\
27.839698912999	1.24\\
27.855704087998	1.25\\
27.871704678998	1.27\\
27.887855638999	1.28\\
27.903878238999	1.29\\
27.920060084	1.3\\
27.936004476	1.32\\
27.951846918	1.33\\
27.967762594998	1.34\\
27.983846148	1.35\\
27.999915012	1.37\\
28.017654875999	1.37\\
28.032722684	1.36\\
28.047723505999	1.35\\
28.063752668	1.34\\
28.079710408999	1.32\\
28.095763601999	1.31\\
28.111947539999	1.3\\
28.127659046999	1.29\\
28.143693611	1.27\\
28.159772592998	1.26\\
28.175751158	1.25\\
28.191693301	1.24\\
28.207697693998	1.22\\
28.223712756	1.21\\
28.239688479999	1.2\\
28.255696291	1.19\\
28.271717287	1.17\\
28.287847695999	1.16\\
28.303837665998	1.15\\
28.319859584999	1.13\\
28.335793176999	1.12\\
28.351769732	1.11\\
28.36787946	1.1\\
28.383786566999	1.08\\
28.399925710998	1.07\\
28.415841414	1.06\\
28.431792747	1.05\\
28.447727403	1.03\\
28.463820099999	1.02\\
28.479840224999	1.01\\
28.495768455	1\\
28.511750242	0.98\\
28.527708846999	0.97\\
28.543843776	0.96\\
28.559833510999	0.94\\
28.575925568001	0.93\\
28.591711637	0.92\\
28.607703506999	0.91\\
28.623706557	0.89\\
28.639704285	0.88\\
28.655697924	0.87\\
28.671668571	0.86\\
28.68765969	0.84\\
28.703807262	0.83\\
28.719805897999	0.82\\
28.735837440999	0.81\\
28.751815636999	0.79\\
28.768248996999	0.78\\
28.783814605999	0.77\\
28.799868697999	0.76\\
28.815960698999	0.74\\
28.831815948999	0.73\\
28.847704809999	0.72\\
28.863702307999	0.7\\
28.879665098999	0.69\\
28.895697709999	0.68\\
28.911705480999	0.67\\
28.927675320999	0.65\\
28.943866033	0.64\\
28.959726257999	0.63\\
28.975627361998	0.62\\
28.991716713999	0.6\\
29.007767892999	0.59\\
29.025093806	0.59\\
29.040243066	0.6\\
29.055679763999	0.61\\
29.071706090999	0.63\\
29.087828260999	0.64\\
29.103865424	0.65\\
29.119761095999	0.67\\
29.135864698999	0.68\\
29.151871623	0.69\\
29.167702874	0.7\\
29.183916095999	0.72\\
29.199851968999	0.73\\
29.215803266999	0.74\\
29.23185421	0.76\\
29.247856318999	0.77\\
29.264008963	0.78\\
29.279864058	0.79\\
29.29580971	0.81\\
29.311853450999	0.82\\
29.327844644	0.83\\
29.343753314999	0.85\\
29.359749513999	0.86\\
29.375808143999	0.87\\
29.391777507001	0.88\\
29.407804549999	0.9\\
29.423842874999	0.91\\
29.439829965	0.92\\
29.455883875	0.94\\
29.471845406999	0.95\\
29.48774196	0.96\\
29.503704253	0.97\\
29.519752901998	0.99\\
29.535696126999	1\\
29.551692818001	1.01\\
29.56769665	1.02\\
29.583680230999	1.04\\
29.599715854999	1.05\\
29.615841101998	1.06\\
29.631883717	1.08\\
29.647813896999	1.09\\
29.663570108	1.1\\
29.681733532	1.12\\
29.696744121	1.13\\
29.711782663	1.14\\
29.727686488	1.15\\
29.743702435	1.17\\
29.759662741	1.18\\
29.775763546	1.19\\
29.791697934999	1.2\\
29.807702418	1.22\\
29.823670333999	1.23\\
29.839621601999	1.24\\
29.855625995	1.25\\
29.871677013999	1.27\\
29.887686254	1.28\\
29.903710843	1.29\\
29.919707769999	1.31\\
29.935699865	1.32\\
29.951648733	1.33\\
29.967857239	1.34\\
29.983873894	1.36\\
29.999821268999	1.37\\
30.017228687	1.38\\
30.032289497	1.36\\
30.047735229	1.35\\
30.063835248999	1.34\\
30.079806738	1.33\\
30.095714026	1.31\\
30.111694915998	1.3\\
30.127711926	1.29\\
30.143828755999	1.28\\
30.159835603999	1.26\\
30.175715666999	1.25\\
30.191880488999	1.24\\
30.207944446999	1.22\\
30.223987248	1.21\\
30.239842743	1.2\\
30.255934886	1.19\\
30.272038252998	1.17\\
30.287754832999	1.16\\
30.303794858001	1.15\\
30.319835188	1.13\\
30.335784581999	1.12\\
30.351706948998	1.11\\
30.367805375999	1.1\\
30.383877327	1.08\\
30.399804880999	1.07\\
30.415874787999	1.06\\
30.431813567999	1.04\\
30.447814488	1.03\\
30.463669789999	1.02\\
30.47984536	1.01\\
30.495857917	0.99\\
30.511806929998	0.98\\
30.527718657999	0.97\\
30.543691339001	0.95\\
30.559701219999	0.94\\
30.575680390999	0.93\\
30.59169133	0.92\\
30.607700528999	0.9\\
30.623680657	0.89\\
30.639637837999	0.88\\
30.655696701	0.87\\
30.67182066	0.85\\
30.687717888	0.84\\
30.703763938	0.83\\
30.719788877999	0.81\\
30.735731486	0.8\\
30.751710205999	0.79\\
30.767704402	0.78\\
30.783730007999	0.76\\
30.7997029	0.75\\
30.815686034999	0.74\\
30.831714903999	0.72\\
30.847662262999	0.71\\
30.863707985	0.7\\
30.879717636001	0.69\\
30.895843499	0.67\\
30.911895195001	0.66\\
30.927772985998	0.65\\
30.943709686	0.63\\
30.959694625	0.62\\
30.975755702001	0.61\\
30.991685811999	0.6\\
31.007825611999	0.58\\
31.025211196	0.58\\
31.040322684999	0.59\\
31.055684142	0.61\\
31.071663938999	0.62\\
31.087702886	0.63\\
31.103697286999	0.64\\
31.119643888999	0.66\\
31.137933295998	0.67\\
31.153001057	0.68\\
31.168094851	0.7\\
31.183680561	0.71\\
31.199828204	0.72\\
31.215704280999	0.73\\
31.231701809	0.75\\
31.247689383999	0.76\\
31.263697390999	0.77\\
31.279713986999	0.79\\
31.295914549999	0.8\\
31.311700819	0.81\\
31.327867103999	0.82\\
31.343829797999	0.84\\
31.359835915001	0.85\\
31.375949405999	0.86\\
31.391941440999	0.88\\
31.407834878	0.89\\
31.423717399999	0.9\\
31.439701698001	0.91\\
31.455741311	0.93\\
31.471844457999	0.94\\
31.487836668999	0.95\\
31.503834658	0.97\\
31.519884298999	0.98\\
31.535857091	0.99\\
31.552011815	1\\
31.567853965	1.02\\
31.583880423999	1.03\\
31.599737368998	1.04\\
31.615873960999	1.06\\
31.631823214	1.07\\
31.647872744998	1.08\\
31.663829019	1.09\\
31.679824031	1.11\\
31.695863198001	1.12\\
31.711820364999	1.13\\
31.727860282999	1.15\\
31.743795063999	1.16\\
31.759720853999	1.17\\
31.775698141	1.18\\
31.791693338	1.2\\
31.807882576	1.21\\
31.823855041999	1.22\\
31.839743753	1.23\\
31.855847726999	1.25\\
31.871844989	1.26\\
31.887846460998	1.27\\
31.903846233999	1.29\\
31.919823455998	1.3\\
31.935822421999	1.31\\
31.951854252	1.32\\
31.967843135	1.34\\
31.983812176001	1.35\\
31.999724723999	1.36\\
32.017360169001	1.37\\
32.033056118	1.36\\
32.048303145	1.34\\
32.063653834001	1.33\\
32.079634034999	1.32\\
32.095649087999	1.31\\
32.11167482	1.29\\
32.127725201999	1.28\\
32.143756976998	1.27\\
32.159689131999	1.26\\
32.17570007	1.24\\
32.191801543999	1.23\\
32.207807932	1.22\\
32.223862282999	1.21\\
32.239905123	1.19\\
32.255847741999	1.18\\
32.272097707	1.17\\
32.287803023001	1.15\\
32.303818556	1.14\\
32.319801996	1.13\\
32.335897916999	1.12\\
32.351861477999	1.1\\
32.36780086	1.09\\
32.383839074	1.08\\
32.399838731998	1.07\\
32.41588033	1.05\\
32.431820632998	1.04\\
32.447710776999	1.03\\
32.463782740998	1.02\\
32.479839825999	1\\
32.495804729999	0.99\\
32.511842398999	0.98\\
32.527854148998	0.96\\
32.543698729999	0.95\\
32.559771718999	0.94\\
32.575705261999	0.93\\
32.591657696999	0.91\\
32.607731292999	0.9\\
32.623867339	0.89\\
32.639880016999	0.88\\
32.655754424999	0.86\\
32.671609744	0.85\\
32.687715627	0.84\\
32.703617391	0.83\\
32.721792262	0.81\\
32.736867223999	0.8\\
32.751887913	0.79\\
32.767841096	0.78\\
32.783788374	0.76\\
32.799765005	0.75\\
32.815809012999	0.74\\
32.831737424	0.72\\
32.847734195	0.71\\
32.863802903999	0.7\\
32.879923169999	0.69\\
32.895848529999	0.67\\
32.911868408	0.66\\
32.927924669001	0.65\\
32.943857455	0.64\\
32.959846783998	0.62\\
32.975857772	0.61\\
32.991795901	0.6\\
33.007708880999	0.59\\
33.023571969999	0.58\\
33.039690382001	0.6\\
33.055846408998	0.61\\
33.071864497999	0.62\\
33.087798360999	0.63\\
33.103834506999	0.65\\
33.119845830999	0.66\\
33.135741533	0.67\\
33.151951467	0.69\\
33.167863962	0.7\\
33.183785027	0.71\\
33.199900010999	0.72\\
33.21601099	0.74\\
33.231932415999	0.75\\
33.247955881999	0.76\\
33.263839973	0.78\\
33.279812478999	0.79\\
33.295845717999	0.8\\
33.311794790999	0.81\\
33.327701127	0.83\\
33.343704187	0.84\\
33.359701494998	0.85\\
33.375706222999	0.86\\
33.391675642999	0.88\\
33.407684144998	0.89\\
33.423701763999	0.9\\
33.439723896999	0.92\\
33.455840841	0.93\\
33.471749897999	0.94\\
33.487732021	0.95\\
33.503689292	0.97\\
33.519730615	0.98\\
33.535758431001	0.99\\
33.551807045	1.01\\
33.567689761999	1.02\\
33.583958338999	1.03\\
33.599849721	1.04\\
33.615717138999	1.06\\
33.631705150999	1.07\\
33.647697791999	1.08\\
33.663697099	1.1\\
33.680008393	1.11\\
33.695816682	1.12\\
33.71170992	1.13\\
33.727648893999	1.15\\
33.743689658	1.16\\
33.75970862	1.17\\
33.775846662	1.18\\
33.791824299	1.2\\
33.807866032999	1.21\\
33.823820964001	1.22\\
33.839854287	1.24\\
33.855834685999	1.25\\
33.871827082999	1.26\\
33.887849651999	1.27\\
33.903802814	1.29\\
33.919887223999	1.3\\
33.935690485999	1.31\\
33.951832255999	1.33\\
33.967818482999	1.34\\
33.983881258999	1.35\\
33.999983439	1.36\\
34.017784545	1.37\\
34.033302124	1.36\\
34.049454504	1.35\\
34.065100403999	1.33\\
34.080526722	1.32\\
34.096079687999	1.31\\
34.111698967999	1.3\\
34.127863929999	1.28\\
34.143835548999	1.27\\
34.159827815	1.26\\
34.175980996	1.25\\
34.191804674999	1.23\\
34.207709372	1.22\\
34.223903635999	1.21\\
34.239836927999	1.2\\
34.255843766999	1.18\\
34.271850215	1.17\\
34.287811284999	1.16\\
34.303752955	1.14\\
34.319823259	1.13\\
34.335810542999	1.12\\
34.351705629	1.11\\
34.367695846	1.09\\
34.383705676001	1.08\\
34.399890613998	1.07\\
34.415873799	1.06\\
34.431969916999	1.04\\
34.448967975	1.03\\
34.464643212999	1.02\\
34.479934801	1\\
34.495756585	0.99\\
34.511815557	0.98\\
34.527894992	0.97\\
34.543857909999	0.95\\
34.559895448999	0.94\\
34.575778065999	0.93\\
34.591815072999	0.92\\
34.607849279	0.9\\
34.623785963999	0.89\\
34.639846975999	0.88\\
34.655832821999	0.87\\
34.671968458	0.85\\
34.687766627999	0.84\\
34.703820593999	0.83\\
34.719820176999	0.82\\
34.735820570999	0.8\\
34.751912884	0.79\\
34.767829332999	0.78\\
34.783953873	0.76\\
34.799842348999	0.75\\
34.815668999999	0.74\\
34.831908144998	0.73\\
34.847892292	0.71\\
34.863748443998	0.7\\
34.879900963999	0.69\\
34.895852202001	0.68\\
34.911695754999	0.66\\
34.927845258	0.65\\
34.943883807999	0.64\\
34.959834465	0.63\\
34.975824641	0.61\\
34.99171436	0.6\\
35.007744861999	0.59\\
35.025254952999	0.59\\
35.040903940999	0.6\\
35.056303038	0.61\\
35.071748384	0.62\\
35.087788281999	0.64\\
35.103838715	0.65\\
35.119816478	0.66\\
35.135759692999	0.68\\
35.151810091	0.69\\
35.167814624	0.7\\
35.183821212998	0.71\\
35.199844553998	0.73\\
35.215764110001	0.74\\
35.231959035	0.75\\
35.247863315999	0.77\\
35.263823915998	0.78\\
35.279859108999	0.79\\
35.295703374	0.8\\
35.311693435	0.82\\
35.327834853999	0.83\\
35.343919374	0.84\\
35.359814219	0.85\\
35.375697592999	0.87\\
35.391716062999	0.88\\
35.407694337	0.89\\
35.423856193	0.91\\
35.439779561998	0.92\\
35.455989156	0.93\\
35.471806886	0.94\\
35.490119561	0.96\\
35.505490374	0.97\\
35.520968564	0.98\\
35.536435684998	1\\
35.551891208999	1.01\\
35.567716380999	1.02\\
35.583700674	1.03\\
35.599801144999	1.05\\
35.615669900999	1.06\\
35.631857779999	1.07\\
35.647864238	1.09\\
35.663728378999	1.1\\
35.679686688	1.11\\
35.695672380999	1.12\\
35.711584515999	1.14\\
35.729806096999	1.15\\
35.744902782999	1.16\\
35.760110681	1.18\\
35.775840737	1.19\\
35.791722652999	1.2\\
35.807778879	1.21\\
35.823706823999	1.23\\
35.839694308999	1.24\\
35.855861505999	1.25\\
35.871919321	1.27\\
35.887652866	1.28\\
35.903717639999	1.29\\
35.919702444	1.3\\
35.935698924999	1.32\\
35.951863757001	1.33\\
35.967828335	1.34\\
35.983792993	1.35\\
35.999809302998	1.37\\
36.017628158999	1.37\\
36.033005756999	1.36\\
36.048515684	1.35\\
36.063824887	1.34\\
36.079669848999	1.32\\
36.095728209999	1.31\\
36.111705973	1.3\\
36.12766788	1.29\\
36.143665497999	1.27\\
36.159683521998	1.26\\
36.175841087999	1.25\\
36.191896410999	1.24\\
36.207695695001	1.22\\
36.223887533	1.21\\
36.239875649999	1.2\\
36.255661187999	1.19\\
36.271971127998	1.17\\
36.287909872001	1.16\\
36.303728277999	1.15\\
36.319756417999	1.13\\
36.335822179	1.12\\
36.351695965999	1.11\\
36.367665785999	1.1\\
36.383714275	1.08\\
36.399898690999	1.07\\
36.416074167	1.06\\
36.431769614999	1.05\\
36.447693742001	1.03\\
36.463626581999	1.02\\
36.479722524	1.01\\
36.495704801999	1\\
36.511707165998	0.98\\
36.527710746999	0.97\\
36.543694929	0.96\\
36.5598634	0.95\\
36.575924115999	0.93\\
36.591982275999	0.92\\
36.6079584	0.91\\
36.623697392	0.89\\
36.63981169	0.88\\
36.655838974998	0.87\\
36.67189268	0.86\\
36.687791429	0.84\\
36.703937592999	0.83\\
36.719852268999	0.82\\
36.735891135	0.81\\
36.751871979999	0.79\\
36.767650155	0.78\\
36.783675622001	0.77\\
36.799717324	0.76\\
36.815792759	0.74\\
36.832102594	0.73\\
36.847744912	0.72\\
36.86370491	0.7\\
36.879854011	0.69\\
36.895704851998	0.68\\
36.911809416999	0.67\\
36.927699861998	0.65\\
36.943654455	0.64\\
36.959678384999	0.63\\
36.975700368999	0.62\\
36.991834962	0.6\\
37.007837159999	0.59\\
37.025617002998	0.59\\
37.041016169	0.6\\
37.056526082	0.61\\
37.072106944999	0.63\\
37.087718492999	0.64\\
37.103741398999	0.65\\
37.119670535999	0.66\\
37.135706004999	0.68\\
37.151705271999	0.69\\
37.167676886	0.7\\
37.183732585999	0.72\\
37.199701763999	0.73\\
37.215691739999	0.74\\
37.231735371999	0.75\\
37.247689832	0.77\\
37.263716721	0.78\\
37.279687152999	0.79\\
37.295714599	0.81\\
37.311677707999	0.82\\
37.327659806	0.83\\
37.343841435999	0.84\\
37.359840361999	0.86\\
37.375693652999	0.87\\
37.391699228999	0.88\\
37.407716957999	0.9\\
37.423849396	0.91\\
37.439832295998	0.92\\
37.457367139999	0.94\\
37.472815350999	0.95\\
37.488572995	0.96\\
37.504061944999	0.97\\
37.51993047	0.99\\
37.535864198998	1\\
37.551845002999	1.01\\
37.567850549	1.02\\
37.583819351999	1.04\\
37.599694597999	1.05\\
37.615705725999	1.06\\
37.631704641999	1.07\\
37.647782602	1.09\\
37.663850118999	1.1\\
37.679871976999	1.11\\
37.695875082	1.13\\
37.714285709001	1.14\\
37.729412462998	1.15\\
37.744507830999	1.17\\
37.759693639998	1.18\\
37.775747415001	1.19\\
37.791673431998	1.2\\
37.807709364999	1.22\\
37.82367441	1.23\\
37.839797455999	1.24\\
37.855713525	1.25\\
37.871693322999	1.27\\
37.887683397999	1.28\\
37.903683176001	1.29\\
37.919692882001	1.31\\
37.935688113999	1.32\\
37.951699122999	1.33\\
37.967694910999	1.34\\
37.983689011	1.36\\
37.999708162999	1.37\\
38.017037489999	1.38\\
38.032196913	1.38\\
38.047686214001	1.38\\
38.063690912999	1.38\\
38.079695585998	1.38\\
38.095666899999	1.38\\
38.111695071999	1.38\\
38.127705113998	1.38\\
38.143658671999	1.38\\
38.159663942	1.38\\
38.175699215	1.38\\
38.191669179999	1.38\\
38.207694676999	1.38\\
38.223686110999	1.38\\
38.239757521	1.38\\
38.255734426999	1.38\\
};
\end{axis}
\end{tikzpicture}%
}
      \caption{The orientation of the robot over time for
        $K_{\omega}^T = K_{\omega, max}^T + 1$. The system is marginally stable}
      \label{fig:13_max_plus_one}
    \end{figure}
  \end{minipage}
  \hfill
\end{minipage}
}

