Figure \ref{fig:22_map_7} plots the trajectory of the robot for an experiment
performed at KTH's Smart Mobility Lab of the Automatic Control Department. The
robot was initially placed in the vicinity of node 2. Its first goal was
dictated to be node 1, and from there, nodes 5 and 3.

The robot's bearing and distance tolarance from a target was set to
$\xi = 4^{\circ}$ and the distance tolerance to $\delta$ = 6 cm. The controller's
gains were set to

$$(\dfrac{K_{\Psi}^R}{K_{\Psi,max}^R}, \dfrac{K_{\omega}^T}{K_{\omega, max}^T}, \dfrac{K_{\omega}^R}{K_{\omega,max}^R}, \dfrac{K_{\Psi}^T}{K_{\Psi,max}^T})
\equiv (0.2, 0.2, 0.5, 0.5)$$
where $K_{*,max} > 0$.

The robot's distance errors from nodes 1, 5 and 3 were $e_1 = 7.42$ cm,
$e_5 = 7.07$ cm and $e_1 = 8.54$ cm and respectively.

It wasn't before this experiment that we considered a
positive $K_{omega}^T$, instead of a negative one, as the theoretical analysis
pointed to. The second component of the line-following controller showed
significantly worse performance when $K_{omega}^T < 0$ than when $K_{omega}^T > 0$.

When the robot was asked to go from node 2 to node 1, the angle between it and
node 1 was $1.68^{\circ}$. The augmented error in distance, relative to the
other two, is hence the fact that $K_{omega}^T < 0$ during the experimental
phase.

\begin{figure}[H]\centering
  \scalebox{1}{% This file was created by matlab2tikz.
%
%The latest updates can be retrieved from
%  http://www.mathworks.com/matlabcentral/fileexchange/22022-matlab2tikz-matlab2tikz
%where you can also make suggestions and rate matlab2tikz.
%
\definecolor{mycolor1}{rgb}{0.00000,0.44700,0.74100}%
%
\begin{tikzpicture}

\begin{axis}[%
width=4.133in,
height=3.26in,
at={(0.693in,0.44in)},
scale only axis,
xmin=-0.85,
xmax=3,
ymin=-1.8,
ymax=2,
axis background/.style={fill=white}
]
\addplot [color=mycolor1,solid,forget plot]
  table[row sep=crcr]{%
1.18	1.68\\
1.18	1.68\\
1.18	1.68\\
1.18	1.68\\
1.18	1.68\\
1.18	1.68\\
1.18	1.68\\
1.18	1.68\\
1.18	1.68\\
1.18	1.68\\
1.18	1.68\\
1.18	1.68\\
1.18	1.68\\
1.18	1.68\\
1.18	1.68\\
1.18	1.68\\
1.18	1.68\\
1.18	1.68\\
1.18	1.68\\
1.18	1.68\\
1.18	1.68\\
1.18	1.68\\
1.18	1.68\\
1.18	1.68\\
1.18	1.68\\
1.18	1.68\\
1.18	1.68\\
1.18	1.68\\
1.18	1.68\\
1.18	1.68\\
1.18	1.68\\
1.18	1.68\\
1.18	1.68\\
1.18	1.68\\
1.18	1.68\\
1.18	1.68\\
1.18	1.68\\
1.18	1.68\\
1.18	1.68\\
1.18	1.68\\
1.18	1.68\\
1.18	1.68\\
1.18	1.68\\
1.18	1.68\\
1.18	1.68\\
1.18	1.68\\
1.18	1.68\\
1.18	1.68\\
1.18	1.68\\
1.18	1.68\\
1.18	1.68\\
1.18	1.68\\
1.18	1.68\\
1.18	1.68\\
1.18	1.68\\
1.18	1.68\\
1.18	1.68\\
1.18	1.68\\
1.18	1.68\\
1.18	1.68\\
1.18	1.68\\
1.18	1.68\\
1.18	1.68\\
1.18	1.68\\
1.18	1.68\\
1.18	1.68\\
1.18	1.68\\
1.18	1.68\\
1.18	1.68\\
1.18	1.68\\
1.18	1.68\\
1.18	1.68\\
1.18	1.68\\
1.18	1.68\\
1.18	1.68\\
1.18	1.68\\
1.18	1.68\\
1.18	1.68\\
1.18	1.68\\
1.18	1.68\\
1.18	1.68\\
1.18	1.68\\
1.18	1.68\\
1.18	1.68\\
1.18	1.68\\
1.18	1.68\\
1.18	1.68\\
1.18	1.68\\
1.18	1.68\\
1.18	1.68\\
1.18	1.68\\
1.18	1.68\\
1.18	1.68\\
1.18	1.68\\
1.18	1.68\\
1.18	1.68\\
1.18	1.68\\
1.18	1.68\\
1.18	1.68\\
1.18	1.68\\
1.18	1.68\\
1.18	1.68\\
1.18	1.68\\
1.18	1.68\\
1.18	1.68\\
1.18	1.68\\
1.18	1.68\\
1.18	1.68\\
1.18	1.68\\
1.18	1.68\\
1.18	1.68\\
1.18	1.68\\
1.18	1.68\\
1.18	1.68\\
1.18	1.68\\
1.18	1.68\\
1.18	1.68\\
1.18	1.68\\
1.18	1.68\\
1.18	1.68\\
1.18	1.68\\
1.18	1.68\\
1.18	1.68\\
1.18	1.68\\
1.18	1.68\\
1.18	1.68\\
1.18	1.68\\
1.18	1.68\\
1.18	1.68\\
1.18	1.68\\
1.18	1.68\\
1.18	1.68\\
1.18	1.68\\
1.18	1.68\\
1.18	1.68\\
1.18	1.68\\
1.18	1.68\\
1.18	1.68\\
1.18	1.68\\
1.18	1.68\\
1.18	1.68\\
1.18	1.68\\
1.18	1.68\\
1.18	1.68\\
1.18	1.68\\
1.18	1.68\\
1.18	1.68\\
1.18	1.68\\
1.18	1.68\\
1.18	1.68\\
1.18	1.68\\
1.18	1.68\\
1.18	1.68\\
1.18	1.68\\
1.18	1.68\\
1.18	1.68\\
1.18	1.68\\
1.18	1.68\\
1.18	1.68\\
1.18	1.68\\
1.18	1.68\\
1.18	1.68\\
1.18	1.68\\
1.18	1.68\\
1.18	1.68\\
1.18	1.68\\
1.18	1.68\\
1.18	1.68\\
1.18	1.68\\
1.18	1.68\\
1.18	1.68\\
1.18	1.68\\
1.18	1.68\\
1.18	1.68\\
1.18	1.68\\
1.18	1.68\\
1.18	1.68\\
1.18	1.68\\
1.18	1.68\\
1.18	1.68\\
1.18	1.68\\
1.18	1.68\\
1.18	1.68\\
1.18	1.68\\
1.18	1.68\\
1.18	1.68\\
1.18	1.68\\
1.18	1.68\\
1.18	1.68\\
1.18	1.68\\
1.18	1.68\\
1.18	1.68\\
1.18	1.68\\
1.18	1.68\\
1.18	1.68\\
1.18	1.68\\
1.18	1.68\\
1.18	1.68\\
1.18	1.68\\
1.18	1.68\\
1.18	1.68\\
1.18	1.68\\
1.18	1.68\\
1.18	1.68\\
1.18	1.68\\
1.18	1.68\\
1.18	1.68\\
1.18	1.68\\
1.18	1.68\\
1.18	1.68\\
1.18	1.68\\
1.18	1.68\\
1.18	1.68\\
1.18	1.68\\
1.18	1.68\\
1.18	1.68\\
1.18	1.68\\
1.18	1.68\\
1.18	1.68\\
1.18	1.68\\
1.18	1.68\\
1.18	1.68\\
1.18	1.68\\
1.18	1.68\\
1.18	1.68\\
1.18	1.68\\
1.18	1.68\\
1.18	1.68\\
1.18	1.68\\
1.18	1.68\\
1.18	1.68\\
1.18	1.68\\
1.18	1.68\\
1.18	1.68\\
1.18	1.68\\
1.18	1.68\\
1.18	1.68\\
1.18	1.68\\
1.18	1.68\\
1.18	1.68\\
1.18	1.68\\
1.18	1.68\\
1.18	1.68\\
1.18	1.68\\
1.18	1.68\\
1.18	1.68\\
1.18	1.68\\
1.18	1.68\\
1.18	1.68\\
1.18	1.68\\
1.18	1.68\\
1.18	1.68\\
1.18	1.68\\
1.18	1.68\\
1.18	1.68\\
1.18	1.68\\
1.18	1.68\\
1.18	1.68\\
1.18	1.68\\
1.18	1.68\\
1.18	1.68\\
1.18	1.68\\
1.18	1.68\\
1.18	1.68\\
1.18	1.68\\
1.18	1.68\\
1.18	1.68\\
1.18	1.68\\
1.18	1.68\\
1.18	1.68\\
1.18	1.68\\
1.18	1.68\\
1.18	1.68\\
1.18	1.68\\
1.18	1.68\\
1.18	1.68\\
1.18	1.68\\
1.18	1.68\\
1.18	1.68\\
1.18	1.68\\
1.18	1.68\\
1.18	1.68\\
1.18	1.68\\
1.18	1.68\\
1.18	1.68\\
1.18	1.68\\
1.18	1.68\\
1.18	1.68\\
1.18	1.68\\
1.18	1.68\\
1.18	1.68\\
1.18	1.68\\
1.18	1.68\\
1.18	1.68\\
1.18	1.68\\
1.18	1.68\\
1.18	1.68\\
1.18	1.68\\
1.18	1.68\\
1.18	1.68\\
1.18	1.68\\
1.18	1.68\\
1.18	1.68\\
1.18	1.68\\
1.18	1.68\\
1.18	1.68\\
1.18	1.68\\
1.18	1.68\\
1.18	1.68\\
1.18	1.68\\
1.18	1.68\\
1.18	1.68\\
1.18	1.68\\
1.18	1.68\\
1.18	1.68\\
1.18	1.68\\
1.18	1.68\\
1.18	1.68\\
1.18	1.68\\
1.18	1.68\\
1.18	1.68\\
1.18	1.68\\
1.18	1.68\\
1.18	1.68\\
1.18	1.68\\
1.18	1.68\\
1.18	1.68\\
1.18	1.68\\
1.18	1.68\\
1.18	1.68\\
1.18	1.68\\
1.18	1.68\\
1.18	1.68\\
1.18	1.68\\
1.18	1.68\\
1.18	1.68\\
1.18	1.68\\
1.18	1.68\\
1.18	1.68\\
1.18	1.68\\
1.18	1.68\\
1.18	1.68\\
1.18	1.68\\
1.18	1.68\\
1.18	1.68\\
1.18	1.68\\
1.18	1.68\\
1.18	1.68\\
1.18	1.68\\
1.18	1.68\\
1.18	1.68\\
1.18	1.68\\
1.18	1.68\\
1.18	1.68\\
1.18	1.68\\
1.18	1.68\\
1.18	1.68\\
1.18	1.68\\
1.18	1.68\\
1.18	1.68\\
1.18	1.68\\
1.18	1.68\\
1.18	1.68\\
1.18	1.68\\
1.18	1.68\\
1.18	1.68\\
1.18	1.68\\
1.18	1.68\\
1.18	1.68\\
1.18	1.68\\
1.18	1.68\\
1.18	1.68\\
1.18	1.68\\
1.18	1.68\\
1.18	1.68\\
1.18	1.68\\
1.18	1.68\\
1.18	1.68\\
1.18	1.68\\
1.18	1.68\\
1.18	1.68\\
1.18	1.68\\
1.18	1.68\\
1.18	1.68\\
1.18	1.68\\
1.18	1.68\\
1.18	1.68\\
1.18	1.68\\
1.18	1.68\\
1.18	1.68\\
1.18	1.68\\
1.18	1.68\\
1.18	1.68\\
1.18	1.68\\
1.18	1.68\\
1.18	1.68\\
1.18	1.68\\
1.18	1.68\\
1.18	1.68\\
1.18	1.68\\
1.18	1.68\\
1.18	1.68\\
1.18	1.68\\
1.18	1.68\\
1.18	1.68\\
1.18	1.68\\
1.18	1.68\\
1.18	1.68\\
1.18	1.68\\
1.18	1.68\\
1.18	1.68\\
1.18	1.68\\
1.18	1.68\\
1.18	1.68\\
1.18	1.68\\
1.18	1.68\\
1.18	1.68\\
1.18	1.68\\
1.18	1.68\\
1.18	1.68\\
1.18	1.68\\
1.18	1.68\\
1.18	1.68\\
1.18	1.68\\
1.18	1.68\\
1.18	1.68\\
1.18	1.68\\
1.18	1.68\\
1.18	1.68\\
1.18	1.68\\
1.18	1.68\\
1.18	1.68\\
1.18	1.68\\
1.18	1.68\\
1.18	1.68\\
1.18	1.68\\
1.18	1.68\\
1.18	1.68\\
1.18	1.68\\
1.18	1.68\\
1.18	1.68\\
1.18	1.68\\
1.18	1.68\\
1.18	1.68\\
1.18	1.68\\
1.18	1.68\\
1.18	1.68\\
1.18	1.68\\
1.18	1.68\\
1.18	1.68\\
1.18	1.68\\
1.18	1.68\\
1.18	1.68\\
1.18	1.68\\
1.18	1.68\\
1.18	1.68\\
1.18	1.68\\
1.18	1.68\\
1.18	1.68\\
1.18	1.68\\
1.18	1.68\\
1.18	1.68\\
1.18	1.68\\
1.18	1.68\\
1.18	1.68\\
1.18	1.68\\
1.18	1.68\\
1.18	1.68\\
1.18	1.68\\
1.18	1.68\\
1.18	1.68\\
1.18	1.68\\
1.18	1.68\\
1.18	1.68\\
1.18	1.68\\
1.18	1.68\\
1.18	1.68\\
1.18	1.68\\
1.18	1.68\\
1.18	1.68\\
1.18	1.68\\
1.18	1.68\\
1.18	1.68\\
1.18	1.68\\
1.18	1.68\\
1.18	1.68\\
1.18	1.68\\
1.18	1.68\\
1.18	1.68\\
1.18	1.68\\
1.18	1.68\\
1.18	1.68\\
1.18	1.68\\
1.18	1.68\\
1.18	1.68\\
1.18	1.68\\
1.18	1.68\\
1.18	1.68\\
1.18	1.68\\
1.18	1.68\\
1.18	1.68\\
1.18	1.68\\
1.18	1.68\\
1.18	1.68\\
1.18	1.68\\
1.18	1.68\\
1.18	1.68\\
1.18	1.68\\
1.18	1.68\\
1.18	1.68\\
1.18	1.68\\
1.18	1.68\\
1.18	1.68\\
1.18	1.68\\
1.18	1.68\\
1.18	1.68\\
1.18	1.68\\
1.18	1.68\\
1.18	1.68\\
1.18	1.68\\
1.18	1.68\\
1.18	1.68\\
1.18	1.68\\
1.18	1.68\\
1.18	1.68\\
1.18	1.68\\
1.18	1.68\\
1.18	1.68\\
1.18	1.68\\
1.18	1.68\\
1.18	1.68\\
1.18	1.68\\
1.18	1.68\\
1.18	1.68\\
1.18	1.68\\
1.18	1.68\\
1.18	1.68\\
1.18	1.68\\
1.18	1.68\\
1.18	1.68\\
1.18	1.68\\
1.18	1.68\\
1.18	1.68\\
1.18	1.68\\
1.18	1.68\\
1.18	1.68\\
1.18	1.68\\
1.18	1.68\\
1.18	1.68\\
1.18	1.68\\
1.18	1.68\\
1.18	1.68\\
1.18	1.68\\
1.18	1.68\\
1.18	1.68\\
1.18	1.68\\
1.18	1.68\\
1.18	1.68\\
1.18	1.68\\
1.18	1.68\\
1.18	1.68\\
1.18	1.68\\
1.18	1.68\\
1.18	1.68\\
1.18	1.68\\
1.18	1.68\\
1.18	1.68\\
1.18	1.68\\
1.18	1.68\\
1.18	1.68\\
1.18	1.68\\
1.18	1.68\\
1.18	1.68\\
1.18	1.68\\
1.18	1.68\\
1.18	1.68\\
1.18	1.68\\
1.18	1.68\\
1.18	1.68\\
1.18	1.68\\
1.18	1.68\\
1.18	1.68\\
1.18	1.68\\
1.18	1.68\\
1.18	1.68\\
1.18	1.68\\
1.18	1.68\\
1.18	1.68\\
1.18	1.68\\
1.18	1.68\\
1.18	1.68\\
1.18	1.68\\
1.18	1.68\\
1.18	1.68\\
1.18	1.68\\
1.18	1.68\\
1.18	1.68\\
1.18	1.68\\
1.18	1.68\\
1.18	1.68\\
1.18	1.68\\
1.18	1.68\\
1.18	1.68\\
1.18	1.68\\
1.18	1.68\\
1.18	1.68\\
1.18	1.68\\
1.18	1.68\\
1.18	1.68\\
1.18	1.68\\
1.18	1.68\\
1.18	1.68\\
1.18	1.68\\
1.18	1.68\\
1.18	1.68\\
1.18	1.68\\
1.18	1.68\\
1.18	1.68\\
1.18	1.68\\
1.18	1.68\\
1.18	1.68\\
1.18	1.68\\
1.18	1.68\\
1.18	1.68\\
1.18	1.68\\
1.18	1.68\\
1.18	1.68\\
1.18	1.68\\
1.18	1.68\\
1.18	1.68\\
1.18	1.68\\
1.18	1.68\\
1.18	1.68\\
1.18	1.68\\
1.18	1.68\\
1.18	1.68\\
1.18	1.68\\
1.18	1.68\\
1.18	1.68\\
1.18	1.68\\
1.18	1.68\\
1.18	1.68\\
1.18	1.68\\
1.18	1.68\\
1.18	1.68\\
1.18	1.68\\
1.18	1.68\\
1.18	1.68\\
1.18	1.68\\
1.18	1.68\\
1.18	1.68\\
1.18	1.68\\
1.18	1.68\\
1.18	1.68\\
1.18	1.68\\
1.18	1.68\\
1.18	1.68\\
1.18	1.68\\
1.18	1.68\\
1.18	1.68\\
1.18	1.68\\
1.18	1.68\\
1.18	1.68\\
1.18	1.68\\
1.18	1.68\\
1.18	1.68\\
1.18	1.68\\
1.18	1.68\\
1.18	1.68\\
1.18	1.68\\
1.18	1.68\\
1.18	1.68\\
1.18	1.68\\
1.18	1.68\\
1.18	1.68\\
1.18	1.68\\
1.18	1.68\\
1.18	1.68\\
1.18	1.68\\
1.18	1.68\\
1.18	1.68\\
1.18	1.68\\
1.18	1.68\\
1.18	1.68\\
1.18	1.68\\
1.18	1.68\\
1.18	1.68\\
1.18	1.68\\
1.18	1.68\\
1.18	1.68\\
1.18	1.68\\
1.18	1.68\\
1.18	1.68\\
1.18	1.68\\
1.18	1.68\\
1.18	1.68\\
1.18	1.68\\
1.18	1.68\\
1.18	1.68\\
1.18	1.68\\
1.18	1.68\\
1.18	1.68\\
1.18	1.68\\
1.18	1.68\\
1.18	1.68\\
1.18	1.68\\
1.18	1.68\\
1.18	1.68\\
1.18	1.68\\
1.18	1.68\\
1.18	1.68\\
1.18	1.68\\
1.18	1.68\\
1.18	1.68\\
1.18	1.68\\
1.18	1.68\\
1.18	1.68\\
1.18	1.68\\
1.18	1.68\\
1.18	1.68\\
1.18	1.68\\
1.18	1.68\\
1.18	1.68\\
1.18	1.68\\
1.18	1.68\\
1.18	1.68\\
1.18	1.68\\
1.18	1.68\\
1.18	1.68\\
1.18	1.68\\
1.18	1.68\\
1.18	1.68\\
1.18	1.68\\
1.18	1.68\\
1.18	1.68\\
1.18	1.68\\
1.18	1.68\\
1.18	1.68\\
1.18	1.68\\
1.18	1.68\\
1.18	1.68\\
1.18	1.68\\
1.18	1.68\\
1.18	1.68\\
1.18	1.68\\
1.18	1.68\\
1.18	1.68\\
1.18	1.68\\
1.18	1.68\\
1.18	1.68\\
1.18	1.68\\
1.18	1.68\\
1.18	1.68\\
1.18	1.68\\
1.18	1.68\\
1.18	1.68\\
1.18	1.68\\
1.18	1.68\\
1.18	1.68\\
1.18	1.68\\
1.18	1.68\\
1.18	1.68\\
1.18	1.68\\
1.18	1.68\\
1.18	1.68\\
1.18	1.68\\
1.18	1.68\\
1.18	1.68\\
1.18	1.68\\
1.18	1.68\\
1.18	1.68\\
1.18	1.68\\
1.18	1.68\\
1.18	1.68\\
1.18	1.68\\
1.18	1.68\\
1.18	1.68\\
1.18	1.68\\
1.18	1.68\\
1.18	1.68\\
1.18	1.68\\
1.18	1.68\\
1.18	1.68\\
1.18	1.68\\
1.18	1.68\\
1.18	1.68\\
1.18	1.68\\
1.18	1.68\\
1.18	1.68\\
1.18	1.68\\
1.18	1.68\\
1.18	1.68\\
1.18	1.68\\
1.18	1.68\\
1.18	1.68\\
1.18	1.68\\
1.18	1.68\\
1.18	1.68\\
1.18	1.68\\
1.18	1.68\\
1.18	1.68\\
1.18	1.68\\
1.18	1.68\\
1.18	1.68\\
1.18	1.68\\
1.18	1.68\\
1.18	1.68\\
1.18	1.68\\
1.18	1.68\\
1.18	1.68\\
1.18	1.68\\
1.18	1.68\\
1.18	1.68\\
1.18	1.68\\
1.18	1.68\\
1.18	1.68\\
1.18	1.68\\
1.18	1.68\\
1.18	1.68\\
1.18	1.68\\
1.18	1.68\\
1.18	1.68\\
1.18	1.68\\
1.18	1.68\\
1.18	1.68\\
1.18	1.68\\
1.18	1.68\\
1.18	1.69\\
1.18	1.69\\
1.18	1.69\\
1.18	1.69\\
1.18	1.69\\
1.18	1.69\\
1.18	1.69\\
1.18	1.69\\
1.18	1.69\\
1.18	1.69\\
1.18	1.69\\
1.18	1.69\\
1.18	1.69\\
1.18	1.69\\
1.18	1.69\\
1.18	1.69\\
1.18	1.69\\
1.18	1.69\\
1.18	1.69\\
1.18	1.68\\
1.18	1.69\\
1.18	1.69\\
1.18	1.69\\
1.18	1.69\\
1.18	1.69\\
1.18	1.69\\
1.18	1.69\\
1.18	1.69\\
1.18	1.69\\
1.18	1.69\\
1.18	1.69\\
1.18	1.69\\
1.18	1.69\\
1.18	1.69\\
1.18	1.69\\
1.18	1.69\\
1.18	1.69\\
1.18	1.69\\
1.18	1.69\\
1.18	1.69\\
1.18	1.69\\
1.18	1.69\\
1.18	1.69\\
1.18	1.69\\
1.18	1.69\\
1.18	1.69\\
1.18	1.69\\
1.18	1.69\\
1.18	1.69\\
1.18	1.69\\
1.18	1.69\\
1.18	1.69\\
1.18	1.69\\
1.18	1.69\\
1.18	1.69\\
1.18	1.69\\
1.18	1.69\\
1.18	1.69\\
1.18	1.69\\
1.18	1.69\\
1.18	1.69\\
1.18	1.69\\
1.18	1.69\\
1.18	1.69\\
1.18	1.69\\
1.18	1.69\\
1.18	1.69\\
1.18	1.69\\
1.18	1.69\\
1.18	1.69\\
1.18	1.69\\
1.18	1.69\\
1.18	1.69\\
1.18	1.69\\
1.18	1.69\\
1.18	1.69\\
1.18	1.69\\
1.18	1.69\\
1.18	1.69\\
1.18	1.69\\
1.18	1.69\\
1.18	1.69\\
1.18	1.69\\
1.18	1.69\\
1.18	1.69\\
1.18	1.69\\
1.18	1.69\\
1.18	1.69\\
1.18	1.69\\
1.18	1.69\\
1.18	1.69\\
1.18	1.69\\
1.18	1.69\\
1.18	1.69\\
1.18	1.69\\
1.18	1.69\\
1.18	1.69\\
1.18	1.69\\
1.18	1.69\\
1.18	1.69\\
1.18	1.69\\
1.18	1.69\\
1.18	1.69\\
1.18	1.69\\
1.18	1.69\\
1.18	1.69\\
1.18	1.69\\
1.18	1.69\\
1.18	1.69\\
1.18	1.69\\
1.18	1.69\\
1.18	1.69\\
1.18	1.69\\
1.18	1.69\\
1.18	1.69\\
1.18	1.69\\
1.18	1.69\\
1.18	1.69\\
1.18	1.69\\
1.18	1.69\\
1.18	1.69\\
1.18	1.69\\
1.18	1.69\\
1.18	1.69\\
1.18	1.69\\
1.18	1.69\\
1.18	1.69\\
1.18	1.69\\
1.18	1.69\\
1.18	1.69\\
1.18	1.69\\
1.18	1.69\\
1.18	1.69\\
1.18	1.69\\
1.18	1.69\\
1.18	1.69\\
1.18	1.69\\
1.18	1.69\\
1.18	1.69\\
1.18	1.69\\
1.18	1.69\\
1.18	1.69\\
1.18	1.69\\
1.18	1.69\\
1.18	1.69\\
1.18	1.69\\
1.18	1.69\\
1.18	1.69\\
1.18	1.69\\
1.18	1.69\\
1.18	1.69\\
1.18	1.69\\
1.18	1.69\\
1.18	1.69\\
1.18	1.69\\
1.18	1.69\\
1.18	1.69\\
1.18	1.69\\
1.18	1.69\\
1.18	1.69\\
1.18	1.69\\
1.18	1.69\\
1.18	1.69\\
1.18	1.69\\
1.18	1.69\\
1.18	1.69\\
1.18	1.69\\
1.18	1.69\\
1.18	1.69\\
1.18	1.69\\
1.18	1.69\\
1.18	1.69\\
1.18	1.69\\
1.18	1.69\\
1.18	1.69\\
1.18	1.69\\
1.18	1.69\\
1.18	1.69\\
1.18	1.69\\
1.18	1.69\\
1.18	1.69\\
1.18	1.69\\
1.18	1.69\\
1.18	1.69\\
1.18	1.69\\
1.18	1.69\\
1.18	1.69\\
1.18	1.69\\
1.18	1.69\\
1.18	1.69\\
1.18	1.69\\
1.18	1.69\\
1.18	1.69\\
1.18	1.69\\
1.18	1.69\\
1.18	1.69\\
1.18	1.69\\
1.18	1.69\\
1.18	1.69\\
1.18	1.69\\
1.18	1.69\\
1.18	1.69\\
1.18	1.69\\
1.17	1.69\\
1.17	1.69\\
1.17	1.69\\
1.18	1.69\\
1.17	1.69\\
1.17	1.69\\
1.17	1.69\\
1.17	1.69\\
1.17	1.69\\
1.17	1.69\\
1.17	1.69\\
1.17	1.69\\
1.17	1.69\\
1.17	1.69\\
1.17	1.69\\
1.17	1.69\\
1.17	1.69\\
1.17	1.69\\
1.17	1.69\\
1.17	1.69\\
1.17	1.69\\
1.17	1.69\\
1.17	1.69\\
1.17	1.69\\
1.17	1.69\\
1.17	1.69\\
1.17	1.69\\
1.17	1.69\\
1.17	1.69\\
1.17	1.69\\
1.17	1.69\\
1.17	1.69\\
1.17	1.69\\
1.17	1.69\\
1.17	1.69\\
1.17	1.69\\
1.17	1.69\\
1.17	1.69\\
1.17	1.69\\
1.17	1.69\\
1.17	1.69\\
1.17	1.69\\
1.17	1.69\\
1.17	1.69\\
1.17	1.69\\
1.17	1.69\\
1.17	1.69\\
1.17	1.69\\
1.17	1.69\\
1.17	1.69\\
1.17	1.69\\
1.17	1.69\\
1.17	1.69\\
1.18	1.69\\
1.18	1.69\\
1.18	1.69\\
1.18	1.69\\
1.18	1.69\\
1.18	1.69\\
1.18	1.69\\
1.18	1.69\\
1.18	1.69\\
1.18	1.69\\
1.18	1.69\\
1.18	1.69\\
1.18	1.69\\
1.18	1.69\\
1.18	1.69\\
1.18	1.69\\
1.18	1.69\\
1.18	1.69\\
1.18	1.69\\
1.18	1.69\\
1.18	1.69\\
1.18	1.69\\
1.18	1.69\\
1.18	1.69\\
1.18	1.69\\
1.18	1.69\\
1.18	1.69\\
1.18	1.69\\
1.18	1.69\\
1.18	1.69\\
1.18	1.69\\
1.18	1.69\\
1.18	1.69\\
1.18	1.69\\
1.18	1.69\\
1.18	1.69\\
1.18	1.69\\
1.18	1.69\\
1.18	1.69\\
1.18	1.69\\
1.18	1.69\\
1.18	1.69\\
1.18	1.69\\
1.18	1.69\\
1.18	1.69\\
1.18	1.69\\
1.18	1.69\\
1.18	1.69\\
1.18	1.69\\
1.18	1.69\\
1.18	1.69\\
1.18	1.69\\
1.18	1.69\\
1.18	1.69\\
1.18	1.69\\
1.18	1.69\\
1.18	1.69\\
1.18	1.69\\
1.18	1.69\\
1.18	1.69\\
1.18	1.69\\
1.18	1.69\\
1.18	1.69\\
1.18	1.69\\
1.18	1.69\\
1.18	1.69\\
1.18	1.69\\
1.18	1.69\\
1.18	1.69\\
1.18	1.69\\
1.18	1.69\\
1.18	1.69\\
1.18	1.69\\
1.18	1.69\\
1.18	1.69\\
1.18	1.69\\
1.18	1.69\\
1.18	1.69\\
1.18	1.69\\
1.18	1.69\\
1.18	1.69\\
1.18	1.69\\
1.18	1.69\\
1.18	1.69\\
1.18	1.69\\
1.18	1.69\\
1.18	1.69\\
1.18	1.69\\
1.18	1.69\\
1.18	1.69\\
1.18	1.69\\
1.18	1.69\\
1.18	1.69\\
1.18	1.69\\
1.18	1.69\\
1.18	1.69\\
1.18	1.69\\
1.18	1.69\\
1.18	1.69\\
1.18	1.69\\
1.18	1.69\\
1.18	1.69\\
1.18	1.69\\
1.18	1.69\\
1.18	1.69\\
1.18	1.69\\
1.18	1.69\\
1.18	1.69\\
1.18	1.69\\
1.18	1.69\\
1.18	1.69\\
1.18	1.69\\
1.18	1.69\\
1.18	1.69\\
1.18	1.69\\
1.18	1.69\\
1.18	1.69\\
1.18	1.69\\
1.18	1.69\\
1.18	1.69\\
1.18	1.69\\
1.18	1.69\\
1.18	1.69\\
1.18	1.69\\
1.18	1.69\\
1.18	1.69\\
1.18	1.69\\
1.18	1.69\\
1.18	1.69\\
1.18	1.69\\
1.18	1.69\\
1.18	1.69\\
1.18	1.69\\
1.18	1.69\\
1.18	1.69\\
1.18	1.69\\
1.18	1.69\\
1.18	1.69\\
1.18	1.69\\
1.18	1.69\\
1.18	1.69\\
1.18	1.69\\
1.18	1.69\\
1.18	1.69\\
1.18	1.69\\
1.18	1.69\\
1.18	1.69\\
1.18	1.69\\
1.18	1.69\\
1.18	1.69\\
1.18	1.69\\
1.18	1.69\\
1.18	1.69\\
1.18	1.69\\
1.18	1.69\\
1.18	1.69\\
1.18	1.69\\
1.18	1.69\\
1.18	1.69\\
1.18	1.69\\
1.18	1.69\\
1.18	1.69\\
1.18	1.69\\
1.18	1.69\\
1.18	1.69\\
1.18	1.69\\
1.18	1.69\\
1.18	1.69\\
1.18	1.69\\
1.18	1.69\\
1.18	1.69\\
1.18	1.69\\
1.18	1.69\\
1.18	1.69\\
1.18	1.69\\
1.18	1.69\\
1.18	1.69\\
1.18	1.69\\
1.18	1.69\\
1.18	1.69\\
1.18	1.69\\
1.18	1.69\\
1.18	1.69\\
1.18	1.69\\
1.18	1.69\\
1.18	1.69\\
1.18	1.69\\
1.18	1.69\\
1.18	1.69\\
1.18	1.69\\
1.18	1.69\\
1.18	1.69\\
1.18	1.69\\
1.18	1.69\\
1.18	1.69\\
1.18	1.69\\
1.18	1.69\\
1.18	1.69\\
1.18	1.69\\
1.18	1.69\\
1.18	1.69\\
1.18	1.69\\
1.18	1.69\\
1.18	1.69\\
1.18	1.69\\
1.18	1.69\\
1.18	1.69\\
1.18	1.69\\
1.18	1.69\\
1.18	1.69\\
1.18	1.69\\
1.18	1.69\\
1.18	1.69\\
1.18	1.69\\
1.18	1.69\\
1.18	1.69\\
1.18	1.69\\
1.18	1.69\\
1.18	1.69\\
1.18	1.69\\
1.18	1.69\\
1.18	1.69\\
1.18	1.69\\
1.18	1.69\\
1.18	1.69\\
1.18	1.69\\
1.18	1.69\\
1.18	1.69\\
1.18	1.69\\
1.18	1.69\\
1.18	1.69\\
1.18	1.69\\
1.18	1.69\\
1.18	1.69\\
1.18	1.69\\
1.18	1.69\\
1.18	1.69\\
1.18	1.69\\
1.18	1.69\\
1.18	1.69\\
1.18	1.69\\
1.18	1.69\\
1.18	1.69\\
1.18	1.69\\
1.18	1.69\\
1.18	1.69\\
1.18	1.69\\
1.18	1.69\\
1.19	1.69\\
1.19	1.69\\
1.19	1.69\\
1.19	1.69\\
1.19	1.69\\
1.19	1.69\\
1.19	1.69\\
1.19	1.69\\
1.19	1.69\\
1.19	1.69\\
1.19	1.69\\
1.19	1.69\\
1.19	1.69\\
1.19	1.69\\
1.19	1.69\\
1.19	1.69\\
1.19	1.69\\
1.19	1.69\\
1.19	1.69\\
1.19	1.69\\
1.19	1.69\\
1.19	1.69\\
1.19	1.69\\
1.19	1.69\\
1.19	1.69\\
1.19	1.69\\
1.19	1.69\\
1.19	1.69\\
1.19	1.69\\
1.19	1.69\\
1.19	1.69\\
1.19	1.69\\
1.19	1.69\\
1.19	1.69\\
1.19	1.69\\
1.19	1.69\\
1.19	1.69\\
1.19	1.69\\
1.19	1.69\\
1.19	1.69\\
1.19	1.69\\
1.19	1.69\\
1.19	1.69\\
1.19	1.69\\
1.19	1.69\\
1.19	1.69\\
1.19	1.69\\
1.19	1.69\\
1.19	1.69\\
1.19	1.69\\
1.19	1.69\\
1.19	1.69\\
1.19	1.69\\
1.19	1.69\\
1.19	1.69\\
1.19	1.69\\
1.19	1.69\\
1.19	1.69\\
1.19	1.69\\
1.19	1.69\\
1.19	1.69\\
1.19	1.69\\
1.19	1.69\\
1.19	1.69\\
1.19	1.69\\
1.19	1.69\\
1.19	1.69\\
1.19	1.69\\
1.19	1.69\\
1.19	1.69\\
1.19	1.69\\
1.19	1.69\\
1.19	1.69\\
1.19	1.69\\
1.18	1.69\\
1.18	1.69\\
1.18	1.69\\
1.18	1.69\\
1.17	1.69\\
1.17	1.69\\
1.17	1.69\\
1.17	1.69\\
1.16	1.69\\
1.15	1.69\\
1.15	1.69\\
1.14	1.69\\
1.14	1.69\\
1.13	1.69\\
1.12	1.69\\
1.11	1.69\\
1.11	1.69\\
1.1	1.69\\
1.09	1.69\\
1.08	1.69\\
1.08	1.69\\
1.07	1.69\\
1.07	1.69\\
1.06	1.69\\
1.05	1.69\\
1.04	1.69\\
1.04	1.69\\
1.03	1.69\\
1.02	1.69\\
1.01	1.69\\
1.01	1.69\\
1	1.69\\
1	1.69\\
0.99	1.69\\
0.99	1.69\\
0.98	1.7\\
0.97	1.7\\
0.97	1.69\\
0.97	1.69\\
0.96	1.69\\
0.96	1.69\\
0.95	1.69\\
0.95	1.69\\
0.95	1.69\\
0.94	1.69\\
0.94	1.69\\
0.94	1.7\\
0.93	1.7\\
0.93	1.7\\
0.93	1.7\\
0.92	1.7\\
0.92	1.7\\
0.92	1.7\\
0.91	1.7\\
0.91	1.7\\
0.9	1.7\\
0.9	1.7\\
0.89	1.7\\
0.89	1.7\\
0.89	1.7\\
0.88	1.7\\
0.88	1.7\\
0.87	1.7\\
0.87	1.7\\
0.86	1.7\\
0.86	1.7\\
0.86	1.7\\
0.85	1.7\\
0.85	1.7\\
0.84	1.7\\
0.84	1.7\\
0.83	1.7\\
0.83	1.7\\
0.83	1.7\\
0.82	1.7\\
0.82	1.7\\
0.81	1.7\\
0.81	1.7\\
0.8	1.7\\
0.8	1.7\\
0.79	1.7\\
0.79	1.7\\
0.79	1.7\\
0.78	1.7\\
0.78	1.7\\
0.77	1.7\\
0.77	1.7\\
0.76	1.7\\
0.76	1.7\\
0.76	1.7\\
0.75	1.7\\
0.75	1.7\\
0.74	1.7\\
0.74	1.7\\
0.74	1.71\\
0.74	1.71\\
0.73	1.71\\
0.73	1.71\\
0.72	1.71\\
0.72	1.71\\
0.72	1.7\\
0.71	1.7\\
0.71	1.71\\
0.7	1.71\\
0.7	1.7\\
0.7	1.71\\
0.69	1.71\\
0.69	1.71\\
0.68	1.71\\
0.68	1.71\\
0.68	1.71\\
0.67	1.71\\
0.67	1.71\\
0.66	1.71\\
0.66	1.71\\
0.65	1.71\\
0.65	1.71\\
0.65	1.71\\
0.64	1.71\\
0.64	1.71\\
0.64	1.71\\
0.63	1.71\\
0.63	1.71\\
0.62	1.71\\
0.62	1.71\\
0.61	1.71\\
0.61	1.71\\
0.6	1.71\\
0.6	1.71\\
0.6	1.71\\
0.59	1.71\\
0.59	1.71\\
0.59	1.71\\
0.58	1.71\\
0.58	1.71\\
0.57	1.71\\
0.57	1.71\\
0.57	1.71\\
0.56	1.71\\
0.56	1.71\\
0.56	1.71\\
0.55	1.71\\
0.55	1.71\\
0.55	1.71\\
0.54	1.71\\
0.54	1.71\\
0.54	1.71\\
0.54	1.71\\
0.53	1.71\\
0.53	1.71\\
0.52	1.71\\
0.51	1.71\\
0.52	1.71\\
0.52	1.71\\
0.52	1.71\\
0.51	1.71\\
0.51	1.71\\
0.51	1.71\\
0.51	1.71\\
0.5	1.72\\
0.5	1.72\\
0.5	1.72\\
0.5	1.72\\
0.49	1.72\\
0.49	1.72\\
0.48	1.72\\
0.48	1.72\\
0.48	1.72\\
0.47	1.72\\
0.47	1.72\\
0.47	1.72\\
0.47	1.72\\
0.46	1.72\\
0.46	1.72\\
0.45	1.72\\
0.45	1.72\\
0.45	1.72\\
0.44	1.72\\
0.44	1.72\\
0.44	1.72\\
0.43	1.72\\
0.43	1.72\\
0.43	1.72\\
0.42	1.72\\
0.42	1.72\\
0.41	1.72\\
0.41	1.72\\
0.41	1.72\\
0.4	1.72\\
0.4	1.72\\
0.4	1.72\\
0.39	1.72\\
0.39	1.72\\
0.39	1.72\\
0.38	1.72\\
0.38	1.72\\
0.38	1.72\\
0.38	1.72\\
0.38	1.72\\
0.37	1.72\\
0.37	1.72\\
0.37	1.72\\
0.37	1.72\\
0.36	1.72\\
0.36	1.72\\
0.36	1.72\\
0.36	1.72\\
0.36	1.72\\
0.35	1.72\\
0.35	1.72\\
0.35	1.72\\
0.35	1.72\\
0.35	1.72\\
0.34	1.72\\
0.34	1.72\\
0.34	1.72\\
0.34	1.72\\
0.34	1.72\\
0.33	1.72\\
0.33	1.72\\
0.33	1.72\\
0.33	1.72\\
0.33	1.72\\
0.32	1.72\\
0.32	1.72\\
0.32	1.72\\
0.31	1.72\\
0.31	1.72\\
0.31	1.72\\
0.31	1.72\\
0.3	1.72\\
0.3	1.73\\
0.29	1.72\\
0.29	1.73\\
0.29	1.73\\
0.29	1.73\\
0.28	1.73\\
0.28	1.73\\
0.28	1.73\\
0.27	1.73\\
0.27	1.72\\
0.27	1.73\\
0.26	1.73\\
0.26	1.73\\
0.26	1.73\\
0.26	1.73\\
0.26	1.73\\
0.25	1.73\\
0.25	1.73\\
0.25	1.73\\
0.24	1.73\\
0.24	1.73\\
0.24	1.73\\
0.24	1.73\\
0.23	1.73\\
0.23	1.73\\
0.23	1.73\\
0.23	1.73\\
0.23	1.73\\
0.22	1.73\\
0.22	1.73\\
0.22	1.73\\
0.22	1.73\\
0.22	1.73\\
0.22	1.73\\
0.21	1.73\\
0.21	1.73\\
0.21	1.73\\
0.21	1.73\\
0.21	1.73\\
0.21	1.73\\
0.21	1.73\\
0.2	1.73\\
0.2	1.73\\
0.2	1.73\\
0.2	1.73\\
0.2	1.73\\
0.19	1.73\\
0.19	1.73\\
0.19	1.73\\
0.19	1.73\\
0.19	1.73\\
0.19	1.73\\
0.18	1.73\\
0.18	1.73\\
0.18	1.73\\
0.18	1.73\\
0.17	1.73\\
0.17	1.73\\
0.17	1.73\\
0.17	1.73\\
0.16	1.73\\
0.16	1.73\\
0.16	1.73\\
0.16	1.73\\
0.16	1.73\\
0.15	1.73\\
0.15	1.73\\
0.15	1.73\\
0.15	1.73\\
0.14	1.73\\
0.14	1.73\\
0.14	1.73\\
0.14	1.73\\
0.13	1.73\\
0.13	1.73\\
0.13	1.73\\
0.13	1.73\\
0.12	1.73\\
0.12	1.73\\
0.12	1.73\\
0.12	1.73\\
0.12	1.73\\
0.12	1.73\\
0.11	1.73\\
0.11	1.74\\
0.11	1.73\\
0.11	1.74\\
0.11	1.73\\
0.11	1.73\\
0.11	1.73\\
0.1	1.73\\
0.1	1.73\\
0.1	1.73\\
0.1	1.74\\
0.1	1.74\\
0.1	1.74\\
0.09	1.73\\
0.09	1.74\\
0.09	1.74\\
0.09	1.74\\
0.09	1.74\\
0.09	1.74\\
0.09	1.74\\
0.09	1.74\\
0.09	1.74\\
0.08	1.74\\
0.08	1.74\\
0.08	1.74\\
0.08	1.74\\
0.08	1.74\\
0.08	1.74\\
0.07	1.74\\
0.07	1.74\\
0.07	1.74\\
0.07	1.74\\
0.07	1.74\\
0.07	1.74\\
0.06	1.74\\
0.06	1.74\\
0.06	1.74\\
0.06	1.74\\
0.06	1.74\\
0.05	1.74\\
0.05	1.74\\
0.05	1.74\\
0.05	1.74\\
0.05	1.74\\
0.04	1.74\\
0.04	1.74\\
0.04	1.74\\
0.04	1.74\\
0.04	1.74\\
0.03	1.74\\
0.03	1.74\\
0.03	1.74\\
0.03	1.74\\
0.03	1.74\\
0.03	1.74\\
0.02	1.74\\
0.02	1.74\\
0.02	1.74\\
0.02	1.74\\
0.02	1.74\\
0.02	1.74\\
0.02	1.74\\
0.01	1.74\\
0.01	1.74\\
0.01	1.74\\
0.01	1.74\\
0.01	1.74\\
0.01	1.74\\
0.01	1.74\\
0.01	1.74\\
0	1.74\\
0	1.74\\
0	1.74\\
0	1.74\\
0	1.74\\
0	1.74\\
-0	1.74\\
-0	1.74\\
-0	1.74\\
-0	1.74\\
-0.01	1.74\\
-0	1.74\\
-0.01	1.74\\
-0.01	1.74\\
-0.01	1.74\\
-0.01	1.74\\
-0.01	1.74\\
-0.01	1.74\\
-0.01	1.74\\
-0.02	1.74\\
-0.02	1.74\\
-0.02	1.74\\
-0.02	1.74\\
-0.02	1.74\\
-0.02	1.74\\
-0.02	1.74\\
-0.03	1.74\\
-0.03	1.74\\
-0.03	1.74\\
-0.03	1.74\\
-0.03	1.74\\
-0.03	1.74\\
-0.03	1.74\\
-0.04	1.74\\
-0.04	1.74\\
-0.04	1.74\\
-0.04	1.74\\
-0.04	1.74\\
-0.04	1.74\\
-0.05	1.74\\
-0.05	1.74\\
-0.05	1.74\\
-0.05	1.74\\
-0.05	1.74\\
-0.05	1.74\\
-0.05	1.74\\
-0.05	1.74\\
-0.05	1.74\\
-0.06	1.74\\
-0.06	1.74\\
-0.06	1.74\\
-0.06	1.74\\
-0.06	1.74\\
-0.06	1.74\\
-0.06	1.74\\
-0.06	1.74\\
-0.06	1.74\\
-0.07	1.74\\
-0.06	1.74\\
-0.07	1.74\\
-0.07	1.74\\
-0.07	1.74\\
-0.07	1.74\\
-0.07	1.74\\
-0.07	1.74\\
-0.07	1.74\\
-0.07	1.74\\
-0.07	1.74\\
-0.07	1.74\\
-0.08	1.74\\
-0.08	1.74\\
-0.08	1.74\\
-0.08	1.74\\
-0.08	1.74\\
-0.08	1.74\\
-0.08	1.74\\
-0.08	1.74\\
-0.08	1.74\\
-0.08	1.74\\
-0.09	1.74\\
-0.09	1.74\\
-0.09	1.74\\
-0.09	1.74\\
-0.09	1.74\\
-0.09	1.74\\
-0.09	1.74\\
-0.09	1.74\\
-0.1	1.74\\
-0.1	1.74\\
-0.1	1.74\\
-0.1	1.74\\
-0.1	1.74\\
-0.1	1.74\\
-0.1	1.74\\
-0.1	1.74\\
-0.11	1.74\\
-0.11	1.75\\
-0.11	1.75\\
-0.11	1.75\\
-0.11	1.75\\
-0.11	1.75\\
-0.11	1.75\\
-0.11	1.75\\
-0.12	1.75\\
-0.12	1.75\\
-0.12	1.75\\
-0.12	1.75\\
-0.12	1.75\\
-0.12	1.75\\
-0.12	1.75\\
-0.12	1.75\\
-0.12	1.75\\
-0.12	1.74\\
-0.12	1.74\\
-0.12	1.75\\
-0.12	1.75\\
-0.13	1.75\\
-0.13	1.75\\
-0.13	1.75\\
-0.13	1.75\\
-0.13	1.75\\
-0.13	1.75\\
-0.13	1.75\\
-0.13	1.75\\
-0.13	1.75\\
-0.13	1.75\\
-0.13	1.75\\
-0.13	1.75\\
-0.14	1.75\\
-0.14	1.75\\
-0.14	1.75\\
-0.14	1.75\\
-0.14	1.75\\
-0.14	1.75\\
-0.14	1.75\\
-0.14	1.75\\
-0.14	1.75\\
-0.14	1.75\\
-0.14	1.75\\
-0.14	1.75\\
-0.14	1.75\\
-0.14	1.75\\
-0.14	1.75\\
-0.14	1.75\\
-0.14	1.75\\
-0.15	1.75\\
-0.15	1.75\\
-0.15	1.75\\
-0.15	1.75\\
-0.15	1.75\\
-0.15	1.75\\
-0.15	1.75\\
-0.15	1.75\\
-0.15	1.75\\
-0.15	1.75\\
-0.16	1.75\\
-0.16	1.75\\
-0.16	1.75\\
-0.16	1.75\\
-0.16	1.75\\
-0.16	1.75\\
-0.16	1.75\\
-0.16	1.75\\
-0.16	1.75\\
-0.17	1.75\\
-0.17	1.75\\
-0.17	1.75\\
-0.17	1.75\\
-0.17	1.75\\
-0.17	1.75\\
-0.17	1.75\\
-0.17	1.75\\
-0.17	1.75\\
-0.17	1.75\\
-0.17	1.75\\
-0.17	1.75\\
-0.17	1.75\\
-0.17	1.75\\
-0.17	1.75\\
-0.17	1.75\\
-0.18	1.75\\
-0.17	1.75\\
-0.18	1.75\\
-0.18	1.75\\
-0.18	1.75\\
-0.18	1.75\\
-0.18	1.75\\
-0.18	1.75\\
-0.18	1.75\\
-0.18	1.75\\
-0.18	1.75\\
-0.18	1.75\\
-0.18	1.75\\
-0.18	1.75\\
-0.18	1.75\\
-0.18	1.75\\
-0.18	1.75\\
-0.18	1.75\\
-0.18	1.75\\
-0.19	1.75\\
-0.19	1.75\\
-0.19	1.75\\
-0.19	1.75\\
-0.19	1.75\\
-0.19	1.75\\
-0.19	1.75\\
-0.19	1.75\\
-0.19	1.75\\
-0.19	1.75\\
-0.19	1.75\\
-0.19	1.75\\
-0.19	1.75\\
-0.2	1.75\\
-0.2	1.75\\
-0.2	1.75\\
-0.2	1.75\\
-0.2	1.75\\
-0.2	1.75\\
-0.2	1.75\\
-0.2	1.75\\
-0.2	1.75\\
-0.2	1.75\\
-0.2	1.75\\
-0.2	1.75\\
-0.2	1.75\\
-0.2	1.75\\
-0.21	1.75\\
-0.21	1.75\\
-0.21	1.75\\
-0.21	1.75\\
-0.21	1.75\\
-0.21	1.75\\
-0.21	1.75\\
-0.21	1.75\\
-0.21	1.75\\
-0.21	1.75\\
-0.21	1.75\\
-0.21	1.75\\
-0.21	1.75\\
-0.21	1.75\\
-0.21	1.75\\
-0.21	1.75\\
-0.22	1.75\\
-0.22	1.75\\
-0.22	1.75\\
-0.22	1.75\\
-0.22	1.75\\
-0.22	1.75\\
-0.22	1.75\\
-0.22	1.75\\
-0.22	1.75\\
-0.22	1.75\\
-0.22	1.75\\
-0.22	1.75\\
-0.22	1.75\\
-0.22	1.75\\
-0.22	1.75\\
-0.22	1.75\\
-0.22	1.75\\
-0.22	1.75\\
-0.22	1.75\\
-0.22	1.75\\
-0.22	1.75\\
-0.22	1.75\\
-0.22	1.75\\
-0.22	1.75\\
-0.22	1.75\\
-0.22	1.75\\
-0.22	1.75\\
-0.23	1.75\\
-0.23	1.75\\
-0.23	1.75\\
-0.23	1.75\\
-0.23	1.75\\
-0.23	1.75\\
-0.23	1.75\\
-0.23	1.75\\
-0.23	1.75\\
-0.23	1.75\\
-0.23	1.75\\
-0.23	1.75\\
-0.23	1.75\\
-0.24	1.75\\
-0.24	1.75\\
-0.24	1.75\\
-0.24	1.75\\
-0.24	1.75\\
-0.24	1.75\\
-0.24	1.75\\
-0.24	1.75\\
-0.24	1.75\\
-0.24	1.75\\
-0.24	1.75\\
-0.24	1.75\\
-0.24	1.75\\
-0.24	1.75\\
-0.24	1.75\\
-0.24	1.75\\
-0.24	1.75\\
-0.25	1.75\\
-0.25	1.75\\
-0.25	1.75\\
-0.25	1.75\\
-0.25	1.75\\
-0.25	1.75\\
-0.25	1.75\\
-0.25	1.75\\
-0.25	1.75\\
-0.25	1.75\\
-0.25	1.75\\
-0.25	1.75\\
-0.25	1.75\\
-0.25	1.75\\
-0.25	1.75\\
-0.25	1.75\\
-0.25	1.75\\
-0.25	1.75\\
-0.25	1.75\\
-0.25	1.75\\
-0.25	1.75\\
-0.25	1.75\\
-0.25	1.75\\
-0.25	1.75\\
-0.25	1.75\\
-0.25	1.75\\
-0.25	1.75\\
-0.25	1.75\\
-0.25	1.75\\
-0.25	1.75\\
-0.25	1.75\\
-0.25	1.75\\
-0.25	1.75\\
-0.25	1.75\\
-0.25	1.75\\
-0.25	1.75\\
-0.25	1.75\\
-0.26	1.75\\
-0.26	1.75\\
-0.26	1.75\\
-0.26	1.75\\
-0.26	1.75\\
-0.26	1.75\\
-0.26	1.75\\
-0.26	1.75\\
-0.26	1.75\\
-0.26	1.75\\
-0.26	1.75\\
-0.27	1.75\\
-0.27	1.75\\
-0.27	1.75\\
-0.27	1.75\\
-0.27	1.75\\
-0.27	1.76\\
-0.27	1.76\\
-0.27	1.76\\
-0.27	1.76\\
-0.27	1.76\\
-0.27	1.76\\
-0.27	1.76\\
-0.27	1.76\\
-0.27	1.76\\
-0.27	1.76\\
-0.27	1.76\\
-0.27	1.76\\
-0.27	1.76\\
-0.27	1.76\\
-0.27	1.76\\
-0.27	1.76\\
-0.27	1.76\\
-0.27	1.76\\
-0.27	1.76\\
-0.27	1.75\\
-0.27	1.75\\
-0.27	1.75\\
-0.27	1.75\\
-0.27	1.75\\
-0.27	1.75\\
-0.27	1.75\\
-0.27	1.75\\
-0.27	1.75\\
-0.27	1.75\\
-0.27	1.75\\
-0.27	1.75\\
-0.27	1.75\\
-0.27	1.75\\
-0.27	1.75\\
-0.27	1.75\\
-0.27	1.75\\
-0.27	1.75\\
-0.27	1.75\\
-0.27	1.75\\
-0.27	1.75\\
-0.27	1.75\\
-0.27	1.75\\
-0.27	1.75\\
-0.27	1.75\\
-0.27	1.75\\
-0.27	1.75\\
-0.27	1.75\\
-0.27	1.75\\
-0.27	1.75\\
-0.27	1.75\\
-0.27	1.75\\
-0.27	1.75\\
-0.27	1.75\\
-0.27	1.75\\
-0.27	1.75\\
-0.27	1.75\\
-0.27	1.75\\
-0.27	1.75\\
-0.27	1.75\\
-0.27	1.75\\
-0.27	1.75\\
-0.27	1.75\\
-0.27	1.75\\
-0.27	1.76\\
-0.27	1.75\\
-0.27	1.75\\
-0.27	1.75\\
-0.28	1.76\\
-0.27	1.75\\
-0.28	1.75\\
-0.28	1.75\\
-0.28	1.76\\
-0.28	1.75\\
-0.28	1.75\\
-0.28	1.75\\
-0.28	1.75\\
-0.28	1.75\\
-0.28	1.75\\
-0.28	1.75\\
-0.28	1.76\\
-0.28	1.76\\
-0.28	1.76\\
-0.28	1.76\\
-0.28	1.76\\
-0.28	1.76\\
-0.29	1.76\\
-0.29	1.76\\
-0.29	1.76\\
-0.29	1.76\\
-0.29	1.76\\
-0.29	1.76\\
-0.29	1.76\\
-0.29	1.76\\
-0.29	1.76\\
-0.29	1.76\\
-0.29	1.76\\
-0.29	1.76\\
-0.29	1.76\\
-0.29	1.76\\
-0.29	1.76\\
-0.29	1.76\\
-0.29	1.76\\
-0.29	1.76\\
-0.29	1.76\\
-0.29	1.76\\
-0.29	1.76\\
-0.29	1.76\\
-0.29	1.76\\
-0.29	1.76\\
-0.29	1.76\\
-0.29	1.76\\
-0.29	1.76\\
-0.29	1.76\\
-0.29	1.76\\
-0.29	1.76\\
-0.29	1.76\\
-0.29	1.76\\
-0.29	1.76\\
-0.29	1.76\\
-0.29	1.76\\
-0.29	1.76\\
-0.29	1.76\\
-0.29	1.76\\
-0.29	1.76\\
-0.29	1.76\\
-0.29	1.76\\
-0.29	1.76\\
-0.29	1.76\\
-0.29	1.76\\
-0.29	1.76\\
-0.29	1.76\\
-0.29	1.76\\
-0.29	1.76\\
-0.29	1.76\\
-0.29	1.76\\
-0.29	1.76\\
-0.29	1.76\\
-0.29	1.76\\
-0.29	1.76\\
-0.29	1.76\\
-0.29	1.76\\
-0.29	1.76\\
-0.29	1.76\\
-0.29	1.76\\
-0.3	1.76\\
-0.29	1.76\\
-0.3	1.76\\
-0.3	1.76\\
-0.3	1.76\\
-0.3	1.76\\
-0.3	1.76\\
-0.3	1.76\\
-0.3	1.76\\
-0.3	1.76\\
-0.3	1.76\\
-0.3	1.76\\
-0.3	1.76\\
-0.3	1.76\\
-0.3	1.76\\
-0.3	1.76\\
-0.3	1.76\\
-0.3	1.76\\
-0.3	1.76\\
-0.3	1.76\\
-0.3	1.76\\
-0.31	1.76\\
-0.31	1.76\\
-0.31	1.76\\
-0.31	1.76\\
-0.31	1.76\\
-0.31	1.76\\
-0.31	1.76\\
-0.31	1.76\\
-0.31	1.76\\
-0.31	1.76\\
-0.31	1.76\\
-0.31	1.76\\
-0.31	1.76\\
-0.31	1.76\\
-0.31	1.76\\
-0.31	1.76\\
-0.31	1.76\\
-0.31	1.76\\
-0.31	1.76\\
-0.31	1.76\\
-0.31	1.76\\
-0.31	1.76\\
-0.31	1.76\\
-0.31	1.76\\
-0.31	1.76\\
-0.31	1.76\\
-0.31	1.76\\
-0.31	1.76\\
-0.31	1.76\\
-0.31	1.76\\
-0.31	1.76\\
-0.31	1.76\\
-0.31	1.76\\
-0.31	1.76\\
-0.31	1.76\\
-0.31	1.76\\
-0.31	1.76\\
-0.31	1.76\\
-0.31	1.76\\
-0.31	1.76\\
-0.31	1.76\\
-0.31	1.76\\
-0.31	1.76\\
-0.31	1.76\\
-0.31	1.76\\
-0.31	1.76\\
-0.31	1.76\\
-0.31	1.76\\
-0.31	1.76\\
-0.31	1.76\\
-0.31	1.76\\
-0.31	1.76\\
-0.31	1.76\\
-0.31	1.76\\
-0.31	1.76\\
-0.31	1.76\\
-0.31	1.76\\
-0.31	1.76\\
-0.31	1.76\\
-0.31	1.76\\
-0.31	1.76\\
-0.31	1.76\\
-0.31	1.76\\
-0.31	1.76\\
-0.31	1.76\\
-0.31	1.76\\
-0.31	1.76\\
-0.31	1.76\\
-0.31	1.76\\
-0.31	1.76\\
-0.31	1.76\\
-0.31	1.76\\
-0.31	1.76\\
-0.31	1.76\\
-0.31	1.76\\
-0.31	1.76\\
-0.31	1.76\\
-0.31	1.76\\
-0.31	1.76\\
-0.31	1.76\\
-0.31	1.76\\
-0.31	1.76\\
-0.31	1.76\\
-0.31	1.76\\
-0.31	1.76\\
-0.31	1.76\\
-0.31	1.76\\
-0.31	1.76\\
-0.31	1.76\\
-0.31	1.76\\
-0.31	1.76\\
-0.31	1.76\\
-0.31	1.76\\
-0.31	1.76\\
-0.31	1.76\\
-0.31	1.76\\
-0.31	1.76\\
-0.31	1.76\\
-0.31	1.76\\
-0.31	1.76\\
-0.31	1.76\\
-0.31	1.76\\
-0.31	1.76\\
-0.31	1.76\\
-0.31	1.76\\
-0.31	1.76\\
-0.31	1.76\\
-0.31	1.76\\
-0.31	1.76\\
-0.31	1.76\\
-0.31	1.76\\
-0.31	1.76\\
-0.31	1.76\\
-0.31	1.76\\
-0.31	1.76\\
-0.31	1.76\\
-0.31	1.76\\
-0.31	1.76\\
-0.31	1.76\\
-0.31	1.76\\
-0.31	1.76\\
-0.31	1.76\\
-0.31	1.76\\
-0.31	1.76\\
-0.31	1.76\\
-0.31	1.76\\
-0.31	1.76\\
-0.31	1.76\\
-0.31	1.76\\
-0.31	1.76\\
-0.31	1.76\\
-0.31	1.76\\
-0.31	1.76\\
-0.31	1.76\\
-0.31	1.76\\
-0.31	1.76\\
-0.31	1.76\\
-0.31	1.76\\
-0.31	1.76\\
-0.31	1.76\\
-0.31	1.76\\
-0.31	1.76\\
-0.31	1.76\\
-0.31	1.76\\
-0.31	1.76\\
-0.31	1.76\\
-0.31	1.76\\
-0.31	1.76\\
-0.31	1.76\\
-0.31	1.76\\
-0.31	1.76\\
-0.31	1.76\\
-0.31	1.76\\
-0.31	1.76\\
-0.31	1.76\\
-0.31	1.76\\
-0.31	1.76\\
-0.31	1.76\\
-0.31	1.76\\
-0.31	1.76\\
-0.31	1.76\\
-0.31	1.76\\
-0.31	1.76\\
-0.31	1.76\\
-0.31	1.76\\
-0.31	1.76\\
-0.31	1.76\\
-0.31	1.76\\
-0.31	1.76\\
-0.31	1.76\\
-0.31	1.76\\
-0.31	1.76\\
-0.31	1.76\\
-0.31	1.76\\
-0.31	1.76\\
-0.31	1.76\\
-0.31	1.76\\
-0.31	1.76\\
-0.31	1.76\\
-0.31	1.76\\
-0.31	1.76\\
-0.31	1.76\\
-0.31	1.76\\
-0.31	1.76\\
-0.31	1.76\\
-0.31	1.76\\
-0.31	1.76\\
-0.31	1.76\\
-0.31	1.76\\
-0.31	1.76\\
-0.31	1.76\\
-0.31	1.76\\
-0.31	1.76\\
-0.31	1.76\\
-0.31	1.76\\
-0.31	1.76\\
-0.31	1.76\\
-0.31	1.76\\
-0.31	1.76\\
-0.31	1.76\\
-0.31	1.76\\
-0.31	1.76\\
-0.31	1.76\\
-0.31	1.76\\
-0.31	1.76\\
-0.31	1.76\\
-0.31	1.76\\
-0.31	1.76\\
-0.31	1.76\\
-0.31	1.76\\
-0.31	1.76\\
-0.31	1.76\\
-0.31	1.76\\
-0.31	1.76\\
-0.31	1.76\\
-0.31	1.76\\
-0.31	1.76\\
-0.31	1.76\\
-0.31	1.76\\
-0.31	1.76\\
-0.31	1.76\\
-0.31	1.76\\
-0.31	1.76\\
-0.31	1.76\\
-0.31	1.76\\
-0.31	1.76\\
-0.31	1.76\\
-0.31	1.76\\
-0.31	1.76\\
-0.31	1.76\\
-0.31	1.76\\
-0.31	1.76\\
-0.31	1.76\\
-0.31	1.76\\
-0.31	1.76\\
-0.31	1.76\\
-0.31	1.76\\
-0.31	1.76\\
-0.31	1.76\\
-0.31	1.76\\
-0.31	1.76\\
-0.31	1.76\\
-0.31	1.76\\
-0.31	1.76\\
-0.31	1.76\\
-0.31	1.76\\
-0.31	1.76\\
-0.31	1.76\\
-0.31	1.76\\
-0.31	1.76\\
-0.31	1.76\\
-0.31	1.76\\
-0.31	1.76\\
-0.31	1.76\\
-0.31	1.76\\
-0.31	1.76\\
-0.31	1.76\\
-0.31	1.76\\
-0.31	1.76\\
-0.31	1.76\\
-0.31	1.76\\
-0.31	1.76\\
-0.31	1.76\\
-0.31	1.76\\
-0.31	1.76\\
-0.31	1.76\\
-0.31	1.76\\
-0.31	1.76\\
-0.31	1.76\\
-0.31	1.76\\
-0.31	1.76\\
-0.31	1.76\\
-0.31	1.76\\
-0.31	1.76\\
-0.31	1.76\\
-0.31	1.76\\
-0.31	1.76\\
-0.31	1.76\\
-0.31	1.76\\
-0.31	1.76\\
-0.31	1.76\\
-0.31	1.76\\
-0.31	1.76\\
-0.31	1.76\\
-0.31	1.76\\
-0.31	1.76\\
-0.31	1.76\\
-0.31	1.76\\
-0.31	1.76\\
-0.31	1.76\\
-0.31	1.76\\
-0.31	1.76\\
-0.31	1.76\\
-0.31	1.76\\
-0.31	1.76\\
-0.31	1.76\\
-0.31	1.76\\
-0.31	1.76\\
-0.31	1.76\\
-0.31	1.76\\
-0.31	1.76\\
-0.31	1.76\\
-0.31	1.76\\
-0.31	1.76\\
-0.31	1.76\\
-0.31	1.76\\
-0.31	1.76\\
-0.31	1.76\\
-0.31	1.76\\
-0.31	1.76\\
-0.31	1.76\\
-0.31	1.76\\
-0.31	1.76\\
-0.31	1.76\\
-0.31	1.76\\
-0.31	1.76\\
-0.31	1.76\\
-0.31	1.76\\
-0.31	1.76\\
-0.31	1.76\\
-0.31	1.76\\
-0.31	1.76\\
-0.31	1.76\\
-0.31	1.76\\
-0.31	1.76\\
-0.31	1.76\\
-0.31	1.76\\
-0.31	1.76\\
-0.31	1.76\\
-0.31	1.76\\
-0.31	1.76\\
-0.31	1.76\\
-0.31	1.76\\
-0.31	1.76\\
-0.31	1.76\\
-0.31	1.76\\
-0.31	1.76\\
-0.31	1.76\\
-0.31	1.76\\
-0.31	1.76\\
-0.31	1.76\\
-0.31	1.76\\
-0.31	1.76\\
-0.31	1.76\\
-0.31	1.76\\
-0.31	1.76\\
-0.31	1.76\\
-0.31	1.76\\
-0.31	1.76\\
-0.31	1.76\\
-0.31	1.76\\
-0.31	1.76\\
-0.31	1.76\\
-0.31	1.76\\
-0.31	1.76\\
-0.31	1.76\\
-0.31	1.76\\
-0.31	1.76\\
-0.31	1.76\\
-0.31	1.76\\
-0.31	1.76\\
-0.31	1.76\\
-0.31	1.76\\
-0.31	1.76\\
-0.31	1.76\\
-0.31	1.76\\
-0.31	1.76\\
-0.31	1.76\\
-0.31	1.76\\
-0.31	1.76\\
-0.31	1.76\\
-0.31	1.76\\
-0.31	1.76\\
-0.31	1.76\\
-0.31	1.76\\
-0.31	1.76\\
-0.31	1.76\\
-0.31	1.76\\
-0.31	1.76\\
-0.31	1.76\\
-0.31	1.76\\
-0.31	1.76\\
-0.31	1.76\\
-0.31	1.76\\
-0.31	1.76\\
-0.31	1.76\\
-0.31	1.76\\
-0.31	1.76\\
-0.31	1.76\\
-0.31	1.76\\
-0.31	1.76\\
-0.31	1.76\\
-0.31	1.76\\
-0.31	1.76\\
-0.31	1.76\\
-0.31	1.76\\
-0.31	1.76\\
-0.31	1.76\\
-0.31	1.76\\
-0.31	1.76\\
-0.31	1.76\\
-0.31	1.76\\
-0.31	1.76\\
-0.32	1.76\\
-0.32	1.76\\
-0.32	1.76\\
-0.32	1.76\\
-0.32	1.76\\
-0.32	1.76\\
-0.32	1.76\\
-0.32	1.76\\
-0.32	1.76\\
-0.32	1.76\\
-0.32	1.76\\
-0.32	1.76\\
-0.32	1.76\\
-0.32	1.76\\
-0.32	1.76\\
-0.32	1.76\\
-0.32	1.76\\
-0.32	1.76\\
-0.32	1.76\\
-0.32	1.76\\
-0.32	1.76\\
-0.32	1.76\\
-0.32	1.76\\
-0.32	1.76\\
-0.32	1.76\\
-0.32	1.76\\
-0.32	1.76\\
-0.31	1.76\\
-0.31	1.76\\
-0.31	1.76\\
-0.31	1.76\\
-0.31	1.76\\
-0.31	1.76\\
-0.31	1.76\\
-0.31	1.76\\
-0.31	1.76\\
-0.31	1.76\\
-0.31	1.76\\
-0.31	1.76\\
-0.31	1.76\\
-0.31	1.76\\
-0.31	1.76\\
-0.31	1.76\\
-0.32	1.76\\
-0.32	1.76\\
-0.32	1.76\\
-0.32	1.76\\
-0.32	1.76\\
-0.32	1.76\\
-0.32	1.76\\
-0.32	1.76\\
-0.32	1.76\\
-0.32	1.76\\
-0.32	1.76\\
-0.32	1.76\\
-0.32	1.76\\
-0.32	1.76\\
-0.32	1.76\\
-0.32	1.76\\
-0.32	1.76\\
-0.32	1.76\\
-0.32	1.76\\
-0.32	1.76\\
-0.32	1.76\\
-0.32	1.76\\
-0.32	1.76\\
-0.32	1.76\\
-0.32	1.76\\
-0.32	1.76\\
-0.32	1.76\\
-0.33	1.76\\
-0.33	1.76\\
-0.33	1.76\\
-0.33	1.76\\
-0.33	1.76\\
-0.33	1.76\\
-0.33	1.76\\
-0.33	1.76\\
-0.33	1.76\\
-0.33	1.76\\
-0.33	1.76\\
-0.33	1.76\\
-0.33	1.76\\
-0.33	1.76\\
-0.33	1.76\\
-0.33	1.76\\
-0.33	1.76\\
-0.33	1.76\\
-0.33	1.76\\
-0.33	1.76\\
-0.33	1.76\\
-0.33	1.76\\
-0.33	1.76\\
-0.33	1.76\\
-0.33	1.76\\
-0.33	1.76\\
-0.33	1.76\\
-0.33	1.76\\
-0.33	1.76\\
-0.33	1.76\\
-0.33	1.76\\
-0.33	1.76\\
-0.33	1.76\\
-0.33	1.76\\
-0.33	1.76\\
-0.33	1.76\\
-0.33	1.76\\
-0.33	1.76\\
-0.33	1.76\\
-0.33	1.76\\
-0.33	1.76\\
-0.33	1.76\\
-0.33	1.76\\
-0.33	1.76\\
-0.33	1.76\\
-0.33	1.76\\
-0.33	1.76\\
-0.33	1.76\\
-0.33	1.76\\
-0.33	1.76\\
-0.33	1.76\\
-0.33	1.76\\
-0.33	1.76\\
-0.33	1.76\\
-0.33	1.76\\
-0.33	1.76\\
-0.33	1.76\\
-0.33	1.76\\
-0.33	1.76\\
-0.33	1.76\\
-0.33	1.76\\
-0.33	1.76\\
-0.33	1.76\\
-0.33	1.76\\
-0.33	1.76\\
-0.33	1.76\\
-0.33	1.76\\
-0.33	1.76\\
-0.33	1.76\\
-0.33	1.76\\
-0.33	1.76\\
-0.33	1.76\\
-0.33	1.76\\
-0.33	1.76\\
-0.33	1.76\\
-0.33	1.76\\
-0.33	1.76\\
-0.33	1.76\\
-0.33	1.76\\
-0.33	1.76\\
-0.33	1.76\\
-0.33	1.76\\
-0.33	1.76\\
-0.33	1.76\\
-0.33	1.76\\
-0.33	1.76\\
-0.33	1.76\\
-0.33	1.76\\
-0.33	1.76\\
-0.33	1.76\\
-0.33	1.76\\
-0.32	1.76\\
-0.33	1.76\\
-0.33	1.76\\
-0.33	1.76\\
-0.33	1.76\\
-0.33	1.76\\
-0.33	1.76\\
-0.33	1.76\\
-0.33	1.76\\
-0.33	1.76\\
-0.33	1.76\\
-0.33	1.76\\
-0.33	1.76\\
-0.33	1.76\\
-0.33	1.76\\
-0.33	1.76\\
-0.33	1.76\\
-0.33	1.76\\
-0.33	1.76\\
-0.33	1.76\\
-0.33	1.76\\
-0.33	1.76\\
-0.33	1.76\\
-0.33	1.76\\
-0.33	1.76\\
-0.33	1.76\\
-0.33	1.76\\
-0.33	1.76\\
-0.33	1.76\\
-0.33	1.76\\
-0.33	1.76\\
-0.33	1.76\\
-0.33	1.76\\
-0.33	1.76\\
-0.33	1.76\\
-0.33	1.76\\
-0.33	1.76\\
-0.33	1.76\\
-0.33	1.76\\
-0.33	1.76\\
-0.33	1.76\\
-0.33	1.76\\
-0.33	1.76\\
-0.33	1.76\\
-0.33	1.76\\
-0.33	1.76\\
-0.33	1.76\\
-0.33	1.76\\
-0.33	1.76\\
-0.33	1.76\\
-0.33	1.76\\
-0.33	1.76\\
-0.33	1.76\\
-0.33	1.76\\
-0.33	1.76\\
-0.33	1.76\\
-0.33	1.76\\
-0.33	1.76\\
-0.33	1.76\\
-0.33	1.76\\
-0.33	1.76\\
-0.33	1.76\\
-0.33	1.76\\
-0.33	1.76\\
-0.33	1.76\\
-0.33	1.76\\
-0.33	1.76\\
-0.33	1.76\\
-0.33	1.76\\
-0.33	1.76\\
-0.33	1.76\\
-0.33	1.76\\
-0.33	1.76\\
-0.33	1.76\\
-0.33	1.76\\
-0.33	1.76\\
-0.33	1.76\\
-0.33	1.76\\
-0.33	1.76\\
-0.33	1.76\\
-0.33	1.76\\
-0.33	1.76\\
-0.33	1.76\\
-0.33	1.76\\
-0.33	1.76\\
-0.33	1.76\\
-0.33	1.76\\
-0.33	1.76\\
-0.33	1.76\\
-0.33	1.76\\
-0.33	1.76\\
-0.33	1.76\\
-0.33	1.76\\
-0.33	1.76\\
-0.33	1.76\\
-0.33	1.76\\
-0.33	1.76\\
-0.33	1.76\\
-0.33	1.76\\
-0.33	1.76\\
-0.33	1.76\\
-0.33	1.76\\
-0.33	1.76\\
-0.33	1.76\\
-0.33	1.76\\
-0.33	1.76\\
-0.33	1.76\\
-0.33	1.76\\
-0.33	1.76\\
-0.33	1.76\\
-0.33	1.76\\
-0.33	1.76\\
-0.33	1.76\\
-0.33	1.76\\
-0.33	1.76\\
-0.33	1.76\\
-0.33	1.76\\
-0.33	1.76\\
-0.33	1.76\\
-0.33	1.76\\
-0.33	1.76\\
-0.33	1.76\\
-0.33	1.76\\
-0.33	1.76\\
-0.33	1.76\\
-0.33	1.76\\
-0.33	1.76\\
-0.33	1.76\\
-0.33	1.76\\
-0.33	1.76\\
-0.33	1.76\\
-0.33	1.76\\
-0.33	1.76\\
-0.33	1.76\\
-0.33	1.76\\
-0.33	1.76\\
-0.33	1.76\\
-0.33	1.76\\
-0.33	1.76\\
-0.33	1.76\\
-0.33	1.76\\
-0.33	1.76\\
-0.33	1.76\\
-0.33	1.76\\
-0.33	1.76\\
-0.33	1.76\\
-0.33	1.76\\
-0.33	1.76\\
-0.33	1.76\\
-0.33	1.76\\
-0.33	1.76\\
-0.33	1.76\\
-0.33	1.76\\
-0.33	1.76\\
-0.33	1.76\\
-0.33	1.76\\
-0.33	1.76\\
-0.33	1.76\\
-0.33	1.76\\
-0.33	1.76\\
-0.33	1.76\\
-0.33	1.76\\
-0.33	1.76\\
-0.33	1.76\\
-0.33	1.76\\
-0.33	1.76\\
-0.33	1.76\\
-0.33	1.76\\
-0.33	1.76\\
-0.33	1.76\\
-0.33	1.76\\
-0.33	1.76\\
-0.33	1.76\\
-0.33	1.76\\
-0.33	1.76\\
-0.33	1.76\\
-0.33	1.76\\
-0.33	1.76\\
-0.33	1.76\\
-0.33	1.76\\
-0.33	1.76\\
-0.33	1.76\\
-0.33	1.76\\
-0.33	1.76\\
-0.33	1.76\\
-0.33	1.76\\
-0.33	1.76\\
-0.33	1.76\\
-0.33	1.76\\
-0.33	1.76\\
-0.33	1.76\\
-0.33	1.76\\
-0.33	1.76\\
-0.33	1.76\\
-0.33	1.76\\
-0.33	1.76\\
-0.33	1.76\\
-0.33	1.76\\
-0.33	1.76\\
-0.33	1.76\\
-0.33	1.76\\
-0.33	1.76\\
-0.33	1.76\\
-0.33	1.76\\
-0.33	1.76\\
-0.33	1.76\\
-0.33	1.76\\
-0.33	1.76\\
-0.33	1.76\\
-0.33	1.76\\
-0.33	1.76\\
-0.33	1.76\\
-0.33	1.76\\
-0.33	1.76\\
-0.33	1.76\\
-0.33	1.76\\
-0.33	1.76\\
-0.33	1.76\\
-0.33	1.76\\
-0.33	1.76\\
-0.33	1.76\\
-0.33	1.76\\
-0.33	1.76\\
-0.33	1.76\\
-0.33	1.76\\
-0.33	1.76\\
-0.33	1.76\\
-0.33	1.76\\
-0.33	1.76\\
-0.33	1.76\\
-0.33	1.76\\
-0.33	1.76\\
-0.33	1.76\\
-0.33	1.76\\
-0.33	1.76\\
-0.33	1.76\\
-0.33	1.76\\
-0.33	1.76\\
-0.33	1.76\\
-0.33	1.76\\
-0.33	1.76\\
-0.33	1.76\\
-0.33	1.76\\
-0.33	1.76\\
-0.33	1.76\\
-0.33	1.76\\
-0.33	1.76\\
-0.33	1.76\\
-0.33	1.76\\
-0.33	1.76\\
-0.33	1.76\\
-0.33	1.76\\
-0.33	1.76\\
-0.33	1.76\\
-0.33	1.76\\
-0.33	1.76\\
-0.33	1.76\\
-0.33	1.76\\
-0.33	1.76\\
-0.33	1.76\\
-0.33	1.76\\
-0.33	1.76\\
-0.33	1.76\\
-0.33	1.76\\
-0.33	1.76\\
-0.33	1.76\\
-0.33	1.76\\
-0.33	1.76\\
-0.33	1.76\\
-0.33	1.76\\
-0.33	1.76\\
-0.33	1.76\\
-0.33	1.76\\
-0.33	1.76\\
-0.33	1.76\\
-0.33	1.76\\
-0.33	1.76\\
-0.33	1.76\\
-0.33	1.76\\
-0.33	1.76\\
-0.33	1.76\\
-0.33	1.76\\
-0.33	1.76\\
-0.33	1.76\\
-0.33	1.76\\
-0.33	1.76\\
-0.33	1.76\\
-0.33	1.76\\
-0.33	1.76\\
-0.33	1.76\\
-0.33	1.76\\
-0.33	1.76\\
-0.33	1.76\\
-0.33	1.76\\
-0.33	1.76\\
-0.33	1.76\\
-0.33	1.76\\
-0.33	1.76\\
-0.33	1.76\\
-0.33	1.76\\
-0.33	1.76\\
-0.33	1.76\\
-0.33	1.76\\
-0.33	1.76\\
-0.33	1.76\\
-0.33	1.76\\
-0.33	1.76\\
-0.33	1.76\\
-0.33	1.76\\
-0.33	1.76\\
-0.33	1.76\\
-0.33	1.76\\
-0.33	1.76\\
-0.33	1.76\\
-0.33	1.76\\
-0.33	1.76\\
-0.33	1.76\\
-0.33	1.76\\
-0.33	1.76\\
-0.33	1.76\\
-0.33	1.76\\
-0.33	1.76\\
-0.33	1.76\\
-0.33	1.76\\
-0.33	1.76\\
-0.33	1.76\\
-0.33	1.76\\
-0.33	1.76\\
-0.33	1.76\\
-0.33	1.76\\
-0.33	1.76\\
-0.33	1.76\\
-0.33	1.76\\
-0.33	1.76\\
-0.33	1.76\\
-0.33	1.76\\
-0.33	1.76\\
-0.33	1.76\\
-0.33	1.76\\
-0.33	1.76\\
-0.33	1.76\\
-0.33	1.76\\
-0.33	1.76\\
-0.33	1.76\\
-0.33	1.76\\
-0.33	1.76\\
-0.33	1.76\\
-0.33	1.76\\
-0.33	1.76\\
-0.33	1.76\\
-0.33	1.76\\
-0.33	1.76\\
-0.33	1.76\\
-0.33	1.76\\
-0.33	1.76\\
-0.33	1.76\\
-0.33	1.76\\
-0.33	1.76\\
-0.33	1.76\\
-0.33	1.76\\
-0.33	1.76\\
-0.33	1.76\\
-0.33	1.76\\
-0.33	1.76\\
-0.33	1.76\\
-0.33	1.76\\
-0.33	1.76\\
-0.33	1.76\\
-0.33	1.76\\
-0.33	1.76\\
-0.33	1.76\\
-0.33	1.76\\
-0.33	1.76\\
-0.33	1.76\\
-0.33	1.76\\
-0.33	1.76\\
-0.33	1.76\\
-0.33	1.76\\
-0.33	1.76\\
-0.33	1.76\\
-0.33	1.76\\
-0.33	1.76\\
-0.33	1.76\\
-0.33	1.76\\
-0.33	1.76\\
-0.33	1.76\\
-0.33	1.76\\
-0.33	1.76\\
-0.33	1.76\\
-0.33	1.76\\
-0.33	1.76\\
-0.33	1.76\\
-0.33	1.76\\
-0.33	1.76\\
-0.33	1.76\\
-0.33	1.76\\
-0.33	1.76\\
-0.33	1.76\\
-0.33	1.76\\
-0.33	1.76\\
-0.33	1.76\\
-0.33	1.76\\
-0.33	1.76\\
-0.33	1.76\\
-0.33	1.76\\
-0.33	1.76\\
-0.33	1.76\\
-0.33	1.76\\
-0.33	1.76\\
-0.33	1.76\\
-0.33	1.76\\
-0.33	1.76\\
-0.33	1.76\\
-0.33	1.76\\
-0.33	1.76\\
-0.33	1.76\\
-0.33	1.76\\
-0.33	1.76\\
-0.33	1.76\\
-0.33	1.76\\
-0.33	1.76\\
-0.33	1.76\\
-0.33	1.76\\
-0.33	1.76\\
-0.33	1.76\\
-0.33	1.76\\
-0.33	1.76\\
-0.33	1.76\\
-0.33	1.76\\
-0.33	1.76\\
-0.33	1.76\\
-0.33	1.76\\
-0.33	1.76\\
-0.33	1.76\\
-0.33	1.76\\
-0.33	1.76\\
-0.33	1.76\\
-0.33	1.76\\
-0.33	1.76\\
-0.33	1.76\\
-0.33	1.76\\
-0.33	1.76\\
-0.33	1.76\\
-0.33	1.76\\
-0.33	1.76\\
-0.33	1.76\\
-0.33	1.76\\
-0.33	1.76\\
-0.33	1.76\\
-0.33	1.76\\
-0.33	1.76\\
-0.33	1.76\\
-0.33	1.76\\
-0.33	1.76\\
-0.33	1.76\\
-0.33	1.76\\
-0.33	1.76\\
-0.33	1.76\\
-0.33	1.76\\
-0.33	1.76\\
-0.33	1.76\\
-0.33	1.76\\
-0.33	1.76\\
-0.33	1.76\\
-0.33	1.76\\
-0.33	1.76\\
-0.33	1.76\\
-0.33	1.76\\
-0.33	1.76\\
-0.33	1.76\\
-0.33	1.76\\
-0.33	1.76\\
-0.33	1.76\\
-0.33	1.76\\
-0.33	1.76\\
-0.33	1.76\\
-0.33	1.76\\
-0.33	1.76\\
-0.33	1.76\\
-0.33	1.76\\
-0.33	1.76\\
-0.33	1.76\\
-0.33	1.76\\
-0.33	1.76\\
-0.33	1.76\\
-0.33	1.76\\
-0.33	1.76\\
-0.33	1.76\\
-0.33	1.76\\
-0.33	1.76\\
-0.33	1.76\\
-0.33	1.76\\
-0.33	1.76\\
-0.33	1.76\\
-0.33	1.76\\
-0.33	1.76\\
-0.33	1.76\\
-0.33	1.76\\
-0.33	1.76\\
-0.33	1.76\\
-0.33	1.76\\
-0.33	1.76\\
-0.33	1.76\\
-0.33	1.76\\
-0.33	1.76\\
-0.33	1.76\\
-0.33	1.76\\
-0.33	1.76\\
-0.33	1.76\\
-0.33	1.76\\
-0.33	1.76\\
-0.33	1.76\\
-0.33	1.76\\
-0.33	1.76\\
-0.33	1.76\\
-0.33	1.76\\
-0.33	1.76\\
-0.33	1.76\\
-0.33	1.76\\
-0.33	1.76\\
-0.33	1.76\\
-0.33	1.76\\
-0.33	1.76\\
-0.33	1.76\\
-0.33	1.76\\
-0.33	1.76\\
-0.33	1.76\\
-0.33	1.76\\
-0.33	1.76\\
-0.33	1.76\\
-0.33	1.76\\
-0.33	1.76\\
-0.33	1.76\\
-0.33	1.76\\
-0.33	1.76\\
-0.33	1.76\\
-0.33	1.76\\
-0.33	1.76\\
-0.33	1.76\\
-0.33	1.76\\
-0.33	1.76\\
-0.33	1.76\\
-0.33	1.76\\
-0.33	1.76\\
-0.33	1.76\\
-0.33	1.76\\
-0.33	1.76\\
-0.33	1.76\\
-0.33	1.76\\
-0.33	1.76\\
-0.33	1.76\\
-0.33	1.76\\
-0.32	1.76\\
-0.33	1.76\\
-0.33	1.76\\
-0.33	1.76\\
-0.33	1.76\\
-0.33	1.76\\
-0.33	1.76\\
-0.33	1.76\\
-0.33	1.76\\
-0.33	1.76\\
-0.33	1.76\\
-0.33	1.76\\
-0.33	1.76\\
-0.33	1.76\\
-0.33	1.76\\
-0.32	1.76\\
-0.33	1.76\\
-0.33	1.76\\
-0.33	1.76\\
-0.33	1.76\\
-0.33	1.76\\
-0.33	1.76\\
-0.33	1.76\\
-0.33	1.76\\
-0.33	1.76\\
-0.33	1.76\\
-0.33	1.76\\
-0.33	1.76\\
-0.33	1.76\\
-0.33	1.76\\
-0.33	1.76\\
-0.33	1.76\\
-0.33	1.76\\
-0.33	1.76\\
-0.33	1.76\\
-0.33	1.76\\
-0.33	1.76\\
-0.33	1.76\\
-0.33	1.76\\
-0.33	1.76\\
-0.33	1.76\\
-0.33	1.76\\
-0.33	1.76\\
-0.33	1.76\\
-0.33	1.76\\
-0.33	1.76\\
-0.33	1.76\\
-0.33	1.76\\
-0.33	1.76\\
-0.33	1.76\\
-0.33	1.76\\
-0.33	1.76\\
-0.33	1.76\\
-0.33	1.76\\
-0.33	1.76\\
-0.33	1.76\\
-0.33	1.76\\
-0.33	1.76\\
-0.33	1.76\\
-0.33	1.76\\
-0.33	1.76\\
-0.33	1.76\\
-0.33	1.76\\
-0.33	1.76\\
-0.33	1.76\\
-0.33	1.76\\
-0.33	1.76\\
-0.33	1.76\\
-0.33	1.76\\
-0.33	1.76\\
-0.33	1.76\\
-0.33	1.76\\
-0.33	1.76\\
-0.33	1.76\\
-0.33	1.76\\
-0.33	1.76\\
-0.33	1.76\\
-0.33	1.76\\
-0.33	1.76\\
-0.33	1.76\\
-0.33	1.76\\
-0.33	1.76\\
-0.33	1.76\\
-0.33	1.76\\
-0.33	1.76\\
-0.33	1.76\\
-0.33	1.76\\
-0.33	1.76\\
-0.33	1.76\\
-0.33	1.76\\
-0.33	1.76\\
-0.33	1.76\\
-0.33	1.76\\
-0.33	1.76\\
-0.33	1.76\\
-0.33	1.76\\
-0.33	1.76\\
-0.33	1.76\\
-0.33	1.76\\
-0.33	1.76\\
-0.33	1.76\\
-0.33	1.76\\
-0.33	1.76\\
-0.33	1.76\\
-0.33	1.76\\
-0.33	1.76\\
-0.33	1.76\\
-0.33	1.76\\
-0.33	1.76\\
-0.33	1.76\\
-0.33	1.76\\
-0.33	1.76\\
-0.33	1.76\\
-0.33	1.76\\
-0.33	1.76\\
-0.33	1.76\\
-0.33	1.76\\
-0.33	1.76\\
-0.33	1.76\\
-0.33	1.76\\
-0.33	1.76\\
-0.33	1.76\\
-0.33	1.76\\
-0.33	1.76\\
-0.33	1.76\\
-0.33	1.76\\
-0.33	1.76\\
-0.33	1.76\\
-0.33	1.76\\
-0.33	1.76\\
-0.33	1.76\\
-0.33	1.76\\
-0.33	1.76\\
-0.33	1.76\\
-0.33	1.76\\
-0.33	1.76\\
-0.33	1.76\\
-0.33	1.76\\
-0.33	1.76\\
-0.33	1.76\\
-0.33	1.76\\
-0.33	1.76\\
-0.33	1.76\\
-0.33	1.76\\
-0.33	1.76\\
-0.33	1.76\\
-0.33	1.76\\
-0.33	1.76\\
-0.33	1.76\\
-0.33	1.76\\
-0.33	1.76\\
-0.33	1.76\\
-0.33	1.76\\
-0.33	1.76\\
-0.33	1.76\\
-0.33	1.76\\
-0.33	1.76\\
-0.33	1.76\\
-0.33	1.76\\
-0.33	1.76\\
-0.33	1.76\\
-0.33	1.76\\
-0.33	1.76\\
-0.33	1.76\\
-0.33	1.76\\
-0.33	1.76\\
-0.33	1.76\\
-0.33	1.76\\
-0.33	1.76\\
-0.33	1.76\\
-0.33	1.76\\
-0.33	1.76\\
-0.33	1.76\\
-0.33	1.76\\
-0.33	1.76\\
-0.33	1.76\\
-0.33	1.76\\
-0.33	1.76\\
-0.33	1.76\\
-0.33	1.76\\
-0.33	1.76\\
-0.33	1.76\\
-0.33	1.76\\
-0.33	1.76\\
-0.33	1.76\\
-0.33	1.76\\
-0.33	1.76\\
-0.33	1.76\\
-0.33	1.76\\
-0.33	1.76\\
-0.33	1.76\\
-0.33	1.76\\
-0.33	1.76\\
-0.33	1.76\\
-0.33	1.76\\
-0.33	1.76\\
-0.33	1.76\\
-0.33	1.76\\
-0.33	1.76\\
-0.33	1.76\\
-0.33	1.76\\
-0.33	1.76\\
-0.33	1.76\\
-0.33	1.76\\
-0.33	1.76\\
-0.33	1.76\\
-0.33	1.76\\
-0.33	1.76\\
-0.33	1.76\\
-0.33	1.76\\
-0.33	1.76\\
-0.33	1.76\\
-0.33	1.76\\
-0.33	1.76\\
-0.33	1.76\\
-0.33	1.76\\
-0.33	1.76\\
-0.33	1.76\\
-0.33	1.76\\
-0.33	1.76\\
-0.33	1.76\\
-0.33	1.76\\
-0.33	1.76\\
-0.33	1.76\\
-0.33	1.76\\
-0.33	1.76\\
-0.33	1.76\\
-0.33	1.76\\
-0.33	1.76\\
-0.33	1.76\\
-0.33	1.76\\
-0.33	1.76\\
-0.33	1.76\\
-0.33	1.76\\
-0.33	1.76\\
-0.33	1.76\\
-0.33	1.76\\
-0.33	1.76\\
-0.33	1.76\\
-0.33	1.76\\
-0.33	1.76\\
-0.33	1.76\\
-0.33	1.76\\
-0.33	1.76\\
-0.33	1.76\\
-0.33	1.76\\
-0.33	1.76\\
-0.33	1.76\\
-0.33	1.76\\
-0.33	1.76\\
-0.33	1.76\\
-0.33	1.76\\
-0.33	1.76\\
-0.33	1.76\\
-0.33	1.76\\
-0.33	1.76\\
-0.33	1.76\\
-0.33	1.76\\
-0.33	1.76\\
-0.33	1.76\\
-0.33	1.76\\
-0.33	1.76\\
-0.33	1.76\\
-0.33	1.76\\
-0.33	1.76\\
-0.33	1.76\\
-0.33	1.76\\
-0.33	1.76\\
-0.33	1.76\\
-0.33	1.76\\
-0.33	1.76\\
-0.33	1.76\\
-0.33	1.76\\
-0.33	1.76\\
-0.33	1.76\\
-0.33	1.76\\
-0.33	1.76\\
-0.33	1.76\\
-0.33	1.76\\
-0.33	1.76\\
-0.33	1.76\\
-0.33	1.76\\
-0.33	1.76\\
-0.33	1.76\\
-0.33	1.76\\
-0.33	1.76\\
-0.33	1.76\\
-0.33	1.76\\
-0.33	1.76\\
-0.33	1.76\\
-0.33	1.76\\
-0.33	1.76\\
-0.33	1.76\\
-0.33	1.76\\
-0.33	1.76\\
-0.33	1.76\\
-0.33	1.76\\
-0.33	1.76\\
-0.33	1.76\\
-0.33	1.76\\
-0.33	1.76\\
-0.33	1.76\\
-0.33	1.76\\
-0.33	1.76\\
-0.33	1.76\\
-0.33	1.76\\
-0.33	1.76\\
-0.33	1.76\\
-0.33	1.76\\
-0.33	1.76\\
-0.33	1.76\\
-0.33	1.76\\
-0.33	1.76\\
-0.33	1.76\\
-0.33	1.76\\
-0.33	1.76\\
-0.33	1.76\\
-0.33	1.76\\
-0.33	1.76\\
-0.33	1.76\\
-0.33	1.76\\
-0.33	1.76\\
-0.33	1.76\\
-0.33	1.76\\
-0.33	1.76\\
-0.33	1.76\\
-0.33	1.76\\
-0.33	1.76\\
-0.33	1.76\\
-0.33	1.76\\
-0.33	1.76\\
-0.33	1.76\\
-0.33	1.76\\
-0.33	1.76\\
-0.33	1.76\\
-0.33	1.76\\
-0.33	1.76\\
-0.33	1.76\\
-0.33	1.76\\
-0.33	1.76\\
-0.33	1.76\\
-0.33	1.76\\
-0.33	1.76\\
-0.33	1.76\\
-0.33	1.76\\
-0.33	1.76\\
-0.33	1.76\\
-0.33	1.76\\
-0.33	1.76\\
-0.33	1.76\\
-0.33	1.76\\
-0.33	1.76\\
-0.33	1.76\\
-0.33	1.76\\
-0.33	1.76\\
-0.33	1.76\\
-0.33	1.76\\
-0.33	1.76\\
-0.33	1.76\\
-0.33	1.76\\
-0.33	1.76\\
-0.33	1.76\\
-0.33	1.76\\
-0.33	1.76\\
-0.33	1.76\\
-0.33	1.76\\
-0.33	1.76\\
-0.33	1.76\\
-0.33	1.76\\
-0.33	1.76\\
-0.33	1.76\\
-0.33	1.76\\
-0.33	1.76\\
-0.33	1.76\\
-0.33	1.76\\
-0.33	1.76\\
-0.33	1.76\\
-0.33	1.76\\
-0.33	1.76\\
-0.33	1.76\\
-0.33	1.76\\
-0.33	1.76\\
-0.33	1.76\\
-0.33	1.76\\
-0.33	1.76\\
-0.33	1.76\\
-0.33	1.76\\
-0.33	1.76\\
-0.33	1.76\\
-0.33	1.76\\
-0.33	1.76\\
-0.33	1.76\\
-0.33	1.76\\
-0.33	1.76\\
-0.33	1.76\\
-0.33	1.76\\
-0.33	1.76\\
-0.33	1.76\\
-0.33	1.76\\
-0.33	1.76\\
-0.33	1.76\\
-0.33	1.76\\
-0.33	1.76\\
-0.33	1.76\\
-0.33	1.76\\
-0.33	1.76\\
-0.33	1.76\\
-0.33	1.76\\
-0.33	1.76\\
-0.33	1.76\\
-0.33	1.76\\
-0.33	1.76\\
-0.33	1.76\\
-0.33	1.76\\
-0.33	1.76\\
-0.33	1.76\\
-0.33	1.76\\
-0.33	1.76\\
-0.33	1.76\\
-0.33	1.76\\
-0.33	1.76\\
-0.33	1.76\\
-0.33	1.76\\
-0.33	1.76\\
-0.33	1.76\\
-0.33	1.76\\
-0.33	1.76\\
-0.33	1.76\\
-0.33	1.76\\
-0.33	1.76\\
-0.33	1.76\\
-0.33	1.76\\
-0.33	1.76\\
-0.33	1.76\\
-0.33	1.76\\
-0.33	1.76\\
-0.33	1.76\\
-0.33	1.76\\
-0.33	1.76\\
-0.33	1.76\\
-0.33	1.76\\
-0.33	1.76\\
-0.33	1.76\\
-0.33	1.76\\
-0.33	1.76\\
-0.33	1.76\\
-0.33	1.76\\
-0.33	1.76\\
-0.33	1.76\\
-0.33	1.76\\
-0.33	1.76\\
-0.33	1.76\\
-0.33	1.76\\
-0.33	1.76\\
-0.33	1.76\\
-0.33	1.76\\
-0.33	1.76\\
-0.33	1.76\\
-0.33	1.76\\
-0.33	1.76\\
-0.33	1.76\\
-0.33	1.76\\
-0.33	1.76\\
-0.33	1.76\\
-0.33	1.76\\
-0.33	1.76\\
-0.33	1.76\\
-0.33	1.76\\
-0.33	1.76\\
-0.33	1.76\\
-0.33	1.76\\
-0.33	1.76\\
-0.33	1.76\\
-0.33	1.76\\
-0.33	1.76\\
-0.33	1.76\\
-0.33	1.76\\
-0.33	1.76\\
-0.33	1.76\\
-0.33	1.76\\
-0.33	1.76\\
-0.33	1.76\\
-0.33	1.76\\
-0.33	1.76\\
-0.33	1.76\\
-0.33	1.76\\
-0.33	1.76\\
-0.33	1.76\\
-0.33	1.76\\
-0.33	1.76\\
-0.33	1.76\\
-0.33	1.76\\
-0.33	1.76\\
-0.33	1.76\\
-0.33	1.76\\
-0.33	1.76\\
-0.33	1.76\\
-0.33	1.76\\
-0.33	1.76\\
-0.33	1.76\\
-0.33	1.76\\
-0.33	1.76\\
-0.33	1.76\\
-0.33	1.76\\
-0.33	1.76\\
-0.33	1.76\\
-0.33	1.76\\
-0.33	1.76\\
-0.33	1.76\\
-0.33	1.76\\
-0.33	1.76\\
-0.33	1.76\\
-0.33	1.76\\
-0.33	1.76\\
-0.33	1.76\\
-0.33	1.76\\
-0.33	1.76\\
-0.33	1.76\\
-0.33	1.76\\
-0.33	1.76\\
-0.33	1.76\\
-0.33	1.76\\
-0.33	1.76\\
-0.33	1.76\\
-0.33	1.76\\
-0.33	1.76\\
-0.33	1.76\\
-0.33	1.76\\
-0.33	1.76\\
-0.33	1.76\\
-0.33	1.76\\
-0.33	1.76\\
-0.33	1.76\\
-0.33	1.76\\
-0.33	1.76\\
-0.33	1.76\\
-0.33	1.76\\
-0.33	1.76\\
-0.33	1.76\\
-0.33	1.76\\
-0.33	1.76\\
-0.33	1.76\\
-0.33	1.76\\
-0.33	1.76\\
-0.33	1.76\\
-0.33	1.76\\
-0.33	1.76\\
-0.33	1.76\\
-0.33	1.76\\
-0.33	1.76\\
-0.33	1.76\\
-0.33	1.76\\
-0.33	1.76\\
-0.33	1.76\\
-0.33	1.76\\
-0.33	1.76\\
-0.33	1.76\\
-0.33	1.76\\
-0.33	1.76\\
-0.33	1.76\\
-0.33	1.76\\
-0.33	1.76\\
-0.33	1.76\\
-0.33	1.76\\
-0.33	1.76\\
-0.33	1.76\\
-0.33	1.76\\
-0.33	1.76\\
-0.33	1.76\\
-0.33	1.76\\
-0.33	1.76\\
-0.33	1.76\\
-0.33	1.76\\
-0.33	1.76\\
-0.33	1.76\\
-0.33	1.76\\
-0.33	1.76\\
-0.33	1.76\\
-0.33	1.76\\
};
\addplot [color=mycolor1,solid,forget plot]
  table[row sep=crcr]{%
-0.33	1.76\\
-0.33	1.76\\
-0.33	1.76\\
-0.33	1.76\\
-0.33	1.76\\
-0.33	1.76\\
-0.33	1.76\\
-0.33	1.76\\
-0.33	1.76\\
-0.33	1.76\\
-0.33	1.76\\
-0.33	1.76\\
-0.33	1.76\\
-0.33	1.76\\
-0.33	1.76\\
-0.33	1.76\\
-0.33	1.76\\
-0.33	1.76\\
-0.33	1.76\\
-0.33	1.76\\
-0.33	1.76\\
-0.33	1.76\\
-0.33	1.76\\
-0.33	1.76\\
-0.33	1.76\\
-0.33	1.76\\
-0.33	1.76\\
-0.33	1.76\\
-0.33	1.76\\
-0.33	1.76\\
-0.33	1.76\\
-0.33	1.76\\
-0.33	1.76\\
-0.33	1.76\\
-0.33	1.76\\
-0.33	1.76\\
-0.33	1.76\\
-0.33	1.76\\
-0.33	1.76\\
-0.33	1.76\\
-0.33	1.76\\
-0.33	1.76\\
-0.33	1.76\\
-0.33	1.76\\
-0.33	1.76\\
-0.33	1.76\\
-0.33	1.76\\
-0.33	1.76\\
-0.33	1.76\\
-0.33	1.76\\
-0.33	1.76\\
-0.33	1.76\\
-0.33	1.76\\
-0.33	1.76\\
-0.33	1.76\\
-0.33	1.76\\
-0.33	1.76\\
-0.33	1.76\\
-0.33	1.76\\
-0.33	1.76\\
-0.33	1.76\\
-0.33	1.76\\
-0.33	1.76\\
-0.33	1.76\\
-0.33	1.76\\
-0.33	1.76\\
-0.33	1.76\\
-0.33	1.76\\
-0.33	1.76\\
-0.33	1.76\\
-0.33	1.76\\
-0.33	1.76\\
-0.33	1.76\\
-0.33	1.76\\
-0.33	1.76\\
-0.33	1.76\\
-0.33	1.76\\
-0.33	1.76\\
-0.33	1.76\\
-0.33	1.76\\
-0.33	1.76\\
-0.33	1.76\\
-0.33	1.76\\
-0.33	1.76\\
-0.33	1.76\\
-0.33	1.76\\
-0.33	1.76\\
-0.33	1.76\\
-0.33	1.76\\
-0.33	1.76\\
-0.33	1.76\\
-0.33	1.76\\
-0.33	1.76\\
-0.33	1.76\\
-0.33	1.76\\
-0.33	1.76\\
-0.33	1.76\\
-0.33	1.76\\
-0.33	1.76\\
-0.33	1.76\\
-0.33	1.76\\
-0.33	1.76\\
-0.33	1.76\\
-0.33	1.76\\
-0.33	1.76\\
-0.33	1.76\\
-0.33	1.76\\
-0.33	1.76\\
-0.33	1.76\\
-0.33	1.76\\
-0.33	1.76\\
-0.33	1.76\\
-0.33	1.76\\
-0.33	1.76\\
-0.33	1.76\\
-0.33	1.76\\
-0.33	1.76\\
-0.33	1.76\\
-0.33	1.76\\
-0.33	1.76\\
-0.33	1.76\\
-0.33	1.76\\
-0.33	1.76\\
-0.33	1.76\\
-0.33	1.76\\
-0.33	1.76\\
-0.33	1.76\\
-0.33	1.76\\
-0.33	1.76\\
-0.33	1.76\\
-0.33	1.76\\
-0.33	1.76\\
-0.33	1.76\\
-0.33	1.76\\
-0.33	1.76\\
-0.33	1.76\\
-0.33	1.76\\
-0.33	1.76\\
-0.33	1.76\\
-0.33	1.76\\
-0.33	1.76\\
-0.33	1.76\\
-0.33	1.76\\
-0.33	1.76\\
-0.33	1.76\\
-0.33	1.76\\
-0.33	1.76\\
-0.33	1.76\\
-0.33	1.76\\
-0.33	1.76\\
-0.33	1.76\\
-0.33	1.76\\
-0.33	1.76\\
-0.33	1.76\\
-0.33	1.76\\
-0.33	1.76\\
-0.33	1.76\\
-0.33	1.76\\
-0.33	1.76\\
-0.33	1.76\\
-0.33	1.76\\
-0.33	1.76\\
-0.33	1.76\\
-0.33	1.76\\
-0.33	1.76\\
-0.33	1.76\\
-0.33	1.76\\
-0.33	1.76\\
-0.33	1.76\\
-0.33	1.76\\
-0.33	1.76\\
-0.33	1.76\\
-0.33	1.76\\
-0.33	1.76\\
-0.33	1.76\\
-0.33	1.76\\
-0.33	1.76\\
-0.33	1.76\\
-0.33	1.76\\
-0.33	1.76\\
-0.33	1.76\\
-0.33	1.76\\
-0.33	1.76\\
-0.33	1.76\\
-0.33	1.76\\
-0.33	1.76\\
-0.33	1.76\\
-0.33	1.76\\
-0.33	1.76\\
-0.33	1.76\\
-0.33	1.76\\
-0.33	1.76\\
-0.33	1.76\\
-0.33	1.76\\
-0.33	1.76\\
-0.33	1.76\\
-0.33	1.76\\
-0.33	1.76\\
-0.33	1.76\\
-0.33	1.76\\
-0.33	1.76\\
-0.33	1.76\\
-0.33	1.76\\
-0.33	1.76\\
-0.33	1.76\\
-0.33	1.76\\
-0.33	1.76\\
-0.33	1.76\\
-0.33	1.76\\
-0.33	1.76\\
-0.33	1.76\\
-0.33	1.76\\
-0.33	1.76\\
-0.33	1.76\\
-0.33	1.76\\
-0.33	1.76\\
-0.33	1.76\\
-0.33	1.76\\
-0.33	1.76\\
-0.33	1.76\\
-0.33	1.76\\
-0.33	1.76\\
-0.33	1.76\\
-0.33	1.76\\
-0.33	1.76\\
-0.33	1.76\\
-0.33	1.76\\
-0.33	1.76\\
-0.33	1.76\\
-0.33	1.76\\
-0.33	1.76\\
-0.33	1.76\\
-0.33	1.76\\
-0.33	1.76\\
-0.33	1.76\\
-0.33	1.76\\
-0.33	1.76\\
-0.33	1.76\\
-0.33	1.76\\
-0.33	1.76\\
-0.33	1.76\\
-0.33	1.76\\
-0.33	1.76\\
-0.33	1.76\\
-0.33	1.76\\
-0.33	1.76\\
-0.33	1.76\\
-0.33	1.76\\
-0.33	1.76\\
-0.33	1.76\\
-0.33	1.76\\
-0.33	1.76\\
-0.33	1.76\\
-0.33	1.76\\
-0.33	1.76\\
-0.33	1.76\\
-0.33	1.76\\
-0.33	1.76\\
-0.33	1.76\\
-0.33	1.76\\
-0.33	1.76\\
-0.33	1.76\\
-0.33	1.76\\
-0.33	1.76\\
-0.33	1.76\\
-0.33	1.76\\
-0.33	1.76\\
-0.33	1.76\\
-0.33	1.76\\
-0.33	1.76\\
-0.33	1.76\\
-0.33	1.76\\
-0.33	1.76\\
-0.33	1.76\\
-0.33	1.76\\
-0.33	1.76\\
-0.33	1.76\\
-0.33	1.76\\
-0.33	1.76\\
-0.33	1.76\\
-0.33	1.76\\
-0.33	1.76\\
-0.33	1.76\\
-0.33	1.76\\
-0.33	1.76\\
-0.33	1.76\\
-0.33	1.76\\
-0.33	1.76\\
-0.33	1.76\\
-0.33	1.76\\
-0.33	1.76\\
-0.33	1.76\\
-0.33	1.76\\
-0.33	1.76\\
-0.33	1.76\\
-0.33	1.76\\
-0.33	1.76\\
-0.33	1.76\\
-0.33	1.76\\
-0.33	1.76\\
-0.33	1.76\\
-0.33	1.76\\
-0.33	1.76\\
-0.33	1.76\\
-0.33	1.76\\
-0.33	1.76\\
-0.33	1.76\\
-0.33	1.76\\
-0.33	1.76\\
-0.33	1.76\\
-0.33	1.76\\
-0.33	1.76\\
-0.33	1.76\\
-0.33	1.76\\
-0.33	1.76\\
-0.33	1.76\\
-0.33	1.76\\
-0.33	1.76\\
-0.33	1.76\\
-0.33	1.76\\
-0.33	1.76\\
-0.33	1.76\\
-0.33	1.76\\
-0.33	1.76\\
-0.33	1.76\\
-0.33	1.76\\
-0.33	1.76\\
-0.33	1.76\\
-0.33	1.76\\
-0.33	1.76\\
-0.33	1.76\\
-0.33	1.76\\
-0.33	1.76\\
-0.33	1.76\\
-0.33	1.76\\
-0.33	1.76\\
-0.33	1.76\\
-0.33	1.76\\
-0.33	1.76\\
-0.33	1.76\\
-0.33	1.76\\
-0.33	1.76\\
-0.33	1.76\\
-0.33	1.76\\
-0.33	1.76\\
-0.33	1.76\\
-0.33	1.76\\
-0.33	1.76\\
-0.33	1.76\\
-0.33	1.76\\
-0.33	1.76\\
-0.33	1.76\\
-0.33	1.76\\
-0.33	1.76\\
-0.33	1.76\\
-0.33	1.76\\
-0.33	1.76\\
-0.33	1.76\\
-0.33	1.76\\
-0.33	1.76\\
-0.33	1.76\\
-0.33	1.76\\
-0.33	1.76\\
-0.33	1.76\\
-0.33	1.76\\
-0.33	1.76\\
-0.33	1.76\\
-0.33	1.76\\
-0.33	1.76\\
-0.33	1.76\\
-0.33	1.76\\
-0.33	1.76\\
-0.33	1.76\\
-0.33	1.76\\
-0.33	1.76\\
-0.33	1.76\\
-0.33	1.76\\
-0.33	1.76\\
-0.33	1.76\\
-0.33	1.76\\
-0.33	1.76\\
-0.33	1.76\\
-0.33	1.76\\
-0.33	1.76\\
-0.33	1.76\\
-0.33	1.76\\
-0.33	1.76\\
-0.33	1.76\\
-0.33	1.76\\
-0.33	1.76\\
-0.33	1.76\\
-0.33	1.76\\
-0.33	1.76\\
-0.33	1.76\\
-0.33	1.76\\
-0.33	1.76\\
-0.33	1.76\\
-0.33	1.76\\
-0.33	1.76\\
-0.33	1.76\\
-0.33	1.76\\
-0.33	1.76\\
-0.33	1.76\\
-0.33	1.76\\
-0.33	1.76\\
-0.33	1.76\\
-0.33	1.76\\
-0.33	1.76\\
-0.33	1.76\\
-0.33	1.76\\
-0.33	1.76\\
-0.33	1.76\\
-0.33	1.76\\
-0.33	1.76\\
-0.33	1.76\\
-0.33	1.76\\
-0.33	1.76\\
-0.33	1.76\\
-0.33	1.76\\
-0.33	1.76\\
-0.33	1.76\\
-0.33	1.76\\
-0.33	1.76\\
-0.33	1.76\\
-0.33	1.76\\
-0.32	1.76\\
-0.32	1.76\\
-0.33	1.76\\
-0.32	1.76\\
-0.32	1.76\\
-0.33	1.76\\
-0.32	1.76\\
-0.33	1.76\\
-0.33	1.76\\
-0.33	1.76\\
-0.33	1.76\\
-0.33	1.76\\
-0.33	1.76\\
-0.33	1.76\\
-0.33	1.76\\
-0.33	1.76\\
-0.33	1.76\\
-0.33	1.76\\
-0.33	1.76\\
-0.33	1.77\\
-0.33	1.77\\
-0.33	1.77\\
-0.33	1.77\\
-0.33	1.77\\
-0.32	1.77\\
-0.33	1.77\\
-0.33	1.77\\
-0.33	1.77\\
-0.33	1.77\\
-0.33	1.77\\
-0.33	1.77\\
-0.33	1.77\\
-0.33	1.77\\
-0.33	1.77\\
-0.33	1.77\\
-0.33	1.77\\
-0.33	1.76\\
-0.33	1.76\\
-0.33	1.77\\
-0.33	1.76\\
-0.33	1.76\\
-0.33	1.76\\
-0.33	1.76\\
-0.33	1.76\\
-0.33	1.76\\
-0.33	1.76\\
-0.33	1.76\\
-0.33	1.76\\
-0.33	1.76\\
-0.33	1.76\\
-0.33	1.76\\
-0.33	1.76\\
-0.33	1.76\\
-0.33	1.76\\
-0.33	1.76\\
-0.33	1.76\\
-0.33	1.76\\
-0.33	1.76\\
-0.33	1.76\\
-0.33	1.76\\
-0.33	1.76\\
-0.33	1.76\\
-0.33	1.76\\
-0.33	1.76\\
-0.33	1.76\\
-0.33	1.76\\
-0.33	1.76\\
-0.33	1.76\\
-0.33	1.76\\
-0.33	1.76\\
-0.33	1.76\\
-0.33	1.76\\
-0.33	1.76\\
-0.33	1.76\\
-0.33	1.76\\
-0.33	1.76\\
-0.33	1.76\\
-0.33	1.76\\
-0.33	1.76\\
-0.33	1.76\\
-0.33	1.76\\
-0.33	1.76\\
-0.33	1.76\\
-0.33	1.76\\
-0.33	1.76\\
-0.33	1.76\\
-0.33	1.76\\
-0.33	1.76\\
-0.33	1.76\\
-0.33	1.76\\
-0.33	1.76\\
-0.33	1.76\\
-0.33	1.76\\
-0.33	1.76\\
-0.33	1.76\\
-0.33	1.76\\
-0.33	1.76\\
-0.33	1.76\\
-0.33	1.76\\
-0.33	1.76\\
-0.33	1.76\\
-0.33	1.76\\
-0.33	1.76\\
-0.33	1.76\\
-0.33	1.76\\
-0.33	1.76\\
-0.33	1.76\\
-0.33	1.76\\
-0.33	1.76\\
-0.33	1.76\\
-0.33	1.76\\
-0.33	1.76\\
-0.33	1.76\\
-0.33	1.76\\
-0.33	1.76\\
-0.33	1.76\\
-0.33	1.76\\
-0.33	1.76\\
-0.33	1.76\\
-0.33	1.76\\
-0.33	1.76\\
-0.33	1.76\\
-0.33	1.76\\
-0.33	1.76\\
-0.33	1.76\\
-0.33	1.76\\
-0.33	1.76\\
-0.33	1.76\\
-0.33	1.76\\
-0.33	1.76\\
-0.33	1.76\\
-0.33	1.76\\
-0.33	1.76\\
-0.33	1.76\\
-0.33	1.76\\
-0.33	1.76\\
-0.34	1.76\\
-0.34	1.76\\
-0.34	1.76\\
-0.34	1.76\\
-0.34	1.76\\
-0.34	1.76\\
-0.34	1.76\\
-0.34	1.76\\
-0.34	1.76\\
-0.34	1.76\\
-0.34	1.76\\
-0.34	1.76\\
-0.34	1.76\\
-0.34	1.76\\
-0.34	1.76\\
-0.34	1.76\\
-0.34	1.76\\
-0.34	1.76\\
-0.34	1.76\\
-0.34	1.76\\
-0.34	1.76\\
-0.34	1.76\\
-0.34	1.76\\
-0.34	1.76\\
-0.34	1.76\\
-0.34	1.76\\
-0.34	1.76\\
-0.34	1.76\\
-0.34	1.76\\
-0.34	1.76\\
-0.34	1.76\\
-0.34	1.76\\
-0.34	1.76\\
-0.34	1.76\\
-0.34	1.76\\
-0.34	1.76\\
-0.34	1.76\\
-0.34	1.76\\
-0.34	1.76\\
-0.34	1.76\\
-0.34	1.76\\
-0.34	1.76\\
-0.34	1.76\\
-0.34	1.76\\
-0.34	1.76\\
-0.34	1.76\\
-0.34	1.76\\
-0.34	1.76\\
-0.34	1.76\\
-0.34	1.76\\
-0.34	1.76\\
-0.34	1.76\\
-0.34	1.76\\
-0.34	1.76\\
-0.34	1.76\\
-0.34	1.76\\
-0.34	1.76\\
-0.34	1.76\\
-0.34	1.76\\
-0.34	1.76\\
-0.34	1.76\\
-0.34	1.76\\
-0.34	1.76\\
-0.34	1.76\\
-0.34	1.76\\
-0.34	1.76\\
-0.34	1.76\\
-0.34	1.76\\
-0.34	1.76\\
-0.34	1.76\\
-0.34	1.76\\
-0.34	1.76\\
-0.34	1.76\\
-0.34	1.76\\
-0.34	1.76\\
-0.34	1.76\\
-0.34	1.76\\
-0.34	1.76\\
-0.34	1.76\\
-0.34	1.76\\
-0.34	1.76\\
-0.34	1.76\\
-0.34	1.76\\
-0.34	1.76\\
-0.34	1.76\\
-0.34	1.76\\
-0.34	1.76\\
-0.34	1.76\\
-0.34	1.76\\
-0.34	1.76\\
-0.34	1.76\\
-0.34	1.76\\
-0.34	1.76\\
-0.34	1.76\\
-0.34	1.76\\
-0.34	1.76\\
-0.34	1.76\\
-0.34	1.76\\
-0.34	1.76\\
-0.34	1.76\\
-0.34	1.76\\
-0.34	1.76\\
-0.34	1.76\\
-0.34	1.76\\
-0.34	1.76\\
-0.34	1.76\\
-0.34	1.76\\
-0.34	1.76\\
-0.34	1.76\\
-0.34	1.76\\
-0.34	1.76\\
-0.34	1.76\\
-0.34	1.76\\
-0.34	1.76\\
-0.34	1.76\\
-0.34	1.76\\
-0.34	1.76\\
-0.34	1.76\\
-0.34	1.76\\
-0.34	1.76\\
-0.34	1.76\\
-0.34	1.76\\
-0.34	1.76\\
-0.34	1.76\\
-0.34	1.76\\
-0.34	1.76\\
-0.34	1.76\\
-0.34	1.76\\
-0.34	1.76\\
-0.34	1.76\\
-0.34	1.76\\
-0.34	1.76\\
-0.34	1.76\\
-0.34	1.76\\
-0.34	1.76\\
-0.34	1.76\\
-0.34	1.76\\
-0.34	1.76\\
-0.34	1.76\\
-0.34	1.76\\
-0.34	1.76\\
-0.34	1.76\\
-0.34	1.76\\
-0.34	1.76\\
-0.33	1.76\\
-0.33	1.76\\
-0.33	1.76\\
-0.33	1.76\\
-0.33	1.76\\
-0.33	1.76\\
-0.33	1.76\\
-0.33	1.76\\
-0.33	1.76\\
-0.33	1.76\\
-0.33	1.76\\
-0.33	1.75\\
-0.33	1.75\\
-0.33	1.75\\
-0.33	1.75\\
-0.33	1.75\\
-0.33	1.75\\
-0.33	1.75\\
-0.33	1.75\\
-0.33	1.75\\
-0.33	1.75\\
-0.33	1.75\\
-0.33	1.75\\
-0.33	1.75\\
-0.33	1.75\\
-0.33	1.75\\
-0.33	1.75\\
-0.33	1.75\\
-0.33	1.75\\
-0.33	1.75\\
-0.33	1.75\\
-0.33	1.75\\
-0.33	1.75\\
-0.33	1.75\\
-0.33	1.75\\
-0.33	1.75\\
-0.33	1.75\\
-0.33	1.75\\
-0.33	1.75\\
-0.33	1.75\\
-0.33	1.75\\
-0.33	1.75\\
-0.33	1.75\\
-0.33	1.75\\
-0.33	1.75\\
-0.33	1.75\\
-0.33	1.75\\
-0.33	1.75\\
-0.33	1.75\\
-0.33	1.75\\
-0.33	1.75\\
-0.33	1.75\\
-0.33	1.75\\
-0.33	1.75\\
-0.33	1.75\\
-0.33	1.75\\
-0.33	1.75\\
-0.33	1.75\\
-0.33	1.75\\
-0.33	1.75\\
-0.33	1.75\\
-0.33	1.75\\
-0.33	1.75\\
-0.33	1.75\\
-0.33	1.75\\
-0.33	1.75\\
-0.33	1.75\\
-0.33	1.75\\
-0.33	1.75\\
-0.33	1.75\\
-0.33	1.75\\
-0.33	1.75\\
-0.33	1.75\\
-0.33	1.75\\
-0.33	1.75\\
-0.33	1.75\\
-0.33	1.75\\
-0.33	1.75\\
-0.33	1.75\\
-0.33	1.75\\
-0.33	1.75\\
-0.33	1.75\\
-0.33	1.75\\
-0.33	1.75\\
-0.33	1.75\\
-0.33	1.75\\
-0.33	1.75\\
-0.33	1.75\\
-0.33	1.75\\
-0.33	1.75\\
-0.33	1.75\\
-0.33	1.75\\
-0.33	1.75\\
-0.33	1.75\\
-0.33	1.75\\
-0.33	1.76\\
-0.33	1.76\\
-0.33	1.76\\
-0.33	1.76\\
-0.33	1.76\\
-0.33	1.76\\
-0.33	1.76\\
-0.33	1.76\\
-0.33	1.76\\
-0.33	1.76\\
-0.33	1.76\\
-0.33	1.76\\
-0.33	1.76\\
-0.34	1.76\\
-0.34	1.76\\
-0.34	1.76\\
-0.34	1.76\\
-0.34	1.76\\
-0.34	1.76\\
-0.34	1.76\\
-0.34	1.76\\
-0.34	1.76\\
-0.34	1.76\\
-0.34	1.76\\
-0.34	1.76\\
-0.34	1.76\\
-0.34	1.76\\
-0.34	1.76\\
-0.34	1.76\\
-0.34	1.76\\
-0.34	1.76\\
-0.34	1.76\\
-0.34	1.76\\
-0.34	1.76\\
-0.34	1.76\\
-0.34	1.76\\
-0.34	1.76\\
-0.34	1.76\\
-0.34	1.76\\
-0.34	1.76\\
-0.34	1.76\\
-0.34	1.76\\
-0.34	1.76\\
-0.34	1.76\\
-0.34	1.76\\
-0.34	1.76\\
-0.34	1.76\\
-0.34	1.76\\
-0.34	1.76\\
-0.34	1.76\\
-0.34	1.76\\
-0.34	1.76\\
-0.34	1.76\\
-0.34	1.76\\
-0.34	1.76\\
-0.34	1.76\\
-0.34	1.76\\
-0.34	1.76\\
-0.34	1.76\\
-0.34	1.76\\
-0.34	1.76\\
-0.34	1.76\\
-0.34	1.76\\
-0.34	1.76\\
-0.34	1.76\\
-0.34	1.76\\
-0.34	1.76\\
-0.34	1.76\\
-0.34	1.76\\
-0.34	1.76\\
-0.34	1.76\\
-0.34	1.76\\
-0.34	1.76\\
-0.34	1.76\\
-0.34	1.76\\
-0.34	1.76\\
-0.34	1.76\\
-0.34	1.76\\
-0.34	1.76\\
-0.34	1.76\\
-0.34	1.76\\
-0.34	1.76\\
-0.34	1.76\\
-0.34	1.76\\
-0.34	1.76\\
-0.34	1.76\\
-0.34	1.76\\
-0.34	1.76\\
-0.34	1.76\\
-0.34	1.76\\
-0.34	1.76\\
-0.34	1.76\\
-0.34	1.76\\
-0.34	1.76\\
-0.34	1.76\\
-0.34	1.76\\
-0.34	1.76\\
-0.34	1.76\\
-0.34	1.76\\
-0.34	1.76\\
-0.34	1.76\\
-0.34	1.76\\
-0.34	1.76\\
-0.34	1.76\\
-0.34	1.76\\
-0.34	1.76\\
-0.34	1.76\\
-0.34	1.76\\
-0.34	1.76\\
-0.34	1.76\\
-0.34	1.76\\
-0.34	1.76\\
-0.34	1.76\\
-0.34	1.76\\
-0.34	1.76\\
-0.34	1.76\\
-0.34	1.76\\
-0.34	1.76\\
-0.34	1.76\\
-0.34	1.76\\
-0.34	1.76\\
-0.34	1.76\\
-0.34	1.76\\
-0.34	1.76\\
-0.34	1.76\\
-0.34	1.76\\
-0.34	1.76\\
-0.34	1.76\\
-0.34	1.76\\
-0.34	1.76\\
-0.34	1.76\\
-0.34	1.76\\
-0.34	1.76\\
-0.34	1.76\\
-0.34	1.76\\
-0.34	1.76\\
-0.34	1.76\\
-0.34	1.76\\
-0.34	1.76\\
-0.34	1.76\\
-0.34	1.76\\
-0.34	1.76\\
-0.34	1.76\\
-0.34	1.76\\
-0.34	1.76\\
-0.34	1.76\\
-0.34	1.76\\
-0.34	1.76\\
-0.34	1.76\\
-0.34	1.76\\
-0.34	1.76\\
-0.34	1.76\\
-0.34	1.76\\
-0.34	1.76\\
-0.34	1.76\\
-0.34	1.76\\
-0.34	1.76\\
-0.34	1.76\\
-0.34	1.76\\
-0.34	1.76\\
-0.34	1.76\\
-0.34	1.76\\
-0.34	1.76\\
-0.34	1.76\\
-0.34	1.76\\
-0.34	1.76\\
-0.34	1.76\\
-0.34	1.76\\
-0.34	1.76\\
-0.34	1.76\\
-0.34	1.76\\
-0.34	1.76\\
-0.34	1.76\\
-0.34	1.76\\
-0.34	1.76\\
-0.34	1.76\\
-0.34	1.76\\
-0.34	1.76\\
-0.34	1.76\\
-0.34	1.76\\
-0.34	1.76\\
-0.34	1.76\\
-0.34	1.76\\
-0.34	1.76\\
-0.34	1.76\\
-0.34	1.76\\
-0.34	1.76\\
-0.34	1.76\\
-0.34	1.76\\
-0.34	1.76\\
-0.34	1.76\\
-0.34	1.76\\
-0.34	1.76\\
-0.34	1.76\\
-0.34	1.76\\
-0.34	1.76\\
-0.34	1.76\\
-0.34	1.76\\
-0.34	1.76\\
-0.34	1.76\\
-0.34	1.76\\
-0.34	1.76\\
-0.34	1.76\\
-0.34	1.76\\
-0.34	1.76\\
-0.34	1.76\\
-0.34	1.76\\
-0.34	1.76\\
-0.34	1.76\\
-0.34	1.76\\
-0.34	1.76\\
-0.34	1.76\\
-0.34	1.76\\
-0.34	1.76\\
-0.34	1.76\\
-0.34	1.76\\
-0.34	1.76\\
-0.34	1.76\\
-0.34	1.76\\
-0.34	1.76\\
-0.34	1.76\\
-0.34	1.76\\
-0.34	1.76\\
-0.34	1.76\\
-0.34	1.76\\
-0.34	1.76\\
-0.34	1.76\\
-0.34	1.76\\
-0.34	1.76\\
-0.34	1.76\\
-0.34	1.76\\
-0.34	1.76\\
-0.34	1.76\\
-0.34	1.76\\
-0.34	1.76\\
-0.34	1.76\\
-0.34	1.76\\
-0.34	1.76\\
-0.34	1.76\\
-0.34	1.76\\
-0.34	1.76\\
-0.34	1.76\\
-0.34	1.76\\
-0.34	1.76\\
-0.34	1.76\\
-0.34	1.76\\
-0.34	1.76\\
-0.34	1.76\\
-0.34	1.76\\
-0.34	1.76\\
-0.34	1.76\\
-0.34	1.76\\
-0.34	1.76\\
-0.34	1.76\\
-0.34	1.76\\
-0.34	1.76\\
-0.34	1.76\\
-0.34	1.76\\
-0.34	1.76\\
-0.34	1.76\\
-0.34	1.76\\
-0.34	1.76\\
-0.34	1.76\\
-0.34	1.76\\
-0.34	1.76\\
-0.34	1.76\\
-0.34	1.76\\
-0.34	1.76\\
-0.34	1.76\\
-0.34	1.76\\
-0.34	1.76\\
-0.34	1.76\\
-0.34	1.76\\
-0.34	1.76\\
-0.34	1.76\\
-0.34	1.76\\
-0.34	1.76\\
-0.34	1.76\\
-0.34	1.76\\
-0.34	1.76\\
-0.34	1.76\\
-0.34	1.76\\
-0.34	1.76\\
-0.34	1.76\\
-0.34	1.76\\
-0.34	1.76\\
-0.34	1.76\\
-0.34	1.76\\
-0.34	1.76\\
-0.34	1.76\\
-0.34	1.76\\
-0.34	1.76\\
-0.34	1.76\\
-0.34	1.76\\
-0.34	1.76\\
-0.34	1.76\\
-0.34	1.76\\
-0.34	1.76\\
-0.34	1.76\\
-0.34	1.76\\
-0.34	1.76\\
-0.34	1.76\\
-0.34	1.76\\
-0.34	1.76\\
-0.34	1.76\\
-0.34	1.76\\
-0.34	1.76\\
-0.34	1.76\\
-0.34	1.76\\
-0.34	1.76\\
-0.34	1.76\\
-0.34	1.76\\
-0.34	1.76\\
-0.34	1.76\\
-0.34	1.76\\
-0.34	1.76\\
-0.34	1.76\\
-0.34	1.76\\
-0.34	1.76\\
-0.34	1.76\\
-0.34	1.76\\
-0.34	1.76\\
-0.34	1.76\\
-0.34	1.76\\
-0.34	1.76\\
-0.34	1.76\\
-0.34	1.76\\
-0.34	1.76\\
-0.34	1.76\\
-0.34	1.76\\
-0.34	1.76\\
-0.34	1.76\\
-0.34	1.76\\
-0.34	1.76\\
-0.34	1.76\\
-0.34	1.76\\
-0.34	1.76\\
-0.34	1.76\\
-0.34	1.76\\
-0.34	1.76\\
-0.34	1.76\\
-0.34	1.76\\
-0.34	1.76\\
-0.34	1.76\\
-0.34	1.76\\
-0.34	1.76\\
-0.34	1.76\\
-0.34	1.76\\
-0.34	1.76\\
-0.34	1.76\\
-0.34	1.76\\
-0.34	1.76\\
-0.33	1.75\\
-0.33	1.75\\
-0.33	1.75\\
-0.32	1.74\\
-0.32	1.74\\
-0.31	1.73\\
-0.31	1.73\\
-0.3	1.72\\
-0.3	1.72\\
-0.3	1.71\\
-0.29	1.71\\
-0.28	1.7\\
-0.28	1.7\\
-0.27	1.69\\
-0.26	1.68\\
-0.26	1.68\\
-0.25	1.67\\
-0.25	1.66\\
-0.24	1.65\\
-0.23	1.65\\
-0.23	1.64\\
-0.22	1.64\\
-0.22	1.63\\
-0.21	1.62\\
-0.2	1.62\\
-0.2	1.61\\
-0.19	1.6\\
-0.18	1.6\\
-0.18	1.59\\
-0.17	1.58\\
-0.17	1.58\\
-0.16	1.57\\
-0.16	1.57\\
-0.16	1.57\\
-0.15	1.56\\
-0.15	1.56\\
-0.14	1.55\\
-0.14	1.55\\
-0.14	1.54\\
-0.13	1.54\\
-0.13	1.54\\
-0.13	1.53\\
-0.12	1.53\\
-0.12	1.53\\
-0.12	1.52\\
-0.11	1.52\\
-0.11	1.52\\
-0.1	1.51\\
-0.1	1.51\\
-0.1	1.51\\
-0.1	1.5\\
-0.09	1.5\\
-0.09	1.5\\
-0.09	1.49\\
-0.08	1.49\\
-0.08	1.48\\
-0.07	1.48\\
-0.07	1.48\\
-0.07	1.47\\
-0.06	1.47\\
-0.06	1.47\\
-0.06	1.46\\
-0.05	1.46\\
-0.05	1.45\\
-0.04	1.45\\
-0.04	1.44\\
-0.04	1.44\\
-0.03	1.43\\
-0.03	1.43\\
-0.03	1.43\\
-0.02	1.42\\
-0.02	1.42\\
-0.01	1.41\\
-0.01	1.41\\
-0	1.4\\
-0	1.4\\
0	1.39\\
0.01	1.39\\
0.01	1.38\\
0.02	1.38\\
0.02	1.37\\
0.02	1.37\\
0.03	1.37\\
0.03	1.36\\
0.03	1.36\\
0.04	1.35\\
0.04	1.35\\
0.05	1.34\\
0.05	1.34\\
0.06	1.34\\
0.06	1.33\\
0.06	1.33\\
0.07	1.32\\
0.07	1.32\\
0.08	1.31\\
0.08	1.31\\
0.08	1.31\\
0.08	1.3\\
0.09	1.3\\
0.09	1.3\\
0.1	1.29\\
0.1	1.29\\
0.1	1.28\\
0.11	1.28\\
0.11	1.27\\
0.12	1.27\\
0.12	1.27\\
0.12	1.26\\
0.12	1.26\\
0.13	1.26\\
0.13	1.25\\
0.14	1.25\\
0.14	1.24\\
0.14	1.24\\
0.15	1.24\\
0.15	1.23\\
0.15	1.23\\
0.16	1.22\\
0.16	1.22\\
0.17	1.22\\
0.17	1.21\\
0.17	1.21\\
0.17	1.21\\
0.18	1.2\\
0.18	1.2\\
0.19	1.19\\
0.19	1.19\\
0.19	1.19\\
0.2	1.18\\
0.2	1.18\\
0.2	1.18\\
0.21	1.17\\
0.21	1.17\\
0.21	1.17\\
0.22	1.16\\
0.22	1.16\\
0.22	1.16\\
0.22	1.15\\
0.23	1.15\\
0.23	1.15\\
0.23	1.14\\
0.24	1.14\\
0.24	1.14\\
0.24	1.13\\
0.25	1.13\\
0.25	1.13\\
0.25	1.12\\
0.25	1.12\\
0.26	1.12\\
0.26	1.12\\
0.26	1.11\\
0.26	1.11\\
0.27	1.11\\
0.27	1.1\\
0.27	1.1\\
0.27	1.1\\
0.28	1.09\\
0.28	1.09\\
0.28	1.09\\
0.28	1.09\\
0.29	1.08\\
0.29	1.08\\
0.29	1.08\\
0.3	1.07\\
0.3	1.07\\
0.3	1.07\\
0.31	1.06\\
0.31	1.06\\
0.31	1.06\\
0.32	1.05\\
0.32	1.05\\
0.32	1.05\\
0.32	1.04\\
0.33	1.04\\
0.33	1.04\\
0.33	1.03\\
0.34	1.03\\
0.34	1.02\\
0.34	1.02\\
0.35	1.02\\
0.35	1.01\\
0.36	1.01\\
0.36	1.01\\
0.36	1\\
0.36	1\\
0.37	1\\
0.37	0.99\\
0.37	0.99\\
0.38	0.99\\
0.38	0.98\\
0.38	0.98\\
0.39	0.98\\
0.39	0.97\\
0.39	0.97\\
0.39	0.97\\
0.39	0.97\\
0.4	0.96\\
0.4	0.96\\
0.4	0.96\\
0.4	0.96\\
0.41	0.95\\
0.41	0.95\\
0.41	0.95\\
0.41	0.95\\
0.42	0.94\\
0.42	0.94\\
0.42	0.94\\
0.42	0.94\\
0.42	0.94\\
0.42	0.93\\
0.43	0.93\\
0.43	0.93\\
0.43	0.93\\
0.43	0.93\\
0.43	0.92\\
0.44	0.92\\
0.44	0.92\\
0.44	0.92\\
0.44	0.91\\
0.44	0.91\\
0.45	0.91\\
0.45	0.91\\
0.45	0.91\\
0.45	0.9\\
0.46	0.9\\
0.46	0.9\\
0.46	0.9\\
0.46	0.89\\
0.47	0.89\\
0.47	0.89\\
0.47	0.88\\
0.47	0.88\\
0.48	0.88\\
0.48	0.88\\
0.48	0.87\\
0.48	0.87\\
0.49	0.87\\
0.49	0.86\\
0.49	0.86\\
0.5	0.86\\
0.5	0.85\\
0.5	0.85\\
0.5	0.85\\
0.51	0.84\\
0.51	0.84\\
0.51	0.84\\
0.51	0.84\\
0.52	0.83\\
0.52	0.83\\
0.52	0.83\\
0.53	0.82\\
0.53	0.82\\
0.53	0.82\\
0.53	0.82\\
0.53	0.81\\
0.54	0.81\\
0.54	0.81\\
0.54	0.81\\
0.54	0.81\\
0.54	0.81\\
0.54	0.8\\
0.55	0.8\\
0.55	0.8\\
0.55	0.8\\
0.55	0.8\\
0.55	0.8\\
0.55	0.79\\
0.55	0.79\\
0.56	0.79\\
0.56	0.79\\
0.56	0.79\\
0.56	0.79\\
0.56	0.79\\
0.56	0.78\\
0.56	0.78\\
0.57	0.78\\
0.57	0.78\\
0.57	0.78\\
0.57	0.78\\
0.57	0.77\\
0.57	0.77\\
0.58	0.77\\
0.58	0.77\\
0.58	0.77\\
0.58	0.76\\
0.58	0.76\\
0.59	0.76\\
0.59	0.76\\
0.59	0.75\\
0.59	0.75\\
0.6	0.75\\
0.6	0.75\\
0.6	0.74\\
0.6	0.74\\
0.6	0.74\\
0.61	0.74\\
0.61	0.73\\
0.61	0.73\\
0.61	0.73\\
0.61	0.73\\
0.62	0.72\\
0.62	0.72\\
0.62	0.72\\
0.62	0.72\\
0.63	0.71\\
0.63	0.71\\
0.63	0.71\\
0.63	0.71\\
0.63	0.71\\
0.63	0.71\\
0.64	0.7\\
0.64	0.7\\
0.64	0.7\\
0.64	0.7\\
0.64	0.7\\
0.64	0.69\\
0.65	0.69\\
0.65	0.69\\
0.65	0.69\\
0.65	0.69\\
0.65	0.69\\
0.65	0.69\\
0.65	0.68\\
0.65	0.68\\
0.66	0.68\\
0.66	0.68\\
0.66	0.68\\
0.66	0.68\\
0.66	0.68\\
0.66	0.68\\
0.66	0.67\\
0.66	0.67\\
0.66	0.67\\
0.67	0.67\\
0.67	0.67\\
0.67	0.67\\
0.67	0.67\\
0.67	0.67\\
0.67	0.66\\
0.67	0.66\\
0.67	0.66\\
0.68	0.66\\
0.68	0.66\\
0.68	0.66\\
0.68	0.66\\
0.68	0.65\\
0.68	0.65\\
0.69	0.65\\
0.69	0.65\\
0.69	0.65\\
0.69	0.64\\
0.69	0.64\\
0.69	0.64\\
0.7	0.64\\
0.7	0.64\\
0.7	0.63\\
0.7	0.63\\
0.7	0.63\\
0.7	0.63\\
0.71	0.63\\
0.71	0.63\\
0.71	0.62\\
0.71	0.62\\
0.71	0.62\\
0.72	0.62\\
0.72	0.62\\
0.72	0.61\\
0.72	0.61\\
0.72	0.61\\
0.72	0.61\\
0.72	0.61\\
0.73	0.61\\
0.73	0.61\\
0.73	0.6\\
0.73	0.6\\
0.73	0.6\\
0.73	0.6\\
0.73	0.6\\
0.73	0.6\\
0.74	0.6\\
0.74	0.6\\
0.74	0.59\\
0.74	0.59\\
0.74	0.59\\
0.74	0.59\\
0.74	0.59\\
0.74	0.59\\
0.74	0.59\\
0.74	0.59\\
0.74	0.59\\
0.74	0.59\\
0.75	0.59\\
0.75	0.58\\
0.75	0.58\\
0.75	0.58\\
0.75	0.58\\
0.75	0.58\\
0.75	0.58\\
0.75	0.58\\
0.75	0.58\\
0.76	0.58\\
0.76	0.57\\
0.76	0.57\\
0.76	0.57\\
0.76	0.57\\
0.76	0.57\\
0.76	0.57\\
0.77	0.56\\
0.77	0.56\\
0.77	0.56\\
0.77	0.56\\
0.77	0.56\\
0.77	0.56\\
0.77	0.56\\
0.78	0.55\\
0.78	0.55\\
0.78	0.55\\
0.78	0.55\\
0.78	0.55\\
0.78	0.55\\
0.78	0.55\\
0.79	0.54\\
0.79	0.54\\
0.79	0.54\\
0.79	0.54\\
0.79	0.54\\
0.79	0.54\\
0.79	0.53\\
0.79	0.53\\
0.79	0.53\\
0.8	0.53\\
0.8	0.53\\
0.8	0.53\\
0.8	0.53\\
0.8	0.53\\
0.8	0.53\\
0.8	0.52\\
0.8	0.52\\
0.8	0.52\\
0.81	0.52\\
0.81	0.52\\
0.81	0.52\\
0.81	0.52\\
0.81	0.52\\
0.81	0.52\\
0.81	0.52\\
0.81	0.51\\
0.81	0.51\\
0.81	0.51\\
0.81	0.51\\
0.81	0.51\\
0.82	0.51\\
0.82	0.51\\
0.82	0.51\\
0.82	0.51\\
0.82	0.51\\
0.82	0.51\\
0.82	0.51\\
0.82	0.51\\
0.82	0.5\\
0.82	0.5\\
0.82	0.5\\
0.82	0.5\\
0.82	0.5\\
0.82	0.5\\
0.83	0.5\\
0.83	0.5\\
0.83	0.5\\
0.83	0.5\\
0.83	0.5\\
0.83	0.49\\
0.83	0.49\\
0.83	0.49\\
0.83	0.49\\
0.83	0.49\\
0.84	0.49\\
0.84	0.49\\
0.84	0.49\\
0.84	0.48\\
0.84	0.48\\
0.84	0.48\\
0.84	0.48\\
0.84	0.48\\
0.84	0.48\\
0.85	0.48\\
0.85	0.48\\
0.85	0.47\\
0.85	0.47\\
0.85	0.47\\
0.85	0.47\\
0.85	0.47\\
0.85	0.47\\
0.85	0.47\\
0.85	0.47\\
0.86	0.47\\
0.86	0.47\\
0.86	0.46\\
0.86	0.46\\
0.86	0.46\\
0.86	0.46\\
0.86	0.46\\
0.86	0.46\\
0.86	0.46\\
0.86	0.46\\
0.86	0.46\\
0.86	0.46\\
0.86	0.46\\
0.86	0.46\\
0.87	0.46\\
0.87	0.46\\
0.87	0.45\\
0.87	0.45\\
0.87	0.45\\
0.87	0.45\\
0.87	0.45\\
0.87	0.45\\
0.87	0.45\\
0.87	0.45\\
0.87	0.45\\
0.87	0.45\\
0.87	0.45\\
0.87	0.45\\
0.87	0.45\\
0.87	0.45\\
0.87	0.44\\
0.88	0.44\\
0.88	0.44\\
0.88	0.44\\
0.88	0.44\\
0.88	0.44\\
0.88	0.44\\
0.88	0.44\\
0.88	0.44\\
0.88	0.44\\
0.88	0.44\\
0.88	0.43\\
0.89	0.43\\
0.89	0.43\\
0.89	0.43\\
0.89	0.43\\
0.89	0.43\\
0.89	0.43\\
0.89	0.43\\
0.89	0.43\\
0.89	0.43\\
0.89	0.42\\
0.89	0.42\\
0.89	0.42\\
0.9	0.42\\
0.9	0.42\\
0.9	0.42\\
0.9	0.42\\
0.9	0.42\\
0.9	0.42\\
0.9	0.42\\
0.9	0.42\\
0.9	0.42\\
0.9	0.42\\
0.9	0.42\\
0.9	0.42\\
0.9	0.41\\
0.9	0.41\\
0.9	0.41\\
0.9	0.41\\
0.9	0.41\\
0.91	0.41\\
0.91	0.41\\
0.91	0.41\\
0.91	0.41\\
0.91	0.41\\
0.91	0.41\\
0.91	0.41\\
0.91	0.41\\
0.91	0.41\\
0.91	0.41\\
0.91	0.41\\
0.91	0.41\\
0.91	0.4\\
0.91	0.4\\
0.91	0.4\\
0.91	0.4\\
0.91	0.4\\
0.91	0.4\\
0.91	0.4\\
0.91	0.4\\
0.91	0.4\\
0.92	0.4\\
0.92	0.4\\
0.92	0.4\\
0.92	0.4\\
0.92	0.4\\
0.92	0.4\\
0.92	0.4\\
0.92	0.39\\
0.92	0.39\\
0.92	0.39\\
0.92	0.39\\
0.92	0.39\\
0.92	0.39\\
0.92	0.39\\
0.92	0.39\\
0.93	0.39\\
0.93	0.39\\
0.93	0.39\\
0.93	0.39\\
0.93	0.39\\
0.93	0.38\\
0.93	0.38\\
0.93	0.38\\
0.93	0.38\\
0.93	0.38\\
0.93	0.38\\
0.93	0.38\\
0.93	0.38\\
0.93	0.38\\
0.94	0.38\\
0.94	0.38\\
0.94	0.38\\
0.94	0.38\\
0.94	0.38\\
0.94	0.38\\
0.94	0.38\\
0.94	0.37\\
0.94	0.37\\
0.94	0.37\\
0.94	0.37\\
0.94	0.37\\
0.94	0.37\\
0.94	0.37\\
0.94	0.37\\
0.94	0.37\\
0.94	0.37\\
0.94	0.37\\
0.94	0.37\\
0.94	0.37\\
0.94	0.37\\
0.94	0.37\\
0.94	0.37\\
0.94	0.37\\
0.94	0.37\\
0.95	0.37\\
0.95	0.37\\
0.95	0.37\\
0.95	0.37\\
0.95	0.37\\
0.95	0.37\\
0.95	0.37\\
0.95	0.37\\
0.95	0.37\\
0.95	0.36\\
0.95	0.36\\
0.95	0.36\\
0.95	0.36\\
0.95	0.36\\
0.95	0.36\\
0.95	0.36\\
0.95	0.36\\
0.95	0.36\\
0.95	0.36\\
0.96	0.36\\
0.96	0.36\\
0.96	0.36\\
0.96	0.36\\
0.96	0.35\\
0.96	0.35\\
0.96	0.35\\
0.96	0.35\\
0.96	0.35\\
0.96	0.35\\
0.96	0.35\\
0.96	0.35\\
0.96	0.35\\
0.96	0.35\\
0.96	0.35\\
0.96	0.35\\
0.96	0.35\\
0.96	0.35\\
0.97	0.35\\
0.97	0.35\\
0.97	0.35\\
0.97	0.35\\
0.97	0.34\\
0.97	0.34\\
0.97	0.34\\
0.97	0.34\\
0.97	0.34\\
0.97	0.34\\
0.97	0.34\\
0.97	0.34\\
0.97	0.34\\
0.97	0.34\\
0.97	0.34\\
0.97	0.34\\
0.97	0.34\\
0.97	0.34\\
0.97	0.34\\
0.97	0.34\\
0.97	0.34\\
0.97	0.34\\
0.97	0.34\\
0.97	0.34\\
0.97	0.34\\
0.97	0.34\\
0.97	0.34\\
0.97	0.34\\
0.97	0.34\\
0.97	0.34\\
0.97	0.34\\
0.97	0.34\\
0.97	0.34\\
0.97	0.34\\
0.97	0.34\\
0.97	0.34\\
0.97	0.34\\
0.97	0.34\\
0.97	0.34\\
0.97	0.34\\
0.97	0.34\\
0.97	0.34\\
0.97	0.34\\
0.97	0.34\\
0.97	0.34\\
0.98	0.33\\
0.98	0.33\\
0.98	0.33\\
0.98	0.33\\
0.98	0.33\\
0.98	0.33\\
0.98	0.33\\
0.98	0.33\\
0.98	0.33\\
0.98	0.33\\
0.98	0.33\\
0.99	0.33\\
0.98	0.33\\
0.99	0.33\\
0.99	0.32\\
0.99	0.32\\
0.99	0.32\\
0.99	0.32\\
0.99	0.32\\
0.99	0.32\\
0.99	0.32\\
0.99	0.32\\
0.99	0.32\\
0.99	0.32\\
0.99	0.32\\
0.99	0.32\\
0.99	0.32\\
0.99	0.32\\
0.99	0.32\\
0.99	0.32\\
0.99	0.32\\
0.99	0.32\\
0.99	0.32\\
0.99	0.32\\
0.99	0.32\\
0.99	0.32\\
0.99	0.32\\
0.99	0.32\\
0.99	0.32\\
0.99	0.32\\
0.99	0.32\\
0.99	0.32\\
0.99	0.32\\
0.99	0.32\\
0.99	0.32\\
0.99	0.32\\
0.99	0.32\\
0.99	0.32\\
0.99	0.32\\
0.99	0.32\\
0.99	0.32\\
0.99	0.32\\
0.99	0.32\\
0.99	0.32\\
0.99	0.32\\
0.99	0.32\\
0.99	0.32\\
0.99	0.32\\
0.99	0.32\\
0.99	0.32\\
0.99	0.32\\
0.99	0.32\\
0.99	0.32\\
0.99	0.32\\
0.99	0.32\\
0.99	0.32\\
0.99	0.32\\
0.99	0.32\\
0.99	0.32\\
0.99	0.32\\
0.99	0.32\\
0.99	0.32\\
0.99	0.32\\
0.99	0.32\\
0.99	0.32\\
0.99	0.32\\
0.99	0.32\\
0.99	0.32\\
0.99	0.32\\
0.99	0.32\\
0.99	0.32\\
0.99	0.32\\
0.99	0.32\\
0.99	0.32\\
0.99	0.32\\
0.99	0.32\\
0.99	0.32\\
0.99	0.32\\
0.99	0.32\\
0.99	0.32\\
0.99	0.32\\
0.99	0.32\\
0.99	0.32\\
0.99	0.32\\
0.99	0.32\\
0.99	0.32\\
0.99	0.32\\
0.99	0.32\\
0.99	0.32\\
0.99	0.32\\
0.99	0.32\\
0.99	0.32\\
0.99	0.32\\
1	0.32\\
1	0.31\\
1	0.31\\
1	0.31\\
1	0.31\\
1	0.31\\
1	0.31\\
1	0.31\\
1	0.31\\
1	0.31\\
1	0.31\\
1	0.31\\
1	0.31\\
1	0.31\\
1	0.31\\
1	0.31\\
1	0.31\\
1	0.31\\
1.01	0.31\\
1.01	0.3\\
1.01	0.3\\
1.01	0.3\\
1.01	0.3\\
1.01	0.3\\
1.01	0.3\\
1.01	0.3\\
1.01	0.3\\
1.01	0.3\\
1.01	0.3\\
1.01	0.3\\
1.01	0.3\\
1.01	0.3\\
1.01	0.3\\
1.01	0.3\\
1.01	0.3\\
1.01	0.3\\
1.01	0.3\\
1.01	0.3\\
1.01	0.3\\
1.01	0.3\\
1.01	0.3\\
1.01	0.3\\
1.01	0.3\\
1.01	0.3\\
1.01	0.3\\
1.01	0.3\\
1.01	0.3\\
1.01	0.3\\
1.01	0.3\\
1.01	0.3\\
1.01	0.3\\
1.01	0.3\\
1.01	0.3\\
1.01	0.3\\
1.01	0.3\\
1.01	0.3\\
1.01	0.3\\
1.01	0.3\\
1.01	0.3\\
1.01	0.3\\
1.01	0.3\\
1.01	0.3\\
1.01	0.3\\
1.01	0.3\\
1.01	0.3\\
1.01	0.3\\
1.01	0.3\\
1.02	0.3\\
1.02	0.29\\
1.02	0.29\\
1.02	0.29\\
1.02	0.29\\
1.02	0.29\\
1.02	0.29\\
1.02	0.29\\
1.02	0.29\\
1.02	0.29\\
1.02	0.29\\
1.02	0.29\\
1.02	0.29\\
1.02	0.29\\
1.02	0.29\\
1.02	0.29\\
1.02	0.29\\
1.02	0.29\\
1.02	0.29\\
1.02	0.29\\
1.02	0.29\\
1.02	0.29\\
1.02	0.29\\
1.02	0.29\\
1.02	0.29\\
1.02	0.29\\
1.02	0.29\\
1.02	0.29\\
1.02	0.29\\
1.02	0.29\\
1.02	0.29\\
1.02	0.29\\
1.02	0.29\\
1.02	0.29\\
1.02	0.29\\
1.02	0.29\\
1.02	0.29\\
1.02	0.29\\
1.02	0.29\\
1.02	0.29\\
1.02	0.29\\
1.02	0.29\\
1.02	0.29\\
1.02	0.29\\
1.02	0.29\\
1.02	0.29\\
1.02	0.29\\
1.02	0.29\\
1.02	0.29\\
1.02	0.29\\
1.02	0.29\\
1.02	0.29\\
1.02	0.29\\
1.02	0.29\\
1.02	0.29\\
1.02	0.29\\
1.02	0.29\\
1.02	0.29\\
1.02	0.29\\
1.02	0.29\\
1.02	0.29\\
1.02	0.29\\
1.02	0.29\\
1.02	0.29\\
1.02	0.29\\
1.02	0.29\\
1.02	0.29\\
1.02	0.29\\
1.02	0.29\\
1.02	0.29\\
1.02	0.29\\
1.02	0.29\\
1.02	0.29\\
1.02	0.29\\
1.02	0.29\\
1.02	0.29\\
1.02	0.29\\
1.02	0.29\\
1.02	0.29\\
1.02	0.29\\
1.02	0.29\\
1.02	0.29\\
1.02	0.29\\
1.02	0.29\\
1.02	0.29\\
1.02	0.29\\
1.02	0.29\\
1.02	0.29\\
1.02	0.29\\
1.02	0.29\\
1.02	0.29\\
1.02	0.29\\
1.02	0.29\\
1.02	0.29\\
1.02	0.29\\
1.02	0.29\\
1.02	0.29\\
1.02	0.29\\
1.02	0.29\\
1.02	0.29\\
1.02	0.29\\
1.02	0.29\\
1.02	0.29\\
1.02	0.29\\
1.02	0.29\\
1.02	0.29\\
1.02	0.29\\
1.02	0.29\\
1.02	0.29\\
1.02	0.29\\
1.02	0.29\\
1.02	0.29\\
1.02	0.29\\
1.02	0.29\\
1.02	0.29\\
1.02	0.29\\
1.02	0.29\\
1.02	0.29\\
1.02	0.29\\
1.02	0.29\\
1.02	0.29\\
1.02	0.29\\
1.02	0.29\\
1.02	0.29\\
1.02	0.29\\
1.02	0.29\\
1.02	0.29\\
1.02	0.29\\
1.02	0.29\\
1.02	0.29\\
1.02	0.29\\
1.02	0.29\\
1.02	0.29\\
1.02	0.29\\
1.02	0.29\\
1.02	0.29\\
1.02	0.29\\
1.02	0.29\\
1.02	0.29\\
1.02	0.29\\
1.02	0.29\\
1.02	0.29\\
1.02	0.29\\
1.02	0.29\\
1.02	0.29\\
1.02	0.29\\
1.02	0.29\\
1.03	0.29\\
1.03	0.29\\
1.02	0.29\\
1.02	0.29\\
1.02	0.29\\
1.03	0.28\\
1.03	0.28\\
1.03	0.28\\
1.03	0.28\\
1.03	0.28\\
1.03	0.28\\
1.03	0.28\\
1.03	0.28\\
1.03	0.28\\
1.03	0.28\\
1.03	0.28\\
1.03	0.28\\
1.03	0.28\\
1.03	0.28\\
1.03	0.28\\
1.03	0.28\\
1.03	0.28\\
1.03	0.28\\
1.03	0.28\\
1.03	0.28\\
1.04	0.28\\
1.03	0.28\\
1.03	0.28\\
1.03	0.28\\
1.03	0.28\\
1.03	0.28\\
1.03	0.28\\
1.03	0.28\\
1.03	0.28\\
1.03	0.28\\
1.03	0.28\\
1.03	0.28\\
1.03	0.28\\
1.04	0.28\\
1.03	0.28\\
1.03	0.28\\
1.04	0.28\\
1.03	0.28\\
1.03	0.28\\
1.04	0.28\\
1.03	0.28\\
1.03	0.28\\
1.03	0.28\\
1.03	0.28\\
1.04	0.28\\
1.03	0.28\\
1.03	0.28\\
1.03	0.28\\
1.03	0.28\\
1.03	0.28\\
1.03	0.28\\
1.03	0.28\\
1.03	0.28\\
1.03	0.28\\
1.03	0.28\\
1.03	0.28\\
1.03	0.28\\
1.03	0.28\\
1.03	0.28\\
1.03	0.28\\
1.03	0.28\\
1.03	0.28\\
1.03	0.28\\
1.03	0.28\\
1.03	0.28\\
1.03	0.28\\
1.03	0.28\\
1.03	0.28\\
1.03	0.28\\
1.03	0.28\\
1.04	0.28\\
1.03	0.28\\
1.03	0.28\\
1.03	0.28\\
1.03	0.28\\
1.03	0.28\\
1.03	0.28\\
1.03	0.28\\
1.03	0.28\\
1.03	0.28\\
1.03	0.28\\
1.03	0.28\\
1.03	0.28\\
1.03	0.28\\
1.03	0.28\\
1.04	0.28\\
1.04	0.28\\
1.03	0.28\\
1.03	0.28\\
1.03	0.28\\
1.03	0.28\\
1.03	0.28\\
1.03	0.28\\
1.03	0.28\\
1.03	0.28\\
1.03	0.28\\
1.03	0.28\\
1.04	0.28\\
1.03	0.28\\
1.03	0.28\\
1.03	0.28\\
1.03	0.28\\
1.04	0.28\\
1.03	0.28\\
1.03	0.28\\
1.03	0.28\\
1.03	0.28\\
1.03	0.28\\
1.03	0.28\\
1.03	0.28\\
1.03	0.28\\
1.03	0.28\\
1.03	0.28\\
1.04	0.28\\
1.03	0.28\\
1.03	0.28\\
1.03	0.28\\
1.03	0.28\\
1.03	0.28\\
1.03	0.28\\
1.03	0.28\\
1.03	0.28\\
1.03	0.28\\
1.03	0.28\\
1.04	0.28\\
1.03	0.28\\
1.03	0.28\\
1.03	0.28\\
1.03	0.28\\
1.03	0.28\\
1.03	0.28\\
1.03	0.28\\
1.04	0.28\\
1.04	0.28\\
1.03	0.28\\
1.03	0.28\\
1.03	0.28\\
1.03	0.28\\
1.03	0.28\\
1.03	0.28\\
1.03	0.28\\
1.03	0.28\\
1.03	0.28\\
1.03	0.28\\
1.03	0.28\\
1.03	0.28\\
1.03	0.28\\
1.03	0.28\\
1.03	0.28\\
1.03	0.28\\
1.03	0.28\\
1.04	0.28\\
1.03	0.28\\
1.04	0.28\\
1.03	0.28\\
1.04	0.28\\
1.03	0.28\\
1.03	0.28\\
1.03	0.28\\
1.03	0.28\\
1.03	0.28\\
1.04	0.28\\
1.03	0.28\\
1.03	0.28\\
1.04	0.28\\
1.04	0.28\\
1.03	0.28\\
1.03	0.28\\
1.03	0.28\\
1.03	0.28\\
1.04	0.28\\
1.04	0.27\\
1.04	0.27\\
1.04	0.27\\
1.04	0.27\\
1.04	0.27\\
1.04	0.27\\
1.04	0.27\\
1.04	0.27\\
1.04	0.27\\
1.04	0.27\\
1.04	0.27\\
1.04	0.27\\
1.04	0.27\\
1.04	0.27\\
1.04	0.27\\
1.04	0.27\\
1.04	0.27\\
1.04	0.27\\
1.04	0.27\\
1.04	0.27\\
1.04	0.27\\
1.04	0.27\\
1.04	0.27\\
1.04	0.27\\
1.04	0.27\\
1.04	0.27\\
1.04	0.27\\
1.04	0.27\\
1.04	0.27\\
1.04	0.27\\
1.04	0.27\\
1.04	0.27\\
1.04	0.27\\
1.04	0.27\\
1.04	0.27\\
1.04	0.27\\
1.04	0.27\\
1.04	0.27\\
1.04	0.27\\
1.04	0.27\\
1.04	0.27\\
1.04	0.27\\
1.04	0.27\\
1.04	0.27\\
1.04	0.27\\
1.04	0.27\\
1.04	0.27\\
1.04	0.27\\
1.04	0.27\\
1.04	0.27\\
1.04	0.27\\
1.04	0.27\\
1.04	0.27\\
1.04	0.27\\
1.04	0.27\\
1.04	0.27\\
1.04	0.27\\
1.04	0.27\\
1.04	0.27\\
1.04	0.27\\
1.04	0.27\\
1.04	0.27\\
1.04	0.27\\
1.04	0.27\\
1.04	0.27\\
1.04	0.27\\
1.04	0.27\\
1.04	0.27\\
1.04	0.27\\
1.04	0.27\\
1.04	0.27\\
1.04	0.27\\
1.04	0.27\\
1.04	0.27\\
1.04	0.27\\
1.04	0.27\\
1.04	0.27\\
1.04	0.27\\
1.04	0.27\\
1.04	0.27\\
1.04	0.27\\
1.04	0.27\\
1.04	0.27\\
1.04	0.27\\
1.04	0.27\\
1.04	0.27\\
1.04	0.27\\
1.04	0.27\\
1.04	0.27\\
1.04	0.27\\
1.04	0.27\\
1.04	0.27\\
1.04	0.27\\
1.04	0.27\\
1.04	0.27\\
1.04	0.27\\
1.04	0.27\\
1.04	0.27\\
1.04	0.27\\
1.04	0.27\\
1.04	0.27\\
1.04	0.27\\
1.04	0.27\\
1.04	0.27\\
1.04	0.27\\
1.04	0.27\\
1.04	0.27\\
1.04	0.27\\
1.04	0.27\\
1.04	0.27\\
1.04	0.27\\
1.04	0.27\\
1.04	0.27\\
1.04	0.27\\
1.04	0.27\\
1.04	0.27\\
1.04	0.27\\
1.04	0.27\\
1.04	0.27\\
1.04	0.27\\
1.04	0.27\\
1.04	0.27\\
1.04	0.27\\
1.04	0.27\\
1.04	0.27\\
1.04	0.27\\
1.04	0.27\\
1.04	0.27\\
1.04	0.27\\
1.04	0.27\\
1.04	0.27\\
1.04	0.27\\
1.04	0.27\\
1.04	0.27\\
1.04	0.27\\
1.04	0.27\\
1.04	0.27\\
1.04	0.27\\
1.04	0.27\\
1.04	0.27\\
1.04	0.27\\
1.04	0.27\\
1.04	0.27\\
1.04	0.27\\
1.04	0.27\\
1.04	0.27\\
1.04	0.27\\
1.04	0.27\\
1.04	0.27\\
1.04	0.27\\
1.04	0.27\\
1.04	0.27\\
1.04	0.27\\
1.04	0.27\\
1.04	0.27\\
1.04	0.27\\
1.04	0.27\\
1.04	0.27\\
1.04	0.27\\
1.04	0.27\\
1.04	0.27\\
1.04	0.27\\
1.04	0.27\\
1.04	0.27\\
1.04	0.27\\
1.04	0.27\\
1.04	0.27\\
1.04	0.27\\
1.04	0.27\\
1.04	0.27\\
1.04	0.27\\
1.04	0.27\\
1.04	0.27\\
1.04	0.27\\
1.04	0.27\\
1.04	0.27\\
1.04	0.27\\
1.04	0.27\\
1.04	0.27\\
1.04	0.27\\
1.04	0.27\\
1.04	0.27\\
1.04	0.27\\
1.04	0.27\\
1.04	0.27\\
1.04	0.27\\
1.04	0.27\\
1.04	0.27\\
1.04	0.27\\
1.04	0.27\\
1.04	0.27\\
1.04	0.27\\
1.04	0.27\\
1.04	0.27\\
1.04	0.27\\
1.04	0.27\\
1.04	0.27\\
1.04	0.27\\
1.04	0.27\\
1.04	0.27\\
1.04	0.27\\
1.04	0.27\\
1.04	0.27\\
1.04	0.27\\
1.04	0.27\\
1.04	0.27\\
1.04	0.27\\
1.04	0.27\\
1.04	0.27\\
1.04	0.27\\
1.04	0.27\\
1.04	0.27\\
1.04	0.27\\
1.04	0.27\\
1.04	0.27\\
1.04	0.27\\
1.04	0.27\\
1.04	0.27\\
1.04	0.27\\
1.04	0.27\\
1.04	0.27\\
1.04	0.27\\
1.04	0.27\\
1.04	0.27\\
1.04	0.27\\
1.04	0.27\\
1.04	0.27\\
1.04	0.27\\
1.04	0.27\\
1.04	0.27\\
1.04	0.27\\
1.04	0.27\\
1.04	0.27\\
1.04	0.27\\
1.04	0.27\\
1.04	0.27\\
1.04	0.27\\
1.04	0.27\\
1.04	0.27\\
1.04	0.27\\
1.04	0.27\\
1.04	0.27\\
1.04	0.27\\
1.04	0.27\\
1.04	0.27\\
1.04	0.27\\
1.04	0.27\\
1.04	0.27\\
1.04	0.27\\
1.04	0.27\\
1.04	0.27\\
1.04	0.27\\
1.04	0.27\\
1.04	0.27\\
1.04	0.27\\
1.04	0.27\\
1.04	0.27\\
1.04	0.27\\
1.04	0.27\\
1.04	0.27\\
1.04	0.27\\
1.04	0.27\\
1.04	0.27\\
1.04	0.27\\
1.04	0.27\\
1.04	0.27\\
1.04	0.27\\
1.04	0.27\\
1.04	0.27\\
1.04	0.27\\
1.04	0.27\\
1.04	0.27\\
1.04	0.27\\
1.04	0.27\\
1.04	0.27\\
1.04	0.27\\
1.04	0.27\\
1.04	0.27\\
1.04	0.27\\
1.04	0.27\\
1.04	0.27\\
1.04	0.27\\
1.04	0.27\\
1.04	0.27\\
1.04	0.27\\
1.04	0.27\\
1.04	0.27\\
1.04	0.27\\
1.04	0.27\\
1.04	0.27\\
1.04	0.27\\
1.04	0.27\\
1.04	0.27\\
1.04	0.27\\
1.04	0.27\\
1.04	0.27\\
1.04	0.27\\
1.04	0.27\\
1.04	0.27\\
1.04	0.27\\
1.04	0.27\\
1.04	0.27\\
1.04	0.27\\
1.04	0.27\\
1.04	0.27\\
1.04	0.27\\
1.04	0.27\\
1.04	0.27\\
1.04	0.27\\
1.04	0.27\\
1.04	0.27\\
1.04	0.27\\
1.04	0.27\\
1.04	0.27\\
1.04	0.27\\
1.04	0.27\\
1.04	0.27\\
1.04	0.27\\
1.04	0.27\\
1.04	0.27\\
1.04	0.27\\
1.04	0.27\\
1.04	0.27\\
1.04	0.27\\
1.04	0.27\\
1.04	0.27\\
1.04	0.27\\
1.04	0.27\\
1.04	0.27\\
1.04	0.27\\
1.04	0.27\\
1.04	0.27\\
1.04	0.27\\
1.04	0.27\\
1.04	0.27\\
1.04	0.27\\
1.04	0.27\\
1.04	0.27\\
1.04	0.27\\
1.04	0.27\\
1.04	0.27\\
1.04	0.27\\
1.04	0.27\\
1.04	0.27\\
1.04	0.27\\
1.04	0.27\\
1.04	0.27\\
1.04	0.27\\
1.04	0.27\\
1.04	0.27\\
1.04	0.27\\
1.04	0.27\\
1.04	0.27\\
1.04	0.27\\
1.04	0.27\\
1.04	0.27\\
1.04	0.27\\
1.04	0.27\\
1.04	0.27\\
1.04	0.27\\
1.04	0.27\\
1.04	0.27\\
1.04	0.27\\
1.04	0.27\\
1.04	0.27\\
1.04	0.27\\
1.04	0.27\\
1.04	0.27\\
1.04	0.27\\
1.04	0.27\\
1.04	0.27\\
1.04	0.27\\
1.04	0.27\\
1.04	0.27\\
1.04	0.27\\
1.04	0.27\\
1.04	0.27\\
1.04	0.27\\
1.04	0.27\\
1.04	0.27\\
1.04	0.27\\
1.04	0.27\\
1.04	0.27\\
1.04	0.27\\
1.04	0.27\\
1.04	0.27\\
1.04	0.27\\
1.04	0.27\\
1.04	0.27\\
1.04	0.27\\
1.04	0.27\\
1.04	0.27\\
1.04	0.27\\
1.04	0.27\\
1.04	0.27\\
1.04	0.27\\
1.04	0.27\\
1.04	0.27\\
1.04	0.27\\
1.04	0.27\\
1.04	0.27\\
1.04	0.27\\
1.04	0.27\\
1.04	0.27\\
1.04	0.27\\
1.04	0.27\\
1.04	0.27\\
1.04	0.27\\
1.04	0.27\\
1.04	0.27\\
1.04	0.27\\
1.04	0.27\\
1.04	0.27\\
1.04	0.27\\
1.04	0.27\\
1.04	0.27\\
1.04	0.27\\
1.04	0.27\\
1.04	0.27\\
1.04	0.27\\
1.04	0.27\\
1.04	0.27\\
1.04	0.27\\
1.04	0.27\\
1.04	0.27\\
1.04	0.27\\
1.04	0.27\\
1.04	0.27\\
1.04	0.27\\
1.04	0.27\\
1.04	0.27\\
1.04	0.27\\
1.04	0.27\\
1.04	0.27\\
1.04	0.27\\
1.04	0.27\\
1.04	0.27\\
1.04	0.27\\
1.04	0.27\\
1.04	0.27\\
1.04	0.27\\
1.04	0.27\\
1.04	0.27\\
1.04	0.27\\
1.04	0.27\\
1.04	0.27\\
1.04	0.27\\
1.04	0.27\\
1.04	0.27\\
1.04	0.27\\
1.04	0.27\\
1.04	0.27\\
1.04	0.27\\
1.04	0.27\\
1.04	0.27\\
1.04	0.27\\
1.04	0.27\\
1.04	0.27\\
1.04	0.27\\
1.04	0.27\\
1.04	0.27\\
1.04	0.27\\
1.04	0.27\\
1.04	0.27\\
1.04	0.27\\
1.04	0.27\\
1.04	0.27\\
1.04	0.27\\
1.04	0.27\\
1.04	0.27\\
1.04	0.27\\
1.04	0.27\\
1.04	0.27\\
1.04	0.27\\
1.04	0.27\\
1.04	0.27\\
1.04	0.27\\
1.04	0.27\\
1.04	0.27\\
1.04	0.27\\
1.04	0.27\\
1.04	0.27\\
1.04	0.27\\
1.04	0.27\\
1.04	0.27\\
1.04	0.27\\
1.04	0.27\\
1.04	0.27\\
1.04	0.27\\
1.04	0.27\\
1.04	0.27\\
1.04	0.27\\
1.04	0.27\\
1.04	0.27\\
1.04	0.27\\
1.04	0.27\\
1.04	0.27\\
1.04	0.27\\
1.04	0.27\\
1.04	0.27\\
1.04	0.27\\
1.04	0.27\\
1.04	0.27\\
1.04	0.27\\
1.04	0.27\\
1.04	0.27\\
1.04	0.27\\
1.04	0.27\\
1.04	0.27\\
1.04	0.27\\
1.04	0.27\\
1.04	0.27\\
1.04	0.27\\
1.04	0.27\\
1.04	0.27\\
1.04	0.27\\
1.04	0.27\\
1.04	0.27\\
1.04	0.27\\
1.04	0.27\\
1.04	0.27\\
1.04	0.27\\
1.04	0.27\\
1.04	0.27\\
1.04	0.27\\
1.04	0.27\\
1.04	0.27\\
1.04	0.27\\
1.04	0.27\\
1.04	0.27\\
1.04	0.27\\
1.04	0.27\\
1.04	0.27\\
1.04	0.27\\
1.04	0.27\\
1.04	0.27\\
1.04	0.27\\
1.04	0.27\\
1.04	0.27\\
1.04	0.27\\
1.04	0.27\\
1.04	0.27\\
1.04	0.27\\
1.04	0.27\\
1.04	0.27\\
1.04	0.27\\
1.04	0.27\\
1.04	0.27\\
1.04	0.27\\
1.04	0.27\\
1.04	0.27\\
1.04	0.27\\
1.04	0.27\\
1.04	0.27\\
1.04	0.27\\
1.04	0.27\\
1.04	0.27\\
1.04	0.27\\
1.04	0.27\\
1.04	0.27\\
1.04	0.27\\
1.04	0.27\\
1.04	0.27\\
1.04	0.27\\
1.04	0.27\\
1.04	0.27\\
1.04	0.27\\
1.04	0.27\\
1.04	0.27\\
1.04	0.27\\
1.04	0.27\\
1.04	0.27\\
1.04	0.27\\
1.04	0.27\\
1.04	0.27\\
1.04	0.27\\
1.04	0.27\\
1.04	0.27\\
1.04	0.27\\
1.04	0.27\\
1.04	0.27\\
1.04	0.27\\
1.04	0.27\\
1.04	0.27\\
1.04	0.27\\
1.04	0.27\\
1.04	0.27\\
1.04	0.27\\
1.04	0.27\\
1.04	0.27\\
1.04	0.27\\
1.04	0.27\\
1.04	0.27\\
1.04	0.27\\
1.04	0.27\\
1.04	0.27\\
1.04	0.27\\
1.04	0.27\\
1.04	0.27\\
1.04	0.27\\
1.04	0.27\\
1.04	0.27\\
1.04	0.27\\
1.04	0.27\\
1.04	0.27\\
1.04	0.27\\
1.04	0.27\\
1.04	0.27\\
1.04	0.27\\
1.04	0.27\\
1.04	0.27\\
1.04	0.27\\
1.04	0.27\\
1.04	0.27\\
1.04	0.27\\
1.04	0.27\\
1.04	0.27\\
1.04	0.27\\
1.04	0.27\\
1.04	0.27\\
1.04	0.27\\
1.04	0.27\\
1.04	0.27\\
1.04	0.27\\
1.04	0.27\\
1.04	0.27\\
1.04	0.27\\
1.04	0.27\\
1.04	0.27\\
1.04	0.27\\
1.04	0.27\\
1.04	0.27\\
1.04	0.27\\
1.04	0.27\\
1.04	0.27\\
1.04	0.27\\
1.04	0.27\\
1.04	0.27\\
1.04	0.27\\
1.04	0.27\\
1.04	0.27\\
1.04	0.27\\
1.04	0.27\\
1.04	0.27\\
1.04	0.27\\
1.04	0.27\\
1.04	0.27\\
1.04	0.27\\
1.04	0.27\\
1.04	0.27\\
1.04	0.27\\
1.04	0.27\\
1.04	0.27\\
1.04	0.27\\
1.04	0.27\\
1.04	0.27\\
1.04	0.27\\
1.04	0.27\\
1.04	0.27\\
1.04	0.27\\
1.04	0.27\\
1.04	0.27\\
1.04	0.27\\
1.04	0.27\\
1.04	0.27\\
1.04	0.27\\
1.04	0.27\\
1.04	0.27\\
1.04	0.27\\
1.04	0.27\\
1.04	0.27\\
1.04	0.27\\
1.04	0.27\\
1.04	0.27\\
1.04	0.27\\
1.04	0.27\\
1.04	0.27\\
1.04	0.27\\
1.04	0.27\\
1.04	0.27\\
1.04	0.27\\
1.04	0.27\\
1.04	0.27\\
1.04	0.27\\
1.04	0.27\\
1.04	0.27\\
1.04	0.27\\
1.04	0.27\\
1.04	0.27\\
1.04	0.27\\
1.04	0.27\\
1.04	0.27\\
1.04	0.27\\
1.04	0.27\\
1.04	0.27\\
1.04	0.27\\
1.04	0.27\\
1.04	0.27\\
1.04	0.27\\
1.04	0.27\\
1.04	0.27\\
1.04	0.27\\
1.04	0.27\\
1.04	0.27\\
1.04	0.27\\
1.04	0.27\\
1.04	0.27\\
1.04	0.27\\
1.04	0.27\\
1.04	0.27\\
1.04	0.27\\
1.04	0.27\\
1.04	0.27\\
1.04	0.27\\
1.04	0.27\\
1.04	0.27\\
1.04	0.27\\
1.04	0.27\\
1.04	0.27\\
1.04	0.27\\
1.04	0.27\\
1.04	0.27\\
1.04	0.27\\
1.04	0.27\\
1.04	0.27\\
1.04	0.27\\
1.04	0.27\\
1.04	0.27\\
1.04	0.27\\
1.04	0.27\\
1.04	0.27\\
1.04	0.27\\
1.04	0.27\\
1.04	0.27\\
1.04	0.27\\
1.04	0.27\\
1.04	0.27\\
1.04	0.27\\
1.04	0.27\\
1.04	0.27\\
1.04	0.27\\
1.04	0.27\\
1.04	0.27\\
1.04	0.27\\
1.04	0.27\\
1.04	0.27\\
1.04	0.27\\
1.04	0.27\\
1.04	0.27\\
1.04	0.27\\
1.04	0.27\\
1.04	0.27\\
1.04	0.27\\
1.04	0.27\\
1.04	0.27\\
1.04	0.27\\
1.04	0.27\\
1.04	0.27\\
1.04	0.27\\
1.04	0.27\\
1.04	0.27\\
1.04	0.27\\
1.04	0.27\\
1.04	0.27\\
1.04	0.27\\
1.04	0.27\\
1.04	0.27\\
1.04	0.27\\
1.04	0.27\\
1.04	0.27\\
1.04	0.27\\
1.04	0.27\\
1.04	0.27\\
1.04	0.27\\
1.04	0.27\\
1.04	0.27\\
1.04	0.27\\
1.04	0.27\\
1.04	0.27\\
1.04	0.27\\
1.04	0.27\\
1.04	0.27\\
1.04	0.27\\
1.04	0.27\\
1.04	0.27\\
1.04	0.27\\
1.04	0.27\\
1.04	0.27\\
1.04	0.27\\
1.04	0.27\\
1.04	0.27\\
1.04	0.27\\
1.04	0.27\\
1.04	0.27\\
1.04	0.27\\
1.04	0.27\\
1.04	0.27\\
1.04	0.27\\
1.04	0.27\\
1.04	0.27\\
1.04	0.27\\
1.04	0.27\\
1.04	0.27\\
1.04	0.27\\
1.04	0.27\\
1.04	0.27\\
1.04	0.27\\
1.04	0.27\\
1.04	0.27\\
1.04	0.27\\
1.04	0.27\\
1.04	0.27\\
1.04	0.27\\
1.04	0.27\\
1.04	0.27\\
1.04	0.27\\
1.04	0.27\\
1.04	0.27\\
1.04	0.27\\
1.04	0.27\\
1.04	0.27\\
1.04	0.27\\
1.04	0.27\\
1.04	0.27\\
1.04	0.27\\
1.04	0.27\\
1.04	0.27\\
1.04	0.27\\
1.04	0.27\\
1.04	0.27\\
1.04	0.27\\
1.04	0.27\\
1.04	0.27\\
1.04	0.27\\
1.04	0.27\\
1.04	0.27\\
1.04	0.27\\
1.04	0.27\\
1.04	0.27\\
1.04	0.27\\
1.04	0.27\\
1.04	0.27\\
1.04	0.27\\
1.04	0.27\\
1.04	0.27\\
1.04	0.27\\
1.04	0.27\\
1.04	0.27\\
1.04	0.27\\
1.04	0.27\\
1.04	0.27\\
1.04	0.27\\
1.04	0.27\\
1.04	0.27\\
1.04	0.27\\
1.04	0.27\\
1.04	0.27\\
1.04	0.27\\
1.04	0.27\\
1.04	0.27\\
1.04	0.27\\
1.04	0.27\\
1.04	0.27\\
1.04	0.27\\
1.04	0.27\\
1.04	0.27\\
1.04	0.27\\
1.04	0.27\\
1.04	0.27\\
1.04	0.27\\
1.04	0.27\\
1.04	0.27\\
1.04	0.27\\
1.04	0.27\\
1.04	0.27\\
1.04	0.27\\
1.04	0.27\\
1.04	0.27\\
1.04	0.27\\
1.04	0.27\\
1.04	0.27\\
1.04	0.27\\
1.04	0.27\\
1.04	0.27\\
1.04	0.27\\
1.04	0.27\\
1.04	0.27\\
1.04	0.27\\
1.04	0.27\\
1.04	0.27\\
1.04	0.27\\
1.04	0.27\\
1.04	0.27\\
1.04	0.27\\
1.04	0.27\\
1.04	0.27\\
1.04	0.27\\
1.04	0.27\\
1.04	0.27\\
1.04	0.27\\
1.04	0.27\\
1.04	0.27\\
1.04	0.27\\
1.04	0.27\\
1.04	0.27\\
1.04	0.27\\
1.04	0.27\\
1.04	0.27\\
1.04	0.27\\
1.04	0.27\\
1.04	0.27\\
1.04	0.27\\
1.04	0.27\\
1.04	0.27\\
1.04	0.27\\
1.04	0.27\\
1.04	0.27\\
1.04	0.27\\
1.04	0.27\\
1.04	0.27\\
1.04	0.27\\
1.04	0.27\\
1.04	0.27\\
1.04	0.27\\
1.04	0.27\\
1.04	0.27\\
1.04	0.27\\
1.04	0.27\\
1.04	0.27\\
1.04	0.27\\
1.04	0.27\\
1.04	0.27\\
1.04	0.27\\
1.04	0.27\\
1.04	0.27\\
1.04	0.27\\
1.04	0.27\\
1.04	0.27\\
1.04	0.27\\
1.04	0.27\\
1.04	0.27\\
1.04	0.27\\
1.04	0.27\\
1.04	0.27\\
1.04	0.27\\
1.04	0.27\\
1.04	0.27\\
1.04	0.27\\
1.04	0.27\\
1.04	0.27\\
1.04	0.27\\
1.04	0.27\\
1.04	0.27\\
1.04	0.27\\
1.04	0.27\\
1.04	0.27\\
1.04	0.27\\
1.04	0.27\\
1.04	0.27\\
1.04	0.27\\
1.04	0.27\\
1.04	0.27\\
1.04	0.27\\
1.04	0.27\\
1.04	0.27\\
1.04	0.27\\
1.04	0.27\\
1.04	0.27\\
1.04	0.27\\
1.04	0.27\\
1.04	0.27\\
1.04	0.27\\
1.04	0.27\\
1.04	0.27\\
1.04	0.27\\
1.04	0.27\\
1.04	0.27\\
1.04	0.27\\
1.04	0.27\\
1.04	0.27\\
1.04	0.27\\
1.04	0.27\\
1.04	0.27\\
1.04	0.27\\
1.04	0.27\\
1.04	0.27\\
1.04	0.27\\
1.04	0.27\\
1.04	0.27\\
1.04	0.27\\
1.04	0.27\\
1.04	0.27\\
1.04	0.27\\
1.04	0.27\\
1.04	0.27\\
1.04	0.27\\
1.04	0.27\\
1.04	0.27\\
1.04	0.27\\
1.04	0.27\\
1.04	0.27\\
1.04	0.27\\
1.04	0.27\\
1.04	0.27\\
1.04	0.27\\
1.04	0.27\\
1.04	0.27\\
1.04	0.27\\
1.04	0.27\\
1.04	0.27\\
1.04	0.27\\
1.04	0.27\\
1.04	0.27\\
1.04	0.27\\
1.04	0.27\\
1.04	0.27\\
1.04	0.27\\
1.04	0.27\\
1.04	0.27\\
1.04	0.27\\
1.04	0.27\\
1.04	0.27\\
1.04	0.27\\
1.04	0.27\\
1.04	0.27\\
1.04	0.27\\
1.04	0.27\\
1.04	0.27\\
1.04	0.27\\
1.04	0.27\\
1.04	0.27\\
1.04	0.27\\
1.04	0.27\\
1.04	0.27\\
1.04	0.27\\
1.04	0.27\\
1.04	0.27\\
1.04	0.27\\
1.04	0.27\\
1.04	0.27\\
1.04	0.27\\
1.04	0.27\\
1.04	0.27\\
1.04	0.27\\
1.04	0.27\\
1.04	0.27\\
1.04	0.27\\
1.04	0.27\\
1.04	0.27\\
1.04	0.27\\
1.04	0.27\\
1.04	0.27\\
1.04	0.27\\
1.04	0.27\\
1.04	0.27\\
1.04	0.27\\
1.04	0.27\\
1.04	0.27\\
1.04	0.27\\
1.04	0.27\\
1.04	0.27\\
1.04	0.27\\
1.04	0.27\\
1.04	0.27\\
1.04	0.27\\
1.04	0.27\\
1.04	0.27\\
1.04	0.27\\
1.04	0.27\\
1.04	0.27\\
1.04	0.27\\
1.04	0.27\\
1.04	0.27\\
1.04	0.27\\
1.04	0.27\\
1.04	0.27\\
1.04	0.27\\
1.04	0.27\\
1.04	0.27\\
1.04	0.27\\
1.04	0.27\\
1.04	0.27\\
1.04	0.27\\
1.04	0.27\\
1.04	0.27\\
1.04	0.27\\
1.04	0.27\\
1.04	0.27\\
1.04	0.27\\
1.04	0.27\\
1.04	0.27\\
1.04	0.27\\
1.04	0.27\\
1.04	0.27\\
1.04	0.27\\
1.04	0.27\\
1.04	0.27\\
1.04	0.27\\
1.04	0.27\\
1.04	0.27\\
1.04	0.27\\
1.04	0.27\\
1.04	0.27\\
1.04	0.27\\
1.04	0.27\\
1.04	0.27\\
1.04	0.27\\
1.04	0.27\\
1.04	0.27\\
1.04	0.27\\
1.04	0.27\\
1.04	0.27\\
1.04	0.27\\
1.04	0.27\\
1.04	0.27\\
1.04	0.27\\
1.04	0.27\\
1.04	0.27\\
1.04	0.27\\
1.04	0.27\\
1.04	0.27\\
1.04	0.27\\
1.04	0.27\\
1.04	0.27\\
1.04	0.27\\
1.04	0.27\\
1.04	0.27\\
1.04	0.27\\
1.04	0.27\\
1.04	0.27\\
1.04	0.27\\
1.04	0.27\\
1.04	0.27\\
1.04	0.27\\
1.04	0.27\\
1.04	0.27\\
1.04	0.27\\
1.04	0.27\\
1.04	0.27\\
1.04	0.27\\
1.04	0.27\\
1.04	0.27\\
1.04	0.27\\
1.04	0.27\\
1.04	0.27\\
1.04	0.27\\
1.04	0.27\\
1.04	0.27\\
1.04	0.27\\
1.04	0.27\\
1.04	0.27\\
1.04	0.27\\
1.04	0.27\\
1.04	0.27\\
1.04	0.27\\
1.04	0.27\\
1.04	0.27\\
1.04	0.27\\
1.04	0.27\\
1.04	0.27\\
1.04	0.27\\
1.04	0.27\\
1.04	0.27\\
1.04	0.27\\
1.04	0.27\\
1.04	0.27\\
1.04	0.27\\
1.04	0.27\\
1.04	0.27\\
1.04	0.27\\
1.04	0.27\\
1.04	0.27\\
1.04	0.27\\
1.04	0.27\\
1.04	0.27\\
1.04	0.27\\
1.04	0.27\\
1.04	0.27\\
1.04	0.27\\
1.04	0.27\\
1.04	0.27\\
1.04	0.27\\
1.04	0.27\\
1.04	0.27\\
1.04	0.27\\
1.04	0.27\\
1.04	0.27\\
1.04	0.27\\
1.04	0.27\\
1.04	0.27\\
1.04	0.27\\
1.04	0.27\\
1.04	0.27\\
1.04	0.27\\
1.04	0.27\\
1.04	0.27\\
1.04	0.27\\
1.04	0.27\\
1.04	0.27\\
1.04	0.27\\
1.04	0.27\\
1.04	0.27\\
1.04	0.27\\
1.04	0.27\\
1.04	0.27\\
1.04	0.27\\
1.04	0.27\\
1.04	0.27\\
1.04	0.27\\
1.04	0.27\\
1.04	0.27\\
1.04	0.27\\
1.04	0.27\\
1.04	0.27\\
1.04	0.27\\
1.04	0.27\\
1.04	0.27\\
1.04	0.27\\
1.04	0.27\\
1.04	0.27\\
1.04	0.27\\
1.04	0.27\\
1.04	0.27\\
1.04	0.27\\
1.04	0.27\\
1.04	0.27\\
1.04	0.27\\
1.04	0.27\\
1.04	0.27\\
1.04	0.27\\
1.04	0.27\\
1.04	0.27\\
1.04	0.27\\
1.04	0.27\\
1.04	0.27\\
1.04	0.27\\
1.04	0.27\\
1.04	0.27\\
1.04	0.27\\
1.04	0.27\\
1.04	0.27\\
1.04	0.27\\
1.04	0.27\\
1.04	0.27\\
1.04	0.27\\
1.04	0.27\\
1.04	0.27\\
1.04	0.27\\
1.04	0.27\\
1.04	0.27\\
1.04	0.27\\
1.04	0.27\\
1.04	0.27\\
1.04	0.27\\
1.04	0.27\\
1.04	0.27\\
1.04	0.27\\
1.04	0.27\\
1.04	0.27\\
1.04	0.27\\
1.04	0.27\\
1.04	0.27\\
1.04	0.27\\
1.04	0.27\\
1.04	0.27\\
1.04	0.27\\
1.04	0.27\\
1.04	0.27\\
1.04	0.27\\
1.04	0.27\\
1.04	0.27\\
1.04	0.27\\
1.04	0.27\\
1.04	0.27\\
1.04	0.27\\
1.04	0.27\\
1.04	0.27\\
1.04	0.27\\
1.04	0.27\\
1.04	0.27\\
1.04	0.27\\
1.04	0.27\\
1.04	0.27\\
1.04	0.27\\
1.04	0.27\\
1.04	0.27\\
1.04	0.27\\
1.04	0.27\\
1.04	0.27\\
1.04	0.27\\
1.04	0.27\\
1.04	0.27\\
1.04	0.27\\
1.04	0.27\\
1.04	0.27\\
1.04	0.27\\
1.04	0.27\\
1.04	0.27\\
1.04	0.27\\
1.04	0.27\\
1.04	0.27\\
1.04	0.27\\
1.04	0.27\\
1.04	0.27\\
1.04	0.27\\
1.04	0.27\\
1.04	0.27\\
1.04	0.27\\
1.04	0.27\\
1.04	0.27\\
1.04	0.27\\
1.04	0.27\\
1.04	0.27\\
1.04	0.27\\
1.04	0.27\\
1.04	0.27\\
1.04	0.27\\
1.04	0.27\\
1.04	0.27\\
1.04	0.27\\
1.04	0.27\\
1.04	0.27\\
1.04	0.27\\
1.04	0.27\\
1.04	0.27\\
1.04	0.27\\
1.04	0.27\\
1.04	0.27\\
1.04	0.27\\
1.04	0.27\\
1.04	0.27\\
1.04	0.27\\
1.04	0.27\\
1.04	0.27\\
1.04	0.27\\
1.04	0.27\\
1.04	0.27\\
1.04	0.27\\
1.04	0.27\\
1.04	0.27\\
1.04	0.27\\
1.04	0.27\\
1.04	0.27\\
1.04	0.27\\
1.04	0.27\\
1.04	0.27\\
1.04	0.27\\
1.04	0.27\\
1.04	0.27\\
1.04	0.27\\
1.04	0.27\\
1.04	0.27\\
1.04	0.27\\
1.04	0.27\\
1.04	0.27\\
1.04	0.27\\
1.04	0.27\\
1.04	0.27\\
1.04	0.27\\
1.04	0.27\\
1.04	0.27\\
1.04	0.27\\
1.04	0.27\\
1.04	0.27\\
1.04	0.27\\
1.04	0.27\\
1.04	0.27\\
1.04	0.27\\
1.04	0.27\\
1.04	0.27\\
1.04	0.27\\
1.04	0.27\\
1.04	0.27\\
1.04	0.27\\
1.04	0.27\\
1.04	0.27\\
1.04	0.27\\
1.04	0.27\\
1.04	0.27\\
1.04	0.27\\
1.04	0.27\\
1.04	0.27\\
1.04	0.27\\
1.04	0.27\\
1.04	0.27\\
1.04	0.27\\
1.04	0.27\\
1.04	0.27\\
1.04	0.27\\
1.04	0.27\\
1.04	0.27\\
1.04	0.27\\
1.04	0.27\\
1.04	0.27\\
1.04	0.27\\
1.04	0.27\\
1.04	0.27\\
1.04	0.27\\
1.04	0.27\\
1.04	0.27\\
1.04	0.27\\
1.04	0.27\\
1.04	0.27\\
1.04	0.27\\
1.04	0.27\\
1.04	0.27\\
1.04	0.27\\
1.04	0.27\\
1.04	0.27\\
1.04	0.27\\
1.04	0.27\\
1.04	0.27\\
1.04	0.27\\
1.04	0.27\\
1.04	0.27\\
1.04	0.27\\
1.04	0.27\\
1.04	0.27\\
1.04	0.27\\
1.04	0.27\\
1.04	0.27\\
1.04	0.27\\
1.04	0.27\\
1.04	0.27\\
1.04	0.27\\
1.04	0.27\\
1.04	0.27\\
1.04	0.27\\
1.04	0.27\\
1.04	0.27\\
1.04	0.27\\
1.04	0.27\\
1.04	0.27\\
1.04	0.27\\
1.04	0.27\\
1.04	0.27\\
1.04	0.27\\
1.04	0.27\\
1.04	0.27\\
1.04	0.27\\
1.04	0.27\\
1.04	0.27\\
1.04	0.27\\
1.04	0.27\\
1.04	0.27\\
1.04	0.27\\
1.04	0.27\\
1.04	0.27\\
1.04	0.27\\
1.04	0.27\\
1.04	0.27\\
1.04	0.27\\
1.04	0.27\\
1.04	0.27\\
1.04	0.27\\
1.04	0.27\\
1.04	0.27\\
1.04	0.27\\
1.04	0.27\\
1.04	0.27\\
1.04	0.27\\
1.04	0.27\\
1.04	0.27\\
1.04	0.27\\
1.04	0.27\\
1.04	0.27\\
1.04	0.27\\
1.04	0.27\\
1.04	0.27\\
1.04	0.27\\
1.04	0.27\\
1.04	0.27\\
1.04	0.27\\
1.04	0.27\\
1.04	0.27\\
1.04	0.27\\
1.04	0.27\\
1.04	0.27\\
1.04	0.27\\
1.04	0.27\\
1.04	0.27\\
1.04	0.27\\
1.04	0.27\\
1.04	0.27\\
1.04	0.27\\
1.04	0.27\\
1.04	0.27\\
1.04	0.27\\
1.04	0.27\\
1.04	0.27\\
1.04	0.27\\
1.04	0.27\\
1.04	0.27\\
1.04	0.27\\
1.04	0.27\\
1.04	0.27\\
1.04	0.27\\
1.04	0.27\\
1.04	0.27\\
1.04	0.27\\
1.04	0.27\\
1.04	0.27\\
1.04	0.27\\
1.04	0.27\\
1.04	0.27\\
1.04	0.27\\
1.04	0.27\\
1.04	0.27\\
1.04	0.27\\
1.04	0.27\\
1.04	0.27\\
1.04	0.27\\
1.04	0.27\\
1.04	0.27\\
1.04	0.27\\
1.04	0.27\\
1.04	0.27\\
1.04	0.27\\
1.04	0.27\\
1.04	0.27\\
1.04	0.27\\
1.04	0.27\\
1.04	0.27\\
1.04	0.27\\
1.04	0.27\\
1.04	0.27\\
1.04	0.27\\
1.04	0.27\\
1.04	0.27\\
1.04	0.27\\
1.04	0.27\\
1.04	0.27\\
1.04	0.27\\
1.04	0.27\\
1.04	0.27\\
1.04	0.27\\
1.04	0.27\\
1.04	0.27\\
1.04	0.27\\
1.04	0.27\\
1.04	0.27\\
1.04	0.27\\
1.04	0.27\\
1.04	0.27\\
1.04	0.27\\
1.04	0.27\\
1.04	0.27\\
1.04	0.27\\
1.04	0.27\\
1.04	0.27\\
1.04	0.27\\
1.04	0.27\\
1.04	0.27\\
1.04	0.27\\
1.04	0.27\\
1.04	0.27\\
1.04	0.27\\
1.04	0.27\\
1.04	0.27\\
1.04	0.27\\
1.04	0.27\\
1.04	0.27\\
1.04	0.27\\
1.04	0.27\\
1.04	0.27\\
1.04	0.27\\
1.04	0.27\\
1.04	0.27\\
1.04	0.27\\
1.04	0.27\\
1.04	0.27\\
1.04	0.27\\
1.04	0.27\\
1.04	0.27\\
1.04	0.27\\
1.04	0.27\\
1.04	0.27\\
1.04	0.27\\
1.04	0.27\\
1.04	0.27\\
1.04	0.27\\
1.04	0.27\\
1.04	0.27\\
1.04	0.27\\
1.04	0.27\\
1.04	0.27\\
1.04	0.27\\
1.04	0.27\\
1.04	0.27\\
1.04	0.27\\
1.04	0.27\\
1.04	0.27\\
1.04	0.27\\
1.04	0.27\\
1.04	0.27\\
1.04	0.27\\
1.04	0.27\\
1.04	0.27\\
1.04	0.27\\
1.04	0.27\\
1.04	0.27\\
1.04	0.27\\
1.04	0.27\\
1.04	0.27\\
1.04	0.27\\
1.04	0.27\\
1.04	0.27\\
1.04	0.27\\
1.04	0.28\\
1.04	0.27\\
1.04	0.27\\
1.04	0.27\\
1.04	0.27\\
1.04	0.27\\
1.04	0.27\\
1.04	0.27\\
1.04	0.27\\
1.04	0.27\\
1.04	0.27\\
1.04	0.27\\
1.04	0.27\\
1.04	0.27\\
1.04	0.27\\
1.04	0.27\\
1.04	0.27\\
1.04	0.27\\
1.04	0.27\\
1.04	0.27\\
1.04	0.27\\
1.04	0.27\\
1.04	0.27\\
1.04	0.27\\
1.04	0.27\\
1.04	0.27\\
};
\addplot [color=mycolor1,solid,forget plot]
  table[row sep=crcr]{%
1.04	0.27\\
1.04	0.28\\
1.04	0.27\\
1.04	0.28\\
1.04	0.27\\
1.04	0.27\\
1.04	0.28\\
1.04	0.28\\
1.04	0.27\\
1.04	0.28\\
1.04	0.27\\
1.04	0.28\\
1.04	0.27\\
1.04	0.28\\
1.04	0.27\\
1.04	0.28\\
1.04	0.27\\
1.04	0.27\\
1.04	0.28\\
1.04	0.27\\
1.04	0.28\\
1.04	0.27\\
1.04	0.28\\
1.04	0.27\\
1.04	0.28\\
1.04	0.28\\
1.04	0.28\\
1.04	0.28\\
1.05	0.28\\
1.05	0.28\\
1.05	0.28\\
1.05	0.28\\
1.05	0.28\\
1.05	0.28\\
1.05	0.28\\
1.05	0.28\\
1.05	0.28\\
1.05	0.28\\
1.05	0.28\\
1.05	0.28\\
1.05	0.28\\
1.05	0.28\\
1.05	0.28\\
1.05	0.28\\
1.05	0.28\\
1.05	0.28\\
1.05	0.28\\
1.05	0.28\\
1.05	0.28\\
1.05	0.28\\
1.05	0.28\\
1.05	0.28\\
1.05	0.28\\
1.05	0.28\\
1.05	0.28\\
1.05	0.28\\
1.06	0.28\\
1.06	0.28\\
1.06	0.28\\
1.06	0.28\\
1.06	0.28\\
1.06	0.28\\
1.06	0.28\\
1.06	0.28\\
1.06	0.28\\
1.06	0.28\\
1.06	0.28\\
1.06	0.28\\
1.06	0.28\\
1.06	0.28\\
1.06	0.28\\
1.06	0.28\\
1.06	0.28\\
1.06	0.28\\
1.06	0.28\\
1.06	0.28\\
1.06	0.28\\
1.06	0.28\\
1.06	0.28\\
1.06	0.28\\
1.06	0.28\\
1.06	0.28\\
1.06	0.28\\
1.06	0.28\\
1.06	0.28\\
1.06	0.28\\
1.06	0.28\\
1.06	0.28\\
1.06	0.28\\
1.06	0.28\\
1.06	0.28\\
1.06	0.28\\
1.06	0.28\\
1.06	0.28\\
1.06	0.28\\
1.06	0.28\\
1.06	0.28\\
1.06	0.28\\
1.06	0.28\\
1.06	0.28\\
1.06	0.28\\
1.06	0.28\\
1.06	0.28\\
1.06	0.28\\
1.06	0.28\\
1.06	0.28\\
1.06	0.28\\
1.06	0.28\\
1.06	0.28\\
1.06	0.28\\
1.06	0.28\\
1.06	0.28\\
1.06	0.28\\
1.06	0.28\\
1.06	0.28\\
1.06	0.28\\
1.06	0.29\\
1.06	0.28\\
1.06	0.29\\
1.06	0.29\\
1.06	0.29\\
1.06	0.29\\
1.07	0.29\\
1.07	0.29\\
1.07	0.29\\
1.07	0.29\\
1.07	0.29\\
1.07	0.29\\
1.07	0.29\\
1.07	0.29\\
1.07	0.29\\
1.07	0.29\\
1.07	0.29\\
1.07	0.29\\
1.07	0.29\\
1.07	0.29\\
1.07	0.29\\
1.07	0.29\\
1.07	0.29\\
1.07	0.29\\
1.07	0.29\\
1.07	0.29\\
1.07	0.29\\
1.07	0.29\\
1.07	0.29\\
1.07	0.29\\
1.07	0.29\\
1.07	0.29\\
1.07	0.29\\
1.07	0.29\\
1.07	0.29\\
1.07	0.29\\
1.07	0.29\\
1.07	0.29\\
1.07	0.29\\
1.07	0.29\\
1.07	0.29\\
1.07	0.29\\
1.07	0.29\\
1.07	0.29\\
1.07	0.29\\
1.07	0.29\\
1.07	0.29\\
1.07	0.29\\
1.07	0.29\\
1.07	0.29\\
1.07	0.29\\
1.07	0.29\\
1.07	0.29\\
1.07	0.29\\
1.07	0.29\\
1.07	0.29\\
1.07	0.29\\
1.07	0.29\\
1.07	0.29\\
1.07	0.29\\
1.06	0.29\\
1.07	0.29\\
1.07	0.29\\
1.07	0.29\\
1.07	0.29\\
1.07	0.29\\
1.07	0.29\\
1.07	0.29\\
1.07	0.29\\
1.07	0.29\\
1.07	0.29\\
1.07	0.29\\
1.07	0.29\\
1.07	0.29\\
1.07	0.29\\
1.07	0.29\\
1.07	0.29\\
1.07	0.29\\
1.07	0.29\\
1.07	0.29\\
1.07	0.29\\
1.07	0.29\\
1.07	0.29\\
1.07	0.29\\
1.07	0.29\\
1.07	0.29\\
1.07	0.29\\
1.07	0.29\\
1.07	0.29\\
1.07	0.29\\
1.07	0.29\\
1.07	0.29\\
1.07	0.29\\
1.07	0.29\\
1.07	0.29\\
1.07	0.29\\
1.07	0.29\\
1.07	0.29\\
1.07	0.29\\
1.07	0.29\\
1.07	0.29\\
1.07	0.29\\
1.07	0.29\\
1.07	0.29\\
1.07	0.29\\
1.07	0.29\\
1.06	0.29\\
1.06	0.29\\
1.06	0.29\\
1.06	0.29\\
1.06	0.29\\
1.06	0.29\\
1.06	0.28\\
1.06	0.28\\
1.06	0.28\\
1.06	0.28\\
1.06	0.28\\
1.06	0.28\\
1.06	0.28\\
1.06	0.28\\
1.06	0.28\\
1.06	0.28\\
1.06	0.28\\
1.06	0.28\\
1.06	0.28\\
1.06	0.28\\
1.06	0.28\\
1.06	0.28\\
1.06	0.28\\
1.06	0.28\\
1.05	0.28\\
1.05	0.28\\
1.05	0.28\\
1.05	0.28\\
1.05	0.28\\
1.05	0.28\\
1.05	0.28\\
1.05	0.28\\
1.05	0.28\\
1.05	0.28\\
1.05	0.28\\
1.05	0.28\\
1.05	0.28\\
1.05	0.28\\
1.05	0.28\\
1.05	0.28\\
1.05	0.28\\
1.05	0.28\\
1.05	0.28\\
1.05	0.27\\
1.05	0.27\\
1.05	0.27\\
1.05	0.28\\
1.05	0.27\\
1.05	0.27\\
1.05	0.27\\
1.05	0.28\\
1.05	0.27\\
1.05	0.27\\
1.05	0.27\\
1.05	0.27\\
1.05	0.27\\
1.05	0.28\\
1.05	0.27\\
1.05	0.27\\
1.05	0.27\\
1.05	0.27\\
1.05	0.27\\
1.05	0.27\\
1.05	0.27\\
1.05	0.27\\
1.05	0.27\\
1.05	0.27\\
1.05	0.27\\
1.05	0.27\\
1.05	0.27\\
1.05	0.27\\
1.05	0.27\\
1.05	0.27\\
1.05	0.27\\
1.05	0.27\\
1.05	0.27\\
1.05	0.27\\
1.05	0.27\\
1.05	0.27\\
1.05	0.27\\
1.05	0.27\\
1.05	0.27\\
1.05	0.27\\
1.05	0.27\\
1.04	0.27\\
1.04	0.27\\
1.04	0.27\\
1.04	0.27\\
1.04	0.27\\
1.04	0.27\\
1.04	0.27\\
1.04	0.27\\
1.04	0.27\\
1.04	0.27\\
1.04	0.27\\
1.04	0.27\\
1.04	0.27\\
1.04	0.27\\
1.04	0.27\\
1.04	0.27\\
1.04	0.27\\
1.04	0.27\\
1.04	0.27\\
1.04	0.27\\
1.04	0.27\\
1.04	0.27\\
1.04	0.27\\
1.04	0.27\\
1.04	0.27\\
1.04	0.27\\
1.04	0.27\\
1.04	0.27\\
1.04	0.27\\
1.04	0.27\\
1.04	0.27\\
1.04	0.27\\
1.04	0.27\\
1.04	0.27\\
1.04	0.27\\
1.05	0.27\\
1.04	0.27\\
1.04	0.27\\
1.05	0.27\\
1.05	0.27\\
1.05	0.27\\
1.05	0.27\\
1.05	0.27\\
1.05	0.27\\
1.05	0.27\\
1.05	0.27\\
1.05	0.27\\
1.05	0.27\\
1.05	0.27\\
1.05	0.27\\
1.05	0.27\\
1.05	0.27\\
1.05	0.27\\
1.05	0.27\\
1.05	0.27\\
1.05	0.27\\
1.05	0.27\\
1.05	0.27\\
1.05	0.27\\
1.05	0.27\\
1.05	0.27\\
1.05	0.27\\
1.05	0.27\\
1.05	0.28\\
1.05	0.28\\
1.06	0.28\\
1.06	0.28\\
1.06	0.29\\
1.07	0.29\\
1.07	0.3\\
1.08	0.3\\
1.08	0.31\\
1.09	0.31\\
1.09	0.32\\
1.1	0.32\\
1.1	0.32\\
1.11	0.33\\
1.12	0.34\\
1.12	0.34\\
1.13	0.35\\
1.14	0.36\\
1.14	0.36\\
1.15	0.37\\
1.16	0.38\\
1.16	0.38\\
1.17	0.39\\
1.18	0.39\\
1.18	0.4\\
1.19	0.4\\
1.2	0.41\\
1.2	0.42\\
1.21	0.42\\
1.22	0.43\\
1.22	0.43\\
1.23	0.44\\
1.23	0.45\\
1.24	0.45\\
1.24	0.46\\
1.25	0.46\\
1.25	0.46\\
1.26	0.47\\
1.26	0.47\\
1.27	0.47\\
1.27	0.48\\
1.27	0.48\\
1.28	0.49\\
1.28	0.49\\
1.28	0.49\\
1.29	0.5\\
1.29	0.5\\
1.3	0.5\\
1.3	0.5\\
1.3	0.51\\
1.3	0.51\\
1.31	0.51\\
1.31	0.52\\
1.32	0.52\\
1.32	0.52\\
1.32	0.53\\
1.33	0.53\\
1.33	0.53\\
1.33	0.54\\
1.34	0.54\\
1.34	0.55\\
1.35	0.55\\
1.35	0.55\\
1.35	0.56\\
1.36	0.56\\
1.36	0.56\\
1.37	0.57\\
1.37	0.57\\
1.37	0.58\\
1.38	0.58\\
1.38	0.58\\
1.39	0.59\\
1.39	0.59\\
1.4	0.59\\
1.4	0.6\\
1.4	0.6\\
1.41	0.61\\
1.41	0.61\\
1.42	0.61\\
1.42	0.62\\
1.43	0.62\\
1.43	0.63\\
1.44	0.63\\
1.44	0.64\\
1.45	0.64\\
1.45	0.64\\
1.45	0.65\\
1.46	0.65\\
1.46	0.65\\
1.47	0.66\\
1.47	0.66\\
1.47	0.66\\
1.48	0.67\\
1.48	0.67\\
1.49	0.68\\
1.49	0.68\\
1.5	0.68\\
1.5	0.69\\
1.5	0.69\\
1.51	0.69\\
1.51	0.7\\
1.51	0.7\\
1.52	0.7\\
1.52	0.71\\
1.53	0.71\\
1.53	0.72\\
1.53	0.72\\
1.54	0.72\\
1.54	0.73\\
1.55	0.73\\
1.55	0.73\\
1.55	0.74\\
1.56	0.74\\
1.56	0.74\\
1.57	0.75\\
1.57	0.75\\
1.57	0.75\\
1.58	0.76\\
1.58	0.76\\
1.59	0.76\\
1.59	0.77\\
1.59	0.77\\
1.6	0.78\\
1.6	0.78\\
1.6	0.78\\
1.61	0.78\\
1.61	0.79\\
1.62	0.79\\
1.62	0.8\\
1.62	0.8\\
1.63	0.8\\
1.63	0.81\\
1.63	0.81\\
1.64	0.81\\
1.64	0.81\\
1.64	0.82\\
1.65	0.82\\
1.65	0.82\\
1.65	0.82\\
1.66	0.83\\
1.66	0.83\\
1.66	0.83\\
1.66	0.84\\
1.67	0.84\\
1.67	0.84\\
1.67	0.84\\
1.68	0.85\\
1.68	0.85\\
1.68	0.85\\
1.68	0.85\\
1.69	0.86\\
1.69	0.86\\
1.69	0.86\\
1.7	0.86\\
1.7	0.87\\
1.7	0.87\\
1.7	0.87\\
1.71	0.87\\
1.71	0.88\\
1.71	0.88\\
1.72	0.88\\
1.72	0.88\\
1.72	0.89\\
1.73	0.89\\
1.73	0.89\\
1.73	0.9\\
1.74	0.9\\
1.74	0.9\\
1.74	0.91\\
1.75	0.91\\
1.75	0.91\\
1.76	0.92\\
1.76	0.92\\
1.76	0.92\\
1.77	0.93\\
1.77	0.93\\
1.77	0.93\\
1.78	0.94\\
1.78	0.94\\
1.78	0.94\\
1.79	0.95\\
1.79	0.95\\
1.8	0.95\\
1.8	0.96\\
1.8	0.96\\
1.81	0.96\\
1.81	0.97\\
1.81	0.97\\
1.82	0.97\\
1.82	0.98\\
1.82	0.98\\
1.83	0.98\\
1.83	0.98\\
1.83	0.99\\
1.83	0.99\\
1.84	0.99\\
1.84	0.99\\
1.84	1\\
1.84	1\\
1.85	1\\
1.85	1\\
1.85	1\\
1.85	1.01\\
1.85	1.01\\
1.86	1.01\\
1.86	1.01\\
1.86	1.01\\
1.86	1.01\\
1.86	1.02\\
1.87	1.02\\
1.87	1.02\\
1.87	1.02\\
1.87	1.02\\
1.87	1.02\\
1.88	1.03\\
1.88	1.03\\
1.88	1.03\\
1.88	1.03\\
1.88	1.03\\
1.89	1.04\\
1.89	1.04\\
1.89	1.04\\
1.89	1.04\\
1.9	1.05\\
1.9	1.05\\
1.9	1.05\\
1.9	1.05\\
1.91	1.06\\
1.91	1.06\\
1.91	1.06\\
1.92	1.06\\
1.92	1.07\\
1.92	1.07\\
1.93	1.07\\
1.93	1.08\\
1.93	1.08\\
1.93	1.08\\
1.94	1.08\\
1.94	1.09\\
1.94	1.09\\
1.95	1.09\\
1.95	1.09\\
1.95	1.1\\
1.95	1.1\\
1.96	1.1\\
1.96	1.1\\
1.96	1.11\\
1.97	1.11\\
1.97	1.11\\
1.97	1.11\\
1.97	1.12\\
1.98	1.12\\
1.98	1.12\\
1.98	1.12\\
1.98	1.12\\
1.99	1.13\\
1.99	1.13\\
1.99	1.13\\
1.99	1.13\\
1.99	1.13\\
2	1.13\\
2	1.14\\
2	1.14\\
2	1.14\\
2	1.14\\
2	1.14\\
2	1.14\\
2.01	1.14\\
2.01	1.15\\
2.01	1.15\\
2.01	1.15\\
2.01	1.15\\
2.01	1.15\\
2.02	1.15\\
2.02	1.15\\
2.02	1.16\\
2.02	1.16\\
2.02	1.16\\
2.02	1.16\\
2.02	1.16\\
2.03	1.16\\
2.03	1.16\\
2.03	1.17\\
2.03	1.17\\
2.03	1.17\\
2.04	1.17\\
2.04	1.17\\
2.04	1.18\\
2.04	1.18\\
2.05	1.18\\
2.05	1.18\\
2.05	1.19\\
2.05	1.19\\
2.06	1.19\\
2.06	1.19\\
2.06	1.19\\
2.06	1.2\\
2.07	1.2\\
2.07	1.2\\
2.07	1.2\\
2.07	1.21\\
2.08	1.21\\
2.08	1.21\\
2.08	1.21\\
2.08	1.21\\
2.08	1.21\\
2.09	1.22\\
2.09	1.22\\
2.09	1.22\\
2.09	1.22\\
2.1	1.22\\
2.1	1.22\\
2.1	1.23\\
2.1	1.23\\
2.1	1.23\\
2.1	1.23\\
2.11	1.23\\
2.11	1.23\\
2.11	1.24\\
2.11	1.24\\
2.11	1.24\\
2.11	1.24\\
2.11	1.24\\
2.12	1.24\\
2.12	1.24\\
2.12	1.24\\
2.12	1.24\\
2.12	1.24\\
2.12	1.25\\
2.12	1.25\\
2.12	1.25\\
2.12	1.25\\
2.13	1.25\\
2.13	1.25\\
2.13	1.25\\
2.13	1.25\\
2.13	1.25\\
2.13	1.26\\
2.13	1.26\\
2.13	1.26\\
2.14	1.26\\
2.14	1.26\\
2.14	1.26\\
2.14	1.26\\
2.14	1.27\\
2.15	1.27\\
2.15	1.27\\
2.15	1.27\\
2.15	1.27\\
2.15	1.27\\
2.15	1.28\\
2.16	1.28\\
2.16	1.28\\
2.16	1.28\\
2.16	1.28\\
2.17	1.29\\
2.17	1.29\\
2.17	1.29\\
2.17	1.29\\
2.17	1.29\\
2.18	1.3\\
2.18	1.3\\
2.18	1.3\\
2.18	1.3\\
2.18	1.3\\
2.18	1.3\\
2.19	1.3\\
2.19	1.31\\
2.19	1.31\\
2.19	1.31\\
2.19	1.31\\
2.19	1.31\\
2.19	1.31\\
2.2	1.32\\
2.2	1.32\\
2.2	1.32\\
2.2	1.32\\
2.2	1.32\\
2.2	1.32\\
2.2	1.32\\
2.2	1.32\\
2.2	1.32\\
2.21	1.33\\
2.21	1.33\\
2.21	1.33\\
2.21	1.33\\
2.21	1.33\\
2.21	1.33\\
2.21	1.33\\
2.21	1.33\\
2.21	1.33\\
2.21	1.33\\
2.22	1.33\\
2.22	1.33\\
2.22	1.34\\
2.22	1.34\\
2.22	1.34\\
2.22	1.34\\
2.22	1.34\\
2.22	1.34\\
2.22	1.34\\
2.22	1.34\\
2.22	1.34\\
2.22	1.34\\
2.23	1.34\\
2.23	1.34\\
2.23	1.35\\
2.23	1.35\\
2.23	1.35\\
2.23	1.35\\
2.24	1.35\\
2.24	1.35\\
2.24	1.35\\
2.24	1.35\\
2.24	1.36\\
2.24	1.36\\
2.24	1.36\\
2.24	1.36\\
2.25	1.36\\
2.25	1.36\\
2.25	1.36\\
2.25	1.37\\
2.25	1.37\\
2.25	1.37\\
2.26	1.37\\
2.26	1.37\\
2.26	1.37\\
2.26	1.37\\
2.26	1.37\\
2.26	1.38\\
2.27	1.38\\
2.27	1.38\\
2.27	1.38\\
2.27	1.38\\
2.27	1.38\\
2.27	1.38\\
2.27	1.38\\
2.27	1.38\\
2.27	1.39\\
2.28	1.39\\
2.28	1.39\\
2.28	1.39\\
2.28	1.39\\
2.28	1.39\\
2.28	1.39\\
2.28	1.39\\
2.28	1.39\\
2.28	1.39\\
2.28	1.39\\
2.28	1.39\\
2.29	1.4\\
2.29	1.4\\
2.29	1.4\\
2.29	1.4\\
2.29	1.4\\
2.29	1.4\\
2.29	1.4\\
2.29	1.4\\
2.29	1.4\\
2.29	1.4\\
2.29	1.4\\
2.29	1.4\\
2.29	1.4\\
2.3	1.4\\
2.3	1.41\\
2.3	1.41\\
2.3	1.41\\
2.3	1.41\\
2.3	1.41\\
2.3	1.41\\
2.3	1.41\\
2.3	1.41\\
2.31	1.42\\
2.31	1.42\\
2.31	1.42\\
2.31	1.42\\
2.31	1.42\\
2.31	1.42\\
2.31	1.42\\
2.31	1.42\\
2.31	1.42\\
2.32	1.42\\
2.32	1.42\\
2.32	1.43\\
2.32	1.43\\
2.32	1.43\\
2.32	1.43\\
2.32	1.43\\
2.32	1.43\\
2.32	1.43\\
2.33	1.43\\
2.33	1.43\\
2.33	1.43\\
2.33	1.44\\
2.33	1.44\\
2.33	1.44\\
2.33	1.44\\
2.33	1.44\\
2.33	1.44\\
2.33	1.44\\
2.34	1.44\\
2.34	1.44\\
2.34	1.44\\
2.34	1.44\\
2.34	1.44\\
2.34	1.45\\
2.34	1.45\\
2.34	1.45\\
2.34	1.45\\
2.34	1.45\\
2.34	1.45\\
2.34	1.45\\
2.35	1.45\\
2.35	1.45\\
2.35	1.45\\
2.35	1.45\\
2.35	1.45\\
2.35	1.45\\
2.35	1.45\\
2.35	1.45\\
2.35	1.45\\
2.35	1.45\\
2.35	1.46\\
2.35	1.46\\
2.35	1.46\\
2.35	1.46\\
2.35	1.46\\
2.35	1.46\\
2.35	1.46\\
2.36	1.46\\
2.36	1.46\\
2.36	1.46\\
2.36	1.46\\
2.36	1.46\\
2.36	1.46\\
2.36	1.46\\
2.36	1.46\\
2.36	1.47\\
2.36	1.47\\
2.36	1.47\\
2.37	1.47\\
2.37	1.47\\
2.37	1.47\\
2.37	1.47\\
2.37	1.47\\
2.37	1.47\\
2.37	1.47\\
2.37	1.48\\
2.38	1.48\\
2.38	1.48\\
2.38	1.48\\
2.38	1.48\\
2.38	1.48\\
2.38	1.48\\
2.38	1.48\\
2.38	1.48\\
2.38	1.48\\
2.38	1.48\\
2.38	1.49\\
2.39	1.49\\
2.39	1.49\\
2.39	1.49\\
2.39	1.49\\
2.39	1.49\\
2.39	1.49\\
2.39	1.49\\
2.39	1.49\\
2.39	1.49\\
2.39	1.49\\
2.39	1.49\\
2.39	1.49\\
2.39	1.49\\
2.39	1.49\\
2.39	1.49\\
2.39	1.49\\
2.39	1.49\\
2.39	1.49\\
2.39	1.49\\
2.39	1.49\\
2.39	1.49\\
2.39	1.49\\
2.39	1.49\\
2.39	1.49\\
2.39	1.49\\
2.39	1.49\\
2.39	1.49\\
2.39	1.49\\
2.4	1.49\\
2.4	1.5\\
2.4	1.5\\
2.4	1.5\\
2.4	1.5\\
2.4	1.5\\
2.4	1.5\\
2.4	1.5\\
2.4	1.5\\
2.4	1.5\\
2.4	1.5\\
2.41	1.5\\
2.41	1.5\\
2.41	1.51\\
2.41	1.51\\
2.41	1.51\\
2.41	1.51\\
2.41	1.51\\
2.41	1.51\\
2.41	1.51\\
2.41	1.51\\
2.41	1.51\\
2.42	1.51\\
2.42	1.51\\
2.42	1.51\\
2.42	1.51\\
2.42	1.51\\
2.42	1.52\\
2.42	1.52\\
2.42	1.52\\
2.42	1.52\\
2.42	1.52\\
2.42	1.52\\
2.42	1.52\\
2.42	1.52\\
2.42	1.52\\
2.42	1.52\\
2.42	1.52\\
2.42	1.52\\
2.42	1.52\\
2.42	1.52\\
2.42	1.52\\
2.42	1.52\\
2.42	1.52\\
2.43	1.52\\
2.43	1.52\\
2.43	1.52\\
2.43	1.52\\
2.43	1.52\\
2.43	1.52\\
2.43	1.52\\
2.43	1.52\\
2.43	1.52\\
2.43	1.52\\
2.43	1.52\\
2.43	1.52\\
2.43	1.52\\
2.43	1.52\\
2.43	1.52\\
2.43	1.52\\
2.43	1.52\\
2.43	1.52\\
2.43	1.52\\
2.43	1.53\\
2.43	1.53\\
2.43	1.53\\
2.43	1.53\\
2.43	1.53\\
2.43	1.53\\
2.43	1.53\\
2.43	1.53\\
2.44	1.53\\
2.44	1.53\\
2.44	1.53\\
2.44	1.53\\
2.44	1.53\\
2.44	1.54\\
2.44	1.54\\
2.44	1.54\\
2.44	1.54\\
2.45	1.54\\
2.45	1.54\\
2.45	1.54\\
2.45	1.54\\
2.45	1.54\\
2.45	1.54\\
2.45	1.54\\
2.45	1.54\\
2.45	1.55\\
2.45	1.55\\
2.45	1.55\\
2.45	1.55\\
2.45	1.55\\
2.45	1.55\\
2.45	1.55\\
2.45	1.55\\
2.46	1.55\\
2.46	1.55\\
2.46	1.55\\
2.46	1.55\\
2.46	1.55\\
2.46	1.55\\
2.46	1.55\\
2.46	1.55\\
2.46	1.55\\
2.46	1.55\\
2.46	1.55\\
2.46	1.55\\
2.46	1.55\\
2.46	1.55\\
2.46	1.55\\
2.46	1.55\\
2.46	1.55\\
2.46	1.55\\
2.46	1.55\\
2.46	1.55\\
2.46	1.55\\
2.46	1.55\\
2.46	1.55\\
2.46	1.55\\
2.46	1.55\\
2.46	1.55\\
2.46	1.55\\
2.46	1.55\\
2.46	1.55\\
2.46	1.55\\
2.46	1.55\\
2.46	1.55\\
2.46	1.55\\
2.46	1.55\\
2.46	1.55\\
2.46	1.55\\
2.46	1.55\\
2.46	1.55\\
2.46	1.55\\
2.46	1.55\\
2.46	1.55\\
2.46	1.55\\
2.46	1.55\\
2.46	1.55\\
2.46	1.55\\
2.46	1.55\\
2.46	1.55\\
2.46	1.55\\
2.46	1.55\\
2.46	1.56\\
2.46	1.56\\
2.47	1.56\\
2.47	1.56\\
2.47	1.56\\
2.47	1.56\\
2.47	1.56\\
2.47	1.56\\
2.47	1.56\\
2.47	1.56\\
2.47	1.56\\
2.47	1.56\\
2.47	1.56\\
2.47	1.56\\
2.47	1.56\\
2.47	1.56\\
2.47	1.56\\
2.47	1.56\\
2.47	1.56\\
2.47	1.56\\
2.48	1.56\\
2.48	1.56\\
2.48	1.56\\
2.48	1.57\\
2.48	1.57\\
2.48	1.57\\
2.48	1.57\\
2.48	1.57\\
2.48	1.57\\
2.48	1.57\\
2.48	1.57\\
2.48	1.57\\
2.48	1.57\\
2.48	1.57\\
2.48	1.57\\
2.48	1.57\\
2.48	1.57\\
2.48	1.57\\
2.48	1.57\\
2.48	1.57\\
2.48	1.57\\
2.48	1.57\\
2.48	1.57\\
2.48	1.57\\
2.48	1.57\\
2.48	1.57\\
2.48	1.57\\
2.48	1.57\\
2.48	1.57\\
2.48	1.57\\
2.48	1.57\\
2.48	1.57\\
2.48	1.57\\
2.48	1.57\\
2.48	1.57\\
2.48	1.57\\
2.48	1.57\\
2.48	1.57\\
2.48	1.57\\
2.48	1.57\\
2.49	1.57\\
2.49	1.57\\
2.49	1.57\\
2.49	1.58\\
2.49	1.58\\
2.49	1.58\\
2.49	1.58\\
2.49	1.58\\
2.49	1.58\\
2.49	1.58\\
2.49	1.58\\
2.49	1.58\\
2.49	1.58\\
2.49	1.58\\
2.49	1.58\\
2.49	1.58\\
2.5	1.58\\
2.5	1.58\\
2.5	1.58\\
2.5	1.58\\
2.5	1.58\\
2.5	1.58\\
2.5	1.58\\
2.5	1.58\\
2.5	1.58\\
2.5	1.58\\
2.5	1.58\\
2.5	1.58\\
2.5	1.58\\
2.5	1.58\\
2.5	1.58\\
2.5	1.58\\
2.5	1.58\\
2.5	1.59\\
2.5	1.59\\
2.5	1.59\\
2.5	1.59\\
2.5	1.59\\
2.5	1.59\\
2.5	1.59\\
2.5	1.59\\
2.5	1.59\\
2.5	1.59\\
2.5	1.59\\
2.5	1.59\\
2.5	1.59\\
2.5	1.59\\
2.5	1.59\\
2.5	1.59\\
2.5	1.59\\
2.5	1.59\\
2.5	1.59\\
2.5	1.59\\
2.5	1.59\\
2.5	1.59\\
2.5	1.59\\
2.5	1.59\\
2.5	1.59\\
2.5	1.59\\
2.5	1.59\\
2.5	1.59\\
2.5	1.59\\
2.5	1.59\\
2.5	1.59\\
2.51	1.59\\
2.51	1.59\\
2.51	1.59\\
2.51	1.59\\
2.51	1.59\\
2.51	1.59\\
2.51	1.59\\
2.51	1.59\\
2.51	1.59\\
2.51	1.59\\
2.51	1.59\\
2.51	1.59\\
2.51	1.59\\
2.51	1.59\\
2.51	1.59\\
2.51	1.6\\
2.51	1.6\\
2.51	1.6\\
2.51	1.6\\
2.51	1.6\\
2.51	1.6\\
2.51	1.6\\
2.51	1.6\\
2.51	1.6\\
2.51	1.6\\
2.51	1.6\\
2.51	1.6\\
2.51	1.6\\
2.51	1.6\\
2.51	1.6\\
2.51	1.6\\
2.51	1.6\\
2.51	1.6\\
2.51	1.6\\
2.51	1.6\\
2.51	1.6\\
2.51	1.6\\
2.51	1.6\\
2.51	1.6\\
2.51	1.6\\
2.51	1.6\\
2.51	1.6\\
2.51	1.6\\
2.51	1.6\\
2.52	1.6\\
2.52	1.6\\
2.52	1.6\\
2.52	1.6\\
2.52	1.6\\
2.52	1.6\\
2.52	1.6\\
2.52	1.6\\
2.52	1.6\\
2.52	1.6\\
2.52	1.6\\
2.52	1.6\\
2.52	1.6\\
2.52	1.6\\
2.52	1.6\\
2.52	1.6\\
2.52	1.6\\
2.52	1.6\\
2.52	1.6\\
2.52	1.6\\
2.52	1.6\\
2.52	1.6\\
2.52	1.6\\
2.52	1.6\\
2.52	1.6\\
2.52	1.6\\
2.52	1.6\\
2.52	1.6\\
2.52	1.6\\
2.52	1.6\\
2.52	1.6\\
2.52	1.6\\
2.52	1.6\\
2.52	1.6\\
2.52	1.6\\
2.52	1.6\\
2.52	1.6\\
2.52	1.6\\
2.52	1.6\\
2.52	1.6\\
2.52	1.6\\
2.52	1.6\\
2.52	1.6\\
2.52	1.6\\
2.52	1.6\\
2.52	1.6\\
2.52	1.6\\
2.52	1.6\\
2.52	1.6\\
2.52	1.6\\
2.52	1.61\\
2.52	1.61\\
2.52	1.61\\
2.52	1.61\\
2.53	1.61\\
2.53	1.61\\
2.53	1.61\\
2.53	1.61\\
2.53	1.61\\
2.53	1.61\\
2.53	1.61\\
2.53	1.61\\
2.53	1.61\\
2.53	1.61\\
2.53	1.61\\
2.53	1.61\\
2.53	1.61\\
2.53	1.61\\
2.53	1.61\\
2.53	1.61\\
2.53	1.61\\
2.53	1.61\\
2.53	1.61\\
2.53	1.61\\
2.53	1.61\\
2.53	1.61\\
2.53	1.61\\
2.53	1.61\\
2.53	1.61\\
2.53	1.61\\
2.53	1.61\\
2.53	1.61\\
2.53	1.61\\
2.53	1.61\\
2.53	1.61\\
2.53	1.61\\
2.53	1.61\\
2.53	1.61\\
2.53	1.61\\
2.53	1.61\\
2.53	1.61\\
2.53	1.61\\
2.53	1.61\\
2.53	1.61\\
2.53	1.61\\
2.53	1.61\\
2.53	1.61\\
2.53	1.61\\
2.53	1.61\\
2.53	1.61\\
2.53	1.61\\
2.53	1.61\\
2.53	1.61\\
2.53	1.61\\
2.53	1.61\\
2.53	1.61\\
2.53	1.61\\
2.53	1.61\\
2.53	1.61\\
2.53	1.61\\
2.53	1.61\\
2.53	1.61\\
2.53	1.61\\
2.53	1.61\\
2.53	1.61\\
2.53	1.61\\
2.53	1.61\\
2.53	1.61\\
2.53	1.61\\
2.53	1.61\\
2.53	1.61\\
2.53	1.61\\
2.53	1.61\\
2.53	1.61\\
2.53	1.61\\
2.53	1.61\\
2.53	1.61\\
2.53	1.61\\
2.53	1.61\\
2.53	1.61\\
2.53	1.61\\
2.53	1.61\\
2.53	1.61\\
2.53	1.61\\
2.53	1.61\\
2.53	1.61\\
2.53	1.61\\
2.53	1.61\\
2.53	1.61\\
2.53	1.61\\
2.53	1.61\\
2.53	1.61\\
2.53	1.61\\
2.53	1.61\\
2.53	1.61\\
2.53	1.61\\
2.53	1.61\\
2.53	1.61\\
2.53	1.61\\
2.53	1.61\\
2.53	1.61\\
2.53	1.61\\
2.53	1.61\\
2.53	1.61\\
2.53	1.61\\
2.53	1.61\\
2.53	1.61\\
2.53	1.61\\
2.53	1.61\\
2.53	1.61\\
2.53	1.61\\
2.53	1.61\\
2.53	1.61\\
2.53	1.61\\
2.53	1.61\\
2.53	1.61\\
2.53	1.61\\
2.53	1.61\\
2.53	1.61\\
2.53	1.61\\
2.53	1.61\\
2.53	1.61\\
2.53	1.61\\
2.53	1.61\\
2.53	1.61\\
2.53	1.61\\
2.53	1.61\\
2.53	1.61\\
2.53	1.61\\
2.53	1.61\\
2.53	1.61\\
2.53	1.61\\
2.53	1.61\\
2.53	1.61\\
2.53	1.61\\
2.53	1.61\\
2.53	1.61\\
2.53	1.61\\
2.53	1.61\\
2.53	1.61\\
2.53	1.61\\
2.53	1.61\\
2.53	1.61\\
2.53	1.61\\
2.53	1.61\\
2.53	1.61\\
2.53	1.61\\
2.53	1.61\\
2.53	1.61\\
2.53	1.61\\
2.53	1.61\\
2.53	1.61\\
2.53	1.61\\
2.53	1.61\\
2.53	1.61\\
2.53	1.61\\
2.53	1.61\\
2.53	1.61\\
2.53	1.61\\
2.53	1.61\\
2.53	1.61\\
2.53	1.61\\
2.53	1.61\\
2.53	1.61\\
2.53	1.61\\
2.53	1.61\\
2.53	1.61\\
2.53	1.61\\
2.53	1.61\\
2.53	1.61\\
2.53	1.61\\
2.53	1.61\\
2.53	1.61\\
2.53	1.61\\
2.53	1.61\\
2.53	1.61\\
2.53	1.61\\
2.53	1.61\\
2.53	1.61\\
2.53	1.61\\
2.53	1.61\\
2.53	1.61\\
2.53	1.61\\
2.53	1.61\\
2.53	1.61\\
2.53	1.61\\
2.53	1.61\\
2.53	1.61\\
2.53	1.61\\
2.53	1.61\\
2.53	1.61\\
2.53	1.61\\
2.53	1.61\\
2.53	1.61\\
2.53	1.61\\
2.53	1.61\\
2.53	1.61\\
2.53	1.61\\
2.53	1.61\\
2.53	1.61\\
2.53	1.61\\
2.53	1.61\\
2.53	1.61\\
2.53	1.61\\
2.53	1.61\\
2.53	1.61\\
2.53	1.61\\
2.53	1.61\\
2.53	1.61\\
2.53	1.61\\
2.53	1.61\\
2.53	1.61\\
2.53	1.61\\
2.53	1.61\\
2.53	1.61\\
2.53	1.61\\
2.53	1.61\\
2.53	1.61\\
2.53	1.61\\
2.53	1.61\\
2.53	1.61\\
2.53	1.61\\
2.53	1.61\\
2.53	1.61\\
2.53	1.61\\
2.53	1.61\\
2.53	1.61\\
2.53	1.61\\
2.53	1.61\\
2.53	1.62\\
2.53	1.62\\
2.53	1.62\\
2.53	1.62\\
2.53	1.62\\
2.53	1.62\\
2.53	1.62\\
2.53	1.61\\
2.53	1.61\\
2.53	1.61\\
2.53	1.61\\
2.53	1.61\\
2.53	1.61\\
2.53	1.61\\
2.53	1.61\\
2.53	1.61\\
2.53	1.61\\
2.53	1.61\\
2.53	1.61\\
2.53	1.61\\
2.53	1.61\\
2.53	1.61\\
2.53	1.61\\
2.53	1.61\\
2.53	1.61\\
2.53	1.61\\
2.53	1.61\\
2.53	1.61\\
2.53	1.61\\
2.53	1.61\\
2.53	1.61\\
2.53	1.61\\
2.53	1.61\\
2.53	1.61\\
2.53	1.61\\
2.53	1.61\\
2.53	1.61\\
2.53	1.61\\
2.53	1.61\\
2.53	1.61\\
2.53	1.61\\
2.53	1.61\\
2.53	1.61\\
2.53	1.61\\
2.53	1.61\\
2.53	1.61\\
2.53	1.61\\
2.53	1.61\\
2.53	1.61\\
2.53	1.61\\
2.53	1.61\\
2.53	1.61\\
2.53	1.61\\
2.53	1.61\\
2.53	1.61\\
2.53	1.61\\
2.53	1.61\\
2.53	1.61\\
2.53	1.61\\
2.53	1.61\\
2.53	1.61\\
2.53	1.61\\
2.53	1.61\\
2.53	1.61\\
2.53	1.61\\
2.53	1.61\\
2.53	1.61\\
2.53	1.61\\
2.53	1.61\\
2.53	1.61\\
2.53	1.61\\
2.53	1.61\\
2.53	1.61\\
2.53	1.61\\
2.53	1.61\\
2.53	1.61\\
2.53	1.61\\
2.53	1.61\\
2.53	1.61\\
2.53	1.61\\
2.53	1.61\\
2.53	1.61\\
2.53	1.61\\
2.53	1.61\\
2.53	1.61\\
2.53	1.61\\
2.53	1.61\\
2.53	1.61\\
2.53	1.61\\
2.53	1.61\\
2.53	1.61\\
2.53	1.61\\
2.53	1.62\\
2.53	1.62\\
2.53	1.62\\
2.53	1.62\\
2.53	1.62\\
2.53	1.62\\
2.53	1.62\\
2.53	1.62\\
2.53	1.62\\
2.53	1.62\\
2.53	1.62\\
2.53	1.62\\
2.53	1.62\\
2.53	1.62\\
2.53	1.62\\
2.53	1.62\\
2.53	1.62\\
2.53	1.62\\
2.53	1.62\\
2.53	1.62\\
2.53	1.62\\
2.53	1.62\\
2.53	1.62\\
2.53	1.62\\
2.53	1.62\\
2.53	1.62\\
2.53	1.62\\
2.53	1.62\\
2.53	1.62\\
2.53	1.62\\
2.53	1.62\\
2.53	1.62\\
2.53	1.62\\
2.53	1.62\\
2.53	1.62\\
2.53	1.62\\
2.53	1.62\\
2.53	1.62\\
2.53	1.62\\
2.53	1.62\\
2.53	1.62\\
2.53	1.62\\
2.53	1.62\\
2.53	1.62\\
2.53	1.62\\
2.53	1.62\\
2.53	1.62\\
2.53	1.62\\
2.53	1.62\\
2.53	1.62\\
2.53	1.62\\
2.53	1.62\\
2.53	1.62\\
2.53	1.62\\
2.53	1.62\\
2.53	1.62\\
2.53	1.62\\
2.53	1.62\\
2.53	1.62\\
2.53	1.62\\
2.53	1.62\\
2.53	1.62\\
2.53	1.62\\
2.53	1.62\\
2.53	1.62\\
2.53	1.62\\
2.53	1.62\\
2.53	1.62\\
2.53	1.62\\
2.53	1.62\\
2.53	1.62\\
2.53	1.62\\
2.53	1.62\\
2.53	1.62\\
2.53	1.62\\
2.53	1.62\\
2.53	1.62\\
2.53	1.62\\
2.53	1.62\\
2.53	1.62\\
2.53	1.62\\
2.53	1.62\\
2.53	1.62\\
2.53	1.61\\
2.53	1.61\\
2.53	1.61\\
2.53	1.61\\
2.53	1.61\\
2.53	1.61\\
2.53	1.61\\
2.53	1.61\\
2.53	1.61\\
2.53	1.61\\
2.53	1.61\\
2.53	1.61\\
2.53	1.61\\
2.53	1.61\\
2.53	1.61\\
2.53	1.61\\
2.53	1.61\\
2.53	1.61\\
2.53	1.61\\
2.53	1.61\\
2.53	1.61\\
2.53	1.61\\
2.53	1.61\\
2.53	1.61\\
2.53	1.61\\
2.53	1.61\\
2.53	1.61\\
2.53	1.61\\
2.53	1.61\\
2.53	1.61\\
2.53	1.61\\
2.53	1.61\\
2.53	1.61\\
2.53	1.61\\
2.53	1.61\\
2.53	1.61\\
2.53	1.61\\
2.53	1.61\\
2.53	1.61\\
2.53	1.61\\
2.53	1.61\\
2.53	1.61\\
2.53	1.61\\
2.53	1.61\\
2.53	1.61\\
2.53	1.61\\
2.53	1.61\\
2.53	1.61\\
2.53	1.61\\
2.53	1.61\\
2.53	1.61\\
2.53	1.61\\
2.53	1.61\\
2.53	1.61\\
2.53	1.61\\
2.53	1.61\\
2.53	1.61\\
2.53	1.61\\
2.53	1.61\\
2.53	1.61\\
2.53	1.61\\
2.53	1.61\\
2.53	1.61\\
2.53	1.61\\
2.53	1.61\\
2.53	1.61\\
2.53	1.61\\
2.53	1.61\\
2.53	1.61\\
2.53	1.61\\
2.53	1.61\\
2.53	1.61\\
2.53	1.61\\
2.53	1.61\\
2.53	1.61\\
2.53	1.61\\
2.53	1.61\\
2.53	1.61\\
2.53	1.61\\
2.53	1.61\\
2.53	1.61\\
2.53	1.61\\
2.53	1.61\\
2.53	1.61\\
2.53	1.61\\
2.53	1.61\\
2.53	1.61\\
2.53	1.61\\
2.53	1.61\\
2.53	1.61\\
2.53	1.61\\
2.53	1.61\\
2.53	1.61\\
2.53	1.61\\
2.53	1.61\\
2.53	1.61\\
2.53	1.61\\
2.53	1.61\\
2.53	1.61\\
2.53	1.61\\
2.53	1.61\\
2.53	1.61\\
2.53	1.61\\
2.53	1.61\\
2.53	1.61\\
2.53	1.61\\
2.53	1.61\\
2.53	1.61\\
2.53	1.61\\
2.53	1.61\\
2.53	1.61\\
2.53	1.61\\
2.53	1.61\\
2.53	1.61\\
2.53	1.61\\
2.53	1.61\\
2.53	1.61\\
2.53	1.61\\
2.53	1.61\\
2.53	1.61\\
2.53	1.61\\
2.53	1.61\\
2.53	1.61\\
2.53	1.61\\
2.53	1.61\\
2.53	1.61\\
2.53	1.61\\
2.53	1.61\\
2.53	1.61\\
2.53	1.61\\
2.53	1.61\\
2.53	1.61\\
2.53	1.61\\
2.53	1.61\\
2.53	1.61\\
2.53	1.61\\
2.53	1.61\\
2.53	1.61\\
2.53	1.61\\
2.53	1.61\\
2.53	1.61\\
2.53	1.61\\
2.53	1.61\\
2.53	1.61\\
2.53	1.61\\
2.53	1.61\\
2.53	1.61\\
2.53	1.61\\
2.53	1.61\\
2.53	1.61\\
2.53	1.61\\
2.53	1.61\\
2.53	1.61\\
2.53	1.61\\
2.53	1.61\\
2.53	1.61\\
2.53	1.61\\
2.53	1.61\\
2.53	1.61\\
2.53	1.61\\
2.53	1.61\\
2.53	1.61\\
2.53	1.61\\
2.53	1.61\\
2.53	1.61\\
2.53	1.61\\
2.53	1.61\\
2.53	1.61\\
2.53	1.61\\
2.53	1.61\\
2.53	1.61\\
2.53	1.61\\
2.53	1.61\\
2.53	1.61\\
2.53	1.61\\
2.53	1.61\\
2.53	1.61\\
2.53	1.61\\
2.53	1.61\\
2.53	1.61\\
2.53	1.61\\
2.53	1.61\\
2.53	1.61\\
2.53	1.61\\
2.53	1.61\\
2.53	1.61\\
2.53	1.61\\
2.53	1.61\\
2.53	1.61\\
2.53	1.61\\
2.53	1.61\\
2.53	1.61\\
2.53	1.61\\
2.53	1.61\\
2.53	1.61\\
2.53	1.61\\
2.53	1.61\\
2.53	1.61\\
2.53	1.61\\
2.53	1.61\\
2.53	1.61\\
2.53	1.61\\
2.53	1.61\\
2.53	1.61\\
2.53	1.61\\
2.53	1.61\\
2.53	1.61\\
2.53	1.62\\
2.53	1.61\\
2.53	1.61\\
2.53	1.61\\
2.53	1.61\\
2.53	1.61\\
2.53	1.61\\
2.53	1.61\\
2.53	1.61\\
2.53	1.61\\
2.53	1.61\\
2.53	1.61\\
2.53	1.61\\
2.53	1.61\\
2.53	1.61\\
2.53	1.62\\
2.53	1.62\\
2.53	1.62\\
2.53	1.62\\
2.53	1.62\\
2.53	1.62\\
2.53	1.61\\
2.53	1.61\\
2.53	1.61\\
2.53	1.62\\
2.53	1.62\\
2.53	1.62\\
2.53	1.62\\
2.53	1.62\\
2.53	1.62\\
2.53	1.62\\
2.53	1.62\\
2.53	1.62\\
2.53	1.62\\
2.53	1.62\\
2.53	1.62\\
2.53	1.62\\
2.53	1.62\\
2.53	1.62\\
2.53	1.62\\
2.53	1.62\\
2.53	1.62\\
2.53	1.62\\
2.53	1.62\\
2.53	1.62\\
2.53	1.62\\
2.53	1.62\\
2.52	1.62\\
2.52	1.62\\
2.52	1.62\\
2.52	1.62\\
2.52	1.62\\
2.52	1.62\\
2.53	1.62\\
2.53	1.62\\
2.53	1.61\\
2.53	1.62\\
2.53	1.62\\
2.53	1.61\\
};
\addplot [color=blue,only marks,mark=asterisk,mark options={solid},forget plot]
  table[row sep=crcr]{%
-0.37	1.68\\
1.09	1.68\\
2.55	1.68\\
-0.37	0.23\\
1.09	0.23\\
2.55	0.23\\
-0.37	-1.22\\
1.09	-1.22\\
2.55	-1.22\\
};
\node[right, align=left, text=black]
at (axis cs:-0.38,1.53) {1};
\node[right, align=left, text=black]
at (axis cs:1.08,1.53) {2};
\node[right, align=left, text=black]
at (axis cs:2.54,1.53) {3};
\node[right, align=left, text=black]
at (axis cs:-0.38,0.08) {4};
\node[right, align=left, text=black]
at (axis cs:1.08,0.08) {5};
\node[right, align=left, text=black]
at (axis cs:2.54,0.08) {6};
\node[right, align=left, text=black]
at (axis cs:-0.38,-1.37) {7};
\node[right, align=left, text=black]
at (axis cs:1.08,-1.37) {8};
\node[right, align=left, text=black]
at (axis cs:2.54,-1.37) {9};
\end{axis}
\end{tikzpicture}%}
  \caption{The trajectory of the robot in real-life conditions. The anglular
    tolerance was set to $\xi = 4^{\circ}$ and the distance tolerance to
    $\delta$ = 6 cm. Direction of travel is clockwise, starting from node 3.}
  \label{fig:22_map_7}
\end{figure}
