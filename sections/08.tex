For the purpose of simulating this part of the controller, the initial point of
the robot was taken to be $I (x_0, y_0) \equiv (0,0)$. The goal was set to
$G (x_g, y_g) \equiv (-0.37, 1.68)$, which is node $1$ in the simulation
environment. The angle between the line connecting $I$ and $G$ and the
x-axis is hence $\theta_R = tan^{-1}(1.68 / -0.37) = 102.42$ degrees.

Figures \ref{fig:8_0.1_max}, \ref{fig:8_0.2_max}, \ref{fig:8_0.5_max},
\ref{fig:8_0.75_max}, \ref{fig:8_max} show the angular response of the robot
for different values of $K_{\Psi}^R$ inside the interval set by inequality
\ref{eq:7.2}. Figure \ref{fig:8_max_plus_one} verifies that the upper limit
for $K_{\Psi}^R$ is indeed $K_{\Psi,max}^R = \frac{2L}{R}$ by showing that the
angular response of the robot cannot converge for $K_{\Psi}^R > K_{\Psi,max}^R$.

Here, one can see that the smaller the value of $K_{\Psi}^R$ is, the larger the
settling time, the lower the rise time and the smoother the response is. However,
as the value of $K_{\Psi}^R$ increases, the steady-state response begins to
oscillate, with the amplitude of this oscillation proportional to the value of
$K_{\Psi}^R$.

Figures \ref{fig:8_0.1_max_magnified}, \ref{fig:8_0.2_max_magnified},
\ref{fig:8_0.5_max_magnified}, \ref{fig:8_0.75_max_magnified} and
\ref{fig:8_0.1_max_magnified} focus on the steady-state value of the
aforementioned responses. As it is evident, none of the responses converge to
the value $\theta_R = 102.42$. This is reasonable since with only a purely
proportional control signal, as the angular error, i.e. $e(\theta) = \theta^R - \theta$,
tends to zero, the product of $K_{\Psi}^R$ and $e(\theta)$ isn't large enough
to force the robot to rotate exactly $\theta^R$ degrees.

Another way to look at this is by looking at the steady-state response of the
system, which is linear, for a step input of magnitude $\theta^R$. Figure
\ref{fig:8_block_diagram} shows the structure of the system. The z-transform of
the input is then $R(z) = \dfrac{\theta^R}{1 - z^{-1}}$ and the equation of the
closed-loop system is
$$Y(z) = \dfrac{K_{\Psi}^R G(z)}{1 + K_{\Psi}^R G(z)} R(z)$$
The steady-state response is
$$lim_{t \to \infty} y(t) = lim_{z \to 1} (1-z^{-1}) \dfrac{\theta^R}{1-z^{-1}}
\dfrac{K_{\Psi}^R G(z)}{1 + K_{\Psi}^R G(z)} = \theta^R \cdot lim_{z \to 1}
\dfrac{K_{\Psi}^R G(z)}{1 + K_{\Psi}^R G(z)}$$

The steady-state response cannot reach exactly $\theta^R$ as the above limit
cannot converge to 1 under our limitations for $K_{\Psi}^R$ and the dynamics of
$G(z)$.

\begin{figure}\centering
  \scalebox{0.9}{% Generated with LaTeXDraw 2.0.8
% Fri Feb 26 16:07:26 CET 2016
% \usepackage[usenames,dvipsnames]{pstricks}
% \usepackage{epsfig}
% \usepackage{pst-grad} % For gradients
% \usepackage{pst-plot} % For axes
\scalebox{1} % Change this value to rescale the drawing.
{
\begin{pspicture}(0,-1.05)(9.869375,1.055625)
\psframe[linewidth=0.04,dimen=outer](8.02,1.03)(6.02,-0.09)
\usefont{T1}{cmr12}{m}{n}
\rput(7.0045314,0.48){$G(z)$}
\psframe[linewidth=0.04,dimen=outer](4.56,0.87)(3.4,0.05)
\usefont{T1}{cmr12}{m}{n}
\rput(3.9621875,0.48){$K_{\Psi}^R$}
\psline[linewidth=0.04cm,arrowsize=0.05291667cm 2.0,arrowlength=1.4,arrowinset=0.4]{->}(4.56,0.35)(6.04,0.33)
\psline[linewidth=0.04cm,arrowsize=0.05291667cm 2.0,arrowlength=1.4,arrowinset=0.4]{->}(1.94,0.39)(3.42,0.37)
\psline[linewidth=0.04cm,arrowsize=0.05291667cm 2.0,arrowlength=1.4,arrowinset=0.4]{->}(8.0,0.39)(9.48,0.37)
\pscircle[linewidth=0.04,dimen=outer](1.7,0.39){0.26}
\psline[linewidth=0.04cm,arrowsize=0.05291667cm 2.0,arrowlength=1.4,arrowinset=0.4]{->}(0.0,0.39)(1.48,0.37)
\psline[linewidth=0.04,arrowsize=0.05291667cm 2.0,arrowlength=1.4,arrowinset=0.4]{->}(8.68,0.37)(8.68,-1.01)(1.68,-1.03)(1.68,0.09)(1.68,0.15)
\usefont{T1}{ptm}{m}{n}
\rput(9.495,0.74){$Y(z)$}
\usefont{T1}{ptm}{m}{n}
\rput(1.4326563,0.9){+}
\usefont{T1}{ptm}{m}{n}
\rput(1.3473438,-0.18){-}
\usefont{T1}{ptm}{m}{n}
\rput(0.50703126,0.8){$R(z)$}
\end{pspicture}
}

}
  \caption{The structure of the system under rotational control. $G(z)$ is the
    discretized transfer function of the linear system whose state-space
    equation is $\dot{\theta} = \frac{R}{L} u_{\Psi}$}
  \label{fig:8_block_diagram}
\end{figure}


\noindent\makebox[\textwidth][c]{%
\begin{minipage}{\linewidth}
  \begin{minipage}{0.45\linewidth}
    \begin{figure}[H]
      \scalebox{0.6}{% This file was created by matlab2tikz.
%
%The latest updates can be retrieved from
%  http://www.mathworks.com/matlabcentral/fileexchange/22022-matlab2tikz-matlab2tikz
%where you can also make suggestions and rate matlab2tikz.
%
\definecolor{mycolor1}{rgb}{0.00000,0.44700,0.74100}%
%
\begin{tikzpicture}

\begin{axis}[%
width=4.133in,
height=3.26in,
at={(0.693in,0.44in)},
scale only axis,
xmin=0,
xmax=34,
xmajorgrids,
ymin=0,
ymax=120,
ymajorgrids,
axis background/.style={fill=white}
]
\addplot [color=mycolor1,solid,forget plot]
  table[row sep=crcr]{%
0	0\\
0.0174300379990004	0.24\\
0.0324506270000003	0.55\\
0.0479367249989994	0.85\\
0.0638417170000002	1.16\\
0.0819980969999991	1.55\\
0.0958595020000003	1.79\\
0.111990205998998	2.13\\
0.128003443999	2.46\\
0.144020304998998	2.79\\
0.159882857999999	3.12\\
0.175851340999	3.42\\
0.192007680999	3.75\\
0.208044713	4.08\\
0.224115767998997	4.4\\
0.239838469999999	4.72\\
0.25818727	5.11\\
0.273270488998998	5.41\\
0.288371083999	5.72\\
0.304055936999999	6.02\\
0.320009518998999	6.34\\
0.336038854998998	6.67\\
0.351893171999998	6.99\\
0.367914987999998	7.31\\
0.383973559999	7.63\\
0.400995524999	8\\
0.416045024	8.3\\
0.432006399999	8.61\\
0.448025189998999	8.93\\
0.464038138999999	9.25\\
0.480015434	9.58\\
0.496034375	9.9\\
0.512015028998997	10.23\\
0.528082830998998	10.55\\
0.549680989999998	11\\
0.561620908	11.24\\
0.576748722998999	11.55\\
0.592070063999999	11.86\\
0.608254215999	12.18\\
0.624067698999997	12.49\\
0.640107527999997	12.81\\
0.655923043998998	13.14\\
0.671984959998999	13.45\\
0.688054060999997	13.78\\
0.704052780997999	14.11\\
0.720024088998999	14.43\\
0.735992154998999	14.75\\
0.751957489999998	15.07\\
0.768220824999999	15.4\\
0.783993932999	15.72\\
0.799905231999997	16.04\\
0.816094957998998	16.37\\
0.832011709999997	16.69\\
0.848004915998999	17.02\\
0.863793581999998	17.34\\
0.8822515	17.72\\
0.895927096999999	17.98\\
0.911987252999999	18.32\\
0.927891984998997	18.63\\
0.946389945999	19.02\\
0.961619046999998	19.33\\
0.976069737	19.62\\
0.991921077999998	19.93\\
1.009968221999	20.27\\
1.024879852999	20.52\\
1.039853554	20.77\\
1.057957825999	21.07\\
1.072917568	21.32\\
1.087888876999	21.57\\
1.104042991	21.82\\
1.120017256	22.09\\
1.136036412999	22.36\\
1.151969417999	22.62\\
1.168006750999	22.89\\
1.184002208999	23.16\\
1.199974418999	23.42\\
1.216305891999	23.69\\
1.231898047999	23.96\\
1.247983779999	24.22\\
1.264015224	24.5\\
1.280097093	24.75\\
1.295856492999	25.02\\
1.312070260999	25.29\\
1.328076495	25.56\\
1.344009986999	25.83\\
1.361950467999	26.13\\
1.377124443999	26.38\\
1.392315348999	26.64\\
1.407866441999	26.89\\
1.424419446	27.17\\
1.440021116	27.42\\
1.456095894	27.68\\
1.472018340999	27.95\\
1.488025036999	28.22\\
1.504037309999	28.48\\
1.520005207	28.75\\
1.536027474	29.02\\
1.551899911999	29.28\\
1.570240232	29.6\\
1.585276435	29.85\\
1.600431330999	30.1\\
1.616136391999	30.35\\
1.632071121	30.62\\
1.648080415	30.88\\
1.663970845	31.14\\
1.682181233999	31.46\\
1.697410846999	31.72\\
1.712734807	31.97\\
1.727958361	32.23\\
1.749574771999	32.58\\
1.761599528999	32.79\\
1.776900814999	33.04\\
1.792267257	33.3\\
1.809894746999	33.59\\
1.825139976999	33.84\\
1.840363133	34.1\\
1.85593825	34.35\\
1.871999549999	34.61\\
1.887803738	34.88\\
1.903839515	35.14\\
1.919961884999	35.41\\
1.936075149	35.67\\
1.951943109999	35.94\\
1.968066234999	36.21\\
1.984011825	36.48\\
1.999938906999	36.74\\
2.017256623999	37\\
2.032199356999	37.19\\
2.047890916	37.39\\
2.063890505	37.6\\
2.079915071999	37.81\\
2.096001938999	38.02\\
2.112032936999	38.24\\
2.128057590999	38.44\\
2.144937938999	38.67\\
2.160079861	38.87\\
2.176075094999	39.07\\
2.191983429999	39.27\\
2.208101595999	39.48\\
2.224045633	39.7\\
2.239809741999	39.9\\
2.256023873999	40.11\\
2.271973884	40.32\\
2.287863977999	40.53\\
2.303949406999	40.74\\
2.320022412999	40.95\\
2.336006296999	41.16\\
2.354573697999	41.41\\
2.369639749	41.61\\
2.384856592	41.81\\
2.400038186	42\\
2.415978520999	42.2\\
2.431964903999	42.41\\
2.44806087	42.62\\
2.463989645999	42.83\\
2.480060068999	43.04\\
2.496163776	43.26\\
2.512003793001	43.46\\
2.528047775	43.67\\
2.54410579	43.88\\
2.560266229999	44.1\\
2.576030780998	44.3\\
2.59190329	44.5\\
2.607926851	44.71\\
2.623859651	44.92\\
2.640011726999	45.13\\
2.656058925999	45.34\\
2.671868436999	45.55\\
2.688024283999	45.76\\
2.70396642	45.97\\
2.719983380999	46.19\\
2.736097559	46.39\\
2.751901967	46.6\\
2.768192910999	46.81\\
2.784010139999	47.02\\
2.799915858	47.23\\
2.816190153999	47.44\\
2.831814283999	47.65\\
2.850254785999	47.89\\
2.86542613	48.09\\
2.880610297999	48.29\\
2.895834915	48.49\\
2.912044984999	48.69\\
2.928004087999	48.9\\
2.943869503999	49.11\\
2.960004231999	49.32\\
2.976060762999	49.53\\
2.992000057	49.74\\
3.009236875999	49.95\\
3.024219209999	50.11\\
3.039888473	50.26\\
3.056004924999	50.42\\
3.071871139999	50.59\\
3.088053345999	50.75\\
3.103908068999	50.92\\
3.120086937999	51.08\\
3.136024108999	51.25\\
3.151933680999	51.41\\
3.170234823	51.61\\
3.185320685	51.77\\
3.201081613	51.93\\
3.216224429	52.08\\
3.232039993999	52.24\\
3.248116384	52.4\\
3.264206779999	52.57\\
3.280253116999	52.74\\
3.295930790999	52.9\\
3.311998323999	53.06\\
3.327981663999	53.23\\
3.34392566	53.39\\
3.359999385	53.56\\
3.378549962999	53.76\\
3.393819269999	53.91\\
3.408922368	54.07\\
3.424086291	54.23\\
3.440034217999	54.38\\
3.456398528999	54.56\\
3.472004142999	54.71\\
3.487791478999	54.88\\
3.50384554	55.04\\
3.522273228999	55.24\\
3.537372677999	55.39\\
3.552483726	55.55\\
3.568083190999	55.71\\
3.583913052	55.87\\
3.600018516	56.03\\
3.616126227999	56.19\\
3.632030574	56.36\\
3.648005750999	56.53\\
3.663860558999	56.69\\
3.679967891	56.86\\
3.696572628999	57.03\\
3.712331441999	57.19\\
3.727837571	57.35\\
3.745929883999	57.54\\
3.760847450999	57.7\\
3.776579752	57.86\\
3.791947991999	58.01\\
3.807953543999	58.17\\
3.824080022999	58.35\\
3.840029635999	58.5\\
3.856091335	58.68\\
3.872018657999	58.84\\
3.888049205999	59\\
3.903929995999	59.16\\
3.919993915999	59.33\\
3.936028266999	59.49\\
3.951944033999	59.66\\
3.968079104999	59.82\\
3.984022903	59.99\\
4.000108506	60.15\\
4.016057264	60.3\\
4.032020855	60.43\\
4.048154955999	60.57\\
4.064045054999	60.7\\
4.080020176	60.83\\
4.096089427999	60.96\\
4.112060354999	61.1\\
4.128039825	61.23\\
4.144017658999	61.36\\
4.161276334999	61.51\\
4.176431502999	61.64\\
4.191971858	61.77\\
4.207943458999	61.9\\
4.224013965	62.03\\
4.239854238	62.16\\
4.255999326001	62.29\\
4.272028487	62.43\\
4.288013627	62.56\\
4.304057674	62.7\\
4.320028175999	62.83\\
4.336054597999	62.96\\
4.351962236999	63.1\\
4.367986194999	63.23\\
4.384050453	63.36\\
4.399909967	63.5\\
4.416040752999	63.63\\
4.431938975999	63.76\\
4.447962996999	63.89\\
4.463958000999	64.03\\
4.482286186	64.19\\
4.497350867999	64.31\\
4.512533074	64.44\\
4.528026731	64.57\\
4.543915373999	64.69\\
4.560190772999	64.83\\
4.576110061	64.96\\
4.592035089998	65.09\\
4.607989397999	65.23\\
4.624072344	65.36\\
4.640046653999	65.49\\
4.656059656	65.63\\
4.671808891	65.76\\
4.690099175999	65.92\\
4.705304392999	66.04\\
4.720498165	66.17\\
4.7359396	66.3\\
4.751987200999	66.42\\
4.768152628999	66.56\\
4.784038197999	66.69\\
4.800071917	66.83\\
4.818437517999	66.98\\
4.833528897999	67.11\\
4.848744887	67.24\\
4.863986880999	67.36\\
4.879998545999	67.49\\
4.896045840999	67.62\\
4.912085674999	67.76\\
4.928697207999	67.9\\
4.943859212	68.03\\
4.960436904999	68.17\\
4.975961495	68.3\\
4.99192647	68.42\\
5.009781466	68.56\\
5.025024462999	68.67\\
5.04011074	68.77\\
5.055874448999	68.87\\
5.072021234999	68.98\\
5.088021239	69.08\\
5.103986361999	69.19\\
5.119914854	69.3\\
5.136079538999	69.41\\
5.151938932	69.52\\
5.168031897	69.62\\
5.183997121	69.73\\
5.199997324999	69.84\\
5.218478833999	69.97\\
5.233569741	70.07\\
5.248701339998	70.17\\
5.26397441	70.28\\
5.280017299999	70.38\\
5.295859715999	70.48\\
5.311922009999	70.59\\
5.328071387999	70.7\\
5.345589873999	70.83\\
5.361996458999	70.94\\
5.377553205999	71.04\\
5.392728549999	71.14\\
5.408081906999	71.25\\
5.424115259999	71.35\\
5.439986802	71.46\\
5.455935986999	71.56\\
5.471969615	71.67\\
5.488010734999	71.78\\
5.50391891	71.89\\
5.520069117	71.99\\
5.536107422	72.1\\
5.551888649999	72.21\\
5.568130746999	72.32\\
5.584012358	72.43\\
5.600019125999	72.53\\
5.616315641	72.64\\
5.632054712999	72.75\\
5.648006667999	72.86\\
5.663965429999	72.97\\
5.680370297999	73.05\\
5.696037229999	73.18\\
5.711911581	73.29\\
5.728039937999	73.4\\
5.74396125	73.51\\
5.759996491	73.61\\
5.776010407	73.72\\
5.791988297999	73.83\\
5.808011313999	73.94\\
5.823856184999	74.04\\
5.839962602	74.15\\
5.856026040999	74.26\\
5.872010759	74.37\\
5.887813002	74.47\\
5.903837720999	74.58\\
5.919806392999	74.69\\
5.938156772	74.82\\
5.953362993	74.92\\
5.968693609001	75.02\\
5.984073679	75.13\\
6.000050269	75.23\\
6.01762837	75.33\\
6.032871644	75.41\\
6.048241441	75.49\\
6.064001175999	75.57\\
6.080146546999	75.65\\
6.097613905998	75.75\\
6.112767851999	75.83\\
6.128064112	75.9\\
6.144299521999	75.98\\
6.160850463999	76.07\\
6.176111682	76.15\\
6.192025755	76.23\\
6.208005837999	76.31\\
6.223992929999	76.39\\
6.240025719	76.48\\
6.256063637999	76.56\\
6.272079406999	76.64\\
6.288184568999	76.73\\
6.304077979999	76.81\\
6.319927441999	76.89\\
6.336073578	76.97\\
6.351912856	77.05\\
6.368057432999	77.14\\
6.383960774	77.22\\
6.400022891999	77.3\\
6.416082533	77.38\\
6.432004958999	77.47\\
6.448018887999	77.55\\
6.463904531999	77.63\\
6.480006424	77.71\\
6.495827855	77.8\\
6.512043066	77.88\\
6.528170748	77.96\\
6.544107979	78.04\\
6.559990125	78.13\\
6.576109266999	78.21\\
6.591993713999	78.29\\
6.608008726	78.37\\
6.624082582	78.46\\
6.640033125	78.54\\
6.655990778	78.62\\
6.671903263999	78.7\\
6.687967759	78.78\\
6.704050907	78.87\\
6.720026838999	78.95\\
6.736085488	79.03\\
6.751955824	79.12\\
6.768058355999	79.2\\
6.783978331	79.28\\
6.800066644	79.36\\
6.816128960999	79.45\\
6.831928957999	79.53\\
6.850405115	79.63\\
6.865640250999	79.7\\
6.880828882999	79.78\\
6.89585413	79.86\\
6.912091141	79.94\\
6.927879743	80.02\\
6.943874983	80.1\\
6.960064842999	80.19\\
6.976078529999	80.27\\
6.992001869999	80.35\\
7.009532528999	80.44\\
7.024752501999	80.51\\
7.039853299999	80.57\\
7.056058830999	80.64\\
7.072014644	80.71\\
7.087797194	80.78\\
7.103797762999	80.85\\
7.119851067	80.92\\
7.136030307	80.99\\
7.151964165999	81.06\\
7.168001161999	81.13\\
7.184027970999	81.2\\
7.200045212999	81.27\\
7.216365135999	81.34\\
7.232013594	81.41\\
7.248005705	81.47\\
7.264001495999	81.55\\
7.279970844	81.61\\
7.296102224	81.69\\
7.311943193	81.76\\
7.328165029	81.82\\
7.344316811	81.89\\
7.360084506999	81.97\\
7.375996045	82.03\\
7.392017035999	82.1\\
7.408018661999	82.17\\
7.424114423	82.24\\
7.439994368999	82.31\\
7.455827253	82.38\\
7.471999231	82.45\\
7.487845097	82.52\\
7.503932318999	82.59\\
7.519986832	82.66\\
7.536075719	82.73\\
7.551918491	82.8\\
7.567998386999	82.87\\
7.584005873	82.94\\
7.600090101	83.01\\
7.618474867	83.09\\
7.633623033	83.16\\
7.648694696999	83.23\\
7.663862833999	83.29\\
7.679848073999	83.36\\
7.698113476999	83.44\\
7.713293687	83.51\\
7.728518156	83.57\\
7.744830738	83.65\\
7.759886396	83.71\\
7.775837782999	83.78\\
7.791788658999	83.85\\
7.807860399	83.92\\
7.824065279	83.99\\
7.840040006999	84.06\\
7.856001729999	84.13\\
7.872038050999	84.2\\
7.888012184999	84.27\\
7.904005670999	84.34\\
7.920082967	84.41\\
7.936098413999	84.48\\
7.951986653999	84.55\\
7.967831160999	84.61\\
7.984018417	84.68\\
8.000109231999	84.75\\
8.017665713999	84.83\\
8.032738621	84.88\\
8.048021069999	84.94\\
8.064060883999	84.99\\
8.080091361999	85.05\\
8.095828138999	85.1\\
8.111809260999	85.16\\
8.130070782999	85.23\\
8.145361875	85.28\\
8.160517995	85.34\\
8.176576925999	85.39\\
8.191917566999	85.45\\
8.208003382	85.5\\
8.224055399999	85.56\\
8.239878319	85.62\\
8.256056035	85.67\\
8.271989972999	85.73\\
8.288174182	85.79\\
8.304042287999	85.85\\
8.320017354999	85.9\\
8.336097194	85.96\\
8.352002476999	86.02\\
8.367937003999	86.07\\
8.383860592999	86.13\\
8.400008908	86.19\\
8.416160743999	86.25\\
8.431999549	86.3\\
8.45037326599999	86.37\\
8.465420122998	86.43\\
8.480946016999	86.48\\
8.495989852999	86.53\\
8.511963651	86.59\\
8.528111182	86.65\\
8.544318419999	86.7\\
8.561121029999	86.77\\
8.576106456	86.82\\
8.59199159699999	86.87\\
8.608043774999	86.93\\
8.623802653	86.99\\
8.639863552999	87.04\\
8.656038236999	87.1\\
8.67194906299899	87.16\\
8.688020538	87.22\\
8.70409881299899	87.27\\
8.720006797	87.33\\
8.736035901999	87.39\\
8.752030114	87.44\\
8.768169312	87.5\\
8.784047809	87.56\\
8.800026723999	87.62\\
8.816261035	87.67\\
8.83211868899999	87.73\\
8.848205413999	87.79\\
8.864006856	87.84\\
8.880025667999	87.9\\
8.895852649	87.96\\
8.911918508999	88.01\\
8.92789440499899	88.07\\
8.943988862999	88.13\\
8.960687236	88.19\\
8.976026531999	88.25\\
8.99199760799999	88.3\\
9.009578329999	88.36\\
9.024612418999	88.4\\
9.039829124999	88.44\\
9.058108440999	88.49\\
9.073157934999	88.54\\
9.088346819999	88.58\\
9.104087988999	88.62\\
9.119957168999	88.66\\
9.136131579999	88.71\\
9.152442796	88.76\\
9.16804712	88.8\\
9.184016805	88.84\\
9.200083585999	88.89\\
9.216294287999	88.93\\
9.231974493	88.97\\
9.247959871999	89.02\\
9.263977561	89.06\\
9.280082041	89.11\\
9.295938438999	89.15\\
9.312052896	89.2\\
9.328093946999	89.24\\
9.343983764999	89.29\\
9.360036367	89.33\\
9.376062115	89.37\\
9.392046486999	89.42\\
9.407898814999	89.46\\
9.424052587	89.51\\
9.439809943999	89.55\\
9.458065231	89.6\\
9.473261557	89.65\\
9.489298571999	89.69\\
9.504456908999	89.73\\
9.52024634	89.78\\
9.535832253999	89.82\\
9.552047956	89.86\\
9.567998061	89.91\\
9.584049735999	89.95\\
9.600024235999	90\\
9.61608032399899	90.04\\
9.632110719	90.08\\
9.647958507	90.13\\
9.663985559	90.17\\
9.68003081399999	90.22\\
9.695880082	90.26\\
9.712042518999	90.31\\
9.728043790999	90.35\\
9.746446606	90.4\\
9.761741454999	90.45\\
9.777097008	90.49\\
9.792303352999	90.53\\
9.808016688999	90.57\\
9.823991430999	90.62\\
9.840007850999	90.66\\
9.85594626599999	90.71\\
9.871993503999	90.75\\
9.888074234999	90.8\\
9.904051566999	90.84\\
9.919937335999	90.88\\
9.935953590999	90.93\\
9.951893761999	90.97\\
9.968106178	91.02\\
9.984030965	91.06\\
10.000021703	91.11\\
10.017551453999	91.15\\
10.032704333999	91.18\\
10.048581877	91.21\\
10.063868663	91.24\\
10.080018884999	91.27\\
10.095807503999	91.3\\
10.112131813999	91.33\\
10.128071866	91.37\\
10.144101938999	91.4\\
10.159862424	91.43\\
10.175815257	91.46\\
10.192031154999	91.49\\
10.209336471999	91.53\\
10.224609863999	91.56\\
10.239905970999	91.59\\
10.255894066999	91.62\\
10.272030534999	91.65\\
10.287811974	91.68\\
10.304070736999	91.71\\
10.320035012999	91.75\\
10.336208555999	91.78\\
10.352027293001	91.81\\
10.368151243999	91.84\\
10.384021019999	91.87\\
10.400076939999	91.91\\
10.416428866	91.94\\
10.431958507	91.97\\
10.448128885999	92\\
10.463988792	92.03\\
10.480107364	92.06\\
10.495878566	92.1\\
10.512034559999	92.13\\
10.527966220999	92.16\\
10.543968618	92.19\\
10.560137249999	92.22\\
10.575942292	92.25\\
10.591960439999	92.29\\
10.608049406999	92.32\\
10.623848424999	92.35\\
10.639905293001	92.38\\
10.656042726999	92.41\\
10.672074196	92.44\\
10.687824421	92.48\\
10.70380645	92.51\\
10.71990224	92.54\\
10.735957559	92.57\\
10.752054957	92.6\\
10.767937524	92.63\\
10.784032605999	92.67\\
10.799938759	92.7\\
10.816186924999	92.73\\
10.832159930999	92.76\\
10.848273646999	92.79\\
10.864073510999	92.83\\
10.88004276	92.86\\
10.895924444999	92.89\\
10.914138121	92.93\\
10.929242312999	92.96\\
10.944366873999	92.99\\
10.960503088999	93.02\\
10.976001484999	93.05\\
10.991967083999	93.08\\
11.009760097999	93.11\\
11.024955476999	93.14\\
11.040096050999	93.16\\
11.056072413	93.18\\
11.072056226999	93.21\\
11.088062693999	93.23\\
11.103975979	93.26\\
11.120017312999	93.29\\
11.136103415999	93.31\\
11.151812137999	93.34\\
11.167972691	93.36\\
11.184109149	93.39\\
11.200054507999	93.41\\
11.218515790999	93.44\\
11.234020351	93.47\\
11.249154871999	93.49\\
11.26425083	93.52\\
11.280090165	93.54\\
11.295867134	93.56\\
11.312024210999	93.59\\
11.32809245	93.62\\
11.344115684999	93.64\\
11.360256029999	93.67\\
11.376031241999	93.69\\
11.392012451999	93.72\\
11.407983450999	93.74\\
11.424021811	93.77\\
11.439853637	93.79\\
11.455998929	93.82\\
11.472070116999	93.84\\
11.489842371	93.87\\
11.505041399	93.9\\
11.520288200999	93.92\\
11.536074957999	93.95\\
11.551803049999	93.97\\
11.567837566	94\\
11.583970991999	94.02\\
11.600071949	94.05\\
11.618870576999	94.08\\
11.632097677999	94.1\\
11.648019501999	94.12\\
11.663796695999	94.15\\
11.68008317	94.17\\
11.6958865	94.2\\
11.711953227999	94.22\\
11.728109207999	94.25\\
11.743988467	94.28\\
11.760029403	94.3\\
11.776044880999	94.33\\
11.791997462999	94.35\\
11.808017712	94.38\\
11.824073451001	94.4\\
11.839851606	94.43\\
11.856053425999	94.45\\
11.872024503999	94.48\\
11.888003076001	94.5\\
11.90405321	94.53\\
11.919994213999	94.55\\
11.935832547999	94.58\\
11.951912097	94.6\\
11.967946799999	94.63\\
11.984017629999	94.66\\
12.000010917	94.68\\
12.017426229	94.71\\
12.032026545	94.73\\
12.048039432999	94.76\\
12.063816092	94.78\\
12.080031384999	94.81\\
12.096048672	94.83\\
12.114643029	94.86\\
12.129666163999	94.89\\
12.144935832	94.91\\
12.160347234999	94.94\\
12.175836604999	94.96\\
12.192090373999	94.98\\
12.208026217	95.01\\
12.224019255	95.04\\
12.239802726	95.06\\
12.255842594	95.09\\
12.272071326999	95.11\\
12.28803713	95.14\\
12.304071167999	95.16\\
12.320025436999	95.19\\
12.336198232	95.21\\
12.351959685	95.24\\
12.368076703	95.26\\
12.383939728999	95.29\\
12.400024153999	95.32\\
12.415936285	95.34\\
12.432043832	95.37\\
12.44803717	95.39\\
12.463840036999	95.42\\
12.480018215999	95.44\\
12.496114302	95.47\\
12.511933913999	95.49\\
12.528032114999	95.52\\
12.547074997998	95.55\\
12.55980854	95.57\\
12.578015246999	95.6\\
12.593206943	95.62\\
12.608455897999	95.65\\
12.624086559	95.67\\
12.639838569	95.7\\
12.655844234001	95.72\\
12.672026922999	95.75\\
12.687988484001	95.77\\
12.703914163	95.8\\
12.720082252	95.82\\
12.736082132999	95.85\\
12.752031643999	95.87\\
12.768074341999	95.9\\
12.784011043999	95.92\\
12.799984557	95.95\\
12.818526072	95.98\\
12.833770632999	96\\
12.848993260999	96.03\\
12.864366664999	96.05\\
12.880045404999	96.08\\
12.896037535	96.1\\
12.912022424999	96.13\\
12.928042988	96.15\\
12.943985717	96.18\\
12.962111873999	96.21\\
12.978052292999	96.23\\
12.993309147	96.26\\
13.007797884	96.28\\
13.024108363	96.3\\
13.039909236999	96.31\\
13.055976271	96.33\\
13.072156689999	96.35\\
13.087902147	96.37\\
13.104071703	96.39\\
13.11983134	96.41\\
13.135866918999	96.43\\
13.152006358	96.45\\
13.167976353999	96.47\\
13.184096104	96.49\\
13.200016936999	96.51\\
13.216326684	96.52\\
13.232067682	96.54\\
13.247909118999	96.56\\
13.263988495999	96.58\\
13.280058446999	96.6\\
13.295976297999	96.62\\
13.31212356	96.64\\
13.328023913999	96.66\\
13.343990028	96.68\\
13.360796479999	96.7\\
13.376065332	96.72\\
13.392106356999	96.73\\
13.408054753	96.75\\
13.423908527	96.77\\
13.439919319999	96.79\\
13.456033488	96.81\\
13.472035986999	96.83\\
13.487912981	96.85\\
13.504035021	96.87\\
13.519828585	96.89\\
13.536026608	96.9\\
13.551813215999	96.92\\
13.568058942999	96.94\\
13.583988868	96.96\\
13.600059644999	96.98\\
13.616316665999	97\\
13.631939623999	97.02\\
13.648077976	97.04\\
13.663860090999	97.06\\
13.680032222999	97.08\\
13.696097354999	97.09\\
13.711947359999	97.11\\
13.728039571999	97.13\\
13.744047510999	97.15\\
13.759953026999	97.17\\
13.776007766999	97.19\\
13.791840941999	97.21\\
13.807911361	97.23\\
13.824061562999	97.25\\
13.839888982	97.27\\
13.855969465	97.28\\
13.872035288	97.3\\
13.887997795999	97.32\\
13.904109422999	97.34\\
13.919990595999	97.36\\
13.935903421	97.38\\
13.951894644	97.4\\
13.968047545999	97.42\\
13.983960389999	97.44\\
14.000027504999	97.46\\
14.01762777	97.47\\
14.032759472999	97.49\\
14.048374838	97.5\\
14.063800805	97.51\\
14.079998809999	97.52\\
14.096081339998	97.54\\
14.111912038999	97.55\\
14.128117726	97.56\\
14.144014474999	97.57\\
14.159938832	97.59\\
14.176042758	97.6\\
14.192012026999	97.61\\
14.208009789	97.63\\
14.224067612999	97.64\\
14.239805932	97.65\\
14.255826628999	97.66\\
14.272077641	97.68\\
14.288024243	97.69\\
14.304021734999	97.7\\
14.319999465999	97.71\\
14.336157849999	97.73\\
14.352090108999	97.74\\
14.368103097	97.75\\
14.384006842999	97.76\\
14.400081658999	97.78\\
14.416192245999	97.79\\
14.432196137999	97.8\\
14.447979796	97.82\\
14.46409117	97.83\\
14.480046634999	97.84\\
14.496136177	97.85\\
14.513334149999	97.87\\
14.528424364999	97.88\\
14.543815893999	97.89\\
14.560003601999	97.9\\
14.576045335999	97.92\\
14.592015085	97.93\\
14.607956404999	97.94\\
14.624059075999	97.95\\
14.639973939	97.97\\
14.656213887999	97.98\\
14.672025283999	97.99\\
14.688025052	98.01\\
14.704114061	98.02\\
14.720064633	98.03\\
14.735987156	98.04\\
14.751898318	98.06\\
14.767947569999	98.07\\
14.783993743	98.08\\
14.800090361999	98.09\\
14.818686234001	98.11\\
14.833766929999	98.12\\
14.848820121	98.13\\
14.863988592	98.15\\
14.880014509	98.16\\
14.89607542	98.17\\
14.911843433	98.18\\
14.930134549999	98.2\\
14.945579429	98.21\\
14.959932290999	98.22\\
14.976137776999	98.23\\
14.992018078999	98.25\\
15.009683929	98.26\\
15.024846948	98.27\\
15.039952969999	98.28\\
15.0559247	98.3\\
15.072010814999	98.31\\
15.088096781	98.32\\
15.103931924999	98.34\\
15.120077323	98.35\\
15.136085877999	98.36\\
15.151923712999	98.37\\
15.168080642999	98.39\\
15.183992532999	98.4\\
15.199963401	98.41\\
15.218365478999	98.43\\
15.233532024	98.44\\
15.248764099	98.45\\
15.264094413	98.46\\
15.280030842999	98.47\\
15.296066682999	98.49\\
15.312054043001	98.5\\
15.328062259	98.51\\
15.343987944999	98.53\\
15.360045861999	98.54\\
15.376065293999	98.55\\
15.392045958999	98.56\\
15.407944021	98.58\\
15.423929495	98.59\\
15.43991654	98.6\\
15.455970223999	98.61\\
15.472050248999	98.63\\
15.488056330999	98.64\\
15.504070401	98.65\\
15.519940671999	98.67\\
15.535891090999	98.68\\
15.552001083999	98.69\\
15.568010676999	98.7\\
15.584000333999	98.72\\
15.600053601	98.73\\
15.616059977999	98.74\\
15.631947607	98.75\\
15.648108472999	98.77\\
15.664005153999	98.78\\
15.679953207	98.79\\
15.6960606	98.8\\
15.712200088999	98.82\\
15.727955030998	98.83\\
15.749656115	98.85\\
15.760379321999	98.86\\
15.775940858	98.87\\
15.791941946999	98.88\\
15.808003819	98.89\\
15.824034679999	98.91\\
15.840042271999	98.92\\
15.856050964998	98.93\\
15.872229271999	98.94\\
15.888012106999	98.96\\
15.904064659	98.97\\
15.91995492	98.98\\
15.935977149999	98.99\\
15.951805555999	99.01\\
15.968026054	99.02\\
15.984002198999	99.03\\
16.000036915	99.05\\
16.017714374	99.06\\
16.036241287999	99.06\\
16.048227626999	99.07\\
16.064003908	99.07\\
16.080095318	99.08\\
16.096034948999	99.09\\
16.111989300999	99.09\\
16.127873466999	99.1\\
16.144027335999	99.11\\
16.16175697	99.11\\
16.176962144	99.12\\
16.192398837999	99.13\\
16.208061165999	99.13\\
16.224134149	99.14\\
16.240054608	99.14\\
16.256070479999	99.15\\
16.271997082	99.16\\
16.287989472999	99.16\\
16.30409693	99.17\\
16.320069710999	99.18\\
16.335988991	99.18\\
16.352005967999	99.19\\
16.368094012	99.2\\
16.384029422	99.2\\
16.400677424999	99.21\\
16.415895880999	99.21\\
16.431966736999	99.22\\
16.447934663999	99.23\\
16.464034167999	99.23\\
16.479996660999	99.24\\
16.49597695	99.25\\
16.512073690999	99.25\\
16.528041069999	99.26\\
16.543909139999	99.27\\
16.559946457999	99.27\\
16.576171701001	99.28\\
16.591989234999	99.28\\
16.608101278	99.29\\
16.624508303999	99.3\\
16.639934724999	99.3\\
16.656084604999	99.31\\
16.672056818999	99.32\\
16.687870764	99.32\\
16.703915946	99.33\\
16.719956606	99.34\\
16.735844821	99.34\\
16.754045167999	99.35\\
16.769098644999	99.35\\
16.784368763	99.36\\
16.800088105	99.37\\
16.816438134999	99.37\\
16.831982347999	99.38\\
16.850227987999	99.39\\
16.865423618	99.39\\
16.880702198	99.4\\
16.896017358	99.4\\
16.912017189999	99.41\\
16.927937043999	99.42\\
16.943984375	99.42\\
16.961156573999	99.43\\
16.976113892999	99.44\\
16.991970229999	99.44\\
17.008444463	99.45\\
17.024075033999	99.46\\
17.039835035	99.46\\
17.056070679999	99.47\\
17.072049715999	99.47\\
17.087850441999	99.48\\
17.104057568999	99.49\\
17.119988880999	99.49\\
17.136113286	99.5\\
17.151862217	99.51\\
17.167957228999	99.51\\
17.18398495	99.52\\
17.200104649	99.53\\
17.218550832999	99.53\\
17.232926274	99.54\\
17.248137278	99.54\\
17.26387245	99.55\\
17.279970776999	99.56\\
17.296061349999	99.56\\
17.311988568	99.57\\
17.327872132999	99.58\\
17.344109498	99.58\\
17.360939495999	99.59\\
17.376223236999	99.6\\
17.391893642999	99.6\\
17.408056335	99.61\\
17.424313682	99.61\\
17.440001361	99.62\\
17.455967295	99.63\\
17.472023309999	99.63\\
17.487967858	99.64\\
17.504971807	99.65\\
17.520140914999	99.65\\
17.536103998	99.66\\
17.551967871999	99.66\\
17.567978700999	99.67\\
17.584065223999	99.68\\
17.60011293	99.68\\
17.616333253999	99.69\\
17.631884960999	99.7\\
17.647969191999	99.7\\
17.664987641	99.71\\
17.680100542	99.72\\
17.696044604	99.72\\
17.711796486	99.73\\
17.730134252	99.74\\
17.745402875	99.74\\
17.760505993	99.75\\
17.776049182	99.75\\
17.791998161999	99.76\\
17.808054979	99.77\\
17.824025455	99.77\\
17.840003354	99.78\\
17.856105553999	99.79\\
17.872035170999	99.79\\
17.887997814999	99.8\\
17.904001475999	99.8\\
17.920053290999	99.81\\
17.935916165999	99.82\\
17.952010519999	99.82\\
17.967968245	99.83\\
17.983976764	99.84\\
18.000092871999	99.84\\
18.017608078	99.85\\
18.032729605999	99.86\\
18.047910663	99.86\\
18.064054511999	99.87\\
18.080047638999	99.87\\
18.096095526	99.88\\
18.112035257	99.89\\
18.128437739	99.89\\
18.144065436999	99.9\\
18.159924115	99.91\\
18.176005127999	99.91\\
18.191979129999	99.92\\
18.208025093	99.92\\
18.224024293999	99.93\\
18.240000453	99.94\\
18.255998251	99.94\\
18.272025358999	99.95\\
18.288359305999	99.96\\
18.304005299	99.96\\
18.319960547999	99.97\\
18.336018537999	99.98\\
18.351802871999	99.98\\
18.369980340999	99.99\\
18.385104022999	100\\
18.400257116	100\\
18.416241009	100.01\\
18.432067687999	100.01\\
18.447955486999	100.02\\
18.463991415999	100.03\\
18.480091927	100.03\\
18.496088361999	100.04\\
18.512036934999	100.05\\
18.527898818	100.05\\
18.544323797999	100.06\\
18.560143847999	100.06\\
18.576016829999	100.07\\
18.592041628999	100.08\\
18.607951371	100.08\\
18.624014828999	100.09\\
18.641157909999	100.1\\
18.656310766	100.1\\
18.672027071999	100.11\\
18.687989443999	100.12\\
18.703933064999	100.12\\
18.719808654999	100.13\\
18.736029138999	100.13\\
18.751922731999	100.14\\
18.767956998	100.15\\
18.783964884999	100.15\\
18.799903131999	100.16\\
18.818292355	100.17\\
18.833377278	100.17\\
18.848546647999	100.18\\
18.864017621999	100.18\\
18.880089888999	100.19\\
18.896087050999	100.2\\
18.913062920999	100.2\\
18.928432550999	100.21\\
18.944024673	100.22\\
18.960022436999	100.22\\
18.976051463999	100.23\\
18.992014304	100.24\\
19.009559223999	100.24\\
19.024605875999	100.25\\
19.040156094999	100.25\\
19.05605804	100.26\\
19.072099803	100.27\\
19.087958116	100.27\\
19.10404495	100.28\\
19.120145246	100.29\\
19.135979366999	100.29\\
19.152084318	100.3\\
19.167932496999	100.31\\
19.184005887	100.31\\
19.200006661999	100.32\\
19.218521771999	100.33\\
19.234051876	100.33\\
19.249188941	100.34\\
19.264575524	100.34\\
19.279936205999	100.35\\
19.296040325999	100.36\\
19.312866729	100.36\\
19.328141177999	100.37\\
19.344189149999	100.37\\
19.360711006	100.38\\
19.376121715999	100.39\\
19.391994101999	100.39\\
19.408076809	100.4\\
19.424083699	100.41\\
19.439815111999	100.41\\
19.456020111	100.42\\
19.471967913999	100.43\\
19.488116641999	100.43\\
19.504021713999	100.44\\
19.519999397	100.44\\
19.536005307	100.45\\
19.551803008	100.46\\
19.570162304999	100.46\\
19.58536714	100.47\\
19.600403257999	100.48\\
19.616290003999	100.48\\
19.631994267999	100.49\\
19.650442038999	100.5\\
19.665641326999	100.5\\
19.680888878999	100.51\\
19.696128528999	100.51\\
19.711887293999	100.52\\
19.72800738	100.53\\
19.744040668999	100.53\\
19.759878538999	100.54\\
19.77604073	100.55\\
19.791939476	100.55\\
19.808081894	100.56\\
19.824061521	100.57\\
19.840782703999	100.57\\
19.856002352999	100.58\\
19.872040502999	100.58\\
19.888065045999	100.59\\
19.903949894	100.6\\
19.919957668999	100.6\\
19.935938593	100.61\\
19.95212016	100.62\\
19.967966250999	100.62\\
19.984028040999	100.63\\
20.00004834	100.63\\
20.017641960999	100.64\\
20.032894441	100.65\\
20.048139604999	100.65\\
20.064044583999	100.66\\
20.080014207999	100.67\\
20.096028047999	100.67\\
20.112028699999	100.68\\
20.128437771	100.69\\
20.146166155998	100.69\\
20.161859586999	100.7\\
20.177214057	100.71\\
20.192502436	100.71\\
20.208031220999	100.72\\
20.224042435	100.72\\
20.240277332999	100.73\\
20.256384235999	100.74\\
20.271775258	100.74\\
20.287983995	100.75\\
20.304063226999	100.76\\
20.320027724999	100.76\\
20.336073016	100.77\\
20.351907595999	100.77\\
20.367952783999	100.78\\
20.384081814999	100.79\\
20.400118635999	100.79\\
20.416171264	100.8\\
20.432003085999	100.81\\
20.448067653999	100.81\\
20.464095575999	100.82\\
20.479898357999	100.83\\
20.495931497998	100.83\\
20.512762644999	100.84\\
20.529310129999	100.85\\
20.544741265	100.85\\
20.561385144999	100.86\\
20.576546880999	100.86\\
20.59197883	100.87\\
20.607787805	100.88\\
20.626082811	100.88\\
20.641620573999	100.89\\
20.656745405998	100.9\\
20.671929663	100.9\\
20.688012349999	100.91\\
20.704016764	100.91\\
20.720184026999	100.92\\
20.735902930999	100.93\\
20.751816108	100.93\\
20.770156674999	100.94\\
20.785534557999	100.95\\
20.800777472	100.95\\
20.816402253999	100.96\\
20.831975321	100.96\\
20.850868717	100.97\\
20.866075125999	100.98\\
20.881223522999	100.98\\
20.896411424999	100.99\\
20.911979493	101\\
20.928058024	101\\
20.944107491	101.01\\
20.961958420999	101.02\\
20.977079430999	101.02\\
20.992237653999	101.03\\
21.009678505	101.03\\
21.024918534999	101.03\\
21.040069728999	101.03\\
21.055969424	101.03\\
21.072009136999	101.03\\
21.087996591999	101.03\\
21.103891936999	101.03\\
21.120091693	101.03\\
21.136024282999	101.03\\
21.152154441	101.03\\
21.167949866999	101.03\\
21.184028424	101.03\\
21.200040517999	101.03\\
21.218484944999	101.03\\
21.233770583999	101.03\\
21.248996484999	101.03\\
21.264097038	101.03\\
21.280010304999	101.03\\
21.296019988	101.03\\
21.312119489	101.03\\
21.328972625999	101.03\\
21.344313097999	101.03\\
21.360530062999	101.03\\
21.376048639999	101.03\\
21.3920632	101.03\\
21.408327729	101.03\\
21.424097410999	101.03\\
21.439811637999	101.03\\
21.458225741	101.03\\
21.473409147999	101.03\\
21.488686934	101.03\\
21.504037724999	101.03\\
21.519812976999	101.03\\
21.536064527	101.03\\
21.552026212999	101.03\\
21.568087286	101.03\\
21.584008703999	101.03\\
21.599973995	101.03\\
21.616199532999	101.03\\
21.631992326999	101.03\\
21.647925049999	101.03\\
21.664128536999	101.03\\
21.680005755999	101.03\\
21.696141075999	101.03\\
21.71227551	101.03\\
21.727938005999	101.03\\
21.743976130999	101.03\\
21.759841598999	101.03\\
21.776020677999	101.03\\
21.792006073999	101.03\\
21.808005916	101.03\\
21.823971061	101.03\\
21.84032695	101.03\\
21.856152545999	101.03\\
21.872090766	101.03\\
21.888006236999	101.03\\
21.904080628999	101.03\\
21.919994242999	101.03\\
21.936112065999	101.03\\
21.952119717	101.03\\
21.967831791	101.03\\
21.983796529999	101.03\\
22.000046335	101.03\\
22.017892427999	101.03\\
22.032996535999	101.03\\
22.048279286999	101.03\\
22.063965491999	101.03\\
22.080025542999	101.03\\
22.096051779999	101.03\\
22.114068049999	101.03\\
22.129231363	101.03\\
22.144664528999	101.03\\
22.160777602	101.03\\
22.175846785	101.03\\
22.191987843999	101.03\\
22.208087086999	101.03\\
22.223947483999	101.03\\
22.240002924999	101.03\\
22.255857957	101.03\\
22.274043565	101.03\\
22.289192352999	101.03\\
22.304644428999	101.03\\
22.319955518999	101.03\\
22.335840836999	101.03\\
22.351799123	101.03\\
22.370235443	101.03\\
22.385457853999	101.03\\
22.400744371999	101.03\\
22.415939241999	101.03\\
22.432020182999	101.03\\
22.448039111999	101.03\\
22.464018548	101.03\\
22.480060899999	101.03\\
22.495911836	101.03\\
22.512086975999	101.03\\
22.528096319999	101.03\\
22.544011401	101.03\\
22.560271932	101.03\\
22.576138734999	101.03\\
22.592012430999	101.03\\
22.608034064	101.03\\
22.624046439999	101.03\\
22.640035557999	101.03\\
22.655833617999	101.03\\
22.672091670999	101.03\\
22.687805594	101.03\\
22.706110389999	101.03\\
22.721300647	101.03\\
22.736400113	101.03\\
22.751806778999	101.03\\
22.767889709999	101.03\\
22.783967040999	101.03\\
22.800044716	101.03\\
22.816156109999	101.03\\
22.832001915	101.03\\
22.848197385999	101.03\\
22.863995946999	101.03\\
22.880054687999	101.03\\
22.895831674	101.03\\
22.912027811999	101.03\\
22.928054617	101.03\\
22.944008249999	101.03\\
22.960436774999	101.03\\
22.976033918999	101.03\\
22.992044320999	101.03\\
23.009644358	101.03\\
23.024791621	101.03\\
23.04002891	101.03\\
23.056107055999	101.03\\
23.071936799999	101.03\\
23.087983591999	101.03\\
23.104077121999	101.03\\
23.120021081	101.03\\
23.136043286	101.03\\
23.152055140999	101.03\\
23.167936080999	101.03\\
23.184027920999	101.03\\
23.199809842	101.03\\
23.218301541	101.03\\
23.23354756	101.03\\
23.248878561999	101.03\\
23.264282998999	101.03\\
23.279971154	101.03\\
23.295921705999	101.03\\
23.312035597999	101.03\\
23.330370118	101.03\\
23.345557036999	101.03\\
23.360624596999	101.03\\
23.376032748999	101.03\\
23.392078293001	101.03\\
23.408611537999	101.03\\
23.424035844999	101.03\\
23.440038363999	101.03\\
23.456019759	101.03\\
23.47196233	101.03\\
23.487986053999	101.03\\
23.503885189999	101.03\\
23.520476339998	101.03\\
23.536225986999	101.03\\
23.551949435	101.03\\
23.5679244	101.03\\
23.583994146	101.03\\
23.600032581	101.03\\
23.618439917	101.03\\
23.63345641	101.03\\
23.648461132999	101.03\\
23.663832750999	101.03\\
23.682126788999	101.03\\
23.697317132	101.03\\
23.712634066	101.03\\
23.728157839998	101.03\\
23.743952887999	101.03\\
23.76012646	101.03\\
23.77602999	101.03\\
23.792002029	101.03\\
23.807770458999	101.03\\
23.823996397	101.03\\
23.840065582999	101.03\\
23.856090260999	101.03\\
23.871993325	101.03\\
23.88798767	101.03\\
23.904057606	101.03\\
23.919903682999	101.03\\
23.935984267999	101.03\\
23.952839953999	101.03\\
23.968112821999	101.03\\
23.983897587	101.03\\
23.999915491999	101.03\\
24.017403472999	101.03\\
24.032488418999	101.03\\
24.048035609001	101.03\\
24.063997981999	101.03\\
24.080026689999	101.03\\
24.096017502	101.03\\
24.112089522999	101.03\\
24.128013356999	101.03\\
24.143967442999	101.03\\
24.160037911999	101.03\\
24.175911377999	101.03\\
24.192174801999	101.03\\
24.209293538999	101.03\\
24.224682762999	101.03\\
24.240015396	101.03\\
24.256096637	101.03\\
24.272026296	101.03\\
24.287995549999	101.03\\
24.304026877999	101.03\\
24.319988194	101.03\\
24.335996676999	101.03\\
24.351892871999	101.03\\
24.368071731999	101.03\\
24.383966055999	101.03\\
24.399941734001	101.03\\
24.416117033999	101.03\\
24.432026167999	101.03\\
24.4504257	101.03\\
24.465698028	101.03\\
24.480770403999	101.03\\
24.495848377	101.03\\
24.512194541	101.03\\
24.527991691999	101.03\\
24.544030035	101.03\\
24.561394819	101.03\\
24.576608302	101.03\\
24.591954613	101.03\\
24.607889965	101.03\\
24.624041703999	101.03\\
24.639914759	101.03\\
24.655836328	101.03\\
24.671812121999	101.03\\
24.687927146	101.03\\
24.704045512999	101.03\\
24.720072479	101.03\\
24.736041239999	101.03\\
24.752029884999	101.03\\
24.768060098	101.03\\
24.784000521	101.03\\
24.799963727999	101.03\\
24.816332906	101.03\\
24.832059759999	101.03\\
24.847998899999	101.03\\
24.863816246	101.03\\
24.882083932999	101.03\\
24.897357076001	101.03\\
24.912578415999	101.03\\
24.928060479999	101.03\\
24.943953123999	101.03\\
24.959849656999	101.03\\
24.976015321	101.03\\
24.991974917999	101.03\\
25.010011955	101.03\\
25.025357847	101.03\\
25.039796722999	101.03\\
25.058087529999	101.03\\
25.073255696999	101.03\\
25.088601151	101.03\\
25.104034925999	101.03\\
25.119905073	101.03\\
25.135914062999	101.03\\
25.1519887	101.03\\
25.168049531	101.03\\
25.184032029	101.03\\
25.199915531999	101.03\\
25.218453524999	101.03\\
25.233596637999	101.03\\
25.248978712999	101.03\\
25.264242847999	101.03\\
25.280009686999	101.03\\
25.296123177999	101.03\\
25.312078269	101.03\\
25.328007391	101.03\\
25.343991212999	101.03\\
25.360124222	101.03\\
25.376128494	101.03\\
25.392057446	101.03\\
25.407967213999	101.03\\
25.423944337	101.03\\
25.439977936	101.03\\
25.456144472	101.03\\
25.472117397999	101.03\\
25.487884724	101.03\\
25.503966552	101.03\\
25.519833832999	101.03\\
25.535970402	101.03\\
25.551793687999	101.03\\
25.570161671	101.03\\
25.585351998999	101.03\\
25.601040411	101.03\\
25.616253524	101.03\\
25.632076973999	101.03\\
25.647999928999	101.03\\
25.66400279	101.03\\
25.680072130999	101.03\\
25.696064793001	101.03\\
25.713143643999	101.03\\
25.728432281	101.03\\
25.744497621999	101.03\\
25.759963179	101.03\\
25.776063694999	101.03\\
25.792009630999	101.03\\
25.808648127999	101.03\\
25.823953062999	101.03\\
25.840041206	101.03\\
25.855989070999	101.03\\
25.871928976999	101.03\\
25.888029149	101.03\\
25.904050599	101.03\\
25.919911502999	101.03\\
25.936024457	101.03\\
25.951878661999	101.03\\
25.968013427	101.03\\
25.984099679	101.03\\
26.000004663999	101.03\\
26.01745681	101.03\\
26.032585075	101.03\\
26.048102274	101.03\\
26.063913137999	101.03\\
26.08009312	101.03\\
26.09607226	101.03\\
26.111959319999	101.03\\
26.12809096	101.03\\
26.143954153	101.03\\
26.160039583999	101.03\\
26.176043016999	101.03\\
26.192122079999	101.03\\
26.208000114	101.03\\
26.223834489	101.03\\
26.239858366999	101.03\\
26.256040995999	101.03\\
26.271976086999	101.03\\
26.287974957	101.03\\
26.30399363	101.03\\
26.320014668001	101.03\\
26.335974745999	101.03\\
26.352047889999	101.03\\
26.369256455999	101.03\\
26.384204746	101.03\\
26.399809000999	101.03\\
26.415812274999	101.03\\
26.431807338	101.03\\
26.449915657999	101.03\\
26.464988389	101.03\\
26.480123748999	101.03\\
26.495930116	101.03\\
26.511928387	101.03\\
26.528284756	101.03\\
26.544653785	101.03\\
26.560053010999	101.03\\
26.575926198999	101.03\\
26.592006381999	101.03\\
26.608066807	101.03\\
26.624103426999	101.03\\
26.640011507999	101.03\\
26.655891413	101.03\\
26.671880047	101.03\\
26.688018094	101.03\\
26.706433752	101.03\\
26.721412093	101.03\\
26.736503587	101.03\\
26.753763302	101.03\\
26.769000694999	101.03\\
26.784579610999	101.03\\
26.799858932999	101.03\\
26.81583364	101.03\\
26.834136767999	101.03\\
26.849326097999	101.03\\
26.864658337	101.03\\
26.879997305999	101.03\\
26.895816856999	101.03\\
26.911920227999	101.03\\
26.930104788999	101.03\\
26.945445280998	101.03\\
26.960692163999	101.03\\
26.976066391999	101.03\\
26.992006454	101.03\\
27.00928259	101.03\\
27.025011603999	101.03\\
27.041014884	101.03\\
27.056137533	101.03\\
27.072001682999	101.03\\
27.087876480999	101.03\\
27.103916004999	101.03\\
27.11987604	101.03\\
27.136093988	101.03\\
27.151903885	101.03\\
27.167904130999	101.03\\
27.183846619	101.03\\
27.200022088999	101.03\\
27.21861688	101.03\\
27.233922126999	101.03\\
27.249259393999	101.03\\
27.264488764	101.03\\
27.280004956	101.03\\
27.296072870999	101.03\\
27.311963828999	101.03\\
27.328096547999	101.03\\
27.344028936	101.03\\
27.360786153	101.03\\
27.376436936	101.03\\
27.392853193	101.03\\
27.409092545	101.03\\
27.424289833999	101.03\\
27.440046670999	101.03\\
27.456065857999	101.03\\
27.472022301999	101.03\\
27.487870021999	101.03\\
27.503948184999	101.03\\
27.520040772999	101.03\\
27.536032682999	101.03\\
27.551978707999	101.03\\
27.567924734999	101.03\\
27.583973066999	101.03\\
27.599834109999	101.03\\
27.618217526999	101.03\\
27.63334012	101.03\\
27.648542666	101.03\\
27.663992932999	101.03\\
27.680022888	101.03\\
27.695991559999	101.03\\
27.712019220999	101.03\\
27.728178154999	101.03\\
27.744952668001	101.03\\
27.760192824	101.03\\
27.776060229999	101.03\\
27.791850636999	101.03\\
27.808222845999	101.03\\
27.823971895999	101.03\\
27.839940097	101.03\\
27.856062320999	101.03\\
27.871962998999	101.03\\
27.888035467	101.03\\
27.903905996999	101.03\\
27.920036457	101.03\\
27.935924446999	101.03\\
27.951951037999	101.03\\
27.968010695999	101.03\\
27.984000623999	101.03\\
27.999864564	101.03\\
28.017386875	101.03\\
28.032605016	101.03\\
28.048026557	101.03\\
28.063902336999	101.03\\
28.0800032	101.03\\
28.095857908	101.03\\
28.112071300999	101.03\\
28.128089992	101.03\\
28.147021154999	101.03\\
28.16002725	101.03\\
28.176018952999	101.03\\
28.191873623999	101.03\\
28.207861529999	101.03\\
28.224049238	101.03\\
28.240036065999	101.03\\
28.256065774	101.03\\
28.272010255	101.03\\
28.288036064999	101.03\\
28.304027759	101.03\\
28.319931643999	101.03\\
28.335915505999	101.03\\
28.352044259	101.03\\
28.368237206	101.03\\
28.383989107	101.03\\
28.400579637999	101.03\\
28.416439077999	101.03\\
28.432019781	101.03\\
28.448121495	101.03\\
28.463931847999	101.03\\
28.482325506999	101.03\\
28.49767795	101.03\\
28.51289858	101.03\\
28.528113312999	101.03\\
28.546542045	101.03\\
28.561720576001	101.03\\
28.577191569999	101.03\\
28.592530754999	101.03\\
28.608006354	101.03\\
28.624036399999	101.03\\
28.639836196999	101.03\\
28.656083893999	101.03\\
28.672130903999	101.03\\
28.687805818999	101.03\\
28.703838847999	101.03\\
28.722195644	101.03\\
28.737772177	101.03\\
28.752962424	101.03\\
28.768110438999	101.03\\
28.783996379999	101.03\\
28.799898618	101.03\\
28.818277137	101.03\\
28.833375696999	101.03\\
28.848615005	101.03\\
28.864066373999	101.03\\
28.879980568999	101.03\\
28.896044897	101.03\\
28.912038075999	101.03\\
28.928115693999	101.03\\
28.943869287999	101.03\\
28.960036073999	101.03\\
28.975941880999	101.03\\
28.992024092999	101.03\\
29.009701312	101.03\\
29.024949731	101.03\\
29.040258187999	101.03\\
29.056089382999	101.03\\
29.071998651	101.03\\
29.087875345999	101.03\\
29.104045307999	101.03\\
29.119958688999	101.03\\
29.136120975	101.03\\
29.151911679	101.03\\
29.168004437999	101.03\\
29.184032	101.03\\
29.199869116999	101.03\\
29.215911366	101.03\\
29.231922654999	101.03\\
29.248068392999	101.03\\
29.263915062999	101.03\\
29.279993488999	101.03\\
29.295814186	101.03\\
29.312018281	101.03\\
29.328003308999	101.03\\
29.344118524999	101.03\\
29.360614886999	101.03\\
29.376116373999	101.03\\
29.391944856	101.03\\
29.409216522999	101.03\\
29.424332059999	101.03\\
29.44000426	101.03\\
29.45593188	101.03\\
29.471969741999	101.03\\
29.487888185999	101.03\\
29.50401777	101.03\\
29.519995542999	101.03\\
29.535982668999	101.03\\
29.551924595	101.03\\
29.568024905998	101.03\\
29.584010821999	101.03\\
29.599913516	101.03\\
29.616156883	101.03\\
29.631935411999	101.03\\
29.648067913999	101.03\\
29.663980659	101.03\\
29.680067713999	101.03\\
29.696037495999	101.03\\
29.711991477999	101.03\\
29.727807703999	101.03\\
29.744155169999	101.03\\
29.761667486	101.03\\
29.777244561	101.03\\
29.792848795999	101.03\\
29.809831533	101.03\\
29.824019627999	101.03\\
29.839989637999	101.03\\
29.856052793999	101.03\\
29.871947464998	101.03\\
29.887911785	101.03\\
29.904040720999	101.03\\
29.920088121999	101.03\\
29.936069042999	101.03\\
29.952023175999	101.03\\
29.968041901	101.03\\
29.983961417	101.03\\
29.999932408999	101.03\\
30.018131226999	101.03\\
30.033302099999	101.03\\
30.048515130999	101.03\\
30.064029953999	101.03\\
30.079909252	101.03\\
30.095999679	101.03\\
30.112024469	101.03\\
30.128004141	101.03\\
30.144302013999	101.03\\
30.161657255999	101.03\\
30.177257201999	101.03\\
30.192558088999	101.03\\
30.209296672	101.03\\
30.224025436999	101.03\\
30.23999635	101.03\\
30.256065877	101.03\\
30.271931057999	101.03\\
30.288054441	101.03\\
30.304031977999	101.03\\
30.320061528	101.03\\
30.336006597	101.03\\
30.352068189999	101.03\\
30.368254389999	101.03\\
30.383892807999	101.03\\
30.399847247998	101.03\\
30.416052034	101.03\\
30.432018036	101.03\\
30.447833604999	101.03\\
30.464028946999	101.03\\
30.479998049	101.03\\
30.495841373999	101.03\\
30.514215509	101.03\\
30.529441656999	101.03\\
30.544562526	101.03\\
30.560027667999	101.03\\
30.576000509	101.03\\
30.591996116999	101.03\\
30.60783938	101.03\\
30.623812320999	101.03\\
30.640000993	101.03\\
30.656053524999	101.03\\
30.671924108999	101.03\\
30.687834593999	101.03\\
30.703833248999	101.03\\
30.720123783999	101.03\\
30.736113736999	101.03\\
30.751981657	101.03\\
30.768167866	101.03\\
30.783993944	101.03\\
30.799923427999	101.03\\
30.816254937	101.03\\
30.832012344999	101.03\\
30.847955889999	101.03\\
30.863978849999	101.03\\
30.880056184	101.03\\
30.896081288999	101.03\\
30.911997205999	101.03\\
30.92807466	101.03\\
30.946386759999	101.03\\
30.961531592	101.03\\
30.977435795	101.03\\
30.992750108999	101.03\\
31.008809943	101.03\\
31.024028242999	101.03\\
31.040056474999	101.03\\
31.056069580999	101.03\\
31.072012732	101.03\\
31.087813897999	101.03\\
31.106010602999	101.03\\
31.121108784	101.03\\
31.136256408999	101.03\\
31.151981270999	101.03\\
31.168057987999	101.03\\
31.183998542	101.03\\
31.199923112999	101.03\\
31.218223772999	101.03\\
31.233407883999	101.03\\
31.248763533	101.03\\
31.264167170999	101.03\\
31.280464886999	101.03\\
31.296340892999	101.03\\
31.312199793001	101.03\\
31.328043616	101.03\\
31.346707125	101.03\\
31.361862075999	101.03\\
31.377163880999	101.03\\
31.392444743999	101.03\\
31.408093554	101.03\\
31.423933408999	101.03\\
31.439960094	101.03\\
31.456064193	101.03\\
31.472083679999	101.03\\
31.487935262	101.03\\
31.504078309	101.03\\
31.519898901999	101.03\\
31.536061762999	101.03\\
31.551947861999	101.03\\
31.568095820999	101.03\\
31.584043031999	101.03\\
31.599898200999	101.03\\
31.615918139999	101.03\\
31.632020832999	101.03\\
31.648041368	101.03\\
31.665071155998	101.03\\
31.680127320999	101.03\\
31.695939485999	101.03\\
31.712037707	101.03\\
31.728077986	101.03\\
31.744104028999	101.03\\
31.759973208999	101.03\\
31.776000821	101.03\\
31.792026932999	101.03\\
31.807921136999	101.03\\
31.823995889	101.03\\
31.83981299	101.03\\
31.857956719999	101.03\\
31.873210821999	101.03\\
31.888430646999	101.03\\
31.904206877	101.03\\
31.919887082999	101.03\\
31.935931539	101.03\\
31.951969262999	101.03\\
31.968063849999	101.03\\
31.984031955999	101.03\\
32.000068338999	101.03\\
32.018539915999	101.03\\
32.033699745999	101.03\\
32.048876457999	101.03\\
32.064165635999	101.03\\
32.079997420999	101.03\\
32.096043366	101.03\\
32.112043257	101.03\\
32.128003029999	101.03\\
32.144041307999	101.03\\
32.160145213999	101.03\\
32.176056694999	101.03\\
32.192043672	101.03\\
32.207789915999	101.03\\
32.224078493	101.03\\
32.239931830999	101.03\\
32.256167863	101.03\\
32.272022715999	101.03\\
32.287866983	101.03\\
32.304062342	101.03\\
32.319948858999	101.03\\
32.335970075	101.03\\
32.351984821999	101.03\\
32.368227412999	101.03\\
32.384047617999	101.03\\
32.399974279	101.03\\
32.415805491999	101.03\\
32.434264332	101.03\\
32.449953396	101.03\\
32.465196782999	101.03\\
32.480379316	101.03\\
32.496070215999	101.03\\
32.511996440999	101.03\\
32.527955082	101.03\\
32.543929225	101.03\\
32.559920082	101.03\\
32.575949057999	101.03\\
32.592025668999	101.03\\
32.607986759	101.03\\
32.623991125999	101.03\\
32.639995976	101.03\\
32.656095220999	101.03\\
32.672015750999	101.03\\
32.687997453	101.03\\
32.704052680999	101.03\\
32.719974526999	101.03\\
32.736014363999	101.03\\
32.752016108	101.03\\
32.768261995999	101.03\\
32.784058254999	101.03\\
32.799965509	101.03\\
32.816104535	101.03\\
32.83200579	101.03\\
32.848018386999	101.03\\
32.863885141999	101.03\\
32.880101962	101.03\\
32.895841035	101.03\\
32.913907736999	101.03\\
32.929049047	101.03\\
32.944228996999	101.03\\
32.959862404	101.03\\
32.976023908	101.03\\
32.991857002999	101.03\\
33.009244781999	101.03\\
33.024422667	101.03\\
33.040013121999	101.03\\
33.055953964998	101.03\\
33.071884991999	101.03\\
33.087998201999	101.03\\
33.104080874999	101.03\\
33.120004927999	101.03\\
33.136093384	101.03\\
33.152026397999	101.03\\
33.168003128	101.03\\
33.184059108	101.03\\
33.199866684999	101.03\\
33.216057719	101.03\\
33.232042361	101.03\\
33.248054609999	101.03\\
33.263776510999	101.03\\
33.282116139999	101.03\\
33.297292479999	101.03\\
33.312542604999	101.03\\
33.327923371	101.03\\
33.344980838999	101.03\\
33.360323241	101.03\\
33.376079547	101.03\\
33.392028913	101.03\\
33.408092502	101.03\\
33.424285233	101.03\\
33.439990399999	101.03\\
33.456195564999	101.03\\
33.472067137999	101.03\\
33.487939349999	101.03\\
33.504001993999	101.03\\
33.520031059	101.03\\
33.536077772999	101.03\\
33.552030149	101.03\\
33.568069731	101.03\\
33.584012694999	101.03\\
33.599935415	101.03\\
33.616270104999	101.03\\
33.632107934999	101.03\\
33.648032311999	101.03\\
33.664011061	101.03\\
33.679931219	101.03\\
33.696085812	101.03\\
33.711968543999	101.03\\
33.727970361999	101.03\\
33.744093122998	101.03\\
33.760019889999	101.03\\
33.778531274	101.03\\
33.793520005999	101.03\\
33.809779422999	101.03\\
};
\end{axis}
\end{tikzpicture}%
}
      \caption{The orientation of the robot over time for
        $K_{\Psi}^R = 0.1 K_{\Psi, max}^R$}
      \label{fig:8_0.1_max}
    \end{figure}
  \end{minipage}
  \hfill
  \begin{minipage}{0.45\linewidth}
    \begin{figure}[H]
      \scalebox{0.6}{% This file was created by matlab2tikz.
%
%The latest updates can be retrieved from
%  http://www.mathworks.com/matlabcentral/fileexchange/22022-matlab2tikz-matlab2tikz
%where you can also make suggestions and rate matlab2tikz.
%
\definecolor{mycolor1}{rgb}{0.00000,0.44700,0.74100}%
%
\begin{tikzpicture}

\begin{axis}[%
width=4.133in,
height=3.26in,
at={(0.693in,0.44in)},
scale only axis,
xmin=25,
xmax=34,
xmajorgrids,
ymin=100,
ymax=102,
ymajorgrids,
axis background/.style={fill=white}
]
\addplot [color=mycolor1,solid,forget plot]
  table[row sep=crcr]{%
25.010011955	101.03\\
25.025357847	101.03\\
25.039796722999	101.03\\
25.058087529999	101.03\\
25.073255696999	101.03\\
25.088601151	101.03\\
25.104034925999	101.03\\
25.119905073	101.03\\
25.135914062999	101.03\\
25.1519887	101.03\\
25.168049531	101.03\\
25.184032029	101.03\\
25.199915531999	101.03\\
25.218453524999	101.03\\
25.233596637999	101.03\\
25.248978712999	101.03\\
25.264242847999	101.03\\
25.280009686999	101.03\\
25.296123177999	101.03\\
25.312078269	101.03\\
25.328007391	101.03\\
25.343991212999	101.03\\
25.360124222	101.03\\
25.376128494	101.03\\
25.392057446	101.03\\
25.407967213999	101.03\\
25.423944337	101.03\\
25.439977936	101.03\\
25.456144472	101.03\\
25.472117397999	101.03\\
25.487884724	101.03\\
25.503966552	101.03\\
25.519833832999	101.03\\
25.535970402	101.03\\
25.551793687999	101.03\\
25.570161671	101.03\\
25.585351998999	101.03\\
25.601040411	101.03\\
25.616253524	101.03\\
25.632076973999	101.03\\
25.647999928999	101.03\\
25.66400279	101.03\\
25.680072130999	101.03\\
25.696064793001	101.03\\
25.713143643999	101.03\\
25.728432281	101.03\\
25.744497621999	101.03\\
25.759963179	101.03\\
25.776063694999	101.03\\
25.792009630999	101.03\\
25.808648127999	101.03\\
25.823953062999	101.03\\
25.840041206	101.03\\
25.855989070999	101.03\\
25.871928976999	101.03\\
25.888029149	101.03\\
25.904050599	101.03\\
25.919911502999	101.03\\
25.936024457	101.03\\
25.951878661999	101.03\\
25.968013427	101.03\\
25.984099679	101.03\\
26.000004663999	101.03\\
26.01745681	101.03\\
26.032585075	101.03\\
26.048102274	101.03\\
26.063913137999	101.03\\
26.08009312	101.03\\
26.09607226	101.03\\
26.111959319999	101.03\\
26.12809096	101.03\\
26.143954153	101.03\\
26.160039583999	101.03\\
26.176043016999	101.03\\
26.192122079999	101.03\\
26.208000114	101.03\\
26.223834489	101.03\\
26.239858366999	101.03\\
26.256040995999	101.03\\
26.271976086999	101.03\\
26.287974957	101.03\\
26.30399363	101.03\\
26.320014668001	101.03\\
26.335974745999	101.03\\
26.352047889999	101.03\\
26.369256455999	101.03\\
26.384204746	101.03\\
26.399809000999	101.03\\
26.415812274999	101.03\\
26.431807338	101.03\\
26.449915657999	101.03\\
26.464988389	101.03\\
26.480123748999	101.03\\
26.495930116	101.03\\
26.511928387	101.03\\
26.528284756	101.03\\
26.544653785	101.03\\
26.560053010999	101.03\\
26.575926198999	101.03\\
26.592006381999	101.03\\
26.608066807	101.03\\
26.624103426999	101.03\\
26.640011507999	101.03\\
26.655891413	101.03\\
26.671880047	101.03\\
26.688018094	101.03\\
26.706433752	101.03\\
26.721412093	101.03\\
26.736503587	101.03\\
26.753763302	101.03\\
26.769000694999	101.03\\
26.784579610999	101.03\\
26.799858932999	101.03\\
26.81583364	101.03\\
26.834136767999	101.03\\
26.849326097999	101.03\\
26.864658337	101.03\\
26.879997305999	101.03\\
26.895816856999	101.03\\
26.911920227999	101.03\\
26.930104788999	101.03\\
26.945445280998	101.03\\
26.960692163999	101.03\\
26.976066391999	101.03\\
26.992006454	101.03\\
27.00928259	101.03\\
27.025011603999	101.03\\
27.041014884	101.03\\
27.056137533	101.03\\
27.072001682999	101.03\\
27.087876480999	101.03\\
27.103916004999	101.03\\
27.11987604	101.03\\
27.136093988	101.03\\
27.151903885	101.03\\
27.167904130999	101.03\\
27.183846619	101.03\\
27.200022088999	101.03\\
27.21861688	101.03\\
27.233922126999	101.03\\
27.249259393999	101.03\\
27.264488764	101.03\\
27.280004956	101.03\\
27.296072870999	101.03\\
27.311963828999	101.03\\
27.328096547999	101.03\\
27.344028936	101.03\\
27.360786153	101.03\\
27.376436936	101.03\\
27.392853193	101.03\\
27.409092545	101.03\\
27.424289833999	101.03\\
27.440046670999	101.03\\
27.456065857999	101.03\\
27.472022301999	101.03\\
27.487870021999	101.03\\
27.503948184999	101.03\\
27.520040772999	101.03\\
27.536032682999	101.03\\
27.551978707999	101.03\\
27.567924734999	101.03\\
27.583973066999	101.03\\
27.599834109999	101.03\\
27.618217526999	101.03\\
27.63334012	101.03\\
27.648542666	101.03\\
27.663992932999	101.03\\
27.680022888	101.03\\
27.695991559999	101.03\\
27.712019220999	101.03\\
27.728178154999	101.03\\
27.744952668001	101.03\\
27.760192824	101.03\\
27.776060229999	101.03\\
27.791850636999	101.03\\
27.808222845999	101.03\\
27.823971895999	101.03\\
27.839940097	101.03\\
27.856062320999	101.03\\
27.871962998999	101.03\\
27.888035467	101.03\\
27.903905996999	101.03\\
27.920036457	101.03\\
27.935924446999	101.03\\
27.951951037999	101.03\\
27.968010695999	101.03\\
27.984000623999	101.03\\
27.999864564	101.03\\
28.017386875	101.03\\
28.032605016	101.03\\
28.048026557	101.03\\
28.063902336999	101.03\\
28.0800032	101.03\\
28.095857908	101.03\\
28.112071300999	101.03\\
28.128089992	101.03\\
28.147021154999	101.03\\
28.16002725	101.03\\
28.176018952999	101.03\\
28.191873623999	101.03\\
28.207861529999	101.03\\
28.224049238	101.03\\
28.240036065999	101.03\\
28.256065774	101.03\\
28.272010255	101.03\\
28.288036064999	101.03\\
28.304027759	101.03\\
28.319931643999	101.03\\
28.335915505999	101.03\\
28.352044259	101.03\\
28.368237206	101.03\\
28.383989107	101.03\\
28.400579637999	101.03\\
28.416439077999	101.03\\
28.432019781	101.03\\
28.448121495	101.03\\
28.463931847999	101.03\\
28.482325506999	101.03\\
28.49767795	101.03\\
28.51289858	101.03\\
28.528113312999	101.03\\
28.546542045	101.03\\
28.561720576001	101.03\\
28.577191569999	101.03\\
28.592530754999	101.03\\
28.608006354	101.03\\
28.624036399999	101.03\\
28.639836196999	101.03\\
28.656083893999	101.03\\
28.672130903999	101.03\\
28.687805818999	101.03\\
28.703838847999	101.03\\
28.722195644	101.03\\
28.737772177	101.03\\
28.752962424	101.03\\
28.768110438999	101.03\\
28.783996379999	101.03\\
28.799898618	101.03\\
28.818277137	101.03\\
28.833375696999	101.03\\
28.848615005	101.03\\
28.864066373999	101.03\\
28.879980568999	101.03\\
28.896044897	101.03\\
28.912038075999	101.03\\
28.928115693999	101.03\\
28.943869287999	101.03\\
28.960036073999	101.03\\
28.975941880999	101.03\\
28.992024092999	101.03\\
29.009701312	101.03\\
29.024949731	101.03\\
29.040258187999	101.03\\
29.056089382999	101.03\\
29.071998651	101.03\\
29.087875345999	101.03\\
29.104045307999	101.03\\
29.119958688999	101.03\\
29.136120975	101.03\\
29.151911679	101.03\\
29.168004437999	101.03\\
29.184032	101.03\\
29.199869116999	101.03\\
29.215911366	101.03\\
29.231922654999	101.03\\
29.248068392999	101.03\\
29.263915062999	101.03\\
29.279993488999	101.03\\
29.295814186	101.03\\
29.312018281	101.03\\
29.328003308999	101.03\\
29.344118524999	101.03\\
29.360614886999	101.03\\
29.376116373999	101.03\\
29.391944856	101.03\\
29.409216522999	101.03\\
29.424332059999	101.03\\
29.44000426	101.03\\
29.45593188	101.03\\
29.471969741999	101.03\\
29.487888185999	101.03\\
29.50401777	101.03\\
29.519995542999	101.03\\
29.535982668999	101.03\\
29.551924595	101.03\\
29.568024905998	101.03\\
29.584010821999	101.03\\
29.599913516	101.03\\
29.616156883	101.03\\
29.631935411999	101.03\\
29.648067913999	101.03\\
29.663980659	101.03\\
29.680067713999	101.03\\
29.696037495999	101.03\\
29.711991477999	101.03\\
29.727807703999	101.03\\
29.744155169999	101.03\\
29.761667486	101.03\\
29.777244561	101.03\\
29.792848795999	101.03\\
29.809831533	101.03\\
29.824019627999	101.03\\
29.839989637999	101.03\\
29.856052793999	101.03\\
29.871947464998	101.03\\
29.887911785	101.03\\
29.904040720999	101.03\\
29.920088121999	101.03\\
29.936069042999	101.03\\
29.952023175999	101.03\\
29.968041901	101.03\\
29.983961417	101.03\\
29.999932408999	101.03\\
30.018131226999	101.03\\
30.033302099999	101.03\\
30.048515130999	101.03\\
30.064029953999	101.03\\
30.079909252	101.03\\
30.095999679	101.03\\
30.112024469	101.03\\
30.128004141	101.03\\
30.144302013999	101.03\\
30.161657255999	101.03\\
30.177257201999	101.03\\
30.192558088999	101.03\\
30.209296672	101.03\\
30.224025436999	101.03\\
30.23999635	101.03\\
30.256065877	101.03\\
30.271931057999	101.03\\
30.288054441	101.03\\
30.304031977999	101.03\\
30.320061528	101.03\\
30.336006597	101.03\\
30.352068189999	101.03\\
30.368254389999	101.03\\
30.383892807999	101.03\\
30.399847247998	101.03\\
30.416052034	101.03\\
30.432018036	101.03\\
30.447833604999	101.03\\
30.464028946999	101.03\\
30.479998049	101.03\\
30.495841373999	101.03\\
30.514215509	101.03\\
30.529441656999	101.03\\
30.544562526	101.03\\
30.560027667999	101.03\\
30.576000509	101.03\\
30.591996116999	101.03\\
30.60783938	101.03\\
30.623812320999	101.03\\
30.640000993	101.03\\
30.656053524999	101.03\\
30.671924108999	101.03\\
30.687834593999	101.03\\
30.703833248999	101.03\\
30.720123783999	101.03\\
30.736113736999	101.03\\
30.751981657	101.03\\
30.768167866	101.03\\
30.783993944	101.03\\
30.799923427999	101.03\\
30.816254937	101.03\\
30.832012344999	101.03\\
30.847955889999	101.03\\
30.863978849999	101.03\\
30.880056184	101.03\\
30.896081288999	101.03\\
30.911997205999	101.03\\
30.92807466	101.03\\
30.946386759999	101.03\\
30.961531592	101.03\\
30.977435795	101.03\\
30.992750108999	101.03\\
31.008809943	101.03\\
31.024028242999	101.03\\
31.040056474999	101.03\\
31.056069580999	101.03\\
31.072012732	101.03\\
31.087813897999	101.03\\
31.106010602999	101.03\\
31.121108784	101.03\\
31.136256408999	101.03\\
31.151981270999	101.03\\
31.168057987999	101.03\\
31.183998542	101.03\\
31.199923112999	101.03\\
31.218223772999	101.03\\
31.233407883999	101.03\\
31.248763533	101.03\\
31.264167170999	101.03\\
31.280464886999	101.03\\
31.296340892999	101.03\\
31.312199793001	101.03\\
31.328043616	101.03\\
31.346707125	101.03\\
31.361862075999	101.03\\
31.377163880999	101.03\\
31.392444743999	101.03\\
31.408093554	101.03\\
31.423933408999	101.03\\
31.439960094	101.03\\
31.456064193	101.03\\
31.472083679999	101.03\\
31.487935262	101.03\\
31.504078309	101.03\\
31.519898901999	101.03\\
31.536061762999	101.03\\
31.551947861999	101.03\\
31.568095820999	101.03\\
31.584043031999	101.03\\
31.599898200999	101.03\\
31.615918139999	101.03\\
31.632020832999	101.03\\
31.648041368	101.03\\
31.665071155998	101.03\\
31.680127320999	101.03\\
31.695939485999	101.03\\
31.712037707	101.03\\
31.728077986	101.03\\
31.744104028999	101.03\\
31.759973208999	101.03\\
31.776000821	101.03\\
31.792026932999	101.03\\
31.807921136999	101.03\\
31.823995889	101.03\\
31.83981299	101.03\\
31.857956719999	101.03\\
31.873210821999	101.03\\
31.888430646999	101.03\\
31.904206877	101.03\\
31.919887082999	101.03\\
31.935931539	101.03\\
31.951969262999	101.03\\
31.968063849999	101.03\\
31.984031955999	101.03\\
32.000068338999	101.03\\
32.018539915999	101.03\\
32.033699745999	101.03\\
32.048876457999	101.03\\
32.064165635999	101.03\\
32.079997420999	101.03\\
32.096043366	101.03\\
32.112043257	101.03\\
32.128003029999	101.03\\
32.144041307999	101.03\\
32.160145213999	101.03\\
32.176056694999	101.03\\
32.192043672	101.03\\
32.207789915999	101.03\\
32.224078493	101.03\\
32.239931830999	101.03\\
32.256167863	101.03\\
32.272022715999	101.03\\
32.287866983	101.03\\
32.304062342	101.03\\
32.319948858999	101.03\\
32.335970075	101.03\\
32.351984821999	101.03\\
32.368227412999	101.03\\
32.384047617999	101.03\\
32.399974279	101.03\\
32.415805491999	101.03\\
32.434264332	101.03\\
32.449953396	101.03\\
32.465196782999	101.03\\
32.480379316	101.03\\
32.496070215999	101.03\\
32.511996440999	101.03\\
32.527955082	101.03\\
32.543929225	101.03\\
32.559920082	101.03\\
32.575949057999	101.03\\
32.592025668999	101.03\\
32.607986759	101.03\\
32.623991125999	101.03\\
32.639995976	101.03\\
32.656095220999	101.03\\
32.672015750999	101.03\\
32.687997453	101.03\\
32.704052680999	101.03\\
32.719974526999	101.03\\
32.736014363999	101.03\\
32.752016108	101.03\\
32.768261995999	101.03\\
32.784058254999	101.03\\
32.799965509	101.03\\
32.816104535	101.03\\
32.83200579	101.03\\
32.848018386999	101.03\\
32.863885141999	101.03\\
32.880101962	101.03\\
32.895841035	101.03\\
32.913907736999	101.03\\
32.929049047	101.03\\
32.944228996999	101.03\\
32.959862404	101.03\\
32.976023908	101.03\\
32.991857002999	101.03\\
33.009244781999	101.03\\
33.024422667	101.03\\
33.040013121999	101.03\\
33.055953964998	101.03\\
33.071884991999	101.03\\
33.087998201999	101.03\\
33.104080874999	101.03\\
33.120004927999	101.03\\
33.136093384	101.03\\
33.152026397999	101.03\\
33.168003128	101.03\\
33.184059108	101.03\\
33.199866684999	101.03\\
33.216057719	101.03\\
33.232042361	101.03\\
33.248054609999	101.03\\
33.263776510999	101.03\\
33.282116139999	101.03\\
33.297292479999	101.03\\
33.312542604999	101.03\\
33.327923371	101.03\\
33.344980838999	101.03\\
33.360323241	101.03\\
33.376079547	101.03\\
33.392028913	101.03\\
33.408092502	101.03\\
33.424285233	101.03\\
33.439990399999	101.03\\
33.456195564999	101.03\\
33.472067137999	101.03\\
33.487939349999	101.03\\
33.504001993999	101.03\\
33.520031059	101.03\\
33.536077772999	101.03\\
33.552030149	101.03\\
33.568069731	101.03\\
33.584012694999	101.03\\
33.599935415	101.03\\
33.616270104999	101.03\\
33.632107934999	101.03\\
33.648032311999	101.03\\
33.664011061	101.03\\
33.679931219	101.03\\
33.696085812	101.03\\
33.711968543999	101.03\\
33.727970361999	101.03\\
33.744093122998	101.03\\
33.760019889999	101.03\\
33.778531274	101.03\\
33.793520005999	101.03\\
33.809779422999	101.03\\
};
\end{axis}
\end{tikzpicture}%
}
      \caption{The steady state orientation of the robot for
        $K_{\Psi}^R = 0.1 K_{\Psi, max}^R$}
      \label{fig:8_0.1_max_magnified}
    \end{figure}
  \end{minipage}
\end{minipage}
}

\noindent\makebox[\textwidth][c]{%
\begin{minipage}{\linewidth}
  \begin{minipage}{0.45\linewidth}
    \begin{figure}[H]
      \scalebox{0.6}{% This file was created by matlab2tikz.
%
%The latest updates can be retrieved from
%  http://www.mathworks.com/matlabcentral/fileexchange/22022-matlab2tikz-matlab2tikz
%where you can also make suggestions and rate matlab2tikz.
%
\definecolor{mycolor1}{rgb}{0.00000,0.44700,0.74100}%
%
\begin{tikzpicture}

\begin{axis}[%
width=4.133in,
height=3.26in,
at={(0.693in,0.44in)},
scale only axis,
xmin=0,
xmax=19,
xmajorgrids,
xlabel={Time (seconds)},
ymin=0,
ymax=120,
ymajorgrids,
ylabel={Angle (degrees)},
axis background/.style={fill=white}
]
\addplot [color=mycolor1,solid,forget plot]
  table[row sep=crcr]{%
0	0\\
0.0163819549999982	0.23\\
0.0319267739999965	0.9\\
0.0480634459999949	1.53\\
0.0640696999999955	2.19\\
0.0799117170000002	2.84\\
0.0959398420000002	3.49\\
0.111933950999998	4.14\\
0.127945803999998	4.81\\
0.143964200999998	5.45\\
0.159925566	6.11\\
0.175943472999997	6.75\\
0.191963118000001	7.42\\
0.207920979999997	8.07\\
0.223958387999997	8.71\\
0.241862258000001	9.48\\
0.256941710000001	10.1\\
0.272081295999996	10.71\\
0.287960424999997	11.34\\
0.30411807	12.02\\
0.31995318	12.64\\
0.335933748999998	13.29\\
0.352001651000001	13.96\\
0.367956747000001	14.6\\
0.383967920999996	15.25\\
0.399994861999996	15.91\\
0.416005332000002	16.56\\
0.432011798999995	17.21\\
0.44792332	17.86\\
0.466274148999996	18.64\\
0.481272140999995	19.26\\
0.496313144999996	19.87\\
0.512044608999997	20.49\\
0.527936045999996	21.14\\
0.543893684	21.78\\
0.559924740999997	22.43\\
0.575938414999999	23.09\\
0.591949170999996	23.74\\
0.607892527999997	24.39\\
0.623884052999996	25.04\\
0.639980056999996	25.69\\
0.655865383000001	26.36\\
0.671890030999996	27.01\\
0.687927497000001	27.66\\
0.703930026000001	28.31\\
0.719906651000001	28.96\\
0.735951912	29.61\\
0.751981611999996	30.28\\
0.767926740999997	30.93\\
0.784057691	31.58\\
0.799913892999999	32.23\\
0.815941984999999	32.88\\
0.832085587999999	33.55\\
0.847827853	34.2\\
0.86387132	34.84\\
0.879930254999995	35.49\\
0.895970214000002	36.15\\
0.911961982999995	36.8\\
0.927945519000001	37.46\\
0.943951159999996	38.12\\
0.959947564999998	38.77\\
0.976024171999998	39.42\\
0.991956957000002	40.08\\
1.007850292	40.72\\
1.02531389	41.32\\
1.04052084	41.7\\
1.056116374	42.08\\
1.071965362	42.45\\
1.087933297	42.84\\
1.104001089	43.23\\
1.119976431	43.64\\
1.13598326	44.02\\
1.151877825	44.43\\
1.167786715	44.81\\
1.184057979	45.2\\
1.199910301	45.6\\
1.21589509	45.99\\
1.232148758	46.38\\
1.247879296	46.78\\
1.263872516	47.16\\
1.279852784	47.58\\
1.296000723	47.96\\
1.311833583	48.35\\
1.327941475	48.74\\
1.343837091	49.14\\
1.359864372	49.52\\
1.375946628	49.92\\
1.3920773	50.32\\
1.40806457099999	50.71\\
1.423940789	51.11\\
1.440027995	51.5\\
1.455954376	51.89\\
1.471963715	52.28\\
1.487938857	52.67\\
1.50404586	53.07\\
1.519968731	53.47\\
1.535955268	53.86\\
1.551995041	54.25\\
1.56792866	54.64\\
1.584019309	55.03\\
1.599907319	55.43\\
1.615968644	55.82\\
1.63197895	56.21\\
1.647916592	56.61\\
1.664224772	57\\
1.679915275	57.4\\
1.696002851	57.78\\
1.711962939	58.18\\
1.728062037	58.57\\
1.74407513	58.96\\
1.759998063	59.38\\
1.775841514	59.75\\
1.791898313	60.14\\
1.807943143	60.53\\
1.823914426	60.93\\
1.839970817	61.32\\
1.855924545	61.71\\
1.871945601	62.11\\
1.887901704	62.5\\
1.903886859	62.9\\
1.91998539	63.29\\
1.935957293	63.68\\
1.95193859	64.08\\
1.967924411	64.47\\
1.983983408	64.86\\
2.000008583	65.26\\
2.015897323	65.59\\
2.032055907	65.82\\
2.047961336	66.06\\
2.064218462	66.29\\
2.079919772	66.53\\
2.095933386	66.76\\
2.111925086	66.99\\
2.127944818	67.23\\
2.143963109	67.46\\
2.159941336	67.7\\
2.175898122	67.93\\
2.192007924	68.17\\
2.208010362	68.4\\
2.223978909	68.64\\
2.240014703	68.87\\
2.255925046	69.11\\
2.271956698	69.34\\
2.287938785	69.57\\
2.303910951	69.82\\
2.319942245	70.04\\
2.335909491	70.28\\
2.351955561	70.51\\
2.367924432	70.75\\
2.384002759	70.98\\
2.400075582	71.22\\
2.41595655999999	71.45\\
2.432070767	71.69\\
2.447991034	71.92\\
2.464277813	72.16\\
2.479930118	72.4\\
2.495981157	72.62\\
2.512107123	72.86\\
2.52794038	73.1\\
2.543931525	73.33\\
2.559945348	73.56\\
2.575986173	73.8\\
2.592076456	74.03\\
2.608030249	74.27\\
2.623889575	74.5\\
2.639993988	74.74\\
2.65593616	74.97\\
2.671860439	75.2\\
2.68783076599999	75.44\\
2.703849456	75.67\\
2.7199492	75.91\\
2.73591432099999	76.15\\
2.752029796	76.38\\
2.767934764	76.62\\
2.78402629899999	76.85\\
2.799963726	77.08\\
2.815950428	77.32\\
2.832078466	77.55\\
2.847947959	77.79\\
2.864120468	78.02\\
2.879904309	78.26\\
2.895943002	78.49\\
2.912032768	78.72\\
2.927942918	78.96\\
2.943954547	79.2\\
2.95994567	79.43\\
2.97593563	79.66\\
2.991927836	79.9\\
3.00784666	80.13\\
3.025277748	80.35\\
3.040389568	80.48\\
3.056019623	80.61\\
3.072084657	80.75\\
3.087911001	80.89\\
3.103953165	81.03\\
3.120011535	81.17\\
3.135952276	81.31\\
3.153288672	81.46\\
3.168440131	81.6\\
3.184030686	81.73\\
3.199861359	81.86\\
3.21593023	82\\
3.231979177	82.14\\
3.247846626	82.28\\
3.266183542	82.45\\
3.28151771	82.58\\
3.296665161	82.71\\
3.311938978	82.85\\
3.327951454	82.98\\
3.34405694599999	83.12\\
3.359992125	83.26\\
3.376067667	83.4\\
3.391918367	83.54\\
3.407989485	83.68\\
3.423943494	83.82\\
3.440014854	83.96\\
3.456135301	84.11\\
3.471967939	84.24\\
3.488005116	84.38\\
3.50392897	84.52\\
3.519951105	84.65\\
3.535923456	84.79\\
3.552021125	84.94\\
3.567956181	85.07\\
3.584042495	85.21\\
3.599923759	85.35\\
3.615930111	85.49\\
3.632068387	85.63\\
3.647989296	85.77\\
3.664214991	85.91\\
3.679894442	86.05\\
3.69593593499999	86.19\\
3.712023336	86.33\\
3.727869811	86.47\\
3.743892237	86.61\\
3.75993525	86.75\\
3.775934672	86.89\\
3.791999793	87.03\\
3.807926956	87.17\\
3.823934691	87.31\\
3.839962122	87.45\\
3.85589611	87.58\\
3.872013599	87.72\\
3.888054781	87.86\\
3.904036708	88\\
3.919977983	88.14\\
3.936042086	88.28\\
3.951976506	88.42\\
3.967981154	88.56\\
3.983962054	88.7\\
3.999955031	88.84\\
4.018586887	88.98\\
4.033833772	89.07\\
4.049048853	89.16\\
4.064258858	89.24\\
4.079925405	89.33\\
4.095894137	89.41\\
4.111993104	89.5\\
4.128112077	89.59\\
4.14393804	89.68\\
4.159832722	89.77\\
4.175896891	89.86\\
4.191955038	89.95\\
4.208021677	90.04\\
4.223977034	90.12\\
4.239959934	90.21\\
4.255931442	90.3\\
4.27199511	90.39\\
4.288023437	90.48\\
4.303914528	90.57\\
4.319807783	90.66\\
4.335898538	90.74\\
4.351836626	90.83\\
4.36795099399999	90.92\\
4.383943042	91.01\\
4.399893343	91.1\\
4.415949509	91.19\\
4.432064197	91.28\\
4.447916874	91.37\\
4.466267448	91.47\\
4.481365521	91.56\\
4.49655676599999	91.64\\
4.511986832	91.73\\
4.527894741	91.81\\
4.54392671	91.9\\
4.559939644	91.99\\
4.575965691	92.08\\
4.592069351	92.17\\
4.607929029	92.26\\
4.623960734	92.34\\
4.640082643	92.43\\
4.655936751	92.52\\
4.671995137	92.61\\
4.687902194	92.7\\
4.703982975	92.79\\
4.720019491	92.88\\
4.735817356	92.97\\
4.75402094	93.07\\
4.769178292	93.15\\
4.784298666	93.24\\
4.799907464	93.32\\
4.81593669599999	93.41\\
4.832050377	93.5\\
4.847952819	93.59\\
4.864200509	93.68\\
4.879845993	93.77\\
4.895932863	93.85\\
4.912039879	93.94\\
4.927924828	94.03\\
4.944019398	94.12\\
4.959967429	94.21\\
4.975869552	94.3\\
4.991943752	94.38\\
5.007975126	94.47\\
5.025388874	94.56\\
5.040525083	94.6\\
5.05595790199999	94.65\\
5.071884125	94.7\\
5.087918307	94.75\\
5.103937148	94.8\\
5.119998868	94.85\\
5.135895274	94.9\\
5.15201484	94.95\\
5.167933753	95\\
5.183940337	95.06\\
5.200012552	95.11\\
5.215903947	95.16\\
5.231775977	95.21\\
5.250078509	95.27\\
5.265154431	95.32\\
5.280080095	95.36\\
5.295826559	95.41\\
5.311985096	95.46\\
5.32800897199999	95.51\\
5.34394198	95.56\\
5.35998024	95.61\\
5.375908831	95.66\\
5.391929561	95.71\\
5.408157336	95.77\\
5.42392541299999	95.82\\
5.441000729	95.87\\
5.456254129	95.92\\
5.47405072	95.98\\
5.489246731	96.03\\
5.504573678	96.07\\
5.519967164	96.12\\
5.5359174	96.17\\
5.552081474	96.22\\
5.567905698	96.27\\
5.5839678	96.32\\
5.599891134	96.37\\
5.615963361	96.43\\
5.63214957099999	96.48\\
5.647952658	96.53\\
5.664155521	96.58\\
5.67992896	96.63\\
5.695952348	96.68\\
5.711956383	96.73\\
5.727940195	96.78\\
5.744037854	96.83\\
5.759940226	96.88\\
5.775974587	96.93\\
5.791945554	96.98\\
5.807989898	97.03\\
5.823973343	97.09\\
5.839979931	97.14\\
5.855943176	97.19\\
5.871937265	97.24\\
5.887877866	97.29\\
5.903979863	97.34\\
5.92001784	97.39\\
5.935946434	97.44\\
5.95202471	97.49\\
5.96792434	97.54\\
5.984047908	97.59\\
5.999955061	97.64\\
6.01596462	97.69\\
6.032045109	97.72\\
6.047871594	97.75\\
6.064262796	97.78\\
6.079804808	97.81\\
6.09582217	97.84\\
6.112018837	97.88\\
6.12801859	97.91\\
6.143956156	97.94\\
6.159843059	97.97\\
6.175962957	98\\
6.191971355	98.03\\
6.20795294599999	98.07\\
6.223957214	98.1\\
6.23997432099999	98.13\\
6.255921915	98.16\\
6.271955419	98.19\\
6.287978384	98.23\\
6.303980281	98.26\\
6.319958595	98.29\\
6.335922171	98.32\\
6.351909316	98.35\\
6.367911838	98.38\\
6.38397428799999	98.42\\
6.399972096	98.45\\
6.415956654	98.48\\
6.432105796	98.51\\
6.447890581	98.54\\
6.466280777	98.58\\
6.481386438	98.61\\
6.496449607	98.64\\
6.511994649	98.67\\
6.527996521	98.7\\
6.54397301599999	98.73\\
6.559900575	98.76\\
6.575937577	98.8\\
6.591956187	98.83\\
6.608027444	98.86\\
6.62402977	98.89\\
6.640019966	98.92\\
6.65590795	98.95\\
6.671974679	98.99\\
6.687976055	99.02\\
6.704002529	99.05\\
6.719961198	99.08\\
6.735940464	99.11\\
6.752014678	99.14\\
6.767966264	99.18\\
6.783978645	99.21\\
6.799942687	99.24\\
6.815958796	99.27\\
6.832076566	99.3\\
6.847983548	99.34\\
6.864141343	99.37\\
6.87989676	99.4\\
6.896047166	99.43\\
6.911974916	99.46\\
6.927945081	99.49\\
6.943965153	99.53\\
6.960010676	99.56\\
6.975945356	99.59\\
6.991927401	99.62\\
7.007932714	99.65\\
7.023998749	99.68\\
7.039890509	99.69\\
7.055831291	99.71\\
7.071973694	99.73\\
7.087890312	99.75\\
7.103949327	99.77\\
7.119942402	99.79\\
7.136038587	99.81\\
7.151974406	99.83\\
7.167833798	99.85\\
7.184007828	99.87\\
7.200008594	99.89\\
7.215969572	99.9\\
7.23206201	99.92\\
7.24787499	99.94\\
7.26414530999999	99.96\\
7.279883756	99.98\\
7.296011806	100\\
7.311973986	100.02\\
7.327904813	100.04\\
7.343925076	100.06\\
7.359789702	100.08\\
7.375910385	100.09\\
7.392038923	100.11\\
7.407890901	100.13\\
7.424007684	100.15\\
7.43997162	100.17\\
7.455876764	100.19\\
7.471989482	100.21\\
7.487930566	100.23\\
7.50392215	100.25\\
7.519977829	100.27\\
7.535951117	100.28\\
7.551990149	100.3\\
7.567931403	100.32\\
7.583937205	100.34\\
7.599922114	100.36\\
7.616018199	100.38\\
7.631991421	100.4\\
7.647955519	100.42\\
7.666325936	100.44\\
7.681466791	100.46\\
7.696563426	100.48\\
7.71214891	100.49\\
7.727953945	100.51\\
7.743946256	100.53\\
7.759970669	100.55\\
7.77590242799999	100.57\\
7.791988811	100.59\\
7.807981498	100.61\\
7.823956785	100.63\\
7.839984341	100.65\\
7.855915708	100.67\\
7.871982919	100.68\\
7.887949155	100.7\\
7.903969369	100.72\\
7.920008265	100.74\\
7.935993818	100.76\\
7.951993904	100.78\\
7.967965518	100.8\\
7.984029461	100.82\\
7.999983464	100.84\\
8.01614165599999	100.85\\
8.031878431	100.87\\
8.047943724	100.88\\
8.064213949	100.89\\
8.07993502	100.9\\
8.095967877	100.92\\
8.112036917	100.93\\
8.127966972	100.94\\
8.143945493	100.95\\
8.159932886	100.97\\
8.176071408	100.98\\
8.192318187	100.99\\
8.207947924	101.01\\
8.223793026	101.02\\
8.23990698	101.03\\
8.25585417	101.04\\
8.27199252699999	101.06\\
8.287929941	101.07\\
8.303861566	101.08\\
8.319992186	101.09\\
8.335928474	101.11\\
8.352017434	101.12\\
8.36792585	101.13\\
8.384014333	101.14\\
8.399974586	101.16\\
8.415980012	101.17\\
8.4320926	101.18\\
8.447898257	101.2\\
8.464168518	101.21\\
8.479891979	101.22\\
8.495975898	101.23\\
8.512073036	101.25\\
8.527944864	101.26\\
8.544047632	101.27\\
8.55997371	101.28\\
8.575880625	101.3\\
8.591938953	101.31\\
8.60800501599999	101.32\\
8.623861418	101.33\\
8.639934632	101.35\\
8.655968341	101.36\\
8.67200166	101.37\\
8.68791869	101.39\\
8.703997296	101.4\\
8.720015791	101.41\\
8.73589582499999	101.42\\
8.751975396	101.44\\
8.767867599	101.45\\
8.78397691	101.46\\
8.800015551	101.47\\
8.815938144	101.49\\
8.832009328	101.5\\
8.847906954	101.51\\
8.863987668	101.53\\
8.879964215	101.54\\
8.895966784	101.55\\
8.912000609	101.56\\
8.927956246	101.58\\
8.943918435	101.59\\
8.959972934	101.6\\
8.975988729	101.61\\
8.992023496	101.63\\
9.007957585	101.64\\
9.025499894	101.65\\
9.040769668	101.66\\
9.055949891	101.66\\
9.071998179	101.67\\
9.087917113	101.67\\
9.103975268	101.68\\
9.120039437	101.69\\
9.135946284	101.69\\
9.151971859	101.7\\
9.167968345	101.71\\
9.18401866699999	101.71\\
9.200011656	101.72\\
9.215865805	101.73\\
9.232014517	101.73\\
9.247873798	101.74\\
9.26611223299999	101.75\\
9.281230703	101.75\\
9.296384322	101.76\\
9.311951008	101.76\\
9.327949212	101.77\\
9.343887485	101.78\\
9.359940949	101.78\\
9.375956778	101.79\\
9.39196718	101.8\\
9.407944101	101.8\\
9.423988419	101.81\\
9.439964704	101.81\\
9.455948826	101.82\\
9.471929741	101.83\\
9.487918518	101.83\\
9.503928444	101.84\\
9.519978682	101.85\\
9.535937576	101.85\\
9.551871719	101.86\\
9.56788909699999	101.87\\
9.583982271	101.87\\
9.59989059699999	101.88\\
9.615935065	101.88\\
9.63209872199999	101.89\\
9.6479323	101.9\\
9.664150368	101.9\\
9.679921777	101.91\\
9.695944993	101.92\\
9.711896642	101.92\\
9.727973922	101.93\\
9.743833828	101.93\\
9.759941976	101.94\\
9.775841925	101.95\\
9.79182053799999	101.95\\
9.807995568	101.96\\
9.823953709	101.97\\
9.840172213	101.97\\
9.855931729	101.98\\
9.871984512	101.99\\
9.88795579	101.99\\
9.90393268499999	102\\
9.919992283	102\\
9.935947254	102.01\\
9.952006547	102.02\\
9.967979423	102.02\\
9.983946938	102.03\\
9.999882796	102.04\\
10.018199473	102.04\\
10.033654041	102.04\\
10.048908545	102.04\\
10.064245375	102.04\\
10.080033038	102.04\\
10.095946092	102.04\\
10.111955943	102.04\\
10.127937309	102.04\\
10.144013725	102.04\\
10.159947573	102.04\\
10.175949085	102.04\\
10.191966798	102.04\\
10.207960472	102.04\\
10.224226809	102.04\\
10.239982308	102.04\\
10.255925462	102.04\\
10.271940844	102.04\\
10.287976323	102.04\\
10.303857917	102.04\\
10.319888883	102.04\\
10.335850011	102.04\\
10.35183024	102.04\\
10.367831771	102.04\\
10.383945476	102.04\\
10.399916918	102.04\\
10.415977471	102.04\\
10.43203982	102.04\\
10.447865278	102.04\\
10.466194009	102.04\\
10.481335973	102.04\\
10.496565413	102.04\\
10.51195398	102.04\\
10.527950691	102.04\\
10.543950724	102.04\\
10.55997843	102.04\\
10.575940783	102.04\\
10.591820879	102.04\\
10.607849708	102.04\\
10.623955091	102.04\\
10.639920302	102.04\\
10.655965624	102.04\\
10.671861921	102.04\\
10.687818117	102.04\\
10.703988423	102.04\\
10.719938952	102.04\\
10.736202136	102.04\\
10.751919307	102.04\\
10.767954103	102.04\\
10.784036966	102.04\\
10.79990022	102.04\\
10.815968852	102.04\\
10.832009448	102.04\\
10.847902007	102.04\\
10.864155209	102.04\\
10.879835047	102.04\\
10.895968242	102.04\\
10.911972504	102.04\\
10.927963572	102.04\\
10.944012982	102.04\\
10.959931471	102.04\\
10.975912269	102.04\\
10.991941029	102.04\\
11.00793403	102.04\\
11.02399862	102.04\\
11.039920435	102.04\\
11.055992826	102.04\\
11.071998835	102.04\\
11.087928468	102.04\\
11.103959642	102.04\\
11.119982732	102.04\\
11.135922815	102.04\\
11.15199361	102.04\\
11.167958527	102.04\\
11.183925009	102.04\\
11.199912134	102.04\\
11.215802487	102.04\\
11.231936138	102.04\\
11.248037957	102.04\\
11.263933083	102.04\\
11.280297392	102.04\\
11.295924359	102.04\\
11.31193208	102.04\\
11.328017196	102.04\\
11.343944088	102.04\\
11.359958945	102.04\\
11.375917098	102.04\\
11.392045523	102.04\\
11.408037362	102.04\\
11.423980611	102.04\\
11.439976157	102.04\\
11.456004506	102.04\\
11.471928918	102.04\\
11.488012616	102.04\\
11.503945827	102.04\\
11.51995647	102.04\\
11.535867401	102.04\\
11.551874971	102.04\\
11.56800941	102.04\\
11.58382962	102.04\\
11.59996525	102.04\\
11.61593964	102.04\\
11.631849917	102.04\\
11.648119319	102.04\\
11.663943183	102.04\\
11.680282369	102.04\\
11.69597445	102.04\\
11.711955219	102.04\\
11.727954515	102.04\\
11.743849037	102.04\\
11.762120918	102.04\\
11.777272566	102.04\\
11.792386126	102.04\\
11.808012187	102.04\\
11.8239772	102.04\\
11.839929415	102.04\\
11.8559522	102.04\\
11.871937269	102.04\\
11.887968944	102.04\\
11.903925188	102.04\\
11.919925071	102.04\\
11.935990336	102.04\\
11.951937004	102.04\\
11.967987101	102.04\\
11.983918378	102.04\\
12.000009893	102.04\\
12.015948938	102.04\\
12.031893192	102.04\\
12.048075125	102.04\\
12.063909583	102.04\\
12.082448347	102.04\\
12.097759092	102.04\\
12.112916255	102.04\\
12.128102753	102.04\\
12.143962082	102.04\\
12.160060358	102.04\\
12.175968599	102.04\\
12.191999933	102.04\\
12.208125214	102.04\\
12.22400487	102.04\\
12.239936082	102.04\\
12.256048871	102.04\\
12.272037142	102.04\\
12.287900681	102.04\\
12.303834592	102.04\\
12.319905806	102.04\\
12.335811065	102.04\\
12.351920157	102.04\\
12.368007854	102.04\\
12.383944099	102.04\\
12.4001321	102.04\\
12.415945905	102.04\\
12.431942774	102.04\\
12.448046462	102.04\\
12.46403758	102.04\\
12.480293123	102.04\\
12.49599434	102.04\\
12.511955865	102.04\\
12.528031239	102.04\\
12.543939051	102.04\\
12.562376581	102.04\\
12.577594974	102.04\\
12.592754194	102.04\\
12.608178993	102.04\\
12.623980602	102.04\\
12.639933553	102.04\\
12.655992209	102.04\\
12.67190307	102.04\\
12.687862107	102.04\\
12.703907141	102.04\\
12.719947906	102.04\\
12.735939019	102.04\\
12.75194242	102.04\\
12.767935217	102.04\\
12.783892598	102.04\\
12.800072459	102.04\\
12.815956955	102.04\\
12.831963869	102.04\\
12.84808214	102.04\\
12.863895339	102.04\\
12.879994976	102.04\\
12.895903798	102.04\\
12.91194648	102.04\\
12.9279653	102.04\\
12.943919737	102.04\\
12.95999961	102.04\\
12.975935898	102.04\\
12.991816185	102.04\\
13.007923937	102.04\\
13.023912665	102.04\\
13.0400329	102.04\\
13.05686738	102.04\\
13.072040739	102.04\\
13.087986555	102.04\\
13.104003922	102.04\\
13.119914618	102.04\\
13.135986609	102.04\\
13.151887062	102.04\\
13.168005387	102.04\\
13.183911597	102.04\\
13.200139133	102.04\\
13.216051156	102.04\\
13.231952269	102.04\\
13.248079105	102.04\\
13.263920764	102.04\\
13.282390314	102.04\\
13.297601696	102.04\\
13.312718025	102.04\\
13.328097242	102.04\\
13.343929782	102.04\\
13.359949011	102.04\\
13.375859413	102.04\\
13.391988411	102.04\\
13.407823475	102.04\\
13.423959411	102.04\\
13.439980636	102.04\\
13.456036208	102.04\\
13.471954591	102.04\\
13.487984431	102.04\\
13.503876446	102.04\\
13.519880281	102.04\\
13.535952444	102.04\\
13.551951319	102.04\\
13.568006592	102.04\\
13.583935258	102.04\\
13.60002309	102.04\\
13.615891174	102.04\\
13.631976651	102.04\\
13.648109581	102.04\\
13.663833739	102.04\\
13.680022844	102.04\\
13.695929687	102.04\\
13.711939422	102.04\\
13.72799018	102.04\\
13.743852866	102.04\\
13.759897591	102.04\\
13.775933206	102.04\\
13.791961503	102.04\\
13.808055945	102.04\\
13.8239973	102.04\\
13.839971675	102.04\\
13.856009414	102.04\\
13.872039109	102.04\\
13.887926603	102.04\\
13.903930208	102.04\\
13.919936501	102.04\\
13.936022434	102.04\\
13.951923728	102.04\\
13.967940243	102.04\\
13.984050857	102.04\\
13.999962616	102.04\\
14.015902347	102.04\\
14.03187592	102.04\\
14.047924503	102.04\\
14.063954143	102.04\\
14.080234451	102.04\\
14.095931027	102.04\\
14.111953597	102.04\\
14.128077736	102.04\\
14.143969884	102.04\\
14.159957269	102.04\\
14.175854235	102.04\\
14.191986672	102.04\\
14.207974908	102.04\\
14.223936173	102.04\\
14.239976182	102.04\\
14.255938585	102.04\\
14.271934043	102.04\\
14.287972949	102.04\\
14.30389672	102.04\\
14.319941805	102.04\\
14.33596713	102.04\\
14.351896247	102.04\\
14.367964881	102.04\\
14.383886321	102.04\\
14.400095805	102.04\\
14.415901071	102.04\\
14.431916308	102.04\\
14.448110106	102.04\\
14.463967967	102.04\\
14.480286647	102.04\\
14.495894822	102.04\\
14.5118949	102.04\\
14.527917379	102.04\\
14.543899031	102.04\\
14.559964394	102.04\\
14.576027368	102.04\\
14.591933305	102.04\\
14.608041586	102.04\\
14.623963267	102.04\\
14.639954635	102.04\\
14.65593966	102.04\\
14.671937903	102.04\\
14.688053624	102.04\\
14.703915879	102.04\\
14.719962764	102.04\\
14.735907652	102.04\\
14.751943256	102.04\\
14.76795916	102.04\\
14.783923716	102.04\\
14.800001578	102.04\\
14.815904796	102.04\\
14.831949125	102.04\\
14.84806934	102.04\\
14.863864872	102.04\\
14.882215813	102.04\\
14.897449914	102.04\\
14.912774476	102.04\\
14.927983045	102.04\\
14.943917087	102.04\\
14.959940304	102.04\\
14.97594149	102.04\\
14.991876185	102.04\\
15.010243455	102.04\\
15.024359322	102.04\\
15.039953139	102.04\\
15.055977208	102.04\\
15.071936216	102.04\\
15.088012528	102.04\\
15.103938653	102.04\\
15.119926624	102.04\\
15.136136393	102.04\\
15.151921368	102.04\\
15.167964639	102.04\\
15.183962738	102.04\\
15.199980401	102.04\\
15.215969044	102.04\\
15.231945338	102.04\\
15.248056157	102.04\\
15.263926438	102.04\\
15.280415999	102.04\\
15.295841995	102.04\\
15.31199244	102.04\\
15.328036318	102.04\\
15.343880837	102.04\\
15.36003883	102.04\\
15.375906293	102.04\\
15.391842779	102.04\\
15.407973245	102.04\\
15.423932862	102.04\\
15.439936082	102.04\\
15.455957594	102.04\\
15.471937436	102.04\\
15.487970274	102.04\\
15.503948282	102.04\\
15.519931472	102.04\\
15.536013631	102.04\\
15.551934761	102.04\\
15.567989062	102.04\\
15.583944433	102.04\\
15.599972039	102.04\\
15.615920822	102.04\\
15.631966969	102.04\\
15.648046131	102.04\\
15.664001108	102.04\\
15.680592816	102.04\\
15.69608023	102.04\\
15.71194716	102.04\\
15.727926177	102.04\\
15.743880996	102.04\\
15.759928642	102.04\\
15.775891741	102.04\\
15.791862515	102.04\\
15.808061141	102.04\\
15.824095963	102.04\\
15.839976511	102.04\\
15.855991188	102.04\\
15.871944352	102.04\\
15.887962322	102.04\\
15.903888515	102.04\\
15.919958405	102.04\\
15.93610655	102.04\\
15.951947932	102.04\\
15.967966197	102.04\\
15.98398156	102.04\\
16.00004352	102.04\\
16.017397626	102.04\\
16.032816181	102.04\\
16.048107442	102.04\\
16.063875665	102.04\\
16.082201882	102.04\\
16.097241252	102.04\\
16.112457348	102.04\\
16.127939993	102.04\\
16.144002793	102.04\\
16.159897566	102.04\\
16.175981722	102.04\\
16.191997767	102.04\\
16.207927174	102.04\\
16.223892207	102.04\\
16.239759597	102.04\\
16.255769925	102.04\\
16.273823543	102.04\\
16.288735446	102.04\\
16.303769195	102.04\\
16.319912432	102.04\\
16.335977042	102.04\\
16.351840179	102.04\\
16.367877823	102.04\\
16.383955292	102.04\\
16.400006471	102.04\\
16.415932872	102.04\\
16.43196373	102.04\\
16.44796837	102.04\\
16.464006306	102.04\\
16.480140485	102.04\\
16.49591404	102.04\\
16.511937972	102.04\\
16.527927687	102.04\\
16.543937563	102.04\\
16.560002713	102.04\\
16.575994323	102.04\\
16.591911794	102.04\\
16.60816041	102.04\\
16.624054238	102.04\\
16.640045977	102.04\\
16.656011838	102.04\\
16.671955111	102.04\\
16.688085521	102.04\\
16.703975168	102.04\\
16.719919711	102.04\\
16.735996508	102.04\\
16.751895945	102.04\\
16.767971392	102.04\\
16.783924689	102.04\\
16.800010968	102.04\\
16.815965614	102.04\\
16.831969189	102.04\\
16.848078704	102.04\\
16.863935449	102.04\\
16.882287891	102.04\\
16.897545761	102.04\\
16.912722483	102.04\\
16.927962393	102.04\\
16.943951471	102.04\\
16.959988257	102.04\\
16.975964823	102.04\\
16.994466762	102.04\\
17.007933162	102.04\\
17.025313227	102.04\\
17.0405263	102.04\\
17.056189352	102.04\\
17.071816515	102.04\\
17.087823145	102.04\\
17.103830838	102.04\\
17.119942735	102.04\\
17.136000445	102.04\\
17.152054329	102.04\\
17.168086986	102.04\\
17.183967659	102.04\\
17.199970368	102.04\\
17.215922437	102.04\\
17.231987373	102.04\\
17.248180789	102.04\\
17.263898211	102.04\\
17.282258596	102.04\\
17.297556954	102.04\\
17.312806836	102.04\\
17.328116358	102.04\\
17.343935797	102.04\\
17.360039716	102.04\\
17.376006073	102.04\\
17.391882838	102.04\\
17.407973708	102.04\\
17.423865803	102.04\\
17.439924871	102.04\\
17.455968187	102.04\\
17.471964748	102.04\\
17.488005032	102.04\\
17.50397855	102.04\\
};
\end{axis}
\end{tikzpicture}%
}
      \caption{The orientation of the robot over time for
        $K_{\Psi}^R = 0.2 K_{\Psi, max}^R$}
      \label{fig:8_0.2_max}
    \end{figure}
  \end{minipage}
  \hfill
  \begin{minipage}{0.45\linewidth}
    \begin{figure}[H]
      \scalebox{0.6}{% This file was created by matlab2tikz.
%
%The latest updates can be retrieved from
%  http://www.mathworks.com/matlabcentral/fileexchange/22022-matlab2tikz-matlab2tikz
%where you can also make suggestions and rate matlab2tikz.
%
\definecolor{mycolor1}{rgb}{0.00000,0.44700,0.74100}%
%
\begin{tikzpicture}

\begin{axis}[%
width=4.133in,
height=3.26in,
at={(0.693in,0.44in)},
scale only axis,
xmin=14,
xmax=18,
xmajorgrids,
ymin=100,
ymax=104,
ymajorgrids,
axis background/.style={fill=white}
]
\addplot [color=mycolor1,solid,forget plot]
  table[row sep=crcr]{%
14.015902347	102.04\\
14.03187592	102.04\\
14.047924503	102.04\\
14.063954143	102.04\\
14.080234451	102.04\\
14.095931027	102.04\\
14.111953597	102.04\\
14.128077736	102.04\\
14.143969884	102.04\\
14.159957269	102.04\\
14.175854235	102.04\\
14.191986672	102.04\\
14.207974908	102.04\\
14.223936173	102.04\\
14.239976182	102.04\\
14.255938585	102.04\\
14.271934043	102.04\\
14.287972949	102.04\\
14.30389672	102.04\\
14.319941805	102.04\\
14.33596713	102.04\\
14.351896247	102.04\\
14.367964881	102.04\\
14.383886321	102.04\\
14.400095805	102.04\\
14.415901071	102.04\\
14.431916308	102.04\\
14.448110106	102.04\\
14.463967967	102.04\\
14.480286647	102.04\\
14.495894822	102.04\\
14.5118949	102.04\\
14.527917379	102.04\\
14.543899031	102.04\\
14.559964394	102.04\\
14.576027368	102.04\\
14.591933305	102.04\\
14.608041586	102.04\\
14.623963267	102.04\\
14.639954635	102.04\\
14.65593966	102.04\\
14.671937903	102.04\\
14.688053624	102.04\\
14.703915879	102.04\\
14.719962764	102.04\\
14.735907652	102.04\\
14.751943256	102.04\\
14.76795916	102.04\\
14.783923716	102.04\\
14.800001578	102.04\\
14.815904796	102.04\\
14.831949125	102.04\\
14.84806934	102.04\\
14.863864872	102.04\\
14.882215813	102.04\\
14.897449914	102.04\\
14.912774476	102.04\\
14.927983045	102.04\\
14.943917087	102.04\\
14.959940304	102.04\\
14.97594149	102.04\\
14.991876185	102.04\\
15.010243455	102.04\\
15.024359322	102.04\\
15.039953139	102.04\\
15.055977208	102.04\\
15.071936216	102.04\\
15.088012528	102.04\\
15.103938653	102.04\\
15.119926624	102.04\\
15.136136393	102.04\\
15.151921368	102.04\\
15.167964639	102.04\\
15.183962738	102.04\\
15.199980401	102.04\\
15.215969044	102.04\\
15.231945338	102.04\\
15.248056157	102.04\\
15.263926438	102.04\\
15.280415999	102.04\\
15.295841995	102.04\\
15.31199244	102.04\\
15.328036318	102.04\\
15.343880837	102.04\\
15.36003883	102.04\\
15.375906293	102.04\\
15.391842779	102.04\\
15.407973245	102.04\\
15.423932862	102.04\\
15.439936082	102.04\\
15.455957594	102.04\\
15.471937436	102.04\\
15.487970274	102.04\\
15.503948282	102.04\\
15.519931472	102.04\\
15.536013631	102.04\\
15.551934761	102.04\\
15.567989062	102.04\\
15.583944433	102.04\\
15.599972039	102.04\\
15.615920822	102.04\\
15.631966969	102.04\\
15.648046131	102.04\\
15.664001108	102.04\\
15.680592816	102.04\\
15.69608023	102.04\\
15.71194716	102.04\\
15.727926177	102.04\\
15.743880996	102.04\\
15.759928642	102.04\\
15.775891741	102.04\\
15.791862515	102.04\\
15.808061141	102.04\\
15.824095963	102.04\\
15.839976511	102.04\\
15.855991188	102.04\\
15.871944352	102.04\\
15.887962322	102.04\\
15.903888515	102.04\\
15.919958405	102.04\\
15.93610655	102.04\\
15.951947932	102.04\\
15.967966197	102.04\\
15.98398156	102.04\\
16.00004352	102.04\\
16.017397626	102.04\\
16.032816181	102.04\\
16.048107442	102.04\\
16.063875665	102.04\\
16.082201882	102.04\\
16.097241252	102.04\\
16.112457348	102.04\\
16.127939993	102.04\\
16.144002793	102.04\\
16.159897566	102.04\\
16.175981722	102.04\\
16.191997767	102.04\\
16.207927174	102.04\\
16.223892207	102.04\\
16.239759597	102.04\\
16.255769925	102.04\\
16.273823543	102.04\\
16.288735446	102.04\\
16.303769195	102.04\\
16.319912432	102.04\\
16.335977042	102.04\\
16.351840179	102.04\\
16.367877823	102.04\\
16.383955292	102.04\\
16.400006471	102.04\\
16.415932872	102.04\\
16.43196373	102.04\\
16.44796837	102.04\\
16.464006306	102.04\\
16.480140485	102.04\\
16.49591404	102.04\\
16.511937972	102.04\\
16.527927687	102.04\\
16.543937563	102.04\\
16.560002713	102.04\\
16.575994323	102.04\\
16.591911794	102.04\\
16.60816041	102.04\\
16.624054238	102.04\\
16.640045977	102.04\\
16.656011838	102.04\\
16.671955111	102.04\\
16.688085521	102.04\\
16.703975168	102.04\\
16.719919711	102.04\\
16.735996508	102.04\\
16.751895945	102.04\\
16.767971392	102.04\\
16.783924689	102.04\\
16.800010968	102.04\\
16.815965614	102.04\\
16.831969189	102.04\\
16.848078704	102.04\\
16.863935449	102.04\\
16.882287891	102.04\\
16.897545761	102.04\\
16.912722483	102.04\\
16.927962393	102.04\\
16.943951471	102.04\\
16.959988257	102.04\\
16.975964823	102.04\\
16.994466762	102.04\\
17.007933162	102.04\\
17.025313227	102.04\\
17.0405263	102.04\\
17.056189352	102.04\\
17.071816515	102.04\\
17.087823145	102.04\\
17.103830838	102.04\\
17.119942735	102.04\\
17.136000445	102.04\\
17.152054329	102.04\\
17.168086986	102.04\\
17.183967659	102.04\\
17.199970368	102.04\\
17.215922437	102.04\\
17.231987373	102.04\\
17.248180789	102.04\\
17.263898211	102.04\\
17.282258596	102.04\\
17.297556954	102.04\\
17.312806836	102.04\\
17.328116358	102.04\\
17.343935797	102.04\\
17.360039716	102.04\\
17.376006073	102.04\\
17.391882838	102.04\\
17.407973708	102.04\\
17.423865803	102.04\\
17.439924871	102.04\\
17.455968187	102.04\\
17.471964748	102.04\\
17.488005032	102.04\\
17.50397855	102.04\\
};
\end{axis}
\end{tikzpicture}%
}
      \caption{The steady state orientation of the robot for
        $K_{\Psi}^R = 0.2 K_{\Psi, max}^R$}
      \label{fig:8_0.2_max_magnified}
    \end{figure}
  \end{minipage}
\end{minipage}
}

\noindent\makebox[\textwidth][c]{%
\begin{minipage}{\linewidth}
  \begin{minipage}{0.45\linewidth}
    \begin{figure}[H]
      \scalebox{0.6}{% This file was created by matlab2tikz.
%
%The latest updates can be retrieved from
%  http://www.mathworks.com/matlabcentral/fileexchange/22022-matlab2tikz-matlab2tikz
%where you can also make suggestions and rate matlab2tikz.
%
\definecolor{mycolor1}{rgb}{0.00000,0.44700,0.74100}%
%
\begin{tikzpicture}

\begin{axis}[%
width=4.133in,
height=3.26in,
at={(0.693in,0.44in)},
scale only axis,
xmin=0,
xmax=13,
xmajorgrids,
ymin=0,
ymax=120,
ymajorgrids,
axis background/.style={fill=white}
]
\addplot [color=mycolor1,solid,forget plot]
  table[row sep=crcr]{%
0	0\\
0.0159279149999917	1.01\\
0.0320604899999946	2.62\\
0.0479482419999987	4.26\\
0.0639531620000005	5.88\\
0.0823400259999931	7.85\\
0.0974942259999961	9.4\\
0.112662259000003	10.95\\
0.128097980000004	12.53\\
0.143864966999993	14.1\\
0.159962746999994	15.69\\
0.175990738999993	17.35\\
0.192034471000001	19\\
0.208057030000001	20.64\\
0.223997402999997	22.28\\
0.239995335999995	23.92\\
0.256012627999991	25.54\\
0.271964907999992	27.19\\
0.288076484999999	28.85\\
0.303949754999995	30.46\\
0.319961520999998	32.06\\
0.335928357999995	33.71\\
0.351977016000003	35.35\\
0.368026355999991	36.99\\
0.383969680999994	38.64\\
0.400000303000003	40.26\\
0.415996971000001	41.9\\
0.432050186999992	43.54\\
0.447953967999995	45.17\\
0.466299186999992	47.12\\
0.481524890000001	48.68\\
0.496918542999998	50.25\\
0.512161976999998	51.81\\
0.528028853	53.37\\
0.543837179000005	54.99\\
0.559842495999992	56.61\\
0.575983403999999	58.25\\
0.592313557999998	59.91\\
0.608000377999991	61.57\\
0.624037209999993	63.18\\
0.639982296000004	64.81\\
0.655936201999992	66.44\\
0.671931459999993	68.08\\
0.687977320999995	69.69\\
0.704037152999997	71.38\\
0.720005109999999	72.99\\
0.735955991999999	74.61\\
0.751966814999998	76.25\\
0.767955472000003	77.88\\
0.783970875	79.52\\
0.799905444000006	81.17\\
0.815968686999991	82.8\\
0.831969055999994	84.44\\
0.847948958000004	86.08\\
0.863986415999994	87.71\\
0.880320471000001	89.35\\
0.895986788000002	91.01\\
0.912058968999997	92.63\\
0.927938980000004	94.27\\
0.943977365999997	95.89\\
0.959896142000005	97.59\\
0.975891067000002	99.15\\
0.991980902999997	100.8\\
1.00980414	102.14\\
1.025186086	102.18\\
1.040423045	102.22\\
1.05593677	102.25\\
1.071987598	102.29\\
1.087977183	102.33\\
1.103975773	102.37\\
1.11999392200001	102.4\\
1.13590967900001	102.44\\
1.15199748099999	102.48\\
1.167950272	102.52\\
1.183946617	102.55\\
1.19989355499999	102.59\\
1.215839301	102.63\\
1.23206491599999	102.67\\
1.24794579700001	102.71\\
1.263934861	102.75\\
1.280003625	102.78\\
1.295991899	102.82\\
1.31200920299999	102.86\\
1.32801146600001	102.9\\
1.344022595	102.94\\
1.35997639399999	102.97\\
1.375958125	103.01\\
1.392005039	103.05\\
1.407982167	103.09\\
1.423972587	103.13\\
1.43999666599999	103.16\\
1.455930595	103.2\\
1.471943741	103.24\\
1.487980377	103.28\\
1.50398922499999	103.32\\
1.51999730599999	103.35\\
1.535995024	103.39\\
1.55194119499999	103.43\\
1.567978914	103.47\\
1.583947456	103.51\\
1.599965951	103.54\\
1.615892546	103.58\\
1.63200293099999	103.62\\
1.647959325	103.66\\
1.663939623	103.7\\
1.682328723	103.74\\
1.697465719	103.78\\
1.712631927	103.81\\
1.72802956999999	103.85\\
1.74397298999999	103.89\\
1.75993346699999	103.93\\
1.77598193099999	103.96\\
1.792001874	104\\
1.808089037	104.04\\
1.82389161399999	104.08\\
1.839960855	104.12\\
1.856055829	104.15\\
1.871921693	104.19\\
1.88804473	104.23\\
1.903979911	104.27\\
1.919980926	104.31\\
1.93586151899999	104.34\\
1.951995546	104.38\\
1.96788328299999	104.42\\
1.983996454	104.46\\
2.00003771600001	104.5\\
2.017517083	104.5\\
2.032708513	104.48\\
2.047949008	104.46\\
2.06396468699999	104.44\\
2.080210447	104.42\\
2.09589648999999	104.4\\
2.11210142200001	104.39\\
2.12793465099999	104.37\\
2.143964396	104.35\\
2.15996205	104.33\\
2.178247053	104.31\\
2.19334619400001	104.29\\
2.209148027	104.27\\
2.224235277	104.25\\
2.24004651199999	104.23\\
2.255787973	104.21\\
2.271751542	104.2\\
2.287907344	104.18\\
2.303946978	104.16\\
2.320041245	104.14\\
2.335969395	104.12\\
2.352013774	104.1\\
2.36798392900001	104.08\\
2.383972426	104.06\\
2.400037597	104.04\\
2.415985001	104.02\\
2.43200528399999	104\\
2.447966837	103.99\\
2.463941491	103.97\\
2.480204777	103.95\\
2.496031286	103.93\\
2.512036191	103.91\\
2.52796845999999	103.89\\
2.543964082	103.87\\
2.559962678	103.85\\
2.575951456	103.83\\
2.59193276099999	103.81\\
2.607821535	103.8\\
2.62393333399999	103.78\\
2.63999433	103.76\\
2.656180977	103.74\\
2.67195844	103.72\\
2.688021772	103.7\\
2.703976542	103.68\\
2.720025301	103.66\\
2.73586151199999	103.64\\
2.752039639	103.62\\
2.76796152499999	103.6\\
2.783945322	103.59\\
2.799887905	103.57\\
2.815912984	103.55\\
2.831966148	103.53\\
2.847985455	103.51\\
2.86397692900001	103.49\\
2.882449663	103.47\\
2.89775463	103.45\\
2.913137925	103.43\\
2.92842631	103.41\\
2.943951426	103.4\\
2.96002115700001	103.38\\
2.97596801000001	103.36\\
2.991887701	103.34\\
3.009583404	103.32\\
3.02475806	103.32\\
3.039994362	103.31\\
3.055970897	103.3\\
3.071980817	103.3\\
3.088003211	103.29\\
3.10395876199999	103.28\\
3.12003538599999	103.28\\
3.13597804899999	103.27\\
3.15202048800001	103.27\\
3.168220098	103.26\\
3.183971367	103.25\\
3.200161507	103.25\\
3.21599857099999	103.24\\
3.23213440199999	103.23\\
3.247867655	103.23\\
3.263975192	103.22\\
3.280290264	103.21\\
3.295851949	103.21\\
3.311907892	103.2\\
3.327899495	103.2\\
3.343846663	103.19\\
3.35997376199999	103.18\\
3.375962735	103.18\\
3.391981861	103.17\\
3.408086888	103.16\\
3.424119295	103.16\\
3.44003578299999	103.15\\
3.455842743	103.14\\
3.471891169	103.14\\
3.48801612899999	103.13\\
3.503981803	103.13\\
3.52026023	103.12\\
3.535968214	103.11\\
3.552006583	103.11\\
3.567950825	103.1\\
3.583944734	103.09\\
3.599990867	103.09\\
3.615946755	103.08\\
3.63205338	103.08\\
3.64798606	103.07\\
3.66394004899999	103.06\\
3.68028188699999	103.06\\
3.69595493799999	103.05\\
3.71196123099999	103.04\\
3.728071154	103.04\\
3.744040066	103.03\\
3.75990754899999	103.02\\
3.77595541499999	103.02\\
3.791940154	103.01\\
3.807952494	103.01\\
3.823901575	103\\
3.839930226	102.99\\
3.85594834999999	102.99\\
3.871964845	102.98\\
3.88802462899999	102.97\\
3.90394565799999	102.97\\
3.92002038	102.96\\
3.935998058	102.95\\
3.951989983	102.95\\
3.967946933	102.94\\
3.983970705	102.94\\
4.00002915099999	102.93\\
4.018017376	102.93\\
4.03353249699999	102.94\\
4.048954771	102.94\\
4.064260147	102.95\\
4.08024893099999	102.96\\
4.095951633	102.96\\
4.111961058	102.97\\
4.12784188699999	102.97\\
4.143934064	102.98\\
4.159957053	102.99\\
4.175990542	102.99\\
4.19200017200001	103\\
4.208054157	103.01\\
4.223967831	103.01\\
4.239969707	103.02\\
4.255946314	103.03\\
4.27193914399999	103.03\\
4.28798227	103.04\\
4.304030006	103.04\\
4.32006857099999	103.05\\
4.33599106900001	103.06\\
4.352008612	103.06\\
4.368065759	103.07\\
4.383962002	103.08\\
4.399989979	103.08\\
4.415972362	103.09\\
4.432018073	103.1\\
4.44798656	103.1\\
4.46397412	103.11\\
4.480047485	103.11\\
4.495801131	103.12\\
4.514037934	103.13\\
4.52935157099999	103.13\\
4.544663935	103.14\\
4.559955245	103.15\\
4.575975991	103.15\\
4.591962454	103.16\\
4.607979345	103.17\\
4.624041729	103.17\\
4.64000881199999	103.18\\
4.655942385	103.18\\
4.67194144599999	103.19\\
4.68790428399999	103.2\\
4.703998712	103.2\\
4.72000157099999	103.21\\
4.73599610699999	103.22\\
4.752017759	103.22\\
4.767917161	103.23\\
4.783992897	103.23\\
4.800036877	103.24\\
4.816031447	103.25\\
4.831991727	103.25\\
4.84793241499999	103.26\\
4.86395662199999	103.27\\
4.88237512799999	103.27\\
4.89753390899999	103.28\\
4.912754936	103.29\\
4.927968241	103.29\\
4.94400804899999	103.3\\
4.959883789	103.3\\
4.97596315899999	103.31\\
4.99197851099999	103.32\\
5.009822845	103.32\\
5.024930796	103.31\\
5.040166126	103.31\\
5.055956684	103.3\\
5.07197327499999	103.3\\
5.087988448	103.29\\
5.104009032	103.28\\
5.120006735	103.28\\
5.13594100399999	103.27\\
5.15200925	103.26\\
5.167945831	103.26\\
5.184010162	103.25\\
5.200098859	103.25\\
5.21593009199999	103.24\\
5.232012961	103.23\\
5.247930116	103.23\\
5.26398682799999	103.22\\
5.282453447	103.21\\
5.297675602	103.21\\
5.312911979	103.2\\
5.32816722399999	103.19\\
5.344003075	103.19\\
5.35991228200001	103.18\\
5.376018355	103.18\\
5.392038309	103.17\\
5.407976903	103.16\\
5.42399754700001	103.16\\
5.43996348899999	103.15\\
5.455976796	103.14\\
5.47198467299999	103.14\\
5.487943281	103.13\\
5.503844352	103.12\\
5.522096892	103.12\\
5.53725930499999	103.11\\
5.552451206	103.11\\
5.567983064	103.1\\
5.583943035	103.09\\
5.60000980400001	103.09\\
5.61602636	103.08\\
5.63211643699999	103.07\\
5.647953346	103.07\\
5.66397320899999	103.06\\
5.682383235	103.05\\
5.697578068	103.05\\
5.712848898	103.04\\
5.728088876	103.04\\
5.744090888	103.03\\
5.759951758	103.02\\
5.775921616	103.02\\
5.791956756	103.01\\
5.810011222	103\\
5.82513746699999	103\\
5.840344574	102.99\\
5.855953175	102.99\\
5.87193248099999	102.98\\
5.88796548899999	102.97\\
5.903912602	102.97\\
5.91989976099999	102.96\\
5.93597278999999	102.95\\
5.952035375	102.95\\
5.96796734299999	102.94\\
5.98383762	102.93\\
5.999919263	102.93\\
6.01745702	102.93\\
6.032845836	102.94\\
6.04805238699999	102.94\\
6.063957289	102.95\\
6.082284774	102.96\\
6.09759010699999	102.96\\
6.112709088	102.97\\
6.12793768699999	102.97\\
6.143926949	102.98\\
6.159966228	102.99\\
6.175967424	102.99\\
6.192003619	103\\
6.208084781	103.01\\
6.22396189999999	103.01\\
6.24000836399999	103.02\\
6.25605751899999	103.02\\
6.271970495	103.03\\
6.287991134	103.04\\
6.303947448	103.04\\
6.320009853	103.05\\
6.336146941	103.06\\
6.351994419	103.06\\
6.36798011499999	103.07\\
6.383978368	103.08\\
6.400003057	103.08\\
6.415959993	103.09\\
6.432074155	103.09\\
6.44795572499999	103.1\\
6.463948491	103.11\\
6.47995638699999	103.11\\
6.495972928	103.12\\
6.512233309	103.13\\
6.52797965899999	103.13\\
6.544050007	103.14\\
6.55992439299999	103.15\\
6.575794397	103.15\\
6.59196289	103.16\\
6.608508418	103.16\\
6.623978094	103.17\\
6.63998456199999	103.18\\
6.655961984	103.18\\
6.671931617	103.19\\
6.687953346	103.2\\
6.70395510699999	103.2\\
6.719989649	103.21\\
6.735970413	103.21\\
6.751955006	103.22\\
6.76798170999999	103.23\\
6.783796693	103.23\\
6.799806638	103.24\\
6.816006265	103.25\\
6.831996536	103.25\\
6.84789128	103.26\\
6.863881506	103.27\\
6.882360961	103.27\\
6.89790377699999	103.28\\
6.913106513	103.29\\
6.928365766	103.29\\
6.944330735	103.3\\
6.95992589	103.3\\
6.975920546	103.31\\
6.99180443699999	103.32\\
7.009213007	103.32\\
7.02428542900001	103.31\\
7.039976065	103.31\\
7.05596627499999	103.3\\
7.072197948	103.29\\
7.08799878	103.29\\
7.10398408	103.28\\
7.12003390799999	103.28\\
7.13597204899999	103.27\\
7.15201500499999	103.26\\
7.167883142	103.26\\
7.183939213	103.25\\
7.199985487	103.24\\
7.215953903	103.24\\
7.232049889	103.23\\
7.24802769599999	103.22\\
7.263985028	103.22\\
7.280239598	103.21\\
7.295960303	103.21\\
7.314261552	103.2\\
7.329542802	103.19\\
7.344755124	103.19\\
7.359976061	103.18\\
7.37595684999999	103.17\\
7.391998698	103.17\\
7.407969375	103.16\\
7.42397179899999	103.16\\
7.43995039299999	103.15\\
7.45600047	103.14\\
7.47194293699999	103.14\\
7.487996517	103.13\\
7.50401659999999	103.12\\
7.520080955	103.12\\
7.53601289299999	103.11\\
7.551836948	103.1\\
7.567937418	103.1\\
7.58393684999999	103.09\\
7.60003667399999	103.09\\
7.61595628	103.08\\
7.63205367399999	103.07\\
7.647980065	103.07\\
7.663949544	103.06\\
7.682464554	103.05\\
7.697774668	103.05\\
7.71305699599999	103.04\\
7.728424764	103.03\\
7.74398346099999	103.03\\
7.759926131	103.02\\
7.77606992299999	103.02\\
7.79193990199999	103.01\\
7.808055116	103\\
7.82398930499999	103\\
7.83997642	102.99\\
7.855979344	102.98\\
7.87192945899999	102.98\\
7.88807040199999	102.97\\
7.903944329	102.96\\
7.920087531	102.96\\
7.93602487199999	102.95\\
7.952012377	102.95\\
7.967931847	102.94\\
7.98402269	102.93\\
8.00000768099999	102.93\\
8.01588256	102.93\\
8.031952756	102.93\\
8.04798336	102.94\\
8.063950764	102.95\\
8.08040924599999	102.95\\
8.09596215099999	102.96\\
8.114296516	102.97\\
8.129466462	102.97\\
8.14459689399999	102.98\\
8.159953089	102.99\\
8.17608321600001	102.99\\
8.191985462	103\\
8.208002413	103\\
8.224021029	103.01\\
8.241678685	103.02\\
8.25742670199999	103.02\\
8.27250796699999	103.03\\
8.288272882	103.04\\
8.303987117	103.04\\
8.31991531199999	103.05\\
8.33595715099999	103.05\\
8.351956132	103.06\\
8.367970939	103.07\\
8.383948375	103.07\\
8.400004966	103.08\\
8.415965219	103.09\\
8.43198973800001	103.09\\
8.447946207	103.1\\
8.46395063500001	103.11\\
8.48027256999999	103.11\\
8.49596410699999	103.12\\
8.51432398999999	103.13\\
8.52955379700001	103.13\\
8.544714377	103.14\\
8.56005293099999	103.14\\
8.575938685	103.15\\
8.591799786	103.16\\
8.607947957	103.16\\
8.623992932	103.17\\
8.64010945899999	103.18\\
8.655949565	103.18\\
8.671984147	103.19\\
8.687972447	103.19\\
8.70396498899999	103.2\\
8.719996258	103.21\\
8.736027171	103.21\\
8.75188036499999	103.22\\
8.76795265799999	103.23\\
8.783945265	103.23\\
8.800048712	103.24\\
8.816024891	103.25\\
8.832102244	103.25\\
8.84794421600001	103.26\\
8.863953619	103.26\\
8.880301015	103.27\\
8.89598485799999	103.28\\
8.91202697	103.28\\
8.92795733499999	103.29\\
8.944041288	103.3\\
8.95984452599999	103.3\\
8.97589579799999	103.31\\
8.991871899	103.31\\
9.009354462	103.32\\
9.02456673999999	103.31\\
9.04004298899999	103.31\\
9.055997293	103.3\\
9.07196044	103.29\\
9.08801074699999	103.29\\
9.10393895199999	103.28\\
9.119986257	103.27\\
9.135975721	103.27\\
9.15204366499999	103.26\\
9.167948983	103.26\\
9.183978559	103.25\\
9.20228556999999	103.24\\
9.21734012	103.24\\
9.232504619	103.23\\
9.247932897	103.22\\
9.263954273	103.22\\
9.282294271	103.21\\
9.29738893099999	103.2\\
9.31254101799999	103.2\\
9.327963117	103.19\\
9.343951005	103.19\\
9.36000622399999	103.18\\
9.37594868099999	103.17\\
9.392016348	103.17\\
9.407989671	103.16\\
9.424279295	103.15\\
9.440004846	103.15\\
9.455975272	103.14\\
9.471955876	103.13\\
9.487986248	103.13\\
9.50396090799999	103.12\\
9.520081645	103.12\\
9.53594598099999	103.11\\
9.55202454700001	103.1\\
9.56800059100001	103.1\\
9.583953623	103.09\\
9.60004175499999	103.08\\
9.615961892	103.08\\
9.63213210599999	103.07\\
9.648050539	103.06\\
9.663978208	103.06\\
9.682444789	103.05\\
9.697681603	103.04\\
9.712805795	103.04\\
9.728073076	103.03\\
9.74396904799999	103.03\\
9.759879589	103.02\\
9.77601393699999	103.01\\
9.791991089	103.01\\
9.80783461499999	103\\
9.826114412	102.99\\
9.841650749	102.99\\
9.856896918	102.98\\
9.872106174	102.98\\
9.887970841	102.97\\
9.90399367	102.96\\
9.920035016	102.96\\
9.935958642	102.95\\
9.95200258	102.94\\
9.96792537199999	102.94\\
9.983955037	102.93\\
9.999969483	102.93\\
10.015856343	102.93\\
10.031883321	102.93\\
10.047953476	102.94\\
10.063869806	102.95\\
10.080099513	102.95\\
10.095971833	102.96\\
10.112009993	102.96\\
10.128033729	102.97\\
10.143954527	102.98\\
10.159907158	102.98\\
10.175921528	102.99\\
10.191997491	103\\
10.208109366	103\\
10.223962506	103.01\\
10.239968616	103.02\\
10.255881828	103.02\\
10.27200198	103.03\\
10.287881464	103.03\\
10.304005223	103.04\\
10.320030829	103.05\\
10.335978769	103.05\\
10.352004701	103.06\\
10.367945169	103.07\\
10.383976593	103.07\\
10.400033435	103.08\\
10.415962948	103.09\\
10.432040227	103.09\\
10.447861782	103.1\\
10.463886082	103.1\\
10.4801055	103.11\\
10.495913437	103.12\\
10.511960386	103.12\\
10.527964118	103.13\\
10.543966822	103.14\\
10.559954101	103.14\\
10.575959988	103.15\\
10.59201182	103.16\\
10.607946094	103.16\\
10.623857733	103.17\\
10.639983909	103.17\\
10.65595854	103.18\\
10.671956398	103.19\\
10.688025728	103.19\\
10.703945519	103.2\\
10.720089271	103.21\\
10.736218619	103.21\\
10.751975381	103.22\\
10.767942349	103.22\\
10.783939762	103.23\\
10.800094977	103.24\\
10.81592575	103.24\\
10.832003606	103.25\\
10.847945297	103.26\\
10.863962133	103.26\\
10.880201989	103.27\\
10.89594714	103.28\\
10.911970125	103.28\\
10.927863753	103.29\\
10.943962749	103.29\\
10.95991764	103.3\\
10.975996988	103.31\\
10.992008292	103.31\\
11.009491197	103.32\\
11.024715551	103.31\\
11.039961254	103.3\\
11.055908904	103.3\\
11.071949293	103.29\\
11.087937034	103.29\\
11.103957784	103.28\\
11.119994851	103.27\\
11.135907712	103.27\\
11.151966598	103.26\\
11.167949939	103.25\\
11.184128952	103.25\\
11.200008698	103.24\\
11.215853027	103.23\\
11.231933272	103.23\\
11.247891704	103.22\\
11.263857943	103.22\\
11.27992405	103.21\\
11.295985699	103.2\\
11.312177296	103.2\\
11.327967128	103.19\\
11.343937965	103.18\\
11.359821641	103.18\\
11.375964081	103.17\\
11.391840727	103.17\\
11.40795033	103.16\\
11.423966936	103.15\\
11.439946355	103.15\\
11.455949443	103.14\\
11.472066862	103.13\\
11.487980805	103.13\\
11.503955997	103.12\\
11.520041524	103.11\\
11.53595109	103.11\\
11.55196435	103.1\\
11.570246342	103.09\\
11.585279102	103.09\\
11.600334488	103.08\\
11.615937082	103.08\\
11.63199717	103.07\\
11.6478534	103.06\\
11.663953136	103.06\\
11.680098454	103.05\\
11.695916064	103.04\\
11.712017785	103.04\\
11.727965252	103.03\\
11.743962067	103.03\\
11.759930974	103.02\\
11.77594525	103.01\\
11.792031508	103.01\\
11.807868977	103\\
11.823841854	102.99\\
11.840097998	102.99\\
11.855947069	102.98\\
11.872011444	102.97\\
11.887948188	102.97\\
11.903955227	102.96\\
11.919955531	102.96\\
11.935935494	102.95\\
11.951981822	102.94\\
11.968022502	102.94\\
11.984005191	102.93\\
12.000051675	102.92\\
12.017678164	102.92\\
12.032866753	102.93\\
12.048213844	102.94\\
12.063999627	102.94\\
12.080193658	102.95\\
12.095951505	102.95\\
12.11200541	102.96\\
12.127963955	102.97\\
12.144193017	102.97\\
12.159944672	102.98\\
12.175956753	102.99\\
12.191942002	102.99\\
12.208013232	103\\
12.223974131	103.01\\
12.239939601	103.01\\
12.255942476	103.02\\
12.271909985	103.02\\
12.287976098	103.03\\
12.303858823	103.04\\
12.320051801	103.04\\
12.33593982	103.05\\
12.352007471	103.06\\
12.367905764	103.06\\
12.383961052	103.07\\
12.400094945	103.08\\
12.415951684	103.08\\
12.43203739	103.09\\
12.447947975	103.09\\
12.463953481	103.1\\
12.482375089	103.11\\
12.497695734	103.11\\
12.513055403	103.12\\
12.528469538	103.13\\
12.543928439	103.13\\
12.55995169	103.14\\
12.575923845	103.14\\
12.592003013	103.15\\
12.608043873	103.16\\
12.624005494	103.16\\
12.639860188	103.17\\
12.655989234	103.18\\
12.671943718	103.18\\
};
\end{axis}
\end{tikzpicture}%
}
      \caption{The orientation of the robot over time for
        $K_{\Psi}^R = 0.5 K_{\Psi, max}^R$}
      \label{fig:8_0.5_max}
    \end{figure}
  \end{minipage}
  \hfill
  \begin{minipage}{0.45\linewidth}
    \begin{figure}[H]
      \scalebox{0.6}{% This file was created by matlab2tikz.
%
%The latest updates can be retrieved from
%  http://www.mathworks.com/matlabcentral/fileexchange/22022-matlab2tikz-matlab2tikz
%where you can also make suggestions and rate matlab2tikz.
%
\definecolor{mycolor1}{rgb}{0.00000,0.44700,0.74100}%
%
\begin{tikzpicture}

\begin{axis}[%
width=4.133in,
height=3.26in,
at={(0.693in,0.44in)},
scale only axis,
xmin=6,
xmax=13,
xmajorgrids,
xlabel={Time (seconds)},
ymin=102,
ymax=104,
ymajorgrids,
ylabel={Angle (degrees)},
axis background/.style={fill=white}
]
\addplot [color=mycolor1,solid,forget plot]
  table[row sep=crcr]{%
6.01745702	102.93\\
6.032845836	102.94\\
6.04805238699999	102.94\\
6.063957289	102.95\\
6.082284774	102.96\\
6.09759010699999	102.96\\
6.112709088	102.97\\
6.12793768699999	102.97\\
6.143926949	102.98\\
6.159966228	102.99\\
6.175967424	102.99\\
6.192003619	103\\
6.208084781	103.01\\
6.22396189999999	103.01\\
6.24000836399999	103.02\\
6.25605751899999	103.02\\
6.271970495	103.03\\
6.287991134	103.04\\
6.303947448	103.04\\
6.320009853	103.05\\
6.336146941	103.06\\
6.351994419	103.06\\
6.36798011499999	103.07\\
6.383978368	103.08\\
6.400003057	103.08\\
6.415959993	103.09\\
6.432074155	103.09\\
6.44795572499999	103.1\\
6.463948491	103.11\\
6.47995638699999	103.11\\
6.495972928	103.12\\
6.512233309	103.13\\
6.52797965899999	103.13\\
6.544050007	103.14\\
6.55992439299999	103.15\\
6.575794397	103.15\\
6.59196289	103.16\\
6.608508418	103.16\\
6.623978094	103.17\\
6.63998456199999	103.18\\
6.655961984	103.18\\
6.671931617	103.19\\
6.687953346	103.2\\
6.70395510699999	103.2\\
6.719989649	103.21\\
6.735970413	103.21\\
6.751955006	103.22\\
6.76798170999999	103.23\\
6.783796693	103.23\\
6.799806638	103.24\\
6.816006265	103.25\\
6.831996536	103.25\\
6.84789128	103.26\\
6.863881506	103.27\\
6.882360961	103.27\\
6.89790377699999	103.28\\
6.913106513	103.29\\
6.928365766	103.29\\
6.944330735	103.3\\
6.95992589	103.3\\
6.975920546	103.31\\
6.99180443699999	103.32\\
7.009213007	103.32\\
7.02428542900001	103.31\\
7.039976065	103.31\\
7.05596627499999	103.3\\
7.072197948	103.29\\
7.08799878	103.29\\
7.10398408	103.28\\
7.12003390799999	103.28\\
7.13597204899999	103.27\\
7.15201500499999	103.26\\
7.167883142	103.26\\
7.183939213	103.25\\
7.199985487	103.24\\
7.215953903	103.24\\
7.232049889	103.23\\
7.24802769599999	103.22\\
7.263985028	103.22\\
7.280239598	103.21\\
7.295960303	103.21\\
7.314261552	103.2\\
7.329542802	103.19\\
7.344755124	103.19\\
7.359976061	103.18\\
7.37595684999999	103.17\\
7.391998698	103.17\\
7.407969375	103.16\\
7.42397179899999	103.16\\
7.43995039299999	103.15\\
7.45600047	103.14\\
7.47194293699999	103.14\\
7.487996517	103.13\\
7.50401659999999	103.12\\
7.520080955	103.12\\
7.53601289299999	103.11\\
7.551836948	103.1\\
7.567937418	103.1\\
7.58393684999999	103.09\\
7.60003667399999	103.09\\
7.61595628	103.08\\
7.63205367399999	103.07\\
7.647980065	103.07\\
7.663949544	103.06\\
7.682464554	103.05\\
7.697774668	103.05\\
7.71305699599999	103.04\\
7.728424764	103.03\\
7.74398346099999	103.03\\
7.759926131	103.02\\
7.77606992299999	103.02\\
7.79193990199999	103.01\\
7.808055116	103\\
7.82398930499999	103\\
7.83997642	102.99\\
7.855979344	102.98\\
7.87192945899999	102.98\\
7.88807040199999	102.97\\
7.903944329	102.96\\
7.920087531	102.96\\
7.93602487199999	102.95\\
7.952012377	102.95\\
7.967931847	102.94\\
7.98402269	102.93\\
8.00000768099999	102.93\\
8.01588256	102.93\\
8.031952756	102.93\\
8.04798336	102.94\\
8.063950764	102.95\\
8.08040924599999	102.95\\
8.09596215099999	102.96\\
8.114296516	102.97\\
8.129466462	102.97\\
8.14459689399999	102.98\\
8.159953089	102.99\\
8.17608321600001	102.99\\
8.191985462	103\\
8.208002413	103\\
8.224021029	103.01\\
8.241678685	103.02\\
8.25742670199999	103.02\\
8.27250796699999	103.03\\
8.288272882	103.04\\
8.303987117	103.04\\
8.31991531199999	103.05\\
8.33595715099999	103.05\\
8.351956132	103.06\\
8.367970939	103.07\\
8.383948375	103.07\\
8.400004966	103.08\\
8.415965219	103.09\\
8.43198973800001	103.09\\
8.447946207	103.1\\
8.46395063500001	103.11\\
8.48027256999999	103.11\\
8.49596410699999	103.12\\
8.51432398999999	103.13\\
8.52955379700001	103.13\\
8.544714377	103.14\\
8.56005293099999	103.14\\
8.575938685	103.15\\
8.591799786	103.16\\
8.607947957	103.16\\
8.623992932	103.17\\
8.64010945899999	103.18\\
8.655949565	103.18\\
8.671984147	103.19\\
8.687972447	103.19\\
8.70396498899999	103.2\\
8.719996258	103.21\\
8.736027171	103.21\\
8.75188036499999	103.22\\
8.76795265799999	103.23\\
8.783945265	103.23\\
8.800048712	103.24\\
8.816024891	103.25\\
8.832102244	103.25\\
8.84794421600001	103.26\\
8.863953619	103.26\\
8.880301015	103.27\\
8.89598485799999	103.28\\
8.91202697	103.28\\
8.92795733499999	103.29\\
8.944041288	103.3\\
8.95984452599999	103.3\\
8.97589579799999	103.31\\
8.991871899	103.31\\
9.009354462	103.32\\
9.02456673999999	103.31\\
9.04004298899999	103.31\\
9.055997293	103.3\\
9.07196044	103.29\\
9.08801074699999	103.29\\
9.10393895199999	103.28\\
9.119986257	103.27\\
9.135975721	103.27\\
9.15204366499999	103.26\\
9.167948983	103.26\\
9.183978559	103.25\\
9.20228556999999	103.24\\
9.21734012	103.24\\
9.232504619	103.23\\
9.247932897	103.22\\
9.263954273	103.22\\
9.282294271	103.21\\
9.29738893099999	103.2\\
9.31254101799999	103.2\\
9.327963117	103.19\\
9.343951005	103.19\\
9.36000622399999	103.18\\
9.37594868099999	103.17\\
9.392016348	103.17\\
9.407989671	103.16\\
9.424279295	103.15\\
9.440004846	103.15\\
9.455975272	103.14\\
9.471955876	103.13\\
9.487986248	103.13\\
9.50396090799999	103.12\\
9.520081645	103.12\\
9.53594598099999	103.11\\
9.55202454700001	103.1\\
9.56800059100001	103.1\\
9.583953623	103.09\\
9.60004175499999	103.08\\
9.615961892	103.08\\
9.63213210599999	103.07\\
9.648050539	103.06\\
9.663978208	103.06\\
9.682444789	103.05\\
9.697681603	103.04\\
9.712805795	103.04\\
9.728073076	103.03\\
9.74396904799999	103.03\\
9.759879589	103.02\\
9.77601393699999	103.01\\
9.791991089	103.01\\
9.80783461499999	103\\
9.826114412	102.99\\
9.841650749	102.99\\
9.856896918	102.98\\
9.872106174	102.98\\
9.887970841	102.97\\
9.90399367	102.96\\
9.920035016	102.96\\
9.935958642	102.95\\
9.95200258	102.94\\
9.96792537199999	102.94\\
9.983955037	102.93\\
9.999969483	102.93\\
10.015856343	102.93\\
10.031883321	102.93\\
10.047953476	102.94\\
10.063869806	102.95\\
10.080099513	102.95\\
10.095971833	102.96\\
10.112009993	102.96\\
10.128033729	102.97\\
10.143954527	102.98\\
10.159907158	102.98\\
10.175921528	102.99\\
10.191997491	103\\
10.208109366	103\\
10.223962506	103.01\\
10.239968616	103.02\\
10.255881828	103.02\\
10.27200198	103.03\\
10.287881464	103.03\\
10.304005223	103.04\\
10.320030829	103.05\\
10.335978769	103.05\\
10.352004701	103.06\\
10.367945169	103.07\\
10.383976593	103.07\\
10.400033435	103.08\\
10.415962948	103.09\\
10.432040227	103.09\\
10.447861782	103.1\\
10.463886082	103.1\\
10.4801055	103.11\\
10.495913437	103.12\\
10.511960386	103.12\\
10.527964118	103.13\\
10.543966822	103.14\\
10.559954101	103.14\\
10.575959988	103.15\\
10.59201182	103.16\\
10.607946094	103.16\\
10.623857733	103.17\\
10.639983909	103.17\\
10.65595854	103.18\\
10.671956398	103.19\\
10.688025728	103.19\\
10.703945519	103.2\\
10.720089271	103.21\\
10.736218619	103.21\\
10.751975381	103.22\\
10.767942349	103.22\\
10.783939762	103.23\\
10.800094977	103.24\\
10.81592575	103.24\\
10.832003606	103.25\\
10.847945297	103.26\\
10.863962133	103.26\\
10.880201989	103.27\\
10.89594714	103.28\\
10.911970125	103.28\\
10.927863753	103.29\\
10.943962749	103.29\\
10.95991764	103.3\\
10.975996988	103.31\\
10.992008292	103.31\\
11.009491197	103.32\\
11.024715551	103.31\\
11.039961254	103.3\\
11.055908904	103.3\\
11.071949293	103.29\\
11.087937034	103.29\\
11.103957784	103.28\\
11.119994851	103.27\\
11.135907712	103.27\\
11.151966598	103.26\\
11.167949939	103.25\\
11.184128952	103.25\\
11.200008698	103.24\\
11.215853027	103.23\\
11.231933272	103.23\\
11.247891704	103.22\\
11.263857943	103.22\\
11.27992405	103.21\\
11.295985699	103.2\\
11.312177296	103.2\\
11.327967128	103.19\\
11.343937965	103.18\\
11.359821641	103.18\\
11.375964081	103.17\\
11.391840727	103.17\\
11.40795033	103.16\\
11.423966936	103.15\\
11.439946355	103.15\\
11.455949443	103.14\\
11.472066862	103.13\\
11.487980805	103.13\\
11.503955997	103.12\\
11.520041524	103.11\\
11.53595109	103.11\\
11.55196435	103.1\\
11.570246342	103.09\\
11.585279102	103.09\\
11.600334488	103.08\\
11.615937082	103.08\\
11.63199717	103.07\\
11.6478534	103.06\\
11.663953136	103.06\\
11.680098454	103.05\\
11.695916064	103.04\\
11.712017785	103.04\\
11.727965252	103.03\\
11.743962067	103.03\\
11.759930974	103.02\\
11.77594525	103.01\\
11.792031508	103.01\\
11.807868977	103\\
11.823841854	102.99\\
11.840097998	102.99\\
11.855947069	102.98\\
11.872011444	102.97\\
11.887948188	102.97\\
11.903955227	102.96\\
11.919955531	102.96\\
11.935935494	102.95\\
11.951981822	102.94\\
11.968022502	102.94\\
11.984005191	102.93\\
12.000051675	102.92\\
12.017678164	102.92\\
12.032866753	102.93\\
12.048213844	102.94\\
12.063999627	102.94\\
12.080193658	102.95\\
12.095951505	102.95\\
12.11200541	102.96\\
12.127963955	102.97\\
12.144193017	102.97\\
12.159944672	102.98\\
12.175956753	102.99\\
12.191942002	102.99\\
12.208013232	103\\
12.223974131	103.01\\
12.239939601	103.01\\
12.255942476	103.02\\
12.271909985	103.02\\
12.287976098	103.03\\
12.303858823	103.04\\
12.320051801	103.04\\
12.33593982	103.05\\
12.352007471	103.06\\
12.367905764	103.06\\
12.383961052	103.07\\
12.400094945	103.08\\
12.415951684	103.08\\
12.43203739	103.09\\
12.447947975	103.09\\
12.463953481	103.1\\
12.482375089	103.11\\
12.497695734	103.11\\
12.513055403	103.12\\
12.528469538	103.13\\
12.543928439	103.13\\
12.55995169	103.14\\
12.575923845	103.14\\
12.592003013	103.15\\
12.608043873	103.16\\
12.624005494	103.16\\
12.639860188	103.17\\
12.655989234	103.18\\
12.671943718	103.18\\
};
\end{axis}
\end{tikzpicture}%
}
      \caption{The steady state orientation of the robot for
        $K_{\Psi}^R = 0.5 K_{\Psi, max}^R$}
      \label{fig:8_0.5_max_magnified}
    \end{figure}
  \end{minipage}
\end{minipage}
}

\noindent\makebox[\textwidth][c]{%
\begin{minipage}{\linewidth}
  \begin{minipage}{0.45\linewidth}
    \begin{figure}[H]
      \scalebox{0.6}{% This file was created by matlab2tikz.
%
%The latest updates can be retrieved from
%  http://www.mathworks.com/matlabcentral/fileexchange/22022-matlab2tikz-matlab2tikz
%where you can also make suggestions and rate matlab2tikz.
%
\definecolor{mycolor1}{rgb}{0.00000,0.44700,0.74100}%
%
\begin{tikzpicture}

\begin{axis}[%
width=4.133in,
height=3.26in,
at={(0.693in,0.44in)},
scale only axis,
xmin=0,
xmax=19,
xmajorgrids,
ymin=0,
ymax=160,
ymajorgrids,
axis background/.style={fill=white}
]
\addplot [color=mycolor1,solid,forget plot]
  table[row sep=crcr]{%
0	0\\
0.0182288320000023	0.85\\
0.0331749149999991	3.23\\
0.0481669790000021	5.53\\
0.0636149580000006	7.83\\
0.0796094020010009	10.19\\
0.0955824979999997	12.65\\
0.111696319000002	15.11\\
0.127601390999999	17.6\\
0.143713008001	20.04\\
0.159778670000002	22.51\\
0.175692708000001	25.01\\
0.191676530999999	27.44\\
0.207817142999999	29.92\\
0.223637052000001	32.36\\
0.239747409000002	34.78\\
0.255689962000001	37.24\\
0.271530411000002	39.72\\
0.287679194000002	42.11\\
0.303722962000001	44.62\\
0.319760923	47.09\\
0.335856275000002	49.52\\
0.351704664000001	52\\
0.367747718000002	54.42\\
0.383533931000002	56.86\\
0.399569867001001	59.29\\
0.415515673	61.76\\
0.433847346999999	64.69\\
0.449054166999999	67.02\\
0.464571094999	69.4\\
0.479810434	71.73\\
0.495602515999999	74.08\\
0.511754009	76.52\\
0.527665068	78.97\\
0.543757801001001	81.42\\
0.559778476	83.91\\
0.575752055	86.33\\
0.591713585999999	88.79\\
0.607700949999999	91.27\\
0.623653275000002	93.7\\
0.639777153999999	96.13\\
0.655591296	98.61\\
0.671749923	101.03\\
0.688149151999999	103.69\\
0.703679081	106.05\\
0.719661802999999	108.44\\
0.7356835	110.89\\
0.751612980999999	113.34\\
0.770087521000002	116.27\\
0.785360686999999	118.6\\
0.800636302000001	120.95\\
0.815964943	123.3\\
0.831901903999001	125.74\\
0.847656693999	128.1\\
0.863692634999	130.49\\
0.879679519000001	132.96\\
0.895636093001001	135.41\\
0.911723824999999	137.87\\
0.927685176001001	140.31\\
0.943728554000001	142.79\\
0.959665602000002	145.23\\
0.975658900000002	147.7\\
0.991677039999999	150.13\\
1.007803073	152.59\\
1.025327839	153.42\\
1.040320836	152.31\\
1.055548305	151.2\\
1.071578242	150.07\\
1.087572811999	148.88\\
1.103692657001	147.69\\
1.119688047	146.48\\
1.135743943999	145.3\\
1.151515964	144.1\\
1.167740502	142.95\\
1.183650486	141.74\\
1.199734440001	140.56\\
1.215686785	139.36\\
1.231766298	138.19\\
1.247763049	136.99\\
1.263737329	135.81\\
1.279660054	134.61\\
1.295737395	133.44\\
1.311670391	132.2\\
1.327690226	131.08\\
1.343686539	129.88\\
1.359690932	128.7\\
1.375693740999	127.52\\
1.391695273	126.32\\
1.407621899	125.14\\
1.423647488	123.96\\
1.439730051	122.73\\
1.455563167	121.58\\
1.471729495999	120.41\\
1.487725129	119.21\\
1.503611969	118.03\\
1.521695368	116.63\\
1.536606986	115.52\\
1.551723227	114.4\\
1.567688465	113.29\\
1.583760227	112.09\\
1.599701322	110.9\\
1.615635928	109.71\\
1.631609898999	108.55\\
1.647681919	107.36\\
1.663746526	106.16\\
1.679818624	104.99\\
1.695505417999	103.79\\
1.711509221001	102.63\\
1.72765803	101.43\\
1.743707478	100.24\\
1.759670813	99.05\\
1.775536672	97.87\\
1.791663357	96.7\\
1.807815917	95.49\\
1.823761987001	94.3\\
1.83966953	93.1\\
1.855674304	91.95\\
1.872918429	90.6\\
1.887966957	89.48\\
1.903701001001	88.35\\
1.919573846999	87.18\\
1.935700832999	86.01\\
1.951682100001	84.81\\
1.967816283	83.63\\
1.983618609	82.43\\
2.001994051	81.03\\
2.016720415	80.58\\
2.031714362001	81.06\\
2.047555908	81.54\\
2.063590221001	82.05\\
2.079574196	82.57\\
2.09776278	83.18\\
2.112937753	83.67\\
2.127516634	84.1\\
2.14378221	84.62\\
2.159733633001	85.15\\
2.175770403999	85.66\\
2.19179139	86.17\\
2.207745423	86.69\\
2.223667025999	87.2\\
2.239698007	87.71\\
2.255688443	88.22\\
2.271675323	88.76\\
2.287811271	89.25\\
2.303763438	89.77\\
2.319729663	90.28\\
2.335708725001	90.8\\
2.351664202	91.31\\
2.370106406	91.92\\
2.385455563999	92.42\\
2.400661522	92.9\\
2.416152135	93.4\\
2.431671954	93.89\\
2.447755158	94.39\\
2.463730178	94.91\\
2.47968415	95.42\\
2.495660546	95.93\\
2.511658485	96.45\\
2.527678934	96.96\\
2.543581235001	97.48\\
2.559533288	97.98\\
2.575570484	98.52\\
2.591659030999	99.01\\
2.607665438999	99.53\\
2.623674803001	100.03\\
2.639717002	100.56\\
2.655530248	101.06\\
2.673803534001	101.67\\
2.688986063	102.16\\
2.704278773	102.65\\
2.719612059	103.14\\
2.735613377	103.63\\
2.751724387	104.15\\
2.767782475	104.66\\
2.783695897	105.18\\
2.799881171	105.69\\
2.815747294	106.21\\
2.831686487	106.72\\
2.847732622	107.23\\
2.863734605	107.74\\
2.879705082	108.26\\
2.895709925	108.77\\
2.91165914	109.29\\
2.929782466001	109.89\\
2.94368791	110.31\\
2.959855989	110.86\\
2.975718624999	111.34\\
2.991566431	111.86\\
3.007575669999	112.36\\
3.025593690999	112.24\\
3.040796311	112.02\\
3.055970386	111.8\\
3.071821096	111.58\\
3.08762299	111.36\\
3.103575740999	111.14\\
3.119720071	110.91\\
3.135822622	110.68\\
3.151712751001	110.45\\
3.167785889	110.22\\
3.183666462	109.99\\
3.199727463	109.76\\
3.215709628	109.53\\
3.23170693	109.3\\
3.247712557	109.08\\
3.263561407	108.85\\
3.279704007	108.62\\
3.295691424999	108.39\\
3.311615027001	108.16\\
3.327669867001	107.94\\
3.343724867001	107.71\\
3.35968618	107.48\\
3.37571995	107.25\\
3.391682252999	107.02\\
3.407648294	106.79\\
3.42380624	106.57\\
3.43966732	106.33\\
3.455739536	106.11\\
3.471720029	105.88\\
3.487742142	105.65\\
3.503712648	105.42\\
3.519672485	105.2\\
3.535645637	104.97\\
3.55165866	104.74\\
3.570077933	104.47\\
3.585188379	104.25\\
3.600362976	104.04\\
3.61584188	103.82\\
3.631751501001	103.59\\
3.647762724	103.37\\
3.664859723001	103.12\\
3.679943271	102.9\\
3.695733061	102.68\\
3.711674547	102.46\\
3.727598068	102.23\\
3.743751287	102\\
3.759694807	101.77\\
3.775783929999	101.54\\
3.791501952001	101.31\\
3.807545221	101.09\\
3.823795112	100.86\\
3.839796653	100.63\\
3.855821179	100.4\\
3.871697745	100.17\\
3.887675022	99.95\\
3.903708787	99.72\\
3.919742963	99.49\\
3.935660295	99.26\\
3.951666415	99.03\\
3.967712173	98.8\\
3.983711061999	98.57\\
4.002208672	98.3\\
4.017746804	98.21\\
4.032951979	98.31\\
4.048927111	98.41\\
4.064260662	98.51\\
4.07972116	98.61\\
4.095516144	98.71\\
4.111595119001	98.81\\
4.127720381001	98.91\\
4.143780893	99.01\\
4.159700206	99.11\\
4.175818984	99.21\\
4.19178739	99.32\\
4.207741003	99.42\\
4.22368991	99.52\\
4.239619232	99.62\\
4.255657302	99.72\\
4.272410061999	99.83\\
4.287730815	99.93\\
4.303721191	100.02\\
4.319627873	100.13\\
4.335656495	100.23\\
4.351665359	100.33\\
4.367820709001	100.43\\
4.383655768001	100.53\\
4.399979101	100.63\\
4.415694233	100.74\\
4.431657632	100.84\\
4.447684124	100.94\\
4.463957653	101.04\\
4.480614074	101.15\\
4.49586797	101.25\\
4.511572126	101.35\\
4.527696123	101.45\\
4.543714287	101.55\\
4.559689829001	101.65\\
4.575773405	101.75\\
4.591659628	101.85\\
4.60771482	101.95\\
4.623682019	102.06\\
4.639750147	102.16\\
4.655658287	102.26\\
4.671752222	102.36\\
4.687607334	102.46\\
4.703740482	102.56\\
4.719639261	102.66\\
4.735822323	102.76\\
4.751594387	102.87\\
4.767757879	102.97\\
4.783579882	103.07\\
4.799749929999	103.17\\
4.815708266	103.27\\
4.831749667	103.37\\
4.847679749	103.48\\
4.866089553	103.6\\
4.880364425	103.69\\
4.895533226	103.78\\
4.911527924999	103.88\\
4.927482078	103.98\\
4.943818049	104.09\\
4.959781403999	104.19\\
4.975561235	104.29\\
4.991691962	104.39\\
5.007702341	104.49\\
5.023667798	104.51\\
5.039644419	104.48\\
5.055695904	104.45\\
5.071652367	104.41\\
5.087712657999	104.38\\
5.103561521	104.35\\
5.119663821	104.32\\
5.135657917	104.29\\
5.151701436	104.26\\
5.167871982	104.22\\
5.183616287999	104.19\\
5.199762592	104.16\\
5.215694155	104.13\\
5.23170996	104.1\\
5.247692383	104.07\\
5.263652664	104.03\\
5.279824272	104\\
5.295753375	103.97\\
5.311816957001	103.94\\
5.327646458	103.91\\
5.343568342	103.88\\
5.359524338	103.84\\
5.375725282001	103.81\\
5.391681516	103.78\\
5.407733785	103.75\\
5.423726253	103.72\\
5.439707755	103.68\\
5.455704355999	103.65\\
5.471731544	103.62\\
5.487663994	103.59\\
5.503701655	103.56\\
5.519576981	103.53\\
5.535730557	103.5\\
5.551628824001	103.46\\
5.568015761999	103.43\\
5.583647335	103.4\\
5.599832853	103.37\\
5.615626609	103.34\\
5.631740159001	103.3\\
5.647950177	103.27\\
5.663756889	103.24\\
5.679640061	103.21\\
5.695789624	103.18\\
5.711781458	103.14\\
5.7276325	103.12\\
5.743530473001	103.08\\
5.75967682	103.05\\
5.775589198	103.02\\
5.791550831	102.99\\
5.80771883	102.96\\
5.823674086	102.92\\
5.839550179	102.89\\
5.855625439	102.86\\
5.871726725001	102.83\\
5.887685950999	102.8\\
5.903570193	102.77\\
5.919760542999	102.73\\
5.936025735	102.7\\
5.951662037	102.67\\
5.967774466999	102.64\\
5.983617648999	102.61\\
5.999584672	102.58\\
6.017815105	102.55\\
6.032956134999	102.56\\
6.048245563	102.57\\
6.063754520001	102.57\\
6.081292823	102.58\\
6.096330447001	102.58\\
6.112968561999	102.59\\
6.128230852	102.6\\
6.144811522	102.6\\
6.160085691	102.61\\
6.175512804	102.62\\
6.191708144	102.62\\
6.207638936	102.63\\
6.223682132	102.63\\
6.239775817	102.64\\
6.255672501	102.65\\
6.271731107	102.65\\
6.287649837999	102.66\\
6.303522429999	102.67\\
6.319573955	102.67\\
6.335642075999	102.68\\
6.35170369	102.69\\
6.3677814	102.69\\
6.383672656	102.7\\
6.399850016	102.7\\
6.415675068	102.71\\
6.431693487	102.72\\
6.447621577	102.72\\
6.46367436	102.73\\
6.479704081	102.74\\
6.495634719	102.74\\
6.511660266	102.75\\
6.527682601	102.76\\
6.543592383	102.76\\
6.559585781	102.77\\
6.575717856	102.77\\
6.591695733	102.78\\
6.607704897	102.79\\
6.623654772	102.79\\
6.639546494001	102.8\\
6.655515477	102.81\\
6.671703519	102.81\\
6.687753240999	102.82\\
6.703746339001	102.83\\
6.719663229	102.83\\
6.735644060001	102.84\\
6.751692376001	102.84\\
6.767569525	102.85\\
6.783619440001	102.86\\
6.800271474	102.86\\
6.816073939	102.87\\
6.831862918	102.88\\
6.847687469	102.88\\
6.863671739	102.89\\
6.879691060001	102.89\\
6.895777694	102.9\\
6.911723688	102.91\\
6.927693658	102.91\\
6.943500634	102.92\\
6.959611567	102.93\\
6.975682763	102.93\\
6.991635162999	102.94\\
7.007694566	102.95\\
7.025070903999	102.95\\
7.040524263001	102.96\\
7.055690596	102.97\\
7.071554278	102.97\\
7.087555631	102.98\\
7.103684530999	102.98\\
7.119669961	102.99\\
7.135707082999	103\\
7.151746719	103\\
7.170248561	103.01\\
7.185562755	103.02\\
7.200743554	103.02\\
7.216280932	103.03\\
7.231672853999	103.03\\
7.247832365	103.04\\
7.263724006	103.05\\
7.279565987	103.05\\
7.295827030999	103.06\\
7.311730275001	103.07\\
7.327655984	103.07\\
7.34374534	103.08\\
7.359678191	103.09\\
7.375782957999	103.09\\
7.391667795	103.1\\
7.40775505	103.1\\
7.423693237001	103.11\\
7.439745509	103.12\\
7.455686293	103.12\\
7.471560005	103.13\\
7.487495520999	103.14\\
7.503618528	103.14\\
7.519697836	103.15\\
7.535689696	103.15\\
7.551747979	103.16\\
7.567776804999	103.17\\
7.583717476	103.17\\
7.599714219	103.18\\
7.615549060001	103.19\\
7.631717309	103.19\\
7.64767117	103.2\\
7.663595843001	103.21\\
7.679499892	103.21\\
7.697854024	103.22\\
7.713120718	103.23\\
7.728287228	103.23\\
7.743638435	103.24\\
7.759725153	103.24\\
7.775794875	103.25\\
7.791776553	103.26\\
7.807685411	103.26\\
7.823643752	103.27\\
7.839745476	103.28\\
7.855678992999	103.28\\
7.871955497999	103.29\\
7.887750192001	103.29\\
7.903800829	103.3\\
7.919702079001	103.31\\
7.935604414	103.31\\
7.951658039	103.32\\
7.967731305	103.33\\
7.983626844	103.33\\
7.999807832999	103.34\\
8.020538862001	103.34\\
8.032428537001	103.33\\
8.047797785	103.32\\
8.063756409	103.3\\
8.079751756	103.29\\
8.095736054	103.28\\
8.111628278	103.27\\
8.127651497999	103.25\\
8.143701735	103.24\\
8.159659478	103.23\\
8.175791139	103.21\\
8.191698407	103.2\\
8.207587485001	103.19\\
8.223705374	103.18\\
8.239665221	103.16\\
8.255711739	103.15\\
8.273143879001	103.14\\
8.288173367	103.12\\
8.303621724	103.11\\
8.319694327	103.1\\
8.335708297	103.09\\
8.354027563	103.07\\
8.369210969	103.06\\
8.384392152	103.05\\
8.399834689	103.04\\
8.415705992	103.02\\
8.431752197001	103.01\\
8.447560138	103\\
8.463793990999	102.99\\
8.479645476	102.97\\
8.495671495999	102.96\\
8.511780766	102.95\\
8.527655278999	102.94\\
8.544773933	102.92\\
8.560126548	102.91\\
8.575754121	102.9\\
8.591666156	102.88\\
8.607641551001	102.87\\
8.623665740999	102.86\\
8.639496241	102.85\\
8.655529002	102.83\\
8.671694257	102.82\\
8.687609847	102.81\\
8.703653539001	102.8\\
8.719687018	102.78\\
8.735734333	102.77\\
8.751576198	102.76\\
8.767815139	102.75\\
8.783724148	102.73\\
8.799775558	102.72\\
8.815680402	102.71\\
8.831639624001	102.69\\
8.847662226	102.68\\
8.863595004001	102.67\\
8.879576392	102.66\\
8.897893839	102.64\\
8.913202397	102.63\\
8.92829041	102.62\\
8.94371067	102.61\\
8.959640589001	102.59\\
8.975708661	102.58\\
8.991666052	102.57\\
9.007692686	102.55\\
9.025160797	102.55\\
9.040212579001	102.56\\
9.055669958	102.56\\
9.071554785999	102.57\\
9.089787545	102.58\\
9.105056673	102.58\\
9.120333575	102.59\\
9.135903358	102.59\\
9.151614917	102.6\\
9.167760507	102.61\\
9.183647209	102.61\\
9.199853106	102.62\\
9.215520778	102.63\\
9.231655545	102.63\\
9.247657392	102.64\\
9.263764538	102.65\\
9.27960798	102.65\\
9.295690654	102.66\\
9.313737328	102.67\\
9.329306511	102.67\\
9.344639928	102.68\\
9.359747753	102.68\\
9.376376211	102.69\\
9.391690529	102.7\\
9.407715529	102.7\\
9.423669257	102.71\\
9.439547961	102.71\\
9.45567642	102.72\\
9.471678493	102.73\\
9.487593485001	102.73\\
9.503675537	102.74\\
9.519665888001	102.75\\
9.535768091	102.75\\
9.551636668	102.76\\
9.567838039	102.77\\
9.583695701	102.77\\
9.602041199	102.78\\
9.617172347	102.79\\
9.632479574	102.79\\
9.647906297	102.8\\
9.664580375	102.8\\
9.679865153	102.81\\
9.695685983	102.82\\
9.71165668	102.82\\
9.727682541	102.83\\
9.74371141	102.84\\
9.759682417	102.84\\
9.775698371	102.85\\
9.791664262	102.85\\
9.807636373	102.86\\
9.823626948	102.87\\
9.840281952001	102.87\\
9.855640641	102.88\\
9.87154053	102.89\\
9.887717485999	102.89\\
9.903738823	102.9\\
9.919730717	102.91\\
9.935786437	102.91\\
9.951695659001	102.92\\
9.967739775	102.92\\
9.983592219	102.93\\
9.999934565	102.94\\
10.016015919	102.94\\
10.031633335	102.95\\
10.047741526	102.96\\
10.063804096	102.96\\
10.079694296001	102.97\\
10.095799351	102.97\\
10.111570058	102.98\\
10.127735619	102.99\\
10.143761952	102.99\\
10.159575868	103\\
10.175652087999	103.01\\
10.191579703	103.01\\
10.207631266	103.02\\
10.223560008	103.03\\
10.239734978999	103.03\\
10.25577828	103.04\\
10.271525993	103.04\\
10.287596237	103.05\\
10.303701419	103.06\\
10.319634646	103.06\\
10.335844014	103.07\\
10.351593559	103.08\\
10.367889598	103.08\\
10.383658768	103.09\\
10.399756181	103.1\\
10.415688568	103.1\\
10.431580551	103.11\\
10.447570909	103.11\\
10.463674214	103.12\\
10.479642014	103.13\\
10.495898476	103.13\\
10.511499045	103.14\\
10.527667445001	103.15\\
10.543713384	103.15\\
10.559659556	103.16\\
10.575610691	103.17\\
10.591696143	103.17\\
10.607696877999	103.18\\
10.623711434	103.18\\
10.639710941	103.19\\
10.655631925	103.2\\
10.671518410999	103.2\\
10.689831565001	103.21\\
10.704989155999	103.22\\
10.720087515	103.22\\
10.735694372	103.23\\
10.751711855	103.23\\
10.768009071	103.24\\
10.783714758001	103.25\\
10.79970596	103.25\\
10.815568581	103.26\\
10.831687294	103.27\\
10.847713552	103.27\\
10.863894976	103.28\\
10.879650258	103.29\\
10.895778773999	103.29\\
10.9116898	103.3\\
10.927694801	103.3\\
10.943711429999	103.31\\
10.959671393	103.32\\
10.975718984	103.32\\
10.991667836001	103.33\\
11.007710601	103.34\\
11.025385737	103.33\\
11.040500136999	103.32\\
11.05582761	103.31\\
11.071615712999	103.3\\
11.087556479	103.28\\
11.103633144	103.27\\
11.119666532	103.26\\
11.136046648	103.25\\
11.151706825	103.23\\
11.168013583	103.22\\
11.18382118	103.21\\
11.199714256	103.2\\
11.215626174	103.18\\
11.231733671	103.17\\
11.247702679	103.16\\
11.263715829	103.14\\
11.279704136	103.13\\
11.29837435	103.12\\
11.313763245	103.1\\
11.329075709	103.09\\
11.344391313	103.08\\
11.359708523999	103.07\\
11.375689636	103.06\\
11.391620515	103.04\\
11.407852745	103.03\\
11.423544903	103.02\\
11.439719809	103.01\\
11.455632410999	102.99\\
11.471538961	102.98\\
11.487642333	102.97\\
11.5037961	102.95\\
11.519710193	102.94\\
11.535804037	102.93\\
11.551968504	102.92\\
11.567972392	102.9\\
11.583754596	102.89\\
11.602370836	102.88\\
11.617675942001	102.86\\
11.633039408	102.85\\
11.648381461	102.84\\
11.664308848001	102.83\\
11.679732407999	102.81\\
11.695750117	102.8\\
11.711716936	102.79\\
11.727687534001	102.78\\
11.743674856	102.76\\
11.759702637	102.75\\
11.775749447001	102.74\\
11.791705278999	102.73\\
11.807574401	102.71\\
11.823699593	102.7\\
11.839720458	102.69\\
11.855684234	102.68\\
11.871573584001	102.66\\
11.887726436	102.65\\
11.90373349	102.64\\
11.919749058	102.62\\
11.93579258	102.61\\
11.951695112	102.6\\
11.970185148999	102.58\\
11.985573914999	102.57\\
12.000919537	102.56\\
12.01659102	102.56\\
12.031820926999	102.56\\
12.047653479	102.57\\
12.063781653	102.57\\
12.07964014	102.58\\
12.095638278	102.59\\
12.111687666	102.59\\
12.12764863	102.6\\
12.143724846001	102.61\\
12.159672476	102.61\\
12.175707824	102.62\\
12.191689310001	102.62\\
12.207696534	102.63\\
12.223707476999	102.64\\
12.239727117	102.64\\
12.255709022	102.65\\
12.271710824001	102.66\\
12.287705971	102.66\\
12.303724165	102.67\\
12.319702414	102.68\\
12.335603520001	102.68\\
12.351806235999	102.69\\
12.367656461001	102.69\\
12.383684952	102.7\\
12.399710327999	102.71\\
12.415720158	102.71\\
12.431725639	102.72\\
12.447669955	102.73\\
12.46361005	102.73\\
12.479828626	102.74\\
12.495943216	102.75\\
12.511638828	102.75\\
12.527622929	102.76\\
12.54366311	102.76\\
12.559641607	102.77\\
12.575618551	102.78\\
12.591712211001	102.78\\
12.607658625	102.79\\
12.623699967	102.8\\
12.639662399	102.8\\
12.655546898999	102.81\\
12.671706014	102.82\\
12.687645400999	102.82\\
12.703725014	102.83\\
12.719704811	102.83\\
12.735718421001	102.84\\
12.751682518	102.85\\
12.767596836	102.85\\
12.783923234	102.86\\
12.799724908	102.87\\
12.815721734	102.87\\
12.831737477	102.88\\
12.847678506	102.88\\
12.863733033	102.89\\
12.879665921	102.9\\
12.895768866	102.9\\
12.911621504999	102.91\\
12.927592183999	102.92\\
12.943662946	102.92\\
12.959748957999	102.93\\
12.975798903999	102.94\\
12.991688356	102.94\\
13.007702236	102.95\\
13.025434183	102.96\\
13.040779127	102.96\\
13.055937303	102.97\\
13.071575109	102.97\\
13.087751684	102.98\\
13.103694191	102.99\\
13.119747951	102.99\\
13.135678962999	103\\
13.152181453	103.01\\
13.167668103	103.01\\
13.183874741	103.02\\
13.199729675	103.02\\
13.215752259999	103.03\\
13.231591892	103.04\\
13.249938276	103.04\\
13.265259035	103.05\\
13.280645164	103.06\\
13.296002154	103.06\\
13.311672702001	103.07\\
13.327661744001	103.08\\
13.343663475001	103.08\\
13.359673476	103.09\\
13.3756062	103.09\\
13.39181473	103.1\\
13.407789992	103.11\\
13.423738999999	103.11\\
13.439681781	103.12\\
13.455722138	103.13\\
13.471651157	103.13\\
13.487745226	103.14\\
13.503597648999	103.14\\
13.519625030999	103.15\\
13.535603887	103.16\\
13.553949881	103.17\\
13.569026929	103.17\\
13.584174849	103.18\\
13.599690824001	103.18\\
13.615726549	103.19\\
13.631733948	103.2\\
13.648943068	103.2\\
13.66411992	103.21\\
13.679697571	103.21\\
13.695916915	103.22\\
13.711637806	103.23\\
13.727727482	103.23\\
13.743776039999	103.24\\
13.75977603	103.25\\
13.775673549	103.25\\
13.791704346	103.26\\
13.807700145	103.27\\
13.823746898999	103.27\\
13.839714860001	103.28\\
13.855714602	103.28\\
13.872988717	103.29\\
13.888162609	103.3\\
13.90372908	103.3\\
13.919737303	103.31\\
13.935661713	103.32\\
13.95178071	103.32\\
13.967660928001	103.33\\
13.986125422	103.34\\
14.001382115	103.34\\
14.017222538	103.34\\
14.032466009	103.33\\
14.047616221001	103.32\\
14.063725817	103.3\\
14.079753894	103.29\\
14.095682311	103.28\\
14.111712263999	103.27\\
14.127669848001	103.25\\
14.143506954999	103.24\\
14.159710587	103.23\\
14.175659975001	103.22\\
14.192098186	103.2\\
14.207556248	103.19\\
14.223767364	103.18\\
14.239674435	103.16\\
14.255688966	103.15\\
14.27166965	103.14\\
14.287686599	103.13\\
14.303674563001	103.11\\
14.319787096	103.1\\
14.335650699001	103.09\\
14.351932497	103.08\\
14.367651056	103.06\\
14.386060267	103.05\\
14.401443697	103.04\\
14.416752617999	103.02\\
14.43219208	103.01\\
14.44770073	103\\
14.46373714	102.99\\
14.479664278	102.97\\
14.49567932	102.96\\
14.511611693999	102.95\\
14.527690572	102.94\\
14.543681454	102.92\\
14.559709421001	102.91\\
14.575508599	102.9\\
14.591740663	102.89\\
14.60763122	102.87\\
14.62376315	102.86\\
14.639693344	102.85\\
14.655680985	102.83\\
14.671697013	102.82\\
14.687623413	102.81\\
14.703597864	102.8\\
14.719638546	102.78\\
14.735754669	102.77\\
14.751688499	102.76\\
14.76769374	102.75\\
14.784016825999	102.73\\
14.799655099	102.72\\
14.815807028	102.71\\
14.831632933	102.7\\
14.847693936	102.68\\
14.863661852	102.67\\
14.879753517	102.66\\
14.895735108	102.64\\
14.911668951	102.63\\
14.927727273	102.62\\
14.94371841	102.61\\
14.959661083	102.59\\
14.977986032	102.58\\
14.993191704	102.57\\
15.00921932	102.55\\
15.026913318	102.55\\
15.039677829001	102.56\\
15.055740806999	102.56\\
15.071684954	102.57\\
15.087776086	102.57\\
15.103682419	102.58\\
15.119723034001	102.59\\
15.135674653	102.59\\
15.151724887	102.6\\
15.167662311999	102.61\\
15.183730734	102.61\\
15.199650921999	102.62\\
15.215805804999	102.63\\
15.231731891	102.63\\
15.24765221	102.64\\
15.263717184	102.64\\
15.279710231	102.65\\
15.295677535	102.66\\
15.311756737	102.66\\
15.327733152	102.67\\
15.343672222	102.68\\
15.359713479	102.68\\
15.37564682	102.69\\
15.391708995	102.69\\
15.407739534	102.7\\
15.423768756	102.71\\
15.439649442001	102.71\\
15.455740818	102.72\\
15.47166707	102.73\\
15.487791029	102.73\\
15.503740758	102.74\\
15.519669214	102.75\\
15.535600727	102.75\\
15.551652698	102.76\\
15.567587516999	102.76\\
15.58385214	102.77\\
15.599661308001	102.78\\
15.617999029	102.78\\
15.633215481	102.79\\
15.648458939	102.8\\
15.663917439	102.8\\
15.679827148	102.81\\
15.695780581	102.82\\
15.711853278	102.82\\
15.727621831	102.83\\
15.743673547	102.83\\
15.759699719	102.84\\
15.775655683	102.85\\
15.791806097	102.85\\
15.807689866	102.86\\
15.823721035	102.87\\
15.839704003	102.87\\
15.855696671	102.88\\
15.871661609	102.89\\
15.887704219	102.89\\
15.903722879999	102.9\\
15.919724737	102.9\\
15.935684038	102.91\\
15.951678004999	102.92\\
15.967690019	102.92\\
15.983920706	102.93\\
15.999692492001	102.94\\
16.020569018	102.94\\
16.032566767	102.95\\
16.047793522	102.96\\
16.06382477	102.96\\
16.079656177	102.97\\
16.095767386999	102.97\\
16.111691849	102.98\\
16.127813883	102.99\\
16.143639976	102.99\\
16.159616718001	103\\
16.175842383001	103.01\\
16.191701527	103.01\\
16.207752976	103.02\\
16.223692561999	103.02\\
16.239691478	103.03\\
16.255711908	103.04\\
16.271510041001	103.04\\
16.289807474	103.05\\
16.304999375	103.06\\
16.32012901	103.06\\
16.335604162	103.07\\
16.351825627	103.08\\
16.367582990999	103.08\\
16.386176858	103.09\\
16.401475719	103.1\\
16.4168202	103.1\\
16.432197186	103.11\\
16.447746538	103.11\\
16.463683247	103.12\\
16.479709064	103.13\\
16.495778153	103.13\\
16.511698496	103.14\\
16.527582847	103.15\\
16.543701874999	103.15\\
16.559537226	103.16\\
16.575698033	103.16\\
16.591699206	103.17\\
16.60769364	103.18\\
16.623668027001	103.18\\
16.639576368	103.19\\
16.658158095	103.2\\
16.673404795	103.2\\
16.688717575	103.21\\
16.703976616	103.22\\
16.719687917	103.22\\
16.735672711	103.23\\
16.751639886	103.23\\
16.767697483	103.24\\
16.786053051001	103.25\\
16.801208936	103.25\\
16.816320147	103.26\\
16.831620586	103.27\\
16.847663471	103.27\\
16.863653061	103.28\\
16.87976373	103.28\\
16.895693203	103.29\\
16.911788482	103.3\\
16.92774195	103.3\\
16.943675346999	103.31\\
16.959690424	103.32\\
16.975708057	103.32\\
16.99181857	103.33\\
17.007685141	103.34\\
17.025389207	103.33\\
17.040532197	103.32\\
17.055961577	103.31\\
17.071720775	103.3\\
17.087586778999	103.28\\
17.10355967	103.27\\
17.119727484	103.26\\
17.135614244	103.25\\
17.151680865	103.23\\
17.167660682	103.22\\
17.183968242	103.21\\
17.199726417	103.19\\
17.215802486	103.18\\
17.231699226	103.17\\
17.247634726	103.16\\
17.26362957	103.14\\
17.279631518	103.13\\
17.295707797	103.12\\
17.311686082	103.11\\
17.327668679999	103.09\\
17.343664599	103.08\\
17.359753494	103.07\\
17.375664435001	103.05\\
17.391704524	103.04\\
17.407713362	103.03\\
17.423777283	103.02\\
17.439630069	103\\
17.455702096	102.99\\
17.471668188	102.98\\
17.487642573	102.97\\
17.503738301999	102.95\\
17.519682325	102.94\\
17.535646009	102.93\\
17.551735004	102.92\\
17.567598706	102.9\\
17.583691888001	102.89\\
17.599650045	102.88\\
17.615789469999	102.86\\
17.631806155	102.85\\
17.647658834	102.84\\
17.663749459	102.83\\
17.679755318999	102.81\\
17.695725916	102.8\\
17.711710123	102.79\\
17.727702609	102.78\\
17.743730953	102.76\\
17.759742085	102.75\\
17.775631049	102.74\\
17.791839160999	102.73\\
17.807640914999	102.71\\
17.823735577001	102.7\\
17.83966178	102.69\\
17.855758478	102.67\\
17.871658863	102.66\\
17.887770148	102.65\\
17.903582303	102.64\\
17.919566895001	102.62\\
17.935770551	102.61\\
17.951544459001	102.6\\
17.967526514	102.59\\
17.985880912	102.57\\
18.000889362001	102.56\\
18.015783034001	102.55\\
18.031652008001	102.56\\
18.047612993	102.57\\
18.063693799	102.57\\
18.079735682	102.58\\
18.095676779	102.59\\
18.111850101999	102.59\\
18.127778717	102.6\\
18.14370966	102.6\\
18.159856681	102.61\\
18.175649019	102.62\\
18.191684002	102.62\\
18.207823175	102.63\\
18.223726313	102.64\\
18.23960877	102.64\\
18.255714347	102.65\\
18.271578926	102.66\\
18.28772807	102.66\\
18.303682745	102.67\\
18.31966805	102.67\\
18.335643672	102.68\\
18.351788403	102.69\\
18.367630774	102.69\\
18.384355865999	102.7\\
18.402954199001	102.71\\
18.415801798	102.71\\
18.431737742	102.72\\
18.447733078	102.73\\
18.463765925	102.73\\
18.479739853	102.74\\
18.495736744	102.74\\
18.511661919999	102.75\\
};
\end{axis}
\end{tikzpicture}%
}
      \caption{The orientation of the robot over time for
        $K_{\Psi}^R = 0.75 K_{\Psi, max}^R$}
      \label{fig:8_0.75_max}
    \end{figure}
  \end{minipage}
  \hfill
  \begin{minipage}{0.45\linewidth}
    \begin{figure}[H]
      \scalebox{0.6}{% This file was created by matlab2tikz.
%
%The latest updates can be retrieved from
%  http://www.mathworks.com/matlabcentral/fileexchange/22022-matlab2tikz-matlab2tikz
%where you can also make suggestions and rate matlab2tikz.
%
\definecolor{mycolor1}{rgb}{0.00000,0.44700,0.74100}%
%
\begin{tikzpicture}

\begin{axis}[%
width=4.133in,
height=3.26in,
at={(0.693in,0.44in)},
scale only axis,
xmin=8.27314387,
xmax=19,
xmajorgrids,
xlabel={Time (seconds)},
ymin=102,
ymax=104,
ymajorgrids,
ylabel={Angle (degrees)},
axis background/.style={fill=white}
]
\addplot [color=mycolor1,solid,forget plot]
  table[row sep=crcr]{%
8.273143879001	103.14\\
8.288173367	103.12\\
8.303621724	103.11\\
8.319694327	103.1\\
8.335708297	103.09\\
8.354027563	103.07\\
8.369210969	103.06\\
8.384392152	103.05\\
8.399834689	103.04\\
8.415705992	103.02\\
8.431752197001	103.01\\
8.447560138	103\\
8.463793990999	102.99\\
8.479645476	102.97\\
8.495671495999	102.96\\
8.511780766	102.95\\
8.527655278999	102.94\\
8.544773933	102.92\\
8.560126548	102.91\\
8.575754121	102.9\\
8.591666156	102.88\\
8.607641551001	102.87\\
8.623665740999	102.86\\
8.639496241	102.85\\
8.655529002	102.83\\
8.671694257	102.82\\
8.687609847	102.81\\
8.703653539001	102.8\\
8.719687018	102.78\\
8.735734333	102.77\\
8.751576198	102.76\\
8.767815139	102.75\\
8.783724148	102.73\\
8.799775558	102.72\\
8.815680402	102.71\\
8.831639624001	102.69\\
8.847662226	102.68\\
8.863595004001	102.67\\
8.879576392	102.66\\
8.897893839	102.64\\
8.913202397	102.63\\
8.92829041	102.62\\
8.94371067	102.61\\
8.959640589001	102.59\\
8.975708661	102.58\\
8.991666052	102.57\\
9.007692686	102.55\\
9.025160797	102.55\\
9.040212579001	102.56\\
9.055669958	102.56\\
9.071554785999	102.57\\
9.089787545	102.58\\
9.105056673	102.58\\
9.120333575	102.59\\
9.135903358	102.59\\
9.151614917	102.6\\
9.167760507	102.61\\
9.183647209	102.61\\
9.199853106	102.62\\
9.215520778	102.63\\
9.231655545	102.63\\
9.247657392	102.64\\
9.263764538	102.65\\
9.27960798	102.65\\
9.295690654	102.66\\
9.313737328	102.67\\
9.329306511	102.67\\
9.344639928	102.68\\
9.359747753	102.68\\
9.376376211	102.69\\
9.391690529	102.7\\
9.407715529	102.7\\
9.423669257	102.71\\
9.439547961	102.71\\
9.45567642	102.72\\
9.471678493	102.73\\
9.487593485001	102.73\\
9.503675537	102.74\\
9.519665888001	102.75\\
9.535768091	102.75\\
9.551636668	102.76\\
9.567838039	102.77\\
9.583695701	102.77\\
9.602041199	102.78\\
9.617172347	102.79\\
9.632479574	102.79\\
9.647906297	102.8\\
9.664580375	102.8\\
9.679865153	102.81\\
9.695685983	102.82\\
9.71165668	102.82\\
9.727682541	102.83\\
9.74371141	102.84\\
9.759682417	102.84\\
9.775698371	102.85\\
9.791664262	102.85\\
9.807636373	102.86\\
9.823626948	102.87\\
9.840281952001	102.87\\
9.855640641	102.88\\
9.87154053	102.89\\
9.887717485999	102.89\\
9.903738823	102.9\\
9.919730717	102.91\\
9.935786437	102.91\\
9.951695659001	102.92\\
9.967739775	102.92\\
9.983592219	102.93\\
9.999934565	102.94\\
10.016015919	102.94\\
10.031633335	102.95\\
10.047741526	102.96\\
10.063804096	102.96\\
10.079694296001	102.97\\
10.095799351	102.97\\
10.111570058	102.98\\
10.127735619	102.99\\
10.143761952	102.99\\
10.159575868	103\\
10.175652087999	103.01\\
10.191579703	103.01\\
10.207631266	103.02\\
10.223560008	103.03\\
10.239734978999	103.03\\
10.25577828	103.04\\
10.271525993	103.04\\
10.287596237	103.05\\
10.303701419	103.06\\
10.319634646	103.06\\
10.335844014	103.07\\
10.351593559	103.08\\
10.367889598	103.08\\
10.383658768	103.09\\
10.399756181	103.1\\
10.415688568	103.1\\
10.431580551	103.11\\
10.447570909	103.11\\
10.463674214	103.12\\
10.479642014	103.13\\
10.495898476	103.13\\
10.511499045	103.14\\
10.527667445001	103.15\\
10.543713384	103.15\\
10.559659556	103.16\\
10.575610691	103.17\\
10.591696143	103.17\\
10.607696877999	103.18\\
10.623711434	103.18\\
10.639710941	103.19\\
10.655631925	103.2\\
10.671518410999	103.2\\
10.689831565001	103.21\\
10.704989155999	103.22\\
10.720087515	103.22\\
10.735694372	103.23\\
10.751711855	103.23\\
10.768009071	103.24\\
10.783714758001	103.25\\
10.79970596	103.25\\
10.815568581	103.26\\
10.831687294	103.27\\
10.847713552	103.27\\
10.863894976	103.28\\
10.879650258	103.29\\
10.895778773999	103.29\\
10.9116898	103.3\\
10.927694801	103.3\\
10.943711429999	103.31\\
10.959671393	103.32\\
10.975718984	103.32\\
10.991667836001	103.33\\
11.007710601	103.34\\
11.025385737	103.33\\
11.040500136999	103.32\\
11.05582761	103.31\\
11.071615712999	103.3\\
11.087556479	103.28\\
11.103633144	103.27\\
11.119666532	103.26\\
11.136046648	103.25\\
11.151706825	103.23\\
11.168013583	103.22\\
11.18382118	103.21\\
11.199714256	103.2\\
11.215626174	103.18\\
11.231733671	103.17\\
11.247702679	103.16\\
11.263715829	103.14\\
11.279704136	103.13\\
11.29837435	103.12\\
11.313763245	103.1\\
11.329075709	103.09\\
11.344391313	103.08\\
11.359708523999	103.07\\
11.375689636	103.06\\
11.391620515	103.04\\
11.407852745	103.03\\
11.423544903	103.02\\
11.439719809	103.01\\
11.455632410999	102.99\\
11.471538961	102.98\\
11.487642333	102.97\\
11.5037961	102.95\\
11.519710193	102.94\\
11.535804037	102.93\\
11.551968504	102.92\\
11.567972392	102.9\\
11.583754596	102.89\\
11.602370836	102.88\\
11.617675942001	102.86\\
11.633039408	102.85\\
11.648381461	102.84\\
11.664308848001	102.83\\
11.679732407999	102.81\\
11.695750117	102.8\\
11.711716936	102.79\\
11.727687534001	102.78\\
11.743674856	102.76\\
11.759702637	102.75\\
11.775749447001	102.74\\
11.791705278999	102.73\\
11.807574401	102.71\\
11.823699593	102.7\\
11.839720458	102.69\\
11.855684234	102.68\\
11.871573584001	102.66\\
11.887726436	102.65\\
11.90373349	102.64\\
11.919749058	102.62\\
11.93579258	102.61\\
11.951695112	102.6\\
11.970185148999	102.58\\
11.985573914999	102.57\\
12.000919537	102.56\\
12.01659102	102.56\\
12.031820926999	102.56\\
12.047653479	102.57\\
12.063781653	102.57\\
12.07964014	102.58\\
12.095638278	102.59\\
12.111687666	102.59\\
12.12764863	102.6\\
12.143724846001	102.61\\
12.159672476	102.61\\
12.175707824	102.62\\
12.191689310001	102.62\\
12.207696534	102.63\\
12.223707476999	102.64\\
12.239727117	102.64\\
12.255709022	102.65\\
12.271710824001	102.66\\
12.287705971	102.66\\
12.303724165	102.67\\
12.319702414	102.68\\
12.335603520001	102.68\\
12.351806235999	102.69\\
12.367656461001	102.69\\
12.383684952	102.7\\
12.399710327999	102.71\\
12.415720158	102.71\\
12.431725639	102.72\\
12.447669955	102.73\\
12.46361005	102.73\\
12.479828626	102.74\\
12.495943216	102.75\\
12.511638828	102.75\\
12.527622929	102.76\\
12.54366311	102.76\\
12.559641607	102.77\\
12.575618551	102.78\\
12.591712211001	102.78\\
12.607658625	102.79\\
12.623699967	102.8\\
12.639662399	102.8\\
12.655546898999	102.81\\
12.671706014	102.82\\
12.687645400999	102.82\\
12.703725014	102.83\\
12.719704811	102.83\\
12.735718421001	102.84\\
12.751682518	102.85\\
12.767596836	102.85\\
12.783923234	102.86\\
12.799724908	102.87\\
12.815721734	102.87\\
12.831737477	102.88\\
12.847678506	102.88\\
12.863733033	102.89\\
12.879665921	102.9\\
12.895768866	102.9\\
12.911621504999	102.91\\
12.927592183999	102.92\\
12.943662946	102.92\\
12.959748957999	102.93\\
12.975798903999	102.94\\
12.991688356	102.94\\
13.007702236	102.95\\
13.025434183	102.96\\
13.040779127	102.96\\
13.055937303	102.97\\
13.071575109	102.97\\
13.087751684	102.98\\
13.103694191	102.99\\
13.119747951	102.99\\
13.135678962999	103\\
13.152181453	103.01\\
13.167668103	103.01\\
13.183874741	103.02\\
13.199729675	103.02\\
13.215752259999	103.03\\
13.231591892	103.04\\
13.249938276	103.04\\
13.265259035	103.05\\
13.280645164	103.06\\
13.296002154	103.06\\
13.311672702001	103.07\\
13.327661744001	103.08\\
13.343663475001	103.08\\
13.359673476	103.09\\
13.3756062	103.09\\
13.39181473	103.1\\
13.407789992	103.11\\
13.423738999999	103.11\\
13.439681781	103.12\\
13.455722138	103.13\\
13.471651157	103.13\\
13.487745226	103.14\\
13.503597648999	103.14\\
13.519625030999	103.15\\
13.535603887	103.16\\
13.553949881	103.17\\
13.569026929	103.17\\
13.584174849	103.18\\
13.599690824001	103.18\\
13.615726549	103.19\\
13.631733948	103.2\\
13.648943068	103.2\\
13.66411992	103.21\\
13.679697571	103.21\\
13.695916915	103.22\\
13.711637806	103.23\\
13.727727482	103.23\\
13.743776039999	103.24\\
13.75977603	103.25\\
13.775673549	103.25\\
13.791704346	103.26\\
13.807700145	103.27\\
13.823746898999	103.27\\
13.839714860001	103.28\\
13.855714602	103.28\\
13.872988717	103.29\\
13.888162609	103.3\\
13.90372908	103.3\\
13.919737303	103.31\\
13.935661713	103.32\\
13.95178071	103.32\\
13.967660928001	103.33\\
13.986125422	103.34\\
14.001382115	103.34\\
14.017222538	103.34\\
14.032466009	103.33\\
14.047616221001	103.32\\
14.063725817	103.3\\
14.079753894	103.29\\
14.095682311	103.28\\
14.111712263999	103.27\\
14.127669848001	103.25\\
14.143506954999	103.24\\
14.159710587	103.23\\
14.175659975001	103.22\\
14.192098186	103.2\\
14.207556248	103.19\\
14.223767364	103.18\\
14.239674435	103.16\\
14.255688966	103.15\\
14.27166965	103.14\\
14.287686599	103.13\\
14.303674563001	103.11\\
14.319787096	103.1\\
14.335650699001	103.09\\
14.351932497	103.08\\
14.367651056	103.06\\
14.386060267	103.05\\
14.401443697	103.04\\
14.416752617999	103.02\\
14.43219208	103.01\\
14.44770073	103\\
14.46373714	102.99\\
14.479664278	102.97\\
14.49567932	102.96\\
14.511611693999	102.95\\
14.527690572	102.94\\
14.543681454	102.92\\
14.559709421001	102.91\\
14.575508599	102.9\\
14.591740663	102.89\\
14.60763122	102.87\\
14.62376315	102.86\\
14.639693344	102.85\\
14.655680985	102.83\\
14.671697013	102.82\\
14.687623413	102.81\\
14.703597864	102.8\\
14.719638546	102.78\\
14.735754669	102.77\\
14.751688499	102.76\\
14.76769374	102.75\\
14.784016825999	102.73\\
14.799655099	102.72\\
14.815807028	102.71\\
14.831632933	102.7\\
14.847693936	102.68\\
14.863661852	102.67\\
14.879753517	102.66\\
14.895735108	102.64\\
14.911668951	102.63\\
14.927727273	102.62\\
14.94371841	102.61\\
14.959661083	102.59\\
14.977986032	102.58\\
14.993191704	102.57\\
15.00921932	102.55\\
15.026913318	102.55\\
15.039677829001	102.56\\
15.055740806999	102.56\\
15.071684954	102.57\\
15.087776086	102.57\\
15.103682419	102.58\\
15.119723034001	102.59\\
15.135674653	102.59\\
15.151724887	102.6\\
15.167662311999	102.61\\
15.183730734	102.61\\
15.199650921999	102.62\\
15.215805804999	102.63\\
15.231731891	102.63\\
15.24765221	102.64\\
15.263717184	102.64\\
15.279710231	102.65\\
15.295677535	102.66\\
15.311756737	102.66\\
15.327733152	102.67\\
15.343672222	102.68\\
15.359713479	102.68\\
15.37564682	102.69\\
15.391708995	102.69\\
15.407739534	102.7\\
15.423768756	102.71\\
15.439649442001	102.71\\
15.455740818	102.72\\
15.47166707	102.73\\
15.487791029	102.73\\
15.503740758	102.74\\
15.519669214	102.75\\
15.535600727	102.75\\
15.551652698	102.76\\
15.567587516999	102.76\\
15.58385214	102.77\\
15.599661308001	102.78\\
15.617999029	102.78\\
15.633215481	102.79\\
15.648458939	102.8\\
15.663917439	102.8\\
15.679827148	102.81\\
15.695780581	102.82\\
15.711853278	102.82\\
15.727621831	102.83\\
15.743673547	102.83\\
15.759699719	102.84\\
15.775655683	102.85\\
15.791806097	102.85\\
15.807689866	102.86\\
15.823721035	102.87\\
15.839704003	102.87\\
15.855696671	102.88\\
15.871661609	102.89\\
15.887704219	102.89\\
15.903722879999	102.9\\
15.919724737	102.9\\
15.935684038	102.91\\
15.951678004999	102.92\\
15.967690019	102.92\\
15.983920706	102.93\\
15.999692492001	102.94\\
16.020569018	102.94\\
16.032566767	102.95\\
16.047793522	102.96\\
16.06382477	102.96\\
16.079656177	102.97\\
16.095767386999	102.97\\
16.111691849	102.98\\
16.127813883	102.99\\
16.143639976	102.99\\
16.159616718001	103\\
16.175842383001	103.01\\
16.191701527	103.01\\
16.207752976	103.02\\
16.223692561999	103.02\\
16.239691478	103.03\\
16.255711908	103.04\\
16.271510041001	103.04\\
16.289807474	103.05\\
16.304999375	103.06\\
16.32012901	103.06\\
16.335604162	103.07\\
16.351825627	103.08\\
16.367582990999	103.08\\
16.386176858	103.09\\
16.401475719	103.1\\
16.4168202	103.1\\
16.432197186	103.11\\
16.447746538	103.11\\
16.463683247	103.12\\
16.479709064	103.13\\
16.495778153	103.13\\
16.511698496	103.14\\
16.527582847	103.15\\
16.543701874999	103.15\\
16.559537226	103.16\\
16.575698033	103.16\\
16.591699206	103.17\\
16.60769364	103.18\\
16.623668027001	103.18\\
16.639576368	103.19\\
16.658158095	103.2\\
16.673404795	103.2\\
16.688717575	103.21\\
16.703976616	103.22\\
16.719687917	103.22\\
16.735672711	103.23\\
16.751639886	103.23\\
16.767697483	103.24\\
16.786053051001	103.25\\
16.801208936	103.25\\
16.816320147	103.26\\
16.831620586	103.27\\
16.847663471	103.27\\
16.863653061	103.28\\
16.87976373	103.28\\
16.895693203	103.29\\
16.911788482	103.3\\
16.92774195	103.3\\
16.943675346999	103.31\\
16.959690424	103.32\\
16.975708057	103.32\\
16.99181857	103.33\\
17.007685141	103.34\\
17.025389207	103.33\\
17.040532197	103.32\\
17.055961577	103.31\\
17.071720775	103.3\\
17.087586778999	103.28\\
17.10355967	103.27\\
17.119727484	103.26\\
17.135614244	103.25\\
17.151680865	103.23\\
17.167660682	103.22\\
17.183968242	103.21\\
17.199726417	103.19\\
17.215802486	103.18\\
17.231699226	103.17\\
17.247634726	103.16\\
17.26362957	103.14\\
17.279631518	103.13\\
17.295707797	103.12\\
17.311686082	103.11\\
17.327668679999	103.09\\
17.343664599	103.08\\
17.359753494	103.07\\
17.375664435001	103.05\\
17.391704524	103.04\\
17.407713362	103.03\\
17.423777283	103.02\\
17.439630069	103\\
17.455702096	102.99\\
17.471668188	102.98\\
17.487642573	102.97\\
17.503738301999	102.95\\
17.519682325	102.94\\
17.535646009	102.93\\
17.551735004	102.92\\
17.567598706	102.9\\
17.583691888001	102.89\\
17.599650045	102.88\\
17.615789469999	102.86\\
17.631806155	102.85\\
17.647658834	102.84\\
17.663749459	102.83\\
17.679755318999	102.81\\
17.695725916	102.8\\
17.711710123	102.79\\
17.727702609	102.78\\
17.743730953	102.76\\
17.759742085	102.75\\
17.775631049	102.74\\
17.791839160999	102.73\\
17.807640914999	102.71\\
17.823735577001	102.7\\
17.83966178	102.69\\
17.855758478	102.67\\
17.871658863	102.66\\
17.887770148	102.65\\
17.903582303	102.64\\
17.919566895001	102.62\\
17.935770551	102.61\\
17.951544459001	102.6\\
17.967526514	102.59\\
17.985880912	102.57\\
18.000889362001	102.56\\
18.015783034001	102.55\\
18.031652008001	102.56\\
18.047612993	102.57\\
18.063693799	102.57\\
18.079735682	102.58\\
18.095676779	102.59\\
18.111850101999	102.59\\
18.127778717	102.6\\
18.14370966	102.6\\
18.159856681	102.61\\
18.175649019	102.62\\
18.191684002	102.62\\
18.207823175	102.63\\
18.223726313	102.64\\
18.23960877	102.64\\
18.255714347	102.65\\
18.271578926	102.66\\
18.28772807	102.66\\
18.303682745	102.67\\
18.31966805	102.67\\
18.335643672	102.68\\
18.351788403	102.69\\
18.367630774	102.69\\
18.384355865999	102.7\\
18.402954199001	102.71\\
18.415801798	102.71\\
18.431737742	102.72\\
18.447733078	102.73\\
18.463765925	102.73\\
18.479739853	102.74\\
18.495736744	102.74\\
18.511661919999	102.75\\
};
\end{axis}
\end{tikzpicture}%
}
      \caption{The steady state orientation of the robot for
        $K_{\Psi}^R = 0.75 K_{\Psi, max}^R$}
      \label{fig:8_0.75_max_magnified}
    \end{figure}
  \end{minipage}
\end{minipage}
}

\noindent\makebox[\textwidth][c]{%
\begin{minipage}{\linewidth}
  \begin{minipage}{0.45\linewidth}
    \begin{figure}[H]
      \scalebox{0.6}{% This file was created by matlab2tikz.
%
%The latest updates can be retrieved from
%  http://www.mathworks.com/matlabcentral/fileexchange/22022-matlab2tikz-matlab2tikz
%where you can also make suggestions and rate matlab2tikz.
%
\definecolor{mycolor1}{rgb}{0.00000,0.44700,0.74100}%
%
\begin{tikzpicture}

\begin{axis}[%
width=4.133in,
height=3.26in,
at={(0.693in,0.44in)},
scale only axis,
xmin=0,
xmax=105,
xmajorgrids,
ymin=-200,
ymax=200,
ymajorgrids,
axis background/.style={fill=white}
]
\addplot [color=mycolor1,solid,forget plot]
  table[row sep=crcr]{%
0	0\\
0.017628879999999	1.81\\
0.0329686569999978	4.94\\
0.0484092639990002	8.1\\
0.0669318689990006	11.89\\
0.0800320119999982	14.36\\
0.0960033350000009	17.7\\
0.112024882999998	20.99\\
0.128021026000001	24.24\\
0.144072445999999	27.5\\
0.159972066	30.77\\
0.17597693	34.04\\
0.191958870999999	37.33\\
0.207965610999998	40.58\\
0.224009333000001	43.87\\
0.240072048	47.13\\
0.256011259999998	50.42\\
0.272087042999998	53.71\\
0.288026489	56.97\\
0.304031283	60.24\\
0.319997880998999	63.5\\
0.336051604998998	66.77\\
0.351950548	70.04\\
0.368043166999001	73.31\\
0.384165548	76.61\\
0.400026512998998	79.91\\
0.416458789999999	83.17\\
0.432051310998999	86.5\\
0.448116042999998	89.69\\
0.464153083999	92.98\\
0.479844719	96.29\\
0.495885344000001	99.43\\
0.511953702	102.73\\
0.527979057999	106.04\\
0.544061507998999	109.28\\
0.559997469999999	112.62\\
0.575974905999999	115.83\\
0.592039421999	119.21\\
0.608004595998	122.43\\
0.624015700999997	125.69\\
0.640019292999998	128.94\\
0.656013945999999	132.22\\
0.671958886	135.52\\
0.687890022999	138.75\\
0.704070099999998	142.02\\
0.720078710000001	145.33\\
0.735915204999998	148.62\\
0.752100480999999	151.84\\
0.767971250999998	155.15\\
0.783939956999	158.39\\
0.800004859000001	161.67\\
0.816046296	164.97\\
0.834690483001001	168.92\\
0.850142287999999	172.08\\
0.865372621999998	175.2\\
0.880744491	178.34\\
0.896070860999998	-178.54\\
0.911959083000001	-175.38\\
0.928055338	-172.13\\
0.944044066	-168.84\\
0.959926491	-165.58\\
0.975993761999998	-162.33\\
0.991987149999	-159.05\\
1.009753745999	-154.26\\
1.025150873	-149.39\\
1.040681112999	-144.45\\
1.056052886	-139.64\\
1.072065041999	-134.76\\
1.087914888	-129.63\\
1.104057962	-124.65\\
1.119975077	-119.5\\
1.13603659	-114.49\\
1.152075862	-109.37\\
1.167976304999	-104.29\\
1.184023725	-99.26\\
1.199995128	-94.19\\
1.216017247	-89.15\\
1.232119461	-84.06\\
1.248066199	-78.88\\
1.264299880999	-73.89\\
1.28001892	-68.69\\
1.295957604	-63.76\\
1.312223783	-58.35\\
1.328008493	-53.56\\
1.343956302999	-48.54\\
1.359974592	-43.23\\
1.375997352	-38.39\\
1.391965453	-33.3\\
1.408014227	-28.16\\
1.423839046	-23.16\\
1.440077355999	-18.11\\
1.455893204	-12.91\\
1.472034249	-7.96\\
1.487964028	-2.84\\
1.504035593999	2.28\\
1.519974838	7.34\\
1.536006873	12.36\\
1.551960251	17.48\\
1.567997605	22.56\\
1.584154847	27.55\\
1.600037567	32.72\\
1.616006173	37.8\\
1.632396224	42.86\\
1.648234750001	48.06\\
1.664024208999	53.09\\
1.679975545999	58.07\\
1.695984168	63.07\\
1.712139203	68.35\\
1.727949644001	73.3\\
1.743974717999	78.24\\
1.759962897	83.38\\
1.775985666999	88.43\\
1.791858501999	93.65\\
1.807998002	98.71\\
1.823970883999	103.62\\
1.840039745	108.79\\
1.85592743	113.83\\
1.872018279998	118.87\\
1.887968996	123.99\\
1.903980736001	129.01\\
1.920034224999	134.15\\
1.935919518	139.22\\
1.952030608001	144.26\\
1.967944272999	149.38\\
1.984035991	154.39\\
1.9999327	159.49\\
2.015943894	160.37\\
2.032139412	158.57\\
2.048017958	156.72\\
2.064011987999	154.93\\
2.079825736999	153.13\\
2.096064313	151.33\\
2.111979107	149.48\\
2.128008046	147.73\\
2.144160980999	145.92\\
2.160074317999	144.09\\
2.176010696	142.28\\
2.19202879	140.48\\
2.207971413001	138.67\\
2.223970263999	136.87\\
2.240109873	135.05\\
2.255923638999	133.25\\
2.272028500001	131.41\\
2.28798097	129.64\\
2.304045029	127.83\\
2.319965191	126.02\\
2.336015317999	124.2\\
2.352027834	122.33\\
2.367910224	120.59\\
2.383977178	118.79\\
2.399937168999	116.99\\
2.415993699	115.18\\
2.434515388999	113\\
2.449798324	111.27\\
2.465166406	109.55\\
2.480559775	107.8\\
2.496059459	106.07\\
2.511819362	104.32\\
2.530166454	102.2\\
2.545456851	100.46\\
2.560649338	98.75\\
2.57607329	97.02\\
2.592033638	95.27\\
2.60801977	93.48\\
2.624052189	91.67\\
2.639935359	89.86\\
2.656002925999	88.07\\
2.672026592	86.26\\
2.687989	84.46\\
2.704049934	82.65\\
2.720054986	80.84\\
2.736096182	79.02\\
2.75214881	77.21\\
2.768018708	75.39\\
2.784071401	73.61\\
2.800030281001	71.78\\
2.81605053	69.98\\
2.832065397	68.19\\
2.847967265	66.36\\
2.864180248	64.56\\
2.880144048999	62.72\\
2.896184213	60.95\\
2.912007359999	59.13\\
2.927978296999	57.34\\
2.944047793	55.54\\
2.960024883999	53.64\\
2.975982879	51.92\\
2.991991859999	50.09\\
3.016043862999	50.64\\
3.024582619999	51.53\\
3.039901956	53.1\\
3.055851373999	54.7\\
3.071907685	56.38\\
3.087892511	58.05\\
3.104026821	59.73\\
3.119977010999	61.42\\
3.135971862999	63.1\\
3.152042986999	64.78\\
3.168018819999	66.46\\
3.184117897999	68.13\\
3.199924352	69.82\\
3.216019591	71.47\\
3.234437836	73.49\\
3.249716875	75.09\\
3.265348357999	76.72\\
3.280565456	78.31\\
3.296087931	79.91\\
3.31196022	81.53\\
3.328002543999	83.18\\
3.344116385	84.86\\
3.359978104999	86.56\\
3.376003369999	88.22\\
3.392044176	89.92\\
3.407977899	91.57\\
3.423979645999	93.23\\
3.440151442999	94.98\\
3.455966474	96.65\\
3.472080599	98.26\\
3.488006654	99.94\\
3.504110267	101.61\\
3.520008053	103.34\\
3.535905739	104.96\\
3.551937675	106.62\\
3.567901362999	108.31\\
3.58634538	110.31\\
3.601523254	111.9\\
3.616587856	113.48\\
3.632483688	115.06\\
3.647905808	116.73\\
3.666316434	118.68\\
3.681639733001	120.28\\
3.696811173	121.87\\
3.712558637999	123.52\\
3.728042736999	125.1\\
3.744093215	126.75\\
3.759923460999	128.4\\
3.775956061	130.07\\
3.791989262999	131.77\\
3.807942131	133.43\\
3.823967966	135.11\\
3.840069742	136.79\\
3.855983037	138.45\\
3.871975668	140.13\\
3.887938673	141.8\\
3.904012392	143.52\\
3.920001260999	145.17\\
3.935986784999	146.82\\
3.951999743	148.5\\
3.967960086	150.17\\
3.984073241999	151.84\\
3.999951763999	153.54\\
4.017694552	152.92\\
4.032654998	151.41\\
4.047869789	149.9\\
4.063907765999	148.36\\
4.079864004	146.74\\
4.096006587	145.11\\
4.111982585	143.46\\
4.127983821	141.86\\
4.144096379	140.23\\
4.160074673	138.61\\
4.175919766001	137.01\\
4.19201142	135.38\\
4.207995857	133.77\\
4.224012933	132.15\\
4.239945209	130.55\\
4.255987348	128.93\\
4.272048132	127.3\\
4.287979225999	125.67\\
4.304044612999	124.06\\
4.320072954	122.44\\
4.336029524	120.82\\
4.352045196999	119.21\\
4.368029956	117.58\\
4.384094576	115.98\\
4.400019864999	114.34\\
4.415981691	112.73\\
4.434514987	110.8\\
4.449643173999	109.27\\
4.464861739999	107.73\\
4.480299544	106.17\\
4.496119633999	104.64\\
4.512006380001	103.01\\
4.527925704999	101.43\\
4.544101221	99.81\\
4.560105894999	98.17\\
4.575937456	96.57\\
4.592013446	94.95\\
4.607956375	93.34\\
4.623934576	91.75\\
4.640154587	90.12\\
4.655958927	88.48\\
4.672093217	86.88\\
4.687989834	85.23\\
4.704525905	83.5\\
4.720003906	81.96\\
4.736063909	80.39\\
4.752173763	78.77\\
4.767980599999	77.16\\
4.784027302	75.56\\
4.800152001999	73.93\\
4.815921480999	72.29\\
4.834440821001	70.38\\
4.849703857999	68.83\\
4.86491243	67.29\\
4.880177414	65.75\\
4.896078187	64.21\\
4.911854791999	62.59\\
4.928024647	61.01\\
4.944061312998	59.38\\
4.959836986001	57.76\\
4.975970746	56.07\\
4.991896043	54.54\\
5.009418010999	54.43\\
5.024410322	55.88\\
5.039891765	57.33\\
5.055864847001	58.83\\
5.071941435999	60.38\\
5.087992109	61.93\\
5.103905807	63.5\\
5.119994666	65.03\\
5.136000637001	66.6\\
5.152049465999	68.16\\
5.167939468	69.74\\
5.184029785	71.23\\
5.199950673	72.78\\
5.215987968999	74.33\\
5.234537888	76.2\\
5.249559748999	77.66\\
5.264664727	79.12\\
5.280234173999	80.62\\
5.295891845	82.08\\
5.311931699999	83.61\\
5.327981987	85.17\\
5.344062392	86.71\\
5.36051446	88.29\\
5.376032855999	89.86\\
5.392001709	91.37\\
5.408017062	92.89\\
5.424018968999	94.44\\
5.439939587	96.02\\
5.456055874	97.54\\
5.472049873	99.1\\
5.488011274	100.65\\
5.504188062998	102.19\\
5.519969661	103.77\\
5.535985524999	105.28\\
5.552051902999	106.84\\
5.568039998999	108.39\\
5.584132038001	109.94\\
5.600052458999	111.5\\
5.615982477999	113.02\\
5.634353031001	114.87\\
5.649565944	116.34\\
5.664694772	117.8\\
5.680127772999	119.3\\
5.696015253	120.78\\
5.711937281999	122.3\\
5.727968177001	123.83\\
5.744035204999	125.39\\
5.759999744999	126.94\\
5.77585039	128.48\\
5.79198581	130.02\\
5.807962515	131.58\\
5.823975387	133.12\\
5.840066544	134.7\\
5.855991649999	136.22\\
5.871915782	137.79\\
5.887983217001	139.31\\
5.903905328999	140.86\\
5.922070114999	142.7\\
5.93732134	144.17\\
5.952473812	145.64\\
5.967928928999	147.11\\
5.984066239	148.62\\
6.000015577	150.17\\
6.017290317999	150.25\\
6.032240747	148.82\\
6.047853254	147.41\\
6.063865078	145.92\\
6.07987965	144.4\\
6.098122274	142.56\\
6.113205210999	141.13\\
6.128343866	139.69\\
6.144049321	138.24\\
6.160040675	136.74\\
6.176088456	135.22\\
6.191977073	133.69\\
6.207883473	132.19\\
6.223992357	130.68\\
6.239962713	129.12\\
6.256001099	127.61\\
6.272052814	126.09\\
6.288031554	124.55\\
6.304058187	123.04\\
6.319975165	121.51\\
6.335995466	120\\
6.351978384	118.48\\
6.368004132	116.95\\
6.384046336	115.44\\
6.400005657	113.91\\
6.416082347999	112.39\\
6.43243047	110.86\\
6.448021402	109.31\\
6.464517604	107.76\\
6.479938301	106.26\\
6.496153084	104.8\\
6.511985948	103.24\\
6.528014371	101.74\\
6.544057522	100.23\\
6.560054763999	98.67\\
6.5761332	97.18\\
6.592004952	95.65\\
6.607968768	94.14\\
6.62392447	92.63\\
6.639996306	91.1\\
6.656046595	89.56\\
6.672057281	88.03\\
6.687926763999	86.5\\
6.704350562998	84.97\\
6.719977175	83.43\\
6.735993118	81.95\\
6.752056565	80.41\\
6.767888711999	78.91\\
6.784227840999	77.38\\
6.799999046999	75.84\\
6.815997222	74.34\\
6.834693227	72.49\\
6.849881721	71.05\\
6.86511031	69.61\\
6.880472257	68.14\\
6.896077001	66.68\\
6.912039713	65.19\\
6.928037791	63.66\\
6.944162130999	62.16\\
6.959959654998	60.61\\
6.975991225	59.13\\
6.992076350999	57.6\\
7.009587336999	57.48\\
7.024484303999	58.84\\
7.039932965	60.19\\
7.05586758	61.6\\
7.074091479	63.34\\
7.089334605999	64.72\\
7.105171159	66.16\\
7.12042498	67.54\\
7.136023418	68.93\\
7.152057109	70.34\\
7.168070977	71.81\\
7.184129096	73.26\\
7.200032055	74.7\\
7.215931132999	76.15\\
7.234456206999	77.89\\
7.249702612	79.28\\
7.265002904	80.66\\
7.280253382	82.05\\
7.296057281	83.42\\
7.311815331999	84.85\\
7.330073678	86.57\\
7.345167637001	87.94\\
7.359986628	89.2\\
7.375966675999	90.66\\
7.392054498	92.12\\
7.407921177001	93.57\\
7.423982752	95.01\\
7.44013302	96.49\\
7.45601407	97.99\\
7.472051350999	99.39\\
7.487977111	100.85\\
7.503966665	102.29\\
7.519967239999	103.74\\
7.535961857	105.2\\
7.552030013	106.66\\
7.567937401	108.11\\
7.583983630999	109.54\\
7.599831913	110.98\\
7.616031168	112.43\\
7.632495097999	113.93\\
7.648071867	115.41\\
7.664111354	116.82\\
7.680461727	118.38\\
7.696031268	119.76\\
7.712068884	121.18\\
7.727949873999	122.62\\
7.74402025	124.06\\
7.759934617	125.52\\
7.776034294	126.97\\
7.791863235	128.42\\
7.81006471	130.14\\
7.825226217	131.51\\
7.840503565	132.9\\
7.856300657	134.34\\
7.871826892999	135.71\\
7.887807284	137.11\\
7.905880668	138.84\\
7.920765693	140.19\\
7.935990623	141.54\\
7.951948967	142.95\\
7.967974277	144.41\\
7.984042187	145.88\\
7.999976162	147.32\\
8.01750315799999	147.4\\
8.032471172	146.07\\
8.047870672	144.74\\
8.063895291	143.37\\
8.079891623001	141.96\\
8.098220714	140.24\\
8.113483029998	138.88\\
8.128629243001	137.54\\
8.144025121	136.18\\
8.160065656	134.77\\
8.175989322	133.4\\
8.192012540999	131.99\\
8.208054468999	130.55\\
8.223932102	129.14\\
8.240057869	127.72\\
8.256008229	126.29\\
8.272037428999	124.88\\
8.288104659	123.47\\
8.303921583	121.99\\
8.319874722999	120.63\\
8.335984656	119.21\\
8.352052772999	117.77\\
8.368072260999	116.34\\
8.384080418	114.93\\
8.400014230999	113.5\\
8.415997670999	112.08\\
8.434486892	110.38\\
8.450307073	108.98\\
8.465556183	107.62\\
8.480909765999	106.26\\
8.496309694	104.91\\
8.51271451	103.44\\
8.527885225999	102.09\\
8.544044116	100.74\\
8.559966967	99.25\\
8.575982704	97.88\\
8.591892497	96.47\\
8.60805733799999	95.05\\
8.623996600999	93.6\\
8.640049911999	92.2\\
8.655973723	90.78\\
8.672057394999	89.29\\
8.687972967	87.92\\
8.703894948999	86.52\\
8.720038607	85.11\\
8.736020057999	83.67\\
8.75206786	82.25\\
8.76791887	80.84\\
8.784003189	79.44\\
8.800095421999	77.99\\
8.81595664599899	76.55\\
8.834375477	74.87\\
8.849466227999	73.54\\
8.864950633	72.16\\
8.880254111	70.8\\
8.896270543999	69.45\\
8.911992534	68.02\\
8.92808611	66.62\\
8.946404609	64.93\\
8.960188789	63.71\\
8.976039994	62.36\\
8.992021345	60.95\\
9.009563350999	60.85\\
9.024502946999	62.12\\
9.039897436	63.38\\
9.055862045999	64.69\\
9.071910322	66.05\\
9.08787604	67.4\\
9.104025751999	68.78\\
9.119948229	70.16\\
9.135935819	71.51\\
9.154347876	73.13\\
9.169456578	74.41\\
9.184662149999	75.7\\
9.199925236999	76.99\\
9.216008065	78.28\\
9.234438942	79.92\\
9.249575472999	81.21\\
9.264742652999	82.49\\
9.280105982	83.79\\
9.296058032	85.08\\
9.311970527	86.48\\
9.328006135	87.78\\
9.343972783	89.14\\
9.359889937	90.5\\
9.37599071	91.84\\
9.392060491	93.22\\
9.407986506	94.57\\
9.42397357	95.92\\
9.440030130999	97.3\\
9.45592813	98.64\\
9.472033295	100\\
9.487963845	101.37\\
9.504078378	102.72\\
9.519893171999	104.09\\
9.536027714	105.44\\
9.551971114999	106.82\\
9.567914196	108.15\\
9.584289329	109.5\\
9.60003867	110.89\\
9.615995428999	112.24\\
9.63216559	113.58\\
9.648054475999	114.96\\
9.664135081999	116.3\\
9.680010248	117.66\\
9.696085972	119\\
9.71200232	120.37\\
9.728044249	121.72\\
9.746335406	123.34\\
9.759992811	124.49\\
9.775960149	125.78\\
9.792022872	127.15\\
9.80798517	128.5\\
9.823984321	129.86\\
9.840058339	131.22\\
9.855984140999	132.58\\
9.872057472	133.95\\
9.888016140999	135.3\\
9.904068402	136.65\\
9.920008755	138.02\\
9.935996195	139.35\\
9.95215127	140.71\\
9.968073059	142.09\\
9.984110546	143.43\\
9.999988198999	144.79\\
10.017415991	144.87\\
10.032346887999	143.63\\
10.047872218	142.4\\
10.066079766001	140.84\\
10.081130428999	139.59\\
10.096425388999	138.33\\
10.112512319	136.99\\
10.127987005999	135.74\\
10.144082560999	134.45\\
10.160264706999	133.03\\
10.175984986	131.77\\
10.192026612999	130.47\\
10.207989105999	129.14\\
10.223981236	127.82\\
10.240124277999	126.49\\
10.256097276	125.16\\
10.271948358001	123.84\\
10.288041841	122.5\\
10.303951234	121.19\\
10.31999156	119.86\\
10.33592093	118.54\\
10.352035539	117.22\\
10.368030851	115.88\\
10.384056888999	114.56\\
10.399920595998	113.22\\
10.415972885	111.92\\
10.432156515	110.58\\
10.448044362999	109.24\\
10.464087402999	107.94\\
10.479982383	106.6\\
10.496148132	105.29\\
10.513372022	103.78\\
10.528540703001	102.53\\
10.544053647	101.26\\
10.560011815	99.98\\
10.576109067999	98.65\\
10.592005682	97.3\\
10.607979361	96.01\\
10.624052746	94.68\\
10.639914897999	93.3\\
10.6559713	92.05\\
10.672234842	90.62\\
10.687913741	89.37\\
10.703948256	88.07\\
10.720129877	86.65\\
10.735925734	85.41\\
10.752046949	84.09\\
10.767943135	82.75\\
10.784094395	81.44\\
10.799970908	80.1\\
10.815983970998	78.78\\
10.834804408	77.16\\
10.850017948	75.9\\
10.865337556	74.63\\
10.880525637	73.37\\
10.895977541999	72.11\\
10.911889388999	70.79\\
10.927952345998	69.5\\
10.944094451	68.16\\
10.960065507	66.83\\
10.976092337	65.45\\
10.992096920999	64.19\\
11.008127901	63.83\\
11.023903133	65.11\\
11.039909378	66.32\\
11.055901694	67.55\\
11.071907626999	68.78\\
11.088015064001	70\\
11.104065385999	71.25\\
11.120023845	72.48\\
11.135981461999	73.7\\
11.152058976	74.92\\
11.167929535	76.15\\
11.184038691	77.36\\
11.199930698	78.59\\
11.215989691	79.8\\
11.234484114	81.28\\
11.249723097	82.45\\
11.264956352	83.61\\
11.280343225	84.79\\
11.296122062	85.95\\
11.311980047	87.18\\
11.32806518	88.39\\
11.343981203	89.6\\
11.360010653	90.85\\
11.376036583	92.05\\
11.392077147999	93.29\\
11.408007754	94.51\\
11.424012934001	95.72\\
11.440062915	96.95\\
11.456235635	98.24\\
11.472028443	99.41\\
11.488078569999	100.62\\
11.504073301	101.84\\
11.519977904	103.06\\
11.53599318	104.29\\
11.552029972	105.53\\
11.56798032	106.74\\
11.584033375999	107.97\\
11.599972084	109.2\\
11.616018603	110.42\\
11.634523696001	111.89\\
11.649688028999	113.05\\
11.664886989	114.2\\
11.680390174001	115.39\\
11.695926237	116.56\\
11.71190931	117.79\\
11.727875589	118.97\\
11.743911986	120.19\\
11.760053644001	121.42\\
11.775966746	122.65\\
11.792023588	123.87\\
11.807967375	125.09\\
11.823922444999	126.31\\
11.840028759	127.55\\
11.856032155	128.77\\
11.872016336999	130\\
11.888009083	131.21\\
11.904035998	132.44\\
11.919977699	133.68\\
11.935857483001	134.89\\
11.952036954999	136.11\\
11.968035449	137.34\\
11.984256403	138.57\\
12.000016939001	139.82\\
12.017560494999	139.88\\
12.03247585	138.79\\
12.047864833	137.71\\
12.063885265	136.58\\
12.079873098	135.41\\
12.098173656	134\\
12.113212676	132.91\\
12.128469734	131.8\\
12.144058034	130.69\\
12.159962854	129.54\\
12.17596301	128.38\\
12.192009549001	127.19\\
12.207986063	126.06\\
12.224010767999	124.89\\
12.240375397	123.69\\
12.256033720998	122.51\\
12.272056555	121.38\\
12.288006851	120.21\\
12.30410434	118.99\\
12.319972825	117.88\\
12.335965135999	116.71\\
12.352154592999	115.54\\
12.368009687998	114.36\\
12.384056838	113.18\\
12.399982026001	112.04\\
12.416007881	110.87\\
12.434541034	109.47\\
12.449684559999	108.37\\
12.464789587	107.27\\
12.480078069	106.15\\
12.495975579999	105.05\\
12.511942448	103.87\\
12.528017511	102.72\\
12.544077605	101.54\\
12.5600892	100.36\\
12.575945170001	99.2\\
12.592158551	98.05\\
12.607946706999	96.86\\
12.623922414	95.72\\
12.639998149	94.54\\
12.6560086	93.37\\
12.67203345	92.18\\
12.687876757	91.04\\
12.704038814	89.88\\
12.719965741999	88.71\\
12.736030500999	87.54\\
12.751999638999	86.37\\
12.767980077	85.2\\
12.784061721	84.04\\
12.799972553	82.86\\
12.816076605	81.71\\
12.831988899	80.54\\
12.848046979999	79.37\\
12.864161968	78.2\\
12.880015775	77.04\\
12.896016324	75.88\\
12.912078343	74.71\\
12.927985635999	73.53\\
12.944097798	72.37\\
12.959996339	71.2\\
12.975861052	70.04\\
12.992098198	68.87\\
13.00960153	68.77\\
13.02450654	69.79\\
13.039865182001	70.82\\
13.055856437998	71.87\\
13.071920678999	72.97\\
13.087844427	74.07\\
13.106111223	75.39\\
13.12134075	76.43\\
13.136617964	77.48\\
13.152053774	78.53\\
13.168000591	79.57\\
13.184044243	80.68\\
13.200011772	81.78\\
13.216034046	82.87\\
13.234467029998	84.19\\
13.249781112999	85.24\\
13.264989827	86.28\\
13.280114475999	87.32\\
13.296129588	88.36\\
13.31195268	89.46\\
13.328113161999	90.55\\
13.343881206	91.65\\
13.360058158	92.74\\
13.376000479999	93.84\\
13.392023942	94.94\\
13.407864623	96.03\\
13.423842306	97.13\\
13.439898691	98.24\\
13.455990404	99.35\\
13.4720969	100.44\\
13.48797321	101.52\\
13.504027147999	102.61\\
13.519922819	103.72\\
13.536081244	104.81\\
13.552035464	105.93\\
13.567996302999	107.02\\
13.584000005	108.11\\
13.599998765999	109.2\\
13.616023118	110.32\\
13.632283366	111.4\\
13.647991022999	112.52\\
13.664130556	113.6\\
13.679980522999	114.7\\
13.695958349	115.79\\
13.712011871	116.89\\
13.72798595	117.99\\
13.743994911999	119.08\\
13.760276003999	120.18\\
13.776010256	121.29\\
13.791952939	122.36\\
13.808018536	123.47\\
13.824021074999	124.57\\
13.840118793	125.66\\
13.855964266001	126.76\\
13.872050843999	127.91\\
13.887993182	128.95\\
13.904067209	130.05\\
13.920044975999	131.16\\
13.936018086	132.25\\
13.952172590999	133.35\\
13.968054261	134.47\\
13.984049949	135.55\\
14.000022445	136.63\\
14.017789032	136.68\\
14.032773384	135.68\\
14.047867139999	134.68\\
14.063894609	133.66\\
14.079895909	132.59\\
14.098202946999	131.3\\
14.113566948	130.27\\
14.128859491999	129.24\\
14.144041526001	128.23\\
14.159980909999	127.2\\
14.175934249	126.15\\
14.192146814	125.07\\
14.207996872	123.98\\
14.22387525	122.93\\
14.240025834	121.86\\
14.256004509	120.79\\
14.272060368	119.71\\
14.288016057999	118.63\\
14.304065411	117.56\\
14.320007643	116.49\\
14.33601371	115.42\\
14.352183195	114.35\\
14.367864873999	113.26\\
14.386233612	112\\
14.401300614999	110.99\\
14.416361739	109.98\\
14.432382099	108.96\\
14.447999071	107.87\\
14.464229885	106.85\\
14.480112862999	105.75\\
14.495975845	104.69\\
14.511940483001	103.63\\
14.527941032	102.57\\
14.544077284	101.5\\
14.560016394	100.42\\
14.576375321	99.34\\
14.592109015999	98.21\\
14.608098411999	97.19\\
14.623973263999	96.12\\
14.639993149	95.03\\
14.655960068	93.98\\
14.671906573	92.92\\
14.688046392	91.84\\
14.703958363	90.75\\
14.72001856	89.7\\
14.735897001	88.62\\
14.752022284	87.56\\
14.767986111	86.48\\
14.784016453	85.41\\
14.799976505	84.33\\
14.815993752	83.26\\
14.834507559001	81.98\\
14.849794325	80.95\\
14.865104264	79.93\\
14.880440738	78.9\\
14.895948347	77.87\\
14.911939267	76.84\\
14.927940298	75.78\\
14.944281434	74.7\\
14.959935807	73.6\\
14.975986306	72.56\\
14.991892911	71.48\\
15.009550467	71.42\\
15.02452736	72.35\\
15.039907303999	73.3\\
15.055840508	74.25\\
15.07397783	75.45\\
15.088937404	76.39\\
15.104323495	77.35\\
15.119972821	78.3\\
15.135955296	79.28\\
15.152026376	80.29\\
15.168004269001	81.29\\
15.186341385	82.49\\
15.201449420999	83.43\\
15.216660781	84.39\\
15.232433589	85.34\\
15.248071822	86.33\\
15.263966189	87.31\\
15.279820655	88.3\\
15.295982123	89.29\\
15.311974734	90.31\\
15.327986107999	91.3\\
15.344047215001	92.31\\
15.359959454	93.31\\
15.376008557999	94.31\\
15.392066658	95.31\\
15.40797341	96.32\\
15.424022279	97.33\\
15.440065109	98.33\\
15.455997363	99.33\\
15.471840787	100.33\\
15.488019229	101.31\\
15.503926421	102.34\\
15.520010175	103.32\\
15.536008609	104.33\\
15.552173525	105.34\\
15.567989878	106.35\\
15.584068944999	107.37\\
15.600046157	108.34\\
15.61609285	109.35\\
15.634569718999	110.56\\
15.649782828	111.51\\
15.665004074999	112.46\\
15.680390798	113.43\\
15.696055972	114.38\\
15.711845974	115.37\\
15.728014559999	116.35\\
15.744022796	117.37\\
15.760030151	118.38\\
15.775881149	119.37\\
15.79206209	120.36\\
15.808008632	121.38\\
15.824023326	122.37\\
15.840145046999	123.38\\
15.855994563	124.4\\
15.872064826	125.39\\
15.888007224	126.39\\
15.903974984	127.4\\
15.920003246	128.39\\
15.936013581999	129.39\\
15.952010809	130.39\\
15.96832517	131.39\\
15.983819406	132.41\\
16.00206495	133.57\\
16.019403022	133.58\\
16.03456471	132.66\\
16.049764352999	131.73\\
16.064994267	130.8\\
16.08076548	129.84\\
16.096035166999	128.91\\
16.112120323	127.97\\
16.128034145999	126.99\\
16.144062734	126.03\\
16.160118437998	125.05\\
16.175974956999	124.06\\
16.191942256	123.09\\
16.208035774	122.12\\
16.224006607	121.14\\
16.239858303	120.16\\
16.255977838	119.21\\
16.2721797	118.22\\
16.28800235	117.23\\
16.304018704999	116.26\\
16.31995119	115.29\\
16.335959381	114.32\\
16.352013461	113.33\\
16.368063464	112.35\\
16.384035183	111.37\\
16.399968324	110.39\\
16.415970772999	109.43\\
16.434320122	108.26\\
16.449412328	107.34\\
16.464997478	106.39\\
16.480153784999	105.46\\
16.495998586	104.54\\
16.51190672	103.57\\
16.528009477	102.6\\
16.544085477	101.59\\
16.560015907	100.62\\
16.576020225999	99.65\\
16.592056892	98.67\\
16.607972854	97.7\\
16.62400711	96.72\\
16.640064001999	95.74\\
16.656006229	94.77\\
16.671870535	93.8\\
16.687981304	92.83\\
16.704004005	91.84\\
16.720034144999	90.86\\
16.736013943	89.9\\
16.751919159	88.92\\
16.767913144	87.94\\
16.783998248	86.97\\
16.800050272999	85.98\\
16.816044236	85\\
16.832235359999	84.03\\
16.84796163	83.04\\
16.86413231	82.09\\
16.879940187	81.09\\
16.896020772	80.14\\
16.911931434001	79.15\\
16.928013862999	78.18\\
16.944047248	77.2\\
16.959877193	76.18\\
16.97602161	75.25\\
16.992036733001	74.22\\
17.009627918	74.19\\
17.024793991	75.05\\
17.040094482999	75.92\\
17.05595728	76.78\\
17.072031306	77.68\\
17.087965579999	78.59\\
17.104019866	79.49\\
17.120061128	80.4\\
17.135997291	81.32\\
17.152042130999	82.22\\
17.16798919	83.13\\
17.184006903	84.04\\
17.199978909	84.94\\
17.216009458999	85.84\\
17.234523791999	86.94\\
17.249798842	87.8\\
17.264880623	88.66\\
17.279978974	89.46\\
17.296017157	90.38\\
17.31201062	91.32\\
17.328059404	92.19\\
17.344004482	93.1\\
17.360257709	94.01\\
17.376006172	94.93\\
17.392080183	95.87\\
17.407799265	96.73\\
17.425817778	97.78\\
17.440860018	98.63\\
17.455829196	99.48\\
17.473918075	100.51\\
17.489143678	101.37\\
17.504282684	102.23\\
17.520007598	103.08\\
17.535875063	103.97\\
17.552018309	104.88\\
17.567828176	105.8\\
17.584108955	106.7\\
17.599959525	107.61\\
17.616097661	108.51\\
17.632501504	109.43\\
17.648007092	110.36\\
17.664151020999	111.24\\
17.679889131	112.15\\
17.69603556	113.04\\
17.712005996	113.96\\
17.727918677999	114.87\\
17.744091622	115.76\\
17.76008281	116.68\\
17.775876505999	117.59\\
17.792010953001	118.49\\
17.808075462	119.4\\
17.82398914	120.31\\
17.840040824999	121.21\\
17.856006696	122.12\\
17.872046214	123.04\\
17.887962411	123.93\\
17.904029126	124.83\\
17.920004914	125.75\\
17.936019951	126.65\\
17.952117796	127.56\\
17.968079777999	128.47\\
17.984094744999	129.39\\
18.000065512	130.28\\
18.017497706	130.34\\
18.03253838	129.51\\
18.048045005	128.67\\
18.066601663999	127.64\\
18.081832378	126.8\\
18.097029364999	125.96\\
18.112272596	125.12\\
18.128010836999	124.29\\
18.1439226	123.41\\
18.160088439	122.52\\
18.175982658	121.64\\
18.191984906	120.76\\
18.208027295	119.89\\
18.224033838	119\\
18.240024315	118.11\\
18.256037963	117.24\\
18.272029286999	116.36\\
18.287981344	115.46\\
18.304106518	114.6\\
18.319988183	113.72\\
18.336005810999	112.84\\
18.351972774	111.96\\
18.368023783	111.07\\
18.384092361001	110.19\\
18.399939213	109.31\\
18.415990671999	108.43\\
18.434531329999	107.37\\
18.449667437998	106.53\\
18.465152720998	105.68\\
18.480198088	104.85\\
18.495990235	104.02\\
18.512027734999	103.13\\
18.527834328	102.26\\
18.5440827	101.39\\
18.560148588	100.48\\
18.576001552	99.61\\
18.592075339	98.72\\
18.607902619	97.84\\
18.624041307	96.97\\
18.640771659999	96\\
18.655953864999	95.16\\
18.672027038999	94.33\\
18.687995736	93.44\\
18.70395202	92.56\\
18.719908687	91.69\\
18.735990763999	90.8\\
18.752067531	89.91\\
18.768009647999	89.03\\
18.783935543	88.15\\
18.799979555	87.27\\
18.816018331999	86.39\\
18.834582348	85.32\\
18.849761394	84.49\\
18.865073888999	83.65\\
18.880342684001	82.8\\
18.89610797	81.96\\
18.911972321	81.06\\
18.928001916	80.22\\
18.943993460001	79.33\\
18.959990347	78.41\\
18.975887741	77.57\\
18.992004519	76.69\\
19.009709885	76.64\\
19.024927638	77.45\\
19.040451401	78.26\\
19.055956041	79.07\\
19.072031218	79.89\\
19.088041967999	80.73\\
19.103912186	81.58\\
19.119862254	82.41\\
19.135983513999	83.25\\
19.152031085	84.11\\
19.167980574	84.95\\
19.184192171	85.79\\
19.200045231	86.65\\
19.215858188	87.48\\
19.234773623	88.51\\
19.250032175	89.31\\
19.265406936	90.12\\
19.280767359	90.93\\
19.29604106	91.73\\
19.31188201	92.55\\
19.328025361	93.37\\
19.34408366	94.23\\
19.360014050999	95.09\\
19.376090939	95.91\\
19.39209498	96.77\\
19.407949465	97.6\\
19.423978137	98.43\\
19.440051451	99.28\\
19.456015472999	100.13\\
19.472088872	100.98\\
19.487986328	101.83\\
19.504009791999	102.66\\
19.520072657	103.51\\
19.536079704	104.39\\
19.552034719	105.19\\
19.56803992	106.04\\
19.584230605	106.88\\
19.600029135	107.73\\
19.616021953999	108.57\\
19.634639953999	109.58\\
19.649848926	110.39\\
19.665024189999	111.19\\
19.680410888999	112\\
19.696148441	112.8\\
19.712009459	113.63\\
19.728005833	114.47\\
19.74395588	115.31\\
19.760728033	116.23\\
19.776014863	117.04\\
19.792276312	117.9\\
19.807992824999	118.69\\
19.823941837	119.52\\
19.840065770999	120.37\\
19.856023538	121.22\\
19.872009428999	122.1\\
19.888001359001	122.9\\
19.904054642999	123.75\\
19.919949869	124.59\\
19.936042215999	125.43\\
19.952111454	126.32\\
19.967947453	127.12\\
19.983960831	127.96\\
19.999976484	128.8\\
20.017493156	128.85\\
20.03282739	128.07\\
20.048049222999	127.29\\
20.06424053	126.51\\
20.080012102999	125.65\\
20.095978322	124.86\\
20.112029015	124.05\\
20.128083406	123.24\\
20.143983081999	122.41\\
20.159991804999	121.6\\
20.176057888	120.79\\
20.1920508	119.95\\
20.208075574	119.14\\
20.223995666	118.32\\
20.240051751	117.49\\
20.2560456	116.69\\
20.272116409	115.87\\
20.287985515	115.05\\
20.304025434	114.24\\
20.320067036	113.42\\
20.336013102	112.6\\
20.352014154998	111.78\\
20.368054161999	110.97\\
20.38410214	110.15\\
20.400005823	109.33\\
20.415995498	108.52\\
20.434443119999	107.53\\
20.449770913	106.75\\
20.465435375999	105.95\\
20.480650323999	105.17\\
20.496020347	104.39\\
20.511824336999	103.6\\
20.527958810999	102.8\\
20.543974525	101.96\\
20.560063089	101.15\\
20.576043376999	100.33\\
20.592003472	99.5\\
20.608019383	98.69\\
20.623984262	97.88\\
20.64004381	97.06\\
20.655983481	96.25\\
20.672025359	95.43\\
20.688009872	94.6\\
20.704024604001	93.78\\
20.719946175	92.97\\
20.73588318	92.15\\
20.75204121	91.34\\
20.770311039	90.36\\
20.785395727	89.59\\
20.800616484	88.81\\
20.816032923999	88.04\\
20.834601970998	87.07\\
20.849730296	86.3\\
20.864924850999	85.52\\
20.880242606999	84.74\\
20.896004104	83.96\\
20.912067836999	83.15\\
20.928000086	82.33\\
20.943957354	81.52\\
20.960086883999	80.69\\
20.975915477	79.88\\
20.992106288	79.05\\
21.009762692	78.99\\
21.024977781	79.71\\
21.040581821	80.44\\
21.056033902	81.15\\
21.071952322	81.87\\
21.088016369	82.62\\
21.10402227	83.38\\
21.120014346	84.12\\
21.135950284	84.87\\
21.15203378	85.62\\
21.168013352999	86.37\\
21.184059850999	87.12\\
21.200048859	87.86\\
21.21597963	88.61\\
21.234472621	89.51\\
21.249756944999	90.22\\
21.265078215	90.93\\
21.280297473	91.65\\
21.296009779	92.36\\
21.31201338	93.09\\
21.328014765999	93.85\\
21.3440416	94.6\\
21.359896396	95.35\\
21.375940980999	96.08\\
21.392043736	96.85\\
21.407993208	97.58\\
21.423972855	98.33\\
21.440022556	99.11\\
21.456239704	99.88\\
21.471990541999	100.61\\
21.487973704	101.33\\
21.503975085	102.09\\
21.519974857999	102.82\\
21.535978564	103.56\\
21.552075699	104.31\\
21.567947154	105.07\\
21.584094978	105.82\\
21.599928538	106.57\\
21.616016533999	107.32\\
21.634517393	108.22\\
21.649739913999	108.93\\
21.664869699	109.64\\
21.680116004	110.35\\
21.696027441	111.06\\
21.711938341	111.81\\
21.72801598	112.55\\
21.744070383	113.3\\
21.760023018	114.05\\
21.775926895999	114.79\\
21.792024252	115.54\\
21.807955454	116.29\\
21.823999227	117.04\\
21.840029718999	117.79\\
21.855974482999	118.55\\
21.872056741	119.29\\
21.887967016001	120.04\\
21.904030022	120.78\\
21.91997938	121.53\\
21.936109733001	122.28\\
21.95205623	123.04\\
21.968029663	123.79\\
21.9840204	124.53\\
21.999970239	125.27\\
22.017351734	125.32\\
22.032540787	124.65\\
22.04783459	123.97\\
22.066224973	123.14\\
22.081763605999	122.44\\
22.09693493	121.76\\
22.112235623	121.08\\
22.127929696999	120.4\\
22.143982327	119.69\\
22.15996359	118.97\\
22.175969597	118.25\\
22.192017972	117.54\\
22.208001048	116.82\\
22.223996736999	116.1\\
22.240029716	115.38\\
22.255991581999	114.67\\
22.271982092999	113.96\\
22.287991652999	113.23\\
22.304061534999	112.52\\
22.319964868999	111.79\\
22.336085239999	111.09\\
22.352055095	110.37\\
22.368077744	109.65\\
22.384043388999	108.93\\
22.399936331999	108.2\\
22.416055496	107.51\\
22.434517437	106.64\\
22.449687704	105.96\\
22.465292515999	105.26\\
22.480592902999	104.58\\
22.495998288	103.9\\
22.512004225	103.2\\
22.527986422999	102.49\\
22.544024192	101.78\\
22.560001327	101.05\\
22.575952285	100.34\\
22.591984495	99.62\\
22.608023302	98.91\\
22.623925439999	98.18\\
22.640069182	97.47\\
22.656011589	96.75\\
22.672055376999	96.01\\
22.687967946	95.32\\
22.704058850999	94.61\\
22.720014392	93.87\\
22.735912647999	93.17\\
22.751924676	92.47\\
22.767938114999	91.74\\
22.783938391001	91.02\\
22.799989482	90.31\\
22.816048090001	89.59\\
22.834502135	88.73\\
22.849594850999	88.05\\
22.864729698	87.37\\
22.879978818	86.69\\
22.895949350999	86.01\\
22.912043635999	85.29\\
22.928026781	84.57\\
22.944087095998	83.85\\
22.960069812998	83.13\\
22.976015923	82.42\\
22.992062013999	81.7\\
23.009534189	81.65\\
23.024608769	82.29\\
23.040070609	82.94\\
23.05597376	83.6\\
23.072030131	84.29\\
23.087929331	84.97\\
23.104014829	85.66\\
23.119969593	86.34\\
23.135969015	87.03\\
23.152023921999	87.72\\
23.167934052001	88.4\\
23.184052251999	89.08\\
23.200019701999	89.77\\
23.216017901	90.44\\
23.234439724	91.27\\
23.249852111	91.93\\
23.265173720998	92.59\\
23.280474357999	93.25\\
23.29601101	93.89\\
23.31201694	94.57\\
23.327991418	95.25\\
23.344044114999	95.93\\
23.360084846	96.62\\
23.375958888	97.31\\
23.392076915	97.98\\
23.407992618	98.68\\
23.423957795	99.36\\
23.440052899	100.06\\
23.456163232	100.77\\
23.472039729	101.42\\
23.488012621	102.1\\
23.504052721	102.79\\
23.520007762	103.47\\
23.536022002	104.15\\
23.552045388	104.85\\
23.568023758	105.53\\
23.584123241	106.22\\
23.600043614999	106.91\\
23.616114987999	107.58\\
23.632397291	108.27\\
23.64808088	108.97\\
23.666428737999	109.77\\
23.681589427999	110.42\\
23.6966781	111.06\\
23.711999093999	111.72\\
23.727992953001	112.37\\
23.744068753	113.06\\
23.760031532	113.74\\
23.775939701	114.43\\
23.792027218	115.11\\
23.80803032	115.8\\
23.823977697	116.48\\
23.840075692	117.16\\
23.855981661999	117.85\\
23.871928043	118.55\\
23.888001709	119.22\\
23.903998673999	119.9\\
23.919929475999	120.58\\
23.935990222	121.27\\
23.952108703	121.96\\
23.968095432999	122.65\\
23.984006417	123.33\\
23.999991632	124.01\\
24.017577141001	124.04\\
24.032937975	123.39\\
24.048311552	122.73\\
24.064127338	122.07\\
24.080155053	121.39\\
24.095962092999	120.71\\
24.111982303	120.03\\
24.127988966	119.35\\
24.144101578999	118.67\\
24.160018985	117.97\\
24.176007988	117.29\\
24.191894669	116.6\\
24.207921776	115.94\\
24.224001448	115.24\\
24.240180122	114.54\\
24.256010735	113.86\\
24.272017609	113.18\\
24.288020412	112.49\\
24.303968885999	111.8\\
24.32000889	111.13\\
24.335997908	110.44\\
24.351990196	109.76\\
24.368043114999	109.08\\
24.383991663	108.39\\
24.399972271	107.71\\
24.416003975	107.01\\
24.434474945	106.2\\
24.450102279	105.53\\
24.465295897999	104.88\\
24.480771653	104.21\\
24.496060803999	103.56\\
24.512003107	102.9\\
24.528073124	102.22\\
24.543998588	101.53\\
24.55995197	100.85\\
24.575917684999	100.17\\
24.592023366	99.49\\
24.607957111	98.8\\
24.623969682	98.11\\
24.640020125	97.43\\
24.656025559	96.74\\
24.672080602999	96.05\\
24.68798977	95.38\\
24.703961252	94.68\\
24.719822383	94.02\\
24.735900435	93.34\\
24.752035566	92.62\\
24.767987496	91.94\\
24.783984532	91.26\\
24.800016437998	90.58\\
24.815988181	89.9\\
24.834403765999	89.07\\
24.849631809	88.42\\
24.864910397	87.77\\
24.880189866999	87.12\\
24.896009149	86.46\\
24.911996761	85.78\\
24.927961333	85.1\\
24.943925779	84.41\\
24.960013817	83.74\\
24.975866423999	83.05\\
24.992021992	82.36\\
25.009681012	82.31\\
25.025092525	82.94\\
25.040296258	83.56\\
25.05617909	84.18\\
25.072032887	84.83\\
25.08804854	85.48\\
25.104011262	86.13\\
25.120005885	86.78\\
25.135998161	87.44\\
25.152077116999	88.11\\
25.167922989999	88.74\\
25.184030689	89.39\\
25.200034891001	90.05\\
25.215951822	90.7\\
25.234357067	91.48\\
25.249414418	92.1\\
25.264503545001	92.71\\
25.279999724	93.34\\
25.296006925	93.96\\
25.311937687	94.62\\
25.328014934	95.27\\
25.344008833	95.93\\
25.359948472	96.59\\
25.376002551	97.23\\
25.392012532	97.89\\
25.40789613	98.54\\
25.424009998	99.19\\
25.440062941	99.85\\
25.456035371	100.5\\
25.472028905	101.16\\
25.487951893	101.8\\
25.504156645	102.5\\
25.519992689	103.13\\
25.535995817	103.76\\
25.552146026	104.42\\
25.568001331	105.09\\
25.583997508	105.73\\
25.599903878	106.38\\
25.615990184999	107.03\\
25.634316020999	107.81\\
25.649362066	108.43\\
25.664585819	109.05\\
25.679921402999	109.67\\
25.696023710999	110.29\\
25.71205505	110.96\\
25.727958248001	111.61\\
25.74398532	112.26\\
25.760544184001	112.91\\
25.776044392	113.59\\
25.792072949999	114.25\\
25.808013737999	114.87\\
25.823985674001	115.52\\
25.839904151	116.19\\
25.85616817	116.83\\
25.872056569	117.5\\
25.888041277999	118.15\\
25.904072102	118.81\\
25.919908407	119.45\\
25.935921913	120.09\\
25.952062463	120.75\\
25.968063349	121.41\\
25.984055864	122.07\\
25.999969616999	122.72\\
26.017540861999	122.76\\
26.032839043999	122.17\\
26.047993605999	121.58\\
26.064009916	120.99\\
26.079827198999	120.36\\
26.098141008	119.63\\
26.113453933	119.03\\
26.128610349	118.45\\
26.144031263999	117.86\\
26.159918010999	117.23\\
26.176022550999	116.64\\
26.191981972	116\\
26.207959309	115.39\\
26.223974827	114.77\\
26.240054325	114.15\\
26.256056788	113.52\\
26.272041682	112.9\\
26.287993578	112.28\\
26.303883399	111.67\\
26.319939525	111.05\\
26.336009536999	110.42\\
26.352000102	109.79\\
26.367898156	109.18\\
26.384068214	108.55\\
26.399955734	107.93\\
26.415989541999	107.32\\
26.434405703	106.57\\
26.449444282	105.98\\
26.46475617	105.39\\
26.479978314	104.8\\
26.496166689	104.2\\
26.512157321	103.55\\
26.527995557	102.95\\
26.544065633999	102.34\\
26.560052783	101.71\\
26.575960456999	101.09\\
26.592075178999	100.47\\
26.607934059001	99.85\\
26.623991892	99.23\\
26.640124706999	98.61\\
26.655995239	97.98\\
26.672010438	97.36\\
26.688019642	96.75\\
26.704033349	96.13\\
26.720001628	95.5\\
26.735975346	94.87\\
26.751926117	94.26\\
26.768012141001	93.63\\
26.784168308	93.02\\
26.799892089	92.39\\
26.816047264999	91.78\\
26.834489079999	91.03\\
26.849754114999	90.43\\
26.864877747	89.85\\
26.880225446	89.25\\
26.895972687	88.66\\
26.911969347	88.04\\
26.928016948	87.42\\
26.943926956999	86.81\\
26.96011478	86.18\\
26.976006372	85.55\\
26.991959567	84.94\\
27.00989269	84.9\\
27.025224464	85.46\\
27.040535048	86.02\\
27.056012232	86.58\\
27.072056364999	87.14\\
27.088016083	87.73\\
27.104071093999	88.31\\
27.119972397999	88.89\\
27.135917826	89.47\\
27.151831198	90.05\\
27.167886529	90.64\\
27.183955097	91.22\\
27.199986645999	91.81\\
27.215969935999	92.39\\
27.234443427	93.09\\
27.249546388	93.64\\
27.264776384	94.2\\
27.280064931	94.76\\
27.295995189	95.31\\
27.3119701	95.89\\
27.327969060999	96.47\\
27.343965476	97.05\\
27.359904676	97.65\\
27.376058257	98.23\\
27.392009248999	98.82\\
27.408022122	99.4\\
27.423972186	99.97\\
27.440025044	100.56\\
27.455981073	101.17\\
27.471949042	101.73\\
27.488003735	102.31\\
27.504399645999	102.91\\
27.519948727999	103.5\\
27.536019876	104.06\\
27.551957571001	104.65\\
27.56805625	105.23\\
27.584031987999	105.81\\
27.60000521	106.4\\
27.616027779	106.98\\
27.634544428	107.68\\
27.649759671999	108.24\\
27.664976525	108.79\\
27.680240467	109.35\\
27.696027619	109.9\\
27.711987028	110.5\\
27.728021614999	111.06\\
27.744090809999	111.65\\
27.760037378	112.23\\
27.776026908	112.82\\
27.792116614	113.4\\
27.807921916	114\\
27.82405684	114.56\\
27.839919945	115.15\\
27.856036736001	115.73\\
27.872040015999	116.31\\
27.887971323	116.9\\
27.904043059	117.48\\
27.919954230999	118.07\\
27.935983335	118.64\\
27.952056764999	119.24\\
27.967934581999	119.82\\
27.984016020999	120.39\\
27.99989418	120.98\\
28.015850532	120.86\\
28.034244811	120.2\\
28.049391954999	119.67\\
28.064723569	119.14\\
28.079943883	118.61\\
28.097052482	118.01\\
28.111956521999	117.53\\
28.128009922	116.96\\
28.144077158	116.4\\
28.160014489	115.84\\
28.176002971	115.28\\
28.191995552	114.72\\
28.207991165	114.17\\
28.223982307999	113.61\\
28.240053628	113.05\\
28.255864515	112.49\\
28.272017151	111.94\\
28.288007307	111.38\\
28.304012366	110.82\\
28.319973341	110.26\\
28.336043727	109.7\\
28.352089857	109.14\\
28.367995696	108.58\\
28.384031444999	108.03\\
28.400052247	107.47\\
28.41599321	106.91\\
28.434443156999	106.24\\
28.449611310999	105.71\\
28.464098809001	105.21\\
28.479922827	104.67\\
28.4959888	104.12\\
28.512077611	103.56\\
28.528036619	103\\
28.544114892	102.45\\
28.559978068	101.88\\
28.575994795	101.33\\
28.592002194	100.76\\
28.607988675999	100.22\\
28.624093487	99.66\\
28.640106009	99.09\\
28.656036539	98.54\\
28.672001964	97.98\\
28.68804255	97.42\\
28.703869684	96.85\\
28.719909882	96.32\\
28.735976163001	95.76\\
28.752132201	95.17\\
28.768095573	94.63\\
28.783955012	94.07\\
28.799946022999	93.52\\
28.815915305	92.96\\
28.834280314999	92.3\\
28.849457387	91.77\\
28.864673864	91.24\\
28.879983175999	90.7\\
28.896004959	90.17\\
28.912011066	89.61\\
28.927982559	89.05\\
28.9440542	88.49\\
28.959905001999	87.93\\
28.975905854	87.38\\
28.991975517999	86.82\\
29.009853364999	86.79\\
29.025078006	87.29\\
29.040235003	87.78\\
29.056251027	88.27\\
29.072022768	88.8\\
29.088005741999	89.31\\
29.104097377	89.83\\
29.120010178	90.35\\
29.136009781	90.87\\
29.152037246	91.39\\
29.168050739	91.91\\
29.184035512999	92.43\\
29.200079246	92.95\\
29.215913354	93.46\\
29.234475409	94.09\\
29.249875946	94.59\\
29.265128324	95.09\\
29.280435884	95.59\\
29.295969435	96.08\\
29.312010511	96.59\\
29.328036263	97.1\\
29.344079128	97.62\\
29.360133129	98.15\\
29.375938193	98.67\\
29.392017322	99.18\\
29.408008138999	99.7\\
29.423937517	100.22\\
29.440065114	100.75\\
29.456704342	101.31\\
29.472169206999	101.82\\
29.488041454999	102.32\\
29.504080087999	102.83\\
29.520010350999	103.35\\
29.536025057	103.87\\
29.55206689	104.39\\
29.568006266001	104.91\\
29.584101364	105.43\\
29.599946596	105.95\\
29.61602834	106.46\\
29.634463017	107.09\\
29.649576967	107.58\\
29.664765395	108.08\\
29.67997877	108.57\\
29.695976783	109.06\\
29.712001607999	109.59\\
29.727915313	110.11\\
29.743989511	110.63\\
29.759939725999	111.15\\
29.775945435	111.67\\
29.791978721	112.19\\
29.80797472	112.7\\
29.824025788	113.22\\
29.840055258999	113.75\\
29.855970071	114.27\\
29.872066144999	114.79\\
29.888035218	115.31\\
29.904056871	115.83\\
29.919974927	116.35\\
29.936006868	116.86\\
29.951973256	117.38\\
29.968026734	117.9\\
29.984062057	118.43\\
29.999900452	118.95\\
30.017497020999	118.98\\
30.03274079	118.51\\
30.048017684	118.04\\
30.066347584	117.47\\
30.081743692999	117\\
30.096898806	116.53\\
30.112229907	116.05\\
30.127972866999	115.59\\
30.143997727	115.1\\
30.160025803999	114.6\\
30.176062229999	114.1\\
30.191979527999	113.6\\
30.207981061	113.12\\
30.224009364	112.62\\
30.239981112	112.13\\
30.255882287	111.63\\
30.274345637	111.04\\
30.289488484	110.57\\
30.304749668	110.1\\
30.320063315	109.63\\
30.336003713	109.16\\
30.352072156	108.66\\
30.367992734	108.17\\
30.384138181	107.67\\
30.400007852999	107.18\\
30.416011857	106.69\\
30.432140636	106.19\\
30.448036984	105.69\\
30.46435821	105.2\\
30.479984225999	104.69\\
30.496078453	104.21\\
30.511908344	103.72\\
30.527977609999	103.23\\
30.544027501	102.73\\
30.560055784	102.23\\
30.575935626	101.74\\
30.592021065	101.24\\
30.607868125	100.75\\
30.62401139	100.26\\
30.640072928	99.76\\
30.656007698	99.26\\
30.672034597	98.77\\
30.687884738	98.28\\
30.704062111	97.79\\
30.719910908999	97.28\\
30.735945067	96.8\\
30.752026715	96.3\\
30.768014415	95.8\\
30.784134019	95.31\\
30.799989298999	94.81\\
30.816035347	94.32\\
30.832429478	93.82\\
30.848012844	93.32\\
30.866506209	92.73\\
30.88170125	92.26\\
30.896821021	91.8\\
30.91258014	91.31\\
30.927894659999	90.84\\
30.944056053	90.37\\
30.960028663999	89.87\\
30.975924991999	89.37\\
30.991960295	88.87\\
31.009604556	88.85\\
31.024889533	89.28\\
31.040148972	89.72\\
31.056025263999	90.16\\
31.072068035	90.62\\
31.087918408	91.06\\
31.104029463	91.52\\
31.119978706999	91.98\\
31.136022642	92.44\\
31.151893484	92.89\\
31.167882361	93.34\\
31.184022458	93.8\\
31.200017442	94.26\\
31.215912389	94.71\\
31.234331421999	95.26\\
31.249446302999	95.69\\
31.264551608001	96.12\\
31.279937909999	96.56\\
31.296009079	97\\
31.311812771	97.45\\
31.328117541	97.91\\
31.344062424001	98.37\\
31.360003513999	98.83\\
31.375817184	99.28\\
31.391984157	99.73\\
31.407976219	100.19\\
31.424010706	100.65\\
31.440042287	101.12\\
31.455853178	101.56\\
31.472058604999	102.02\\
31.488010468999	102.48\\
31.503948781	102.93\\
31.51999263	103.39\\
31.53598906	103.85\\
31.551962407	104.3\\
31.568000753	104.76\\
31.584063608001	105.22\\
31.599868421	105.67\\
31.618196239	106.21\\
31.633417151	106.65\\
31.648595557	107.08\\
31.664423739999	107.52\\
31.679925658	107.97\\
31.695989949	108.41\\
31.712062565	108.87\\
31.727967302	109.33\\
31.743882552999	109.78\\
31.760012742	110.25\\
31.776039873	110.69\\
31.792014697	111.15\\
31.807976895	111.61\\
31.823959522999	112.06\\
31.840079203	112.52\\
31.856014328999	112.98\\
31.872020520999	113.44\\
31.888009805	113.9\\
31.904115107	114.35\\
31.920022755	114.81\\
31.935942878	115.27\\
31.951964465	115.72\\
31.967978694	116.19\\
31.983873482	116.63\\
31.999891960999	117.09\\
32.015960718	116.98\\
32.032093843	116.52\\
32.048088178	116.05\\
32.064486819999	115.59\\
32.079922989999	115.11\\
32.096005524999	114.67\\
32.112090185	114.2\\
32.128021035	113.74\\
32.1439596	113.28\\
32.160001368	112.82\\
32.176013428	112.35\\
32.191992163	111.87\\
32.207988458	111.43\\
32.223985538999	110.96\\
32.240146182	110.49\\
32.255976880999	110.04\\
32.271993622	109.58\\
32.28786704	109.11\\
32.304072296	108.63\\
32.319931885	108.18\\
32.335977100999	107.72\\
32.352067732999	107.26\\
32.36802413	106.79\\
32.384005961	106.34\\
32.399935721	105.87\\
32.415997997	105.41\\
32.434333923	104.85\\
32.449435214	104.42\\
32.463895548	104.02\\
32.480025935	103.55\\
32.496003607999	103.09\\
32.512075999	102.63\\
32.528020536999	102.16\\
32.544026293	101.71\\
32.561061619999	101.19\\
32.576060913	100.75\\
32.592083862999	100.3\\
32.608023669	99.85\\
32.623977924001	99.39\\
32.640148313	98.93\\
32.655936418	98.45\\
32.672049014	98\\
32.687935739	97.53\\
32.703987185	97.08\\
32.719960093	96.61\\
32.73598895	96.15\\
32.752064149	95.68\\
32.767995221	95.22\\
32.784049116	94.76\\
32.799960678999	94.3\\
32.815980911999	93.83\\
32.834537347	93.27\\
32.849953685999	92.83\\
32.865311717	92.38\\
32.880604487999	91.94\\
32.896023424001	91.5\\
32.912023012	91.05\\
32.927830669	90.6\\
32.94386917	90.14\\
32.961936069999	89.59\\
32.976790326	89.16\\
32.992882786999	88.69\\
33.007798331	88.6\\
33.025784857	89.14\\
33.040736564999	89.57\\
33.055832678999	90\\
33.073896632	90.51\\
33.088787149999	90.94\\
33.103854128	91.36\\
33.121924633	91.88\\
33.136813931	92.31\\
33.151872149	92.73\\
33.169930267	93.25\\
33.184814934	93.68\\
33.200056748	94.11\\
33.215829229001	94.54\\
33.233899947	95.08\\
33.249069089	95.51\\
33.264328395999	95.95\\
33.280008812	96.38\\
33.296079638	96.83\\
33.311978333	97.3\\
33.328029199999	97.75\\
33.346452327	98.29\\
33.36174893	98.73\\
33.376980722001	99.16\\
33.392311968999	99.6\\
33.407968362999	100.04\\
33.426279465	100.57\\
33.441536137999	101\\
33.456593083	101.43\\
33.472163538	101.87\\
33.487983402	102.31\\
33.504028859	102.77\\
33.519992649999	103.23\\
33.536004354	103.68\\
33.551985715	104.14\\
33.567982143	104.59\\
33.584027992	105.05\\
33.599957129	105.51\\
33.615986188	105.97\\
33.632303317999	106.42\\
33.647898614999	106.88\\
33.664083421999	107.33\\
33.680060130999	107.79\\
33.696058611	108.25\\
33.712064193	108.71\\
33.727954728	109.16\\
33.744068889999	109.62\\
33.759996729	110.08\\
33.775927343	110.53\\
33.792055078	110.98\\
33.808035456999	111.45\\
33.823988599	111.9\\
33.839917725	112.35\\
33.856109069	112.8\\
33.8720449	113.27\\
33.88795154	113.74\\
33.904020796999	114.18\\
33.919926567999	114.63\\
33.935971807	115.09\\
33.951972568	115.55\\
33.967929823	116.01\\
33.984022868001	116.46\\
33.999925461	116.92\\
34.017426455	116.95\\
34.032849593	116.54\\
34.048138717	116.12\\
34.064070876	115.71\\
34.080064218999	115.26\\
34.095897252	114.86\\
34.112009805	114.43\\
34.127991372	113.99\\
34.147333229	113.45\\
34.159830405	113.13\\
34.175882585999	112.71\\
34.192086522	112.27\\
34.207979467	111.83\\
34.224084323	111.4\\
34.240026027999	110.97\\
34.256017743001	110.54\\
34.271959763999	110.11\\
34.287948491	109.68\\
34.304028983001	109.24\\
34.319946513	108.82\\
34.335982123	108.39\\
34.351981554	107.95\\
34.367942416	107.53\\
34.384001203001	107.1\\
34.399996878	106.66\\
34.416045916999	106.23\\
34.434437117	105.71\\
34.449815986001	105.3\\
34.465312157	104.88\\
34.480453218	104.47\\
34.496008979	104.06\\
34.512078133999	103.64\\
34.527998441	103.2\\
34.544090198	102.78\\
34.559992638	102.34\\
34.575966036	101.91\\
34.592015942999	101.48\\
34.608020626999	101.05\\
34.624022675	100.62\\
34.639988546	100.19\\
34.655973289	99.76\\
34.672090291999	99.33\\
34.687998852	98.89\\
34.703977892	98.47\\
34.719955585999	98.04\\
34.735972077	97.6\\
34.752030982999	97.18\\
34.768063565	96.74\\
34.784067186	96.31\\
34.800006225999	95.87\\
34.816004656	95.44\\
34.832234174001	95.02\\
34.847999172	94.58\\
34.864465742	94.15\\
34.880002067	93.7\\
34.895930915	93.29\\
34.912009689999	92.86\\
34.928122379	92.43\\
34.944071081	92\\
34.959821652	91.56\\
34.976028480999	91.14\\
34.992070260999	90.7\\
35.009689783	90.66\\
35.024945549001	91.04\\
35.040402335	91.42\\
35.05611855	91.79\\
35.072046394	92.18\\
35.087979571	92.57\\
35.104097486	92.96\\
35.119970808	93.36\\
35.136013045999	93.75\\
35.152022611	94.14\\
35.167955329999	94.54\\
35.184115277999	94.93\\
35.199952486	95.32\\
35.215940771	95.71\\
35.23441892	96.19\\
35.249521222	96.56\\
35.264809702	96.93\\
35.280463009	97.32\\
35.295911221	97.69\\
35.312002085	98.09\\
35.328007431	98.47\\
35.344020718	98.86\\
35.360062326	99.26\\
35.375956399	99.64\\
35.391954453	100.04\\
35.407983632	100.43\\
35.423940352	100.82\\
35.440023719	101.22\\
35.455998345	101.61\\
35.472016067999	102\\
35.487948984	102.4\\
35.504046732999	102.79\\
35.520009958	103.19\\
35.535975241999	103.58\\
35.551920479	103.97\\
35.567979031001	104.36\\
35.584006638999	104.76\\
35.599872065	105.15\\
35.615984453999	105.54\\
35.634533436	106.02\\
35.649601305	106.39\\
35.664770084	106.76\\
35.680009647999	107.14\\
35.696027711999	107.51\\
35.711957274999	107.91\\
35.728029945	108.29\\
35.744058576	108.69\\
35.759910546	109.09\\
35.776006154	109.47\\
35.792071003999	109.87\\
35.807998928	110.27\\
35.823993855	110.66\\
35.840078956	111.05\\
35.856012323	111.44\\
35.872066158	111.84\\
35.888015326	112.23\\
35.904074027	112.62\\
35.919903944	113.02\\
35.935884854001	113.4\\
35.952115816	113.81\\
35.968104010999	114.2\\
35.984023488	114.59\\
35.999965038001	114.98\\
36.016015435999	114.91\\
36.034483233001	114.47\\
36.049724571999	114.12\\
36.064900147	113.77\\
36.080292095	113.41\\
36.095908482999	113.06\\
36.111820598	112.7\\
36.130090371	112.27\\
36.145167817999	111.92\\
36.160417158999	111.57\\
36.175943322	111.22\\
36.192003717999	110.87\\
36.207967258999	110.5\\
36.223828901999	110.13\\
36.239994593	109.76\\
36.256024833	109.39\\
36.272045729	109.02\\
36.287980293999	108.66\\
36.304046531001	108.29\\
36.319974197	107.92\\
36.336001675999	107.55\\
36.352101315	107.18\\
36.368076718	106.81\\
36.384092590999	106.45\\
36.399986419	106.08\\
36.416022678999	105.72\\
36.434644732	105.27\\
36.449911888	104.92\\
36.46519003	104.57\\
36.480629552	104.21\\
36.496111272	103.85\\
36.512007458	103.5\\
36.527970339	103.14\\
36.544036303999	102.77\\
36.560105795	102.4\\
36.575999132	102.04\\
36.591926543	101.66\\
36.607983277	101.3\\
36.623974209	100.94\\
36.639837642	100.57\\
36.655966287	100.2\\
36.672057532	99.83\\
36.68805753	99.46\\
36.703952661	99.09\\
36.719913279	98.73\\
36.736031536	98.36\\
36.752019145	97.98\\
36.767986298999	97.62\\
36.784109864	97.25\\
36.800072385	96.88\\
36.815993029	96.52\\
36.832448639999	96.15\\
36.848059121	95.77\\
36.864009761	95.41\\
36.879815465999	95.05\\
36.896003197	94.69\\
36.911915952	94.31\\
36.927926317999	93.94\\
36.944074964	93.57\\
36.959925151	93.2\\
36.976065880001	92.84\\
36.992080771	92.47\\
37.009566032	92.43\\
37.024680987999	92.74\\
37.040963517	93.08\\
37.056145159999	93.39\\
37.07200317	93.7\\
37.087979676	94.03\\
37.104021784	94.36\\
37.119958424001	94.69\\
37.135940987	95.02\\
37.152057118	95.35\\
37.168002859	95.68\\
37.184003924001	96.01\\
37.19995826	96.34\\
37.215978809	96.66\\
37.234392037	97.06\\
37.249545246	97.38\\
37.264757304	97.69\\
37.280034597	98\\
37.296003093	98.32\\
37.311827446999	98.65\\
37.327886140999	98.97\\
37.346301343	99.37\\
37.361551105	99.68\\
37.376731125	100\\
37.391977944999	100.31\\
37.407968888	100.62\\
37.424112034999	100.95\\
37.440027182	101.29\\
37.456038991999	101.62\\
37.472066501	101.95\\
37.487928944999	102.28\\
37.50405208	102.6\\
37.519984243999	102.94\\
37.53602286	103.26\\
37.552095027	103.6\\
37.567974163	103.92\\
37.584024126999	104.26\\
37.599981862999	104.58\\
37.616005357	104.92\\
37.634691652999	105.31\\
37.649842052999	105.63\\
37.665048695	105.94\\
37.680229765	106.25\\
37.695865446	106.57\\
37.711974896	106.9\\
37.728024541999	107.22\\
37.744141539	107.55\\
37.760841027999	107.91\\
37.776008744	108.23\\
37.79193041	108.54\\
37.8080153	108.87\\
37.823965538999	109.21\\
37.840073833999	109.53\\
37.856008048	109.86\\
37.872031725	110.19\\
37.888021559	110.52\\
37.904063918	110.85\\
37.919990094	111.18\\
37.935879221	111.51\\
37.952027109	111.84\\
37.967949965	112.17\\
37.984116691	112.5\\
38.000054965999	112.83\\
38.015906036999	112.78\\
38.031863935	112.47\\
38.048028746	112.17\\
38.064030133	111.86\\
38.080009461	111.55\\
38.09584196	111.25\\
38.111975824	110.95\\
38.127984390999	110.64\\
38.144244498999	110.34\\
38.160008372	110.03\\
38.176009546999	109.73\\
38.191942444	109.42\\
38.207989633999	109.12\\
38.223952873	108.82\\
38.240078003	108.51\\
38.256056798	108.2\\
38.272144071	107.9\\
38.287854732999	107.6\\
38.304073434999	107.3\\
38.320267026	106.99\\
38.335980107	106.68\\
38.352058612	106.38\\
38.367968159999	106.08\\
38.384115380999	105.77\\
38.399969390999	105.46\\
38.416004958	105.16\\
38.432246092	104.86\\
38.447995949	104.55\\
38.464166268	104.25\\
38.479950229	103.94\\
38.496015647999	103.64\\
38.511920077	103.34\\
38.527987645	103.03\\
38.544148017	102.73\\
38.560100122	102.42\\
38.575937963	102.12\\
38.591882266001	101.81\\
38.607941916999	101.51\\
38.624083657	101.21\\
38.640040467	100.9\\
38.655970636	100.6\\
38.672068751	100.29\\
38.688003011	99.99\\
38.703936923999	99.68\\
38.72002379	99.38\\
38.735947905	99.08\\
38.751982331999	98.77\\
38.767999427999	98.47\\
38.784072983001	98.16\\
38.799962661	97.86\\
38.816046772	97.55\\
38.831992206999	97.25\\
38.847929407	96.95\\
38.864143463	96.64\\
38.880497599	96.31\\
38.896061402999	96.02\\
38.911994535	95.73\\
38.928120348	95.42\\
38.944064735	95.12\\
38.960089869999	94.81\\
38.975954972999	94.51\\
38.991938916999	94.19\\
39.009782095	94.17\\
39.025026528999	94.43\\
39.040577531	94.69\\
39.055958994999	94.94\\
39.072016393	95.2\\
39.08790704	95.46\\
39.10402002	95.73\\
39.119988706999	96\\
39.136037182	96.26\\
39.152034592	96.53\\
39.167977992	96.8\\
39.183970663	97.06\\
39.2000485	97.33\\
39.215972509	97.59\\
39.234470806	97.91\\
39.249751312	98.17\\
39.265118199999	98.42\\
39.28054101	98.68\\
39.295932050999	98.93\\
39.311895699	99.19\\
39.328021205	99.46\\
39.344077185	99.73\\
39.359962064	100\\
39.376018599	100.26\\
39.392021173	100.54\\
39.408013934	100.79\\
39.423984175	101.06\\
39.439967936	101.32\\
39.456045621	101.59\\
39.472139547	101.86\\
39.487996751	102.12\\
39.504099046	102.39\\
39.519976611	102.66\\
39.535997531999	102.92\\
39.552029527	103.19\\
39.567977558	103.45\\
39.584066888	103.72\\
39.600060372	103.99\\
39.615997236	104.25\\
39.634462402999	104.57\\
39.649786926	104.83\\
39.665151137	105.08\\
39.68054255	105.34\\
39.695894696999	105.59\\
39.712025123	105.85\\
39.728674804	106.14\\
39.743906116	106.39\\
39.760039454999	106.65\\
39.776002837	106.92\\
39.791987223	107.18\\
39.807934852	107.45\\
39.823923419	107.71\\
39.840020652	107.98\\
39.855974566	108.25\\
39.87205876	108.52\\
39.888018542	108.78\\
39.904026303	109.05\\
39.919901807999	109.31\\
39.936062542	109.58\\
39.951945744	109.85\\
39.968003243	110.11\\
39.984065708	110.38\\
40.000043675	110.65\\
40.017441937998	110.67\\
40.032631755	110.44\\
40.048028475	110.22\\
40.064127366999	109.98\\
40.079990656001	109.74\\
40.096093918	109.5\\
40.112004784	109.26\\
40.127911382	109.02\\
40.14395859	108.78\\
40.16003228	108.54\\
40.176074533999	108.29\\
40.191890042	108.05\\
40.210355576999	107.77\\
40.225396884	107.54\\
40.240430998	107.31\\
40.255899767	107.08\\
40.272112744	106.85\\
40.287911098	106.61\\
40.304075050999	106.37\\
40.320016585	106.12\\
40.336068375	105.89\\
40.352035259	105.64\\
40.368058154	105.4\\
40.384033336	105.16\\
40.399969705	104.92\\
40.416027277999	104.68\\
40.432358517	104.43\\
40.4482481	104.19\\
40.464052258999	103.95\\
40.479995458	103.71\\
40.496084088	103.48\\
40.511891125	103.23\\
40.527928335	102.99\\
40.544046852	102.75\\
40.559914166	102.51\\
40.575948593	102.27\\
40.591955534999	102.03\\
40.60795838	101.79\\
40.624039581	101.55\\
40.639935874	101.31\\
40.658078678999	101.02\\
40.67325254	100.79\\
40.688320607999	100.57\\
40.703908946	100.34\\
40.720034277	100.1\\
40.736029814	99.86\\
40.752040242	99.62\\
40.768016843999	99.38\\
40.78408579	99.14\\
40.799980079999	98.9\\
40.816000249	98.66\\
40.832117828999	98.42\\
40.848008128	98.17\\
40.864180444	97.93\\
40.879999284	97.69\\
40.895999838	97.45\\
40.911923763999	97.21\\
40.927996502	96.97\\
40.943977863	96.73\\
40.959872475	96.49\\
40.976007689	96.25\\
40.992055713	96\\
41.0096938	95.97\\
41.024825279	96.17\\
41.040053533	96.36\\
41.055945907	96.55\\
41.072079362	96.75\\
41.088110069999	96.96\\
41.104036902	97.17\\
41.119969008	97.36\\
41.136002611001	97.57\\
41.152006397999	97.77\\
41.168020613	97.98\\
41.184160087	98.18\\
41.200039760999	98.38\\
41.216012864999	98.58\\
41.232016684	98.78\\
41.247969296999	98.99\\
41.264015695	99.19\\
41.279932989999	99.39\\
41.296004753	99.59\\
41.311992662	99.8\\
41.328024655	100\\
41.343990524999	100.21\\
41.360073622	100.41\\
41.376058843999	100.61\\
41.392057394001	100.82\\
41.407975592	101.02\\
41.424083864999	101.22\\
41.440027274999	101.42\\
41.455853219	101.62\\
41.472015374	101.83\\
41.488095093	102.03\\
41.503835052	102.23\\
41.520020806	102.43\\
41.535986296	102.64\\
41.552059215001	102.84\\
41.56801003	103.05\\
41.584053282	103.25\\
41.599977206999	103.45\\
41.616090161	103.65\\
41.634683875001	103.9\\
41.649974699999	104.09\\
41.665271067999	104.29\\
41.681046248	104.49\\
41.696237685999	104.68\\
41.711966236	104.87\\
41.727953272999	105.07\\
41.743935557	105.28\\
41.760057321999	105.48\\
41.775906732999	105.69\\
41.791983579	105.88\\
41.808003852999	106.09\\
41.823969017	106.29\\
41.840032043	106.5\\
41.855980198	106.7\\
41.872034513999	106.9\\
41.888006043	107.1\\
41.903898423	107.31\\
41.919879265999	107.51\\
41.936032838	107.71\\
41.952004623	107.93\\
41.968003788	108.12\\
41.984062034	108.32\\
42.000123876999	108.53\\
42.017640974	108.55\\
42.032995454	108.38\\
42.048231675	108.21\\
42.064430755	108.04\\
42.079989439	107.86\\
42.096264599	107.69\\
42.111959237	107.51\\
42.127981628999	107.33\\
42.144095841	107.16\\
42.159843373001	106.98\\
42.175976241	106.8\\
42.191977876999	106.62\\
42.208018458	106.44\\
42.224045081	106.27\\
42.239936647	106.09\\
42.255905857	105.91\\
42.272057803999	105.73\\
42.287980873	105.56\\
42.304009817	105.38\\
42.319974251999	105.2\\
42.335992709	105.03\\
42.351959763	104.85\\
42.367975546999	104.67\\
42.384119991999	104.49\\
42.400056364999	104.31\\
42.416016954	104.14\\
42.432087979	103.96\\
42.447859869	103.78\\
42.466136744	103.57\\
42.481365383999	103.4\\
42.496446894	103.24\\
42.511962444	103.06\\
42.530331232999	102.86\\
42.545566175999	102.69\\
42.561257424001	102.52\\
42.576515105	102.35\\
42.591911393	102.18\\
42.608019361001	102.01\\
42.623976718	101.83\\
42.639860586	101.65\\
42.655998849	101.47\\
42.672027116999	101.3\\
42.687906501	101.12\\
42.704073182	100.94\\
42.719908397999	100.76\\
42.736038625	100.59\\
42.752051658	100.41\\
42.767950747	100.23\\
42.783958581999	100.05\\
42.79997363	99.88\\
42.81600968	99.7\\
42.834345578001	99.48\\
42.84944076	99.32\\
42.864660594	99.15\\
42.879970861001	98.98\\
42.896045997	98.81\\
42.912808017999	98.61\\
42.928142292	98.44\\
42.943874904	98.27\\
42.959863722	98.1\\
42.978108803999	97.89\\
42.993417545	97.72\\
43.007938375	97.68\\
43.024008622	97.85\\
43.040060262999	98.02\\
43.055919197	98.19\\
43.071997116	98.36\\
43.087962836	98.53\\
43.103903905	98.71\\
43.119980903	98.88\\
43.135870754	99.05\\
43.152032982	99.22\\
43.167983062	99.39\\
43.184034876999	99.56\\
43.199971557	99.73\\
43.215989922	99.9\\
43.232229974	100.07\\
43.248052179	100.25\\
43.266513903	100.45\\
43.281744581999	100.62\\
43.297606060999	100.79\\
43.312821221	100.95\\
43.327980938	101.11\\
43.344077263999	101.28\\
43.359968607	101.45\\
43.375943997	101.62\\
43.392058644	101.79\\
43.408027935	101.96\\
43.423978034999	102.13\\
43.440061156	102.3\\
43.456084862999	102.48\\
43.471917667	102.65\\
43.487950911	102.82\\
43.504084805	102.99\\
43.520048889	103.16\\
43.535978361999	103.33\\
43.551983483001	103.5\\
43.568009864	103.67\\
43.584091498	103.84\\
43.599915118	104.02\\
43.616026757	104.18\\
43.634405147	104.39\\
43.649477322	104.55\\
43.664630715999	104.71\\
43.680441444	104.88\\
43.696016971	105.05\\
43.712285569999	105.22\\
43.727903744999	105.39\\
43.744497353	105.56\\
43.760029357999	105.73\\
43.775842254	105.89\\
43.791983521	106.07\\
43.807984904	106.24\\
43.823983539	106.41\\
43.83996734	106.59\\
43.856012982	106.75\\
43.872001604	106.93\\
43.8879781	107.1\\
43.903957376	107.27\\
43.920022520999	107.44\\
43.935998498999	107.61\\
43.952022253	107.78\\
43.967938387	107.95\\
43.984018078999	108.12\\
43.999994022	108.29\\
44.017393977999	108.3\\
44.032640866999	108.13\\
44.048016220998	107.96\\
44.066384559	107.76\\
44.081745878999	107.59\\
44.096980662	107.42\\
44.112398008999	107.25\\
44.12801098	107.08\\
44.144018869	106.9\\
44.159998643	106.73\\
44.176014913001	106.55\\
44.191953642	106.37\\
44.207975356999	106.19\\
44.223941661	106.02\\
44.240067244	105.84\\
44.256073576	105.66\\
44.271833279	105.48\\
44.287993977	105.31\\
44.304036236	105.12\\
44.319959364	104.95\\
44.336016122	104.77\\
44.352060455	104.59\\
44.367970565	104.42\\
44.384001733001	104.24\\
44.399951913	104.06\\
44.415995041	103.88\\
44.432167181	103.71\\
44.447932054	103.53\\
44.464102515999	103.35\\
44.480058105	103.17\\
44.495961057999	103\\
44.511830665	102.82\\
44.527986981999	102.64\\
44.544052022999	102.46\\
44.560008772999	102.29\\
44.575882087	102.11\\
44.594069284999	101.9\\
44.609318451	101.73\\
44.624591921999	101.56\\
44.640067651	101.39\\
44.655897225999	101.22\\
44.672030268	101.04\\
44.687894706	100.87\\
44.704072311	100.69\\
44.719991319	100.51\\
44.736005279	100.33\\
44.751970162	100.16\\
44.767945408	99.98\\
44.784007927	99.8\\
44.800037892001	99.62\\
44.8160699	99.44\\
44.832173444	99.27\\
44.84804037	99.09\\
44.864346562	98.91\\
44.879919638999	98.73\\
44.896121923999	98.56\\
44.91192654	98.38\\
44.928074618	98.2\\
44.944017456	98.02\\
44.960091916999	97.85\\
44.976006714	97.67\\
44.992046831999	97.49\\
45.009758756	97.48\\
45.024897627	97.64\\
45.04024046	97.81\\
45.055985781	97.97\\
45.071968435999	98.14\\
45.088017038001	98.31\\
45.103980319	98.48\\
45.119974493	98.65\\
45.1359696	98.83\\
45.151965097	99\\
45.167985426	99.17\\
45.183920850999	99.34\\
45.200024300999	99.51\\
45.215905046	99.68\\
45.232477540999	99.85\\
45.248044124	100.03\\
45.264017366999	100.2\\
45.279922696001	100.37\\
45.296008549001	100.54\\
45.312063524999	100.71\\
45.328021612999	100.88\\
45.343889246	101.05\\
45.360031527999	101.22\\
45.376063774	101.39\\
45.392015878999	101.57\\
45.407913562998	101.73\\
45.423924642999	101.9\\
45.440025796	102.08\\
45.45606551	102.25\\
45.472127881	102.43\\
45.487948294	102.59\\
45.50398047	102.76\\
45.519932918	102.93\\
45.536015004	103.1\\
45.552073053999	103.28\\
45.568043395	103.45\\
45.584029045001	103.62\\
45.599990762	103.79\\
45.615953598	103.96\\
45.632231626	104.13\\
45.647987262999	104.31\\
45.664368897	104.48\\
45.679953628999	104.65\\
45.696020768	104.82\\
45.712017584	104.99\\
45.727918913	105.16\\
45.743981824	105.33\\
45.759902869999	105.5\\
45.776027043	105.67\\
45.791918409999	105.85\\
45.807893578	106.01\\
45.826168542	106.22\\
45.841345845	106.38\\
45.856439914	106.54\\
45.872030624	106.71\\
45.887923428	106.87\\
45.904024137999	107.04\\
45.919942569	107.22\\
45.93601654	107.39\\
45.952047192999	107.56\\
45.968041227	107.73\\
45.983963831	107.9\\
45.999927845	108.07\\
46.017819326	108.08\\
46.033131991999	107.91\\
46.048636076	107.74\\
46.064238517999	107.57\\
46.07993176	107.39\\
46.096031736001	107.22\\
46.111991976	107.04\\
46.127968171	106.86\\
46.144046758	106.69\\
46.160013326	106.5\\
46.176074527999	106.33\\
46.191980411999	106.15\\
46.208003232	105.98\\
46.223985133999	105.8\\
46.239906243001	105.62\\
46.258280248	105.41\\
46.273436378	105.24\\
46.288538742	105.07\\
46.304004069	104.91\\
46.319892965	104.73\\
46.336044375	104.56\\
46.352028758	104.37\\
46.367927191	104.2\\
46.384026283	104.02\\
46.399887673	103.85\\
46.416015931	103.67\\
46.434582678	103.45\\
46.449904602	103.28\\
46.465090328	103.11\\
46.480380717999	102.94\\
46.495953349	102.77\\
46.511906427999	102.6\\
46.527990018	102.43\\
46.54405301	102.25\\
46.560057868001	102.07\\
46.576065163	101.89\\
46.592114819	101.71\\
46.607930318	101.54\\
46.623925901	101.36\\
46.640006743001	101.18\\
46.656052915999	101\\
46.671979968	100.83\\
46.687991027	100.65\\
46.704031985	100.47\\
46.720013807	100.29\\
46.735960321	100.12\\
46.751997432999	99.94\\
46.767967042	99.76\\
46.783958646	99.58\\
46.799994822	99.41\\
46.815934846	99.23\\
46.832043496	99.05\\
46.848064895	98.87\\
46.864284279998	98.69\\
46.880000022	98.51\\
46.895980849	98.34\\
46.911940573	98.16\\
46.927982113	97.99\\
46.943842859	97.81\\
46.960043107	97.63\\
46.976001739999	97.45\\
46.991985380999	97.27\\
47.009644306	97.26\\
47.024844165	97.42\\
47.040085899	97.59\\
47.055941637	97.75\\
47.072057876999	97.92\\
47.087970376999	98.09\\
47.104088362	98.26\\
47.119996246	98.44\\
47.136013985	98.6\\
47.151973003	98.78\\
47.168023467	98.95\\
47.184151717001	99.12\\
47.200058423	99.29\\
47.216061412	99.46\\
47.234797518	99.67\\
47.249965375	99.83\\
47.265430034	100\\
47.280515774999	100.16\\
47.295999107	100.32\\
47.311985534	100.49\\
47.327994572	100.66\\
47.343999853	100.83\\
47.359983854	101\\
47.376014987999	101.17\\
47.391980252	101.34\\
47.407992168	101.51\\
47.423982318	101.69\\
47.439951314	101.86\\
47.456251309001	102.04\\
47.472021592	102.2\\
47.488012589	102.37\\
47.504073701	102.55\\
47.519927661	102.72\\
47.536018883	102.89\\
47.551949116999	103.06\\
47.567912960999	103.23\\
47.584128069999	103.4\\
47.599947677999	103.57\\
47.616017303999	103.74\\
47.632385522	103.91\\
47.648010185	104.09\\
47.666636546999	104.29\\
47.681863758	104.46\\
47.696971052	104.62\\
47.712101093	104.78\\
47.727993843	104.94\\
47.743986845	105.11\\
47.760069704999	105.28\\
47.776010456	105.46\\
47.792019843999	105.63\\
47.808001025	105.8\\
47.823983362	105.97\\
47.840068995	106.14\\
47.855951359	106.31\\
47.872129002	106.48\\
47.888016255	106.66\\
47.904046907	106.82\\
47.919933194	107\\
47.936017336	107.17\\
47.95208622	107.35\\
47.968054633	107.51\\
47.984036409	107.68\\
47.999974465999	107.85\\
48.017425088	107.87\\
48.03274087	107.73\\
48.04799871	107.59\\
48.064707626999	107.45\\
48.079999979	107.3\\
48.095977480999	107.16\\
48.111980126	107.02\\
48.128007556	106.87\\
48.144089084	106.73\\
48.159968465	106.57\\
48.175974355	106.43\\
48.192193505	106.29\\
48.207981342001	106.14\\
48.224015198	106\\
48.240081179	105.85\\
48.256001507	105.7\\
48.272010781	105.56\\
48.288016637001	105.41\\
48.30396059	105.27\\
48.319986625	105.12\\
48.336100661999	104.98\\
48.352019195	104.83\\
48.368273505	104.68\\
48.384055138	104.53\\
48.399968224	104.39\\
48.415990006	104.25\\
48.432555071	104.1\\
48.448510382	103.94\\
48.464081473	103.81\\
48.480022636	103.66\\
48.495966970998	103.52\\
48.512061798	103.37\\
48.528028970998	103.22\\
48.544068623999	103.08\\
48.560003204	102.93\\
48.576007675999	102.79\\
48.591936213	102.64\\
48.607807244	102.5\\
48.626173687998	102.32\\
48.641592326999	102.18\\
48.656890684999	102.04\\
48.672132842	101.9\\
48.688007317999	101.76\\
48.703972451	101.62\\
48.720005942	101.48\\
48.735988916	101.33\\
48.752045888	101.18\\
48.768002536999	101.04\\
48.784075585	100.89\\
48.799983901	100.74\\
48.816028185	100.6\\
48.832478539	100.45\\
48.847992574999	100.3\\
48.864381966	100.16\\
48.879801822	100.01\\
48.898112525	99.84\\
48.913421465	99.7\\
48.928572636	99.57\\
48.94399812	99.43\\
48.959951414	99.28\\
48.975950875	99.14\\
48.992037109	98.99\\
49.009718074	98.99\\
49.024973698	99.12\\
49.040062977	99.25\\
49.05595748	99.38\\
49.072033223	99.52\\
49.088007881	99.66\\
49.104066943	99.8\\
49.119925449999	99.94\\
49.135990536	100.08\\
49.151936622	100.22\\
49.167895842	100.36\\
49.183960995	100.5\\
49.200012392	100.64\\
49.216014786	100.78\\
49.234549075	100.95\\
49.249709258999	101.08\\
49.265003239	101.21\\
49.280193593999	101.34\\
49.295938211	101.48\\
49.311801122	101.61\\
49.327994346	101.75\\
49.344086934999	101.89\\
49.360034614	102.03\\
49.375955737	102.17\\
49.392004076	102.31\\
49.408023031	102.45\\
49.424107701999	102.59\\
49.440048779	102.73\\
49.456682912	102.88\\
49.472037008	103.01\\
49.487990802	103.15\\
49.504053805	103.29\\
49.519991026	103.43\\
49.536007376	103.57\\
49.552022573	103.71\\
49.567943146	103.85\\
49.5839703	103.99\\
49.599973807	104.13\\
49.616072437998	104.27\\
49.632375772	104.41\\
49.647926607	104.55\\
49.664502047	104.68\\
49.68001585	104.83\\
49.695972243	104.96\\
49.711957276	105.1\\
49.727941714	105.24\\
49.744009511	105.38\\
49.760006243999	105.52\\
49.775972287	105.66\\
49.792063581	105.81\\
49.807982021	105.94\\
49.823960474999	106.08\\
49.839819758	106.22\\
49.855932241999	106.36\\
49.871946529	106.5\\
49.88806986	106.64\\
49.904032599	106.78\\
49.919986097	106.92\\
49.93595606	107.05\\
49.952041077	107.2\\
49.968026342001	107.34\\
49.983973147	107.48\\
49.999964012	107.61\\
50.017528342	107.62\\
50.032882214	107.48\\
50.048261364999	107.34\\
50.064350595	107.2\\
50.080035621	107.05\\
50.096101977999	106.91\\
50.111898195	106.76\\
50.128022059	106.62\\
50.144054395	106.47\\
50.159946152	106.32\\
50.176093454999	106.18\\
50.191988784999	106.03\\
50.207949503999	105.89\\
50.223909083	105.74\\
50.240075095	105.6\\
50.256062579	105.45\\
50.272031817999	105.3\\
50.288006392999	105.16\\
50.304083737	105.01\\
50.319989114999	104.86\\
50.336011739	104.72\\
50.351951887	104.57\\
50.367949381	104.43\\
50.384039722	104.28\\
50.399986946	104.14\\
50.418280918999	103.96\\
50.433394342	103.83\\
50.448882202	103.68\\
50.46401508	103.55\\
50.479991738	103.41\\
50.496069937	103.26\\
50.511858001999	103.12\\
50.527995744999	102.97\\
50.543987934999	102.83\\
50.559897543999	102.68\\
50.57595829	102.54\\
50.591957706	102.39\\
50.608009807	102.24\\
50.624005747	102.09\\
50.640042696	101.95\\
50.655996301	101.8\\
50.672063878	101.66\\
50.688011893	101.51\\
50.704104262999	101.36\\
50.720021461	101.22\\
50.73600453	101.07\\
50.752051154	100.93\\
50.768016451	100.78\\
50.784059624	100.64\\
50.799955917	100.49\\
50.816030699999	100.35\\
50.83246168	100.2\\
50.847946616	100.05\\
50.866440933	99.88\\
50.881762001999	99.74\\
50.897070069	99.6\\
50.91230052	99.46\\
50.928042849	99.32\\
50.944046890999	99.18\\
50.96001133	99.03\\
50.976103373	98.89\\
50.992041507	98.74\\
51.009782682999	98.73\\
51.024949307	98.86\\
51.040148209	99\\
51.055987914	99.13\\
51.072032001999	99.27\\
51.087968155	99.41\\
51.104055431	99.55\\
51.119976114999	99.69\\
51.135952453	99.82\\
51.151991956999	99.96\\
51.167956018	100.1\\
51.184028143	100.24\\
51.199980951	100.38\\
51.215998564	100.52\\
51.232190592	100.66\\
51.248166516001	100.8\\
51.264274475	100.94\\
51.279972048	101.08\\
51.296001436	101.22\\
51.311967683	101.36\\
51.328081093	101.5\\
51.344020139	101.64\\
51.359904388999	101.78\\
51.376005798	101.92\\
51.392072871	102.06\\
51.408054079	102.2\\
51.424012907	102.34\\
51.440075991	102.48\\
51.456061425999	102.62\\
51.472005102999	102.76\\
51.487835711	102.89\\
51.503988524999	103.03\\
51.519962777	103.17\\
51.536114102	103.31\\
51.552043939999	103.45\\
51.567992938	103.59\\
51.584030818	103.73\\
51.599912894	103.87\\
51.615925767	104.01\\
51.634573193	104.18\\
51.649865895	104.31\\
51.665046003999	104.44\\
51.680322352	104.58\\
51.695961998	104.71\\
51.712008711999	104.85\\
51.728002321001	104.99\\
51.744015444	105.13\\
51.759989698	105.27\\
51.776083143	105.41\\
51.791972611	105.55\\
51.807992308	105.68\\
51.824114654	105.82\\
51.840099027999	105.96\\
51.856000204999	106.1\\
51.872069431	106.25\\
51.887983915	106.39\\
51.904054237	106.52\\
51.919994534999	106.66\\
51.935977293999	106.8\\
51.951986022	106.94\\
51.968122623	107.08\\
51.983867209	107.22\\
51.999982754	107.36\\
52.017552551	107.3\\
52.032956387	107.16\\
52.048416293	107.02\\
52.064100224	106.88\\
52.079929725999	106.74\\
52.095974562998	106.59\\
52.11200481	106.45\\
52.128004477	106.3\\
52.143980956999	106.16\\
52.160079038	106.01\\
52.176048757	105.86\\
52.191826581999	105.72\\
52.208015272999	105.57\\
52.224031011	105.43\\
52.240063102	105.28\\
52.256020372	105.13\\
52.272138293999	104.98\\
52.287970371	104.84\\
52.304071923999	104.7\\
52.320025988	104.55\\
52.33592497	104.41\\
52.352040622	104.26\\
52.367970326999	104.12\\
52.384027203	103.97\\
52.400001448	103.82\\
52.416070468	103.68\\
52.432435634	103.53\\
52.448044769999	103.38\\
52.464250614999	103.24\\
52.48009042	103.09\\
52.496000928	102.95\\
52.511975977	102.8\\
52.527919848	102.66\\
52.543981595	102.51\\
52.560020713	102.36\\
52.576133436	102.22\\
52.592069920001	102.07\\
52.608041474	101.93\\
52.624077692999	101.77\\
52.639865543	101.64\\
52.655935855999	101.49\\
52.672003231	101.34\\
52.687982623	101.2\\
52.704112691	101.05\\
52.71994832	100.9\\
52.735998411	100.76\\
52.751975936	100.61\\
52.767937071	100.47\\
52.784012838	100.32\\
52.799984872	100.18\\
52.816003833	100.03\\
52.832261581	99.89\\
52.848081590999	99.74\\
52.86412842	99.59\\
52.880015987999	99.45\\
52.898463147999	99.27\\
52.913759205999	99.13\\
52.928920958999	98.99\\
52.944227908	98.86\\
52.95990136	98.72\\
52.976054864999	98.57\\
52.991991824999	98.43\\
53.009709382	98.42\\
53.024837123	98.55\\
53.040067947	98.68\\
53.055902913	98.81\\
53.072075535	98.95\\
53.088009629	99.09\\
53.10393721	99.23\\
53.119980192999	99.37\\
53.135991176	99.51\\
53.15201331	99.65\\
53.167991949	99.79\\
53.183961692999	99.93\\
53.200126619	100.07\\
53.216028885	100.21\\
53.23224063	100.35\\
53.247990087	100.49\\
53.264358514	100.63\\
53.280136508	100.78\\
53.296027249	100.91\\
53.311970468	101.05\\
53.328085677	101.19\\
53.344024958	101.33\\
53.360004895	101.47\\
53.376045947	101.6\\
53.391971345998	101.75\\
53.407977522999	101.88\\
53.424009817	102.02\\
53.439860329999	102.16\\
53.456015603	102.3\\
53.472043069	102.44\\
53.487943706	102.58\\
53.504128503	102.72\\
53.519993529998	102.86\\
53.535938564	103\\
53.552046085	103.14\\
53.568017494	103.28\\
53.583998901	103.42\\
53.600002641001	103.56\\
53.615986427999	103.7\\
53.634757708	103.87\\
53.650000765	104\\
53.665371642	104.13\\
53.680709334	104.27\\
53.696438288001	104.4\\
53.711882775	104.54\\
53.728003443	104.67\\
53.744092732999	104.81\\
53.759982048	104.95\\
53.775911677999	105.09\\
53.791980126999	105.23\\
53.807967546999	105.37\\
53.823979622	105.51\\
53.840069263	105.65\\
53.855982980999	105.79\\
53.872017612	105.93\\
53.888046656999	106.07\\
53.904060881	106.21\\
53.919955561	106.35\\
53.936023397999	106.49\\
53.954646435	106.66\\
53.969743833999	106.79\\
53.984916933	106.92\\
54.000117352999	107.05\\
54.017838437998	107.05\\
54.033527602	106.9\\
54.048922675001	106.76\\
54.064491243	106.62\\
54.080038074	106.48\\
54.096085179	106.34\\
54.111847331	106.19\\
54.127968771	106.05\\
54.143980326	105.9\\
54.159963631	105.76\\
54.17605896	105.61\\
54.192003612	105.46\\
54.208007875001	105.32\\
54.223982956999	105.17\\
54.239921168999	105.03\\
54.255870117	104.88\\
54.272070113999	104.74\\
54.287905041999	104.59\\
54.304074319999	104.44\\
54.320060901	104.3\\
54.335993607	104.15\\
54.352067503	104.01\\
54.367936878	103.86\\
54.384042718	103.72\\
54.399945625	103.57\\
54.416035727999	103.42\\
54.432251279	103.27\\
54.448036441	103.13\\
54.466482303999	102.96\\
54.481734607	102.82\\
54.496963251	102.68\\
54.512187561	102.54\\
54.527976274999	102.4\\
54.543969643	102.26\\
54.560032704	102.11\\
54.575998363	101.97\\
54.591922256	101.82\\
54.60800455	101.67\\
54.623982938	101.53\\
54.639857293	101.38\\
54.655961732	101.24\\
54.67204788	101.09\\
54.687820551	100.94\\
54.703993336999	100.8\\
54.720991387999	100.64\\
54.736192164	100.5\\
54.752030862	100.36\\
54.768024144999	100.21\\
54.784022347999	100.07\\
54.799953861999	99.92\\
54.815995869999	99.78\\
54.832542717	99.63\\
54.847998482	99.48\\
54.864305005001	99.34\\
54.879821498	99.19\\
54.896044977	99.05\\
54.911919801	98.9\\
54.927918517	98.76\\
54.944077765	98.61\\
54.960015632	98.46\\
54.975988579	98.32\\
54.992033319	98.17\\
55.00975089	98.16\\
55.024986716	98.3\\
55.040324071999	98.43\\
55.055948796	98.56\\
55.072036616	98.7\\
55.088021364999	98.84\\
55.10402905	98.98\\
55.120177595	99.12\\
55.13598456	99.26\\
55.152032337	99.4\\
55.167989839	99.54\\
55.184185084	99.68\\
55.200049286	99.82\\
55.215952579	99.95\\
55.234520183	100.12\\
55.249771258	100.26\\
55.265088613	100.39\\
55.280431782	100.52\\
55.296125472999	100.66\\
55.311921741999	100.8\\
55.327990165	100.93\\
55.343986053999	101.07\\
55.359911347	101.21\\
55.375962703	101.35\\
55.391981097	101.49\\
55.407978163	101.63\\
55.4240195	101.77\\
55.439939286	101.91\\
55.455975667	102.05\\
55.472075621	102.19\\
55.487825553	102.33\\
55.504009934	102.47\\
55.520067491999	102.61\\
55.535988552	102.75\\
55.552031230999	102.88\\
55.567977732999	103.03\\
55.58399602	103.16\\
55.599978742	103.3\\
55.615991798999	103.44\\
55.634419470998	103.61\\
55.649518637	103.74\\
55.664700475	103.87\\
55.680014498	104.01\\
55.696080640999	104.14\\
55.711991617	104.28\\
55.727991383	104.42\\
55.744034553	104.56\\
55.760031001999	104.7\\
55.775982791	104.84\\
55.791988247	104.98\\
55.808078239999	105.12\\
55.824054912	105.26\\
55.840773992	105.41\\
55.856233684001	105.55\\
55.872046048	105.68\\
55.888010553999	105.82\\
55.904019858001	105.96\\
55.919969637	106.09\\
55.936025091	106.23\\
55.952032782	106.37\\
55.96803493	106.51\\
55.98399584	106.65\\
55.999891052	106.79\\
56.017346161999	106.81\\
56.032591414	106.7\\
56.048021208999	106.59\\
56.064042795999	106.48\\
56.0799838	106.37\\
56.096001589	106.26\\
56.112015017	106.14\\
56.128016413	106.03\\
56.144069126	105.91\\
56.160112116	105.8\\
56.175983402999	105.68\\
56.192031980999	105.57\\
56.207932129	105.46\\
56.223977649001	105.34\\
56.24006715	105.23\\
56.256026877	105.11\\
56.272014181	105\\
56.288102379	104.89\\
56.304040868	104.77\\
56.320015647999	104.66\\
56.336034845	104.54\\
56.352046525	104.43\\
56.368031123	104.31\\
56.384009104999	104.2\\
56.399967659999	104.09\\
56.415947377	103.97\\
56.434440092	103.84\\
56.448064484	103.74\\
56.464000923	103.63\\
56.480009875	103.51\\
56.496015824	103.4\\
56.512017415	103.29\\
56.527999482	103.17\\
56.543980522	103.06\\
56.559935686	102.94\\
56.575967689	102.83\\
56.592039017	102.72\\
56.607939644	102.6\\
56.623983269	102.49\\
56.640131658	102.37\\
56.656020221	102.26\\
56.672049222	102.14\\
56.68804356	102.03\\
56.70385562	101.92\\
56.722091817	101.78\\
56.737175247	101.68\\
56.752421812	101.57\\
56.767993224	101.46\\
56.784033714	101.35\\
56.800074031	101.23\\
56.816091036	101.12\\
56.831967824999	101\\
56.848026885	100.89\\
56.864198031999	100.78\\
56.880035537	100.66\\
56.896075569999	100.55\\
56.912083556	100.43\\
56.928062336	100.32\\
56.944239665999	100.2\\
56.960009192	100.09\\
56.975981487999	99.98\\
56.992038657	99.86\\
57.009727216	99.85\\
57.025394211	99.96\\
57.041703058	100.07\\
57.056996034	100.17\\
57.072278428999	100.28\\
57.08799107	100.38\\
57.103965395	100.49\\
57.119977118	100.59\\
57.135919156	100.7\\
57.152060619999	100.81\\
57.167931447	100.92\\
57.184028288	101.02\\
57.199977657	101.13\\
57.216018673	101.24\\
57.232502984	101.35\\
57.248016218	101.46\\
57.264104211	101.56\\
57.280013906	101.67\\
57.295973891001	101.78\\
57.311975902	101.89\\
57.328001941	101.99\\
57.343848996	102.1\\
57.360009089	102.21\\
57.375994859999	102.32\\
57.392017543999	102.43\\
57.408032315	102.53\\
57.424012062	102.64\\
57.440048684001	102.75\\
57.456021982	102.86\\
57.471941589	102.97\\
57.488021295999	103.07\\
57.504065324	103.18\\
57.520063453	103.29\\
57.535982866999	103.4\\
57.552018552	103.5\\
57.568022829999	103.61\\
57.584108728	103.72\\
57.599951299001	103.83\\
57.615969583	103.93\\
57.634452354999	104.06\\
57.649649869	104.17\\
57.664995017	104.27\\
57.680345367	104.37\\
57.696093189	104.48\\
57.711936329999	104.59\\
57.727996802999	104.69\\
57.744021149999	104.8\\
57.760000750001	104.91\\
57.775937571	105.01\\
57.792014147999	105.12\\
57.808040951999	105.23\\
57.82392863	105.34\\
57.839937128	105.44\\
57.855924112	105.55\\
57.872030055	105.66\\
57.887994913001	105.77\\
57.90409342	105.88\\
57.919998116	105.98\\
57.936014411	106.09\\
57.952069805	106.2\\
57.968108451	106.31\\
57.984233697	106.42\\
57.999981509	106.52\\
58.017497052999	106.52\\
58.032737458999	106.41\\
58.048019434	106.31\\
58.066531036	106.17\\
58.081717695	106.07\\
58.096964541	105.96\\
58.112375048	105.85\\
58.127990718999	105.74\\
58.144020007	105.63\\
58.159998111999	105.51\\
58.17602848	105.4\\
58.191916222999	105.28\\
58.208029723	105.17\\
58.224024449999	105.05\\
58.239961698999	104.94\\
58.256000661999	104.83\\
58.272050833	104.71\\
58.287930449	104.6\\
58.304462496	104.48\\
58.320067791	104.37\\
58.336002446	104.26\\
58.351937098	104.14\\
58.368007754	104.03\\
58.384038136	103.91\\
58.399920495	103.8\\
58.415998727999	103.69\\
58.432074201999	103.57\\
58.447936178	103.46\\
58.466423314	103.32\\
58.481592173	103.21\\
58.496725605999	103.1\\
58.511854081	103\\
58.527986185999	102.89\\
58.543996692999	102.77\\
58.560042212	102.66\\
58.5759267	102.54\\
58.592046319999	102.43\\
58.607982788999	102.32\\
58.624001854999	102.2\\
58.639873369	102.09\\
58.658035536999	101.95\\
58.673406604	101.84\\
58.688464767	101.74\\
58.703966595	101.63\\
58.720009402	101.52\\
58.735927112	101.4\\
58.751965227	101.29\\
58.76790164	101.18\\
58.784003695	101.06\\
58.79992559	100.95\\
58.815892708999	100.83\\
58.834511109	100.69\\
58.849786423999	100.59\\
58.864850593	100.48\\
58.8801921	100.37\\
58.895983928999	100.26\\
58.911802373	100.15\\
58.927988217	100.03\\
58.94408101	99.92\\
58.959914524	99.8\\
58.976004422999	99.69\\
58.992018362999	99.58\\
59.009731645	99.57\\
59.024877737999	99.67\\
59.040047739	99.77\\
59.056219201	99.87\\
59.072061617	99.98\\
59.087947220998	100.09\\
59.104024277999	100.2\\
59.119868517	100.31\\
59.136048845	100.41\\
59.152052963	100.52\\
59.167962962	100.63\\
59.183951347	100.74\\
59.200000265999	100.84\\
59.216091803	100.95\\
59.234513786999	101.08\\
59.249644538999	101.19\\
59.26483855	101.29\\
59.279978708999	101.38\\
59.296012759	101.49\\
59.312005716	101.6\\
59.327994438	101.71\\
59.344019277999	101.81\\
59.360004788	101.92\\
59.375953625	102.03\\
59.392103729	102.14\\
59.407970455	102.25\\
59.423998937	102.36\\
59.440023275	102.46\\
59.455994914	102.57\\
59.471999704999	102.68\\
59.487950763999	102.79\\
59.504068626	102.89\\
59.519970496	103\\
59.535986676	103.11\\
59.552749142	103.23\\
59.567986156	103.33\\
59.583958836	103.43\\
59.599996706	103.54\\
59.615992154	103.65\\
59.634353981	103.78\\
59.649451714	103.88\\
59.664579174001	103.98\\
59.679923591	104.08\\
59.696009769	104.19\\
59.712025217	104.3\\
59.728029071001	104.4\\
59.743992479001	104.51\\
59.760015349	104.62\\
59.775916482	104.73\\
59.792000562	104.83\\
59.808017046999	104.94\\
59.823979716	105.05\\
59.840033165	105.16\\
59.855992144	105.27\\
59.871962494999	105.37\\
59.887974836999	105.48\\
59.903881918	105.59\\
59.919974763	105.7\\
59.936011374	105.81\\
59.952044829	105.91\\
59.968010233001	106.02\\
59.98407844	106.13\\
59.999998346	106.24\\
60.01771446	106.24\\
60.032951574999	106.13\\
60.048287980999	106.02\\
60.06450951	105.91\\
60.080008147999	105.8\\
60.095975964	105.69\\
60.111976303	105.57\\
60.128002354999	105.46\\
60.144135026999	105.34\\
60.160383079	105.22\\
60.176057789	105.11\\
60.192010587	105\\
60.20823563	104.89\\
60.223969374	104.77\\
60.239978904	104.66\\
60.256026504	104.54\\
60.272036921	104.43\\
60.287970374	104.32\\
60.304031495	104.2\\
60.319998066999	104.09\\
60.335961057001	103.97\\
60.351978385	103.86\\
60.367989998	103.75\\
60.384101005	103.63\\
60.399989519	103.52\\
60.415963974	103.4\\
60.434441393	103.27\\
60.449630923	103.16\\
60.464788222	103.05\\
60.480131671	102.94\\
60.496093767999	102.83\\
60.512054427999	102.72\\
60.528012385	102.6\\
60.544105289	102.49\\
60.559991318	102.37\\
60.576042753999	102.26\\
60.592004029	102.15\\
60.608006477	102.03\\
60.624007487	101.92\\
60.640029545	101.8\\
60.655926639	101.69\\
60.672027687	101.58\\
60.688077878999	101.46\\
60.70444744	101.34\\
60.720007509	101.23\\
60.736124918	101.12\\
60.751967465999	101\\
60.7680868	100.89\\
60.784249310999	100.78\\
60.800080402	100.66\\
60.815970044	100.55\\
60.832438152999	100.43\\
60.848022328999	100.32\\
60.866257614	100.18\\
60.881428599999	100.08\\
60.896697363	99.97\\
60.911967803	99.86\\
60.928077915	99.75\\
60.944028091	99.63\\
60.960007096	99.52\\
60.975999536	99.41\\
60.992074059001	99.29\\
61.009531827	99.28\\
61.024652261	99.39\\
61.040051838	99.49\\
61.055932827	99.59\\
61.072029964	99.7\\
61.088010378	99.81\\
61.104058625	99.92\\
61.120032284999	100.02\\
61.136023893	100.13\\
61.152020298	100.24\\
61.168034984	100.35\\
61.18400054	100.46\\
61.200032104001	100.56\\
61.216042215	100.67\\
61.232721442999	100.78\\
61.248025804	100.89\\
61.264058564	100.99\\
61.280041269	101.1\\
61.296094559	101.21\\
61.311825372	101.32\\
61.328001533999	101.42\\
61.344054999	101.53\\
61.359922152	101.64\\
61.376014612999	101.75\\
61.391994132	101.86\\
61.407947399	101.96\\
61.423968081999	102.07\\
61.440032561	102.18\\
61.455904504	102.29\\
61.471952165	102.4\\
61.488033107	102.5\\
61.504064720998	102.61\\
61.519923519	102.72\\
61.536036307999	102.83\\
61.552002553999	102.94\\
61.567956099	103.04\\
61.584021272999	103.15\\
61.599948019999	103.26\\
61.616027444	103.37\\
61.632400026999	103.47\\
61.648036895999	103.58\\
61.666551008999	103.71\\
61.681805524	103.81\\
61.696970216	103.92\\
61.712826581	104.02\\
61.727826448	104.11\\
61.7439848	104.23\\
61.760039255	104.34\\
61.775916039	104.44\\
61.791988122	104.55\\
61.808050121	104.66\\
61.823970474	104.77\\
61.839911062998	104.88\\
61.855992996	104.98\\
61.872018126	105.09\\
61.888008387	105.2\\
61.904037541	105.31\\
61.919931678	105.41\\
61.935997477	105.52\\
61.952026079999	105.63\\
61.967985626	105.74\\
61.984012335001	105.84\\
62.00003473	105.95\\
62.017565875001	105.97\\
62.032802607	105.89\\
62.048037563	105.82\\
62.066688477	105.72\\
62.08183712	105.64\\
62.096819581	105.56\\
62.112139232	105.48\\
62.127994541999	105.41\\
62.146208258999	105.31\\
62.16151273	105.23\\
62.176761704999	105.15\\
62.192178692	105.07\\
62.207985505	104.99\\
62.224249828999	104.91\\
62.240071005	104.83\\
62.255983826999	104.75\\
62.27203431	104.67\\
62.287996923	104.58\\
62.303997986001	104.5\\
62.319948123999	104.42\\
62.336043402999	104.34\\
62.352098004	104.25\\
62.368000942999	104.17\\
62.384087056	104.09\\
62.400011974	104\\
62.416007848	103.92\\
62.432270605999	103.84\\
62.448014954	103.76\\
62.464030123999	103.67\\
62.479999506	103.59\\
62.496032730999	103.51\\
62.51198522	103.43\\
62.528007177999	103.35\\
62.543954267	103.26\\
62.559957301	103.18\\
62.576014919	103.1\\
62.591954475999	103.01\\
62.607912333	102.93\\
62.62396901	102.85\\
62.640052046	102.77\\
62.656007313	102.69\\
62.672067336	102.6\\
62.687945326	102.52\\
62.704017529998	102.44\\
62.719869748	102.36\\
62.736006366	102.27\\
62.752078452	102.19\\
62.767975066	102.11\\
62.783994937	102.03\\
62.799978920999	101.94\\
62.815973697	101.86\\
62.834341229001	101.76\\
62.849478425	101.69\\
62.865078696	101.61\\
62.880368289	101.53\\
62.89596334	101.45\\
62.911870684	101.37\\
62.927981732999	101.29\\
62.94396241	101.2\\
62.960124116	101.12\\
62.97735083	101.03\\
62.992663283	100.95\\
63.007796978	100.93\\
63.02610359	101.02\\
63.041452513	101.09\\
63.057003243001	101.17\\
63.072349751	101.24\\
63.088076743	101.31\\
63.104058048	101.39\\
63.120034487	101.46\\
63.136001772	101.54\\
63.152041725999	101.62\\
63.168045067	101.69\\
63.184063321	101.77\\
63.199973515	101.84\\
63.215989685	101.92\\
63.232273827	102\\
63.247951685	102.07\\
63.2641707	102.15\\
63.280020971	102.23\\
63.295994373	102.3\\
63.311977411	102.38\\
63.328018762	102.45\\
63.343976535	102.53\\
63.360223624	102.61\\
63.375976293999	102.68\\
63.391989097	102.76\\
63.408039675	102.83\\
63.424023562	102.91\\
63.440121262001	102.99\\
63.457874317999	103.07\\
63.473095571	103.15\\
63.489042239	103.22\\
63.504475046	103.3\\
63.51986338	103.37\\
63.536048645	103.44\\
63.552066633999	103.52\\
63.567992563	103.6\\
63.583974137	103.67\\
63.599904083999	103.75\\
63.616078763999	103.82\\
63.634458196	103.91\\
63.649567154	103.99\\
63.66466864	104.06\\
63.680061938	104.13\\
63.696051621	104.2\\
63.713199642999	104.29\\
63.72827292	104.36\\
63.746382698999	104.45\\
63.761630887	104.52\\
63.776885354	104.59\\
63.792349878	104.67\\
63.808069921	104.74\\
63.823973435	104.81\\
63.839970175999	104.89\\
63.855999455	104.96\\
63.872021318	105.04\\
63.888007269	105.12\\
63.904013291	105.19\\
63.919902475999	105.27\\
63.936022959	105.35\\
63.951939316	105.42\\
63.968016716	105.5\\
63.984034352	105.57\\
63.999920277999	105.65\\
};
\addplot [color=mycolor1,solid,forget plot]
  table[row sep=crcr]{%
63.999920277999	105.65\\
64.017641917	105.65\\
64.032924146	105.57\\
64.048097204	105.49\\
64.064352923	105.41\\
64.079964673999	105.33\\
64.095983925	105.25\\
64.111998407	105.17\\
64.127964493	105.08\\
64.144051764	105\\
64.160027873	104.92\\
64.175980307	104.84\\
64.192015266001	104.76\\
64.208062671	104.67\\
64.223977597	104.59\\
64.240032173	104.51\\
64.256002644001	104.43\\
64.27202087	104.34\\
64.288817794	104.25\\
64.304037682	104.17\\
64.320005337999	104.09\\
64.336013388999	104.01\\
64.352070666999	103.93\\
64.369043525	103.84\\
64.384301581999	103.76\\
64.400014529	103.68\\
64.416054458	103.6\\
64.434505247	103.5\\
64.449757039	103.42\\
64.465488304	103.34\\
64.480855828999	103.26\\
64.496118613	103.18\\
64.512002833999	103.1\\
64.52808147	103.02\\
64.544078163001	102.94\\
64.560089399	102.86\\
64.575880208	102.78\\
64.591997086	102.69\\
64.607824244	102.61\\
64.624004383999	102.53\\
64.6400408	102.45\\
64.655919593	102.36\\
64.672046619	102.28\\
64.68802576	102.2\\
64.703936013999	102.12\\
64.720066779	102.03\\
64.735997069	101.95\\
64.752002138	101.87\\
64.768015878	101.79\\
64.783933762	101.7\\
64.800100209	101.62\\
64.816068162	101.54\\
64.832331095	101.46\\
64.847955246	101.37\\
64.866380046	101.28\\
64.881590855	101.2\\
64.896698466	101.12\\
64.911985	101.04\\
64.928009739	100.96\\
64.943937989	100.88\\
64.960019961999	100.8\\
64.97585293	100.72\\
64.994260399	100.62\\
65.008665747	100.6\\
65.023951022	100.67\\
65.039941446	100.74\\
65.055930843	100.82\\
65.072075034999	100.9\\
65.087913524999	100.97\\
65.104069904998	101.05\\
65.119950815	101.12\\
65.136029864999	101.2\\
65.152025107	101.28\\
65.167974972	101.35\\
65.184052937	101.43\\
65.199901261	101.5\\
65.218091013	101.59\\
65.233172113	101.67\\
65.248268012	101.74\\
65.264362494	101.81\\
65.279886239	101.89\\
65.295961135	101.96\\
65.312011406999	102.04\\
65.327996977	102.11\\
65.343973331999	102.19\\
65.360001208001	102.26\\
65.376008482999	102.34\\
65.39201896	102.42\\
65.407975845	102.49\\
65.424061993	102.57\\
65.440091208	102.65\\
65.455867478	102.72\\
65.472054211	102.8\\
65.487850204	102.87\\
65.504052133	102.95\\
65.519973651	103.03\\
65.535994357	103.1\\
65.552020659	103.18\\
65.567967462	103.25\\
65.583971885999	103.33\\
65.599949848	103.41\\
65.616180187998	103.48\\
65.632439194	103.56\\
65.647963743999	103.64\\
65.666560352999	103.73\\
65.681645737999	103.8\\
65.696735411999	103.87\\
65.712071439999	103.94\\
65.728013201	104.02\\
65.743983608001	104.09\\
65.760006576	104.17\\
65.775998961999	104.24\\
65.791999287	104.32\\
65.808071500001	104.4\\
65.823984364	104.47\\
65.840021426	104.55\\
65.855992522	104.62\\
65.87215239	104.7\\
65.887978534	104.78\\
65.904046104	104.85\\
65.919900005	104.93\\
65.935987067	105\\
65.952032012	105.08\\
65.967957329999	105.16\\
65.98405486	105.23\\
65.999984584	105.31\\
66.015978788	105.29\\
66.032171191	105.21\\
66.047934713	105.12\\
66.064276214	105.04\\
66.079928256	104.96\\
66.096021616	104.88\\
66.111804696	104.79\\
66.127943194	104.71\\
66.144022622	104.63\\
66.160038220998	104.54\\
66.176266191	104.46\\
66.19203188	104.38\\
66.208005446	104.3\\
66.223985419	104.22\\
66.240086387	104.13\\
66.255978418999	104.05\\
66.272041956	103.97\\
66.288000194	103.89\\
66.30406102	103.8\\
66.319993225	103.72\\
66.336014175	103.64\\
66.352075121	103.56\\
66.368003112999	103.47\\
66.384092968	103.39\\
66.400001333001	103.31\\
66.415988708	103.23\\
66.434515539	103.13\\
66.449757991	103.05\\
66.464820602999	102.97\\
66.480247954999	102.89\\
66.496107633	102.81\\
66.512031979999	102.73\\
66.527983772999	102.65\\
66.544055836	102.57\\
66.560032755	102.48\\
66.575984366	102.4\\
66.592014881	102.32\\
66.60796246	102.24\\
66.623996494999	102.16\\
66.639888458999	102.07\\
66.655967159999	101.99\\
66.672080151	101.91\\
66.688023622	101.82\\
66.704038386	101.74\\
66.719988699	101.66\\
66.736041735	101.58\\
66.752008507	101.49\\
66.767974379	101.41\\
66.783949246	101.33\\
66.799933657	101.25\\
66.81600419	101.17\\
66.834566845	101.07\\
66.849818678999	100.99\\
66.865160346	100.91\\
66.880486885999	100.83\\
66.895939081	100.75\\
66.911986965	100.67\\
66.927946284999	100.59\\
66.943965803	100.51\\
66.959877500001	100.42\\
66.975937443	100.34\\
66.991984399	100.26\\
67.009621869	100.25\\
67.024729369	100.33\\
67.040140326999	100.4\\
67.055987165	100.47\\
67.07201364	100.55\\
67.088004277	100.62\\
67.104060805	100.7\\
67.120076706999	100.78\\
67.136031589	100.85\\
67.151976581999	100.93\\
67.167990556	101\\
67.183936477	101.08\\
67.200020199999	101.15\\
67.215992731	101.23\\
67.234396944	101.32\\
67.249709258999	101.4\\
67.264988675	101.47\\
67.280162571	101.54\\
67.296017548	101.61\\
67.311866197	101.69\\
67.328029345	101.76\\
67.343938060999	101.84\\
67.359981623	101.92\\
67.376099263	101.99\\
67.391996563	102.07\\
67.408016230999	102.15\\
67.424016229	102.22\\
67.440140546	102.3\\
67.45740328	102.38\\
67.47257052	102.46\\
67.488115250999	102.53\\
67.503986982	102.6\\
67.519986303999	102.68\\
67.535996416	102.75\\
67.552132682	102.83\\
67.567932152999	102.91\\
67.584055145	102.98\\
67.599934418	103.06\\
67.615989685999	103.13\\
67.632338243	103.21\\
67.647954699999	103.29\\
67.664241671	103.36\\
67.679985213	103.44\\
67.695937420001	103.51\\
67.711951132	103.59\\
67.728052951	103.67\\
67.743936441999	103.74\\
67.760040704	103.82\\
67.778415145	103.91\\
67.794001964	103.98\\
67.809114064	104.06\\
67.824387507999	104.13\\
67.840068141001	104.2\\
67.856006354	104.28\\
67.871832498	104.35\\
67.887822640999	104.43\\
67.904015248999	104.5\\
67.920015772999	104.58\\
67.935838897999	104.66\\
67.952025236	104.73\\
67.967852713	104.81\\
67.984025152	104.88\\
68.000044088	104.96\\
68.017682962	104.98\\
68.03308351	104.94\\
68.048264958	104.9\\
68.063970569	104.85\\
68.079904479	104.81\\
68.095938953	104.77\\
68.111916419	104.72\\
68.127998005999	104.68\\
68.144003428999	104.63\\
68.160012286	104.59\\
68.176018402999	104.54\\
68.192016812998	104.5\\
68.208259666	104.46\\
68.224019104	104.41\\
68.240036776999	104.37\\
68.256241053	104.32\\
68.272040848	104.28\\
68.287952016001	104.23\\
68.303994378999	104.19\\
68.319987263	104.15\\
68.335985585	104.1\\
68.352027854	104.06\\
68.367951983001	104.01\\
68.383973203	103.97\\
68.399962217001	103.92\\
68.415993448	103.88\\
68.432339786	103.83\\
68.448029977	103.79\\
68.46410497	103.75\\
68.480002335	103.7\\
68.496017712999	103.66\\
68.511801107	103.61\\
68.527996587	103.57\\
68.544058925999	103.52\\
68.559964048999	103.48\\
68.576008433	103.44\\
68.592026198	103.39\\
68.607993739999	103.35\\
68.6240022	103.3\\
68.640116182	103.26\\
68.656020086	103.21\\
68.671939699999	103.17\\
68.688062749	103.12\\
68.703996264999	103.08\\
68.720036864999	103.04\\
68.736017614	102.99\\
68.752012093999	102.95\\
68.767989121	102.9\\
68.784111466	102.86\\
68.800037455999	102.81\\
68.816027206	102.77\\
68.832400655	102.72\\
68.848035531	102.68\\
68.864168074999	102.64\\
68.879921702	102.59\\
68.896074774999	102.55\\
68.912604253	102.5\\
68.928040503	102.46\\
68.944124437	102.41\\
68.96004731	102.37\\
68.976059786999	102.33\\
68.991997281999	102.28\\
69.009645451999	102.26\\
69.024896206	102.27\\
69.040124259	102.29\\
69.055967998	102.3\\
69.072018015999	102.31\\
69.087973948	102.32\\
69.104054699	102.34\\
69.120047175	102.35\\
69.136098104	102.36\\
69.152019489	102.37\\
69.167864873999	102.39\\
69.184066747	102.4\\
69.200136343	102.41\\
69.216052656	102.42\\
69.234518129	102.44\\
69.249780279	102.45\\
69.264939666	102.46\\
69.28014375	102.48\\
69.29597569	102.49\\
69.311920289	102.5\\
69.327948515999	102.51\\
69.344062623	102.53\\
69.360005211	102.54\\
69.375979941	102.55\\
69.393344644001	102.57\\
69.40846132	102.58\\
69.424139543	102.59\\
69.440028835	102.6\\
69.455904020999	102.61\\
69.472069206	102.63\\
69.487998606	102.64\\
69.504008666	102.65\\
69.519922016001	102.67\\
69.535958704999	102.68\\
69.552067536999	102.69\\
69.567985417	102.7\\
69.584039356999	102.72\\
69.599918395999	102.73\\
69.615951086999	102.74\\
69.632182676	102.75\\
69.647937706	102.77\\
69.666390878	102.78\\
69.681779812	102.79\\
69.697104478	102.81\\
69.712371094	102.82\\
69.727961571	102.83\\
69.744027104001	102.84\\
69.759987052	102.86\\
69.776023217	102.87\\
69.791983755	102.88\\
69.807991069999	102.89\\
69.8239806	102.91\\
69.839883828	102.92\\
69.855824491	102.93\\
69.872053654	102.94\\
69.887976894001	102.96\\
69.903940393	102.97\\
69.919971663	102.98\\
69.935981333	102.99\\
69.951968902	103.01\\
69.967997711	103.02\\
69.984037843	103.03\\
69.999931002	103.05\\
70.017350382	103.05\\
70.033348676	103.03\\
70.048502265	103.02\\
70.064156281	103.01\\
70.079979119999	103\\
70.096017019	102.98\\
70.111984789	102.97\\
70.128035352	102.96\\
70.144036196999	102.95\\
70.160033589	102.93\\
70.176098097999	102.92\\
70.191939784	102.91\\
70.207982437998	102.9\\
70.223998817999	102.88\\
70.240473321001	102.87\\
70.256246013999	102.86\\
70.272064196	102.84\\
70.287877472999	102.83\\
70.304030625001	102.82\\
70.319978335	102.81\\
70.335987277999	102.79\\
70.351918549001	102.78\\
70.368031112999	102.77\\
70.384119242	102.76\\
70.399970678	102.74\\
70.415992024	102.73\\
70.432003812998	102.72\\
70.447994171999	102.71\\
70.464197459	102.69\\
70.480026293999	102.68\\
70.496149967	102.67\\
70.512128451	102.65\\
70.527989785	102.64\\
70.544124173999	102.63\\
70.561282264	102.62\\
70.576490465	102.6\\
70.591940725999	102.59\\
70.607923249	102.58\\
70.623997326	102.57\\
70.640025959	102.55\\
70.655992706999	102.54\\
70.671890999	102.53\\
70.687966418	102.52\\
70.704035157	102.5\\
70.71994276	102.49\\
70.736024704999	102.48\\
70.752037069	102.46\\
70.767978847	102.45\\
70.784047987	102.44\\
70.79996205	102.43\\
70.815989421999	102.41\\
70.834534373	102.4\\
70.849769562	102.39\\
70.865172192999	102.37\\
70.880382909	102.36\\
70.896011079	102.35\\
70.911838968999	102.34\\
70.927996487999	102.33\\
70.944006874	102.31\\
70.959899656	102.3\\
70.975919405	102.29\\
70.991961741001	102.27\\
71.009794043	102.27\\
71.024878927	102.29\\
71.040458427	102.3\\
71.055782901001	102.31\\
71.07204029	102.32\\
71.08805234	102.34\\
71.104241001	102.35\\
71.119974019001	102.36\\
71.136012609	102.37\\
71.151959126	102.39\\
71.167996897	102.4\\
71.184035159	102.41\\
71.20001956	102.42\\
71.215991747	102.44\\
71.234399638999	102.45\\
71.249547385	102.46\\
71.264675843	102.48\\
71.279981921999	102.49\\
71.296062242	102.5\\
71.31198191	102.51\\
71.327939934	102.53\\
71.343999782	102.54\\
71.359957397	102.55\\
71.375986286	102.56\\
71.392000951999	102.58\\
71.407968046	102.59\\
71.424004326	102.6\\
71.440086331	102.61\\
71.455998141001	102.63\\
71.472032038	102.64\\
71.487985511	102.65\\
71.503995217	102.67\\
71.51993782	102.68\\
71.53590659	102.69\\
71.552038052	102.7\\
71.567985295	102.72\\
71.584010561	102.73\\
71.59988804	102.74\\
71.615987399	102.75\\
71.634913276	102.77\\
71.647973815	102.78\\
71.66398984	102.79\\
71.680100161999	102.8\\
71.696071178999	102.82\\
71.711917595	102.83\\
71.727960101	102.84\\
71.743926956	102.86\\
71.759898801	102.87\\
71.776102739	102.88\\
71.792019982	102.89\\
71.808017112	102.91\\
71.824017236999	102.92\\
71.840076258999	102.93\\
71.856012427	102.94\\
71.872040869999	102.96\\
71.887981386	102.97\\
71.903904943	102.98\\
71.919857456	102.99\\
71.936006965	103.01\\
71.952018144	103.02\\
71.96806309	103.03\\
71.984121038	103.05\\
72.000076104	103.06\\
72.016084476	103.06\\
72.032367894	103.04\\
72.048012284	103.03\\
72.06409068	103.02\\
72.079967608001	103\\
72.096096564	102.99\\
72.111937251	102.98\\
72.128012033999	102.97\\
72.144012503	102.95\\
72.160059698	102.94\\
72.176017472	102.93\\
72.191963704	102.92\\
72.207993738	102.9\\
72.223970328	102.89\\
72.23990602	102.88\\
72.255867681	102.87\\
72.272014487	102.85\\
72.287931824999	102.84\\
72.303949203999	102.83\\
72.319998374	102.81\\
72.33600606	102.8\\
72.352035896	102.79\\
72.368013348	102.78\\
72.384052534999	102.76\\
72.400029633	102.75\\
72.415989833	102.74\\
72.434429824999	102.72\\
72.449684343999	102.71\\
72.464851335999	102.7\\
72.480109913	102.69\\
72.496079282	102.68\\
72.512016362	102.66\\
72.528028463	102.65\\
72.544051277	102.64\\
72.560062952	102.62\\
72.576072781999	102.61\\
72.591970590001	102.6\\
72.608005581	102.59\\
72.62400802	102.57\\
72.639973965001	102.56\\
72.656019166	102.55\\
72.67206563	102.54\\
72.688030593	102.52\\
72.704027569	102.51\\
72.720011911999	102.5\\
72.735999723	102.49\\
72.752051032	102.47\\
72.767951319	102.46\\
72.783945387999	102.45\\
72.800014734	102.43\\
72.816020811	102.42\\
72.832510246	102.41\\
72.848014208	102.4\\
72.866586602	102.38\\
72.881902674001	102.37\\
72.897489781	102.36\\
72.912916133999	102.34\\
72.928224107999	102.33\\
72.944157654998	102.32\\
72.959910071	102.31\\
72.975969208	102.29\\
72.992009784999	102.28\\
73.009460158	102.28\\
73.024718067	102.29\\
73.040108117	102.31\\
73.056067703	102.32\\
73.072044974999	102.33\\
73.087975801	102.34\\
73.104062871	102.36\\
73.120000288001	102.37\\
73.136009562	102.38\\
73.152171757	102.39\\
73.168026645999	102.41\\
73.184085664	102.42\\
73.199926256	102.43\\
73.215985161999	102.44\\
73.234529644	102.46\\
73.249861461999	102.47\\
73.265082413001	102.48\\
73.280498027	102.5\\
73.296102402999	102.51\\
73.311992838	102.52\\
73.328005975	102.53\\
73.344237512999	102.55\\
73.359990476	102.56\\
73.37612606	102.57\\
73.392076277	102.58\\
73.407933677	102.6\\
73.423931449999	102.61\\
73.439921054	102.62\\
73.456011758	102.63\\
73.472023579	102.65\\
73.487995358001	102.66\\
73.503978291	102.67\\
73.519980017	102.69\\
73.535992361	102.7\\
73.551985366	102.71\\
73.5680184	102.72\\
73.584053755999	102.74\\
73.599967515999	102.75\\
73.616125938	102.76\\
73.634505159999	102.78\\
73.649790562	102.79\\
73.665019217	102.8\\
73.680222323	102.81\\
73.696071043999	102.83\\
73.711992654	102.84\\
73.728193076	102.85\\
73.743834066	102.86\\
73.761193661	102.88\\
73.77640441	102.89\\
73.792023274	102.9\\
73.808023641001	102.91\\
73.82396143	102.93\\
73.840041152999	102.94\\
73.855974560999	102.95\\
73.871939999	102.96\\
73.887872095	102.98\\
73.904067411	102.99\\
73.920019406999	103\\
73.936042447	103.02\\
73.952064478	103.03\\
73.967930513	103.04\\
73.984091991999	103.05\\
74.000028591	103.07\\
74.017536362	103.07\\
74.032880562998	103.05\\
74.048065161	103.04\\
74.064174354	103.03\\
74.079941894001	103.02\\
74.095996928	103\\
74.111931593	102.99\\
74.127946489	102.98\\
74.144035299001	102.97\\
74.15995885	102.95\\
74.176002239999	102.94\\
74.192119288	102.93\\
74.207969277	102.92\\
74.224267847	102.9\\
74.240055673	102.89\\
74.256088385999	102.88\\
74.271949883999	102.87\\
74.287983715	102.85\\
74.303986684	102.84\\
74.320013531	102.83\\
74.336003497	102.81\\
74.351949544	102.8\\
74.367986376	102.79\\
74.38395429	102.78\\
74.400023822	102.76\\
74.416001815	102.75\\
74.434508401	102.74\\
74.448077883999	102.73\\
74.464666727	102.71\\
74.480611635	102.7\\
74.496006119999	102.69\\
74.511923204999	102.68\\
74.528024303	102.66\\
74.54400829	102.65\\
74.559953228	102.64\\
74.576085312	102.62\\
74.591965815	102.61\\
74.608233123	102.6\\
74.624011382	102.59\\
74.639993130999	102.57\\
74.655959173999	102.56\\
74.671962645	102.55\\
74.688001761	102.54\\
74.704060413	102.52\\
74.719960971	102.51\\
74.73596567	102.5\\
74.752023357999	102.48\\
74.767922305	102.47\\
74.784067169	102.46\\
74.799870099999	102.45\\
74.815933397	102.43\\
74.832091147	102.42\\
74.847985855	102.41\\
74.864140947	102.4\\
74.879986194	102.38\\
74.896018157	102.37\\
74.911886034999	102.36\\
74.930102716	102.34\\
74.945232682	102.33\\
74.961218307	102.32\\
74.976606819999	102.31\\
74.99204547	102.29\\
75.009794585	102.29\\
75.0250794	102.31\\
75.040406531999	102.32\\
75.055947961	102.33\\
75.071950045001	102.34\\
75.088017855999	102.36\\
75.104040730999	102.37\\
75.119999425	102.38\\
75.135992908	102.39\\
75.152036229999	102.41\\
75.167927473	102.42\\
75.184013418999	102.43\\
75.200011085	102.44\\
75.21604044	102.46\\
75.232181928	102.47\\
75.24794557	102.48\\
75.266507499	102.5\\
75.281737716	102.51\\
75.296961821	102.52\\
75.31224779	102.53\\
75.327982106	102.55\\
75.343998416	102.56\\
75.359999932	102.57\\
75.37606468	102.58\\
75.392105387	102.6\\
75.407971637	102.61\\
75.423979203	102.62\\
75.440119022999	102.63\\
75.455927287	102.65\\
75.474157720998	102.66\\
75.48946976	102.67\\
75.505759123	102.69\\
75.521096469	102.7\\
75.536397472999	102.71\\
75.551979989999	102.72\\
75.567964321001	102.74\\
75.584009977999	102.75\\
75.600005727	102.76\\
75.615984647	102.77\\
75.634430107	102.79\\
75.649554317	102.8\\
75.664770678	102.81\\
75.680824443	102.83\\
75.696694121	102.84\\
75.712163576	102.85\\
75.727820763999	102.86\\
75.746094064	102.88\\
75.761340415	102.89\\
75.776471936	102.9\\
75.791934970998	102.91\\
75.808019772999	102.93\\
75.824053652999	102.94\\
75.839950147	102.95\\
75.855886170999	102.96\\
75.87205279	102.98\\
75.887984644001	102.99\\
75.903926328999	103\\
75.919970277999	103.02\\
75.936012461	103.03\\
75.952064899	103.04\\
75.968012263999	103.05\\
75.984359947	103.07\\
76.000012067999	103.08\\
76.017766609	103.08\\
76.033048809	103.07\\
76.048242156	103.06\\
76.064428579	103.04\\
76.079940172	103.03\\
76.095986448	103.02\\
76.111991894	103.01\\
76.127994352	102.99\\
76.143967253999	102.98\\
76.160042432	102.97\\
76.176097169	102.95\\
76.191940203	102.94\\
76.208067022999	102.93\\
76.223992121	102.92\\
76.240064583999	102.9\\
76.256004059	102.89\\
76.272016345	102.88\\
76.287960065	102.87\\
76.304066864999	102.85\\
76.319951078001	102.84\\
76.335994868	102.83\\
76.351981701	102.82\\
76.367930571	102.8\\
76.383979111	102.79\\
76.39998585	102.78\\
76.416030359	102.76\\
76.431987138999	102.75\\
76.448039072	102.74\\
76.46409251	102.73\\
76.480092116	102.71\\
76.496031763	102.7\\
76.511896989999	102.69\\
76.528038344	102.68\\
76.543945694	102.66\\
76.559914777	102.65\\
76.575890557	102.64\\
76.591979017	102.62\\
76.607964805	102.61\\
76.62389166	102.6\\
76.640021467	102.59\\
76.655935640999	102.57\\
76.672149237999	102.56\\
76.687979791999	102.55\\
76.70402188	102.54\\
76.719991824999	102.52\\
76.736001721	102.51\\
76.752066771	102.5\\
76.768001229	102.49\\
76.78414942	102.47\\
76.799964259	102.46\\
76.815975915	102.45\\
76.832112809999	102.43\\
76.847988585	102.42\\
76.866665951	102.41\\
76.881952927	102.39\\
76.897064876999	102.38\\
76.912969619	102.37\\
76.928073059	102.36\\
76.94402117	102.35\\
76.960017665	102.33\\
76.975952875	102.32\\
76.991942592	102.31\\
77.009621614999	102.31\\
77.02478884	102.32\\
77.040056899	102.33\\
77.056018836	102.34\\
77.072024993	102.36\\
77.087911323	102.37\\
77.104038284	102.38\\
77.120034894999	102.39\\
77.13600507	102.41\\
77.152017236	102.42\\
77.168006976	102.43\\
77.183971581	102.45\\
77.200044887999	102.46\\
77.215897831	102.47\\
77.234380669	102.49\\
77.249675022999	102.5\\
77.264889364999	102.51\\
77.280195219	102.52\\
77.295997428999	102.53\\
77.312020795001	102.55\\
77.328102248001	102.56\\
77.344039397999	102.57\\
77.360021972	102.58\\
77.376063586	102.6\\
77.391979943	102.61\\
77.408039505001	102.62\\
77.423983482	102.64\\
77.440038253999	102.65\\
77.456038652999	102.66\\
77.472062495	102.67\\
77.488001163999	102.69\\
77.504029261	102.7\\
77.519940588	102.71\\
77.535997453999	102.72\\
77.55205139	102.74\\
77.567940833	102.75\\
77.584031266001	102.76\\
77.599921048999	102.77\\
77.616020935	102.79\\
77.632480674001	102.8\\
77.648000602999	102.81\\
77.664128236	102.83\\
77.67998323	102.84\\
77.696079256	102.85\\
77.71201801	102.86\\
77.728005298999	102.88\\
77.743981843	102.89\\
77.760001697	102.9\\
77.775934961999	102.91\\
77.791948974	102.93\\
77.807973423	102.94\\
77.823970569	102.95\\
77.840056376999	102.97\\
77.855989894	102.98\\
77.872346326	102.99\\
77.887982935	103\\
77.904025184	103.02\\
77.920014599	103.03\\
77.936006708	103.04\\
77.952020583999	103.05\\
77.968008864	103.07\\
77.983971481	103.08\\
78.000067703	103.09\\
78.017399355999	103.09\\
78.032496736999	103.08\\
78.047936784	103.06\\
78.064367799001	103.05\\
78.080005576	103.04\\
78.096066003	103.03\\
78.111946291	103.01\\
78.127978616	103\\
78.143990561	102.99\\
78.160034484	102.98\\
78.176081878999	102.96\\
78.192022453	102.95\\
78.207979638999	102.94\\
78.223968029998	102.92\\
78.239999239	102.91\\
78.255877208	102.9\\
78.272061276001	102.89\\
78.287936998	102.87\\
78.304013789	102.86\\
78.320004848	102.85\\
78.335975855	102.84\\
78.352029825	102.82\\
78.36798222	102.81\\
78.383959979	102.8\\
78.399945448	102.78\\
78.416023278	102.77\\
78.434578303	102.76\\
78.449705942	102.74\\
78.464966958	102.73\\
78.480314609	102.72\\
78.495959703	102.71\\
78.511995472999	102.7\\
78.528014659999	102.68\\
78.544011685999	102.67\\
78.560055097999	102.66\\
78.57839971	102.64\\
78.59360653	102.63\\
78.60874004	102.62\\
78.624081715	102.61\\
78.640052587	102.59\\
78.655989848	102.58\\
78.671971725	102.57\\
78.687990626	102.56\\
78.703905189	102.54\\
78.719892328999	102.53\\
78.735875725	102.52\\
78.752012618	102.51\\
78.767991153	102.49\\
78.784005302	102.48\\
78.800068003	102.47\\
78.816032767	102.46\\
78.834472866	102.44\\
78.849791468999	102.43\\
78.864960649	102.42\\
78.880234229	102.4\\
78.895915828	102.39\\
78.911934052	102.38\\
78.92790384	102.37\\
78.944019736	102.35\\
78.959904336	102.34\\
78.975848493	102.33\\
78.992030926	102.32\\
79.009773958001	102.31\\
79.025248281999	102.33\\
79.040671993	102.34\\
79.055898717	102.35\\
79.071956937	102.36\\
79.087971224	102.38\\
79.104028973	102.39\\
79.119985602	102.4\\
79.135878382	102.41\\
79.151943149001	102.43\\
79.167993058	102.44\\
79.184041481	102.45\\
79.200008056	102.47\\
79.215987953	102.48\\
79.232002454999	102.49\\
79.247986163	102.5\\
79.266529307	102.52\\
79.281720822	102.53\\
79.297034467999	102.54\\
79.312312145	102.55\\
79.328021434	102.57\\
79.344047135	102.58\\
79.360039982	102.59\\
79.376104785	102.6\\
79.391967536999	102.62\\
79.40799992	102.63\\
79.424063533	102.64\\
79.439881291	102.66\\
79.455881839	102.67\\
79.471958822	102.68\\
79.487951497	102.69\\
79.503985821	102.71\\
79.519996795	102.72\\
79.535932194	102.73\\
79.552007458	102.74\\
79.567967493	102.76\\
79.583948522999	102.77\\
79.599959586	102.78\\
79.615993616	102.79\\
79.632284678	102.81\\
79.647997306	102.82\\
79.664073486	102.83\\
79.680005387	102.85\\
79.696011042	102.86\\
79.712026135	102.87\\
79.727930586999	102.88\\
79.744020056	102.9\\
79.760159555	102.91\\
79.775973206999	102.92\\
79.792000090999	102.93\\
79.808037292	102.95\\
79.824012498999	102.96\\
79.840035925	102.97\\
79.855968153	102.99\\
79.872071081	103\\
79.88803414	103.01\\
79.903852547	103.02\\
79.919967931	103.04\\
79.936035417	103.05\\
79.951983072	103.06\\
79.96802119	103.07\\
79.983994491	103.09\\
80.000011814	103.1\\
80.01761825	103.1\\
80.032789805	103.09\\
80.048163253	103.08\\
80.064335152	103.06\\
80.079991947	103.05\\
80.096092798	103.04\\
80.112053633	103.03\\
80.127955383	103.01\\
80.144043877	103\\
80.159994417	102.99\\
80.176082831	102.97\\
80.192049543	102.96\\
80.208062652	102.95\\
80.223974479	102.94\\
80.240051978	102.92\\
80.255949505	102.91\\
80.271922175	102.9\\
80.288095921999	102.89\\
80.304051324	102.87\\
80.31992259	102.86\\
80.335989395	102.85\\
80.352039256	102.84\\
80.368021217999	102.82\\
80.384099658999	102.81\\
80.399964421999	102.8\\
80.416054894	102.78\\
80.432016029	102.77\\
80.447986085	102.76\\
80.46409113	102.75\\
80.479999475	102.73\\
80.496005878	102.72\\
80.512024288	102.71\\
80.527927897	102.7\\
80.544062750999	102.68\\
80.560082453999	102.67\\
80.575980550999	102.66\\
80.592058921	102.64\\
80.608023296999	102.63\\
80.624018164	102.62\\
80.640079855	102.61\\
80.655947532	102.59\\
80.671992181	102.58\\
80.687949923	102.57\\
80.703995352	102.56\\
80.720037643	102.54\\
80.735998673	102.53\\
80.752065653	102.52\\
80.768040579999	102.51\\
80.784069862999	102.49\\
80.799985079	102.48\\
80.816012899999	102.47\\
80.832338812	102.45\\
80.847956014	102.44\\
80.866557557999	102.43\\
80.881864941	102.41\\
80.897160477	102.4\\
80.912469028	102.39\\
80.927945277999	102.38\\
80.943967247	102.37\\
80.959969981	102.35\\
80.976014904998	102.34\\
80.991992227	102.33\\
81.009655857	102.33\\
81.025092796999	102.34\\
81.040329135	102.35\\
81.055957415999	102.36\\
81.071943099	102.38\\
81.088012928	102.39\\
81.104095769	102.4\\
81.120051269	102.41\\
81.135895382	102.43\\
81.154208387	102.44\\
81.16932168	102.45\\
81.184549638	102.47\\
81.200003157	102.48\\
81.216042214	102.49\\
81.232304942	102.5\\
81.248051045001	102.52\\
81.266527317	102.53\\
81.281628011	102.54\\
81.296799075	102.56\\
81.312014111	102.57\\
81.328022991	102.58\\
81.344043902	102.59\\
81.360048656	102.6\\
81.376034462	102.62\\
81.391988434999	102.63\\
81.408006194	102.64\\
81.424009018	102.66\\
81.440053206999	102.67\\
81.455871618	102.68\\
81.472067249	102.69\\
81.488012267999	102.71\\
81.503953897	102.72\\
81.519908699	102.73\\
81.53600453	102.74\\
81.552058194	102.76\\
81.567917043999	102.77\\
81.584080239999	102.78\\
81.599981691	102.8\\
81.615998538	102.81\\
81.632558243	102.82\\
81.648017035	102.83\\
81.664198071	102.85\\
81.680007964	102.86\\
81.695969125001	102.87\\
81.711910486999	102.88\\
81.728007106	102.9\\
81.743983666	102.91\\
81.760032847	102.92\\
81.776058942999	102.93\\
81.792029345	102.95\\
81.807960161	102.96\\
81.823962239	102.97\\
81.839962592999	102.99\\
81.855940551	103\\
81.872024805	103.01\\
81.887952118	103.02\\
81.904084503	103.04\\
81.919988081	103.05\\
81.935990364	103.06\\
81.952032647999	103.07\\
81.967919546	103.09\\
81.984024344	103.1\\
81.999967274	103.11\\
82.017750069	103.11\\
82.033101671999	103.1\\
82.048303706	103.09\\
82.063920218	103.08\\
82.079990090999	103.06\\
82.095966531	103.05\\
82.111925590999	103.04\\
82.127985279998	103.03\\
82.144037500999	103.01\\
82.160023362	103\\
82.175960839	102.99\\
82.192048574999	102.97\\
82.208069676	102.96\\
82.223987262999	102.95\\
82.239931479	102.94\\
82.25596123	102.92\\
82.271945838	102.91\\
82.287889814	102.9\\
82.303965267999	102.89\\
82.319980362	102.87\\
82.336030078001	102.86\\
82.351935079	102.85\\
82.368015343999	102.84\\
82.384051638999	102.82\\
82.399985435	102.81\\
82.416006984	102.8\\
82.432099998	102.78\\
82.448074402999	102.77\\
82.466827814	102.76\\
82.479923810999	102.75\\
82.496144906	102.73\\
82.512011427	102.72\\
82.527954923	102.71\\
82.544029890999	102.7\\
82.559962505	102.68\\
82.575980258999	102.67\\
82.591975225999	102.66\\
82.607925916999	102.64\\
82.624018656	102.63\\
82.639994869	102.62\\
82.656027166	102.61\\
82.672030482	102.59\\
82.687989467	102.58\\
82.704079553999	102.57\\
82.71997273	102.56\\
82.73593557	102.54\\
82.752004491	102.53\\
82.768012573	102.52\\
82.78408198	102.51\\
82.799981850999	102.49\\
82.816010819999	102.48\\
82.832457796	102.47\\
82.847974539	102.45\\
82.866649188	102.44\\
82.881798891001	102.43\\
82.897132985	102.42\\
82.912369857999	102.4\\
82.927854716	102.39\\
82.944043326	102.38\\
82.960306428	102.37\\
82.975856385	102.35\\
82.991904315	102.34\\
83.009504878999	102.34\\
83.025275347	102.35\\
83.040624973	102.36\\
83.055999363	102.38\\
83.072039484	102.39\\
83.088057644001	102.4\\
83.10395053	102.41\\
83.119991685	102.43\\
83.135969562998	102.44\\
83.151925247	102.45\\
83.167947762	102.47\\
83.183944977	102.48\\
83.19997708	102.49\\
83.21599214	102.5\\
83.234345052	102.52\\
83.249535986001	102.53\\
83.264690777	102.54\\
83.280440589	102.56\\
83.296056496	102.57\\
83.312005349999	102.58\\
83.328012588	102.59\\
83.343921944	102.6\\
83.360044751	102.62\\
83.376006704999	102.63\\
83.39201697	102.64\\
83.408031526	102.66\\
83.423974099	102.67\\
83.439881825	102.68\\
83.456039397999	102.69\\
83.472139317	102.71\\
83.488046133999	102.72\\
83.504105333001	102.73\\
83.519969622	102.74\\
83.535976781001	102.76\\
83.552083992	102.77\\
83.567951257	102.78\\
83.584023479999	102.8\\
83.600006645999	102.81\\
83.615997864	102.82\\
83.632700085	102.83\\
83.647960589	102.85\\
83.666486676	102.86\\
83.681610913	102.87\\
83.697174587	102.89\\
83.712585411	102.9\\
83.727978863	102.91\\
83.744085393	102.92\\
83.760446543	102.94\\
83.776150085	102.95\\
83.792043471	102.96\\
83.807876246	102.97\\
83.823979663001	102.99\\
83.839996324	103\\
83.856021945	103.01\\
83.871910982	103.02\\
83.888037015	103.04\\
83.904030098	103.05\\
83.920001336	103.06\\
83.935984786	103.07\\
83.952040927	103.09\\
83.967969755	103.1\\
83.984051326	103.11\\
83.999981601	103.12\\
84.015859515999	103.12\\
84.032416515	103.11\\
84.047989899	103.1\\
84.06652094	103.08\\
84.081659691	103.07\\
84.096777373	103.06\\
84.111970173	103.05\\
84.128279516001	103.03\\
84.144156956	103.02\\
84.160002305	103.01\\
84.176168664	103\\
84.192035646	102.98\\
84.207952876	102.97\\
84.224013131	102.96\\
84.239948362999	102.94\\
84.255932625999	102.93\\
84.272061345	102.92\\
84.287970241	102.91\\
84.304024703	102.89\\
84.320082289	102.88\\
84.335998541	102.87\\
84.352109965	102.86\\
84.367971659	102.84\\
84.384048206	102.83\\
84.400011630999	102.82\\
84.416118757	102.81\\
84.432279169	102.79\\
84.44807121	102.78\\
84.466089649	102.76\\
84.481396496	102.75\\
84.496650668	102.74\\
84.512035626999	102.73\\
84.528039975	102.72\\
84.544093511	102.7\\
84.560067152999	102.69\\
84.576000248999	102.68\\
84.592037864999	102.67\\
84.607989549001	102.65\\
84.623990928	102.64\\
84.63992569	102.63\\
84.655978970998	102.61\\
84.671973827	102.6\\
84.687974272999	102.59\\
84.704435057	102.58\\
84.720012687	102.56\\
84.735890682	102.55\\
84.752174107	102.54\\
84.768023043	102.53\\
84.784043366999	102.51\\
84.800038815	102.5\\
84.816020151	102.49\\
84.832052732	102.48\\
84.847930724	102.46\\
84.866217732	102.45\\
84.881353141001	102.44\\
84.896666079	102.42\\
84.911984504	102.41\\
84.927970915	102.4\\
84.94391892	102.39\\
84.959917991	102.37\\
84.975978732	102.36\\
84.992020755999	102.35\\
85.009721138999	102.35\\
85.025213269999	102.36\\
85.040675619999	102.37\\
85.055931486	102.38\\
85.072037569	102.4\\
85.088015015999	102.41\\
85.104049265	102.42\\
85.119996145	102.43\\
85.135990309	102.45\\
85.152047357	102.46\\
85.167990300999	102.47\\
85.184073532	102.49\\
85.199958782	102.5\\
85.216014048	102.51\\
85.232334605999	102.52\\
85.248011644	102.54\\
85.266559548	102.55\\
85.281743391001	102.56\\
85.296915122	102.58\\
85.313651655	102.59\\
85.328894632	102.6\\
85.344093105	102.61\\
85.359931864	102.63\\
85.375971949	102.64\\
85.391961277999	102.65\\
85.407988908	102.66\\
85.423976293	102.68\\
85.440054081999	102.69\\
85.456280334	102.7\\
85.471895907	102.71\\
85.487991119	102.73\\
85.504075383	102.74\\
85.519946029	102.75\\
85.536020904	102.77\\
85.551930722	102.78\\
85.567954336	102.79\\
85.584025687	102.8\\
85.59990298	102.82\\
85.616005291	102.83\\
85.632184054	102.84\\
85.647908906999	102.85\\
85.664080871	102.87\\
85.679984829999	102.88\\
85.698107041	102.89\\
85.713136779	102.91\\
85.728250062	102.92\\
85.745946361	102.93\\
85.761598189999	102.94\\
85.776850287	102.96\\
85.792337137	102.97\\
85.808015746	102.98\\
85.823965611	102.99\\
85.840018845998	103.01\\
85.856010669	103.02\\
85.872047326	103.03\\
85.887840281001	103.04\\
85.904009741	103.06\\
85.919995526	103.07\\
85.936031518	103.08\\
85.951975551	103.09\\
85.967959966	103.11\\
85.983958951	103.12\\
85.999985016001	103.13\\
86.017344414	103.13\\
86.032723794	103.12\\
86.047960372	103.11\\
86.066337383999	103.09\\
86.081468354	103.08\\
86.096636797	103.07\\
86.112015204999	103.06\\
86.128010114999	103.05\\
86.14405736	103.03\\
86.160125885	103.02\\
86.176428964	103.01\\
86.192159347	102.99\\
86.208015412	102.98\\
86.224012739	102.97\\
86.240033414	102.96\\
86.255983668999	102.94\\
86.272037119	102.93\\
86.287839005	102.92\\
86.304036589	102.91\\
86.319893478	102.89\\
86.335968005	102.88\\
86.351912621	102.87\\
86.367982578	102.86\\
86.384072319	102.84\\
86.399942189	102.83\\
86.415987553999	102.82\\
86.432485446	102.81\\
86.448057393	102.79\\
86.464049723	102.78\\
86.480054585999	102.77\\
86.49600945	102.75\\
86.511957032	102.74\\
86.527952510999	102.73\\
86.544090126	102.72\\
86.559993169	102.7\\
86.575969505	102.69\\
86.591962575	102.68\\
86.607829436	102.67\\
86.624066803999	102.65\\
86.640041651001	102.64\\
86.656006232999	102.63\\
86.672025984999	102.61\\
86.687946828	102.6\\
86.704028201999	102.59\\
86.719984751	102.58\\
86.736003958	102.56\\
86.752001154	102.55\\
86.76800248	102.54\\
86.783971253	102.53\\
86.799937636	102.51\\
86.815991219	102.5\\
86.83216179	102.49\\
86.847980814	102.47\\
86.866629354	102.46\\
86.882034569	102.45\\
86.897422251	102.44\\
86.912691118999	102.42\\
86.927896916999	102.41\\
86.944028854	102.4\\
86.959960753	102.39\\
86.976048008999	102.37\\
86.991972816	102.36\\
87.009778137001	102.36\\
87.024943288	102.37\\
87.04032096	102.38\\
87.055980927	102.4\\
87.071964059001	102.41\\
87.087995701999	102.42\\
87.103987341	102.43\\
87.119957501	102.45\\
87.136082593	102.46\\
87.152036135999	102.47\\
87.168103091	102.49\\
87.184033651001	102.5\\
87.199987537	102.51\\
87.215998772999	102.52\\
87.234456353	102.54\\
87.249682201	102.55\\
87.265141943	102.56\\
87.28041521	102.58\\
87.295916513	102.59\\
87.311934279998	102.6\\
87.328012977	102.61\\
87.344016668999	102.63\\
87.359999768	102.64\\
87.375960224	102.65\\
87.391979729	102.66\\
87.407999265999	102.68\\
87.423998006	102.69\\
87.440052619	102.7\\
87.456657838	102.72\\
87.472074357999	102.73\\
87.487984362999	102.74\\
87.503962076	102.75\\
87.519938456	102.76\\
87.535966392	102.78\\
87.552034185999	102.79\\
87.567992868	102.8\\
87.584032333	102.82\\
87.599942138999	102.83\\
87.615974258999	102.84\\
87.632442239	102.85\\
87.648041934001	102.87\\
87.664250411999	102.88\\
87.680050455999	102.89\\
87.695974605999	102.9\\
87.711809953	102.92\\
87.730146753	102.93\\
87.745381993	102.94\\
87.760410887	102.96\\
87.776003504	102.97\\
87.792267288	102.98\\
87.808048741	102.99\\
87.829893338	103.01\\
87.842020048999	103.02\\
87.857300447	103.03\\
87.872659552	103.04\\
87.888140758999	103.06\\
87.904043311	103.07\\
87.920008566	103.08\\
87.936019458999	103.09\\
87.952051915	103.11\\
87.96797761	103.12\\
87.984048854999	103.13\\
87.999969538	103.15\\
88.017640809999	103.15\\
88.033383904998	103.13\\
88.048752272	103.12\\
88.064378076	103.11\\
88.080008509	103.1\\
88.096006053999	103.08\\
88.111987619999	103.07\\
88.128003284	103.06\\
88.144126561999	103.05\\
88.159896274	103.03\\
88.175924854999	103.02\\
88.192014755	103.01\\
88.208006947	103\\
88.223956191	102.98\\
88.240033822	102.97\\
88.256036774	102.96\\
88.272041778999	102.94\\
88.288053748001	102.93\\
88.304027097001	102.92\\
88.319982694	102.91\\
88.336011422	102.89\\
88.352041722	102.88\\
88.367931185	102.87\\
88.384073326999	102.86\\
88.400109055	102.84\\
88.416093649	102.83\\
88.432020272	102.82\\
88.448030131	102.81\\
88.463930984	102.79\\
88.479954431	102.78\\
88.495946111	102.77\\
88.512072033	102.75\\
88.52794769	102.74\\
88.544031122	102.73\\
88.560010081999	102.72\\
88.578561515	102.7\\
88.591878373	102.69\\
88.607873863	102.68\\
88.623978783	102.67\\
88.64003094	102.65\\
88.655986001	102.64\\
88.67208345	102.63\\
88.688021613	102.61\\
88.703913840001	102.6\\
88.719999524	102.59\\
88.735879783999	102.58\\
88.751988795999	102.56\\
88.768086475	102.55\\
88.783939375	102.54\\
88.799989743	102.53\\
88.816001738	102.51\\
88.832070526	102.5\\
88.848021263	102.49\\
88.863880503	102.48\\
88.882052263	102.46\\
88.897178991999	102.45\\
88.912450628999	102.44\\
88.927939331999	102.42\\
88.944149935999	102.41\\
88.960072744999	102.4\\
88.975976622	102.39\\
88.992020438	102.37\\
89.009786796	102.37\\
89.025095632	102.39\\
89.040255484	102.4\\
89.055929932999	102.41\\
89.072039998	102.42\\
89.087981348	102.43\\
89.104026123	102.45\\
89.120001522	102.46\\
89.135942279	102.47\\
89.152027862	102.49\\
89.167902159	102.5\\
89.183954165	102.51\\
89.199975817	102.52\\
89.215880099	102.54\\
89.234460121	102.55\\
89.249609937	102.56\\
89.264830701	102.58\\
89.280195239999	102.59\\
89.296010795	102.6\\
89.312084043999	102.61\\
89.328081958	102.63\\
89.346506337	102.64\\
89.359940982	102.65\\
89.376014806	102.66\\
89.39198286	102.68\\
89.407978255	102.69\\
89.424015415999	102.7\\
89.440051966	102.71\\
89.457562279	102.73\\
89.472956718	102.74\\
89.488110692	102.75\\
89.503969451	102.76\\
89.519990147	102.78\\
89.53601218	102.79\\
89.552009835	102.8\\
89.568017739	102.82\\
89.584021429999	102.83\\
89.600021977	102.84\\
89.615991819	102.85\\
89.634473312	102.87\\
89.649814692	102.88\\
89.665173953	102.89\\
89.680535544	102.91\\
89.695928956999	102.92\\
89.711983258	102.93\\
89.72794753	102.94\\
89.744110232	102.95\\
89.759991982	102.97\\
89.775904856999	102.98\\
89.791936137001	102.99\\
89.807935161999	103.01\\
89.82400323	103.02\\
89.840120312998	103.03\\
89.856037126	103.04\\
89.871888859	103.06\\
89.888067693	103.07\\
89.904066027	103.08\\
89.920005869	103.09\\
89.935981261	103.11\\
89.951955819999	103.12\\
89.967924128	103.13\\
89.984091395	103.15\\
89.999906764999	103.16\\
90.017487797	103.16\\
90.033170055	103.15\\
90.048276515	103.13\\
90.064414343	103.12\\
90.07999919	103.11\\
90.096037274	103.1\\
90.11224115	103.08\\
90.128005984	103.07\\
90.143970224	103.06\\
90.160020533	103.05\\
90.17594035	103.03\\
90.191936809	103.02\\
90.207848117	103.01\\
90.223973975	103\\
90.240058250001	102.98\\
90.255998587	102.97\\
90.272035429	102.96\\
90.288058364	102.94\\
90.303981619	102.93\\
90.319971072	102.92\\
90.335982000001	102.91\\
90.352029335	102.89\\
90.368051514	102.88\\
90.384010188	102.87\\
90.39989114	102.86\\
90.418122844	102.84\\
90.433398986	102.83\\
90.448537831	102.82\\
90.464560767	102.8\\
90.479968673	102.79\\
90.495970328999	102.78\\
90.514596421	102.76\\
90.530404885	102.75\\
90.545643968	102.74\\
90.560890960001	102.73\\
90.576166324	102.72\\
90.591944664	102.7\\
90.608007415	102.69\\
90.623988024	102.68\\
90.640121704999	102.67\\
90.656011277999	102.65\\
90.672007277	102.64\\
90.687954665	102.63\\
90.703994359	102.61\\
90.71998925	102.6\\
90.735941484999	102.59\\
90.752035932	102.58\\
90.768127222	102.56\\
90.784029033	102.55\\
90.799963733001	102.54\\
90.815980529	102.53\\
90.832320166999	102.51\\
90.848032568	102.5\\
90.866690477	102.49\\
90.881982824	102.47\\
90.897247522	102.46\\
90.912455253	102.45\\
90.927947600999	102.44\\
90.944085241	102.42\\
90.959892923	102.41\\
90.976006185	102.4\\
90.992042263	102.39\\
91.009733489999	102.39\\
91.025018021	102.4\\
91.040489142999	102.41\\
91.056112459	102.42\\
91.07202097	102.43\\
91.087952493	102.45\\
91.10394101	102.46\\
91.119997100999	102.47\\
91.135937651	102.49\\
91.151969639	102.5\\
91.168036039	102.51\\
91.184089256	102.52\\
91.19998842	102.54\\
91.216019609	102.55\\
91.234444659999	102.56\\
91.249566998	102.58\\
91.264876776999	102.59\\
91.2800697	102.6\\
91.296051893	102.61\\
91.312019402	102.62\\
91.327921758	102.64\\
91.343994286999	102.65\\
91.359902532	102.66\\
91.375930590999	102.68\\
91.39201264	102.69\\
91.408111244	102.7\\
91.424056929	102.71\\
91.440294109	102.73\\
91.45702503	102.74\\
91.472248622	102.75\\
91.488033308	102.77\\
91.503910599	102.78\\
91.520008608001	102.79\\
91.535958151999	102.8\\
91.552030222999	102.82\\
91.568019522	102.83\\
91.583928574	102.84\\
91.599982333	102.85\\
91.616006029998	102.87\\
91.632479968	102.88\\
91.647847168999	102.89\\
91.666311101	102.91\\
91.682042279	102.92\\
91.696985426	102.93\\
91.711914571	102.94\\
91.727858585	102.95\\
91.746053625	102.97\\
91.761141909	102.98\\
91.776187352	102.99\\
91.791923652999	103.01\\
91.807865708	103.02\\
91.82397861	103.03\\
91.842441154	103.05\\
91.857650314	103.06\\
91.873012027999	103.07\\
91.888295118	103.08\\
91.904003876999	103.09\\
91.920076143	103.11\\
91.935982353	103.12\\
91.95202613	103.13\\
91.968014176	103.15\\
91.984006921999	103.16\\
92.000009843	103.17\\
92.017437899	103.17\\
92.033112118	103.16\\
92.048395242	103.15\\
92.06464934	103.13\\
92.079916947	103.12\\
92.096003843	103.11\\
92.111931717001	103.1\\
92.127980600999	103.08\\
92.144029721	103.07\\
92.159909602	103.06\\
92.175935894	103.05\\
92.19202972	103.03\\
92.208017711	103.02\\
92.223999257	103.01\\
92.240013257	103\\
92.255953710001	102.98\\
92.271979882	102.97\\
92.287936331	102.96\\
92.30398491	102.94\\
92.319977856999	102.93\\
92.335981112	102.92\\
92.351895899	102.91\\
92.368012381	102.89\\
92.384008187998	102.88\\
92.400101402	102.87\\
92.416009732	102.86\\
92.432061393	102.84\\
92.448045024	102.83\\
92.464365868	102.82\\
92.479986227	102.8\\
92.496004079999	102.79\\
92.511828486	102.78\\
92.528022294	102.77\\
92.544005634	102.75\\
92.559943205	102.74\\
92.576046730999	102.73\\
92.592150850999	102.72\\
92.607983569	102.7\\
92.623999003	102.69\\
92.640058220998	102.68\\
92.655971404998	102.67\\
92.67203122	102.65\\
92.687949986	102.64\\
92.703928194	102.63\\
92.719997699	102.61\\
92.736009272999	102.6\\
92.75196236	102.59\\
92.768022989999	102.58\\
92.783878645999	102.56\\
92.799863951	102.55\\
92.816000919	102.54\\
92.832108585999	102.53\\
92.84808218	102.51\\
92.864280798	102.5\\
92.879986619	102.49\\
92.896231071999	102.48\\
92.912004678	102.46\\
92.928081524	102.45\\
92.943988064999	102.44\\
92.959987206	102.42\\
92.975950138	102.41\\
92.99190263	102.4\\
93.009749626	102.4\\
93.024962074	102.41\\
93.040294824	102.42\\
93.056041138	102.44\\
93.071961094	102.45\\
93.087975674001	102.46\\
93.104090888	102.47\\
93.119939274	102.49\\
93.135959762999	102.5\\
93.152012906	102.51\\
93.168031677	102.52\\
93.184039722999	102.54\\
93.200057757	102.55\\
93.216091537	102.56\\
93.232353704	102.57\\
93.247879466	102.59\\
93.266869529	102.6\\
93.279971716	102.61\\
93.296063994999	102.63\\
93.311990982	102.64\\
93.327928166	102.65\\
93.344012927999	102.66\\
93.360068793	102.68\\
93.375921937998	102.69\\
93.392076086	102.7\\
93.407973506	102.71\\
93.424028093	102.73\\
93.440057302	102.74\\
93.455855287	102.75\\
93.472039322	102.76\\
93.48800371	102.78\\
93.504105043	102.79\\
93.519959506	102.8\\
93.535973506	102.82\\
93.551979671999	102.83\\
93.567969703	102.84\\
93.584062326	102.85\\
93.599992919	102.87\\
93.616029919	102.88\\
93.632020616	102.89\\
93.648048633999	102.9\\
93.666684111	102.92\\
93.681964182	102.93\\
93.69727192	102.94\\
93.712556288	102.96\\
93.728060583	102.97\\
93.744159546999	102.98\\
93.759969689999	102.99\\
93.77606076	103.01\\
93.791971796	103.02\\
93.808039391001	103.03\\
93.82398459	103.04\\
93.839975357999	103.06\\
93.856021541	103.07\\
93.872063715	103.08\\
93.887977141001	103.09\\
93.90406947	103.11\\
93.919964757	103.12\\
93.935992872	103.13\\
93.952019008999	103.15\\
93.967921932001	103.16\\
93.984054355	103.17\\
94.000017114	103.18\\
94.017648806	103.18\\
94.033546873	103.17\\
94.048744712	103.16\\
94.064343331999	103.15\\
94.079896758	103.13\\
94.095986588	103.12\\
94.111950119999	103.11\\
94.128043929	103.1\\
94.144068013	103.08\\
94.160058397999	103.07\\
94.175983875001	103.06\\
94.191962713	103.05\\
94.208044765999	103.03\\
94.224036616	103.02\\
94.239857625	103.01\\
94.255982489999	103\\
94.272039512001	102.98\\
94.288038904	102.97\\
94.304046724	102.96\\
94.320020489999	102.94\\
94.336033784999	102.93\\
94.352000349	102.92\\
94.367997885999	102.91\\
94.384030267	102.89\\
94.399944281	102.88\\
94.416008504	102.87\\
94.432265349	102.86\\
94.448124845	102.84\\
94.466496371	102.83\\
94.481826378	102.82\\
94.497045392	102.8\\
94.512572468999	102.79\\
94.527992178	102.78\\
94.54400686	102.77\\
94.560154968	102.75\\
94.57594564	102.74\\
94.591971659999	102.73\\
94.608022041999	102.72\\
94.623963402	102.7\\
94.640050416999	102.69\\
94.655970259	102.68\\
94.671860336999	102.67\\
94.688021105	102.65\\
94.70404173	102.64\\
94.719950646999	102.63\\
94.735976318	102.61\\
94.751976757	102.6\\
94.768036435	102.59\\
94.784081460001	102.58\\
94.800021463	102.56\\
94.816120595	102.55\\
94.834692323999	102.54\\
94.849838956999	102.52\\
94.865149163	102.51\\
94.880198423	102.5\\
94.896114762	102.49\\
94.911966196	102.48\\
94.928005892	102.46\\
94.943906518	102.45\\
94.959969921	102.44\\
94.976542544	102.42\\
94.992047894	102.41\\
95.00986678	102.41\\
95.025048454	102.42\\
95.040297829999	102.44\\
95.056066027	102.45\\
95.072030255	102.46\\
95.087997423	102.47\\
95.103956142	102.49\\
95.119993123999	102.5\\
95.135975291999	102.51\\
95.152034986	102.52\\
95.167985877	102.54\\
95.18404518	102.55\\
95.200039685	102.56\\
95.215869077	102.57\\
95.234290019	102.59\\
95.249401973	102.6\\
95.264641003999	102.61\\
95.280010012999	102.63\\
95.296051716	102.64\\
95.311954097999	102.65\\
95.32801106	102.66\\
95.343937899	102.68\\
95.360004496	102.69\\
95.376017685	102.7\\
95.391977471	102.71\\
95.407978581999	102.73\\
95.424096272	102.74\\
95.440082704	102.75\\
95.455925626	102.76\\
95.471907506	102.78\\
95.487849986	102.79\\
95.504032177	102.8\\
95.520258428	102.82\\
95.535994767999	102.83\\
95.551980508	102.84\\
95.567871622	102.85\\
95.584055415	102.87\\
95.599928816	102.88\\
95.615998873001	102.89\\
95.632491958999	102.9\\
95.647991232	102.92\\
95.664287992	102.93\\
95.679947972999	102.94\\
95.696047928999	102.95\\
95.712018912	102.97\\
95.727990956	102.98\\
95.743992418	102.99\\
95.760059413	103.01\\
95.776229087	103.02\\
95.792125804	103.03\\
95.807991894	103.04\\
95.823976181	103.06\\
95.840031319	103.07\\
95.855944632	103.08\\
95.872015319	103.09\\
95.888005178999	103.11\\
95.904015616	103.12\\
95.919814368001	103.13\\
95.935999631	103.15\\
95.952015871	103.16\\
95.967938771	103.17\\
95.983935479999	103.18\\
95.999996952	103.2\\
96.017867702	103.19\\
96.033452824	103.18\\
96.048807312	103.17\\
96.064491476	103.15\\
96.080024821	103.14\\
96.095990351	103.13\\
96.112012544	103.12\\
96.128002545001	103.1\\
96.146266908999	103.09\\
96.161528925999	103.08\\
96.176810314999	103.07\\
96.192103079	103.05\\
96.207983135999	103.04\\
96.224032536	103.03\\
96.240056323	103.02\\
96.255830062998	103\\
96.272009277999	102.99\\
96.287963492	102.98\\
96.304061065	102.96\\
96.320095748	102.95\\
96.336011906	102.94\\
96.351979956999	102.93\\
96.368027442	102.91\\
96.383940487	102.9\\
96.399928324	102.89\\
96.416028734	102.88\\
96.432398411	102.86\\
96.447987114	102.85\\
96.464800984	102.84\\
96.480169842	102.82\\
96.496080060999	102.81\\
96.511888799001	102.8\\
96.527987372	102.79\\
96.544040997	102.77\\
96.560021404998	102.76\\
96.575987821999	102.75\\
96.591935938	102.74\\
96.607989923	102.72\\
96.623961111001	102.71\\
96.639959503	102.7\\
96.655982381	102.69\\
96.672026584	102.67\\
96.6879907	102.66\\
96.704033894001	102.65\\
96.72059371	102.63\\
96.736063535	102.62\\
96.751939484	102.61\\
96.767917877	102.6\\
96.783968635	102.58\\
96.800005515	102.57\\
96.816043498	102.56\\
96.832093781999	102.55\\
96.847873986999	102.53\\
96.86625839	102.52\\
96.881391303	102.51\\
96.896529411999	102.49\\
96.912033062	102.48\\
96.927957288001	102.47\\
96.944065499	102.46\\
96.960017618999	102.44\\
96.975997227999	102.43\\
96.992019662	102.42\\
97.007868237	102.42\\
97.024056295999	102.43\\
97.040039739	102.44\\
97.055932869999	102.46\\
97.071997559	102.47\\
97.087982589	102.48\\
97.104016305	102.49\\
97.119933533999	102.51\\
97.136043927	102.52\\
97.152045007999	102.53\\
97.168100274	102.54\\
97.184044638	102.56\\
97.200001854	102.57\\
97.215993419	102.58\\
97.234560867	102.6\\
97.249834166999	102.61\\
97.265100246	102.62\\
97.280498003999	102.63\\
97.296023757	102.65\\
97.311892335	102.66\\
97.328084661999	102.67\\
97.343906085001	102.68\\
97.360049899	102.7\\
97.376036095	102.71\\
97.39195868	102.72\\
97.407954251	102.73\\
97.423956284	102.75\\
97.440026556	102.76\\
97.456053425999	102.77\\
97.472056781	102.78\\
97.488024586	102.8\\
97.503910421	102.81\\
97.520006231	102.82\\
97.536099009	102.84\\
97.55203857	102.85\\
97.567987897999	102.86\\
97.583944748	102.87\\
97.602030284999	102.89\\
97.61715975	102.9\\
97.632443719	102.91\\
97.647939329	102.92\\
97.664056077	102.94\\
97.680058036999	102.95\\
97.696067248	102.96\\
97.711973104999	102.98\\
97.727965168	102.99\\
97.744023438	103\\
97.760022684001	103.01\\
97.775987187	103.03\\
97.792011577	103.04\\
97.808011518	103.05\\
97.823976581999	103.06\\
97.840130896999	103.08\\
97.856007317999	103.09\\
97.872162838	103.1\\
97.887945221	103.11\\
97.904063374	103.13\\
97.920059727	103.14\\
97.935992324	103.15\\
97.952035729	103.17\\
97.968118286	103.18\\
97.984035581	103.19\\
98.000031585	103.2\\
98.017666089	103.2\\
98.032723305	103.19\\
98.048054993	103.18\\
98.06663059	103.17\\
98.081722531999	103.15\\
98.096936816	103.14\\
98.112299553999	103.13\\
98.127875022999	103.12\\
98.144051729	103.1\\
98.160066975999	103.09\\
98.175979638	103.08\\
98.191988374	103.07\\
98.208056301	103.05\\
98.223846529	103.04\\
98.239993779	103.03\\
98.255931417	103.02\\
98.2719997	103\\
98.288308909	102.99\\
98.304015998	102.98\\
98.319985647999	102.97\\
98.335981486001	102.95\\
98.352046632	102.94\\
98.367989581	102.93\\
98.384174282	102.91\\
98.400013967	102.9\\
98.416023823	102.89\\
98.432442053999	102.88\\
98.44807067	102.86\\
98.464101409999	102.85\\
98.479988871	102.84\\
98.496009079	102.83\\
98.511816331999	102.81\\
98.528015135	102.8\\
98.544047551	102.79\\
98.560005954999	102.77\\
98.575946779998	102.76\\
98.591991594	102.75\\
98.608011987999	102.74\\
98.623912793	102.72\\
98.640197728	102.71\\
98.656016894001	102.7\\
98.672097188	102.69\\
98.687953038999	102.67\\
98.70646642	102.66\\
98.721712029	102.65\\
98.741051342	102.63\\
98.753566464	102.62\\
98.768931292	102.61\\
98.784603548	102.6\\
98.799984727999	102.58\\
98.815999494999	102.57\\
98.832055245	102.56\\
98.847940647999	102.55\\
98.866455622	102.53\\
98.881708001	102.52\\
98.896844666	102.51\\
98.912102797	102.5\\
98.928481539	102.48\\
98.944054049001	102.47\\
98.960259446	102.46\\
98.976105980999	102.44\\
98.992144312998	102.43\\
99.009619665999	102.43\\
99.024945885	102.44\\
99.040088468999	102.46\\
99.056005339	102.47\\
99.072189524999	102.48\\
99.088323256	102.49\\
99.104134600999	102.51\\
99.119949187	102.52\\
99.135951948999	102.53\\
99.151962687	102.54\\
99.167980152999	102.56\\
99.184027262999	102.57\\
99.200077345	102.58\\
99.216120670999	102.59\\
99.232510277	102.61\\
99.248004213	102.62\\
99.264196206999	102.63\\
99.279960137	102.65\\
99.295971696	102.66\\
99.311959283	102.67\\
99.328068636	102.68\\
99.343999031	102.7\\
99.359981303	102.71\\
99.37599171	102.72\\
99.392036556	102.73\\
99.407958758	102.75\\
99.423981892	102.76\\
99.440111718999	102.77\\
99.455891024	102.79\\
99.472030676	102.8\\
99.488007727	102.81\\
99.504010807	102.82\\
99.520087875	102.84\\
99.536015529	102.85\\
99.552052661	102.86\\
99.568012126999	102.87\\
99.584030652999	102.89\\
99.600066095	102.9\\
99.615924347	102.91\\
99.634291027999	102.93\\
99.649345666	102.94\\
99.664512538	102.95\\
};
\end{axis}
\end{tikzpicture}%
}
      \caption{The orientation of the robot over time for
        $K_{\Psi}^R = K_{\Psi, max}^R$. This is the upper limit value of $K_{\Psi}^R$
          before the system becomes unstable}
      \label{fig:8_max}
    \end{figure}
  \end{minipage}
  \hfill
  \begin{minipage}{0.45\linewidth}
    \begin{figure}[H]
      \scalebox{0.6}{% This file was created by matlab2tikz.
%
%The latest updates can be retrieved from
%  http://www.mathworks.com/matlabcentral/fileexchange/22022-matlab2tikz-matlab2tikz
%where you can also make suggestions and rate matlab2tikz.
%
\definecolor{mycolor1}{rgb}{0.00000,0.44700,0.74100}%
%
\begin{tikzpicture}

\begin{axis}[%
width=4.133in,
height=3.26in,
at={(0.693in,0.44in)},
scale only axis,
xmin=86.544090126,
xmax=101,
xmajorgrids,
xlabel={Time (seconds)},
ymin=102,
ymax=104,
ymajorgrids,
ylabel={Angle (degrees)},
axis background/.style={fill=white}
]
\addplot [color=mycolor1,solid,forget plot]
  table[row sep=crcr]{%
86.544090126	102.72\\
86.559993169	102.7\\
86.575969505	102.69\\
86.591962575	102.68\\
86.607829436	102.67\\
86.624066803999	102.65\\
86.640041651001	102.64\\
86.656006232999	102.63\\
86.672025984999	102.61\\
86.687946828	102.6\\
86.704028201999	102.59\\
86.719984751	102.58\\
86.736003958	102.56\\
86.752001154	102.55\\
86.76800248	102.54\\
86.783971253	102.53\\
86.799937636	102.51\\
86.815991219	102.5\\
86.83216179	102.49\\
86.847980814	102.47\\
86.866629354	102.46\\
86.882034569	102.45\\
86.897422251	102.44\\
86.912691118999	102.42\\
86.927896916999	102.41\\
86.944028854	102.4\\
86.959960753	102.39\\
86.976048008999	102.37\\
86.991972816	102.36\\
87.009778137001	102.36\\
87.024943288	102.37\\
87.04032096	102.38\\
87.055980927	102.4\\
87.071964059001	102.41\\
87.087995701999	102.42\\
87.103987341	102.43\\
87.119957501	102.45\\
87.136082593	102.46\\
87.152036135999	102.47\\
87.168103091	102.49\\
87.184033651001	102.5\\
87.199987537	102.51\\
87.215998772999	102.52\\
87.234456353	102.54\\
87.249682201	102.55\\
87.265141943	102.56\\
87.28041521	102.58\\
87.295916513	102.59\\
87.311934279998	102.6\\
87.328012977	102.61\\
87.344016668999	102.63\\
87.359999768	102.64\\
87.375960224	102.65\\
87.391979729	102.66\\
87.407999265999	102.68\\
87.423998006	102.69\\
87.440052619	102.7\\
87.456657838	102.72\\
87.472074357999	102.73\\
87.487984362999	102.74\\
87.503962076	102.75\\
87.519938456	102.76\\
87.535966392	102.78\\
87.552034185999	102.79\\
87.567992868	102.8\\
87.584032333	102.82\\
87.599942138999	102.83\\
87.615974258999	102.84\\
87.632442239	102.85\\
87.648041934001	102.87\\
87.664250411999	102.88\\
87.680050455999	102.89\\
87.695974605999	102.9\\
87.711809953	102.92\\
87.730146753	102.93\\
87.745381993	102.94\\
87.760410887	102.96\\
87.776003504	102.97\\
87.792267288	102.98\\
87.808048741	102.99\\
87.829893338	103.01\\
87.842020048999	103.02\\
87.857300447	103.03\\
87.872659552	103.04\\
87.888140758999	103.06\\
87.904043311	103.07\\
87.920008566	103.08\\
87.936019458999	103.09\\
87.952051915	103.11\\
87.96797761	103.12\\
87.984048854999	103.13\\
87.999969538	103.15\\
88.017640809999	103.15\\
88.033383904998	103.13\\
88.048752272	103.12\\
88.064378076	103.11\\
88.080008509	103.1\\
88.096006053999	103.08\\
88.111987619999	103.07\\
88.128003284	103.06\\
88.144126561999	103.05\\
88.159896274	103.03\\
88.175924854999	103.02\\
88.192014755	103.01\\
88.208006947	103\\
88.223956191	102.98\\
88.240033822	102.97\\
88.256036774	102.96\\
88.272041778999	102.94\\
88.288053748001	102.93\\
88.304027097001	102.92\\
88.319982694	102.91\\
88.336011422	102.89\\
88.352041722	102.88\\
88.367931185	102.87\\
88.384073326999	102.86\\
88.400109055	102.84\\
88.416093649	102.83\\
88.432020272	102.82\\
88.448030131	102.81\\
88.463930984	102.79\\
88.479954431	102.78\\
88.495946111	102.77\\
88.512072033	102.75\\
88.52794769	102.74\\
88.544031122	102.73\\
88.560010081999	102.72\\
88.578561515	102.7\\
88.591878373	102.69\\
88.607873863	102.68\\
88.623978783	102.67\\
88.64003094	102.65\\
88.655986001	102.64\\
88.67208345	102.63\\
88.688021613	102.61\\
88.703913840001	102.6\\
88.719999524	102.59\\
88.735879783999	102.58\\
88.751988795999	102.56\\
88.768086475	102.55\\
88.783939375	102.54\\
88.799989743	102.53\\
88.816001738	102.51\\
88.832070526	102.5\\
88.848021263	102.49\\
88.863880503	102.48\\
88.882052263	102.46\\
88.897178991999	102.45\\
88.912450628999	102.44\\
88.927939331999	102.42\\
88.944149935999	102.41\\
88.960072744999	102.4\\
88.975976622	102.39\\
88.992020438	102.37\\
89.009786796	102.37\\
89.025095632	102.39\\
89.040255484	102.4\\
89.055929932999	102.41\\
89.072039998	102.42\\
89.087981348	102.43\\
89.104026123	102.45\\
89.120001522	102.46\\
89.135942279	102.47\\
89.152027862	102.49\\
89.167902159	102.5\\
89.183954165	102.51\\
89.199975817	102.52\\
89.215880099	102.54\\
89.234460121	102.55\\
89.249609937	102.56\\
89.264830701	102.58\\
89.280195239999	102.59\\
89.296010795	102.6\\
89.312084043999	102.61\\
89.328081958	102.63\\
89.346506337	102.64\\
89.359940982	102.65\\
89.376014806	102.66\\
89.39198286	102.68\\
89.407978255	102.69\\
89.424015415999	102.7\\
89.440051966	102.71\\
89.457562279	102.73\\
89.472956718	102.74\\
89.488110692	102.75\\
89.503969451	102.76\\
89.519990147	102.78\\
89.53601218	102.79\\
89.552009835	102.8\\
89.568017739	102.82\\
89.584021429999	102.83\\
89.600021977	102.84\\
89.615991819	102.85\\
89.634473312	102.87\\
89.649814692	102.88\\
89.665173953	102.89\\
89.680535544	102.91\\
89.695928956999	102.92\\
89.711983258	102.93\\
89.72794753	102.94\\
89.744110232	102.95\\
89.759991982	102.97\\
89.775904856999	102.98\\
89.791936137001	102.99\\
89.807935161999	103.01\\
89.82400323	103.02\\
89.840120312998	103.03\\
89.856037126	103.04\\
89.871888859	103.06\\
89.888067693	103.07\\
89.904066027	103.08\\
89.920005869	103.09\\
89.935981261	103.11\\
89.951955819999	103.12\\
89.967924128	103.13\\
89.984091395	103.15\\
89.999906764999	103.16\\
90.017487797	103.16\\
90.033170055	103.15\\
90.048276515	103.13\\
90.064414343	103.12\\
90.07999919	103.11\\
90.096037274	103.1\\
90.11224115	103.08\\
90.128005984	103.07\\
90.143970224	103.06\\
90.160020533	103.05\\
90.17594035	103.03\\
90.191936809	103.02\\
90.207848117	103.01\\
90.223973975	103\\
90.240058250001	102.98\\
90.255998587	102.97\\
90.272035429	102.96\\
90.288058364	102.94\\
90.303981619	102.93\\
90.319971072	102.92\\
90.335982000001	102.91\\
90.352029335	102.89\\
90.368051514	102.88\\
90.384010188	102.87\\
90.39989114	102.86\\
90.418122844	102.84\\
90.433398986	102.83\\
90.448537831	102.82\\
90.464560767	102.8\\
90.479968673	102.79\\
90.495970328999	102.78\\
90.514596421	102.76\\
90.530404885	102.75\\
90.545643968	102.74\\
90.560890960001	102.73\\
90.576166324	102.72\\
90.591944664	102.7\\
90.608007415	102.69\\
90.623988024	102.68\\
90.640121704999	102.67\\
90.656011277999	102.65\\
90.672007277	102.64\\
90.687954665	102.63\\
90.703994359	102.61\\
90.71998925	102.6\\
90.735941484999	102.59\\
90.752035932	102.58\\
90.768127222	102.56\\
90.784029033	102.55\\
90.799963733001	102.54\\
90.815980529	102.53\\
90.832320166999	102.51\\
90.848032568	102.5\\
90.866690477	102.49\\
90.881982824	102.47\\
90.897247522	102.46\\
90.912455253	102.45\\
90.927947600999	102.44\\
90.944085241	102.42\\
90.959892923	102.41\\
90.976006185	102.4\\
90.992042263	102.39\\
91.009733489999	102.39\\
91.025018021	102.4\\
91.040489142999	102.41\\
91.056112459	102.42\\
91.07202097	102.43\\
91.087952493	102.45\\
91.10394101	102.46\\
91.119997100999	102.47\\
91.135937651	102.49\\
91.151969639	102.5\\
91.168036039	102.51\\
91.184089256	102.52\\
91.19998842	102.54\\
91.216019609	102.55\\
91.234444659999	102.56\\
91.249566998	102.58\\
91.264876776999	102.59\\
91.2800697	102.6\\
91.296051893	102.61\\
91.312019402	102.62\\
91.327921758	102.64\\
91.343994286999	102.65\\
91.359902532	102.66\\
91.375930590999	102.68\\
91.39201264	102.69\\
91.408111244	102.7\\
91.424056929	102.71\\
91.440294109	102.73\\
91.45702503	102.74\\
91.472248622	102.75\\
91.488033308	102.77\\
91.503910599	102.78\\
91.520008608001	102.79\\
91.535958151999	102.8\\
91.552030222999	102.82\\
91.568019522	102.83\\
91.583928574	102.84\\
91.599982333	102.85\\
91.616006029998	102.87\\
91.632479968	102.88\\
91.647847168999	102.89\\
91.666311101	102.91\\
91.682042279	102.92\\
91.696985426	102.93\\
91.711914571	102.94\\
91.727858585	102.95\\
91.746053625	102.97\\
91.761141909	102.98\\
91.776187352	102.99\\
91.791923652999	103.01\\
91.807865708	103.02\\
91.82397861	103.03\\
91.842441154	103.05\\
91.857650314	103.06\\
91.873012027999	103.07\\
91.888295118	103.08\\
91.904003876999	103.09\\
91.920076143	103.11\\
91.935982353	103.12\\
91.95202613	103.13\\
91.968014176	103.15\\
91.984006921999	103.16\\
92.000009843	103.17\\
92.017437899	103.17\\
92.033112118	103.16\\
92.048395242	103.15\\
92.06464934	103.13\\
92.079916947	103.12\\
92.096003843	103.11\\
92.111931717001	103.1\\
92.127980600999	103.08\\
92.144029721	103.07\\
92.159909602	103.06\\
92.175935894	103.05\\
92.19202972	103.03\\
92.208017711	103.02\\
92.223999257	103.01\\
92.240013257	103\\
92.255953710001	102.98\\
92.271979882	102.97\\
92.287936331	102.96\\
92.30398491	102.94\\
92.319977856999	102.93\\
92.335981112	102.92\\
92.351895899	102.91\\
92.368012381	102.89\\
92.384008187998	102.88\\
92.400101402	102.87\\
92.416009732	102.86\\
92.432061393	102.84\\
92.448045024	102.83\\
92.464365868	102.82\\
92.479986227	102.8\\
92.496004079999	102.79\\
92.511828486	102.78\\
92.528022294	102.77\\
92.544005634	102.75\\
92.559943205	102.74\\
92.576046730999	102.73\\
92.592150850999	102.72\\
92.607983569	102.7\\
92.623999003	102.69\\
92.640058220998	102.68\\
92.655971404998	102.67\\
92.67203122	102.65\\
92.687949986	102.64\\
92.703928194	102.63\\
92.719997699	102.61\\
92.736009272999	102.6\\
92.75196236	102.59\\
92.768022989999	102.58\\
92.783878645999	102.56\\
92.799863951	102.55\\
92.816000919	102.54\\
92.832108585999	102.53\\
92.84808218	102.51\\
92.864280798	102.5\\
92.879986619	102.49\\
92.896231071999	102.48\\
92.912004678	102.46\\
92.928081524	102.45\\
92.943988064999	102.44\\
92.959987206	102.42\\
92.975950138	102.41\\
92.99190263	102.4\\
93.009749626	102.4\\
93.024962074	102.41\\
93.040294824	102.42\\
93.056041138	102.44\\
93.071961094	102.45\\
93.087975674001	102.46\\
93.104090888	102.47\\
93.119939274	102.49\\
93.135959762999	102.5\\
93.152012906	102.51\\
93.168031677	102.52\\
93.184039722999	102.54\\
93.200057757	102.55\\
93.216091537	102.56\\
93.232353704	102.57\\
93.247879466	102.59\\
93.266869529	102.6\\
93.279971716	102.61\\
93.296063994999	102.63\\
93.311990982	102.64\\
93.327928166	102.65\\
93.344012927999	102.66\\
93.360068793	102.68\\
93.375921937998	102.69\\
93.392076086	102.7\\
93.407973506	102.71\\
93.424028093	102.73\\
93.440057302	102.74\\
93.455855287	102.75\\
93.472039322	102.76\\
93.48800371	102.78\\
93.504105043	102.79\\
93.519959506	102.8\\
93.535973506	102.82\\
93.551979671999	102.83\\
93.567969703	102.84\\
93.584062326	102.85\\
93.599992919	102.87\\
93.616029919	102.88\\
93.632020616	102.89\\
93.648048633999	102.9\\
93.666684111	102.92\\
93.681964182	102.93\\
93.69727192	102.94\\
93.712556288	102.96\\
93.728060583	102.97\\
93.744159546999	102.98\\
93.759969689999	102.99\\
93.77606076	103.01\\
93.791971796	103.02\\
93.808039391001	103.03\\
93.82398459	103.04\\
93.839975357999	103.06\\
93.856021541	103.07\\
93.872063715	103.08\\
93.887977141001	103.09\\
93.90406947	103.11\\
93.919964757	103.12\\
93.935992872	103.13\\
93.952019008999	103.15\\
93.967921932001	103.16\\
93.984054355	103.17\\
94.000017114	103.18\\
94.017648806	103.18\\
94.033546873	103.17\\
94.048744712	103.16\\
94.064343331999	103.15\\
94.079896758	103.13\\
94.095986588	103.12\\
94.111950119999	103.11\\
94.128043929	103.1\\
94.144068013	103.08\\
94.160058397999	103.07\\
94.175983875001	103.06\\
94.191962713	103.05\\
94.208044765999	103.03\\
94.224036616	103.02\\
94.239857625	103.01\\
94.255982489999	103\\
94.272039512001	102.98\\
94.288038904	102.97\\
94.304046724	102.96\\
94.320020489999	102.94\\
94.336033784999	102.93\\
94.352000349	102.92\\
94.367997885999	102.91\\
94.384030267	102.89\\
94.399944281	102.88\\
94.416008504	102.87\\
94.432265349	102.86\\
94.448124845	102.84\\
94.466496371	102.83\\
94.481826378	102.82\\
94.497045392	102.8\\
94.512572468999	102.79\\
94.527992178	102.78\\
94.54400686	102.77\\
94.560154968	102.75\\
94.57594564	102.74\\
94.591971659999	102.73\\
94.608022041999	102.72\\
94.623963402	102.7\\
94.640050416999	102.69\\
94.655970259	102.68\\
94.671860336999	102.67\\
94.688021105	102.65\\
94.70404173	102.64\\
94.719950646999	102.63\\
94.735976318	102.61\\
94.751976757	102.6\\
94.768036435	102.59\\
94.784081460001	102.58\\
94.800021463	102.56\\
94.816120595	102.55\\
94.834692323999	102.54\\
94.849838956999	102.52\\
94.865149163	102.51\\
94.880198423	102.5\\
94.896114762	102.49\\
94.911966196	102.48\\
94.928005892	102.46\\
94.943906518	102.45\\
94.959969921	102.44\\
94.976542544	102.42\\
94.992047894	102.41\\
95.00986678	102.41\\
95.025048454	102.42\\
95.040297829999	102.44\\
95.056066027	102.45\\
95.072030255	102.46\\
95.087997423	102.47\\
95.103956142	102.49\\
95.119993123999	102.5\\
95.135975291999	102.51\\
95.152034986	102.52\\
95.167985877	102.54\\
95.18404518	102.55\\
95.200039685	102.56\\
95.215869077	102.57\\
95.234290019	102.59\\
95.249401973	102.6\\
95.264641003999	102.61\\
95.280010012999	102.63\\
95.296051716	102.64\\
95.311954097999	102.65\\
95.32801106	102.66\\
95.343937899	102.68\\
95.360004496	102.69\\
95.376017685	102.7\\
95.391977471	102.71\\
95.407978581999	102.73\\
95.424096272	102.74\\
95.440082704	102.75\\
95.455925626	102.76\\
95.471907506	102.78\\
95.487849986	102.79\\
95.504032177	102.8\\
95.520258428	102.82\\
95.535994767999	102.83\\
95.551980508	102.84\\
95.567871622	102.85\\
95.584055415	102.87\\
95.599928816	102.88\\
95.615998873001	102.89\\
95.632491958999	102.9\\
95.647991232	102.92\\
95.664287992	102.93\\
95.679947972999	102.94\\
95.696047928999	102.95\\
95.712018912	102.97\\
95.727990956	102.98\\
95.743992418	102.99\\
95.760059413	103.01\\
95.776229087	103.02\\
95.792125804	103.03\\
95.807991894	103.04\\
95.823976181	103.06\\
95.840031319	103.07\\
95.855944632	103.08\\
95.872015319	103.09\\
95.888005178999	103.11\\
95.904015616	103.12\\
95.919814368001	103.13\\
95.935999631	103.15\\
95.952015871	103.16\\
95.967938771	103.17\\
95.983935479999	103.18\\
95.999996952	103.2\\
96.017867702	103.19\\
96.033452824	103.18\\
96.048807312	103.17\\
96.064491476	103.15\\
96.080024821	103.14\\
96.095990351	103.13\\
96.112012544	103.12\\
96.128002545001	103.1\\
96.146266908999	103.09\\
96.161528925999	103.08\\
96.176810314999	103.07\\
96.192103079	103.05\\
96.207983135999	103.04\\
96.224032536	103.03\\
96.240056323	103.02\\
96.255830062998	103\\
96.272009277999	102.99\\
96.287963492	102.98\\
96.304061065	102.96\\
96.320095748	102.95\\
96.336011906	102.94\\
96.351979956999	102.93\\
96.368027442	102.91\\
96.383940487	102.9\\
96.399928324	102.89\\
96.416028734	102.88\\
96.432398411	102.86\\
96.447987114	102.85\\
96.464800984	102.84\\
96.480169842	102.82\\
96.496080060999	102.81\\
96.511888799001	102.8\\
96.527987372	102.79\\
96.544040997	102.77\\
96.560021404998	102.76\\
96.575987821999	102.75\\
96.591935938	102.74\\
96.607989923	102.72\\
96.623961111001	102.71\\
96.639959503	102.7\\
96.655982381	102.69\\
96.672026584	102.67\\
96.6879907	102.66\\
96.704033894001	102.65\\
96.72059371	102.63\\
96.736063535	102.62\\
96.751939484	102.61\\
96.767917877	102.6\\
96.783968635	102.58\\
96.800005515	102.57\\
96.816043498	102.56\\
96.832093781999	102.55\\
96.847873986999	102.53\\
96.86625839	102.52\\
96.881391303	102.51\\
96.896529411999	102.49\\
96.912033062	102.48\\
96.927957288001	102.47\\
96.944065499	102.46\\
96.960017618999	102.44\\
96.975997227999	102.43\\
96.992019662	102.42\\
97.007868237	102.42\\
97.024056295999	102.43\\
97.040039739	102.44\\
97.055932869999	102.46\\
97.071997559	102.47\\
97.087982589	102.48\\
97.104016305	102.49\\
97.119933533999	102.51\\
97.136043927	102.52\\
97.152045007999	102.53\\
97.168100274	102.54\\
97.184044638	102.56\\
97.200001854	102.57\\
97.215993419	102.58\\
97.234560867	102.6\\
97.249834166999	102.61\\
97.265100246	102.62\\
97.280498003999	102.63\\
97.296023757	102.65\\
97.311892335	102.66\\
97.328084661999	102.67\\
97.343906085001	102.68\\
97.360049899	102.7\\
97.376036095	102.71\\
97.39195868	102.72\\
97.407954251	102.73\\
97.423956284	102.75\\
97.440026556	102.76\\
97.456053425999	102.77\\
97.472056781	102.78\\
97.488024586	102.8\\
97.503910421	102.81\\
97.520006231	102.82\\
97.536099009	102.84\\
97.55203857	102.85\\
97.567987897999	102.86\\
97.583944748	102.87\\
97.602030284999	102.89\\
97.61715975	102.9\\
97.632443719	102.91\\
97.647939329	102.92\\
97.664056077	102.94\\
97.680058036999	102.95\\
97.696067248	102.96\\
97.711973104999	102.98\\
97.727965168	102.99\\
97.744023438	103\\
97.760022684001	103.01\\
97.775987187	103.03\\
97.792011577	103.04\\
97.808011518	103.05\\
97.823976581999	103.06\\
97.840130896999	103.08\\
97.856007317999	103.09\\
97.872162838	103.1\\
97.887945221	103.11\\
97.904063374	103.13\\
97.920059727	103.14\\
97.935992324	103.15\\
97.952035729	103.17\\
97.968118286	103.18\\
97.984035581	103.19\\
98.000031585	103.2\\
98.017666089	103.2\\
98.032723305	103.19\\
98.048054993	103.18\\
98.06663059	103.17\\
98.081722531999	103.15\\
98.096936816	103.14\\
98.112299553999	103.13\\
98.127875022999	103.12\\
98.144051729	103.1\\
98.160066975999	103.09\\
98.175979638	103.08\\
98.191988374	103.07\\
98.208056301	103.05\\
98.223846529	103.04\\
98.239993779	103.03\\
98.255931417	103.02\\
98.2719997	103\\
98.288308909	102.99\\
98.304015998	102.98\\
98.319985647999	102.97\\
98.335981486001	102.95\\
98.352046632	102.94\\
98.367989581	102.93\\
98.384174282	102.91\\
98.400013967	102.9\\
98.416023823	102.89\\
98.432442053999	102.88\\
98.44807067	102.86\\
98.464101409999	102.85\\
98.479988871	102.84\\
98.496009079	102.83\\
98.511816331999	102.81\\
98.528015135	102.8\\
98.544047551	102.79\\
98.560005954999	102.77\\
98.575946779998	102.76\\
98.591991594	102.75\\
98.608011987999	102.74\\
98.623912793	102.72\\
98.640197728	102.71\\
98.656016894001	102.7\\
98.672097188	102.69\\
98.687953038999	102.67\\
98.70646642	102.66\\
98.721712029	102.65\\
98.741051342	102.63\\
98.753566464	102.62\\
98.768931292	102.61\\
98.784603548	102.6\\
98.799984727999	102.58\\
98.815999494999	102.57\\
98.832055245	102.56\\
98.847940647999	102.55\\
98.866455622	102.53\\
98.881708001	102.52\\
98.896844666	102.51\\
98.912102797	102.5\\
98.928481539	102.48\\
98.944054049001	102.47\\
98.960259446	102.46\\
98.976105980999	102.44\\
98.992144312998	102.43\\
99.009619665999	102.43\\
99.024945885	102.44\\
99.040088468999	102.46\\
99.056005339	102.47\\
99.072189524999	102.48\\
99.088323256	102.49\\
99.104134600999	102.51\\
99.119949187	102.52\\
99.135951948999	102.53\\
99.151962687	102.54\\
99.167980152999	102.56\\
99.184027262999	102.57\\
99.200077345	102.58\\
99.216120670999	102.59\\
99.232510277	102.61\\
99.248004213	102.62\\
99.264196206999	102.63\\
99.279960137	102.65\\
99.295971696	102.66\\
99.311959283	102.67\\
99.328068636	102.68\\
99.343999031	102.7\\
99.359981303	102.71\\
99.37599171	102.72\\
99.392036556	102.73\\
99.407958758	102.75\\
99.423981892	102.76\\
99.440111718999	102.77\\
99.455891024	102.79\\
99.472030676	102.8\\
99.488007727	102.81\\
99.504010807	102.82\\
99.520087875	102.84\\
99.536015529	102.85\\
99.552052661	102.86\\
99.568012126999	102.87\\
99.584030652999	102.89\\
99.600066095	102.9\\
99.615924347	102.91\\
99.634291027999	102.93\\
99.649345666	102.94\\
99.664512538	102.95\\
};
\end{axis}
\end{tikzpicture}%
}
      \caption{The steady state orientation of the robot for
        $K_{\Psi}^R = K_{\Psi, max}^R$}
      \label{fig:8_max_magnified}
    \end{figure}
  \end{minipage}
\end{minipage}
}

\noindent\makebox[\textwidth][c]{%
\begin{minipage}{\linewidth}
  \begin{minipage}{0.45\linewidth}
    \begin{figure}[H]
      \scalebox{0.6}{% This file was created by matlab2tikz.
%
%The latest updates can be retrieved from
%  http://www.mathworks.com/matlabcentral/fileexchange/22022-matlab2tikz-matlab2tikz
%where you can also make suggestions and rate matlab2tikz.
%
\definecolor{mycolor1}{rgb}{0.00000,0.44700,0.74100}%
%
\begin{tikzpicture}

\begin{axis}[%
width=4.133in,
height=3.26in,
at={(0.693in,0.44in)},
scale only axis,
xmin=0,
xmax=25,
xmajorgrids,
ymin=-200,
ymax=200,
ymajorgrids,
axis background/.style={fill=white}
]
\addplot [color=mycolor1,solid,forget plot]
  table[row sep=crcr]{%
0	0\\
0.0160502309999987	2.2\\
0.032189881000001	5.81\\
0.0482074939990006	9.54\\
0.0641874769989997	13.13\\
0.0800688759989999	16.66\\
0.0961943489999995	20.29\\
0.112198871999998	23.9\\
0.128054989	27.44\\
0.144058223999999	30.99\\
0.160049797998998	34.6\\
0.176056485999998	38.22\\
0.192204634999998	41.89\\
0.208202482999999	45.55\\
0.224389566	49.13\\
0.240183300999001	52.77\\
0.256107296999998	56.26\\
0.272072601	59.86\\
0.288039945	63.44\\
0.304200015999999	67.04\\
0.320181542999998	70.75\\
0.336218059	74.36\\
0.352172609000001	77.94\\
0.368222324000001	81.57\\
0.384171798	85.11\\
0.400286976999998	88.78\\
0.416212701999001	92.39\\
0.432199599998999	95.97\\
0.449536763999999	99.97\\
0.465034974998999	103.46\\
0.480592094	106.97\\
0.496244509	110.45\\
0.512191807999	113.99\\
0.52827552	117.59\\
0.544184606998999	121.2\\
0.560206013999999	124.76\\
0.576200158999998	128.38\\
0.592105256999999	131.93\\
0.608077342	135.52\\
0.624074228	139.13\\
0.640028246998999	142.73\\
0.656170015999999	146.35\\
0.672144206	149.99\\
0.688168127999999	153.55\\
0.704205530999999	157.22\\
0.720124936000001	160.77\\
0.736230324000001	164.34\\
0.752183835999001	168.03\\
0.768392701999001	171.57\\
0.784207204999998	175.21\\
0.800072638	178.78\\
0.816077066	-177.7\\
0.832030568999998	-174.09\\
0.848183267999001	-170.4\\
0.864200754999999	-166.82\\
0.880073279998999	-163.24\\
0.896057531999998	-159.69\\
0.912194635998998	-156.09\\
0.928064254000001	-152.43\\
0.944048515999999	-148.87\\
0.960238007999998	-145.22\\
0.976228921	-141.54\\
0.992194946998	-137.94\\
1.011447444	-133.14\\
1.024138328	-129.01\\
1.040248934	-123.97\\
1.056061496999	-118.87\\
1.072068876	-113.91\\
1.088049740999	-108.79\\
1.104025293999	-103.77\\
1.120029314	-98.71\\
1.136085376	-93.62\\
1.154351793	-87.53\\
1.169398515	-82.76\\
1.184436467	-78\\
1.200102598	-73.18\\
1.216073825	-68.25\\
1.234393132	-62.16\\
1.249689775001	-57.3\\
1.264809543	-52.51\\
1.280024551	-47.71\\
1.296195566999	-42.83\\
1.312233857	-37.6\\
1.328210392	-32.59\\
1.344201378999	-27.52\\
1.360088701999	-22.49\\
1.376053891	-17.45\\
1.392038745	-12.39\\
1.408159356	-7.33\\
1.424063121	-2.19\\
1.441313224	3.46\\
1.456444793	8.25\\
1.472135705	13.17\\
1.488172066	18.12\\
1.504202277	23.24\\
1.520255991999	28.33\\
1.53629597	33.38\\
1.552205142999	38.46\\
1.568114495	43.48\\
1.584065482	48.48\\
1.600042267999	53.5\\
1.616055727	58.58\\
1.632203458999	63.73\\
1.648245533	68.89\\
1.664177717	73.96\\
1.680048001	78.92\\
1.696040331999	83.91\\
1.712179119	89.08\\
1.728234557999	94.25\\
1.744192571	99.36\\
1.760169130998	104.41\\
1.776176686	109.49\\
1.792205875	114.58\\
1.808178557	119.57\\
1.824220836	124.72\\
1.840200688	129.74\\
1.856052003999	134.73\\
1.872072132999	139.75\\
1.888033764	144.82\\
1.904200740999	150\\
1.920199711	155.17\\
1.936161944	160.22\\
1.95220293	165.22\\
1.96814452	170.36\\
1.984201127	175.42\\
2.000062595999	-179.61\\
2.017321212	-173.89\\
2.032375521	-169.11\\
2.048051771	-164.3\\
2.064066255998	-159.3\\
2.080057674999	-154.21\\
2.09608714	-149.13\\
2.112090915999	-144.07\\
2.128059483	-139.04\\
2.144057803	-133.99\\
2.160041509	-128.93\\
2.176055554	-123.86\\
2.192057019999	-118.77\\
2.208024529999	-113.71\\
2.224027223	-108.65\\
2.240059645	-103.56\\
2.256038676	-98.49\\
2.272053953	-93.42\\
2.288064916	-88.34\\
2.304050727	-83.27\\
2.320045328	-78.15\\
2.336045802	-73.12\\
2.352064227	-68.04\\
2.368044960999	-62.97\\
2.384055734	-57.9\\
2.400058312999	-52.84\\
2.416189352999	-47.65\\
2.43220212	-42.49\\
2.44814071	-37.28\\
2.464191130998	-32.36\\
2.480092871	-27.34\\
2.496038056	-22.36\\
2.512047665	-17.31\\
2.528022561	-12.17\\
2.54401191	-7.16\\
2.560172262	-2.01\\
2.576200038	3.17\\
2.592197936	8.14\\
2.608193955	13.32\\
2.624372859	18.39\\
2.640228766	23.49\\
2.656309111	28.57\\
2.672197661	33.6\\
2.688217118	38.72\\
2.704186292	43.76\\
2.720101446998	48.75\\
2.736172702	53.86\\
2.752184257	58.96\\
2.768194529999	63.95\\
2.784095302	69\\
2.800057125	74\\
2.816202654999	79.19\\
2.832074142	84.27\\
2.848060187999	89.25\\
2.864058085999	94.31\\
2.880077509	99.39\\
2.896051798	104.46\\
2.912076297	109.55\\
2.928074987	114.61\\
2.944048625999	119.69\\
2.960026764	124.75\\
2.976065014	129.8\\
2.992215187	135.02\\
3.011163029	138.68\\
3.024123885	137.77\\
3.040197271001	136.62\\
3.056181734	135.45\\
3.072335083999	134.29\\
3.088141879	133.18\\
3.104039484	132.04\\
3.12019242	130.92\\
3.136210209001	129.73\\
3.152188575	128.61\\
3.168187585	127.46\\
3.183989817	126.34\\
3.200188602	125.22\\
3.216207588	124.04\\
3.232084828	122.91\\
3.248178017	121.79\\
3.264170007	120.61\\
3.280206643	119.47\\
3.296172851	118.34\\
3.312083661	117.22\\
3.328079885	116.09\\
3.344192217999	114.94\\
3.360249901999	113.77\\
3.376344082999	112.63\\
3.392051533	111.49\\
3.408040737	110.38\\
3.424198778001	109.21\\
3.440630421	107.99\\
3.456172804	106.88\\
3.472171219999	105.78\\
3.488199557	104.63\\
3.504156628	103.49\\
3.520240831999	102.35\\
3.536214585999	101.2\\
3.552177926	100.08\\
3.568185144999	98.92\\
3.584181821	97.78\\
3.600184891	96.65\\
3.61620784	95.5\\
3.632216403	94.36\\
3.648217639	93.21\\
3.664169573	92.08\\
3.680346783	90.96\\
3.696195913998	89.78\\
3.712120085	88.66\\
3.728106953	87.53\\
3.744127227999	86.41\\
3.760019527	85.26\\
3.776176811	84.1\\
3.79217933	82.94\\
3.808177257999	81.8\\
3.824198729999	80.67\\
3.840211290999	79.51\\
3.856200875999	78.38\\
3.872201269999	77.23\\
3.888215295	76.09\\
3.904199552	74.97\\
3.920206736	73.81\\
3.936199994	72.68\\
3.95207986	71.56\\
3.96823063	70.39\\
3.98423594	69.24\\
4.000065397	68.13\\
4.017511548	68.12\\
4.032905632	69.27\\
4.048590156	70.45\\
4.064202901	71.63\\
4.080140657	72.8\\
4.096276377999	73.97\\
4.111928095999	75.2\\
4.130028416	76.59\\
4.144911133	77.71\\
4.160162387	78.86\\
4.176198898	80.01\\
4.192067210999	81.24\\
4.208061493	82.43\\
4.224310189	83.63\\
4.240208160999	84.86\\
4.256178897999	86.05\\
4.272109128	87.23\\
4.288217186	88.45\\
4.304218367001	89.67\\
4.320163465	90.87\\
4.336362435	92.07\\
4.352241418	93.3\\
4.368090628999	94.48\\
4.384355869	95.68\\
4.400293790999	96.91\\
4.416209854	98.11\\
4.432168734	99.31\\
4.448430733	100.57\\
4.464082522999	101.72\\
4.480218861	102.91\\
4.496182057	104.13\\
4.512167982	105.33\\
4.528045107	106.52\\
4.544387300999	107.72\\
4.560197088	108.96\\
4.576209057999	110.16\\
4.592158456	111.35\\
4.608067047	112.53\\
4.624163743999	113.74\\
4.640190353	114.98\\
4.65622995	116.17\\
4.672065922	117.36\\
4.688083540001	118.55\\
4.704193552	119.78\\
4.720093409	120.98\\
4.736078187	122.17\\
4.752063963	123.37\\
4.768057311	124.57\\
4.784041327	125.79\\
4.800053673999	126.99\\
4.816038708	128.18\\
4.8321268	129.41\\
4.848132691999	130.63\\
4.864024861	131.83\\
4.88009619	133.01\\
4.896079157	134.22\\
4.912080788	135.42\\
4.928044774	136.62\\
4.944057595	137.83\\
4.960195029999	139.06\\
4.976225191	140.28\\
4.992204236999	141.48\\
5.011122752	141.91\\
5.023963755998	140.87\\
5.039987224999	139.5\\
5.056065745	138.15\\
5.072051871999	136.8\\
5.088069694999	135.44\\
5.104180164	134.1\\
5.120279286999	132.7\\
5.136202821	131.34\\
5.152193818999	129.99\\
5.168207283	128.64\\
5.184172748	127.3\\
5.200071250999	125.98\\
5.216094731999	124.64\\
5.232547115	123.26\\
5.248147158	121.85\\
5.264170111999	120.56\\
5.280213708999	119.19\\
5.296037277	117.86\\
5.31218278	116.5\\
5.328218663998	115.14\\
5.34405875	113.8\\
5.360019503999	112.48\\
5.37621269	111.1\\
5.392202355	109.72\\
5.408173647	108.39\\
5.424227803	107.03\\
5.440062207	105.65\\
5.456140868	104.35\\
5.472183006	102.98\\
5.488203601	101.63\\
5.504198608	100.27\\
5.520195077	98.92\\
5.536238864999	97.58\\
5.552233375999	96.22\\
5.568202393	94.87\\
5.584212575	93.52\\
5.600200641	92.17\\
5.616174585	90.83\\
5.632194894	89.47\\
5.648259318	88.12\\
5.664216251	86.77\\
5.680097260999	85.41\\
5.696201331	84.06\\
5.712094166	82.74\\
5.728085423999	81.41\\
5.744081669	80.06\\
5.760025114	78.71\\
5.776174129	77.32\\
5.792225203	75.96\\
5.808211936001	74.61\\
5.824097755	73.29\\
5.840235616	71.93\\
5.856343158	70.54\\
5.872062359999	69.21\\
5.888073116999	67.9\\
5.90401117	66.54\\
5.920200863	65.17\\
5.936152822	63.81\\
5.952263	62.45\\
5.968244	61.1\\
5.984270102	59.76\\
6.000204777	58.41\\
6.017690639001	58.48\\
6.032897669	59.96\\
6.048492233	61.48\\
6.064232612	62.99\\
6.080220708	64.53\\
6.096191495	66.09\\
6.112273503999	67.65\\
6.128072682	69.18\\
6.144083169	70.73\\
6.160261481	72.31\\
6.176221333	73.9\\
6.192331125	75.46\\
6.208173541	76.99\\
6.224247547	78.57\\
6.240132057999	80.13\\
6.256207953	81.68\\
6.272181436	83.23\\
6.288075314	84.77\\
6.304056654	86.32\\
6.320042465	87.87\\
6.336057147	89.43\\
6.352053336	91.01\\
6.3681685	92.58\\
6.384216662	94.18\\
6.400198451999	95.74\\
6.416292901	97.29\\
6.432231861	98.86\\
6.449198065	100.55\\
6.464930552	102.09\\
6.480371096	103.59\\
6.496227689	105.1\\
6.512176772	106.65\\
6.528160517999	108.17\\
6.544066272	109.73\\
6.560199673	111.32\\
6.576187654	112.89\\
6.592183964	114.46\\
6.608178915	116.01\\
6.624198242001	117.57\\
6.640190578	119.1\\
6.656058762	120.66\\
6.672063656	122.2\\
6.688265973	123.8\\
6.704184366	125.39\\
6.720109649999	126.92\\
6.736271093	128.48\\
6.752194438	130.06\\
6.768168026	131.62\\
6.784063733	133.14\\
6.800198361	134.68\\
6.81620981	136.3\\
6.832163548	137.85\\
6.848070871	139.38\\
6.864051295	140.92\\
6.880069735	142.48\\
6.896047533	144.05\\
6.912252014	145.64\\
6.928208263	147.23\\
6.944217106	148.78\\
6.960207687999	150.33\\
6.976083885	151.87\\
6.992064575	153.41\\
7.010907071998	153.97\\
7.026014892	152.2\\
7.041188361	150.52\\
7.056565142	148.82\\
7.072042022	147.13\\
7.088051312	145.41\\
7.103998817	143.63\\
7.120053062999	141.85\\
7.136071698001	140.06\\
7.152279482	138.27\\
7.168240321998	136.47\\
7.184289777	134.71\\
7.200200252	132.91\\
7.216218903	131.15\\
7.23221616	129.37\\
7.248225354	127.6\\
7.26406793	125.86\\
7.280062345	124.11\\
7.296065881	122.33\\
7.312043734	120.55\\
7.328070718	118.78\\
7.344026797	117.01\\
7.36003125	115.23\\
7.376111896	113.43\\
7.392057561	111.64\\
7.408114347	109.86\\
7.424174747	108.07\\
7.440856414	106.16\\
7.456366357	104.45\\
7.472218726	102.73\\
7.488177861999	100.97\\
7.504057262	99.23\\
7.52005313	97.47\\
7.536121976	95.7\\
7.552226368	93.9\\
7.568073561	92.11\\
7.584065135	90.37\\
7.599996315	88.6\\
7.615988854999	86.83\\
7.632064242001	85.05\\
7.648022856999	83.27\\
7.664092788	81.49\\
7.680003599	79.67\\
7.696064001	77.92\\
7.712051217999	76.13\\
7.728194496	74.35\\
7.744079331999	72.59\\
7.760069017	70.84\\
7.776050503	69.06\\
7.792053128	67.29\\
7.808058212	65.51\\
7.824057689	63.73\\
7.840061345	61.96\\
7.856183956999	60.18\\
7.872199465	58.34\\
7.888169179	56.57\\
7.904062295	54.84\\
7.920058538998	53.07\\
7.936057288998	51.31\\
7.952194217999	49.52\\
7.968207191	47.68\\
7.984173201	45.92\\
8.000064673999	44.18\\
8.017801619	44.27\\
8.032909965999	46.21\\
8.048999663	48.26\\
8.064392559	50.23\\
8.080242773	52.22\\
8.096192486999	54.25\\
8.112359939	56.29\\
8.128227257	58.37\\
8.144189394	60.38\\
8.16023459	62.45\\
8.176301213999	64.48\\
8.192075069	66.5\\
8.208060689	68.51\\
8.224085368	70.57\\
8.24004504	72.62\\
8.256016533999	74.66\\
8.272191238	76.75\\
8.288203969999	78.8\\
8.304207912	80.84\\
8.320083292	82.87\\
8.336053262	84.9\\
8.352168918	86.99\\
8.368195054999	89.07\\
8.384192051	91.12\\
8.400073982	93.12\\
8.416065576	95.14\\
8.432044715999	97.19\\
8.448396532	99.4\\
8.464163010999	101.34\\
8.480230386	103.4\\
8.496192173	105.46\\
8.512276010999	107.51\\
8.528216745	109.54\\
8.544099892999	111.56\\
8.560063479	113.59\\
8.576061597	115.63\\
8.592223217	117.71\\
8.608067166	119.75\\
8.624250378999	121.82\\
8.640274893	123.9\\
8.656248340999	125.95\\
8.67219566	127.98\\
8.688204633	130.04\\
8.704177921	132.06\\
8.720178866	134.13\\
8.736211583	136.18\\
8.752198130998	138.22\\
8.768206002001	140.29\\
8.784170364	142.31\\
8.80008135	144.34\\
8.816215647999	146.37\\
8.83203444	148.44\\
8.848045873999	150.45\\
8.864045617001	152.5\\
8.88004996	154.56\\
8.896015729999	156.59\\
8.912045977	158.67\\
8.928027958999	160.7\\
8.944171337001	162.79\\
8.960174010999	164.85\\
8.976161461	166.9\\
8.99216425	168.96\\
9.011240382	169.64\\
9.024119802999	167.8\\
9.04003875	165.48\\
9.056200765	163.16\\
9.072211342	160.85\\
9.088223998999	158.55\\
9.104188276	156.17\\
9.120225248	153.91\\
9.136176711	151.66\\
9.152074055999	149.38\\
9.168206119	147.06\\
9.184214606	144.72\\
9.200112381	142.44\\
9.216185121	140.17\\
9.232167434	137.81\\
9.248131158999	135.53\\
9.264207555	133.25\\
9.280256906	130.92\\
9.296125104999	128.66\\
9.312063844	126.38\\
9.328069158999	124.09\\
9.344054609	121.75\\
9.360061092	119.46\\
9.376054064	117.17\\
9.392059239	114.88\\
9.408058038998	112.57\\
9.424051137	110.28\\
9.440693465999	107.75\\
9.456202436	105.52\\
9.472198711	103.3\\
9.488200307	100.99\\
9.504073071998	98.71\\
9.520045758	96.46\\
9.536045509999	94.17\\
9.552039059	91.87\\
9.568046498999	89.57\\
9.584233391	87.23\\
9.600150760999	84.86\\
9.61617318	82.58\\
9.632197849	80.28\\
9.648304875	77.95\\
9.664068266	75.69\\
9.680216656	73.4\\
9.696191547	71.04\\
9.712249406	68.79\\
9.728171272	66.47\\
9.744215786	64.15\\
9.760163366	61.86\\
9.776035819999	59.61\\
9.792038409	57.34\\
9.808197262	55\\
9.824206150001	52.63\\
9.840169128999	50.35\\
9.856071109999	48.09\\
9.872068522	45.81\\
9.888069041	43.51\\
9.904192675	41.16\\
9.920179958999	38.83\\
9.936068996	36.58\\
9.952075146999	34.3\\
9.968202288998	32\\
9.984240878999	29.6\\
10.000121215999	27.32\\
10.017423517	27.44\\
10.032619798	29.95\\
10.048402804	32.56\\
10.064093871	35.14\\
10.080050055	37.67\\
10.096069918	40.31\\
10.112062515	42.97\\
10.128105259	45.61\\
10.144179636	48.42\\
10.160213861999	50.98\\
10.176264159	53.67\\
10.192168601	56.32\\
10.208083876	58.88\\
10.224206016	61.59\\
10.240097900001	64.23\\
10.256069112	66.83\\
10.272054127	69.48\\
10.288044448001	72.13\\
10.304250001	74.82\\
10.320224687	77.54\\
10.336201864	80.17\\
10.352111826999	82.82\\
10.368361927	85.44\\
10.384169269	88.11\\
10.400352718	90.79\\
10.416203302	93.46\\
10.432293413998	96.07\\
10.448542396999	98.86\\
10.464180755	101.4\\
10.480143759999	104\\
10.496170142999	106.65\\
10.512226564	109.34\\
10.528131234999	111.97\\
10.544187987	114.56\\
10.560036652	117.22\\
10.576180210999	119.9\\
10.592181291	122.57\\
10.608088018	125.17\\
10.624063981	127.8\\
10.640054037	130.44\\
10.656062609999	133.1\\
10.672056578999	135.74\\
10.688077182999	138.39\\
10.704028787	141.08\\
10.720214956999	143.76\\
10.736432227999	146.45\\
10.751967607	149.12\\
10.767917547	151.63\\
10.78614716	154.8\\
10.801431686	157.33\\
10.816738663	159.87\\
10.832029759	162.38\\
10.848072669	164.91\\
10.86402916	167.56\\
10.880071533	170.21\\
10.896036892	172.86\\
10.912063059999	175.51\\
10.928040607	178.17\\
10.944054632	-179.19\\
10.960057205999	-176.52\\
10.976049075	-173.89\\
10.992053243	-171.23\\
11.010689106	-167.17\\
11.025910680999	-162.17\\
11.041531762	-157.23\\
11.056965904	-152.32\\
11.072446363999	-147.42\\
11.088208443	-142.52\\
11.104147979	-137.51\\
11.120296083	-132.46\\
11.136171403	-127.37\\
11.152257622	-122.29\\
11.168193765	-117.25\\
11.184243505998	-112.12\\
11.200219719999	-107.04\\
11.216076042	-102.09\\
11.232199856	-96.97\\
11.248132472	-91.82\\
11.264308719	-86.81\\
11.280222161	-81.66\\
11.296235384	-76.58\\
11.312199430999	-71.59\\
11.328204375	-66.44\\
11.344085814	-61.5\\
11.360090748999	-56.43\\
11.376056404	-51.43\\
11.392221982	-46.28\\
11.408200878	-41.09\\
11.424201552	-36.04\\
11.440138309	-30.81\\
11.456163585999	-25.92\\
11.472215317	-20.79\\
11.488166715	-15.73\\
11.504102337	-10.78\\
11.520039673999	-5.68\\
11.536195399999	-0.58\\
11.552186527	4.57\\
11.568200647999	9.66\\
11.5842532	14.74\\
11.600220168	19.83\\
11.616210519	24.86\\
11.632133638	29.86\\
11.648075991	34.86\\
11.664071197	39.91\\
11.680053660999	44.98\\
11.696052340999	50.02\\
11.712169465	55.2\\
11.728239345999	60.39\\
11.744213941999	65.47\\
11.760078107	70.44\\
11.776056175	75.44\\
11.792044752	80.53\\
11.808056661999	85.61\\
11.824061512	90.71\\
11.840037821	95.75\\
11.856072010999	100.78\\
11.872001368999	105.86\\
11.88806216	110.91\\
11.904045510999	116\\
11.920063174	121.07\\
11.936196078	126.15\\
11.952198627	131.42\\
11.968192109999	136.4\\
11.984215420999	141.56\\
12.000172236	146.61\\
12.018154123	147.33\\
12.033565552	145.85\\
12.048843848	144.39\\
12.064036135	142.95\\
12.080030224999	141.48\\
12.09605791	139.95\\
12.112059983	138.42\\
12.128172090999	136.87\\
12.144188315	135.33\\
12.160055136	133.83\\
12.176048078999	132.3\\
12.192196517999	130.77\\
12.208242585999	129.21\\
12.224174057999	127.68\\
12.240170463	126.17\\
12.256140394	124.64\\
12.272061542	123.14\\
12.288058153	121.59\\
12.304048574	120.09\\
12.320180260999	118.54\\
12.336333682	116.99\\
12.352065807	115.47\\
12.368046346	113.99\\
12.384178984999	112.46\\
12.40019969	110.88\\
12.416268189	109.36\\
12.432044951	107.84\\
12.449441384	106.12\\
12.46454271	104.68\\
12.480034689	103.24\\
12.496062473	101.76\\
12.512049577	100.23\\
12.528198802999	98.69\\
12.544197699	97.12\\
12.560238218	95.58\\
12.576201894	94.06\\
12.592310579	92.52\\
12.608247203	91.01\\
12.624066500999	89.52\\
12.640193713	87.97\\
12.656200441001	86.43\\
12.672043734	84.93\\
12.688008316999	83.42\\
12.70400818	81.89\\
12.720193979	80.35\\
12.736217080001	78.8\\
12.752114856	77.27\\
12.768174944999	75.76\\
12.784188248	74.19\\
12.800184974	72.66\\
12.816197651	71.16\\
12.832174606	69.6\\
12.848285606	68.09\\
12.864065455	66.58\\
12.880068285001	65.07\\
12.896205373999	63.51\\
12.91226195	61.97\\
12.928370254	60.44\\
12.944081096	58.92\\
12.960075465999	57.43\\
12.976200708999	55.87\\
12.992102634	54.33\\
13.011479267	53.82\\
13.024004828	55.19\\
13.040198114999	56.88\\
13.056235232	58.63\\
13.072176397	60.33\\
13.088203330999	62.02\\
13.104191923	63.73\\
13.120142564	65.4\\
13.136063914	67.07\\
13.152200962999	68.77\\
13.168201778	70.53\\
13.184166852	72.21\\
13.200334113999	73.87\\
13.216256956999	75.62\\
13.232311753999	77.33\\
13.248205015	79\\
13.264206842	80.72\\
13.280183653001	82.42\\
13.296291510999	84.11\\
13.312186951999	85.82\\
13.328212493	87.48\\
13.344059068	89.17\\
13.360086102	90.86\\
13.376183536	92.59\\
13.392291389	94.32\\
13.408175864999	96.02\\
13.424076232	97.68\\
13.440081362	99.36\\
13.456201763	101.09\\
13.472127212	102.82\\
13.488195425	104.48\\
13.504205726	106.21\\
13.520221781	107.91\\
13.536079871999	109.58\\
13.552057373999	111.25\\
13.568070136	112.97\\
13.584207064	114.71\\
13.600145193001	116.41\\
13.616065746	118.06\\
13.632049657	119.74\\
13.648053493999	121.45\\
13.664091272999	123.15\\
13.680424165001	124.85\\
13.696055242001	126.6\\
13.712085884	128.26\\
13.728101583	129.96\\
13.74401602	131.66\\
13.76008868	133.34\\
13.77628583	135.08\\
13.792182554001	136.82\\
13.808209167999	138.51\\
13.824205456999	140.2\\
13.84017889	141.91\\
13.856200427	143.59\\
13.872299038	145.31\\
13.888264108	147.03\\
13.904201038	148.71\\
13.920196910999	150.41\\
13.936196564	152.1\\
13.952191911	153.79\\
13.968230336	155.51\\
13.984342724	157.21\\
14.00019844	158.9\\
14.017980097	158.81\\
14.033568953	156.91\\
14.04899266	155.02\\
14.064488425	153.13\\
14.080233315	151.24\\
14.09634674	149.34\\
14.112142355	147.34\\
14.128205349999	145.45\\
14.144198312	143.45\\
14.16020234	141.5\\
14.176077575	139.55\\
14.192200151	137.65\\
14.20820944	135.62\\
14.224303996999	133.7\\
14.240201438	131.76\\
14.256137255	129.78\\
14.27216634	127.85\\
14.288082513	125.91\\
14.304050528	123.98\\
14.32017785	122\\
14.336220745	120.02\\
14.352158366999	118.08\\
14.368074074	116.16\\
14.384216048999	114.18\\
14.400312145	112.19\\
14.416122864	110.24\\
14.432054646999	108.35\\
14.448208970999	106.41\\
14.46414598	104.39\\
14.480058489999	102.49\\
14.496058218	100.55\\
14.512018359999	98.6\\
14.52805236	96.65\\
14.544006109	94.7\\
14.560052066	92.74\\
14.576232027	90.74\\
14.592240845	88.76\\
14.608205868999	86.79\\
14.624166878999	84.86\\
14.640188461	82.9\\
14.656271401999	80.95\\
14.672199215999	79\\
14.688194564	77.03\\
14.704196059999	75.08\\
14.720188972	73.15\\
14.736192543	71.18\\
14.752187304	69.24\\
14.768056396	67.32\\
14.784172332	65.34\\
14.800210784	63.37\\
14.816054904	61.44\\
14.832197873999	59.48\\
14.848344349	57.51\\
14.86431707	55.57\\
14.880222875	53.58\\
14.896335663	51.68\\
14.91220727	49.67\\
14.928181645	47.75\\
14.944043069	45.82\\
14.960250884	43.86\\
14.976830160001	41.88\\
14.992423184999	39.84\\
15.011545552	39.19\\
15.024061345999	41.01\\
15.039992608	43.19\\
15.058249522999	45.84\\
15.073361202	47.94\\
15.08841264	50.04\\
15.104067891	52.13\\
15.120073536999	54.31\\
15.13609558	56.54\\
15.152058825	58.77\\
15.168052047	60.98\\
15.184046769	63.23\\
15.200010114	65.44\\
15.216101634	67.66\\
15.232051759	69.92\\
15.248004580999	72.12\\
15.264042715	74.33\\
15.280050746	76.57\\
15.296057264	78.79\\
15.312089791	81.02\\
15.328052559	83.27\\
15.344081032	85.51\\
15.360170699	87.74\\
15.37606421	89.95\\
15.392046267	92.17\\
15.408050861	94.39\\
15.424060019999	96.63\\
15.440093569	98.84\\
15.456025704	101.05\\
15.472176664	103.29\\
15.488209275001	105.59\\
15.50416773	107.8\\
15.520061071	109.98\\
15.536050399999	112.19\\
15.552055193999	114.41\\
15.568061421	116.64\\
15.584046751	118.86\\
15.600041101	121.09\\
15.616057451	123.31\\
15.632180307	125.58\\
15.648065971	127.8\\
15.664054939	130\\
15.680050177	132.22\\
15.69605917	134.44\\
15.71213486	136.68\\
15.728076019	138.91\\
15.744054835	141.16\\
15.760083227	143.38\\
15.776063069	145.61\\
15.792170481	147.87\\
15.808223795	150.11\\
15.824120859999	152.29\\
15.840061182999	154.52\\
15.856052451	156.73\\
15.872205235	158.99\\
15.888244417	161.24\\
15.904178805999	163.46\\
15.920200098	165.7\\
15.936229608	167.91\\
15.952057406	170.12\\
15.96832958	172.35\\
15.984249998	174.62\\
16.000216215	176.83\\
16.017837243	176.69\\
16.032933627001	174.25\\
16.048317474	171.76\\
16.06394637	169.3\\
16.079966133	166.81\\
16.095925208	164.21\\
16.111911356	161.62\\
16.127940194	159.03\\
16.144063053	156.37\\
16.160051948	153.82\\
16.176045578999	151.2\\
16.192036409	148.58\\
16.20805543	146.01\\
16.22418847	143.42\\
16.240185866999	140.81\\
16.256209609	138.26\\
16.272195665	135.63\\
16.28818068	133.04\\
16.304120932	130.52\\
16.32018707	127.92\\
16.336241998	125.27\\
16.352170002	122.72\\
16.368089142	120.18\\
16.384068283	117.61\\
16.400201422999	114.97\\
16.41625801	112.32\\
16.432166896	109.77\\
16.4503751	106.76\\
16.465938644	104.23\\
16.481257057	101.75\\
16.49669489	99.26\\
16.512263160001	96.76\\
16.528205567	94.27\\
16.544321262	91.66\\
16.560075307999	89.1\\
16.576084647	86.56\\
16.592030798	83.97\\
16.60816473	81.4\\
16.624171668999	78.73\\
16.64020414	76.13\\
16.656199673	73.46\\
16.672193312	70.97\\
16.688135244	68.29\\
16.704079641	65.84\\
16.720093485	63.27\\
16.736166594001	60.68\\
16.752052087	58.08\\
16.768138590999	55.5\\
16.784174355	52.92\\
16.800198238	50.27\\
16.816200055	47.73\\
16.832207986	45.12\\
16.848075734	42.55\\
16.864045965	39.99\\
16.880065672999	37.4\\
16.896050901999	34.8\\
16.912055325	32.22\\
16.928170324	29.58\\
16.944171525001	26.97\\
16.960043773	24.41\\
16.976110055999	21.84\\
16.992289465999	19.22\\
17.010710367001	18.24\\
17.025942801	21.06\\
17.041344672999	23.88\\
17.05654785	26.67\\
17.07218283	29.43\\
17.088060686	32.28\\
17.104166679001	35.26\\
17.120187484999	38.24\\
17.136206506	41.16\\
17.152292935	44.11\\
17.168187017999	47.04\\
17.184303946	49.94\\
17.200242871	52.89\\
17.216197426	55.8\\
17.232247878	58.75\\
17.248206107	61.69\\
17.264225143	64.55\\
17.280265361999	67.54\\
17.296200649	70.48\\
17.312124619	73.33\\
17.328071572	76.24\\
17.34403387	79.16\\
17.360028992001	82.08\\
17.376185414	85.07\\
17.392023936	87.97\\
17.408033772	90.88\\
17.424136297	93.86\\
17.440289367001	96.94\\
17.456106461	99.74\\
17.472218333	102.68\\
17.488164248	105.63\\
17.504154388	108.48\\
17.520065635999	111.41\\
17.536036108	114.31\\
17.552048583	117.24\\
17.568056544	120.16\\
17.584030052	123.1\\
17.600059762	126.03\\
17.616063298999	128.96\\
17.632056672	131.88\\
17.648041801	134.8\\
17.664199096	137.79\\
17.680204336	140.78\\
17.696310463999	143.71\\
17.712100741999	146.62\\
17.728064418	149.46\\
17.744174479999	152.47\\
17.760232529999	155.42\\
17.776051638	158.3\\
17.792052723	161.2\\
17.808052311	164.18\\
17.824194366	167.12\\
17.840188977	170.09\\
17.85616251	173.01\\
17.872259102	175.9\\
17.888214102999	178.89\\
17.904235439	-178.2\\
17.920072032	-175.31\\
17.935957727	-172.43\\
17.951946102	-169.53\\
17.968037182	-166.6\\
17.984069889	-163.65\\
18.000060236	-160.59\\
18.017350835	-156.22\\
18.032598335999	-151.4\\
18.048043377	-146.52\\
18.064072531	-141.64\\
18.080204522999	-136.54\\
18.096187687999	-131.37\\
18.112204043	-126.3\\
18.128193738	-121.36\\
18.144231396001	-116.13\\
18.160238608999	-111.08\\
18.176118885	-106.02\\
18.192170492001	-100.95\\
18.208188731999	-95.88\\
18.224340248	-90.8\\
18.240071738	-85.74\\
18.256213768	-80.76\\
18.272146103	-75.61\\
18.288064410001	-70.65\\
18.304203323	-65.46\\
18.32034181	-60.35\\
18.336211443	-55.22\\
18.352082182999	-50.33\\
18.368080993	-45.3\\
18.384261175	-40.1\\
18.400212868999	-34.96\\
18.416200639001	-29.98\\
18.432255686001	-24.83\\
18.448179637	-19.87\\
18.464199375	-14.7\\
18.480175469001	-9.62\\
18.496197617001	-4.58\\
18.512264948	0.53\\
18.528188382	5.58\\
18.544175822	10.66\\
18.560049221	15.76\\
18.576046866	20.73\\
18.592066599001	25.86\\
18.608019775001	30.91\\
18.623961445	35.86\\
18.639963432	40.87\\
18.655943668	45.94\\
18.672049894	51.06\\
18.688061726999	56.32\\
18.704049324	61.33\\
18.720026696	66.32\\
18.736017625	71.36\\
18.752012010999	76.45\\
18.768019866	81.52\\
18.78417657	86.68\\
18.800076807	91.81\\
18.816163847	96.85\\
18.832202047	101.98\\
18.848060303	107.04\\
18.864209059	112.07\\
18.88020804	117.22\\
18.896235083999	122.3\\
18.912208297001	127.34\\
18.928157017999	132.45\\
18.944208779	137.49\\
18.960198248	142.59\\
18.976178161	147.64\\
18.992024029	152.61\\
19.010089805	155.95\\
19.025400340999	154.29\\
19.040869279999	152.61\\
19.056173349999	150.94\\
19.072090364999	149.29\\
19.088156232	147.55\\
19.104084311999	145.81\\
19.120054889	144.09\\
19.136052807	142.36\\
19.15209464	140.6\\
19.168028019	138.87\\
19.184067009	137.15\\
19.200058212	135.41\\
19.2160562	133.67\\
19.232125305	131.92\\
19.248090835	130.18\\
19.264087625001	128.44\\
19.280046026999	126.66\\
19.296047866999	124.97\\
19.312097005998	123.24\\
19.328047677	121.5\\
19.344055698	119.76\\
19.360195252999	118\\
19.376231938999	116.24\\
19.392196502	114.5\\
19.408197041	112.78\\
19.424245189	111.02\\
19.440142389	109.29\\
19.456237377	107.59\\
19.472149234999	105.85\\
19.488173243	104.09\\
19.504180241	102.34\\
19.520258281	100.63\\
19.536212505998	98.85\\
19.552210439	97.14\\
19.568322512	95.38\\
19.584236149	93.63\\
19.600232724	91.93\\
19.616239738	90.17\\
19.632126153	88.46\\
19.648093929	86.75\\
19.664193667999	84.97\\
19.680362539	83.23\\
19.69619621	81.46\\
19.712206019999	79.76\\
19.728205958	78.01\\
19.744087284	76.29\\
19.760085988999	74.59\\
19.776032689	72.85\\
19.792200241999	71.08\\
19.808212814	69.32\\
19.824169217999	67.62\\
19.840070102	65.9\\
19.856056554	64.17\\
19.872056566999	62.43\\
19.888093217999	60.68\\
19.904198229	58.93\\
19.920218288	57.15\\
19.936206423999	55.42\\
19.952194840999	53.68\\
19.968179113001	51.95\\
19.984187871	50.23\\
20.000137295	48.49\\
20.017848843	48.52\\
20.033232614	50.35\\
20.048956135	52.23\\
20.06440324	54.07\\
20.080194775001	55.91\\
20.096290009	57.79\\
20.112093442	59.69\\
20.12801402	61.57\\
20.144179889	63.51\\
20.160193583	65.45\\
20.176183929	67.34\\
20.192064696998	69.23\\
20.208209553	71.16\\
20.224197573999	73.09\\
20.240117275001	74.96\\
20.256057948999	76.84\\
20.27207247	78.75\\
20.28806657	80.67\\
20.30419779	82.57\\
20.32019001	84.54\\
20.336167899	86.45\\
20.352064075999	88.32\\
20.368193767999	90.25\\
20.384231727	92.16\\
20.400169159	94.08\\
20.41619512	95.94\\
20.432241494	97.91\\
20.448882485	99.95\\
20.464329929	101.79\\
20.480279092999	103.63\\
20.496304981	105.55\\
20.51207431	107.41\\
20.528197828	109.33\\
20.544184541	111.26\\
20.560124979	113.14\\
20.576072884	115.02\\
20.592054987	116.92\\
20.608069753	118.83\\
20.624194484	120.78\\
20.640209917	122.71\\
20.656189599	124.62\\
20.672060734999	126.5\\
20.688199653	128.42\\
20.704169106999	130.33\\
20.720060639001	132.21\\
20.736060302	134.1\\
20.752070602999	136.01\\
20.76810537	137.92\\
20.784118864	139.88\\
20.800175766	141.79\\
20.816042811999	143.68\\
20.832249038	145.6\\
20.848232528	147.53\\
20.864210922	149.44\\
20.880171622	151.34\\
20.896204203	153.26\\
20.912175797001	155.16\\
20.92807432	157.02\\
20.944183354999	158.97\\
20.960228911999	160.91\\
20.976203422	162.81\\
20.992138332	164.7\\
21.010691574	165.36\\
21.026072383	163.27\\
21.041285618	161.21\\
21.05639225	159.17\\
21.0720952	157.12\\
21.088015492001	154.98\\
21.104092309999	152.82\\
21.120213562999	150.64\\
21.136176040999	148.46\\
21.152214571	146.28\\
21.168086549	144.16\\
21.184075090999	142.02\\
21.200314377	139.82\\
21.216242331	137.61\\
21.232173909	135.48\\
21.24821252	133.32\\
21.264197539	131.15\\
21.280177176	128.99\\
21.296066809	126.87\\
21.312060197	124.7\\
21.328074623	122.54\\
21.344236071998	120.35\\
21.360066566	118.19\\
21.376205348	116.07\\
21.39220262	113.85\\
21.40809969	111.71\\
21.424060369	109.59\\
21.440106812	107.43\\
21.456059017	105.27\\
21.472273981999	103.05\\
21.488154129	100.86\\
21.504311052	98.74\\
21.520170126	96.56\\
21.536276345	94.4\\
21.552186795	92.21\\
21.568246063	90.05\\
21.584195281	87.89\\
21.600194215999	85.75\\
21.616173618	83.57\\
21.632135123	81.46\\
21.648177076	79.31\\
21.664062229	77.13\\
21.68017345	74.97\\
21.696222458	72.75\\
21.712124174	70.59\\
21.728293733	68.46\\
21.744241461	66.26\\
21.760121768	64.12\\
21.776180818999	61.95\\
21.792187668	59.79\\
21.808233405	57.65\\
21.824219647	55.45\\
21.840191996999	53.29\\
21.856161286	51.14\\
21.872202682999	48.97\\
21.888134415999	46.78\\
21.904038847	44.69\\
21.920070918	42.57\\
21.936074609	40.39\\
21.95205796	38.23\\
21.968066833	36.05\\
21.984041323	33.9\\
22.000065932	31.74\\
22.017424315	32.76\\
22.032574937	35.13\\
22.048077393	37.53\\
22.064085654	39.95\\
22.080044033	42.49\\
22.096052851	44.99\\
22.112206077	47.52\\
22.128196477999	50.07\\
22.144159487	52.56\\
22.160061778999	55.05\\
22.176061315	57.52\\
22.192016456999	60.05\\
22.208230701	62.6\\
22.224192147	65.11\\
22.24012388	67.65\\
22.256179087999	70.14\\
22.272077731	72.61\\
22.288052812999	75.1\\
22.30407033	77.61\\
22.320058026	80.12\\
22.336059041	82.63\\
22.352137431999	85.14\\
22.368082505998	87.67\\
22.384276529	90.18\\
22.400045668	92.74\\
22.416257236	95.23\\
22.432227425999	97.81\\
22.44821656	100.21\\
22.464111782	102.79\\
22.480111778	105.28\\
22.496076436	107.77\\
22.512057752	110.26\\
22.528066599999	112.76\\
22.544071745	115.27\\
22.560193203	117.82\\
22.576221571	120.42\\
22.592196958	122.89\\
22.60819889	125.41\\
22.624197465	127.93\\
22.640176701	130.4\\
22.656208146001	132.94\\
22.67210942	135.44\\
22.688043463	138\\
22.704119328	140.4\\
22.719959344	142.91\\
22.735941386	145.38\\
22.751958043999	147.94\\
22.767999529	150.45\\
22.784005451	152.95\\
22.800168689	155.51\\
22.816170861999	158.06\\
22.832054463	160.56\\
22.848165571	163.06\\
22.864197757	165.59\\
22.880074285	168.05\\
22.896098969	170.52\\
22.912061016	173.03\\
22.927985989	175.56\\
22.944201892	178.12\\
22.96006514	-179.37\\
22.976063363999	-176.91\\
22.992022338999	-174.41\\
23.010203392	-170.4\\
23.025407806	-165.58\\
23.040587109	-160.76\\
23.056059586	-155.96\\
23.072055379	-151.06\\
23.088050826999	-145.99\\
23.104069491999	-140.89\\
23.120307279999	-135.73\\
23.136216111	-130.52\\
23.152192234999	-125.6\\
23.168200788	-120.43\\
23.184172291	-115.41\\
23.200084109	-110.41\\
23.216043956	-105.37\\
23.232196861	-100.22\\
23.24817116	-95.09\\
23.264056894	-90.13\\
23.280199751	-85.02\\
23.296195014001	-79.88\\
23.312156309	-74.81\\
23.328080786	-69.84\\
23.344117432	-64.81\\
23.360197482	-59.62\\
23.376056614999	-54.58\\
23.392058402	-49.58\\
23.408056076	-44.53\\
23.424006826999	-39.45\\
23.440107473	-34.41\\
23.456145253	-29.16\\
23.472132531	-24.17\\
23.488062029	-19.13\\
23.504055598	-14.08\\
23.520046049	-8.95\\
23.536058620999	-3.87\\
23.552209606	1.24\\
23.56824292	6.39\\
23.584203163	11.48\\
23.60017627	16.48\\
23.616153972	21.57\\
23.632111900001	26.55\\
23.648037519999	31.62\\
23.664049515	36.61\\
23.680091819	41.68\\
23.696184151	46.89\\
23.712108738	51.99\\
23.728060340999	56.96\\
23.744063932999	62.03\\
23.760058679	67.11\\
23.776073837001	72.17\\
23.792082384	77.23\\
23.808017465	82.35\\
23.824008382	87.38\\
23.840177163	92.57\\
23.856232472	97.66\\
23.872201083999	102.79\\
23.888156931	107.86\\
23.904091782	112.89\\
23.922763472	118.97\\
23.938228422	123.87\\
23.953554227	128.73\\
23.968996074	133.63\\
23.984450194	138.51\\
24.000180226	143.41\\
24.017625856	145.52\\
24.032730734999	144.17\\
24.048160355	142.82\\
24.064222272999	141.45\\
24.080006767999	140.02\\
24.096094029	138.6\\
24.114831638	136.86\\
24.130215715	135.48\\
24.146151918	134.1\\
24.161600606	132.68\\
24.177142461	131.3\\
24.192796078999	129.92\\
24.208044067	128.54\\
24.224026826	127.19\\
24.24008224	125.76\\
24.256086252	124.31\\
24.272538650001	122.88\\
24.288415349	121.41\\
24.304122553	119.99\\
24.320108134	118.59\\
24.336239128999	117.17\\
24.352087012	115.75\\
24.368065741999	114.34\\
24.384044768	112.92\\
24.400052713	111.48\\
24.416032425999	110.07\\
24.432038327	108.64\\
24.448038776	107.21\\
24.464013245999	105.79\\
24.48000937	104.36\\
24.496035488	102.94\\
24.514506141999	101.21\\
24.529933746	99.83\\
24.545344233	98.46\\
24.560543637	97.1\\
24.576064042999	95.75\\
24.59205701	94.37\\
24.608048454	92.95\\
24.624501503999	91.48\\
24.640061605	90.03\\
24.656462652	88.64\\
24.672062262999	87.19\\
24.688011019	85.81\\
24.704199921	84.34\\
24.720113914	82.92\\
24.736196182999	81.51\\
24.752181032	80.06\\
24.768161982	78.63\\
24.786499885	76.95\\
24.801755564	75.59\\
24.816823306	74.24\\
24.83210916	72.9\\
};
\end{axis}
\end{tikzpicture}%}
      \caption{The orientation of the robot over time for
        $K_{\Psi}^R = K_{\Psi, max}^R + 1$. The system is indeed unstable}
      \label{fig:8_max_plus_one}
    \end{figure}
  \end{minipage}
  \hfill
\end{minipage}
}
