Examining file \texttt{rot2.slx} one sees that $\dot{\theta} = R/L u_{\Psi}$ is
stable. We can verify this analytically:

\begin{align*}
  \dot{u}_{\Psi} &= K_L (u_{\Psi}^R - u_{\Psi})  \\
  \dot{u}_{\Psi} &= K_L u_{\Psi}^R - K_L u_{\Psi}  \\
  \dot{u}_{\Psi} \cdot e^{K_L t} &= K_L u_{\Psi}^R \cdot e^{K_L t} - K_L u_{\Psi} \cdot e^{K_L t}  \\
  \dot{u}_{\Psi} \cdot e^{K_L t} + K_L u_{\Psi} \cdot e^{K_L t} &= K_L u_{\Psi}^R \cdot e^{K_L t} \\
  \dfrac{d}{dt}(u_{\Psi} e^{K_L t}) &= K_L u_{\Psi}^R e^{K_L t} \\
  u_{\Psi} e^{K_L t} &= \int K_L u_{\Psi}^R e^{K_L t} = u_{\Psi}^R e^{K_L t} + \lambda \\
  u_{\Psi} &= u_{\Psi}^R + \lambda e^{-K_L t}
\end{align*}

Where, from inspection of the parameters of the integrator, the initial
condition is $\lambda = 1$. Hence
\begin{equation}
  u_{\Psi}(t) = \left\{
    \begin{matrix}
    1 + e^{-K_L t} & \theta - \theta^G \leq 0 \\
    -1 +e^{-K_L t} & \theta - \theta^G > 0
    \end{matrix}\right.
\end{equation}

In order to reach a conclusion about the stability of the angular displacement,
it suffices to find a Lyapunov function $V(x)$ such that $V(0) = 0$, $V(x) > 0$
for all $x \neq 0$ and $\dot{V}(x) \leq 0$ for all x. Considering
$x = \theta - \theta^G$ and $V(x) = x^2$:
$$V(0) = 0,\ V(x) > 0,\ \text{for all } x \neq 0\text{, and}$$
\begin{align*}
  \dot{V}(x) &= 2 x \dot{x} = 2 (\theta - \theta^G) \dot{\theta}  \\
             &= 2 (\theta - \theta^G) \dfrac{R}{L} u_{\Psi} \\
             &= 2 (\theta - \theta^G) \dfrac{R}{L} (u_{\Psi}^R + e^{-K_L t})
\end{align*}

\begin{itemize}
  \item When $\theta - \theta^G \leq 0$, $\dot{V}(x) = 2 (\theta - \theta^G) \dfrac{R}{L} (1 + e^{-K_L t}) \leq 0$ \\
    With
    \begin{align}
      0 <&\ e^{-K_L t} \leq 1 \nonumber \\
      0 <&\ 1 < 1 + e^{-K_L t} \leq 2 \label{04.lambda_A}
    \end{align}
  \item When $\theta - \theta^G > 0$, $\dot{V}(x) = 2 (\theta - \theta^G) \dfrac{R}{L} (-1 + e^{-K_L t}) < 0$  \\
    With
    \begin{align}
      0 <&\ e^{-K_L t} \leq 1 \nonumber \\
      -1 <&\ -1 + e^{-K_L t} \leq 0 \label{04.lambda_B}
    \end{align}
\end{itemize}


Hence, from inequalities \ref{04.lambda_A} and \ref{04.lambda_B} we conclude
that $\dot{V}(x) \leq 0$ for all $x$, meaning that the system is stable for all
$\theta \in (-180^{\circ}, 180^{\circ}]$.

From what one can see in the continuous and discrete state trajectory plot in
figure \ref{fig:04}, one can say that the system enters into zeno behaviour,
even though the system is asymptotically stable as $\dot{\theta}=R /Lu_\Psi$
goes to zero. So one can conclude that this control law is not practical as
the discrete part enters zeno-behaviour.

\begin{figure}[H]\centering
  \scalebox{0.9}{% This file was created by matlab2tikz.
%
%The latest updates can be retrieved from
%  http://www.mathworks.com/matlabcentral/fileexchange/22022-matlab2tikz-matlab2tikz
%where you can also make suggestions and rate matlab2tikz.
%
\definecolor{mycolor1}{rgb}{0.00000,0.44700,0.74100}%
\definecolor{mycolor2}{rgb}{0.85000,0.32500,0.09800}%
%
\begin{tikzpicture}

\begin{axis}[%
width=4.537in,
height=3.6in,
at={(0.761in,0.486in)},
scale only axis,
xmin=0,
xmax=3,
xmajorgrids,
ymin=-1,
ymax=1,
ymajorgrids,
axis background/.style={fill=white},
legend style={legend cell align=left,align=left,draw=white!15!black}
]
\addplot [color=mycolor1,solid]
  table[row sep=crcr]{%
0	1\\
0.000200950914520766	0.999999798228826\\
0.0012057054871246	0.9999927604963\\
0.00622947835014376	0.999809935083858\\
0.0313483426652396	0.995562000407491\\
0.0913483426652396	0.968537878464885\\
0.15134834266524	0.926636021704914\\
0.21134834266524	0.876568821315784\\
0.27134834266524	0.822020237342346\\
0.33134834266524	0.765012136691865\\
0.39134834266524	0.706654180093805\\
0.45134834266524	0.647555382380001\\
0.51134834266524	0.588049989014446\\
0.571348342665239	0.528321443854923\\
0.63134834266524	0.468470426349872\\
0.69134834266524	0.40855219237682\\
0.75134834266524	0.348597068005765\\
0.81134834266524	0.28862169708643\\
0.87134834266524	0.22863521425886\\
0.93134834266524	0.16864263288536\\
0.99134834266524	0.108646704448362\\
1.05134834266524	0.0486489390433071\\
1.09999832888327	1.41414657761629e-14\\
1.09999832888328	-9.0205620750794e-17\\
1.15999832888328	-0.030233294041868\\
1.21999832888328	-0.0197560111548758\\
1.25935832776576	-1.09148801108461e-14\\
1.25935832776578	3.46944695195361e-17\\
1.31935832776578	0.0118992429003487\\
1.36111750025393	2.94382573873264e-15\\
1.36111750025394	-3.07739944638286e-15\\
1.42111750025394	-0.00424551537333807\\
1.43652806024928	-4.68462074687537e-15\\
1.4365280602493	1.73472347597681e-18\\
1.4965280602493	1.15330783039061e-05\\
1.49657075331761	8.26178996092899e-17\\
1.49657075331763	-3.75849016894057e-15\\
1.54649475114703	-3.24869830743788e-15\\
1.54649475114705	2.73269769443182e-17\\
1.58926516707352	9.65448153280452e-18\\
1.58926516707353	-2.8213058110441e-15\\
1.62669428325042	-2.15642729909683e-16\\
1.62669428325043	2.26945706964434e-15\\
1.65996161271047	2.19269555583237e-15\\
1.65996161271048	-3.97851375325345e-17\\
1.68989953171323	-2.00837102211446e-15\\
1.68989953171324	1.14976252260299e-17\\
1.71711437070657	1.51297025038563e-17\\
1.71711437070658	-1.83126659723896e-15\\
1.74206081340125	-1.68506193427929e-15\\
1.74206081340127	1.33949790278795e-17\\
1.76508842219257	1.94665111937983e-17\\
1.76508842219258	-1.55316873999643e-15\\
1.78647168804854	-8.04511893302134e-18\\
1.78647168804856	1.45557360381379e-15\\
1.80643008644	1.36252195236022e-15\\
1.80643008644001	-8.83116550807334e-18\\
1.82514186246521	-1.26588226934208e-15\\
1.82514186246522	2.00357173343532e-17\\
1.84275375445231	8.81422484912825e-18\\
1.84275375445232	-1.20489589713977e-15\\
1.85938801059342	-1.60443286803015e-16\\
1.85938801059344	9.87659051223354e-16\\
1.87514755482624	1.09826291940993e-17\\
1.87514755482625	-1.07883028953402e-15\\
1.89011985835063	-1.75996505780499e-17\\
1.89011985835064	1.01865368082928e-15\\
1.9043798874291	3.1411200409389e-16\\
1.90437988742912	-6.74585509522797e-16\\
1.91799237995543	-1.71271756000714e-16\\
1.91799237995545	7.73678199956183e-16\\
1.93101362626831	1.17483469784171e-17\\
1.93101362626833	-8.93939937807349e-16\\
1.94349287839493	-3.29495816481923e-18\\
1.94349287839494	8.64744406181388e-16\\
1.95547347706441	5.44184787292998e-17\\
1.95547347706442	-7.80281668812978e-16\\
1.96699376172967	-1.95673081145216e-16\\
1.96699376172969	6.07307376588283e-16\\
1.9780878118909	1.95134368190762e-16\\
1.97808781189092	-5.7859973593994e-16\\
1.98878605591936	-2.15290364203626e-16\\
1.98878605591937	5.30633954202823e-16\\
1.99911577482883	1.51190298887209e-16\\
1.99911577482884	-5.7001420998657e-16\\
2.00910152202816	-5.75558887659297e-17\\
2.00910152202819	1.33742944833076e-15\\
2.01876547533242	4.93820208249257e-18\\
2.01876547533245	-1.34635717559482e-15\\
2.02812773394501	-2.31063303527627e-15\\
2.02812773394506	1.95664610815743e-18\\
2.03720657042426	1.41201578153418e-15\\
2.0372065704243	-1.72201798176799e-17\\
2.04601864558309	-1.42264096282454e-15\\
2.04601864558313	3.77268474707065e-18\\
2.05457919268063	1.82976057265874e-17\\
2.05457919268066	-1.18128061857032e-15\\
2.06290217602571	-3.6349571898471e-17\\
2.06290217602574	1.12998091515281e-15\\
2.07100042814403	1.12773966597438e-17\\
2.07100042814405	-1.12385517287827e-15\\
2.07888576889484	-4.8450284582946e-19\\
2.07888576889487	1.10573036187292e-15\\
2.08656910931517	1.07330085845434e-15\\
2.0865691093152	-4.77048955893622e-18\\
2.09406054248304	-1.06655508806241e-15\\
2.09406054248307	1.03338019565025e-18\\
2.1013694232998	9.48507494335366e-18\\
2.10136942329983	-1.01649205274789e-15\\
2.10850443877488	-1.61095502172401e-15\\
2.10850443877493	1.41454502191468e-18\\
2.11547367013832	7.26666177317675e-16\\
2.11547367013835	-2.52288763339693e-16\\
2.12228464789564	-1.27122704723925e-17\\
2.12228464789567	9.44096146746065e-16\\
2.12894440076641	4.70272692315588e-18\\
2.12894440076644	-9.30994241117936e-16\\
2.13545949930384	-3.59108088317933e-16\\
2.13545949930387	5.56617536892796e-16\\
2.14183609487481	1.07640946937076e-17\\
2.14183609487484	-8.85615298001734e-16\\
2.14807995458024	-8.58552595336959e-18\\
2.14807995458027	8.69585199474946e-16\\
2.15419649261277	7.60889696518538e-18\\
2.1541964926128	-8.52782606900263e-16\\
2.16019079847927	-1.1649667640062e-16\\
2.1601907984793	7.26743257315875e-16\\
2.16606766245712	9.55859740317533e-17\\
2.16606766245715	-7.3154508709386e-16\\
2.17183159860325	-3.52450409352514e-18\\
2.17183159860328	8.07722995205176e-16\\
2.17748686559312	7.94509281228008e-16\\
2.17748686559315	-1.61613886336121e-18\\
2.18303748563057	-2.79182059415017e-18\\
2.1830374856306	7.78847642033276e-16\\
2.18848726163916	1.9498698445794e-18\\
2.18848726163919	-7.65392523666142e-16\\
2.19383979291884	-1.60673679764668e-17\\
2.19383979291887	7.37740286070078e-16\\
2.19909848942967	1.2064290267743e-17\\
2.1990984894297	-7.28616047162255e-16\\
2.20426658484454	-1.37439566021483e-17\\
2.20426658484457	7.14460432547741e-16\\
2.20934714849606	9.01497165762752e-18\\
2.20934714849609	-7.06845606351925e-16\\
2.21434309632815	-1.16382326952741e-17\\
2.21434309632818	6.92531596576274e-16\\
2.21925720095025	5.19570009845788e-18\\
2.21925720095027	-6.87508267682583e-16\\
2.22409210088093	-1.17970513728843e-17\\
2.22409210088096	6.69781985678918e-16\\
2.22885030905812	6.69918357983426e-16\\
2.22885030905815	-8.72443935671929e-19\\
2.2335342206847	-1.73133534418779e-18\\
2.23353422068473	6.58668489894468e-16\\
2.23814612047082	2.49408851319029e-18\\
2.23814612047085	-6.47837479597927e-16\\
2.24268818932761	-6.39575943746883e-16\\
2.24268818932764	9.63923493975394e-19\\
2.2471625105616	2.84067321938307e-16\\
2.24716251056163	-3.46996787721618e-16\\
2.25157107561385	-6.19152708839161e-16\\
2.25157107561387	2.67874169569172e-18\\
2.2559157893833	1.24584146981004e-16\\
2.25591578938333	-4.88348798926468e-16\\
2.26019847517002	-6.03881551833113e-16\\
2.26019847517005	3.65918233213858e-19\\
2.26442087927042	6.10748871452977e-17\\
2.26442087927045	-5.34657360702281e-16\\
2.2685846752535	-6.86731961986424e-18\\
2.26858467525353	5.8074230578002e-16\\
2.27269146794428	7.0906822080552e-16\\
2.27269146794432	-6.41203941071505e-18\\
2.27674279713816	-9.91917932882148e-18\\
2.27674279713818	5.6198899632781e-16\\
2.28074014106774	8.41373073100747e-16\\
2.28074014106778	-5.02036427837624e-18\\
2.28468491964181	-3.82858892158944e-18\\
2.28468491964184	5.53132419831319e-16\\
2.28857849747368	9.84115065372042e-16\\
2.28857849747373	-3.70915727602658e-18\\
2.29242218671602	-1.30485425524525e-18\\
2.29242218671605	5.41464540982891e-16\\
2.29621724971568	5.67904674431434e-16\\
2.29621724971571	-1.5661639194732e-18\\
2.29996490150324	-5.27953306683e-16\\
2.29996490150327	1.28621953040566e-18\\
2.30366631212859	5.17707807911249e-16\\
2.30366631212862	-5.0260817507702e-18\\
2.30732260885429	-5.62429876976855e-19\\
2.30732260885432	5.15853229273236e-16\\
2.31093487821676	8.04328510669054e-16\\
2.3109348782168	-8.76679100408201e-19\\
2.31450416796486	-5.0352995868187e-16\\
2.31450416796489	6.48191962886353e-19\\
2.31803148888442	4.64576395745302e-18\\
2.31803148888445	-4.93746516382846e-16\\
2.3215178165169	-3.70117399049871e-16\\
2.32151781651693	1.22454070876897e-16\\
2.32496409277908	1.52381227211049e-18\\
2.32496409277911	-4.8542526470902e-16\\
2.32837122749092	-2.23582816757245e-16\\
2.32837122749095	2.57809935848134e-16\\
2.33174009981753	1.4054394177317e-18\\
2.33174009981756	-4.74656934850576e-16\\
2.33507155963107	-1.28167943380837e-16\\
2.3350715596311	3.4259920441771e-16\\
2.33836642879792	2.03232003166513e-16\\
2.33836642879794	-2.62430500575406e-16\\
2.34162550239575	-2.73417364934241e-16\\
2.34162550239578	1.87220539364565e-16\\
2.34484954986541	4.09822068452416e-16\\
2.34484954986543	-4.59218912353919e-17\\
2.34803931610133	-8.65975568570222e-16\\
2.34803931610139	5.65818008765873e-19\\
2.35119552248486	6.68446426479335e-16\\
2.35119552248491	-8.28186464177892e-19\\
2.35431886786343	-2.2166852229645e-18\\
2.35431886786346	4.39300297324524e-16\\
2.35741002947952	2.54931506135127e-17\\
2.35741002947955	-4.11534345555291e-16\\
2.36046966385218	-2.02334995275371e-17\\
2.3604696638522	4.12365284876547e-16\\
2.36349840761393	1.61052727008565e-17\\
2.36349840761396	-4.1212123350862e-16\\
2.36649687830604	-1.3190526450236e-17\\
2.36649687830607	4.10785674309036e-16\\
2.36946567513427	1.07419659579606e-17\\
2.3694656751343	-4.09070009074373e-16\\
2.3724053796877	-8.93153891232297e-18\\
2.37240537968773	4.0678143193001e-16\\
2.37531655662261	3.40824882151449e-18\\
2.37531655662263	-4.08242775522261e-16\\
2.37819975431349	-8.15862134795342e-18\\
2.37819975431352	3.99616591987423e-16\\
2.38105550547312	3.71116897927631e-18\\
2.38105550547315	-4.0015244620568e-16\\
2.38388432774333	-4.9059089513785e-18\\
2.38388432774336	3.95218267529686e-16\\
2.38668672425822	7.43419641981418e-18\\
2.38668672425825	-3.88943447456427e-16\\
2.38946318418124	-3.92145973150877e-16\\
2.38946318418127	6.18334051495639e-19\\
2.39221418321774	5.1586688767951e-16\\
2.39221418321778	-9.53123823897901e-19\\
2.39494018410423	-4.62017615090766e-16\\
2.39494018410426	1.93494088888404e-18\\
2.39764163707554	5.61731074795371e-18\\
2.39764163707557	-3.76534414779151e-16\\
2.40031898031115	-3.7847168500864e-16\\
2.40031898031118	3.32354552678906e-19\\
2.40297264036176	2.79870273684662e-18\\
2.40297264036179	-3.72620063771652e-16\\
2.40560303255718	-3.70864164471994e-16\\
2.4056030325572	1.30750123320542e-18\\
2.40821056139648	8.46609430780673e-19\\
2.40821056139651	-3.68100295965103e-16\\
2.41079562092143	-5.44853943321329e-19\\
2.41079562092145	3.65250453614066e-16\\
2.41335859507376	5.11396141904784e-20\\
2.41335859507379	-3.62630051312617e-16\\
2.41589985803751	-5.37442405032854e-19\\
2.41589985803754	3.59099829746697e-16\\
2.41841977456685	3.05228322543087e-18\\
2.41841977456688	-3.53557799051931e-16\\
2.42091870030032	-3.51305856082537e-16\\
2.42091870030035	2.32023500076634e-18\\
2.42339698206204	5.09217255527091e-16\\
2.42339698206208	-4.51256802649729e-19\\
2.42585495815077	-3.91371573278849e-18\\
2.4258549581508	3.43928093232834e-16\\
2.42829295861694	3.42434668267706e-16\\
2.42829295861697	-2.65370128418852e-18\\
2.43071130552861	-3.46213652822322e-16\\
2.43071130552864	1.42645642273545e-18\\
2.43311031322691	3.0049044375262e-16\\
2.43311031322694	-3.90747945271433e-17\\
2.43549028857125	-9.624676560973e-18\\
2.43549028857127	3.27268755105354e-16\\
2.43785153117486	3.32339306085856e-16\\
2.43785153117489	-1.91212393886738e-18\\
2.44019433363114	-1.99958220825375e-16\\
2.44019433363116	1.31659624947656e-16\\
2.4425189817313	1.02739802546026e-18\\
2.44251898173133	-3.28048184235506e-16\\
2.44482575467359	-3.23704281644631e-16\\
2.44482575467362	2.85667155417433e-18\\
2.44711492526451	1.39861233217683e-16\\
2.44711492526454	-1.84209919572227e-16\\
2.44938676011249	-2.80500254439182e-18\\
2.44938676011252	3.18816001566467e-16\\
2.45164151981423	9.06753005200465e-17\\
2.45164151981427	-2.8338588393224e-16\\
2.45387945913407	-3.14960360530125e-16\\
2.4538794591341	1.887559983397e-18\\
2.45610082717671	7.85938578351216e-17\\
2.45610082717674	-2.35924192677859e-16\\
2.45830586755374	-5.13619603391326e-19\\
2.45830586755377	3.11680331321001e-16\\
2.46049481854394	8.44915364886165e-20\\
2.46049481854397	-3.09833270100394e-16\\
2.46266791324773	-2.63697238348025e-18\\
2.46266791324776	3.0506860389297e-16\\
2.46482537973615	3.03113704790267e-16\\
2.46482537973618	-2.36840999964143e-18\\
2.46696744119457	-3.81669869659236e-17\\
2.4669674411946	2.65164458231347e-16\\
2.46909431606123	2.99649340035997e-16\\
2.46909431606126	-1.5239710907881e-18\\
2.47120621816096	-3.28534963482381e-17\\
2.47120621816099	2.66224625843954e-16\\
2.47330335683412	2.96418756375169e-16\\
2.47330335683415	-5.76935316198585e-19\\
2.47538593706119	-2.52629694100963e-17\\
2.47538593706122	2.69650185901328e-16\\
2.47745415958298	2.92458824407196e-16\\
2.477454159583	-4.53692022373085e-19\\
2.47950822101673	-2.01977123855156e-17\\
2.47950822101676	2.70705324255811e-16\\
2.48154831396834	2.86457913613254e-16\\
2.48154831396837	-2.46629524460851e-18\\
2.48357462714071	-1.71305531439486e-17\\
2.48357462714074	2.69850244495558e-16\\
2.48558734543851	2.69531177772239e-18\\
2.48558734543854	-2.82370714944956e-16\\
2.48758665006946	-4.39784800170353e-18\\
2.48758665006948	2.78776912968434e-16\\
2.48957271864214	8.77870240490277e-18\\
2.48957271864217	-2.72536662427334e-16\\
2.49154572526076	-9.49497464092469e-18\\
2.49154572526079	2.69953000179971e-16\\
2.49350584061673	1.34368012796502e-17\\
2.49350584061675	-2.64196776028668e-16\\
2.49545323207727	-9.61318867662587e-18\\
2.49545323207729	2.66220073041862e-16\\
2.49738806377119	1.14348653785942e-17\\
2.49738806377122	-2.62636170762911e-16\\
2.4993104966719	-9.77298673608131e-18\\
2.49931049667193	2.62548423443531e-16\\
2.50122068867777	5.2100731887816e-18\\
2.5012206886778	-2.65373225383065e-16\\
2.50311879468997	-1.09589387413557e-17\\
2.50311879468999	2.57926694045957e-16\\
2.50500496668782	7.11806784203111e-18\\
2.50500496668785	-2.60083981083495e-16\\
2.50687935380189	-1.4104501470103e-17\\
2.50687935380191	2.51427754348809e-16\\
2.50874210238465	1.09590446204741e-17\\
2.50874210238468	-2.52919882294737e-16\\
2.51059335607914	-1.33462216938532e-17\\
2.51059335607917	2.48920430945818e-16\\
2.51243325588541	1.10231014871103e-17\\
2.51243325588544	-2.49634426780794e-16\\
2.51426194022494	-1.17663993681055e-17\\
2.51426194022496	2.47316838757988e-16\\
2.51607954500316	1.52050884361619e-17\\
2.51607954500319	-2.4231389159459e-16\\
2.51788620367008	-1.31893882497131e-17\\
2.51788620367011	2.42767371858726e-16\\
2.51968204727905	1.12996842141684e-17\\
2.51968204727908	-2.43145175022981e-16\\
2.52146720454381	-9.60940349814283e-18\\
2.52146720454384	2.43310240568577e-16\\
2.52324180189389	6.92801482449093e-18\\
2.52324180189392	-2.44509930389465e-16\\
2.5250059635283	-1.14056692116936e-17\\
2.52500596352833	2.38574823467613e-16\\
2.52675981146771	1.01455488839752e-17\\
2.52675981146774	-2.38368756233414e-16\\
2.52850346560511	-8.13273390350296e-18\\
2.52850346560514	2.38941482854656e-16\\
2.53023704375495	6.58946634338523e-18\\
2.53023704375498	-2.39057023442618e-16\\
2.53196066170097	-9.20920690034477e-18\\
2.531960661701	2.35029143550407e-16\\
2.53367443324254	4.60780629350419e-18\\
2.53367443324257	-2.38244533557743e-16\\
2.53537847023981	-9.11328041906822e-18\\
2.53537847023984	2.32356781071599e-16\\
2.53707288265747	5.40774950284707e-18\\
2.5370728826575	-2.34694539066462e-16\\
2.53875777860737	-1.82755299303996e-18\\
2.5387577786074	2.36936476459166e-16\\
2.5404332643899	4.48268364532697e-18\\
2.54043326438993	-2.32956692156492e-16\\
2.54209944453423	-7.5296464450579e-18\\
2.54209944453426	2.28597251804991e-16\\
2.54375642183746	5.77777055189919e-18\\
2.54375642183749	-2.29046999830203e-16\\
2.54540429740265	-7.18043064277271e-18\\
2.54540429740267	2.26350496912399e-16\\
2.54704317067586	4.04132654024828e-18\\
2.54704317067589	-2.28220825069275e-16\\
2.54867313948218	-7.75696891227727e-19\\
2.54867313948221	2.30217625833088e-16\\
2.55029430006074	3.06033003842979e-18\\
2.55029430006077	-2.26702386162424e-16\\
2.55190674709879	-3.94566475676774e-18\\
2.55190674709882	2.24590150689767e-16\\
2.55351057376496	3.93073580107239e-18\\
2.55351057376499	-2.23383049330591e-16\\
2.55510587174148	-8.74561518040065e-19\\
2.55510587174151	2.25238819108686e-16\\
2.55669273125569	1.5365177663193e-18\\
2.55669273125572	-2.23379608259243e-16\\
2.55827124111063	-3.85646159951002e-18\\
2.55827124111065	2.19868603753315e-16\\
2.55984148871487	1.23648281453407e-18\\
2.5598414887149	-2.21323515189122e-16\\
2.56140356011157	-4.42124728687143e-19\\
2.5614035601116	2.2096477026618e-16\\
2.56295754000673	2.20250747961424e-16\\
2.56295754000676	-1.4055452968501e-20\\
2.56450351179676	-5.75691236557306e-19\\
2.56450351179679	2.18554882121903e-16\\
2.56604155759543	4.00323652740143e-18\\
2.56604155759546	-2.14010669473892e-16\\
2.56757175825995	-2.16755082692775e-16\\
2.56757175825998	1.45186741115307e-19\\
2.56909419341645	2.35901322788283e-18\\
2.56909419341648	-2.13442045668488e-16\\
2.57060894148497	-2.1261453418369e-18\\
2.570608941485	2.12583418957768e-16\\
2.57211607970358	2.11423314692275e-16\\
2.57211607970361	-2.20964426159045e-18\\
2.57361568415202	-5.7497655250806e-19\\
2.57361568415205	2.12005279266598e-16\\
2.57510782977484	4.26613437840549e-18\\
2.57510782977487	-2.07258361046397e-16\\
2.57659259040383	-8.62120721627268e-19\\
2.57659259040386	2.09614939054556e-16\\
2.57807003877987	8.69095508552315e-19\\
2.5780700387799	-2.08572241496486e-16\\
2.57954024657436	-2.08233362143136e-16\\
2.57954024657439	1.84587008052433e-19\\
2.58100328441009	1.53886034181405e-18\\
2.58100328441012	-2.0586188164906e-16\\
2.58245922188164	-2.06178671971333e-16\\
2.58245922188166	2.24926952165419e-19\\
2.58390812757517	1.04803121865979e-18\\
2.5839081275752	-2.04352985097755e-16\\
2.58535006908789	-2.03896619627643e-16\\
2.58535006908792	5.20660564765378e-19\\
2.5867851130469	5.02025839925783e-19\\
2.58678511304693	-2.02938625364298e-16\\
2.58821332512775	-1.31665977694761e-18\\
2.58821332512778	2.01164806023849e-16\\
2.58963477007241	1.87609856882947e-18\\
2.58963477007244	-1.99649781718567e-16\\
2.59104951170689	-9.54844359572012e-19\\
2.59104951170692	1.99616535675387e-16\\
2.59245761295842	4.2055185831176e-19\\
2.59245761295845	-1.99216696419613e-16\\
2.59385913587227	-1.97135192361075e-16\\
2.59385913587229	1.56748740845329e-18\\
2.59525414162812	2.31213524820822e-20\\
2.59525414162814	-1.97756424853437e-16\\
2.5966426905562	-2.1729836168421e-18\\
2.59664269055623	1.94692825797226e-16\\
2.59802484215287	1.16039543306899e-18\\
2.59802484215289	-1.94798201989821e-16\\
2.59940065509584	-1.94325994356616e-16\\
2.59940065509586	7.35409886673945e-19\\
2.60077018725917	1.93098061280893e-16\\
2.6007701872592	-1.07937143770819e-18\\
2.60213349572794	-1.92537722516505e-16\\
2.60213349572797	7.48684481144196e-19\\
2.60349063681251	1.9133784740716e-16\\
2.60349063681254	-1.09852232325002e-18\\
2.60484166606245	-1.90005557225358e-16\\
2.60484166606247	1.55479514913427e-18\\
2.60618663828018	1.90310065569896e-16\\
2.6061866382802	-3.95670265486165e-19\\
2.60752560753433	-1.87866216698389e-16\\
2.60752560753435	1.98560404704169e-18\\
2.60885862717276	1.88157622502025e-16\\
2.60885862717279	-8.60386951063356e-19\\
2.61018574983537	-2.08281431262892e-18\\
2.6101857498354	1.86096473704e-16\\
2.61150702746651	1.85447567057064e-16\\
2.61150702746654	-1.89892875373593e-18\\
2.61282251132715	-1.85372511997003e-16\\
2.61282251132717	1.15352652526235e-18\\
2.61413225200684	1.84235767312009e-16\\
2.61413225200686	-1.48019007532689e-18\\
2.61543629943545	-1.84024736994134e-16\\
2.61543629943548	8.8820668942474e-19\\
2.61673470289456	1.83486474025933e-16\\
2.61673470289459	-6.19670775365525e-19\\
2.61802751102862	-1.9154856008768e-18\\
2.61802751102864	1.81398736334299e-16\\
2.6193147718559	1.81827440884728e-16\\
2.61931477185592	-7.12354708640867e-19\\
2.62059653277912	-1.79430972918708e-16\\
2.62059653277915	2.32971118186375e-18\\
2.621872840596	1.79424527527375e-16\\
2.62187284059603	-1.56130671491629e-18\\
2.6231437415095	-1.79849764536677e-16\\
2.62314374150953	3.77644345577409e-19\\
2.6244092811379	2.5286315755705e-18\\
2.62440928113793	-1.7694207248232e-16\\
2.62566950452458	-3.02787808863811e-19\\
2.62566950452461	1.78412309920517e-16\\
2.62692445614764	1.24540313025984e-20\\
2.62692445614767	-1.77953588640019e-16\\
2.62817417992924	-1.77152573404603e-16\\
2.62817417992926	7.50947647300141e-20\\
2.62941871924487	1.7569140186592e-16\\
2.6294187192449	-8.04879823238598e-19\\
2.63065811693249	-7.00760945175323e-19\\
2.63065811693252	1.75067191523352e-16\\
2.63189241530135	2.24172563446771e-19\\
2.63189241530138	-1.74820175540109e-16\\
2.63312165614058	-1.09133577808816e-18\\
2.6331216561406	1.73234900439764e-16\\
2.63434588072763	1.71867273102193e-16\\
2.63434588072766	-1.74986418990898e-18\\
2.63556512983669	-1.72848620328591e-16\\
2.63556512983672	6.98140436994755e-20\\
2.63677944374682	1.31511129484091e-18\\
2.63677944374684	-1.70903369460552e-16\\
2.63798886224997	-1.71281503497052e-16\\
2.63798886225	2.31696598298553e-19\\
2.63919342465874	1.69814693895314e-16\\
2.63919342465876	-1.02027765474741e-18\\
2.64039316981402	-1.67897031155065e-16\\
2.64039316981405	2.25223413696958e-18\\
2.64158813609263	8.58149659116972e-17\\
2.64158813609267	-1.1279076829004e-16\\
2.64277836141466	-2.64505890114857e-19\\
2.64277836141469	1.68544521668792e-16\\
2.64396388325056	1.67015620581553e-16\\
2.64396388325059	-1.12325171484291e-18\\
2.64514473862831	-2.69210231694568e-18\\
2.64514473862834	1.64787136454331e-16\\
2.64632096414036	1.47307632205893e-18\\
2.64632096414039	-1.65353894140273e-16\\
2.64749259595047	-3.48560675240045e-19\\
2.6474925959505	1.65828186268621e-16\\
2.64865966980027	1.09445921208116e-19\\
2.6486596698003	-1.65425746357002e-16\\
2.64982222101586	-1.64705060285231e-16\\
2.64982222101589	1.86552389187859e-19\\
2.65098028451423	1.52141675705653e-19\\
2.65098028451427	-1.79497974548325e-16\\
2.6521338948096	-1.61691277354231e-16\\
2.65213389480963	1.93416003038579e-18\\
2.6532830860195	4.67288224665496e-17\\
2.65328308601954	-1.72303660506151e-16\\
2.65442789187099	-8.78584924539523e-19\\
2.65442789187102	1.61496248018126e-16\\
2.65556834570635	1.610546923897e-16\\
2.65556834570638	-7.07166631838934e-19\\
2.65670448048908	-1.55007029347537e-18\\
2.65670448048913	2.62838194738276e-16\\
2.6578363288097	3.51306914873721e-19\\
2.65783632880973	-1.60185153512339e-16\\
2.65896392289107	-4.6094474198395e-19\\
2.65896392289109	1.59478979497235e-16\\
2.66008729459409	1.58202203060703e-16\\
2.66008729459412	-1.14200555369071e-18\\
2.66120647542306	-8.66619459193923e-17\\
2.6612064754231	1.19209630286447e-16\\
2.66232149653109	5.76154457700335e-19\\
2.66232149653112	-1.57576669229155e-16\\
2.66343238872516	-1.45426954365193e-18\\
2.66343238872519	1.56120421069478e-16\\
2.66453918247129	4.97506125058793e-19\\
2.66453918247132	-1.5649876024677e-16\\
2.66564190789948	-1.54038698635258e-16\\
2.66564190789951	2.38306102265097e-18\\
2.66674059480871	1.05024806270143e-18\\
2.66674059480875	-2.40022336600022e-16\\
2.66783527267182	-1.54366473916066e-16\\
2.66783527267185	9.16165394129032e-19\\
2.66892597064009	5.29620121939409e-17\\
2.66892597064014	-2.08119833258424e-16\\
2.67001271754809	-1.51028674412961e-16\\
2.67001271754812	3.12934998874121e-18\\
2.67109554191803	1.52281224383713e-16\\
2.67109554191806	-1.31689800496402e-18\\
2.67217447196445	-2.2975298855685e-17\\
2.6721744719645	2.16161941254015e-16\\
2.67324953559868	1.51674848055153e-16\\
2.67324953559871	-8.24983620846087e-19\\
2.67432076043287	-3.65782443244501e-17\\
2.6743207604329	1.46246961667474e-16\\
2.67538817378455	1.33941055251526e-18\\
2.67538817378458	-1.50077641701652e-16\\
2.67645180268066	-2.96190216298088e-19\\
2.67645180268069	1.50579912387039e-16\\
2.67751167386166	1.49274800051442e-16\\
2.67751167386169	-1.07352161641622e-18\\
2.67856781378556	-2.41414580832622e-17\\
2.67856781378561	2.31020613067085e-16\\
2.67962024863202	1.48756581323845e-16\\
2.67962024863205	-5.44073084823079e-19\\
2.68066900430601	-2.18815013880938e-17\\
2.68066900430606	2.08257727575259e-16\\
2.6817141064419	1.11617104879945e-18\\
2.68171410644193	-1.47143830592273e-16\\
2.68275558040695	-1.47463889932327e-16\\
2.68275558040698	2.82478870464409e-19\\
2.68379345130516	1.64135662238774e-17\\
2.68379345130519	-1.60726685498321e-16\\
2.68482774398096	-2.24019700469571e-18\\
2.68482774398099	1.44484544184589e-16\\
2.68585848302262	1.4445120549718e-16\\
2.68585848302265	-1.77710821056402e-18\\
2.68688569276584	-7.55196046926213e-19\\
2.68688569276587	1.44969867593586e-16\\
2.68790939729721	1.44853380711004e-16\\
2.68790939729724	-3.72754053795996e-19\\
2.68892962045758	-7.84442506408128e-18\\
2.68892962045762	2.02473868974051e-16\\
2.68994638584543	1.43787091961864e-16\\
2.68994638584546	-4.61632956253594e-19\\
2.69095971682003	-7.8806158702417e-18\\
2.69095971682006	1.56095612268547e-16\\
2.6919696365048	6.63067979022507e-21\\
2.69196963650483	-1.43268410013123e-16\\
2.69297616779042	-2.44922223676541e-18\\
2.69297616779045	1.40349382275652e-16\\
2.69397933333787	1.42188866521848e-16\\
2.6939793333379	-1.32719474922908e-19\\
2.69497915558157	-3.5254371532561e-18\\
2.6949791555816	1.82652517149802e-16\\
2.69597565673242	1.40446433722562e-16\\
2.69597565673245	-9.3423101870719e-19\\
2.69696885878067	-3.79230547120041e-18\\
2.69696885878072	2.3839361135646e-16\\
2.69795878349902	5.45978908954401e-19\\
2.69795878349905	-1.72811059711336e-16\\
2.6989454524452	-1.54927620008732e-19\\
2.69894545244523	1.3982853642243e-16\\
2.69992888696491	1.39444572416973e-16\\
2.69992888696494	-8.25526251327922e-20\\
2.70090910819461	-1.8561694055929e-16\\
2.70090910819465	1.25475357990414e-18\\
2.70188613706432	8.96749810791188e-19\\
2.70188613706435	-1.37724301440658e-16\\
2.70285999430025	-4.0791915599685e-19\\
2.70285999430027	1.37762001024256e-16\\
2.70383070042738	1.37516427644001e-16\\
2.70383070042741	-2.07198817277183e-19\\
2.70479827577206	-2.03115390552478e-17\\
2.70479827577209	1.16967309081938e-16\\
2.70576274046464	1.34845219909329e-16\\
2.70576274046467	-1.99289978004441e-18\\
2.70672411444193	-2.14747667547342e-18\\
2.70672411444196	1.3425584710544e-16\\
2.70768241744969	1.35425808055112e-16\\
2.70768241744971	-5.39014048196704e-19\\
2.70863766904503	-2.93386173283206e-17\\
2.70863766904506	1.06192609932778e-16\\
2.70958988859885	1.34289331376636e-16\\
2.70958988859888	-8.12202026020917e-19\\
2.7105390952982	-2.67291834417935e-19\\
2.71053909529823	1.6176476407985e-16\\
2.7114853081486	5.95341739189115e-19\\
2.71148530814863	-1.52537637138715e-16\\
2.71242854597628	-4.4519522312094e-19\\
2.71242854597632	1.58478630277806e-16\\
2.71336882743041	1.47390300636312e-16\\
2.71336882743044	-6.11895277607526e-19\\
2.71430617098532	-3.90881220611728e-18\\
2.71430617098536	1.66491677764117e-16\\
2.71524059494281	2.0247727034073e-19\\
2.71524059494286	-2.52533909894752e-16\\
2.71617211743423	-4.60005594203681e-17\\
2.71617211743426	9.03035291568182e-17\\
2.71710075642234	1.3107209179345e-16\\
2.71710075642237	-6.92763763013161e-19\\
2.71802652970369	-6.20087674394252e-20\\
2.71802652970375	2.54442827681565e-16\\
2.71894945491077	3.727587852691e-17\\
2.71894945491082	-1.98031260575041e-16\\
2.71986954951366	-5.31672224690255e-17\\
2.71986954951369	1.10020315425473e-16\\
2.72078683082211	1.63157997620463e-16\\
2.72078683082214	-1.56664368422848e-18\\
2.72170131598764	-1.29455946312644e-16\\
2.72170131598766	3.01824970630799e-19\\
2.72261302200538	1.28585835026299e-16\\
2.7226130220054	-7.77592789191699e-19\\
2.72352196571604	-1.28660112536584e-16\\
2.72352196571607	3.14460981668159e-19\\
2.72442816380774	5.21954275243411e-19\\
2.72442816380777	-1.2806160435001e-16\\
2.72533163281786	-1.66031692551644e-19\\
2.72533163281791	2.22177155655109e-16\\
2.72623238913484	3.23684574893717e-17\\
2.72623238913489	-1.67341073139201e-16\\
2.72713044899984	-4.63767280757328e-17\\
2.72713044899987	9.89750885113382e-17\\
2.72802582850852	1.3648395072958e-16\\
2.72802582850855	-4.95438173527411e-19\\
2.72891854361289	-1.24911435879612e-16\\
2.72891854361292	1.76365163635901e-18\\
2.72980861012304	7.88776271073416e-19\\
2.72980861012307	-1.25506602256511e-16\\
2.73069604370879	-5.67767871993007e-17\\
2.73069604370883	1.24238931424161e-16\\
2.73158085990131	5.56280119994769e-17\\
2.73158085990135	-1.18966859394108e-16\\
2.73246307409468	-5.52621136185977e-17\\
2.73246307409472	1.22737860001451e-16\\
2.73334270154762	5.19265673554817e-17\\
2.73334270154765	-1.11897939315352e-16\\
2.73421975738495	-5.59982845114352e-17\\
2.73421975738499	1.2096372633237e-16\\
2.7350942565993	4.81905564880492e-17\\
2.73509425659933	-1.06923950090838e-16\\
2.73596621405251	-5.43063924475764e-17\\
2.73596621405255	1.17757564351947e-16\\
2.73683564447724	4.51484508928752e-17\\
2.73683564447728	-1.01358811278479e-16\\
2.73770256247841	-5.57602384747547e-17\\
2.73770256247845	1.17234766352479e-16\\
2.73856698253471	4.29592479185822e-17\\
2.73856698253474	-9.50381190771696e-17\\
2.73942891900001	-5.46644359213587e-17\\
2.73942891900005	1.17337320204823e-16\\
2.74028838610483	4.08724598302905e-17\\
2.74028838610486	-8.49014872001977e-17\\
2.74114539795771	-5.83573572963221e-17\\
2.74114539795775	1.24062589752455e-16\\
2.74199996854667	3.42788344227854e-17\\
2.74199996854672	-1.81731775792525e-16\\
2.74285211174055	-1.63039449403794e-17\\
2.74285211174061	2.17986761242312e-16\\
2.74370184129032	1.06288407373879e-17\\
2.74370184129036	-1.49519324566681e-16\\
2.74454917083035	-2.31622284392321e-17\\
2.74454917083038	1.23382865664618e-16\\
2.74539411387982	2.85520744449011e-17\\
2.74539411387987	-1.56924410846378e-16\\
2.74623668384409	-1.84093512098984e-17\\
2.74623668384412	1.04895769403307e-16\\
2.74707689401577	3.23986959038442e-17\\
2.74707689401582	-1.81846906098902e-16\\
2.74791475757625	-1.31994070612924e-17\\
2.74791475757629	1.63296553681788e-16\\
2.74875028759667	1.49686901432616e-17\\
2.74875028759672	-1.88822235654431e-16\\
2.7495834970393	-1.17753196838313e-17\\
2.74958349703934	1.37875523291523e-16\\
2.75041439875865	1.84589291070924e-17\\
2.75041439875868	-1.04981286643755e-16\\
2.75124300550273	-2.52694743584334e-17\\
2.75124300550277	1.51112358621172e-16\\
2.75206932991427	1.29952489599469e-17\\
2.75206932991432	-1.81227940081419e-16\\
2.75289338453182	-9.5107307772326e-18\\
2.75289338453185	1.23877814147223e-16\\
2.75371518179078	1.72549842691337e-17\\
2.75371518179081	-1.04836007258411e-16\\
2.7545347340247	-2.31845590070486e-17\\
2.75453473402474	1.2583283898138e-16\\
2.75535205346634	1.5268483558308e-17\\
2.75535205346639	-2.04025461279854e-16\\
2.75616715224882	-4.1774010597357e-18\\
2.75616715224886	1.65718121616315e-16\\
2.7569800424065	7.74469024326454e-18\\
2.75698004240654	-1.18433417234517e-16\\
2.7577907358762	-1.47676521671868e-17\\
2.75779073587625	1.97358521200297e-16\\
2.75859924449831	3.84674388167375e-18\\
2.75859924449835	-1.53919865868459e-16\\
2.75940558001767	-8.16791224059362e-18\\
2.7594055800177	1.16993848344712e-16\\
2.76020975408464	1.48630095482019e-17\\
2.76020975408469	-1.77727437460547e-16\\
2.76101177825626	-5.35574962081954e-18\\
2.7610117782563	1.63598835256117e-16\\
2.76181166399704	8.01215413124988e-18\\
2.76181166399709	-1.83547314814348e-16\\
2.76260942268007	-3.36596665732323e-18\\
2.76260942268011	1.38155782009223e-16\\
2.76340506558789	7.44259395739282e-18\\
2.76340506558792	-1.14300134679933e-16\\
2.76419860391352	-1.14497844081223e-17\\
2.76419860391356	1.50441498600779e-16\\
2.76499004876145	6.40290402803002e-18\\
2.76499004876149	-1.81386252524494e-16\\
2.76577941114852	-5.34712709011429e-18\\
2.76577941114855	1.11933878658606e-16\\
2.76656670200472	9.84048136534319e-18\\
2.76656670200476	-1.3680744783226e-16\\
2.76735193217434	-5.28980347366436e-18\\
2.76735193217439	1.93216146276419e-16\\
2.76813511241673	2.60169478471609e-18\\
2.76813511241677	-1.64127625351943e-16\\
2.76891625340713	-3.63224271394895e-18\\
2.76891625340716	1.17623964756852e-16\\
2.76969536573759	7.03610747821691e-18\\
2.76969536573762	-1.08723875251419e-16\\
2.77047245991788	-9.18930493480674e-18\\
2.77047245991791	1.1660835240076e-16\\
2.77124754637634	6.84837056639188e-18\\
2.77124754637639	-2.06282906654682e-16\\
2.77202063546074	-2.04958812178389e-19\\
2.77202063546079	1.88379058752005e-16\\
2.77279173743899	1.94383142610776e-18\\
2.77279173743903	-1.21174243798265e-16\\
2.77356086249995	-4.53576217087327e-18\\
2.7735608625	1.69436841100222e-16\\
2.77432802075444	1.17513579158468e-18\\
2.77432802075448	-1.48536673703637e-16\\
2.77509322223584	-1.93767058490547e-18\\
2.7750932222359	1.99996568918573e-16\\
2.77585647690103	1.93159411612565e-18\\
2.77585647690106	-1.45323235842546e-16\\
2.77661779463095	-3.75801387172641e-18\\
2.776617794631	1.87021618970859e-16\\
2.77737718523164	9.84477277836111e-19\\
2.77737718523169	-1.62368427256002e-16\\
2.77813465843478	-1.05585303853209e-18\\
2.77813465843481	1.2492842120631e-16\\
2.77889022389849	3.10545935828946e-18\\
2.77889022389852	-1.14183935656204e-16\\
2.77964389120815	-6.04045499030521e-18\\
2.77964389120818	1.29341881414895e-16\\
2.78039566987709	4.01864359349085e-18\\
2.78039566987714	-1.92711164092984e-16\\
2.78114556934732	-1.91101952040072e-16\\
2.78114556934737	1.46209136352423e-19\\
2.78189359899009	1.05776230373476e-16\\
2.78189359899012	-3.92100153962386e-19\\
2.78263976810677	-1.16031933245263e-18\\
2.78263976810679	1.04745391276667e-16\\
2.78338408592962	4.17118397025175e-18\\
2.78338408592965	-1.24580245997235e-16\\
2.78412656162234	-1.85946065798858e-18\\
2.78412656162239	1.64446126283722e-16\\
2.78486720428075	1.02128185201105e-18\\
2.78486720428078	-1.28737791413547e-16\\
2.78560602293338	-1.25429697620601e-18\\
2.78560602293341	1.36376313025191e-16\\
2.78634302654226	1.85543394276667e-18\\
2.78634302654229	-1.07655569835945e-16\\
2.78707822400351	-2.63170655170073e-18\\
2.78707822400356	1.89762768135045e-16\\
2.7878116241481	8.17569485210488e-17\\
2.78781162414814	-8.08915521641959e-17\\
2.78854323574225	-6.07403190572996e-18\\
2.78854323574229	1.4968874273666e-16\\
2.78927306748829	7.78643308569641e-19\\
2.78927306748832	-1.22233036636694e-16\\
2.79000112802516	-1.66619817967169e-18\\
2.79000112802519	1.12974910274923e-16\\
2.79072742592905	3.15254247875597e-18\\
2.79072742592909	-1.30539455307776e-16\\
2.79145196971405	-4.0314963258487e-19\\
2.7914519697141	1.63495096881136e-16\\
2.7921747678327	1.11597151629209e-16\\
2.79217476783273	-6.15399214682301e-19\\
2.79289582867648	-6.75310252088292e-20\\
2.79289582867651	1.02278019042901e-16\\
2.79361516057663	1.17088739195861e-18\\
2.79361516057667	-1.39217372372484e-16\\
2.79433277180466	-1.08553062454926e-18\\
2.79433277180471	1.94671547403149e-16\\
2.79504867057283	8.01584501114309e-18\\
2.79504867057285	-9.35968726650817e-17\\
2.79576286503459	-3.32485081779567e-18\\
2.79576286503464	1.47147802713575e-16\\
2.79647536328545	1.16694632301733e-16\\
2.79647536328548	-2.39791387772997e-19\\
2.79718617336326	-1.58846801751008e-19\\
2.79718617336329	1.00733833312304e-16\\
2.79789530324895	1.72784464200282e-18\\
2.79789530324898	-1.06790691294156e-16\\
2.79860276086703	-5.79684864554712e-19\\
2.79860276086708	1.70439999425012e-16\\
2.79930855408611	1.61404808164492e-17\\
2.79930855408614	-8.40403839107239e-17\\
2.80001269071921	-2.51560678964524e-18\\
2.80001269071925	1.42720920672371e-16\\
2.80071517852462	7.49664739800908e-17\\
2.80071517852466	-7.46036692547026e-17\\
2.80141602520618	-3.52206225635689e-18\\
2.80141602520623	1.69012522261401e-16\\
2.80211523841395	7.14687688840511e-18\\
2.80211523841398	-9.21016858052492e-17\\
2.80281282574449	-2.91712025897848e-18\\
2.80281282574454	1.85835731485672e-16\\
2.80350879474166	2.97387973825065e-19\\
2.80350879474169	-9.84944833405551e-17\\
2.80420315289676	-1.35223350637106e-18\\
2.8042031528968	1.37250396432837e-16\\
2.8048959076493	4.9133970865917e-17\\
2.80489590764933	-4.92021917597567e-17\\
2.80558706638728	-4.62580408927212e-18\\
2.80558706638732	1.40999772884607e-16\\
2.80627663644787	2.92859126884112e-17\\
2.8062766364479	-6.85950838790378e-17\\
2.80696462511768	-2.7793483648741e-18\\
2.80696462511773	1.43704921494174e-16\\
2.80765103963339	1.43164499584575e-17\\
2.80765103963342	-8.311872934605e-17\\
2.80833588718201	-1.65370940678336e-18\\
2.80833588718206	1.65428611710644e-16\\
2.80901917490158	4.26336994079833e-18\\
2.80901917490161	-9.27259689397104e-17\\
2.80970090988131	-1.1379920733586e-18\\
2.80970090988134	1.04704138125158e-16\\
2.81038109916226	1.21967015351697e-16\\
2.81038109916229	-2.29640227295746e-19\\
2.81105974973766	-9.59731276456526e-17\\
2.81105974973769	3.60379431832203e-19\\
2.8117368685534	9.57526591969888e-17\\
2.81173686855343	-3.63492939657852e-19\\
2.8124124625085	-9.52037768840844e-17\\
2.81241246250853	6.96567139469679e-19\\
2.81308653845549	2.23858234814001e-19\\
2.81308653845552	-9.54608249137965e-17\\
2.81375910320083	-4.7857692392113e-19\\
2.81375910320087	1.24825720115817e-16\\
2.81443016350534	1.48208266456841e-17\\
2.81443016350537	-8.04332725981455e-17\\
2.81509972608444	-8.38440195051099e-19\\
2.81509972608447	1.04601317920656e-16\\
2.81576779760874	5.81707023367383e-17\\
2.81576779760878	-5.88878265687912e-17\\
2.81643438470438	-1.79681992456116e-18\\
2.81643438470441	1.09088954723491e-16\\
2.81709949395344	3.35705494427583e-17\\
2.81709949395346	-6.08426661144908e-17\\
2.81776313189428	-1.77707512333952e-18\\
2.81776313189434	1.8663194080173e-16\\
2.81842530502213	2.32085538622576e-18\\
2.81842530502219	-1.8567261643196e-16\\
2.81908601978918	-3.11122977024263e-18\\
2.81908601978921	9.50749550842078e-17\\
2.81974528260494	6.00477885592192e-17\\
2.81974528260497	-5.83932949852151e-17\\
2.82040309983689	-1.50661684153673e-18\\
2.82040309983695	1.73578274692563e-16\\
2.82105947781082	1.82334926116696e-18\\
2.82105947781087	-1.51041953970515e-16\\
2.82171442281096	-6.05032987245787e-19\\
2.821714422811	1.34493094085648e-16\\
2.82236794108047	2.40913954582293e-17\\
2.8223679410805	-6.86773618802072e-17\\
2.82302003882176	-4.50777037894448e-19\\
2.82302003882182	1.80347039051164e-16\\
2.82367072219707	4.46425075255684e-18\\
2.82367072219711	-1.32644489346019e-16\\
2.82431999732843	-7.62603647834972e-17\\
2.82431999732847	7.78339236976467e-17\\
2.82496787029827	8.10243262330392e-19\\
2.8249678702983	-9.25964109490886e-17\\
2.82561434714966	-2.73660744631512e-17\\
2.82561434714969	6.44052817275587e-17\\
2.8262594338867	1.70211590042468e-16\\
2.82625943388675	-5.787286437666e-20\\
2.82690313647493	-4.4566837043132e-20\\
2.82690313647495	9.13349936021698e-17\\
2.82754546084146	2.11019647244626e-17\\
2.82754546084149	-7.00806622088749e-17\\
2.82818641287554	-8.02436331709116e-19\\
2.82818641287557	1.02979102588515e-16\\
2.82882599842869	1.61485177032807e-18\\
2.82882599842871	-8.9179930026118e-17\\
2.82946422331506	-1.74213666862651e-17\\
2.82946422331508	7.31776264222734e-17\\
2.83010109331176	6.79028329680044e-17\\
2.8301010933118	-6.77102388516225e-17\\
2.83073661415923	-9.18553100598003e-17\\
2.83073661415926	1.18256552784664e-18\\
2.8313707915614	9.77379323717855e-17\\
2.83137079156143	-7.28224995873309e-19\\
2.83200363118622	-2.39971713146534e-19\\
2.83200363118625	8.95964535148466e-17\\
2.83263513866586	4.97475850248374e-18\\
2.83263513866589	-8.4673177079327e-17\\
2.83326531959698	-1.27235151656396e-17\\
2.83326531959701	7.67389433617796e-17\\
2.83389417954101	2.68666889837407e-17\\
2.83389417954104	-6.24046229933685e-17\\
2.83452172402446	-8.74952344335884e-17\\
2.83452172402451	8.65005712536893e-17\\
2.83514795853934	3.9110604830222e-18\\
2.83514795853937	-8.49899558210715e-17\\
2.83577288854319	-5.21162332524952e-18\\
2.83577288854322	8.35061327684943e-17\\
2.8363965194596	4.54245075330638e-18\\
2.83639651945963	-8.39875171434144e-17\\
2.83701885667841	-3.478526086357e-18\\
2.83701885667844	8.48683139489119e-17\\
2.83763990555595	2.72935687737384e-18\\
2.83763990555598	-8.5434368531333e-17\\
2.83825967141537	-5.28280552568216e-19\\
2.8382596714154	8.74561402234779e-17\\
2.83887815954691	3.12490637449058e-19\\
2.83887815954694	-8.74888502536207e-17\\
2.8394953752081	-9.5713672520358e-17\\
2.83949537520813	1.22275492508814e-19\\
2.84011132362409	8.67865639873959e-17\\
2.84011132362412	-6.56331420113875e-19\\
2.84072600998794	-2.05317808564237e-19\\
2.84072600998797	8.70562695348412e-17\\
2.84133943946092	2.54027166115035e-20\\
2.84133943946095	-8.70580397013521e-17\\
2.84195161717268	-1.81523131063537e-19\\
2.84195161717271	8.67247951023341e-17\\
2.84256254822148	9.39456682596157e-17\\
2.84256254822151	-9.14699630083588e-19\\
2.84317223767448	-1.49727962678224e-19\\
2.84317223767453	1.58081995519725e-16\\
2.84378069056816	1.15203809646812e-17\\
2.8437806905682	-1.22095311064704e-16\\
2.8443879119083	-2.40424263659662e-18\\
2.84438791190833	1.08043773142042e-16\\
2.84499390667032	1.21094278594088e-18\\
2.84499390667035	-1.049819417708e-16\\
2.84559867979957	-1.65383431105586e-18\\
2.8455986797996	9.09114978080579e-17\\
2.84620223621155	1.76505701621974e-18\\
2.84620223621159	-1.29436735042917e-16\\
2.84680458079221	-5.70071288757559e-18\\
2.84680458079226	1.62651642317211e-16\\
2.84740571839811	1.52694406062767e-17\\
2.84740571839816	-1.38078902188731e-16\\
2.84800565385664	-8.56668939646053e-18\\
2.84800565385667	1.00558049987511e-16\\
2.84860439196622	1.21152512109212e-18\\
2.84860439196625	-9.97256399391249e-17\\
2.84920193749666	-4.53632465368981e-19\\
2.8492019374967	1.02933895513678e-16\\
2.84979829518935	1.6823827955369e-18\\
2.8497982951894	-1.39864819872819e-16\\
2.85039346975748	-1.21735707528286e-17\\
2.85039346975752	1.10609089350598e-16\\
2.85098746588618	2.72906240107578e-18\\
2.85098746588622	-1.13222725314637e-16\\
2.85158028823283	-4.00897302494949e-18\\
2.85158028823288	1.30123386768384e-16\\
2.85217194142727	1.04095046989827e-17\\
2.85217194142731	-9.85234512056067e-17\\
2.85276243007196	-2.19869735336393e-18\\
2.85276243007201	1.44506112919061e-16\\
2.85335175874231	1.45776156932588e-17\\
2.85335175874235	-1.25302520255782e-16\\
2.85393993198675	-9.89132567605499e-18\\
2.85393993198679	9.44861034041048e-17\\
2.854526954327	1.83766444884781e-18\\
2.85452695432704	-1.34890383970774e-16\\
2.85511283025835	-1.14896354884932e-17\\
2.8551128302584	1.15875463760593e-16\\
2.85569756424978	6.56891917696941e-18\\
2.85569756424981	-8.42283789029001e-17\\
2.85628116074411	-6.25101216086936e-19\\
2.85628116074416	1.45661910882285e-16\\
2.85686362415846	1.51429721352429e-17\\
2.85686362415851	-1.33444784107122e-16\\
2.85744495888414	-1.2513917524586e-17\\
2.85744495888418	1.17732052447495e-16\\
2.85802516928699	8.66387567381e-18\\
2.85802516928702	-9.43025387912895e-17\\
2.85860425970755	-4.28960562828668e-18\\
2.85860425970759	9.59095662602034e-17\\
2.85918223446131	3.56229383980963e-18\\
2.85918223446135	-1.09265317901894e-16\\
2.85975909783889	-8.07365418261259e-18\\
2.85975909783892	8.1502037098859e-17\\
2.86033485410614	8.60286862209238e-19\\
2.86033485410617	-1.00039358939784e-16\\
2.86090950750445	-5.50190581761276e-18\\
2.8609095075045	1.39819906447099e-16\\
2.86148306225093	1.57190149235605e-17\\
2.86148306225098	-1.30598687229899e-16\\
2.86205552253848	-1.31964895952719e-17\\
2.86205552253853	1.22683440616132e-16\\
2.86262689253604	1.20699175774502e-17\\
2.86262689253608	-1.13410936200953e-16\\
2.86319717638876	-1.03225812514932e-17\\
2.8631971763888	9.84726962304011e-17\\
2.86376637821822	7.23109876401404e-18\\
2.86376637821828	-1.50603766621544e-16\\
2.86433450212265	-1.89870367789126e-17\\
2.8643345021227	1.39809268904422e-16\\
2.86490155217696	1.72346273542588e-17\\
2.86490155217701	-1.31199394062541e-16\\
2.865467532433	-1.50766653373406e-17\\
2.86546753243305	1.25544061130551e-16\\
2.86603244691974	1.36009205306256e-17\\
2.86603244691979	-1.19237300305118e-16\\
2.86659629964343	-1.20367939570011e-17\\
2.86659629964348	1.1304678730707e-16\\
2.86715909458777	1.06636691770794e-17\\
2.86715909458781	-1.06692421410204e-16\\
2.86772083571407	-1.00315551237593e-17\\
2.86772083571411	9.71356729211336e-17\\
2.86828152696145	7.1970089966095e-18\\
2.86828152696148	-8.23578059005624e-17\\
2.86884117224696	-4.08708038147042e-18\\
2.868841172247	1.07647837351579e-16\\
2.86939977546583	1.16035994702071e-17\\
2.86939977546587	-9.99248382655175e-17\\
2.86995734049153	-8.82505926079147e-18\\
2.86995734049156	9.01271064214117e-17\\
2.87051387117595	5.72697669920476e-18\\
2.87051387117598	-7.45220108118119e-17\\
2.87106937134961	-2.78236095666502e-18\\
2.87106937134964	8.34805083674786e-17\\
2.87162384482178	5.29685436120576e-18\\
2.87162384482183	-1.31236829775523e-16\\
2.87217729538075	-1.6350039580014e-17\\
2.8721772953808	1.24844564117351e-16\\
2.87272972679377	1.43384331104167e-17\\
2.87272972679382	-1.2169425668061e-16\\
2.87328114280734	-1.45156433217663e-17\\
2.87328114280738	1.18820966240749e-16\\
2.87383154714733	1.44644681388095e-17\\
2.87383154714737	-1.14964579914675e-16\\
2.87438094351914	-1.25799819585688e-17\\
2.87438094351919	1.14174066564315e-16\\
2.87492933560786	1.3298619103962e-17\\
2.87492933560791	-1.1322478168564e-16\\
2.87547672707841	-1.29328514176196e-17\\
2.87547672707845	1.12145411902293e-16\\
2.87602312157567	1.31092096329384e-17\\
2.87602312157571	-1.11740430891832e-16\\
2.87656852272465	-1.29303582952534e-17\\
2.8765685227247	1.10483271870707e-16\\
2.87711293413063	1.36994195436072e-17\\
2.87711293413068	-1.09490382993222e-16\\
2.87765635937929	-1.37128984516887e-17\\
2.87765635937934	1.08047884229686e-16\\
2.87819880203686	1.27521687020584e-17\\
2.87819880203691	-1.07585561404802e-16\\
2.87874026565027	-1.20041137053604e-17\\
2.87874026565032	1.08116674223787e-16\\
2.87928075374727	1.33457146321377e-17\\
2.87928075374731	-1.06559294962419e-16\\
2.87982026983657	-1.24783917716381e-17\\
2.87982026983662	1.07210474701678e-16\\
2.88035881740801	1.34190681816783e-17\\
2.88035881740805	-1.06055299605455e-16\\
2.88089639993264	-1.36137252865079e-17\\
2.88089639993268	1.04454112031666e-16\\
2.8814330208629	1.2107202743561e-17\\
2.88143302086294	-1.04559052300077e-16\\
2.88196868363273	-1.28327956507252e-17\\
2.88196868363277	1.03625909023876e-16\\
2.88250339165771	1.23382698574134e-17\\
2.88250339165775	-1.03912538686654e-16\\
2.88303714833519	-1.26951569326994e-17\\
2.88303714833524	1.04534788611169e-16\\
2.88356995704443	1.27806832721344e-17\\
2.88356995704448	-1.04240463687153e-16\\
2.8841018211467	-1.28381566082751e-17\\
2.88410182114674	1.02793891886281e-16\\
2.88463274398542	1.17390023190371e-17\\
2.88463274398547	-1.03690026274697e-16\\
2.88516272888632	-1.26531005888154e-17\\
2.88516272888636	1.02572453154759e-16\\
2.8856917791575	1.20771959396601e-17\\
2.88569177915754	-1.02944122774772e-16\\
2.8862198980896	-1.26606949340192e-17\\
2.88621989808964	1.00987465086086e-16\\
2.88674708895593	1.20741709401599e-17\\
2.88674708895597	-1.00204865372646e-16\\
2.88727335501255	-1.24648839122351e-17\\
2.8872733550126	9.96172883778672e-17\\
2.88779869949844	1.15189499422215e-17\\
2.88779869949848	-1.00366447660683e-16\\
2.88832312563555	-1.2095112671728e-17\\
2.8883231256356	9.95951191102702e-17\\
2.88884663662902	1.19507142337364e-17\\
2.88884663662906	-9.83846618793489e-17\\
2.88936923566719	-1.23179634005559e-17\\
2.88936923566723	9.66637829252844e-17\\
2.8898909259218	1.10732807446096e-17\\
2.88989092592184	-9.77201115926687e-17\\
2.89041171054805	-1.1325921556018e-17\\
2.89041171054809	9.72783690214245e-17\\
2.89093159268475	1.22398659702023e-17\\
2.89093159268479	-9.61756123606188e-17\\
2.89145057545442	-1.19224900040555e-17\\
2.89145057545446	9.51563182236198e-17\\
2.89196866196339	1.19956508205133e-17\\
2.89196866196343	-9.48976861430346e-17\\
2.89248585530194	-1.13273037748215e-17\\
2.89248585530199	9.42338364783333e-17\\
2.89300215854439	1.17727901655181e-17\\
2.89300215854443	-9.36073381541917e-17\\
2.89351757474921	-1.06561385247827e-17\\
2.89351757474925	9.33994329246714e-17\\
2.89403210695913	1.12189886095108e-17\\
2.89403210695917	-9.26598813446704e-17\\
2.89454575820125	-1.15908062948534e-17\\
2.89454575820129	9.21096705797022e-17\\
2.89505853148717	1.18333058719519e-17\\
2.89505853148721	-9.05511580808951e-17\\
2.89557042981305	-1.09206113276731e-17\\
2.89557042981309	9.01544414319341e-17\\
2.89608145615975	1.14615766949348e-17\\
2.89608145615979	-8.83073355435746e-17\\
2.89659161349291	-1.09903062573644e-17\\
2.89659161349295	8.74784914708097e-17\\
2.89710090476308	1.01759643150988e-17\\
2.89710090476311	-8.81245584067024e-17\\
2.89760933290579	-1.18885540922449e-17\\
2.89760933290583	8.62445605084654e-17\\
2.89811690084171	1.04190065226791e-17\\
2.89811690084175	-8.52965867110612e-17\\
2.89862361147667	-1.05848661616633e-17\\
2.8986236114767	8.49703789375142e-17\\
2.89912946770181	1.04374799472992e-17\\
2.89912946770185	-8.4956075330362e-17\\
2.89963447239368	-1.00453318168613e-17\\
2.89963447239372	8.51874989590584e-17\\
2.90013862841433	1.12122520506021e-17\\
2.90013862841436	-8.27429891702427e-17\\
2.90064193861138	-1.04516387978443e-17\\
2.90064193861142	8.33451013827455e-17\\
2.90114440581818	9.92206701633991e-18\\
2.90114440581822	-8.37181581846457e-17\\
2.90164603285383	-1.04496295761364e-17\\
2.90164603285387	8.30349946798218e-17\\
2.90214682252334	1.01419523026718e-17\\
2.90214682252338	-8.20760516403146e-17\\
2.90264677761769	-9.63700072491943e-18\\
2.90264677761772	8.13174012871504e-17\\
2.9031459009139	9.72489941835115e-18\\
2.90314590091394	-8.10778679797285e-17\\
2.9036441951752	-1.08629957164669e-17\\
2.90364419517523	7.97890271139617e-17\\
2.90414166315103	9.8903247875138e-18\\
2.90414166315107	-7.95085442303154e-17\\
2.90463830757721	-9.31604472108151e-18\\
2.90463830757725	7.88315779626407e-17\\
2.90513413117597	9.80464293812372e-18\\
2.90513413117601	-7.70984418462594e-17\\
2.90562913665609	-9.29817100240546e-18\\
2.90562913665612	7.74607022868046e-17\\
2.90612332671293	1.00590696324746e-17\\
2.90612332671296	-7.65570389004537e-17\\
2.90661670402858	-9.94766494757597e-18\\
2.90661670402861	7.54319971667652e-17\\
2.9071092712719	8.44036899075633e-18\\
2.90710927127194	-7.67982406311711e-17\\
2.90760103109865	-9.85529120703034e-18\\
2.90760103109868	7.52440294715493e-17\\
2.90809198615152	8.78892635727393e-18\\
2.90809198615156	-7.50832354866363e-17\\
2.90858213906028	-9.42204460785773e-18\\
2.90858213906031	7.43136175457302e-17\\
2.9090714924418	9.12397751874976e-18\\
2.90907149244183	-7.33888809060939e-17\\
2.90956004890018	-9.81718051466818e-18\\
2.90956004890021	7.25622859374146e-17\\
2.91004781102682	9.4563345529705e-18\\
2.91004781102685	-7.17072061771614e-17\\
2.9105347814005	-9.88663390762059e-18\\
2.91053478140053	7.00643237374157e-17\\
2.91102096258745	8.62117743777062e-18\\
2.91102096258748	-7.01200442778385e-17\\
2.91150635714146	-8.47426354353631e-18\\
2.91150635714149	7.12182357288435e-17\\
2.91199096760393	8.71850599018545e-18\\
2.91199096760397	-6.97689234786026e-17\\
2.91247479650399	-8.83002399882801e-18\\
2.91247479650402	6.95323655398441e-17\\
2.91295784635852	7.7288728955913e-18\\
2.91295784635855	-6.94351494839926e-17\\
2.91344011967228	-7.97298175257151e-18\\
2.91344011967232	6.90680963589783e-17\\
2.91392161893797	8.19384352561654e-18\\
2.913921618938	-6.87226338887226e-17\\
2.9144023466363	-7.57518811445293e-18\\
2.91440234663633	6.92187201582599e-17\\
2.91488230523608	8.76563336481474e-18\\
2.91488230523611	-6.68395344767787e-17\\
2.91536149719428	-8.68531412733584e-18\\
2.91536149719431	6.68004721131218e-17\\
2.91583992495612	8.65111972158382e-18\\
2.91583992495615	-6.67128579698208e-17\\
2.91631759095515	-8.64412921822713e-18\\
2.91631759095518	6.55406619781332e-17\\
2.91679449761329	7.66751139698119e-18\\
2.91679449761333	-6.63994975519577e-17\\
2.91727064734097	-7.6374834999746e-18\\
2.917270647341	6.52562603263349e-17\\
2.9177460425371	8.46320207079494e-18\\
2.91774604253713	-6.43145859228856e-17\\
2.91822068558927	-7.68334487467638e-18\\
2.9182206855893	6.49798223558839e-17\\
2.9186945788737	7.14254949303054e-18\\
2.91869457887373	-6.43550193384078e-17\\
2.91916772475541	-8.48678747160066e-18\\
2.91916772475544	6.2897961368911e-17\\
2.91964012558821	7.0873732374508e-18\\
2.91964012558824	-6.31370347639123e-17\\
2.92011178371484	-7.61803358864133e-18\\
2.92011178371486	6.2495628577181e-17\\
2.92058270146698	8.46661708595387e-18\\
2.92058270146701	-6.04923402358236e-17\\
2.92105288116537	-6.88527150710817e-18\\
2.92105288116539	6.09211676225736e-17\\
2.92152232511983	8.42497061006335e-18\\
2.92152232511986	-6.03183531337307e-17\\
2.9219910356294	-7.14371664487485e-18\\
2.92199103562942	6.04509460500199e-17\\
2.9224590149823	7.41435607923179e-18\\
2.92245901498233	-6.00743460199658e-17\\
2.92292626545611	-7.40668894213404e-18\\
2.92292626545613	5.99790663939283e-17\\
2.92339278931775	6.17181540748169e-18\\
2.9233927893178	-1.26318919674228e-16\\
2.92385858882371	-6.59309601191311e-18\\
2.92385858882376	1.01922198617015e-16\\
2.92432366621984	7.56957031732277e-18\\
2.92432366621989	-1.11095865288751e-16\\
2.92478802374159	-7.03152820634582e-18\\
2.92478802374164	1.06300145893952e-16\\
2.92525166361406	7.17323913452718e-18\\
2.9252516636141	-1.07011580777751e-16\\
2.92571458805202	-6.81601622391248e-18\\
2.92571458805206	1.06164731878183e-16\\
2.92617679926002	7.07801327522946e-18\\
2.92617679926007	-1.04705213459954e-16\\
2.92663829943244	-6.75484456325295e-18\\
2.92663829943249	1.03831941516489e-16\\
2.92709909075354	6.88349596392333e-18\\
2.92709909075359	-1.03533196139279e-16\\
2.92755917539754	-7.27739937162108e-18\\
2.92755917539759	1.01950217705904e-16\\
2.92801855552865	6.88385454671887e-18\\
2.9280185555287	-1.0115547125876e-16\\
2.92847723330118	-6.50648647946605e-18\\
2.92847723330123	1.01368905363494e-16\\
2.92893521085956	7.1097836281964e-18\\
2.92893521085961	-9.95851213917966e-17\\
2.92939249033842	-6.4382949505381e-18\\
2.92939249033846	9.90792764589826e-17\\
2.92984907386262	6.43901377049041e-18\\
2.92984907386267	-9.89186057239809e-17\\
2.93030496354738	-6.8927785847574e-18\\
2.93030496354743	9.62811883173311e-17\\
2.93076016149825	5.6841208852587e-18\\
2.9307601614983	-9.83440067794225e-17\\
2.93121466981123	-5.62741682708757e-18\\
2.93121466981127	9.622598311043e-17\\
2.93166849057279	5.62665582092402e-18\\
2.93166849057284	-9.60726825546861e-17\\
2.93212162585997	-6.49743174689074e-18\\
2.93212162586002	9.40428140051152e-17\\
2.9325740777404	6.11761687938599e-18\\
2.93257407774044	-9.42711377744661e-17\\
2.93302584827236	-6.37598260781387e-18\\
2.93302584827241	9.28591851365911e-17\\
2.93347693950487	5.1227692382103e-18\\
2.93347693950491	-9.29619776305422e-17\\
2.93392735347768	-6.34020580555003e-18\\
2.93392735347773	9.05991378651457e-17\\
2.93437709222142	5.8454583278341e-18\\
2.93437709222146	-9.09476766200995e-17\\
2.93482615775755	-5.69511701073167e-18\\
2.93482615775759	8.99572155177122e-17\\
2.9352745520985	5.71843150988678e-18\\
2.93527455209854	-8.97921032364114e-17\\
2.93572227724767	-6.04566374662245e-18\\
2.93572227724772	8.83284319579172e-17\\
2.93616933519953	6.51936857673565e-18\\
2.93616933519957	-8.77134620499728e-17\\
2.93661572793961	-5.38222725863705e-18\\
2.93661572793965	8.87104512617544e-17\\
2.93706145744462	5.45509483956777e-18\\
2.93706145744466	-8.65202614609905e-17\\
2.93750652568246	-6.08829994411589e-18\\
2.9375065256825	8.57505823087191e-17\\
2.93795093461228	5.16805530879605e-18\\
2.93795093461232	-8.55483995146765e-17\\
2.93839468618454	-5.10852229293e-18\\
2.93839468618459	8.5474304811307e-17\\
2.93883778234108	5.809466458611e-18\\
2.93883778234112	-8.3656378546236e-17\\
2.9392802250151	-5.86288867770182e-18\\
2.93928022501514	8.24887969815954e-17\\
2.9397220161313	5.19748556781007e-18\\
2.93972201613134	-8.40060770084086e-17\\
2.94016315760588	-5.47222781800527e-18\\
2.94016315760592	8.06638922072866e-17\\
2.94060365134659	4.71683986517695e-18\\
2.94060365134663	-8.32462197987306e-17\\
2.94104349925279	-5.66233708382711e-18\\
2.94104349925283	7.92443191680185e-17\\
2.94148270321552	5.46198856279279e-18\\
2.94148270321555	-8.12700836609802e-17\\
2.94192126511749	-4.98026297586997e-18\\
2.94192126511753	7.87061058306547e-17\\
2.94235918683319	4.38735852435575e-18\\
2.94235918683323	-8.01471119467747e-17\\
2.94279647022891	-4.67514623999152e-18\\
2.94279647022895	7.77964933613383e-17\\
2.94323311716279	5.1064932188874e-18\\
2.94323311716283	-7.72448424749237e-17\\
2.94366912948487	-4.886279128262e-18\\
2.9436691294849	7.73464827062814e-17\\
2.94410450903712	5.27085524737261e-18\\
2.94410450903715	-7.68424674969824e-17\\
2.94453925765352	-4.67313701828363e-18\\
2.94453925765355	7.63562602455008e-17\\
2.94497337716008	4.4116970731092e-18\\
2.94497337716012	-7.55381161675496e-17\\
2.9454068693749	-4.04787450197725e-18\\
2.94540686937494	7.48245930662935e-17\\
2.9458397361082	4.94246371700844e-18\\
2.94583973610824	-7.38150269868884e-17\\
2.9462719791624	-3.97554334767377e-18\\
2.94627197916243	7.46688630810906e-17\\
2.9467036003321	4.48422344203724e-18\\
2.94670360033214	-7.30897703286432e-17\\
2.9471346014042	-4.52140107466843e-18\\
2.94713460140424	7.29411900790649e-17\\
2.9475649841579	4.59586221904923e-18\\
2.94756498415793	-7.18010767010254e-17\\
2.94799475036474	-4.90459414779485e-18\\
2.94799475036477	7.13827181121078e-17\\
2.94842390178867	4.08402353551673e-18\\
2.94842390178871	-7.11413918907082e-17\\
2.94885244018609	-4.49080242303918e-18\\
2.94885244018612	7.06271953288397e-17\\
2.94928036730586	4.79720170245739e-18\\
2.9492803673059	-6.9264270099461e-17\\
2.94970768488939	-3.82379623712031e-18\\
2.94970768488942	7.01333089461414e-17\\
2.95013439467064	4.0507183489232e-18\\
2.95013439467067	-6.88543334499502e-17\\
2.95056049837619	-4.59588041702271e-18\\
2.95056049837623	6.82053473936758e-17\\
2.95098599772528	4.18040041663682e-18\\
2.95098599772532	-6.75730571508851e-17\\
2.95141089442984	-4.08024042498522e-18\\
2.95141089442987	6.75715074280075e-17\\
2.95183519019453	3.08266763728148e-18\\
2.95183519019456	-6.75263867935443e-17\\
2.95225888671679	-3.65888785584051e-18\\
2.95225888671683	6.59108625253782e-17\\
2.95268198568689	3.73016435486298e-18\\
2.95268198568693	-6.57421999848883e-17\\
2.95310448878794	-4.14097037119764e-18\\
2.95310448878798	6.52327456494715e-17\\
2.95352639769597	3.82551966792656e-18\\
2.953526397696	-6.54503355096036e-17\\
2.95394771407992	-3.90736712158715e-18\\
2.95394771407995	6.4336210078459e-17\\
2.95436843960174	4.33800313874654e-18\\
2.95436843960177	-6.28772574517691e-17\\
2.95478857591637	-3.69958059248475e-18\\
2.9547885759164	6.43534510039664e-17\\
2.95520812467182	3.88369031732374e-18\\
2.95520812467186	-6.22125586861689e-17\\
2.95562708750921	-3.78274078177746e-18\\
2.95562708750924	6.31508755635118e-17\\
2.95604546606276	4.40498119371588e-18\\
2.95604546606279	-6.05793248946784e-17\\
2.95646326195989	-4.02390611654791e-18\\
2.95646326195992	6.1796508758341e-17\\
2.95688047682122	3.7189109762195e-18\\
2.95688047682125	-6.10839828233133e-17\\
2.95729711226062	-3.9156273471841e-18\\
2.95729711226065	6.0797416439852e-17\\
2.95771316988524	3.94074716802612e-18\\
2.95771316988527	-5.97587187136339e-17\\
2.95812865129556	-3.56960032616031e-18\\
2.95812865129559	6.0042053302442e-17\\
2.95854355808542	3.17044556234376e-18\\
2.95854355808545	-5.94329129498646e-17\\
2.95895789184204	-3.70912832470514e-18\\
2.95895789184207	5.78888624632971e-17\\
2.9593716541461	3.85246714433201e-18\\
2.95937165414613	-5.94958423693258e-17\\
2.95978484657173	-2.95626161838388e-18\\
2.95978484657176	5.75554776571653e-17\\
2.96019747068656	3.57462462120096e-18\\
2.96019747068659	-5.77698816311681e-17\\
2.96060952805177	-3.41223045539301e-18\\
2.9606095280518	5.69329117496558e-17\\
2.96102102022213	3.39832844442814e-18\\
2.96102102022216	-5.77775694477812e-17\\
2.96143194874599	-3.29547432552146e-18\\
2.96143194874602	5.59734430368693e-17\\
2.96184231516537	3.31062868793374e-18\\
2.9618423151654	-5.58764941603557e-17\\
2.96225212101596	-3.42852301832817e-18\\
2.96225212101599	5.56781887754343e-17\\
2.96266136782717	3.1510618254595e-18\\
2.9626613678272	-5.58751301395256e-17\\
2.96307005712216	-3.5866596855233e-18\\
2.96307005712222	1.11583987463351e-16\\
2.96347819041797	2.3474467613775e-18\\
2.96347819041801	-6.91977879392689e-17\\
2.96388576922529	-3.40522299483377e-18\\
2.96388576922534	9.60777023002334e-17\\
2.96429279504878	1.85790142253392e-18\\
2.96429279504882	-7.58138489665284e-17\\
2.96469926938688	-2.55949327063455e-18\\
2.96469926938693	8.85370510443061e-17\\
2.96510519373204	3.07113425800065e-18\\
2.96510519373208	-7.7993270530078e-17\\
2.9655105695706	-2.48492211123228e-18\\
2.96551056957065	8.47673098194452e-17\\
2.96591539838291	2.87059672619636e-18\\
2.96591539838295	-7.79762248191951e-17\\
2.96631968164333	-2.80864586139981e-18\\
2.96631968164337	8.24147134451643e-17\\
2.96672342082025	2.67208247817037e-18\\
2.96672342082029	-7.79569465478389e-17\\
2.96712661737617	-2.28028838103463e-18\\
2.96712661737621	8.0924476020527e-17\\
2.96752927276769	2.15039620944542e-18\\
2.96752927276773	-7.8262226238298e-17\\
2.96793138844553	-2.39785845662893e-18\\
2.96793138844557	7.87987215215295e-17\\
2.96833296585462	1.94559456158342e-18\\
2.96833296585466	-7.73612623558762e-17\\
2.96873400643407	-2.95801358692127e-18\\
2.96873400643411	7.62419964108149e-17\\
2.96913451161724	1.90991412586094e-18\\
2.96913451161728	-7.71845827808895e-17\\
2.96953448283174	-2.08137915426614e-18\\
2.96953448283178	7.51334714060076e-17\\
2.9699339214995	2.0057884947589e-18\\
2.96993392149954	-7.59930750170308e-17\\
2.97033282903677	-2.49175255514044e-18\\
2.97033282903681	7.36325852717546e-17\\
2.97073120685415	2.35928619748497e-18\\
2.97073120685419	-7.54318768119861e-17\\
2.97112905635662	-2.61307803071573e-18\\
2.97112905635666	7.24261411075751e-17\\
2.97152637894361	2.39001264851887e-18\\
2.97152637894365	-7.43135873536379e-17\\
2.97192317600897	-2.85370817962986e-18\\
2.97192317600901	7.11068020926234e-17\\
2.97231944894104	2.60014795697061e-18\\
2.97231944894107	-7.30204343953378e-17\\
2.97271519912264	-2.07243609107353e-18\\
2.97271519912268	7.08148383556652e-17\\
2.97311042793117	1.95796959713752e-18\\
2.97311042793121	-7.1710385859436e-17\\
2.97350513673855	-1.99849317534649e-18\\
2.97350513673859	7.06976417521174e-17\\
2.97389932691132	2.31131923453394e-18\\
2.97389932691136	-7.02890989019387e-17\\
2.97429299981062	-2.85849052434134e-18\\
2.97429299981066	6.87723354494181e-17\\
2.97468615679226	2.28950110510693e-18\\
2.9746861567923	-6.92483286746959e-17\\
2.97507879920671	-1.67448446145794e-18\\
2.97507879920675	6.88983316003222e-17\\
2.97547092839915	2.54724603448509e-18\\
2.97547092839918	-6.70634336570331e-17\\
2.97586254570949	-1.88746568480717e-18\\
2.97586254570952	6.85011616563855e-17\\
2.9762536524724	2.32676765965398e-18\\
2.97625365247243	-6.62345523611406e-17\\
2.97664425001734	-2.88928439059515e-18\\
2.97664425001737	6.64493686846788e-17\\
2.97703433966858	2.0849083203496e-18\\
2.97703433966861	-6.62980148985066e-17\\
2.97742392274523	-1.83439005439302e-18\\
2.97742392274527	6.64599727265414e-17\\
2.97781300056129	1.75088865309762e-18\\
2.97781300056132	-6.47281634078626e-17\\
2.97820157442561	-1.45347338231235e-18\\
2.97820157442564	6.58024824263894e-17\\
2.97858964564199	1.65089988783283e-18\\
2.97858964564203	-6.37954031162846e-17\\
2.97897721550919	-1.38900582050211e-18\\
2.97897721550922	6.48322980023576e-17\\
2.9793642853209	2.48823910548862e-18\\
2.97936428532094	-6.27890576645944e-17\\
2.97975085636586	-1.7073566190008e-18\\
2.97975085636589	6.34859490588642e-17\\
2.98013692992778	1.96400139816426e-18\\
2.98013692992782	-6.22887246497922e-17\\
2.98052250728547	-2.03019611309351e-18\\
2.9805225072855	6.21397589340485e-17\\
2.98090758971279	2.49907848023551e-18\\
2.98090758971282	-6.07345362376372e-17\\
2.98129217847869	-2.4436532432914e-18\\
2.98129217847872	6.15617395976571e-17\\
2.98167627484728	1.8316545681073e-18\\
2.98167627484731	-6.12398741087357e-17\\
2.9820598800778	-1.77248551171077e-18\\
2.98205988007783	6.03680360828625e-17\\
2.98244299542466	1.6058917499329e-18\\
2.98244299542469	-6.04548962510351e-17\\
2.98282562213748	-1.0459434147937e-18\\
2.98282562213751	6.00861300349722e-17\\
2.9832077614611	2.22667611719807e-18\\
2.98320776146113	-5.88277089430272e-17\\
2.98358941463562	-1.78721036058998e-18\\
2.98358941463565	5.9189650298651e-17\\
2.9839705828964	1.23272782814338e-18\\
2.98397058289643	-5.88203604772604e-17\\
2.9843512674741	-1.77606761735343e-18\\
2.98435126747413	5.82012221994142e-17\\
2.98473146959471	1.76875099804025e-18\\
2.98473146959474	-5.72883727655658e-17\\
2.98511119047954	-1.89168720106334e-18\\
2.98511119047958	5.79341185485857e-17\\
2.98549043134531	1.58761023121486e-18\\
2.98549043134534	-5.64793646906851e-17\\
2.98586919340409	-1.86180095873664e-18\\
2.98586919340413	5.69729609397835e-17\\
2.9862474778634	1.55242320872778e-18\\
2.98624747786343	-5.6368247037459e-17\\
2.98662528592616	-1.97558482187214e-18\\
2.98662528592619	5.50333144271898e-17\\
2.98700261879077	1.40650544674613e-18\\
2.9870026187908	-5.63683975843305e-17\\
2.98737947765114	-2.27471277011571e-18\\
2.98737947765117	5.37553236895178e-17\\
2.98775586369663	1.47821228648228e-18\\
2.98775586369666	-5.53172315579443e-17\\
2.98813177811219	-8.85700952554164e-19\\
2.98813177811222	5.41699283260205e-17\\
2.98850722207827	1.8558018313441e-18\\
2.9885072220783	-5.39640842237915e-17\\
2.98888219677092	-1.55010420788048e-18\\
2.98888219677095	5.33675418319432e-17\\
2.98925670336179	1.59690091705991e-18\\
2.98925670336182	-5.40840322368517e-17\\
2.98963074301812	-1.95336592343835e-18\\
2.98963074301815	5.19970255660832e-17\\
2.99000431690283	1.21541990100632e-18\\
2.99000431690286	-5.26688196801827e-17\\
2.99037742617446	-1.28460942334332e-18\\
2.99037742617449	5.2532192397251e-17\\
2.99075007198726	1.85057177512608e-18\\
2.99075007198732	-1.0399583143164e-16\\
2.99112225549129	-7.63935210097836e-19\\
2.99112225549132	5.87009838392083e-17\\
2.99149397783211	1.98171857290938e-18\\
2.99149397783216	-9.37039518157622e-17\\
2.99186524015123	-1.17805325760516e-18\\
2.99186524015127	6.30831199587667e-17\\
2.99223604358577	1.62480792297591e-18\\
2.99223604358582	-8.64193197836601e-17\\
2.99260638926874	-3.86155827285427e-19\\
2.99260638926878	6.61817188878508e-17\\
2.99297627832889	6.44666312320499e-19\\
2.99297627832894	-8.14361585737189e-17\\
2.99334571189087	-1.05538195917324e-18\\
2.9933457118909	6.78082901530975e-17\\
2.9937146910751	1.25407984129521e-18\\
2.99371469107514	-7.7349867529348e-17\\
2.99408321699794	-8.05419765383442e-19\\
2.99408321699798	6.8706787796523e-17\\
2.99445129077161	1.43803757169665e-18\\
2.99445129077165	-7.53398004428706e-17\\
2.99481891350425	-1.25263930625845e-18\\
2.99481891350429	6.80892654160035e-17\\
2.99518608629991	1.55657027872881e-18\\
2.99518608629995	-7.34043086516986e-17\\
2.99555281025862	-1.47572143886273e-18\\
2.99555281025866	6.85101698600513e-17\\
2.99591908647637	1.05483746753503e-18\\
2.99591908647641	-7.20973665925873e-17\\
2.99628491604515	-1.2966469692075e-18\\
2.99628491604518	6.77075831882659e-17\\
2.99665030005292	9.50154865884296e-19\\
2.99665030005296	-7.12128884197737e-17\\
2.99701523958373	-4.73862201224731e-19\\
2.99701523958376	6.75524903389298e-17\\
2.99737973571762	1.36982432927761e-18\\
2.99737973571766	-6.90002429401257e-17\\
2.99774378953075	-1.15310569712571e-18\\
2.99774378953079	6.67076820939329e-17\\
2.99810740209533	7.94476579469672e-19\\
2.99810740209537	-6.77917308244148e-17\\
2.9984705744797	-1.37906600467136e-18\\
2.99847057447973	6.63180643089888e-17\\
2.99883330774829	8.98174836191463e-19\\
2.99883330774833	-6.75216588359581e-17\\
2.99919560296172	-1.6031142842311e-18\\
2.99919560296176	6.43220566045879e-17\\
2.99955746117675	9.227791167167e-19\\
2.99955746117678	-6.73327998536321e-17\\
2.9999188834463	-1.45777513508793e-18\\
2.99991888344634	6.27049187138419e-17\\
3	1.13548361568151e-07\\
};
\addlegendentry{$\dot{\theta}$};

\addplot [color=mycolor2,solid]
  table[row sep=crcr]{%
0	-1\\
0.000200950914520766	-1\\
0.0012057054871246	-1\\
0.00622947835014376	-1\\
0.0313483426652396	-1\\
0.0913483426652396	-1\\
0.15134834266524	-1\\
0.21134834266524	-1\\
0.27134834266524	-1\\
0.33134834266524	-1\\
0.39134834266524	-1\\
0.45134834266524	-1\\
0.51134834266524	-1\\
0.571348342665239	-1\\
0.63134834266524	-1\\
0.69134834266524	-1\\
0.75134834266524	-1\\
0.81134834266524	-1\\
0.87134834266524	-1\\
0.93134834266524	-1\\
0.99134834266524	-1\\
1.05134834266524	-1\\
1.09999832888327	-1\\
1.09999832888328	1\\
1.15999832888328	1\\
1.21999832888328	1\\
1.25935832776576	1\\
1.25935832776578	-1\\
1.31935832776578	-1\\
1.36111750025393	-1\\
1.36111750025394	1\\
1.42111750025394	1\\
1.43652806024928	1\\
1.4365280602493	-1\\
1.4965280602493	-1\\
1.49657075331761	-1\\
1.49657075331763	1\\
1.54649475114703	1\\
1.54649475114705	-1\\
1.58926516707352	-1\\
1.58926516707353	1\\
1.62669428325042	1\\
1.62669428325043	-1\\
1.65996161271047	-1\\
1.65996161271048	1\\
1.68989953171323	1\\
1.68989953171324	-1\\
1.71711437070657	-1\\
1.71711437070658	1\\
1.74206081340125	1\\
1.74206081340127	-1\\
1.76508842219257	-1\\
1.76508842219258	1\\
1.78647168804854	1\\
1.78647168804856	-1\\
1.80643008644	-1\\
1.80643008644001	1\\
1.82514186246521	1\\
1.82514186246522	-1\\
1.84275375445231	-1\\
1.84275375445232	1\\
1.85938801059342	1\\
1.85938801059344	-1\\
1.87514755482624	-1\\
1.87514755482625	1\\
1.89011985835063	1\\
1.89011985835064	-1\\
1.9043798874291	-1\\
1.90437988742912	1\\
1.91799237995543	1\\
1.91799237995545	-1\\
1.93101362626831	-1\\
1.93101362626833	1\\
1.94349287839493	1\\
1.94349287839494	-1\\
1.95547347706441	-1\\
1.95547347706442	1\\
1.96699376172967	1\\
1.96699376172969	-1\\
1.9780878118909	-1\\
1.97808781189092	1\\
1.98878605591936	1\\
1.98878605591937	-1\\
1.99911577482883	-1\\
1.99911577482884	1\\
2.00910152202816	1\\
2.00910152202819	-1\\
2.01876547533242	-1\\
2.01876547533245	1\\
2.02812773394501	1\\
2.02812773394506	-1\\
2.03720657042426	-1\\
2.0372065704243	1\\
2.04601864558309	1\\
2.04601864558313	-1\\
2.05457919268063	-1\\
2.05457919268066	1\\
2.06290217602571	1\\
2.06290217602574	-1\\
2.07100042814403	-1\\
2.07100042814405	1\\
2.07888576889484	1\\
2.07888576889487	-1\\
2.08656910931517	-1\\
2.0865691093152	1\\
2.09406054248304	1\\
2.09406054248307	-1\\
2.1013694232998	-1\\
2.10136942329983	1\\
2.10850443877488	1\\
2.10850443877493	-1\\
2.11547367013832	-1\\
2.11547367013835	1\\
2.12228464789564	1\\
2.12228464789567	-1\\
2.12894440076641	-1\\
2.12894440076644	1\\
2.13545949930384	1\\
2.13545949930387	-1\\
2.14183609487481	-1\\
2.14183609487484	1\\
2.14807995458024	1\\
2.14807995458027	-1\\
2.15419649261277	-1\\
2.1541964926128	1\\
2.16019079847927	1\\
2.1601907984793	-1\\
2.16606766245712	-1\\
2.16606766245715	1\\
2.17183159860325	1\\
2.17183159860328	-1\\
2.17748686559312	-1\\
2.17748686559315	1\\
2.18303748563057	1\\
2.1830374856306	-1\\
2.18848726163916	-1\\
2.18848726163919	1\\
2.19383979291884	1\\
2.19383979291887	-1\\
2.19909848942967	-1\\
2.1990984894297	1\\
2.20426658484454	1\\
2.20426658484457	-1\\
2.20934714849606	-1\\
2.20934714849609	1\\
2.21434309632815	1\\
2.21434309632818	-1\\
2.21925720095025	-1\\
2.21925720095027	1\\
2.22409210088093	1\\
2.22409210088096	-1\\
2.22885030905812	-1\\
2.22885030905815	1\\
2.2335342206847	1\\
2.23353422068473	-1\\
2.23814612047082	-1\\
2.23814612047085	1\\
2.24268818932761	1\\
2.24268818932764	-1\\
2.2471625105616	-1\\
2.24716251056163	1\\
2.25157107561385	1\\
2.25157107561387	-1\\
2.2559157893833	-1\\
2.25591578938333	1\\
2.26019847517002	1\\
2.26019847517005	-1\\
2.26442087927042	-1\\
2.26442087927045	1\\
2.2685846752535	1\\
2.26858467525353	-1\\
2.27269146794428	-1\\
2.27269146794432	1\\
2.27674279713816	1\\
2.27674279713818	-1\\
2.28074014106774	-1\\
2.28074014106778	1\\
2.28468491964181	1\\
2.28468491964184	-1\\
2.28857849747368	-1\\
2.28857849747373	1\\
2.29242218671602	1\\
2.29242218671605	-1\\
2.29621724971568	-1\\
2.29621724971571	1\\
2.29996490150324	1\\
2.29996490150327	-1\\
2.30366631212859	-1\\
2.30366631212862	1\\
2.30732260885429	1\\
2.30732260885432	-1\\
2.31093487821676	-1\\
2.3109348782168	1\\
2.31450416796486	1\\
2.31450416796489	-1\\
2.31803148888442	-1\\
2.31803148888445	1\\
2.3215178165169	1\\
2.32151781651693	-1\\
2.32496409277908	-1\\
2.32496409277911	1\\
2.32837122749092	1\\
2.32837122749095	-1\\
2.33174009981753	-1\\
2.33174009981756	1\\
2.33507155963107	1\\
2.3350715596311	-1\\
2.33836642879792	-1\\
2.33836642879794	1\\
2.34162550239575	1\\
2.34162550239578	-1\\
2.34484954986541	-1\\
2.34484954986543	1\\
2.34803931610133	1\\
2.34803931610139	-1\\
2.35119552248486	-1\\
2.35119552248491	1\\
2.35431886786343	1\\
2.35431886786346	-1\\
2.35741002947952	-1\\
2.35741002947955	1\\
2.36046966385218	1\\
2.3604696638522	-1\\
2.36349840761393	-1\\
2.36349840761396	1\\
2.36649687830604	1\\
2.36649687830607	-1\\
2.36946567513427	-1\\
2.3694656751343	1\\
2.3724053796877	1\\
2.37240537968773	-1\\
2.37531655662261	-1\\
2.37531655662263	1\\
2.37819975431349	1\\
2.37819975431352	-1\\
2.38105550547312	-1\\
2.38105550547315	1\\
2.38388432774333	1\\
2.38388432774336	-1\\
2.38668672425822	-1\\
2.38668672425825	1\\
2.38946318418124	1\\
2.38946318418127	-1\\
2.39221418321774	-1\\
2.39221418321778	1\\
2.39494018410423	1\\
2.39494018410426	-1\\
2.39764163707554	-1\\
2.39764163707557	1\\
2.40031898031115	1\\
2.40031898031118	-1\\
2.40297264036176	-1\\
2.40297264036179	1\\
2.40560303255718	1\\
2.4056030325572	-1\\
2.40821056139648	-1\\
2.40821056139651	1\\
2.41079562092143	1\\
2.41079562092145	-1\\
2.41335859507376	-1\\
2.41335859507379	1\\
2.41589985803751	1\\
2.41589985803754	-1\\
2.41841977456685	-1\\
2.41841977456688	1\\
2.42091870030032	1\\
2.42091870030035	-1\\
2.42339698206204	-1\\
2.42339698206208	1\\
2.42585495815077	1\\
2.4258549581508	-1\\
2.42829295861694	-1\\
2.42829295861697	1\\
2.43071130552861	1\\
2.43071130552864	-1\\
2.43311031322691	-1\\
2.43311031322694	1\\
2.43549028857125	1\\
2.43549028857127	-1\\
2.43785153117486	-1\\
2.43785153117489	1\\
2.44019433363114	1\\
2.44019433363116	-1\\
2.4425189817313	-1\\
2.44251898173133	1\\
2.44482575467359	1\\
2.44482575467362	-1\\
2.44711492526451	-1\\
2.44711492526454	1\\
2.44938676011249	1\\
2.44938676011252	-1\\
2.45164151981423	-1\\
2.45164151981427	1\\
2.45387945913407	1\\
2.4538794591341	-1\\
2.45610082717671	-1\\
2.45610082717674	1\\
2.45830586755374	1\\
2.45830586755377	-1\\
2.46049481854394	-1\\
2.46049481854397	1\\
2.46266791324773	1\\
2.46266791324776	-1\\
2.46482537973615	-1\\
2.46482537973618	1\\
2.46696744119457	1\\
2.4669674411946	-1\\
2.46909431606123	-1\\
2.46909431606126	1\\
2.47120621816096	1\\
2.47120621816099	-1\\
2.47330335683412	-1\\
2.47330335683415	1\\
2.47538593706119	1\\
2.47538593706122	-1\\
2.47745415958298	-1\\
2.477454159583	1\\
2.47950822101673	1\\
2.47950822101676	-1\\
2.48154831396834	-1\\
2.48154831396837	1\\
2.48357462714071	1\\
2.48357462714074	-1\\
2.48558734543851	-1\\
2.48558734543854	1\\
2.48758665006946	1\\
2.48758665006948	-1\\
2.48957271864214	-1\\
2.48957271864217	1\\
2.49154572526076	1\\
2.49154572526079	-1\\
2.49350584061673	-1\\
2.49350584061675	1\\
2.49545323207727	1\\
2.49545323207729	-1\\
2.49738806377119	-1\\
2.49738806377122	1\\
2.4993104966719	1\\
2.49931049667193	-1\\
2.50122068867777	-1\\
2.5012206886778	1\\
2.50311879468997	1\\
2.50311879468999	-1\\
2.50500496668782	-1\\
2.50500496668785	1\\
2.50687935380189	1\\
2.50687935380191	-1\\
2.50874210238465	-1\\
2.50874210238468	1\\
2.51059335607914	1\\
2.51059335607917	-1\\
2.51243325588541	-1\\
2.51243325588544	1\\
2.51426194022494	1\\
2.51426194022496	-1\\
2.51607954500316	-1\\
2.51607954500319	1\\
2.51788620367008	1\\
2.51788620367011	-1\\
2.51968204727905	-1\\
2.51968204727908	1\\
2.52146720454381	1\\
2.52146720454384	-1\\
2.52324180189389	-1\\
2.52324180189392	1\\
2.5250059635283	1\\
2.52500596352833	-1\\
2.52675981146771	-1\\
2.52675981146774	1\\
2.52850346560511	1\\
2.52850346560514	-1\\
2.53023704375495	-1\\
2.53023704375498	1\\
2.53196066170097	1\\
2.531960661701	-1\\
2.53367443324254	-1\\
2.53367443324257	1\\
2.53537847023981	1\\
2.53537847023984	-1\\
2.53707288265747	-1\\
2.5370728826575	1\\
2.53875777860737	1\\
2.5387577786074	-1\\
2.5404332643899	-1\\
2.54043326438993	1\\
2.54209944453423	1\\
2.54209944453426	-1\\
2.54375642183746	-1\\
2.54375642183749	1\\
2.54540429740265	1\\
2.54540429740267	-1\\
2.54704317067586	-1\\
2.54704317067589	1\\
2.54867313948218	1\\
2.54867313948221	-1\\
2.55029430006074	-1\\
2.55029430006077	1\\
2.55190674709879	1\\
2.55190674709882	-1\\
2.55351057376496	-1\\
2.55351057376499	1\\
2.55510587174148	1\\
2.55510587174151	-1\\
2.55669273125569	-1\\
2.55669273125572	1\\
2.55827124111063	1\\
2.55827124111065	-1\\
2.55984148871487	-1\\
2.5598414887149	1\\
2.56140356011157	1\\
2.5614035601116	-1\\
2.56295754000673	-1\\
2.56295754000676	1\\
2.56450351179676	1\\
2.56450351179679	-1\\
2.56604155759543	-1\\
2.56604155759546	1\\
2.56757175825995	1\\
2.56757175825998	-1\\
2.56909419341645	-1\\
2.56909419341648	1\\
2.57060894148497	1\\
2.570608941485	-1\\
2.57211607970358	-1\\
2.57211607970361	1\\
2.57361568415202	1\\
2.57361568415205	-1\\
2.57510782977484	-1\\
2.57510782977487	1\\
2.57659259040383	1\\
2.57659259040386	-1\\
2.57807003877987	-1\\
2.5780700387799	1\\
2.57954024657436	1\\
2.57954024657439	-1\\
2.58100328441009	-1\\
2.58100328441012	1\\
2.58245922188164	1\\
2.58245922188166	-1\\
2.58390812757517	-1\\
2.5839081275752	1\\
2.58535006908789	1\\
2.58535006908792	-1\\
2.5867851130469	-1\\
2.58678511304693	1\\
2.58821332512775	1\\
2.58821332512778	-1\\
2.58963477007241	-1\\
2.58963477007244	1\\
2.59104951170689	1\\
2.59104951170692	-1\\
2.59245761295842	-1\\
2.59245761295845	1\\
2.59385913587227	1\\
2.59385913587229	-1\\
2.59525414162812	-1\\
2.59525414162814	1\\
2.5966426905562	1\\
2.59664269055623	-1\\
2.59802484215287	-1\\
2.59802484215289	1\\
2.59940065509584	1\\
2.59940065509586	-1\\
2.60077018725917	-1\\
2.6007701872592	1\\
2.60213349572794	1\\
2.60213349572797	-1\\
2.60349063681251	-1\\
2.60349063681254	1\\
2.60484166606245	1\\
2.60484166606247	-1\\
2.60618663828018	-1\\
2.6061866382802	1\\
2.60752560753433	1\\
2.60752560753435	-1\\
2.60885862717276	-1\\
2.60885862717279	1\\
2.61018574983537	1\\
2.6101857498354	-1\\
2.61150702746651	-1\\
2.61150702746654	1\\
2.61282251132715	1\\
2.61282251132717	-1\\
2.61413225200684	-1\\
2.61413225200686	1\\
2.61543629943545	1\\
2.61543629943548	-1\\
2.61673470289456	-1\\
2.61673470289459	1\\
2.61802751102862	1\\
2.61802751102864	-1\\
2.6193147718559	-1\\
2.61931477185592	1\\
2.62059653277912	1\\
2.62059653277915	-1\\
2.621872840596	-1\\
2.62187284059603	1\\
2.6231437415095	1\\
2.62314374150953	-1\\
2.6244092811379	-1\\
2.62440928113793	1\\
2.62566950452458	1\\
2.62566950452461	-1\\
2.62692445614764	-1\\
2.62692445614767	1\\
2.62817417992924	1\\
2.62817417992926	-1\\
2.62941871924487	-1\\
2.6294187192449	1\\
2.63065811693249	1\\
2.63065811693252	-1\\
2.63189241530135	-1\\
2.63189241530138	1\\
2.63312165614058	1\\
2.6331216561406	-1\\
2.63434588072763	-1\\
2.63434588072766	1\\
2.63556512983669	1\\
2.63556512983672	-1\\
2.63677944374682	-1\\
2.63677944374684	1\\
2.63798886224997	1\\
2.63798886225	-1\\
2.63919342465874	-1\\
2.63919342465876	1\\
2.64039316981402	1\\
2.64039316981405	-1\\
2.64158813609263	-1\\
2.64158813609267	1\\
2.64277836141466	1\\
2.64277836141469	-1\\
2.64396388325056	-1\\
2.64396388325059	1\\
2.64514473862831	1\\
2.64514473862834	-1\\
2.64632096414036	-1\\
2.64632096414039	1\\
2.64749259595047	1\\
2.6474925959505	-1\\
2.64865966980027	-1\\
2.6486596698003	1\\
2.64982222101586	1\\
2.64982222101589	-1\\
2.65098028451423	-1\\
2.65098028451427	1\\
2.6521338948096	1\\
2.65213389480963	-1\\
2.6532830860195	-1\\
2.65328308601954	1\\
2.65442789187099	1\\
2.65442789187102	-1\\
2.65556834570635	-1\\
2.65556834570638	1\\
2.65670448048908	1\\
2.65670448048913	-1\\
2.6578363288097	-1\\
2.65783632880973	1\\
2.65896392289107	1\\
2.65896392289109	-1\\
2.66008729459409	-1\\
2.66008729459412	1\\
2.66120647542306	1\\
2.6612064754231	-1\\
2.66232149653109	-1\\
2.66232149653112	1\\
2.66343238872516	1\\
2.66343238872519	-1\\
2.66453918247129	-1\\
2.66453918247132	1\\
2.66564190789948	1\\
2.66564190789951	-1\\
2.66674059480871	-1\\
2.66674059480875	1\\
2.66783527267182	1\\
2.66783527267185	-1\\
2.66892597064009	-1\\
2.66892597064014	1\\
2.67001271754809	1\\
2.67001271754812	-1\\
2.67109554191803	-1\\
2.67109554191806	1\\
2.67217447196445	1\\
2.6721744719645	-1\\
2.67324953559868	-1\\
2.67324953559871	1\\
2.67432076043287	1\\
2.6743207604329	-1\\
2.67538817378455	-1\\
2.67538817378458	1\\
2.67645180268066	1\\
2.67645180268069	-1\\
2.67751167386166	-1\\
2.67751167386169	1\\
2.67856781378556	1\\
2.67856781378561	-1\\
2.67962024863202	-1\\
2.67962024863205	1\\
2.68066900430601	1\\
2.68066900430606	-1\\
2.6817141064419	-1\\
2.68171410644193	1\\
2.68275558040695	1\\
2.68275558040698	-1\\
2.68379345130516	-1\\
2.68379345130519	1\\
2.68482774398096	1\\
2.68482774398099	-1\\
2.68585848302262	-1\\
2.68585848302265	1\\
2.68688569276584	1\\
2.68688569276587	-1\\
2.68790939729721	-1\\
2.68790939729724	1\\
2.68892962045758	1\\
2.68892962045762	-1\\
2.68994638584543	-1\\
2.68994638584546	1\\
2.69095971682003	1\\
2.69095971682006	-1\\
2.6919696365048	-1\\
2.69196963650483	1\\
2.69297616779042	1\\
2.69297616779045	-1\\
2.69397933333787	-1\\
2.6939793333379	1\\
2.69497915558157	1\\
2.6949791555816	-1\\
2.69597565673242	-1\\
2.69597565673245	1\\
2.69696885878067	1\\
2.69696885878072	-1\\
2.69795878349902	-1\\
2.69795878349905	1\\
2.6989454524452	1\\
2.69894545244523	-1\\
2.69992888696491	-1\\
2.69992888696494	1\\
2.70090910819461	1\\
2.70090910819465	-1\\
2.70188613706432	-1\\
2.70188613706435	1\\
2.70285999430025	1\\
2.70285999430027	-1\\
2.70383070042738	-1\\
2.70383070042741	1\\
2.70479827577206	1\\
2.70479827577209	-1\\
2.70576274046464	-1\\
2.70576274046467	1\\
2.70672411444193	1\\
2.70672411444196	-1\\
2.70768241744969	-1\\
2.70768241744971	1\\
2.70863766904503	1\\
2.70863766904506	-1\\
2.70958988859885	-1\\
2.70958988859888	1\\
2.7105390952982	1\\
2.71053909529823	-1\\
2.7114853081486	-1\\
2.71148530814863	1\\
2.71242854597628	1\\
2.71242854597632	-1\\
2.71336882743041	-1\\
2.71336882743044	1\\
2.71430617098532	1\\
2.71430617098536	-1\\
2.71524059494281	-1\\
2.71524059494286	1\\
2.71617211743423	1\\
2.71617211743426	-1\\
2.71710075642234	-1\\
2.71710075642237	1\\
2.71802652970369	1\\
2.71802652970375	-1\\
2.71894945491077	-1\\
2.71894945491082	1\\
2.71986954951366	1\\
2.71986954951369	-1\\
2.72078683082211	-1\\
2.72078683082214	1\\
2.72170131598764	1\\
2.72170131598766	-1\\
2.72261302200538	-1\\
2.7226130220054	1\\
2.72352196571604	1\\
2.72352196571607	-1\\
2.72442816380774	-1\\
2.72442816380777	1\\
2.72533163281786	1\\
2.72533163281791	-1\\
2.72623238913484	-1\\
2.72623238913489	1\\
2.72713044899984	1\\
2.72713044899987	-1\\
2.72802582850852	-1\\
2.72802582850855	1\\
2.72891854361289	1\\
2.72891854361292	-1\\
2.72980861012304	-1\\
2.72980861012307	1\\
2.73069604370879	1\\
2.73069604370883	-1\\
2.73158085990131	-1\\
2.73158085990135	1\\
2.73246307409468	1\\
2.73246307409472	-1\\
2.73334270154762	-1\\
2.73334270154765	1\\
2.73421975738495	1\\
2.73421975738499	-1\\
2.7350942565993	-1\\
2.73509425659933	1\\
2.73596621405251	1\\
2.73596621405255	-1\\
2.73683564447724	-1\\
2.73683564447728	1\\
2.73770256247841	1\\
2.73770256247845	-1\\
2.73856698253471	-1\\
2.73856698253474	1\\
2.73942891900001	1\\
2.73942891900005	-1\\
2.74028838610483	-1\\
2.74028838610486	1\\
2.74114539795771	1\\
2.74114539795775	-1\\
2.74199996854667	-1\\
2.74199996854672	1\\
2.74285211174055	1\\
2.74285211174061	-1\\
2.74370184129032	-1\\
2.74370184129036	1\\
2.74454917083035	1\\
2.74454917083038	-1\\
2.74539411387982	-1\\
2.74539411387987	1\\
2.74623668384409	1\\
2.74623668384412	-1\\
2.74707689401577	-1\\
2.74707689401582	1\\
2.74791475757625	1\\
2.74791475757629	-1\\
2.74875028759667	-1\\
2.74875028759672	1\\
2.7495834970393	1\\
2.74958349703934	-1\\
2.75041439875865	-1\\
2.75041439875868	1\\
2.75124300550273	1\\
2.75124300550277	-1\\
2.75206932991427	-1\\
2.75206932991432	1\\
2.75289338453182	1\\
2.75289338453185	-1\\
2.75371518179078	-1\\
2.75371518179081	1\\
2.7545347340247	1\\
2.75453473402474	-1\\
2.75535205346634	-1\\
2.75535205346639	1\\
2.75616715224882	1\\
2.75616715224886	-1\\
2.7569800424065	-1\\
2.75698004240654	1\\
2.7577907358762	1\\
2.75779073587625	-1\\
2.75859924449831	-1\\
2.75859924449835	1\\
2.75940558001767	1\\
2.7594055800177	-1\\
2.76020975408464	-1\\
2.76020975408469	1\\
2.76101177825626	1\\
2.7610117782563	-1\\
2.76181166399704	-1\\
2.76181166399709	1\\
2.76260942268007	1\\
2.76260942268011	-1\\
2.76340506558789	-1\\
2.76340506558792	1\\
2.76419860391352	1\\
2.76419860391356	-1\\
2.76499004876145	-1\\
2.76499004876149	1\\
2.76577941114852	1\\
2.76577941114855	-1\\
2.76656670200472	-1\\
2.76656670200476	1\\
2.76735193217434	1\\
2.76735193217439	-1\\
2.76813511241673	-1\\
2.76813511241677	1\\
2.76891625340713	1\\
2.76891625340716	-1\\
2.76969536573759	-1\\
2.76969536573762	1\\
2.77047245991788	1\\
2.77047245991791	-1\\
2.77124754637634	-1\\
2.77124754637639	1\\
2.77202063546074	1\\
2.77202063546079	-1\\
2.77279173743899	-1\\
2.77279173743903	1\\
2.77356086249995	1\\
2.7735608625	-1\\
2.77432802075444	-1\\
2.77432802075448	1\\
2.77509322223584	1\\
2.7750932222359	-1\\
2.77585647690103	-1\\
2.77585647690106	1\\
2.77661779463095	1\\
2.776617794631	-1\\
2.77737718523164	-1\\
2.77737718523169	1\\
2.77813465843478	1\\
2.77813465843481	-1\\
2.77889022389849	-1\\
2.77889022389852	1\\
2.77964389120815	1\\
2.77964389120818	-1\\
2.78039566987709	-1\\
2.78039566987714	1\\
2.78114556934732	1\\
2.78114556934737	-1\\
2.78189359899009	-1\\
2.78189359899012	1\\
2.78263976810677	1\\
2.78263976810679	-1\\
2.78338408592962	-1\\
2.78338408592965	1\\
2.78412656162234	1\\
2.78412656162239	-1\\
2.78486720428075	-1\\
2.78486720428078	1\\
2.78560602293338	1\\
2.78560602293341	-1\\
2.78634302654226	-1\\
2.78634302654229	1\\
2.78707822400351	1\\
2.78707822400356	-1\\
2.7878116241481	-1\\
2.78781162414814	1\\
2.78854323574225	1\\
2.78854323574229	-1\\
2.78927306748829	-1\\
2.78927306748832	1\\
2.79000112802516	1\\
2.79000112802519	-1\\
2.79072742592905	-1\\
2.79072742592909	1\\
2.79145196971405	1\\
2.7914519697141	-1\\
2.7921747678327	-1\\
2.79217476783273	1\\
2.79289582867648	1\\
2.79289582867651	-1\\
2.79361516057663	-1\\
2.79361516057667	1\\
2.79433277180466	1\\
2.79433277180471	-1\\
2.79504867057283	-1\\
2.79504867057285	1\\
2.79576286503459	1\\
2.79576286503464	-1\\
2.79647536328545	-1\\
2.79647536328548	1\\
2.79718617336326	1\\
2.79718617336329	-1\\
2.79789530324895	-1\\
2.79789530324898	1\\
2.79860276086703	1\\
2.79860276086708	-1\\
2.79930855408611	-1\\
2.79930855408614	1\\
2.80001269071921	1\\
2.80001269071925	-1\\
2.80071517852462	-1\\
2.80071517852466	1\\
2.80141602520618	1\\
2.80141602520623	-1\\
2.80211523841395	-1\\
2.80211523841398	1\\
2.80281282574449	1\\
2.80281282574454	-1\\
2.80350879474166	-1\\
2.80350879474169	1\\
2.80420315289676	1\\
2.8042031528968	-1\\
2.8048959076493	-1\\
2.80489590764933	1\\
2.80558706638728	1\\
2.80558706638732	-1\\
2.80627663644787	-1\\
2.8062766364479	1\\
2.80696462511768	1\\
2.80696462511773	-1\\
2.80765103963339	-1\\
2.80765103963342	1\\
2.80833588718201	1\\
2.80833588718206	-1\\
2.80901917490158	-1\\
2.80901917490161	1\\
2.80970090988131	1\\
2.80970090988134	-1\\
2.81038109916226	-1\\
2.81038109916229	1\\
2.81105974973766	1\\
2.81105974973769	-1\\
2.8117368685534	-1\\
2.81173686855343	1\\
2.8124124625085	1\\
2.81241246250853	-1\\
2.81308653845549	-1\\
2.81308653845552	1\\
2.81375910320083	1\\
2.81375910320087	-1\\
2.81443016350534	-1\\
2.81443016350537	1\\
2.81509972608444	1\\
2.81509972608447	-1\\
2.81576779760874	-1\\
2.81576779760878	1\\
2.81643438470438	1\\
2.81643438470441	-1\\
2.81709949395344	-1\\
2.81709949395346	1\\
2.81776313189428	1\\
2.81776313189434	-1\\
2.81842530502213	-1\\
2.81842530502219	1\\
2.81908601978918	1\\
2.81908601978921	-1\\
2.81974528260494	-1\\
2.81974528260497	1\\
2.82040309983689	1\\
2.82040309983695	-1\\
2.82105947781082	-1\\
2.82105947781087	1\\
2.82171442281096	1\\
2.821714422811	-1\\
2.82236794108047	-1\\
2.8223679410805	1\\
2.82302003882176	1\\
2.82302003882182	-1\\
2.82367072219707	-1\\
2.82367072219711	1\\
2.82431999732843	1\\
2.82431999732847	-1\\
2.82496787029827	-1\\
2.8249678702983	1\\
2.82561434714966	1\\
2.82561434714969	-1\\
2.8262594338867	-1\\
2.82625943388675	1\\
2.82690313647493	1\\
2.82690313647495	-1\\
2.82754546084146	-1\\
2.82754546084149	1\\
2.82818641287554	1\\
2.82818641287557	-1\\
2.82882599842869	-1\\
2.82882599842871	1\\
2.82946422331506	1\\
2.82946422331508	-1\\
2.83010109331176	-1\\
2.8301010933118	1\\
2.83073661415923	1\\
2.83073661415926	-1\\
2.8313707915614	-1\\
2.83137079156143	1\\
2.83200363118622	1\\
2.83200363118625	-1\\
2.83263513866586	-1\\
2.83263513866589	1\\
2.83326531959698	1\\
2.83326531959701	-1\\
2.83389417954101	-1\\
2.83389417954104	1\\
2.83452172402446	1\\
2.83452172402451	-1\\
2.83514795853934	-1\\
2.83514795853937	1\\
2.83577288854319	1\\
2.83577288854322	-1\\
2.8363965194596	-1\\
2.83639651945963	1\\
2.83701885667841	1\\
2.83701885667844	-1\\
2.83763990555595	-1\\
2.83763990555598	1\\
2.83825967141537	1\\
2.8382596714154	-1\\
2.83887815954691	-1\\
2.83887815954694	1\\
2.8394953752081	1\\
2.83949537520813	-1\\
2.84011132362409	-1\\
2.84011132362412	1\\
2.84072600998794	1\\
2.84072600998797	-1\\
2.84133943946092	-1\\
2.84133943946095	1\\
2.84195161717268	1\\
2.84195161717271	-1\\
2.84256254822148	-1\\
2.84256254822151	1\\
2.84317223767448	1\\
2.84317223767453	-1\\
2.84378069056816	-1\\
2.8437806905682	1\\
2.8443879119083	1\\
2.84438791190833	-1\\
2.84499390667032	-1\\
2.84499390667035	1\\
2.84559867979957	1\\
2.8455986797996	-1\\
2.84620223621155	-1\\
2.84620223621159	1\\
2.84680458079221	1\\
2.84680458079226	-1\\
2.84740571839811	-1\\
2.84740571839816	1\\
2.84800565385664	1\\
2.84800565385667	-1\\
2.84860439196622	-1\\
2.84860439196625	1\\
2.84920193749666	1\\
2.8492019374967	-1\\
2.84979829518935	-1\\
2.8497982951894	1\\
2.85039346975748	1\\
2.85039346975752	-1\\
2.85098746588618	-1\\
2.85098746588622	1\\
2.85158028823283	1\\
2.85158028823288	-1\\
2.85217194142727	-1\\
2.85217194142731	1\\
2.85276243007196	1\\
2.85276243007201	-1\\
2.85335175874231	-1\\
2.85335175874235	1\\
2.85393993198675	1\\
2.85393993198679	-1\\
2.854526954327	-1\\
2.85452695432704	1\\
2.85511283025835	1\\
2.8551128302584	-1\\
2.85569756424978	-1\\
2.85569756424981	1\\
2.85628116074411	1\\
2.85628116074416	-1\\
2.85686362415846	-1\\
2.85686362415851	1\\
2.85744495888414	1\\
2.85744495888418	-1\\
2.85802516928699	-1\\
2.85802516928702	1\\
2.85860425970755	1\\
2.85860425970759	-1\\
2.85918223446131	-1\\
2.85918223446135	1\\
2.85975909783889	1\\
2.85975909783892	-1\\
2.86033485410614	-1\\
2.86033485410617	1\\
2.86090950750445	1\\
2.8609095075045	-1\\
2.86148306225093	-1\\
2.86148306225098	1\\
2.86205552253848	1\\
2.86205552253853	-1\\
2.86262689253604	-1\\
2.86262689253608	1\\
2.86319717638876	1\\
2.8631971763888	-1\\
2.86376637821822	-1\\
2.86376637821828	1\\
2.86433450212265	1\\
2.8643345021227	-1\\
2.86490155217696	-1\\
2.86490155217701	1\\
2.865467532433	1\\
2.86546753243305	-1\\
2.86603244691974	-1\\
2.86603244691979	1\\
2.86659629964343	1\\
2.86659629964348	-1\\
2.86715909458777	-1\\
2.86715909458781	1\\
2.86772083571407	1\\
2.86772083571411	-1\\
2.86828152696145	-1\\
2.86828152696148	1\\
2.86884117224696	1\\
2.868841172247	-1\\
2.86939977546583	-1\\
2.86939977546587	1\\
2.86995734049153	1\\
2.86995734049156	-1\\
2.87051387117595	-1\\
2.87051387117598	1\\
2.87106937134961	1\\
2.87106937134964	-1\\
2.87162384482178	-1\\
2.87162384482183	1\\
2.87217729538075	1\\
2.8721772953808	-1\\
2.87272972679377	-1\\
2.87272972679382	1\\
2.87328114280734	1\\
2.87328114280738	-1\\
2.87383154714733	-1\\
2.87383154714737	1\\
2.87438094351914	1\\
2.87438094351919	-1\\
2.87492933560786	-1\\
2.87492933560791	1\\
2.87547672707841	1\\
2.87547672707845	-1\\
2.87602312157567	-1\\
2.87602312157571	1\\
2.87656852272465	1\\
2.8765685227247	-1\\
2.87711293413063	-1\\
2.87711293413068	1\\
2.87765635937929	1\\
2.87765635937934	-1\\
2.87819880203686	-1\\
2.87819880203691	1\\
2.87874026565027	1\\
2.87874026565032	-1\\
2.87928075374727	-1\\
2.87928075374731	1\\
2.87982026983657	1\\
2.87982026983662	-1\\
2.88035881740801	-1\\
2.88035881740805	1\\
2.88089639993264	1\\
2.88089639993268	-1\\
2.8814330208629	-1\\
2.88143302086294	1\\
2.88196868363273	1\\
2.88196868363277	-1\\
2.88250339165771	-1\\
2.88250339165775	1\\
2.88303714833519	1\\
2.88303714833524	-1\\
2.88356995704443	-1\\
2.88356995704448	1\\
2.8841018211467	1\\
2.88410182114674	-1\\
2.88463274398542	-1\\
2.88463274398547	1\\
2.88516272888632	1\\
2.88516272888636	-1\\
2.8856917791575	-1\\
2.88569177915754	1\\
2.8862198980896	1\\
2.88621989808964	-1\\
2.88674708895593	-1\\
2.88674708895597	1\\
2.88727335501255	1\\
2.8872733550126	-1\\
2.88779869949844	-1\\
2.88779869949848	1\\
2.88832312563555	1\\
2.8883231256356	-1\\
2.88884663662902	-1\\
2.88884663662906	1\\
2.88936923566719	1\\
2.88936923566723	-1\\
2.8898909259218	-1\\
2.88989092592184	1\\
2.89041171054805	1\\
2.89041171054809	-1\\
2.89093159268475	-1\\
2.89093159268479	1\\
2.89145057545442	1\\
2.89145057545446	-1\\
2.89196866196339	-1\\
2.89196866196343	1\\
2.89248585530194	1\\
2.89248585530199	-1\\
2.89300215854439	-1\\
2.89300215854443	1\\
2.89351757474921	1\\
2.89351757474925	-1\\
2.89403210695913	-1\\
2.89403210695917	1\\
2.89454575820125	1\\
2.89454575820129	-1\\
2.89505853148717	-1\\
2.89505853148721	1\\
2.89557042981305	1\\
2.89557042981309	-1\\
2.89608145615975	-1\\
2.89608145615979	1\\
2.89659161349291	1\\
2.89659161349295	-1\\
2.89710090476308	-1\\
2.89710090476311	1\\
2.89760933290579	1\\
2.89760933290583	-1\\
2.89811690084171	-1\\
2.89811690084175	1\\
2.89862361147667	1\\
2.8986236114767	-1\\
2.89912946770181	-1\\
2.89912946770185	1\\
2.89963447239368	1\\
2.89963447239372	-1\\
2.90013862841433	-1\\
2.90013862841436	1\\
2.90064193861138	1\\
2.90064193861142	-1\\
2.90114440581818	-1\\
2.90114440581822	1\\
2.90164603285383	1\\
2.90164603285387	-1\\
2.90214682252334	-1\\
2.90214682252338	1\\
2.90264677761769	1\\
2.90264677761772	-1\\
2.9031459009139	-1\\
2.90314590091394	1\\
2.9036441951752	1\\
2.90364419517523	-1\\
2.90414166315103	-1\\
2.90414166315107	1\\
2.90463830757721	1\\
2.90463830757725	-1\\
2.90513413117597	-1\\
2.90513413117601	1\\
2.90562913665609	1\\
2.90562913665612	-1\\
2.90612332671293	-1\\
2.90612332671296	1\\
2.90661670402858	1\\
2.90661670402861	-1\\
2.9071092712719	-1\\
2.90710927127194	1\\
2.90760103109865	1\\
2.90760103109868	-1\\
2.90809198615152	-1\\
2.90809198615156	1\\
2.90858213906028	1\\
2.90858213906031	-1\\
2.9090714924418	-1\\
2.90907149244183	1\\
2.90956004890018	1\\
2.90956004890021	-1\\
2.91004781102682	-1\\
2.91004781102685	1\\
2.9105347814005	1\\
2.91053478140053	-1\\
2.91102096258745	-1\\
2.91102096258748	1\\
2.91150635714146	1\\
2.91150635714149	-1\\
2.91199096760393	-1\\
2.91199096760397	1\\
2.91247479650399	1\\
2.91247479650402	-1\\
2.91295784635852	-1\\
2.91295784635855	1\\
2.91344011967228	1\\
2.91344011967232	-1\\
2.91392161893797	-1\\
2.913921618938	1\\
2.9144023466363	1\\
2.91440234663633	-1\\
2.91488230523608	-1\\
2.91488230523611	1\\
2.91536149719428	1\\
2.91536149719431	-1\\
2.91583992495612	-1\\
2.91583992495615	1\\
2.91631759095515	1\\
2.91631759095518	-1\\
2.91679449761329	-1\\
2.91679449761333	1\\
2.91727064734097	1\\
2.917270647341	-1\\
2.9177460425371	-1\\
2.91774604253713	1\\
2.91822068558927	1\\
2.9182206855893	-1\\
2.9186945788737	-1\\
2.91869457887373	1\\
2.91916772475541	1\\
2.91916772475544	-1\\
2.91964012558821	-1\\
2.91964012558824	1\\
2.92011178371484	1\\
2.92011178371486	-1\\
2.92058270146698	-1\\
2.92058270146701	1\\
2.92105288116537	1\\
2.92105288116539	-1\\
2.92152232511983	-1\\
2.92152232511986	1\\
2.9219910356294	1\\
2.92199103562942	-1\\
2.9224590149823	-1\\
2.92245901498233	1\\
2.92292626545611	1\\
2.92292626545613	-1\\
2.92339278931775	-1\\
2.9233927893178	1\\
2.92385858882371	1\\
2.92385858882376	-1\\
2.92432366621984	-1\\
2.92432366621989	1\\
2.92478802374159	1\\
2.92478802374164	-1\\
2.92525166361406	-1\\
2.9252516636141	1\\
2.92571458805202	1\\
2.92571458805206	-1\\
2.92617679926002	-1\\
2.92617679926007	1\\
2.92663829943244	1\\
2.92663829943249	-1\\
2.92709909075354	-1\\
2.92709909075359	1\\
2.92755917539754	1\\
2.92755917539759	-1\\
2.92801855552865	-1\\
2.9280185555287	1\\
2.92847723330118	1\\
2.92847723330123	-1\\
2.92893521085956	-1\\
2.92893521085961	1\\
2.92939249033842	1\\
2.92939249033846	-1\\
2.92984907386262	-1\\
2.92984907386267	1\\
2.93030496354738	1\\
2.93030496354743	-1\\
2.93076016149825	-1\\
2.9307601614983	1\\
2.93121466981123	1\\
2.93121466981127	-1\\
2.93166849057279	-1\\
2.93166849057284	1\\
2.93212162585997	1\\
2.93212162586002	-1\\
2.9325740777404	-1\\
2.93257407774044	1\\
2.93302584827236	1\\
2.93302584827241	-1\\
2.93347693950487	-1\\
2.93347693950491	1\\
2.93392735347768	1\\
2.93392735347773	-1\\
2.93437709222142	-1\\
2.93437709222146	1\\
2.93482615775755	1\\
2.93482615775759	-1\\
2.9352745520985	-1\\
2.93527455209854	1\\
2.93572227724767	1\\
2.93572227724772	-1\\
2.93616933519953	-1\\
2.93616933519957	1\\
2.93661572793961	1\\
2.93661572793965	-1\\
2.93706145744462	-1\\
2.93706145744466	1\\
2.93750652568246	1\\
2.9375065256825	-1\\
2.93795093461228	-1\\
2.93795093461232	1\\
2.93839468618454	1\\
2.93839468618459	-1\\
2.93883778234108	-1\\
2.93883778234112	1\\
2.9392802250151	1\\
2.93928022501514	-1\\
2.9397220161313	-1\\
2.93972201613134	1\\
2.94016315760588	1\\
2.94016315760592	-1\\
2.94060365134659	-1\\
2.94060365134663	1\\
2.94104349925279	1\\
2.94104349925283	-1\\
2.94148270321552	-1\\
2.94148270321555	1\\
2.94192126511749	1\\
2.94192126511753	-1\\
2.94235918683319	-1\\
2.94235918683323	1\\
2.94279647022891	1\\
2.94279647022895	-1\\
2.94323311716279	-1\\
2.94323311716283	1\\
2.94366912948487	1\\
2.9436691294849	-1\\
2.94410450903712	-1\\
2.94410450903715	1\\
2.94453925765352	1\\
2.94453925765355	-1\\
2.94497337716008	-1\\
2.94497337716012	1\\
2.9454068693749	1\\
2.94540686937494	-1\\
2.9458397361082	-1\\
2.94583973610824	1\\
2.9462719791624	1\\
2.94627197916243	-1\\
2.9467036003321	-1\\
2.94670360033214	1\\
2.9471346014042	1\\
2.94713460140424	-1\\
2.9475649841579	-1\\
2.94756498415793	1\\
2.94799475036474	1\\
2.94799475036477	-1\\
2.94842390178867	-1\\
2.94842390178871	1\\
2.94885244018609	1\\
2.94885244018612	-1\\
2.94928036730586	-1\\
2.9492803673059	1\\
2.94970768488939	1\\
2.94970768488942	-1\\
2.95013439467064	-1\\
2.95013439467067	1\\
2.95056049837619	1\\
2.95056049837623	-1\\
2.95098599772528	-1\\
2.95098599772532	1\\
2.95141089442984	1\\
2.95141089442987	-1\\
2.95183519019453	-1\\
2.95183519019456	1\\
2.95225888671679	1\\
2.95225888671683	-1\\
2.95268198568689	-1\\
2.95268198568693	1\\
2.95310448878794	1\\
2.95310448878798	-1\\
2.95352639769597	-1\\
2.953526397696	1\\
2.95394771407992	1\\
2.95394771407995	-1\\
2.95436843960174	-1\\
2.95436843960177	1\\
2.95478857591637	1\\
2.9547885759164	-1\\
2.95520812467182	-1\\
2.95520812467186	1\\
2.95562708750921	1\\
2.95562708750924	-1\\
2.95604546606276	-1\\
2.95604546606279	1\\
2.95646326195989	1\\
2.95646326195992	-1\\
2.95688047682122	-1\\
2.95688047682125	1\\
2.95729711226062	1\\
2.95729711226065	-1\\
2.95771316988524	-1\\
2.95771316988527	1\\
2.95812865129556	1\\
2.95812865129559	-1\\
2.95854355808542	-1\\
2.95854355808545	1\\
2.95895789184204	1\\
2.95895789184207	-1\\
2.9593716541461	-1\\
2.95937165414613	1\\
2.95978484657173	1\\
2.95978484657176	-1\\
2.96019747068656	-1\\
2.96019747068659	1\\
2.96060952805177	1\\
2.9606095280518	-1\\
2.96102102022213	-1\\
2.96102102022216	1\\
2.96143194874599	1\\
2.96143194874602	-1\\
2.96184231516537	-1\\
2.9618423151654	1\\
2.96225212101596	1\\
2.96225212101599	-1\\
2.96266136782717	-1\\
2.9626613678272	1\\
2.96307005712216	1\\
2.96307005712222	-1\\
2.96347819041797	-1\\
2.96347819041801	1\\
2.96388576922529	1\\
2.96388576922534	-1\\
2.96429279504878	-1\\
2.96429279504882	1\\
2.96469926938688	1\\
2.96469926938693	-1\\
2.96510519373204	-1\\
2.96510519373208	1\\
2.9655105695706	1\\
2.96551056957065	-1\\
2.96591539838291	-1\\
2.96591539838295	1\\
2.96631968164333	1\\
2.96631968164337	-1\\
2.96672342082025	-1\\
2.96672342082029	1\\
2.96712661737617	1\\
2.96712661737621	-1\\
2.96752927276769	-1\\
2.96752927276773	1\\
2.96793138844553	1\\
2.96793138844557	-1\\
2.96833296585462	-1\\
2.96833296585466	1\\
2.96873400643407	1\\
2.96873400643411	-1\\
2.96913451161724	-1\\
2.96913451161728	1\\
2.96953448283174	1\\
2.96953448283178	-1\\
2.9699339214995	-1\\
2.96993392149954	1\\
2.97033282903677	1\\
2.97033282903681	-1\\
2.97073120685415	-1\\
2.97073120685419	1\\
2.97112905635662	1\\
2.97112905635666	-1\\
2.97152637894361	-1\\
2.97152637894365	1\\
2.97192317600897	1\\
2.97192317600901	-1\\
2.97231944894104	-1\\
2.97231944894107	1\\
2.97271519912264	1\\
2.97271519912268	-1\\
2.97311042793117	-1\\
2.97311042793121	1\\
2.97350513673855	1\\
2.97350513673859	-1\\
2.97389932691132	-1\\
2.97389932691136	1\\
2.97429299981062	1\\
2.97429299981066	-1\\
2.97468615679226	-1\\
2.9746861567923	1\\
2.97507879920671	1\\
2.97507879920675	-1\\
2.97547092839915	-1\\
2.97547092839918	1\\
2.97586254570949	1\\
2.97586254570952	-1\\
2.9762536524724	-1\\
2.97625365247243	1\\
2.97664425001734	1\\
2.97664425001737	-1\\
2.97703433966858	-1\\
2.97703433966861	1\\
2.97742392274523	1\\
2.97742392274527	-1\\
2.97781300056129	-1\\
2.97781300056132	1\\
2.97820157442561	1\\
2.97820157442564	-1\\
2.97858964564199	-1\\
2.97858964564203	1\\
2.97897721550919	1\\
2.97897721550922	-1\\
2.9793642853209	-1\\
2.97936428532094	1\\
2.97975085636586	1\\
2.97975085636589	-1\\
2.98013692992778	-1\\
2.98013692992782	1\\
2.98052250728547	1\\
2.9805225072855	-1\\
2.98090758971279	-1\\
2.98090758971282	1\\
2.98129217847869	1\\
2.98129217847872	-1\\
2.98167627484728	-1\\
2.98167627484731	1\\
2.9820598800778	1\\
2.98205988007783	-1\\
2.98244299542466	-1\\
2.98244299542469	1\\
2.98282562213748	1\\
2.98282562213751	-1\\
2.9832077614611	-1\\
2.98320776146113	1\\
2.98358941463562	1\\
2.98358941463565	-1\\
2.9839705828964	-1\\
2.98397058289643	1\\
2.9843512674741	1\\
2.98435126747413	-1\\
2.98473146959471	-1\\
2.98473146959474	1\\
2.98511119047954	1\\
2.98511119047958	-1\\
2.98549043134531	-1\\
2.98549043134534	1\\
2.98586919340409	1\\
2.98586919340413	-1\\
2.9862474778634	-1\\
2.98624747786343	1\\
2.98662528592616	1\\
2.98662528592619	-1\\
2.98700261879077	-1\\
2.9870026187908	1\\
2.98737947765114	1\\
2.98737947765117	-1\\
2.98775586369663	-1\\
2.98775586369666	1\\
2.98813177811219	1\\
2.98813177811222	-1\\
2.98850722207827	-1\\
2.9885072220783	1\\
2.98888219677092	1\\
2.98888219677095	-1\\
2.98925670336179	-1\\
2.98925670336182	1\\
2.98963074301812	1\\
2.98963074301815	-1\\
2.99000431690283	-1\\
2.99000431690286	1\\
2.99037742617446	1\\
2.99037742617449	-1\\
2.99075007198726	-1\\
2.99075007198732	1\\
2.99112225549129	1\\
2.99112225549132	-1\\
2.99149397783211	-1\\
2.99149397783216	1\\
2.99186524015123	1\\
2.99186524015127	-1\\
2.99223604358577	-1\\
2.99223604358582	1\\
2.99260638926874	1\\
2.99260638926878	-1\\
2.99297627832889	-1\\
2.99297627832894	1\\
2.99334571189087	1\\
2.9933457118909	-1\\
2.9937146910751	-1\\
2.99371469107514	1\\
2.99408321699794	1\\
2.99408321699798	-1\\
2.99445129077161	-1\\
2.99445129077165	1\\
2.99481891350425	1\\
2.99481891350429	-1\\
2.99518608629991	-1\\
2.99518608629995	1\\
2.99555281025862	1\\
2.99555281025866	-1\\
2.99591908647637	-1\\
2.99591908647641	1\\
2.99628491604515	1\\
2.99628491604518	-1\\
2.99665030005292	-1\\
2.99665030005296	1\\
2.99701523958373	1\\
2.99701523958376	-1\\
2.99737973571762	-1\\
2.99737973571766	1\\
2.99774378953075	1\\
2.99774378953079	-1\\
2.99810740209533	-1\\
2.99810740209537	1\\
2.9984705744797	1\\
2.99847057447973	-1\\
2.99883330774829	-1\\
2.99883330774833	1\\
2.99919560296172	1\\
2.99919560296176	-1\\
2.99955746117675	-1\\
2.99955746117678	1\\
2.9999188834463	1\\
2.99991888344634	-1\\
3	-1\\
};
\addlegendentry{$\dot{\psi}$};

\end{axis}
\end{tikzpicture}%
}
  \caption{}
  \label{fig:04}
\end{figure}
