In order to reach a conclusion about the stability of the angular displacement,
it suffices to find a Lyapunov function $V(x)$ such that $V(0) = 0$, $V(x) > 0$
for all $x \neq 0$ and $\dot{V}(x) \leq 0$ for all x. Considering
$x = \theta - \theta^G$ and $V(x) = x^2$:
$$V(0) = 0,\ V(x) > 0,\ \text{for all } x \neq 0\text{, and}$$
\begin{align*}
  \dot{V}(x) &= 2 x \dot{x} = 2 (\theta - \theta^G) \dot{\theta}  \\
             &= 2 (\theta - \theta^G) \dfrac{R}{L} u_{\Psi}
\end{align*}

\begin{itemize}
  \item When $\theta - \theta^G \leq 0$, $\dot{V}(x) = 2 (\theta - \theta^G) \dfrac{R}{L} \leq 0$
  \item When $\theta - \theta^G > 0$, $\dot{V}(x) = -2 (\theta - \theta^G) \dfrac{R}{L} < 0$
\end{itemize}

Hence $\dot{V}(x) \leq 0$ for all $x$, meaning that the system is stable for all
$\theta \in (-180^{\circ}, 180^{\circ}]$.

Practically, as one can see in \ref{fig:03}, the system is stable but it exhibits
Zeno behaviour, which is defined as an infinite number of switches in finite
time.

A solution $\chi=(\tau,q,y)$ exhibits Zeno behaviour if

\begin{equation}
  \tau_\infty= \sum_{i=0}^\infty (\tau'-\tau)<\infty
  \label{eq:03.zeno}
\end{equation}

whereas $\tau$ is the the time trajectory. As we can see from the automaton, if
we start at $q_1$ we switch to $q_2$ from where we are immediately switched back
and vice versa. So the system enters zeno behaviour as in the moment we enter 0 we
are pushed back.

\begin{figure}[H]\centering
  \scalebox{1.4}{\begin{tikzpicture}[->,>=stealth',shorten >=1pt,auto,node distance=2.8cm, semithick]
  \tikzstyle{every state}=[fill=none,draw=black,text=black]

    \node[state]  (A)                  {\tiny{$q_1$}: \linebreak $\dot{\theta}=\frac{R}{L}$};
    \node[state]  (B)   [right of=A]   {\tiny{$q_2$}: $\dot{\theta}=-\frac{R}{L}$};

    \path (A) edge      [bend left,above] node  {\tiny{$\theta-\theta^G > 0$}} (B)
          (B) edge			[bend left,below] node {\tiny{$\theta-\theta^G \leq 0$}} (A);

\end{tikzpicture}
}
  \caption{}
  \label{fig:03_automaton}
\end{figure}

A formal way to look at this is the following: Without loss of generality,
let us assume that at $t=0$ the angular subsystem is at $\theta = \theta_0^0$,
that $\theta - \theta^G \leq 0$ and that $R/L = K = ct > 0$. Then:

\begin{enumerate}
    \item The system is in state $q_1$ for the first time and $\dot{\theta} = K$.
      Which means that
      $$\theta(t) = K t + \theta_0^0$$
      The transition to state $q_2$ will happen when $\theta =\theta^G$, at time
      $t = t_{1\rightarrow2}^1$, and at that time:
      \begin{align}
        \theta^G &= K t_{1\rightarrow2}^1 + \theta_0^0 \Leftrightarrow \nonumber \\
        t_{1\rightarrow2}^1 &= \dfrac{\theta^G - \theta_0^0}{K} \label{eq:theta_1}
      \end{align}

    \item The time is $t_{1\rightarrow2}^1$ and the system is at state $q_2$. Here,
      $\dot{\theta} = -K$ which means that
      $$\theta(t) = -K t + \theta_0^1$$
      and at this time instant
      \begin{align*}
        \theta^G &= -K t_{1\rightarrow2}^1 + \theta_0^1 \Leftrightarrow \\
        \theta^G &= -K \dfrac{\theta^G - \theta_0^0}{K} + \theta_0^1 \Leftrightarrow \\
        \theta^G &= -\theta^G + \theta_0^0 + \theta_0^1 \Leftrightarrow \\
        \theta_0^1 &= 2\theta^G - \theta_0^0
      \end{align*}


      Hence
      $$\theta = -K t + 2\theta_G - \theta_0^0$$
      and at the time of the second switch, at $t_{2\rightarrow1}^1$ when $\theta = \theta^G$:

      \begin{align}
        \theta^G &= -K t_{2\rightarrow1}^1 + 2\theta^G - \theta_0^0 \Leftrightarrow \nonumber \\
        K t_{2\rightarrow1}^1 &= \theta^G - \theta_0^0 \Leftrightarrow \nonumber \\
        t_{2\rightarrow1}^1 &= \dfrac{\theta^G - \theta_0^0}{K} \label{eq:theta_2}
      \end{align}

      Comparing equations \ref{eq:theta_1} and \ref{eq:theta_2} we observe that
      two switches must happen at the exact same time which is
      $t = t_{1\rightarrow2}^1 = t_{2\rightarrow1}^1$. This fact implies that
      the system might exhibit Zeno behaviour, however, we do not know whether
      the behaviour of $\theta$ will be the same in the next round yet.
      Continuing,

    \item The time now is $t_{2\rightarrow1}^1$ and the system is at state $q_1$.
      Here, $\dot{\theta} = K$ which means that
      $$\theta(t) = K t + \theta_0^2$$
      and at this time instant
      \begin{align*}
        \theta^G &= K t_{2\rightarrow1}^1 + \theta_0^2 \Leftrightarrow \\
        \theta^G &= K \dfrac{\theta^G - \theta_0^0}{K} + \theta_0^2 \Leftrightarrow \\
        \theta^G &= \theta^G - \theta_0^0 + \theta_0^2 \Leftrightarrow \\
        \theta_0^2 &= \theta_0^0
      \end{align*}

      So, when going back to state $q_1$ again, this process starts from
      the beginning.
\end{enumerate}

Hence, equation \ref{eq:03.zeno} converges, and an infinite amount of switches
should happen at the same time. Ergo, the angular subsystem exhibits a chattering
Zeno behaviour under this control law.


\begin{figure}[H]\centering
  \scalebox{0.9}{% This file was created by matlab2tikz.
%
%The latest updates can be retrieved from
%  http://www.mathworks.com/matlabcentral/fileexchange/22022-matlab2tikz-matlab2tikz
%where you can also make suggestions and rate matlab2tikz.
%
\definecolor{mycolor1}{rgb}{0.00000,0.44700,0.74100}%
\definecolor{mycolor2}{rgb}{0.85000,0.32500,0.09800}%
%
\begin{tikzpicture}

\begin{axis}[%
width=4.521in,
height=3.566in,
at={(0.758in,0.481in)},
scale only axis,
xmin=0,
xmax=1.2,
xmajorgrids,
ymin=-1,
ymax=1,
ymajorgrids,
axis background/.style={fill=white},
legend style={legend cell align=left,align=left,draw=white!15!black}
]
\addplot [color=mycolor1,solid]
  table[row sep=crcr]{%
0	1\\
0.226	0.774\\
0.452	0.548\\
0.678	0.322\\
0.904	0.0960000000000001\\
0.9718	0.0282000000000001\\
0.99214	0.00786000000000006\\
0.998242	0.00175800000000006\\
0.99879118	0.00120882000000006\\
0.99934036	0.000659640000000058\\
0.99988954	0.000110460000000058\\
0.99993402358	6.59764200000585e-05\\
0.99997850716	2.14928400000585e-05\\
1.00002299074	-4.60162915423131e-06\\
1.00046782654	-0.000133762814095279\\
1.00091266234	-0.000155747141131035\\
1.00135749814	-9.05666994741811e-05\\
1.00202863945173	-1.41733086049118e-06\\
1.00203027032512	-6.28462223699681e-08\\
1.0020319011985	-6.83883057755179e-07\\
1.00203353207189	1.8525212761481e-07\\
1.00204984080577	-8.69391680930732e-07\\
1.00206614953964	1.98070174025784e-06\\
1.00208245827352	5.22290930016067e-07\\
1.00208377928096	-1.27140286124223e-07\\
1.0020851002884	-2.61695360464658e-07\\
1.00208642129585	-1.59115647175157e-08\\
1.00208677796786	4.17392724895189e-08\\
1.00208713463987	4.95823528040483e-08\\
1.00208749131188	1.36053080072028e-08\\
1.00208749997901	6.15492874665764e-09\\
1.00208750864614	1.22431959035905e-08\\
1.00208751731327	5.93698563575904e-09\\
1.00208760398456	-4.34229572725893e-08\\
1.00208769065586	-1.42953723755005e-08\\
1.00208777732716	-5.19816163100558e-08\\
1.00208864404015	-8.89240042277349e-08\\
1.00208951075313	9.68379343376852e-09\\
1.00209037746611	-3.65085342231508e-08\\
1.00209039852724	-1.89114096460397e-08\\
1.00209041958836	1.59056062504417e-09\\
1.00209044064949	-1.27614954642335e-08\\
1.00209065126074	-1.20055861074565e-07\\
1.002090861872	-6.29773282388246e-08\\
1.00209107248325	7.34852007275532e-08\\
1.00209317859581	6.92632636773275e-07\\
1.00209528470836	9.97647648159037e-07\\
1.00209739082091	1.02473932344667e-06\\
1.0020979594713	5.52264824026711e-07\\
1.00209852812169	7.83800937225921e-09\\
1.00209909677207	2.9614050322556e-07\\
1.00210478327596	-1.77351476502521e-06\\
1.00211046977985	-9.72235697344345e-07\\
1.00211615628374	1.63730431397689e-06\\
1.00217302132261	1.00422649960121e-07\\
1.00222988636149	9.50406516910674e-06\\
1.00228675140036	6.4934907466356e-06\\
1.00229135746851	2.63002693168342e-06\\
1.00229596353666	4.98940337034953e-06\\
1.00230056960481	2.23104140872929e-06\\
1.0023466302863	1.45944517099776e-05\\
1.00239269096779	7.1221274444842e-06\\
1.00243875164928	1.78249507964896e-05\\
1.00257693369374	-1.00551384907022e-05\\
1.00271511573821	2.0563791931315e-05\\
1.00285329778268	-8.28111406576903e-06\\
1.00310570971597	1.75955394937514e-05\\
1.00335812164926	-2.81897025882595e-05\\
1.00361053358255	2.81359741125371e-06\\
1.00364705006308	-5.84168800151743e-07\\
1.0036835665436	1.02700549520614e-07\\
1.00372008302413	-3.01764527620874e-06\\
1.00372304085905	-6.18917489279232e-07\\
1.00372599869397	-2.07982111159738e-06\\
1.0037289565289	-3.5683554936814e-07\\
1.00375853487812	-6.8882494706984e-06\\
1.00378811322735	-2.18252577814132e-06\\
1.00381769157657	-8.1457960576167e-06\\
1.00390642662425	1.53377404444514e-05\\
1.00399516167193	-4.72258820782488e-06\\
1.00408389671961	1.74326504717778e-05\\
1.00427626949192	-2.51118872856372e-06\\
1.00429358304143	1.52346573796634e-06\\
1.00431089659094	1.93793904231493e-06\\
1.00432821014045	-6.53866509613695e-07\\
1.00432961253796	-2.95463007919235e-07\\
1.00433101493547	-2.89098675471068e-07\\
1.00433241733298	-3.62366736241759e-09\\
1.00433279598031	-1.77959122443413e-09\\
1.00433317462763	-3.30501123166516e-08\\
1.00433355327496	-2.864501939187e-08\\
1.00433358394539	-2.71607343902513e-09\\
1.00433361461583	-2.36553548874071e-08\\
1.00433364528626	-1.85819858872775e-09\\
1.0043339519906	-3.59383315829441e-08\\
1.00433425869493	2.88229294754866e-08\\
1.00433456539927	-1.41559819243661e-08\\
1.00433464820944	-1.97655531448151e-09\\
1.00433473101961	-2.8191497389101e-09\\
1.00433481382978	7.80612282092813e-09\\
1.0043348205374	2.28034124439636e-09\\
1.00433482724503	7.00760989035005e-09\\
1.00433483395265	2.15107722017623e-09\\
1.00433490102889	-3.21098269208749e-08\\
1.00433496810513	-1.56968519455325e-08\\
1.00433503518137	-4.10328566428753e-08\\
1.00433570594375	-1.08929865077363e-07\\
1.00433637670613	-9.76811642018032e-08\\
1.00433704746851	-1.57670405813203e-08\\
1.00433722857435	4.88988698500821e-09\\
1.0043374096802	1.68442705503631e-08\\
1.00433759078604	2.04714730811439e-08\\
1.00433760545561	8.14247936314675e-09\\
1.00433762012519	1.81657908435146e-08\\
1.00433763479476	7.74568781055814e-09\\
1.00433778149049	2.34207091309426e-08\\
1.00433792818623	-7.67678225405205e-09\\
1.00433807488196	1.24408351522279e-08\\
1.00433811448981	6.61681820998416e-09\\
1.00433815409765	-3.79829850122052e-09\\
1.0043381937055	-7.41547623357569e-09\\
1.00433819691374	-4.7539124975281e-09\\
1.00433820012197	-1.6332169381307e-09\\
1.00433820333021	-3.82798302248742e-09\\
1.00433823541257	-2.15851194564605e-08\\
1.00433826749492	-1.08860491486927e-08\\
1.00433829957728	1.20219579163144e-08\\
1.00433862040085	9.44231341365954e-08\\
1.00433894122441	1.17954694301039e-07\\
1.00433926204798	9.12114276580896e-08\\
1.00433934867035	1.71012316941774e-08\\
1.00433943529271	7.29836702951691e-08\\
1.00433952191507	1.36151932001767e-08\\
1.0043403881387	-4.80305127385287e-07\\
1.00434125436234	-1.74020066528894e-07\\
1.00434212058597	-5.575500536627e-07\\
1.0043507828223	-8.15018717958928e-07\\
1.00435944505863	3.36086724903965e-07\\
1.00436810729496	-6.25024841004708e-08\\
1.00436817044267	-1.28911523359039e-08\\
1.00436823359037	-3.5549760973768e-08\\
1.00436829673807	-3.21306889307896e-09\\
1.0043689282151	-2.02068652453705e-08\\
1.00436955969213	9.41357165770218e-08\\
1.00437019116916	5.74256332098581e-08\\
1.0043702423188	1.38059009466493e-08\\
1.00437029346844	4.89877097685213e-08\\
1.00437034461807	1.23841828437454e-08\\
1.00437085611447	1.21993516433147e-07\\
1.00437136761086	2.0940563779191e-08\\
1.00437187910725	1.27647804065562e-07\\
1.00437201721128	1.57264276733138e-08\\
1.00437215531531	1.08531888471205e-07\\
1.00437229341933	1.22630853780584e-08\\
1.00437367445959	-9.6149612967048e-08\\
1.00437505549985	1.05008735154708e-07\\
1.00437643654012	-2.26804147698635e-08\\
1.00437680942099	1.30627068626778e-08\\
1.00437718230186	1.15774079094218e-08\\
1.00437755518273	-2.37445497022384e-08\\
1.00437756424373	-1.62323859311163e-08\\
1.00437757330474	-7.42038568115942e-09\\
1.00437758236574	-1.3607974583081e-08\\
1.00437767297579	-6.15643702586397e-08\\
1.00437776358585	-3.42796925927223e-08\\
1.0043778541959	2.73904311602947e-08\\
1.00437876029641	2.81211750254732e-07\\
1.00437966639693	3.85391425532685e-07\\
1.00438057249744	3.59234741968836e-07\\
1.00438081714458	1.5274354247541e-07\\
1.00438106179172	2.9921754066975e-07\\
1.00438130643886	1.39683075964104e-07\\
1.00438375291025	-1.01736399137324e-06\\
1.00438619938164	-4.71733228604901e-07\\
1.00438864585304	-1.33988348617424e-06\\
1.00441311056695	-4.0634624322294e-06\\
1.00443757528087	-4.19739165733441e-06\\
1.00446203999479	-1.98833194710778e-06\\
1.00446402163661	-3.12793318171931e-07\\
1.00446600327844	-1.33048568238529e-06\\
1.00446798492027	-1.39400027562285e-07\\
1.00448780133854	-5.575968225959e-06\\
1.00450761775681	-2.34691110468332e-06\\
1.00452743417509	-7.0259663687099e-06\\
1.00472559835781	-5.28872869941694e-06\\
1.00473094879075	-8.25919550445772e-07\\
1.00473629922368	-3.68831188459147e-06\\
1.00474164965662	-4.20151306079977e-07\\
1.00479515398595	-1.71547212181055e-05\\
1.00484865831529	-8.17534696627742e-06\\
1.00490216264462	-2.20707968134967e-05\\
1.00505974035102	7.31622846668174e-06\\
1.00507392234459	3.58133797200025e-06\\
1.00508810433817	3.7457585796928e-06\\
1.00510228633174	7.87921328010528e-07\\
1.00510611547001	9.22253475916498e-08\\
1.00510994460827	-1.68042413365045e-07\\
1.00511377374654	-3.4783541526953e-08\\
1.0051138667946	1.18537918533292e-08\\
1.00511395984266	1.76740004844365e-08\\
1.00511405289072	-5.39584413180555e-09\\
1.00511407801369	-3.65821805733196e-09\\
1.00511410313667	1.31590708218618e-09\\
1.00511412825965	2.12463637082958e-09\\
1.00511413029461	4.17621436865601e-10\\
1.00511413232957	1.85654527330196e-09\\
1.00511413436453	3.74918791073958e-10\\
1.00511415471414	-1.1936111401766e-08\\
1.00511417506375	-4.22521792512496e-09\\
1.00511419541336	-1.37162303928061e-08\\
1.00511439890947	-1.40351616747789e-08\\
1.00511460240558	2.17357236200449e-08\\
1.00511480590168	1.51392647860874e-08\\
1.00511482238487	1.14710967943197e-09\\
1.00511483886805	1.26377972833206e-08\\
1.00511485535124	7.37317065646692e-10\\
1.00511502018308	-7.86167034447155e-08\\
1.00511518501493	-5.07117934111131e-08\\
1.00511534984678	3.73384352338671e-08\\
1.00511539435138	3.50377234314943e-09\\
1.00511543885597	3.48867791009458e-08\\
1.00511548336057	3.09312727894093e-09\\
1.00511592840656	1.142950961755e-07\\
1.00511637345254	2.7169156680552e-08\\
1.00511681849853	1.29786916837807e-07\\
1.00511693866095	3.27980531716254e-08\\
1.00511705882336	1.14179426596899e-07\\
1.00511717898578	3.00107118009376e-08\\
1.00511838060994	-1.20922542437847e-08\\
1.0051195822341	2.26753699715244e-07\\
1.00512078385826	1.40207922425027e-07\\
1.00512088118982	5.7003883509919e-08\\
1.00512097852138	1.23298220903419e-07\\
1.00512107585293	5.40968107071827e-08\\
1.00512204916851	1.62445295083817e-07\\
1.00512302248408	-4.31043852165226e-08\\
1.00512399579965	9.34518193456422e-08\\
1.00512407463821	2.86633814585801e-08\\
1.00512415347677	8.33656623670853e-08\\
1.00512423231533	2.69684194125957e-08\\
1.00512502070094	2.08102446038989e-07\\
1.00512580908656	5.30987579329194e-08\\
1.00512659747217	2.25173254538591e-07\\
1.00512681033628	5.30772655918311e-08\\
1.0051270232004	1.94161703187245e-07\\
1.00512723606451	4.7294850381139e-08\\
1.00512936470567	-2.73381806879775e-07\\
1.00513149334682	-1.5703868383662e-07\\
1.00513362198797	3.18348920617082e-07\\
1.00515490839951	-3.34635621428406e-06\\
1.00517619481104	-2.49312363572431e-06\\
1.00519748122258	2.05997660945247e-06\\
1.00526134045718	-1.04276621097979e-05\\
1.00532519969179	-8.40556713314992e-06\\
1.00538905892639	5.36868467001799e-06\\
1.0055806366302	-7.03283340485272e-06\\
1.00558580922821	-2.77079673039591e-06\\
1.00559098182621	-5.63850289902347e-06\\
1.00559615442421	-2.43016022383236e-06\\
1.00564788040424	1.9726470951003e-05\\
1.00569960638427	-2.38437840375481e-06\\
1.0057513323643	4.25985069391586e-06\\
1.00576356101924	4.58566057153685e-07\\
1.00577578967418	2.5622294475033e-07\\
1.00578801832913	-2.64619229304764e-06\\
1.00591030487855	-3.535080705623e-06\\
1.00591360661538	-7.78573843393524e-07\\
1.00591690835221	-2.64222583683839e-06\\
1.00592021008905	-5.6663105663572e-07\\
1.00595322745739	1.38656532312195e-05\\
1.00598624482574	8.91198515894522e-06\\
1.00601926219408	-7.50734232988512e-06\\
1.00634943587751	-8.13618109803949e-05\\
1.00667960956094	-8.1385655776879e-05\\
1.00700978324438	-2.10704143524718e-05\\
1.00702594883536	-7.62538866620899e-06\\
1.00704211442634	-1.3098621172352e-05\\
1.00705828001733	-4.74678362583172e-06\\
1.00721993592717	-7.87375534513075e-06\\
1.00723448495906	-2.54463944689152e-06\\
1.00724903399095	-1.53058880397116e-06\\
1.00726358302283	2.42157900798318e-06\\
1.00740907334169	1.05557858296126e-05\\
1.00755456366055	-7.37256258415374e-06\\
1.00770005397941	2.42323020556907e-06\\
1.00771056278943	6.46481375253367e-07\\
1.00772107159944	3.33644103690654e-07\\
1.00773158040945	-1.19411570735712e-06\\
1.00773243162306	-5.0006583563854e-07\\
1.00773328283668	-1.00884346204502e-06\\
1.00773413405029	-4.59008835031086e-07\\
1.0077426461864	3.48675331772739e-06\\
1.00775115832251	1.72292533795382e-06\\
1.00775967045862	4.66868335192102e-06\\
1.00784479181972	1.46731158298124e-05\\
1.00792991318083	8.52714502320474e-06\\
1.00793757410333	2.31655123738148e-06\\
1.00794523502583	5.02211461930677e-06\\
1.00795289594833	1.06443978741571e-06\\
1.00802950517332	3.22965803537878e-06\\
1.00803640000357	6.99709321482194e-07\\
1.00804329483382	2.5398644048893e-07\\
1.00805018966407	1.43157701032546e-06\\
1.00811913796656	-5.54690625703768e-06\\
1.00818808626906	5.11443011550334e-06\\
1.00825703457155	-2.71967826338257e-06\\
1.00846387947904	1.57019206376552e-05\\
1.00867072438653	-6.87569627336499e-06\\
1.00887756929401	1.23023110349562e-05\\
1.00888974336525	2.82133674432371e-06\\
1.00890191743649	7.33293006328454e-06\\
1.00891409150772	1.06295799071992e-06\\
1.0090358322201	5.59357106514153e-06\\
1.00904678888422	1.57251182234727e-06\\
1.00905774554833	9.18371713182856e-07\\
1.00906870221244	-2.05172552579356e-06\\
1.00917826885358	-4.63908412586204e-06\\
1.00918122715289	-2.17392365990914e-06\\
1.00918418545221	-3.87671392978601e-06\\
1.00918714375152	-1.99743803761533e-06\\
1.00921672674462	1.11873045522356e-05\\
1.00924630973773	6.1520257057511e-06\\
1.00927589273084	-9.34113389493763e-06\\
1.00957172266192	-7.50984157403639e-05\\
1.00986755259299	-6.93227698967123e-05\\
1.01016338252407	-6.16558103938581e-06\\
1.01017823519288	8.89678425132024e-07\\
1.01019308786169	3.84883308344769e-06\\
1.01020794053049	3.47772781027581e-06\\
1.01021195075107	2.12788395318225e-07\\
1.01021596097165	1.8460623586787e-06\\
1.01021997119223	-3.20310763385126e-07\\
1.01026007339801	3.3411427274554e-06\\
1.01030017560379	-3.44797560852144e-06\\
1.01034027780958	9.60350668027479e-07\\
1.01035110540514	-2.71110815517076e-07\\
1.0103619330007	-2.04816393372796e-07\\
1.01037276059626	1.01957372848493e-06\\
1.0103736376315	3.00624519492386e-07\\
1.01037451466674	8.16646359742745e-07\\
1.01037539170198	2.54722130273606e-07\\
1.01038416205439	-3.73154867453426e-06\\
1.01039293240679	-2.06893167832181e-06\\
1.0104017027592	2.74113266209205e-06\\
1.01048940628324	6.94576309096427e-06\\
1.01049729960041	5.53282254690949e-07\\
1.01050519291757	4.12796168269626e-06\\
1.01051308623473	-2.98547198196751e-07\\
1.01059201940638	1.1761777112963e-05\\
1.01061569935787	-5.08856603168874e-06\\
1.01063937930936	8.44484081921057e-07\\
1.01066305926085	-9.8688945361101e-06\\
1.01089985877578	-2.84815324827291e-05\\
1.0111366582907	1.86435088601621e-05\\
1.01137345780563	-7.30761791387269e-06\\
1.01140857502872	-3.82000140837106e-06\\
1.0114436922518	2.83276061914024e-06\\
1.01147880947489	5.30563880900811e-06\\
1.01148829112512	-1.47088202703709e-06\\
1.01149777277535	5.09781284795692e-07\\
1.01150725442559	-3.68353340168532e-06\\
1.01160207092792	-1.35949792413512e-05\\
1.01169688743025	1.01698002473835e-06\\
1.01179170393258	-9.99515189705379e-06\\
1.01205554623419	9.52518606796701e-06\\
1.0123193885358	-1.92231877503465e-05\\
1.01258323083742	1.86907728167707e-06\\
1.01259744434032	-1.26673407564521e-06\\
1.01261165784323	-1.7475168125617e-06\\
1.01262587134613	2.04203910119256e-08\\
1.01262588067161	1.27391987859842e-08\\
1.01262588999709	3.71033871942166e-09\\
1.01262589932257	9.72480891901381e-09\\
1.01262599257736	1.27682080224522e-08\\
1.01262608583215	-4.70000111102854e-09\\
1.01262617908695	1.41589652148662e-09\\
1.01262620426574	-1.68618349164891e-09\\
1.01262622944453	-2.13564635964757e-09\\
1.01262625462333	-1.81295486457344e-10\\
1.01262625480688	-4.35521283421882e-11\\
1.01262625499043	-7.70906083417901e-11\\
1.01262625517399	5.10160536112519e-12\\
1.01262625700952	2.18399725004087e-10\\
1.01262625884506	1.78548311394613e-10\\
1.01262626068059	-8.317399481633e-11\\
1.01262626072519	-4.65268217100722e-11\\
1.0126262607698	-3.16765901069576e-12\\
1.0126262608144	-3.40143398556569e-11\\
1.01262626085454	-4.47292647372963e-12\\
1.01262626089469	-3.35647064236834e-11\\
1.01262626093483	-4.40161860041755e-12\\
1.01262626093808	-1.2888781663838e-12\\
1.01262626094133	-3.53412810632409e-12\\
1.01262626094459	-1.15170850730445e-12\\
1.01262626094546	-3.10423657052504e-13\\
1.01262626094634	-9.10687244728212e-13\\
1.01262626094722	-2.71496839934029e-13\\
1.012626260956	4.29745482299274e-12\\
1.01262626096478	2.00952882979486e-12\\
1.01262626097356	5.40673820901063e-12\\
1.01262626106135	1.35735361833328e-11\\
1.01262626114914	1.08382332186214e-11\\
1.01262626123694	-1.56057103151803e-12\\
1.01262626124405	-5.35773875211025e-13\\
1.01262626125116	-6.17207201648092e-13\\
1.01262626125827	2.79323769125854e-13\\
1.01262626125844	1.36899584937119e-13\\
1.01262626125862	2.6048480358835e-13\\
1.01262626125879	1.33958175233774e-13\\
1.01262626125883	8.92519991598183e-14\\
1.01262626125888	4.28998008437911e-14\\
1.01262626125893	7.50321495910648e-14\\
1.01262626125897	4.41114063564152e-14\\
1.01262626125901	3.85292139770321e-15\\
1.01262626125905	3.29710122943836e-14\\
1.01262626125909	5.17123439095122e-15\\
1.01262626125913	3.24916214677299e-14\\
1.01262626125917	5.09470036496213e-15\\
1.01262626125954	-2.16049492803937e-13\\
1.01262626125992	-8.12061430569703e-14\\
1.0126262612603	-2.51066991633209e-13\\
1.01262626126408	-3.42986861407073e-13\\
1.01262626126786	1.93191757468336e-13\\
1.01262626127164	2.91898749371072e-14\\
1.01262626127167	6.97647334806129e-15\\
1.01262626127169	1.69120109546662e-14\\
1.01262626127172	2.63048823955546e-15\\
1.012626261272	1.22140901730575e-14\\
1.01262626127227	-3.74211086935217e-14\\
1.01262626127255	-1.95179269760578e-14\\
1.01262626127257	-4.59624459246988e-16\\
1.01262626127259	-1.63158467563651e-14\\
1.01262626127261	4.03044788776224e-17\\
};
\addlegendentry{$\dot{\theta}$};

\addplot [color=mycolor2,solid]
  table[row sep=crcr]{%
0	-1\\
0.226	-1\\
0.452	-1\\
0.678	-1\\
0.904	-1\\
0.9718	-1\\
0.99214	-1\\
0.998242	-1\\
0.99879118	-1\\
0.99934036	-1\\
0.99988954	-1\\
0.99993402358	-1\\
0.99997850716	-1\\
1.00002299074	1\\
1.00046782654	1\\
1.00091266234	1\\
1.00135749814	1\\
1.00202863945173	1\\
1.00203027032512	1\\
1.0020319011985	1\\
1.00203353207189	-1\\
1.00204984080577	1\\
1.00206614953964	-1\\
1.00208245827352	-1\\
1.00208377928096	1\\
1.0020851002884	1\\
1.00208642129585	1\\
1.00208677796786	-1\\
1.00208713463987	-1\\
1.00208749131188	-1\\
1.00208749997901	-1\\
1.00208750864614	-1\\
1.00208751731327	-1\\
1.00208760398456	1\\
1.00208769065586	1\\
1.00208777732716	1\\
1.00208864404015	1\\
1.00208951075313	-1\\
1.00209037746611	1\\
1.00209039852724	1\\
1.00209041958836	-1\\
1.00209044064949	1\\
1.00209065126074	1\\
1.002090861872	1\\
1.00209107248325	-1\\
1.00209317859581	-1\\
1.00209528470836	-1\\
1.00209739082091	-1\\
1.0020979594713	-1\\
1.00209852812169	-1\\
1.00209909677207	-1\\
1.00210478327596	1\\
1.00211046977985	1\\
1.00211615628374	-1\\
1.00217302132261	-1\\
1.00222988636149	-1\\
1.00228675140036	-1\\
1.00229135746851	-1\\
1.00229596353666	-1\\
1.00230056960481	-1\\
1.0023466302863	-1\\
1.00239269096779	-1\\
1.00243875164928	-1\\
1.00257693369374	1\\
1.00271511573821	-1\\
1.00285329778268	1\\
1.00310570971597	-1\\
1.00335812164926	1\\
1.00361053358255	-1\\
1.00364705006308	1\\
1.0036835665436	-1\\
1.00372008302413	1\\
1.00372304085905	1\\
1.00372599869397	1\\
1.0037289565289	1\\
1.00375853487812	1\\
1.00378811322735	1\\
1.00381769157657	1\\
1.00390642662425	-1\\
1.00399516167193	1\\
1.00408389671961	-1\\
1.00427626949192	1\\
1.00429358304143	-1\\
1.00431089659094	-1\\
1.00432821014045	1\\
1.00432961253796	1\\
1.00433101493547	1\\
1.00433241733298	1\\
1.00433279598031	1\\
1.00433317462763	1\\
1.00433355327496	1\\
1.00433358394539	1\\
1.00433361461583	1\\
1.00433364528626	1\\
1.0043339519906	1\\
1.00433425869493	-1\\
1.00433456539927	1\\
1.00433464820944	1\\
1.00433473101961	1\\
1.00433481382978	-1\\
1.0043348205374	-1\\
1.00433482724503	-1\\
1.00433483395265	-1\\
1.00433490102889	1\\
1.00433496810513	1\\
1.00433503518137	1\\
1.00433570594375	1\\
1.00433637670613	1\\
1.00433704746851	1\\
1.00433722857435	-1\\
1.0043374096802	-1\\
1.00433759078604	-1\\
1.00433760545561	-1\\
1.00433762012519	-1\\
1.00433763479476	-1\\
1.00433778149049	-1\\
1.00433792818623	1\\
1.00433807488196	-1\\
1.00433811448981	-1\\
1.00433815409765	1\\
1.0043381937055	1\\
1.00433819691374	1\\
1.00433820012197	1\\
1.00433820333021	1\\
1.00433823541257	1\\
1.00433826749492	1\\
1.00433829957728	-1\\
1.00433862040085	-1\\
1.00433894122441	-1\\
1.00433926204798	-1\\
1.00433934867035	-1\\
1.00433943529271	-1\\
1.00433952191507	-1\\
1.0043403881387	1\\
1.00434125436234	1\\
1.00434212058597	1\\
1.0043507828223	1\\
1.00435944505863	-1\\
1.00436810729496	1\\
1.00436817044267	1\\
1.00436823359037	1\\
1.00436829673807	1\\
1.0043689282151	1\\
1.00436955969213	-1\\
1.00437019116916	-1\\
1.0043702423188	-1\\
1.00437029346844	-1\\
1.00437034461807	-1\\
1.00437085611447	-1\\
1.00437136761086	-1\\
1.00437187910725	-1\\
1.00437201721128	-1\\
1.00437215531531	-1\\
1.00437229341933	-1\\
1.00437367445959	1\\
1.00437505549985	-1\\
1.00437643654012	1\\
1.00437680942099	-1\\
1.00437718230186	-1\\
1.00437755518273	1\\
1.00437756424373	1\\
1.00437757330474	1\\
1.00437758236574	1\\
1.00437767297579	1\\
1.00437776358585	1\\
1.0043778541959	-1\\
1.00437876029641	-1\\
1.00437966639693	-1\\
1.00438057249744	-1\\
1.00438081714458	-1\\
1.00438106179172	-1\\
1.00438130643886	-1\\
1.00438375291025	1\\
1.00438619938164	1\\
1.00438864585304	1\\
1.00441311056695	1\\
1.00443757528087	1\\
1.00446203999479	1\\
1.00446402163661	1\\
1.00446600327844	1\\
1.00446798492027	1\\
1.00448780133854	1\\
1.00450761775681	1\\
1.00452743417509	1\\
1.00472559835781	1\\
1.00473094879075	1\\
1.00473629922368	1\\
1.00474164965662	1\\
1.00479515398595	1\\
1.00484865831529	1\\
1.00490216264462	1\\
1.00505974035102	-1\\
1.00507392234459	-1\\
1.00508810433817	-1\\
1.00510228633174	-1\\
1.00510611547001	-1\\
1.00510994460827	1\\
1.00511377374654	1\\
1.0051138667946	-1\\
1.00511395984266	-1\\
1.00511405289072	1\\
1.00511407801369	1\\
1.00511410313667	-1\\
1.00511412825965	-1\\
1.00511413029461	-1\\
1.00511413232957	-1\\
1.00511413436453	-1\\
1.00511415471414	1\\
1.00511417506375	1\\
1.00511419541336	1\\
1.00511439890947	1\\
1.00511460240558	-1\\
1.00511480590168	-1\\
1.00511482238487	-1\\
1.00511483886805	-1\\
1.00511485535124	-1\\
1.00511502018308	1\\
1.00511518501493	1\\
1.00511534984678	-1\\
1.00511539435138	-1\\
1.00511543885597	-1\\
1.00511548336057	-1\\
1.00511592840656	-1\\
1.00511637345254	-1\\
1.00511681849853	-1\\
1.00511693866095	-1\\
1.00511705882336	-1\\
1.00511717898578	-1\\
1.00511838060994	1\\
1.0051195822341	-1\\
1.00512078385826	-1\\
1.00512088118982	-1\\
1.00512097852138	-1\\
1.00512107585293	-1\\
1.00512204916851	-1\\
1.00512302248408	1\\
1.00512399579965	-1\\
1.00512407463821	-1\\
1.00512415347677	-1\\
1.00512423231533	-1\\
1.00512502070094	-1\\
1.00512580908656	-1\\
1.00512659747217	-1\\
1.00512681033628	-1\\
1.0051270232004	-1\\
1.00512723606451	-1\\
1.00512936470567	1\\
1.00513149334682	1\\
1.00513362198797	-1\\
1.00515490839951	1\\
1.00517619481104	1\\
1.00519748122258	-1\\
1.00526134045718	1\\
1.00532519969179	1\\
1.00538905892639	-1\\
1.0055806366302	1\\
1.00558580922821	1\\
1.00559098182621	1\\
1.00559615442421	1\\
1.00564788040424	-1\\
1.00569960638427	1\\
1.0057513323643	-1\\
1.00576356101924	-1\\
1.00577578967418	-1\\
1.00578801832913	1\\
1.00591030487855	1\\
1.00591360661538	1\\
1.00591690835221	1\\
1.00592021008905	1\\
1.00595322745739	-1\\
1.00598624482574	-1\\
1.00601926219408	1\\
1.00634943587751	1\\
1.00667960956094	1\\
1.00700978324438	1\\
1.00702594883536	1\\
1.00704211442634	1\\
1.00705828001733	1\\
1.00721993592717	1\\
1.00723448495906	1\\
1.00724903399095	1\\
1.00726358302283	-1\\
1.00740907334169	-1\\
1.00755456366055	1\\
1.00770005397941	-1\\
1.00771056278943	-1\\
1.00772107159944	-1\\
1.00773158040945	1\\
1.00773243162306	1\\
1.00773328283668	1\\
1.00773413405029	1\\
1.0077426461864	-1\\
1.00775115832251	-1\\
1.00775967045862	-1\\
1.00784479181972	-1\\
1.00792991318083	-1\\
1.00793757410333	-1\\
1.00794523502583	-1\\
1.00795289594833	-1\\
1.00802950517332	-1\\
1.00803640000357	-1\\
1.00804329483382	-1\\
1.00805018966407	-1\\
1.00811913796656	1\\
1.00818808626906	-1\\
1.00825703457155	1\\
1.00846387947904	-1\\
1.00867072438653	1\\
1.00887756929401	-1\\
1.00888974336525	-1\\
1.00890191743649	-1\\
1.00891409150772	-1\\
1.0090358322201	-1\\
1.00904678888422	-1\\
1.00905774554833	-1\\
1.00906870221244	1\\
1.00917826885358	1\\
1.00918122715289	1\\
1.00918418545221	1\\
1.00918714375152	1\\
1.00921672674462	-1\\
1.00924630973773	-1\\
1.00927589273084	1\\
1.00957172266192	1\\
1.00986755259299	1\\
1.01016338252407	1\\
1.01017823519288	-1\\
1.01019308786169	-1\\
1.01020794053049	-1\\
1.01021195075107	-1\\
1.01021596097165	-1\\
1.01021997119223	1\\
1.01026007339801	-1\\
1.01030017560379	1\\
1.01034027780958	-1\\
1.01035110540514	1\\
1.0103619330007	1\\
1.01037276059626	-1\\
1.0103736376315	-1\\
1.01037451466674	-1\\
1.01037539170198	-1\\
1.01038416205439	1\\
1.01039293240679	1\\
1.0104017027592	-1\\
1.01048940628324	-1\\
1.01049729960041	-1\\
1.01050519291757	-1\\
1.01051308623473	1\\
1.01059201940638	-1\\
1.01061569935787	1\\
1.01063937930936	-1\\
1.01066305926085	1\\
1.01089985877578	1\\
1.0111366582907	-1\\
1.01137345780563	1\\
1.01140857502872	1\\
1.0114436922518	-1\\
1.01147880947489	-1\\
1.01148829112512	1\\
1.01149777277535	-1\\
1.01150725442559	1\\
1.01160207092792	1\\
1.01169688743025	-1\\
1.01179170393258	1\\
1.01205554623419	-1\\
1.0123193885358	1\\
1.01258323083742	-1\\
1.01259744434032	1\\
1.01261165784323	1\\
1.01262587134613	-1\\
1.01262588067161	-1\\
1.01262588999709	-1\\
1.01262589932257	-1\\
1.01262599257736	-1\\
1.01262608583215	1\\
1.01262617908695	-1\\
1.01262620426574	1\\
1.01262622944453	1\\
1.01262625462333	1\\
1.01262625480688	1\\
1.01262625499043	1\\
1.01262625517399	-1\\
1.01262625700952	-1\\
1.01262625884506	-1\\
1.01262626068059	1\\
1.01262626072519	1\\
1.0126262607698	1\\
1.0126262608144	1\\
1.01262626085454	1\\
1.01262626089469	1\\
1.01262626093483	1\\
1.01262626093808	1\\
1.01262626094133	1\\
1.01262626094459	1\\
1.01262626094546	1\\
1.01262626094634	1\\
1.01262626094722	1\\
1.012626260956	-1\\
1.01262626096478	-1\\
1.01262626097356	-1\\
1.01262626106135	-1\\
1.01262626114914	-1\\
1.01262626123694	1\\
1.01262626124405	1\\
1.01262626125116	1\\
1.01262626125827	-1\\
1.01262626125844	-1\\
1.01262626125862	-1\\
1.01262626125879	-1\\
1.01262626125883	-1\\
1.01262626125888	-1\\
1.01262626125893	-1\\
1.01262626125897	-1\\
1.01262626125901	-1\\
1.01262626125905	-1\\
1.01262626125909	-1\\
1.01262626125913	-1\\
1.01262626125917	-1\\
1.01262626125954	1\\
1.01262626125992	1\\
1.0126262612603	1\\
1.01262626126408	1\\
1.01262626126786	-1\\
1.01262626127164	-1\\
1.01262626127167	-1\\
1.01262626127169	-1\\
1.01262626127172	-1\\
1.012626261272	-1\\
1.01262626127227	1\\
1.01262626127255	1\\
1.01262626127257	1\\
1.01262626127259	1\\
1.01262626127261	-1\\
};
\addlegendentry{$\dot{\psi}$};

\end{axis}
\end{tikzpicture}%
}
  \caption{}
  \label{fig:03}
\end{figure}
