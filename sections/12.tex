\begin{align*}
  d_g[k] &= cos(\theta_g) (x_g - x[k]) + sin(\theta_g) (y_g - y[k]) \\
         &= cos(\theta_g) (x_g - x[k-1] - T_s R u_{\omega}^T[k-1] cos(\theta_g)) + \\
         &  sin(\theta_g) (y_g - y[k-1] - T_s R u_{\omega}^T[k-1] sin(\theta_g)) \\
         &= cos(\theta_g) (x_g - x[k-1]) + sin(\theta_g) (y_g - y[k-1]) -T_s R K_{\omega}^T d_g[k-1]  \\
         &= d_g[k-1] - T_s R K_{\omega}^T d_g[k-1]  \\
         &= (1-T_s R K_{\omega}^T)d_g[k-1]
\end{align*}

Hence

\begin{align*}
  d_g[k+1]= (1-T_s R K_{\omega}^T)d_g[k]
\end{align*}

In order for this system to be stable, that is, $d_g \to 0$ as $k \to \infty$,
the solution of the equation inside the parentheses should lie inside the unit
circle:

\begin{align}
  \Big|1 - &T_s R K_{\omega}^T\Big| < 1 \nonumber \\
  -1 < 1 - &T_s R K_{\omega}^T < 1 \nonumber \\
  -2 < - &T_s R K_{\omega}^T < 0 \nonumber \\
   0 <\ &T_s R K_{\omega}^T < 2 \nonumber \\
   0 <\ &K_{\omega}^T < \dfrac{2}{T_s R} \label{eq:12.2}
\end{align}

Hence, the maximum value $K_{\omega}^T$ can take for the system to be marginally
stable is $K_{\omega,max}^T = \dfrac{2}{T_s R}$

Theoretically, the value of $K_{\omega}^T$ can be chosen to be any value inside
the interval defined in inequality \ref{eq:12.2}. However, first, it would be
wise to choose a value that is far enough from the maximum value so as to avoid
overshoot, but close enough to it, so that convergence happens in reasonable
time. Hence, in practice, it is reasonable that one would need to experiment
with different values and choose one that results in balancing a small angular
error, a minimal overshoot, if any, and a quick enough settling time.

Eventually, the choice made for $K_{\omega}^T$ was
$K_{\omega}^T = \frac{1}{2} K_{\omega,max}^T$.
