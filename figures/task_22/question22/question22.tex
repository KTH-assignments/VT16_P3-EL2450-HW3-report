\documentclass[11pt]{article}
\usepackage{amsmath}
\usepackage{graphicx}
\usepackage{subfig}
\usepackage{float}
\usepackage{epstopdf}

%\usepackage{Fourier}
%\usepackage{mathtools}


\begin{document}
After adjusting gain values based on real test results at the lab, the selected values are:\\\\
$K_{\omega} =0.2\:K_{\omega,max}$  and, $ K_{\Psi} = 0.2\:K_{\Psi,max} $ \\

The original values were delivering very short settling times while large
positioning errors. The last effect was due to the high requirements of speed
from the translational controller part, that led $u_{l}$ and $u_{r}$ to their
upper threshold, as commented in question 19.

In the figure ?? the distance with respect to the goal is plotted. There, one
can observe the trade-off  between speed and offset. The positioning error is
around 7 centimeters, which is a well acceptable value for a displacement of
more than 2 meters. \\

%Dont forget to plot the shitty figure!

In the figure ??+1 the bearing error is plotted. Comparing both figures it can
be seen how the translational motion starts whenever the rotational error is
smaller than certain value, prevoiusly defined and adjusted. As well, one can
observe how the line follower part of the controller acts over the bearing so
to correct the error in heading. The final error is bounded inside 2 degrees,
which can be considered as a quite good result.

%The time scale is shifted so that 0 corresponds to the moment where the robot starts to turn.


\end{document}
